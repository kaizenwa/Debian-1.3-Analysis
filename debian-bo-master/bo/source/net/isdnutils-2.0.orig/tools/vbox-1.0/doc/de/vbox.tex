\documentstyle[linuxdoc-sgml-a4,german,isolatin,qwertz,null]{article}
%   EPSF.TEX macro file:
%   Written by Tomas Rokicki of Radical Eye Software, 29 Mar 1989.
%   Revised by Don Knuth, 3 Jan 1990.
%   Revised by Tomas Rokicki to accept bounding boxes with no
%      space after the colon, 18 Jul 1990.
%
%   TeX macros to include an Encapsulated PostScript graphic.
%   Works by finding the bounding box comment,
%   calculating the correct scale values, and inserting a vbox
%   of the appropriate size at the current position in the TeX document.
%
%   To use with the center environment of LaTeX, preface the \epsffile
%   call with a \leavevmode.  (LaTeX should probably supply this itself
%   for the center environment.)
%
%   To use, simply say
%   \input epsf           % somewhere early on in your TeX file
%   \epsfbox{filename.ps} % where you want to insert a vbox for a figure
%
%   Alternatively, you can type
%
%   \epsfbox[0 0 30 50]{filename.ps} % to supply your own BB
%
%   which will not read in the file, and will instead use the bounding
%   box you specify.
%
%   The effect will be to typeset the figure as a TeX box, at the
%   point of your \epsfbox command. By default, the graphic will have its
%   `natural' width (namely the width of its bounding box, as described
%   in filename.ps). The TeX box will have depth zero.
%
%   You can enlarge or reduce the figure by saying
%     \epsfxsize=<dimen> \epsfbox{filename.ps}
%   (or
%     \epsfysize=<dimen> \epsfbox{filename.ps})
%   instead. Then the width of the TeX box will be \epsfxsize and its
%   height will be scaled proportionately (or the height will be
%   \epsfysize and its width will be scaled proportiontally).  The
%   width (and height) is restored to zero after each use.
%
%   A more general facility for sizing is available by defining the
%   \epsfsize macro.    Normally you can redefine this macro
%   to do almost anything.  The first parameter is the natural x size of
%   the PostScript graphic, the second parameter is the natural y size
%   of the PostScript graphic.  It must return the xsize to use, or 0 if
%   natural scaling is to be used.  Common uses include:
%
%      \epsfxsize  % just leave the old value alone
%      0pt         % use the natural sizes
%      #1          % use the natural sizes
%      \hsize      % scale to full width
%      0.5#1       % scale to 50% of natural size
%      \ifnum#1>\hsize\hsize\else#1\fi  % smaller of natural, hsize
%
%   If you want TeX to report the size of the figure (as a message
%   on your terminal when it processes each figure), say `\epsfverbosetrue'.
%
\newread\epsffilein    % file to \read
\newif\ifepsffileok    % continue looking for the bounding box?
\newif\ifepsfbbfound   % success?
\newif\ifepsfverbose   % report what you're making?
\newdimen\epsfxsize    % horizontal size after scaling
\newdimen\epsfysize    % vertical size after scaling
\newdimen\epsftsize    % horizontal size before scaling
\newdimen\epsfrsize    % vertical size before scaling
\newdimen\epsftmp      % register for arithmetic manipulation
\newdimen\pspoints     % conversion factor
%
\pspoints=1bp          % Adobe points are `big'
\epsfxsize=0pt         % Default value, means `use natural size'
\epsfysize=0pt         % ditto
%
\def\epsfbox#1{\global\def\epsfllx{72}\global\def\epsflly{72}%
   \global\def\epsfurx{540}\global\def\epsfury{720}%
   \def\lbracket{[}\def\testit{#1}\ifx\testit\lbracket
   \let\next=\epsfgetlitbb\else\let\next=\epsfnormal\fi\next{#1}}%
%
\def\epsfgetlitbb#1#2 #3 #4 #5]#6{\epsfgrab #2 #3 #4 #5 .\\%
   \epsfsetgraph{#6}}%
%
\def\epsfnormal#1{\epsfgetbb{#1}\epsfsetgraph{#1}}%
%
\def\epsfgetbb#1{%
%
%   The first thing we need to do is to open the
%   PostScript file, if possible.
%
\openin\epsffilein=#1
\ifeof\epsffilein\errmessage{I couldn't open #1, will ignore it}\else
%
%   Okay, we got it. Now we'll scan lines until we find one that doesn't
%   start with %. We're looking for the bounding box comment.
%
   {\epsffileoktrue \chardef\other=12
    \def\do##1{\catcode`##1=\other}\dospecials \catcode`\ =10
    \loop
       \read\epsffilein to \epsffileline
       \ifeof\epsffilein\epsffileokfalse\else
%
%   We check to see if the first character is a % sign;
%   if not, we stop reading (unless the line was entirely blank);
%   if so, we look further and stop only if the line begins with
%   `%%BoundingBox:'.
%
          \expandafter\epsfaux\epsffileline:. \\%
       \fi
   \ifepsffileok\repeat
   \ifepsfbbfound\else
    \ifepsfverbose\message{No bounding box comment in #1; using defaults}\fi\fi
   }\closein\epsffilein\fi}%
%
%   Now we have to calculate the scale and offset values to use.
%   First we compute the natural sizes.
%
\def\epsfsetgraph#1{%
   \epsfrsize=\epsfury\pspoints
   \advance\epsfrsize by-\epsflly\pspoints
   \epsftsize=\epsfurx\pspoints
   \advance\epsftsize by-\epsfllx\pspoints
%
%   If `epsfxsize' is 0, we default to the natural size of the picture.
%   Otherwise we scale the graph to be \epsfxsize wide.
%
   \epsfxsize\epsfsize\epsftsize\epsfrsize
   \ifnum\epsfxsize=0 \ifnum\epsfysize=0
      \epsfxsize=\epsftsize \epsfysize=\epsfrsize
%
%   We have a sticky problem here:  TeX doesn't do floating point arithmetic!
%   Our goal is to compute y = rx/t. The following loop does this reasonably
%   fast, with an error of at most about 16 sp (about 1/4000 pt).
% 
     \else\epsftmp=\epsftsize \divide\epsftmp\epsfrsize
       \epsfxsize=\epsfysize \multiply\epsfxsize\epsftmp
       \multiply\epsftmp\epsfrsize \advance\epsftsize-\epsftmp
       \epsftmp=\epsfysize
       \loop \advance\epsftsize\epsftsize \divide\epsftmp 2
       \ifnum\epsftmp>0
          \ifnum\epsftsize<\epsfrsize\else
             \advance\epsftsize-\epsfrsize \advance\epsfxsize\epsftmp \fi
       \repeat
     \fi
   \else\epsftmp=\epsfrsize \divide\epsftmp\epsftsize
     \epsfysize=\epsfxsize \multiply\epsfysize\epsftmp   
     \multiply\epsftmp\epsftsize \advance\epsfrsize-\epsftmp
     \epsftmp=\epsfxsize
     \loop \advance\epsfrsize\epsfrsize \divide\epsftmp 2
     \ifnum\epsftmp>0
        \ifnum\epsfrsize<\epsftsize\else
           \advance\epsfrsize-\epsftsize \advance\epsfysize\epsftmp \fi
     \repeat     
   \fi
%
%  Finally, we make the vbox and stick in a \special that dvips can parse.
%
   \ifepsfverbose\message{#1: width=\the\epsfxsize, height=\the\epsfysize}\fi
   \epsftmp=10\epsfxsize \divide\epsftmp\pspoints
   \vbox to\epsfysize{\vfil\hbox to\epsfxsize{%
      \special{PSfile=#1 llx=\epsfllx\space lly=\epsflly\space
          urx=\epsfurx\space ury=\epsfury\space rwi=\number\epsftmp}%
      \hfil}}%
\epsfxsize=0pt\epsfysize=0pt}%

%
%   We still need to define the tricky \epsfaux macro. This requires
%   a couple of magic constants for comparison purposes.
%
{\catcode`\%=12 \global\let\epsfpercent=%\global\def\epsfbblit{%BoundingBox}}%
%
%   So we're ready to check for `%BoundingBox:' and to grab the
%   values if they are found.
%
\long\def\epsfaux#1#2:#3\\{\ifx#1\epsfpercent
   \def\testit{#2}\ifx\testit\epsfbblit
      \epsfgrab #3 . . . \\%
      \epsffileokfalse
      \global\epsfbbfoundtrue
   \fi\else\ifx#1\par\else\epsffileokfalse\fi\fi}%
%
%   Here we grab the values and stuff them in the appropriate definitions.
%
\def\epsfgrab #1 #2 #3 #4 #5\\{%
   \global\def\epsfllx{#1}\ifx\epsfllx\empty
      \epsfgrab #2 #3 #4 #5 .\\\else
   \global\def\epsflly{#2}%
   \global\def\epsfurx{#3}\global\def\epsfury{#4}\fi}%
%
%   We default the epsfsize macro.
%
\def\epsfsize#1#2{\epsfxsize}
%
%   Finally, another definition for compatibility with older macros.
%
\let\epsffile=\epsfbox

\title{V B O X}
\author{Michael `Ghandi' Herold {\(<\)}{\tt michael@abadonna.franken.de}{\(>\)}}
\date{Version 1.0 (16. Juni 1996)}
\abstract{Diese Dokumentation beschreibt die Installation und Konfiguration von
{\em vbox\/}, einem Anrufbeantworter f\"{u}r {\em isdn4linux\/}.}


\begin{document}
\maketitle
\tableofcontents





\section{Copyright\label{id-copyright}}





Copyright {\copyright} 1996 bei Michael Herold ({\tt michael@abadonna.franken.de})

Das Paket {\em vbox\/} ist freie Software und kann unter den Bedingungen
der {\em GNU General Public License\/} ver\"{o}ffentlicht oder ver\"{a}ndert
werden.

Die Dokumentation sowie alle Programme von {\em vbox\/} wurden mit gro{\ss}er
Sorgfalt erstellt. Dennoch sind Fehler nicht ganz auszuschlie{\ss}en. Aus diesem
Grund sind die in der vorliegenden Dokumentation enthaltenen Angaben mit
keiner Verpflichtung oder Garantie irgendeiner Art verbunden. Sowohl Michael
Herold als auch die in Kapitel \ref{id-danksagungen} genannten Personen \"{u}bernehmen keine Verantwortung oder
sonstige Haftung f\"{u}r Sch\"{a}den, die auf irgendeine Art aus der Benutzung der
Dokumentation oder den darin beschriebenen Programmen entstehen oder
entstanden sind.






\section{Vorwort\label{id-vorwort}}








\subsection{Was ist vbox?\label{id-vorwort-wasistvbox}}



{\em vbox\/} ist ein Paket mit Programmen zum Betrieb eines
Anrufbeantworters unter Linux. Es wird das Paket {\em isdn4linux\/} von
Fritz Elfert (entweder im Kernel oder als Modul) mit {\tt AUDIO SUPPORT}
ben\"{o}tigt (enthalten ab Kernel pre2.0.4).

{\em vboxgetty\/} basiert auf {\em mgetty-0.99\/} ({\em vgetty\/}) vom
18. April 1996. Die mit {\em vboxgetty\/} aufgezeichneten Sprachdateien
k\"{o}nnen mit den dort enthaltenen {\em pvftools\/} abgespielt oder
konvertiert werden (f\"{u}r die Formate {\tt ULAW} und {\tt ALAW} liegen
Programme zum Abspielen nach `{\tt /dev/audio}' bei). Die
{\em pfvtools\/} sind {\bf nicht} im Paket von {\em vbox\/} enthalten.

{\em vboxgetty\/} wurde speziell f\"{u}r {\em isdn4linux\/} geschrieben und
ist nicht f\"{u}r normale Modems geeignet. Es k\"{o}nnen mit {\em vboxgetty\/}
keine Fax- oder Datenanrufe erkannt oder entgegengenommen werden. Wer so
etwas ben\"{o}tigt, dem kann ich nur {\em mgetty\/}, bzw. {\em vgetty\/}
empfehlen.






\subsection{Anregungen und Fehlermeldungen\label{id-vorwort-report}}



Wer neue Ideen oder Anregungen hat, oder einen Fehler in einem der Programme
von {\em vbox\/} gefunden hat, kann diese an folgende Adresse schicken:

{\tt michael@abadonna.franken.de}

Bei einer Mitteilung \"{u}ber einen Fehler bitte {\bf immer} eine m\"{o}glichst
genaue Beschreibung (am besten reproduzierbar) {\bf und} den
entsprechenden Auszug aus den Logdateien beif\"{u}gen.






\section{Installation\label{id-install}}



F\"{u}r die Installation werden folgende Dinge ben\"{o}tigt:

\begin{itemize}
\item GCC in der Version 2.7.0 oder h\"{o}her,
\item Linux Kernel in der Version 2.0 oder h\"{o}her,
\item NCurses in der Version 1.9.9e oder h\"{o}her.
\end{itemize}


Zudem sollte {\em isdn4linux\/} bereits funktionsf\"{a}hig mit Audiosupport
installiert sein.

F\"{u}r das Abspielen der aufgenommenen Nachrichten in den Formaten
{\tt ADPCM-2}, {\tt ADPCM-3} und {\tt ADPCM-4} werden zus\"{a}tzlich
die {\bf pvftools} aus dem Paket {\em mgetty\/} (Version 0.99 vom 18.
April 1996) ben\"{o}tigt.

F\"{u}r die Formate {\tt ALAW} und {\tt ULAW} liegen Programme zum
Abspielen nach {\tt /dev/audio} bei.

Vor der Installation sollten folgende Dateien an das eigene System angepasst
werden:

\begin{itemize}
\item `{\tt Makefile}' (siehe Kapitel \ref{id-install-makefile})
\item `{\tt src/settings.h}' (siehe Kapitel \ref{id-install-settings})
\end{itemize}


Zum erzeugen aller Programme von {\em vbox\/} tippen Sie

{\tt \$ make vbox}

Die Programme befinden sich danach - sofern sie fehlerfrei erzeugt wurden -
im Unterverzeichnis {\tt bin} des Quelltextverzeichnisses. Mit

{\tt \$ make install}

werden diese in die im `{\tt Makefile}' (siehe Kapitel \ref{id-install-makefile}) angegebenen
Verzeichnisse kopiert.

{\bf Bemerkung:} Vergessen Sie nicht, da{\ss} ebenfalls im
`{\tt Makefile}' (siehe Kapitel \ref{id-install-makefile}) angegebene Spoolverzeichnis per Hand
oder mit

{\tt \$ make install-dir}

zu erzeugen. Die Unterverzeichnisse

\begin{itemize}
\item {\(<\)}Benutzername{\(>\)}
\item {\(<\)}Benutzername{\(>\)}/incoming
\item {\(<\)}Benutzername{\(>\)}/messages
\end{itemize}


werden von {\em vboxgetty\/} zur Laufzeit automatisch erzeugt, falls sie
noch nicht existieren.

{\bf Bemerkung:} Die Standardansagen werden {\bf nicht} automatisch
erzeugt, sondern m\"{u}ssen f\"{u}r jede Person selbst in das entsprechende
Unterverzeichnis kopiert werden.






\subsection{Einstellungen in `Makefile'\label{id-install-makefile}}



Folgende Einstellungen sind im `{\tt Makefile}' m\"{o}glich:

\begin{description}


\item[SPOOLDIR] \mbox{}



Verzeichnis, in dem die Daten f\"{u}r {\em vbox\/} abgelegt werden sollen.
F\"{u}r jeden Benutzer, f\"{u}r den ein {\em vboxgetty\/} l\"{a}uft, wird dort ein
eigenes Verzeichnis angelegt.

{\bf Beispiel:} {\tt /var/spool/vbox}



\item[INSTALLSBINDIR] \mbox{}



Verzeichnis, in das die Programme installiert werden, die nur vom Benutzer
{\tt root} gestartet werden sollten.

{\bf Beispiel:} {\tt /usr/local/sbin}



\item[INSTALLBINDIR] \mbox{}



Verzeichnis, in das alle Programme installiert werden, die von normalen
Benutzern aufgerufen werden d\"{u}rfen.

{\bf Beispiel:} {\tt /usr/local/bin}



\item[INSTALLXSBINDIR] \mbox{}



Verzeichnis, in das alle Programme installiert werden, die unter X11 laufen
und nur vom Benutzer {\tt root} gestartet werden sollten.

{\bf Beispiel:} {\tt /usr/X11/bin}



\item[INSTALLXBINDIR] \mbox{}



Verzeichnis, in das alle Programme installiert werden, die unter X11 laufen
und von normalen Benutzern aufgerufen werden d\"{u}rfen.

{\bf Beispiel:} {\tt /usr/X11/bin}



\item[INSTALLSBIN] \mbox{}



Kommando, mit dem alle Programme installiert werden, die nur vom Benutzer
{\tt root} gestartet werden sollten.

{\bf Beispiel:} {\tt install -s -o root -g root -m 750}



\item[INSTALLBIN] \mbox{}



Kommando, mit dem alle Programme installiert werden, die von normalen
Benutzern gestartet werden k\"{o}nnen.

{\bf Beispiel:} {\tt install -s -o root -g bin -m 755}



\item[CC] \mbox{}



Der Kompiler, der zum \"{U}bersetzen der Quelltexte verwendet werden soll.

{\bf Beispiel:} {\tt gcc}



\item[CCFLAGS] \mbox{}



Flag's, die dem Kompiler \"{u}bergeben werden sollen.

{\bf Beispiel:} {\tt -i486 -O2 -Wall}



\end{description}







\subsection{Einstellungen in `src/settings.h'\label{id-install-settings}}



Folgende Einstellungen sind in `{\tt src/settings.h}' m\"{o}glich (alle
vorgegebenen Einstellungen {\bf m\"{u}ssen} angegeben werden):

\begin{description}


\item[LOGFILE] \mbox{}



Name der Logdatei in die {\em vbox\/} seine Meldungen schreibt. Der
Platzhalter `{\tt \%s}' wird durch den Namen des jeweiligen tty-Devices
ersetzt.

{\bf Beispiel:} {\tt "/var/log/vbox.\%s"}



\item[LOCKFILE] \mbox{}



Name der Datei mit der {\em vbox\/} die einzelnen Devices lockt. Der
Platzhalter `{\tt \%s}' wird durch den Namen des jeweiligen tty-Devices
ersetzt.

{\bf Beispiel:} {\tt "/var/lock/LCK..\%s"}



\item[PIDFILE] \mbox{}



Name der Datei in der {\em vbox\/} seine Proze{\ss}-ID vermerkt. Der
Platzhalter `{\tt \%s}' wird durch den Namen des jeweiligen tty-Devices
ersetzt.

{\bf Beispiel:} {\tt "/var/run/vbox-pid.\%s"}



\item[VBOXRC] \mbox{}



Name der Konfigurationsdatei, die sich im Homeverzeichnis des jeweiligen
Benutzers befinden kann.

{\bf Beispiel:} {\tt ".vboxrc"}



\item[MODEM\_RING\_TIMEOUT] \mbox{}



Anzahl der Sekunden, in denen versucht wird, einen {\tt RING} vom
Modem einzulesen, nachdem ein eingehender Anruf erkannt wurde.

{\bf Beispiel:} {\tt 5}



\item[MODEM\_COMMAND\_TIMEOUT] \mbox{}



Anzahl der Sekunden, in denen versucht wird, da{\ss} Ergebnis eines
AT-Kommandos einzulesen.

{\bf Beispiel:} {\tt 2}



\item[MODEM\_ECHOCMD\_TIMEOUT] \mbox{}



Anzahl der Sekunden, in denen versucht wird, da{\ss} Echo eines AT-Kommandos
einzulesen.

{\bf Beispiel:} {\tt 2}



\item[MODEM\_WAIT\_TIMEOUT] \mbox{}



Anzahl der Sekunden, nach deren Ablauf ein Kommando an das Modem geschickt
wird, um festzustellen, ob es noch reagiert. Der Wert 0 schaltet diese
\"{U}berpr\"{u}fung aus.

{\bf Beispiel:} {\tt 1800}



\item[DEFAULT\_RINGS] \mbox{}



Anzahl der {\tt RING}'s, bevor {\em vbox\/} den Anruf entgegennimmt.
Bei mir pers\"{o}nlich ({\tt Europa 10}) werden 5 {\tt RING}'s
gemeldet, bevor das Telefon einmal klingelt.

{\bf Beispiel:} {\tt 20}



\item[DEFAULT\_RECORD\_TIME] \mbox{}



Voreingestellte maximale Zeit in Sekunden, die ein entgegengenommener
Anruf aufgezeichnet wird.

{\bf Beispiel:} {\tt 60}



\item[VOICE\_COMPRESSION\_MODE] \mbox{}



Kompressionsmodus beim Aufzeichnen von Audiodaten. Folgende Modi sind
m\"{o}glich:

\begin{itemize}
\item {\tt 2} f\"{u}r {\tt ADPCM-2}
\item {\tt 3} f\"{u}r {\tt ADPCM-3}
\item {\tt 4} f\"{u}r {\tt ADPCM-4}
\item {\tt 5} f\"{u}r {\tt ALAW}
\item {\tt 6} f\"{u}r {\tt ULAW}
\end{itemize}


{\bf Beispiel:} {\tt 4}



\item[NEED\_FREE\_DISK\_SPACE] \mbox{}



Anzahl der Byte's, die auf der Partition des Spoolverzeichnisses noch
frei sein m\"{u}ssen, damit ein Anruf entgegengenommen wird.

{\bf Bemerkung:} Wird in dieser Version noch nicht unterst\"{u}tzt.



\item[VBOXPLAYCMD] \mbox{}



Kommando zum Abspielen einer Nachricht in {\em vbox\/}. Dem Kommando
werden folgende Argumente \"{u}bergeben:

\begin{enumerate}
\item Dateiname der ausgew\"{a}hlten Nachricht
\item Eingestellte Lautst\"{a}rke
\end{enumerate}


{\bf Beispiel:} {\tt "/usr/local/bin/vboxplay"}

Im Verzeichnis `{\tt samples}' befindet sich das Beispielskript
`{\tt vboxplay}', das die {\em pvftools\/} und den {\em Network Audio
Server\/} zum Spielen der Nachrichten benutzt.



\end{description}







\section{Programme\label{id-programme}}








\subsection{vboxgetty\label{id-vboxgetty}}



{\em vboxgetty\/} ist das eigentliche Hauptprogramm, welches die Anrufe
entgegennimmt, Ansagetexte abspielt und die Nachrichten der Anrufer
aufzeichnet.

Beim Start von {\em vboxgetty\/} k\"{o}nnen dem Programm folgende Argumente
\"{u}bergeben werden:

\begin{description}


\item[{\tt -s GESCHWINDIGKEIT}] \mbox{}



Die Geschwindigkeit, mit der {\em vboxgetty\/} das Modem ansprechen
soll. Die Voreinstellung von {\tt 38400} sollte in den meisten F\"{a}llen
ausreichend sein und ist auch f\"{u}r {\tt ADPCM-4} geeignet.

{\bf Beispiel:} {\tt vboxgetty -s 38400}



\item[{\tt -d DEVICE}] \mbox{}



Name des Devices, das als Modem benutzt werden soll. Das Device braucht
vorher nicht mit {\em setserial\/} eingestellt zu werden.

{\bf Beispiel:} {\tt vboxgetty -d /dev/ttyI5}



\item[{\tt -u BENUTZERNAME}] \mbox{}



{\em vboxgetty\/} {\bf mu{\ss}} beim Start der Name eines Benutzers
angegeben werden, f\"{u}r den die Nachrichten aufgezeichnet werden sollen.

Der angegebene Benutzer mu{\ss} in der `{\tt /etc/passwd}' eingetragen
sein und auf dem System existieren.

{\bf Beispiel:} {\tt vboxgetty -u michael}



\item[{\tt -i MODEM-INIT-KOMMANDO}] \mbox{}



Kommando zum Initialisieren des Modems. Hier sollte unter anderem die
Telefonnummer angegeben werden, auf die {\em vboxgetty\/} - also der
Anrufbeantworter - reagieren soll.

Benutzer, die den {\em iprofd\/} von {\em isdn4linux\/} benutzen,
sollten hier mindestens {\bf immer} die Telefonnummer angeben, auf die
reagiert werden soll.

{\bf Beispiel:} {\tt vboxgetty -i "ATZ\&E7850413"}



\end{description}


Ein normaler Aufruf von {\em vboxgetty\/} sollte in etwa so aussehen:

{\tt \$ vboxgetty -d /dev/ttyI6 -i "ATZ\&E7850413" -u michael}

F\"{u}r den in diesem Beispiel angegebenen Benutzer `michael' werden im
Spoolverzeichnis folgende Verzeichnisse angelegt, falls diese noch nicht
existieren:

\begin{itemize}
\item {\tt michael}
\item {\tt michael/incoming}
\item {\tt michael/messages}
\end{itemize}


Im {\tt Incoming}-Verzeichnis werden alle Nachrichten abgelegt, die bei
einem Anruf aufgezeichnet werden. Die Dateinamen dieser Nachrichten haben
folgendes Format:

{\tt {\(<\)}Jahr{\(>\)}{\(<\)}Monat{\(>\)}{\(<\)}Tag{\(>\)}{\(<\)}Stunde{\(>\)}{\(<\)}Minute{\(>\)}{\(<\)}Sekunde{\(>\)}-{\(<\)}CallerID{\(>\)}}

Wenn die Rufnummer eines Anrufers (sprich die CallerID) nicht ermittelt
werden konnte (z.B. weil der Anrufer keinen ISDN-Anschlu{\ss} besitzt), wird f\"{u}r
das Feld {\tt {\(<\)}CallerID{\(>\)}} der Wert {\tt 0} eingesetzt.

{\bf Beispiel eines {\tt Incoming}-Verzeichnisses:}

\begin{verbatim}
-rw-------   1 michael  users       24592 Jun 10 11:02 960610110220-9317850413
-rw-------   1 michael  users      131292 Jun 10 11:14 960610111408-9317850413
-rw-------   1 michael  users       28104 Jun 10 11:16 960610111612-0
\end{verbatim}


Wenn {\em vboxgetty\/} eine Nachricht aufzeichnet, wird diese nur mit den
Zugriffsrechten `{\tt --w-------}' ge\"{o}ffnet. Sobald die Nachricht zu Ende
aufgenommen wurde, werden die Zugriffsrechte zu `{\tt -rw-------}'
ge\"{a}ndert.

Programme, welche dem Benutzer eine Liste der eingegangenen Nachrichten
anzeigen, sollten nur solche Dateien aufnehmen, bei denen das {\tt r}-Bit
(Lesbar) gesetzt ist, um sicher zu sein, da{\ss} die Nachricht zu Ende
aufgezeichnet wurde (siehe Kapitel \ref{id-vbox}).

Im {\tt Messages}-Verzeichnis m\"{u}ssen sich alle Ansagetexte befinden, die
von {\em vboxgetty\/} gespielt werden sollen. Es existieren folgende
voreingestellte Namen:

\begin{description}


\item[{\tt standard.msg}] \mbox{}



Wird als normaler Ansagetext gespielt. Dieser kann in der Datei
`{\tt .vboxrc}' (siehe Kapitel \ref{id-vboxrc}) f\"{u}r
jeden Anrufer individuell eingestellt werden.



\item[{\tt beep.msg}] \mbox{}



Wird nach dem Ansagetext kurz vor der Aufnahme gespielt und sollt den
Anrufer darauf aufmerksam machen, da{\ss} er jetzt seine Nachricht
aufsprechen kann.



\item[{\tt timeout.msg}] \mbox{}



Wird gespielt, wenn die maximale Aufzeichnungszeit abgelaufen ist. Hier
k\"{o}nnte z.B. der Text "{\tt Vielen Dank f\"{u}r Ihren Anruf}" gesprochen
werden.



\item[{\tt panic.msg}] \mbox{}



Wird gespielt, wenn die Nachricht des Anrufers nicht aufgezeichnet werden
konnte. Hier k\"{o}nnte z.B. der Text "{\tt Ihr Anruf konnte leider nicht
aufgezeichnet werden. Bitte versuchen Sie es zu einem sp\"{a}teren Zeitpunk
noch einmal}" gesprochen werden.



\end{description}


Einige Beispielansagen befinden sich im Verzeichnis
`{\tt samples/messages}'.

{\bf Bemerkung:} {\em vboxgetty\/} sollte am besten aus der
`{\tt /etc/inittab}', aus `{\tt /etc/rc.d/rc.local}' oder vom Benutzer
{\tt root} gestartet werden. Wenn {\em vboxgetty\/} von einem anderen
Benutzer gestartet werden soll, ist eine Anpassung der Zugriffsrechte f\"{u}r
einige Dateien n\"{o}tig (z.B. die Devices oder die Logdateien).








\subsection{vbox\label{id-vbox}}



{\em vbox\/} ist eine auf {\em ncurses\/} basierende Benutzeroberfl\"{a}che
zum Ansehen und Abspielen der aufgezeichneten Nachrichten.

Beim Programmstart k\"{o}nnen folgende Argumente angegeben werden:

\begin{description}


\item[{\tt -d DEVICE}] \mbox{}



Name des Devices f\"{u}r die Bildschirm Ein- und Ausgabe. Fehlt diese Angabe,
wird `{\tt /dev/tty}' (aktueller Bildschirm) benutzt.

{\bf Beispiel:} {\tt vbox -d /dev/tty12}



\item[{\tt -u BENUTZERNAME}] \mbox{}



Name des Benutzers, dessen Nachrichten eingesehen werden sollen. Diese
Option kann nur benutzt werden, wenn {\em vbox\/} vom Benutzer
{\tt root} gestartet wird. {\em vbox\/} l\"{a}uft dann im sogenannten
`viewmode', in dem alle Funktionen gesperrt sind und nur die Liste der
aufgezeichneten Nachrichten angezeigt wird.

Sofort nach dem Start setzt {\em vbox\/} die Rechte auf den angegebenen
Benutzer zur\"{u}ck.

{\bf Beispiel:} {\tt vbox -u michael}



\item[{\tt -r SEKUNDEN}] \mbox{}



Diese Option ist eigentlich nur im Zusammenhang mit `-u' sinnvoll und
gibt an, nach wievielen Sekunden die Nachrichtenliste neu eingelesen
werden soll.

{\bf Beispiel:} {\tt vbox -r 60}



\end{description}


Beim booten des Rechners k\"{o}nnte {\em vbox\/} z.B. wie folgt aufgerufen
werden:

{\tt \$ vbox -d /dev/tty12 -u michael -r 60 \&}

In diesem Beispiel w\"{u}rde {\em vbox\/} alle Nachrichten des Benutzers
`michael' auf `{\tt /dev/tty12}' (Konsole 12) anzeigen und alle 60
Sekunden pr\"{u}fen, ob sich die Liste ge\"{a}ndert hat.

Wenn {\em vbox\/} von einem normalen Benutzer (also nicht von
{\tt root}) aufgerufen und nicht in den `viewmode' geschaltet wurde,
stehen folgende Funktionen zur Verf\"{u}gung:

\begin{itemize}
\item {\tt [CURSOR HOCH]} und {\tt [CURSOR RUNTER]}


Bewegt den Markierbalken in der Liste hoch oder runter. Alle Funktionen
beziehen sich immer auf den gerade angew\"{a}hlten Eintrag.



\item `{\tt P}', `{\tt p}' oder {\tt [RETURN]}


Spielt den aktuellen Eintrag mit dem in `{\tt src/settings.h}' (siehe
Kapitel \ref{id-install-settings}) unter {\tt VBOXPLAYCMD} eingestellten Programm.

Dem Programm wird beim Aufruf als erstes Argument der Name der Datei und
als zweites Argument die eingestellte Lautst\"{a}rke \"{u}bergeben.



\item `{\tt R}' oder `{\tt r}'


Liest die Liste der Nachrichten neu ein. Diese wird auf dem Bildschirm nur
dann neu dargestellt, wenn \"{A}nderungen gefunden werden konnten.



\item `{\tt N}' oder `{\tt n}'


Markiert einen Eintrag als neu oder als gelesen.



\item `{\tt D}' oder `{\tt d}'


Markiert einen Eintrag als gel\"{o}scht oder hebt diese Markierung auf.



\item `{\tt +}' oder `{\tt -}'


Setzt die Lautst\"{a}rke hoch oder runter.



\item `{\tt Q}' oder `{\tt q}'


Beendet das Programm. Wenn Eintr\"{a}ge als gel\"{o}scht markiert wurden, erscheint
zus\"{a}tzlich eine Sicherheitsabfrage, ob diese auch auf dem Datentr\"{a}ger
gel\"{o}scht werden sollen.



\item `{\tt O}'


Schaltet die Annahme von Anrufen ein oder aus. Dazu wird im
Homeverzeichnis des Benutzers eine Datei mit dem Namen
`{\tt .vboxstop}' erzeugt oder gel\"{o}scht. Existiert diese Datei, nimmt
{\em vboxgetty\/} keine Anrufe mehr entgegen.

Die Datei kann auch von Hand angelegt oder gel\"{o}scht werden.



\end{itemize}


{\em vbox\/} nimmt nur Nachrichten in die Liste auf, die ein gesetztes
{\tt r}-Bit (Lesbar) haben. Gelesene Nachrichten werden mit dem
{\tt x}-Bit (Ausf\"{u}hrbar) versehen.






\subsection{rmdcutheader\label{id-rmdcutheader}}



Das Programm {\em rmdcutheader\/} entfernt den {\tt RAW MODEM DATA
HEADER} einer Voicedatei.

Der Aufruf ist denkbar einfach:

{\tt \$ rmdcutheader {\(<\)}DATEI\_MIT\_HEADER {\(>\)}DATEI\_OHNE\_HEADER}

Um zum Beispiel Nachrichten, die im Format {\tt ULAW} aufgezeichnet
wurden, nach `{\tt /dev/audio}' zu spielen, kann man wie folgt vorgehen:

{\tt \$ rmdcutheader {\(<\)}VOICEMESSAGE {\(>\)}/dev/audio}






\subsection{rmdcatheader\label{id-rmdcatheader}}



Mit {\em rmdcatheader\/} kann ein {\tt RAW MODEM DATA HEADER} f\"{u}r die
Formate {\tt ADPCM-2}, {\tt ADPCM-3}, {\tt ADPCM-4}, {\tt ALAW}
oder {\tt ULAW} erzeugt werden.

Folgende Argumente k\"{o}nnen \"{u}bergeben werden:

\begin{description}


\item[{\tt -2, --adpcm2}] \mbox{}

Erzeugt einen {\tt RAW MODEM DATA HEADER} mit der Kompression
{\tt ADPCM-2}.



\item[{\tt -3, --adpcm3}] \mbox{}

Erzeugt einen {\tt RAW MODEM DATA HEADER} mit der Kompression
{\tt ADPCM-3}.



\item[{\tt -4, --adpcm4}] \mbox{}

Erzeugt einen {\tt RAW MODEM DATA HEADER} mit der Kompression
{\tt ADPCM-4}.



\item[{\tt -u, --ulaw}] \mbox{}

Erzeugt einen {\tt RAW MODEM DATA HEADER} mit der Kompression
{\tt ULAW}.



\item[{\tt -a, --alaw}] \mbox{}

Erzeugt einen {\tt RAW MODEM DATA HEADER} mit der Kompression
{\tt ALAW}.



\end{description}


{\bf Bemerkung:} Alle Header haben als Modemtyp `{\tt ZyXEL 1496}'
eingetragen.






\section{Konfiguration\label{id-konfiguration}}








\subsection{vboxrc\label{id-vboxrc}}



Jeder Benutzer kann sich in seinem Homeverzeichnis eine Datei mit dem Namen
`{\tt .vboxrc}' anlegen, in der einige Einstellungen f\"{u}r {\em vbox\/} und
{\em vboxgetty\/} gemacht werden k\"{o}nnen.

Die Datei ist in verschiedene Sektionen unterteilt. Jede Sektion beginnt mit
einem `{\tt [}', enth\"{a}lt einen Sektionsnamen und endet mit einem
`{\tt ]}'.

Die Gro{\ss}- und Kleinschreibung zwischen den Sektionskennungen `{\tt [}'
und `{\tt ]}' wird ignoriert. Leere Zeichen und solche die mit einem
`{\tt \#}' beginnen werden \"{u}berlesen. Kommentare am Ende einer Zeile
(beginnen mit `{\tt \#}') werden entfernt.

Folgende Sektionen existieren:

\begin{description}


\item[{\tt [CALLERID]}] \mbox{}



{\bf Format:} {\tt CALLERID-PATTERN ALIAS}

Hier k\"{o}nnen sogenannte `Aliase' f\"{u}r {\tt CALLERID}'s angegeben werden.
Die {\tt CALLERID} wird bei ISDN-Ger\"{a}ten - sofern nicht abgeschaltet -
mit \"{u}bermittelt und dient zum Idendifizieren des Anrufers.

\begin{description}


\item[{\tt CALLERID-PATTERN}] \mbox{}



{\tt CALLERID} f\"{u}r die ein Alias definiert werden soll (ohne
f\"{u}hrende Nullen). Es k\"{o}nnen hier die \"{u}blichen UN*X-Pattern angegeben
werden. In der {\tt CALLERID} d\"{u}rfen keine Leerzeichen enthalten
sein.



\item[{\tt ALIAS}] \mbox{}



Name, der dieser {\tt CALLERID} zugeordnet werden soll. Der
Aliasname darf Leerzeichen enthalten.



\end{description}


Als letzter Eintrag sollte {\bf immer} `{\tt * *** Unknown ***}'
angegeben werden (siehe Beispiel). Dieses Pattern gilt f\"{u}r alle Anrufe,
bei denen keine {\tt CALLERID} \"{u}bermittelt oder in der Liste gefunden
wurde.

{\bf Beispiel:}

\begin{verbatim}
   [CALLERID]

   913469992[6-8]          Bernd Braun
   91437850413             Michael Herold
   *                       *** Unknown ***
   
\end{verbatim}


Die {\bf erste} \"{U}bereinstimmung einer {\tt CALLERID} mit einem der
Pattern wird benutzt.



\item[{\tt [DEBUG]}] \mbox{}



{\bf Format:} {\tt FLAG + FLAG + FLAG ...}

In dieser Sektion k\"{o}nnen Flag's angegeben werden, die bestimmen, welche
Informationen in die Logdatei von {\em vboxgetty\/} aufgenommen werden
sollen.

Folgende Flag's k\"{o}nnen angegeben werden:

\begin{itemize}
\item {\tt FATAL}
Fatale Fehler



\item {\tt ERRORS}
Normal Fehler



\item {\tt WARNINGS}
Warnungen



\item {\tt INFOS}
Informationen



\item {\tt DEBUG}
Informationen zum Debuggen



\item {\tt JUNK}
Erweiterte Informationen zum Debuggen



\end{itemize}


{\bf Beispiel:} {\tt FATAL + ERRORS + WARNINGS + INFOS}



\item[{\tt [RINGS]}] \mbox{}



{\bf Format:} {\tt ZEIT RINGS}

Hier kann angegeben werden, zu welcher Zeit nach wievielen {\tt RING}'s
ein eingehender Anruf entgegengenommen werden soll.

\begin{description}


\item[{\tt ZEIT}] \mbox{}



Eine oder mehrere Zeitzonen; Die Zeitangaben k\"{o}nnen durch Kommata
voneinander getrennt werden. Bei einer Zeitzone von
`{\tt STARTZEIT}-{\tt ENDZEIT}' ist die letzte Zahl inklusiv
(z.B. {\tt 18-22} hei{\ss}t von 18 bis einschlie{\ss}lich 22 Uhr).



\item[{\tt RINGS}] \mbox{}



Anzahl der {\tt RING}'s bis ein Anruf entgegengenommen wird. Eine
Anzahl von 0 {\tt RING}'s hei{\ss}t, da{\ss} der Anruf nicht
entgegengenommen werden soll.



\end{description}


{\bf Beispiel:}

\begin{verbatim}
      0-8     5 
      *       30
   
\end{verbatim}


Im oben angegebenen Beispiel w\"{u}rde {\em vboxgetty\/} von 0 bis 8 Uhr nach
5 {\tt RING}'s und zu allen anderen Zeiten nach 30 {\tt RING}'s den
Anruf entgegennehmen.

Die {\bf erste} \"{U}bereinstimming der aktuellen Zeit mit einer
Zeitangabe wird benutzt.

{\bf Achtung!}

Beim einem Fehler (z.B. einer Falschen Zeitangabe) wird immer 0
{\tt RING}'s angenommen. Achten Sie also bitte auf die Ausgabe in der
Logdatei um solche Situationen auszuschlie{\ss}en!



\item[{\tt Alias Sektionen}] \mbox{}



{\bf Format:} {\tt ZEIT ANSAGETEXT AUFNAHMEZEIT}

Mit {\tt ALIAS SEKTIONEN} k\"{o}nnen f\"{u}r Aliase, die in der Sektion
{\tt [CALLERID]} definiert wurden, eigene Einstellungen gemacht
werden.

\begin{description}


\item[{\tt ZEIT}] \mbox{}



Eine oder mehrere Zeitzonen; Die Zeitangaben k\"{o}nnen durch Kommata
voneinander getrennt werden. Bei einer Zeitzone von
`{\tt STARTZEIT}-{\tt ENDZEIT}' ist die letzte Zahl inklusiv
(z.B. {\tt 18-22} hei{\ss}t von 18 bis einschlie{\ss}lich 22 Uhr).



\item[{\tt ANSAGETEXT}] \mbox{}



Ansagetext der gespielt werden soll. Der Ansagetext wird aus dem
{\tt Message}-Verzeichnis des Benutzers geladen, f\"{u}r den
{\em vboxgetty\/} Nachrichten aufzeichnen soll.



\item[{\tt AUFNAHMEZEIT}] \mbox{}



Maximale Aufzeichnungszeit in Sekunden. Eine Aufzeichnungszeit von 0
Sekunden unterdr\"{u}ckt die Aufzeichnung.



\end{description}


Wenn f\"{u}r eine {\tt CALLERID} keine {\tt ALIAS SEKTION} gefunden
wurde, wird immer die Datei `{\tt standard.msg}' als Ansagetext
gespielt und 60 Sekunden lang aufgezeichnet.

{\bf Beispiel:}

Haben Sie Beispielsweise in der Sektion {\tt [CALLERID]} eine Alias
`Peter Muster' mit der {\tt CALLERID} 931271111 definiert, k\"{o}nnen Sie
sich f\"{u}r diesen eine {\tt ALIAS SEKTION} erstellen:

\begin{verbatim}
   [Peter Muster]

   0-7,8                   standard.msg            30
   *                       standard.msg            60
   
\end{verbatim}


Jedesmal wenn bei einem Anruf die {\tt CALLERID} 931271111 erkannt
wird, werden die Einstellungen der {\tt ALIAS SEKTION} f\"{u}r den Alias
der {\tt CALLERID} (also in diesem Beispiel `Peter Muster') benutzt.

{\bf Bemerkung:} Legen Sie sich eine {\tt ALIAS SEKTION} `{\tt [***
Unknown ***]}' an, die benutzt wird, wenn keine {\tt CALLERID}
ermittelt werden konnte.



\end{description}







\section{Danksagungen\label{id-danksagungen}}



Bedanken m\"{o}chte ich mich bei:

\begin{itemize}
\item {\bf Fritz Elfert} ({\tt fritz@wuemaus.franken.de})


F\"{u}r {\em isdn4linux\/}, sein Hilfe bei einigen Problemen und den
schnellen Einbau der DTMF-Erkennung.



\item {\bf Nicole Sauvage} ({\tt nicole@abadonna.franken.de})


F\"{u}r die englische \"{U}bersetzung dieser Dokumentation.



\end{itemize}


Und nat\"{u}rlich bei alle anderen, die mich mit zahlreichen neuen Ideen und
Anregungen versorgt haben.



\end{document}
