\documentstyle{article}
\setlength{\oddsidemargin}{0mm}
\setlength{\evensidemargin}{0mm}
\setlength{\textwidth}{160mm}
\setlength{\topmargin}{-10mm}
\setlength{\textheight}{210mm}
\parindent=0mm
\parskip=5mm

\def\mytitle{The JED Editor}

\newlength{\pagewidth}
\setlength{\pagewidth}{\textwidth}
%\addtolength\pagewidth\marginparwidth
% \addtolength\pagewidth\marginparsep

\makeatletter
\def\ps@headings{\def\@oddfoot{}%
\def\@oddhead{\makebox[\textwidth][l]{\underline{\hbox to \pagewidth{\bf
\mytitle\hfill\thepage}}}}%
\def\@evenfoot{}%
\def\@evenhead{\makebox[\textwidth][l]{\underline{\hbox to \pagewidth{\bf
\mytitle\hfill\thepage}}}}}%
\pagestyle{headings}

\newcommand{\slang}{{\bf S-Lang}}
\newcommand{\jed}{{\bf jed}}
\newcommand{\exmp}[1]{{\tt #1}}
\newcommand{\var}[1]{{\tt #1}}
\newcommand{\key}[1]{{\sc #1}}

\begin{document}
\tableofcontents
\pagebreak

\section{Introduction}

 This document presents some basic information that users should know in
 order to use \jed{} effectively.  Any questions, comments, or bug reports,
 should be email-ed to the author. Please be sure to include the version
 number.  To be notified of future releases of \jed{}, email to the address
 below and your email address will be placed on the announcement list. The
 email address is:
\begin{verbatim}
      davis@space.mit.edu  (Internet)
      davis@ohstpy (BITNET)
\end{verbatim}

\section{Installing JED}

 Building \jed{} from its sources requires the use of a C compiler that
 understands function prototypes.  \jed{} has been successfully built with
 \verb|cc| on the ULTRIX, VMS, and IRIX operating systems.  In addition, it
 has been created using \verb|gcc| under SunOS and Borland's BCC 3.0 for the
 PC.

 Detailed installation instructions are in separate, operating system
 dependent files.  They are:
\begin{verbatim}
       UNIX:   install.unx
        VMS:   install.vms
      IBM-PC:   install.pc
\end{verbatim}

  When \jed{} starts up, it will first try to load a site initialization file
  called \verb|site.sl|.  Site specific commands are placed here.  Most
  likely, \verb|site.sl| will define some functions, default hooks,
  etc\ldots  What goes in it is left to the discretion of the user or system
  manager. See the file \verb|site.sl| for examples.

  When loading \verb|site.sl| as well as other \verb|S-Lang| files (the
  user's personal initialization file, \verb|.jedrc| or \verb|jed.rc|, is a
  special case, see below), \jed{} searches all directories specified by the
  environment variable \verb|JED_LIBRARY|, and if the file is not found, \jed{}
  will look for it in the default directory. The environment variable
  \verb|JED_LIBRARY| is a comma separated list of directories.  Here are
  some examples of setting this variable for different systems:

\begin{verbatim}
      VMS:    define/job JED_LIBRARY  dev$lib:[jedfiles]
      UNIX:   setenv JED_LIBRARY '/usr/local/lib/jed,~/jed'
      IBMPC:  set JED_LIBRARY = c:\editors\jed\lib
\end{verbatim}


  You will probably want to put define \verb|JED_LIBRARY| in your login
  startup file, e.g., \verb|autoexec.bat|, \verb|login.com|, or
  \verb|.cshrc|.

  \jed{} versions 0.92 and later allow the value of \verb|JED_LIBRARY| to be
  specified at compile time an it may only be necessary to define
  \verb|JED_LIBRARY| as an environment variable to override its
  pre--compiled value.

\section{Startup Files}

  Once \jed{} has loaded the startup file \verb|site.sl|, it will try to load
  the user's personal initialization file.  It first looks in the directory
  pointed to by the environment variable \verb|JED_HOME|. If that fails, it
  then searches the \verb|HOME| directory and upon failure simply loads the
  one supplied in \verb|JED_LIBRARY|.

  The name of the user initialization file varies according to the operating
  system.  On Unix systems this file must be called \verb|.jedrc| while on
  VMS and MSDOS, it goes by the name \verb|jed.rc|.  For VMS systems, the
  \verb|HOME| directory corresponds to the \verb|SYS$LOGIN| logical name
  while for the other two systems, it corresponds to the \verb|HOME|
  environment variable.

  The purpose of this file is to allow an individual user to tailor \jed{} to
  his or her personal taste.  Most likely, this will involve choosing an
  initial set of key-bindings, setting some variables, and so on.

\section{Starting JED}

  Normally, \jed{} is started as

\begin{verbatim}
     jed <file name>
\end{verbatim}
  or 
\begin{verbatim}
     jed <file list>
\end{verbatim}

  However, \jed{} also takes the switches defined in the following table:
\begin{verbatim}
     Switch          Meaning
     ------          -------
      -batch         run JED in batch mode.  This is a non-interactive mode.
      -n             do not load jed.rc (.jedrc) file.
      -g <n>         goto line `n' in buffer
      -l <file>      load `file' as S-Lang code.
      -f <function>  execute S-Lang function named `function'
      -s <string>    search forward for `string'
      -2             split window
      -i <file>      insert <file> into current buffer.
\end{verbatim} 

  For example, the command line:
\begin{verbatim}
                jed slang.c -g 1012 -s error -2 file.c -f eob
\end{verbatim}


  will start up \jed{}, read in the file \var{slang.c}, goto line 1012 of
  \var{slang.c} and start searching for the string \var{error}, split the
  window, read in \var{file.c} and goto the end of the file.

  If the \var{-batch} parameter is used, it must be the first parameter.
  Similarly, if \var{-n} is used, it must also be the first parameter unless
  used with the \var{-batch} parameter in which case it must the second.
  \jed{} should only be run in batch mode when non--interactive operation is
  desired. For example, \jed{} is distributed with a file, \verb|mkdoc.sl|,
  that contains S--Lang code to produce a help file for functions and
  variables.  In fact, the help file \verb|jed_funs.hlp| was created by
  entering

\begin{verbatim}
                jed -batch -n -l mkdoc.sl
\end{verbatim}

  at the command line.

  Now suppose that you want to read in a file with the name of one of the
  switches, say \verb|-2|.  How can this be done?  The answer depends upon
  the operating system.  For Unix, instead of \var{jed -2}, use 
  \var{jed ./-2}; for VMS, use \var{jed []-2}.
  The case for MS-DOS is similar to Unix except that one must use the
  backslash.

  It is possible to completely change \jed{}'s command line syntax through the
  use of the user defined function \verb|command_line_hook|. In fact, the
  behavior described above is dictated by the value of
  \verb|command_line_hook| as distributed in \verb|site.sl|.  See the
  section on hooks for details.

\section{Emulating Other Editors}

  \jed{}'s ability to create new functions using the S--Lang programming
  language as well as allowing the user to choose key bindings, makes the
  emulation of other editors possible.  Currently, \jed{} provides reasonable
  emulation of the Emacs, EDT, and Wordstar editors.

\subsection{Emacs Emulation}

  Emacs Emulation is provided by the \slang{} code in \verb|emacs.sl|. The
  basic functionality of Emacs is emulated; most Emacs users should have no
  problem with \jed{}.  To enable Emacs emulation in \jed{}, make sure that the
  line
\begin{verbatim}
      evalfile ("emacs"); pop ():
\end{verbatim}
  is in your \verb|jed.rc| (\verb|.jedrc|) startup file.  \jed{} is distributed
  with this line already present in the default \verb|jed.rc| file.

\subsection{EDT Emulation}

  For EDT emulation, \verb|edt.sl| must be loaded.  This is accomplished by
  ensuring that the line

\begin{verbatim}
      evalfile ("edt"); pop ();
\end{verbatim}


  is in present in the \verb|jed.rc| (\verb|.jedrc|) Startup File.  \jed{} is
  distributed with EDT emulation enabled on VMS and Unix systems but the
  above line is commented out in the \verb|jed.rc| file on MS-DOS systems.

  This emulation provides a near identical emulation of the EDT keypad key
  commands.  In addition, the smaller keypad on the newer DEC terminals is
  also setup.  It is possible to have both EDT and Emacs emulation at the
  same time.  The only restriction is that \verb|emacs.sl| must be loaded
  before \verb|edt.sl| is loaded.

  One minor difference between \jed{}'s EDT emulation and the real EDT concerns
  the \key{Ctrl-H} key.  EDT normally binds this to move the cursor to the
  beginning of the line.  However, \jed{} uses it as a help key. Nevertheless,
  it is possible to re-bind it.  See the section on re-binding keys as well as
  the file \verb|edt.sl| for hints.  Alternatively, simply put
\begin{verbatim}
      unsetkey ("^H"); setkey ("bol", "^H");
\end{verbatim}
  in the \verb|jed.rc| startup file after \verb|edt.sl| is loaded.  Keep in
  mind that the \key{Ctrl-H} key will no longer function as a help key if
  this is done.

  EDT emulation for PCs only work with the enhanced keyboard.  When
  \verb|edt.sl| is loaded, a variable \verb|NUMLOCK_IS_GOLD| is set which
  instructs \jed{} to interpret the Num-Lock key on the square numeric keypad to
  function as the EDT GOLD key.  In fact, this keypad should behave exactly
  like the keypad on VTxxx terminals.  The only other problem that remains
  concerns the \key{+} key on the PC keypad.  This key occupies two VTxxx
  key positions, the minus and the comma (delete word and character) keys.
  Thus a decision had to be made about which key to emulate.  I chose the
  \key{+} key to return the characters \key{Esc O l} which \jed{} maps
  to the delete character function.  This may be changed to the delete word
  function if you prefer.  See the file \verb|edt.sl| for details.

  The \key{GOLD}--\key{GOLD} key combination toggles the keypad between
  application and numeric states.  On the PC, this is not possible. Instead,
  the PC \key{F1} key has been instructed to perform this task.

\subsection{Wordstar Emulation}

  \verb|wordstar.sl| contains the \slang{} code for \jed{}'s Wordstar
  emulation. Adding the line
\begin{verbatim}
      evalfile ("wordstar"); pop ();
\end{verbatim}
  to your \verb|jed.rc| (\verb|.jedrc|) startup file will enable \jed{}'s
  Wordstar emulation.


\section{File Types and Sizes}

  \jed{} is primarily a text editor; however, it can also edit binary files
  (see the section on editing binary files).  As a result, \jed{} may edit
  lines of arbitrary length (actually this depends upon the size of an
  integer).  It is capable of editing arbitrarily large buffers as long as
  there is enough memory for the buffer as well as the overhead involved.
  This editor employs a linked list representation; hence, the overhead can
  be quite high.

\section{Backup and Auto-save Files}

  On UNIX and MS-DOS systems, \jed{} creates backup files by appending a \var{~}
  character to the filename.  The VMS operating system handles backup files
  itself.  \jed{} periodically auto-saves its buffers.  On UNIX and MS-DOS,
  auto-save files are prefixed with the pound sign \var{\#}.  On VMS, they are
  prefixed with \verb|\_$|.  The auto-save interval may be changed by setting
  the variable \verb|MAX_HITS| to the desired value. The default is 300
  ``hits'' on the buffer. A ``hit'' is defined as a key which MAY change the
  state of the buffer. Cursor movement keys do not cause hits on the buffer.

  Like many of \jed{}'s features, the names of auto-save and backup files can be
  controlled by the user.  The file \verb|site.sl| defines two functions,
  \verb|make_backup_filename|, and \verb|make_autosave_filename| that
  generate the file names described in the previous paragraph.  Like all
  S--Lang functions, these functions may be overloaded and replaced with
  different ones.  See also information about \verb|find_file_hook| in the
  section on hooks.

  On UNIX systems, \jed{} catches most signals and tries to auto-save its
  buffers in the event of a crash or if the user accidently disconnects from
  the system (\var{SIGHUP}).

  If an auto-save file exists and you is desire to recover data from the
  auto-save file, use the function \verb|recover_file|.  Whenever \jed{} finds a
  file, it checks to see if an auto-save file exists as well as the file's
  date.  If the dates are such that the auto-save file is more recent \jed{}
  will display a message in the mini-buffer alerting the user of this fact
  and that the function \verb|recover_file| should be considered.

\section{Status line and Windows}

  \jed{} supports multiple windows.  Each window may contain the same buffer or
  different buffers.  A status line is displayed immediately below each
  window.  The status line contains information such as the \jed{} version
  number, the buffer name, ``mode'', etc.  Please beware of the following
  indicators:

   \var{**}\\
        buffer has been modified since last save.
        
   \var{\%\%}\\
        buffer is read only.
        
   \var{m}\\
        Mark set indicator.  This means a region is being defined.
        
   \var{d}\\
        File changed on disk indicator.  This indicates that the
        file associated with the buffer is newer than the buffer itself.
        
   \var{s}\\
        spot pushed indicator.
        
   \var{+}\\
        Undo is enabled for the buffer.
        
   \var{[Macro]}\\
        A macro is being defined.
        
   \var{[Narrow]}
        Buffer is narrowed to a region of LINES.

\section{Mini-Buffer}

  The Mini-Buffer consists of a single line located at the bottom of the
  screen. Much of the dialog between the user and \jed{} takes place in this
  buffer.  For example, when you search for a string, \jed{} will prompt you
  for the string in the Mini-Buffer.

  The Mini-Buffer also provides a direct link to the \slang{} interpreter. To
  access the interpreter, press \key{Ctrl-X} \key{Esc} and the \var{S-Lang>}
  prompt will appear in the Mini-Buffer.  Enter any valid \slang{} expression
  for evaluation by the interpreter.

  It is possible to recall data previously entered into the Mini-Buffer by
  using the up and down arrow keys.  This makes it possible to use and edit
  previous expressions in a convenient and efficient manner.

\subsection{Command Line Completion}

  The \jed{} editor has several hundred built--in functions as well as many
  more written in the \slang{} extension language.  Many of these functions
  are bound to keys and many are not.  It is simply unreasonable to require
  the user to remember if a function is bound to a key or not and, if it is,
  to remember the key to which it is bound. This is especially true of those
  functions that are bound but rarely used.  More often than not, one simply
  forgets the exact name or spelling of a function and requires a little
  help.  For this reason, \jed{} supports command line completion in the
  mini-buffer.  This function, called \verb|emacs_escape_x|, is bound to the
  key \key{Esc X}. This is one binding that must be remembered!

  As an example, suppose that you are editing several buffers and you wish
  to insert the contents of one buffer into the current buffer.  The
  function that does this is called \verb|insert_buffer| and has no default
  key-binding. Pressing \key{Esc X} produces the prompt \var{M-x}. This
  prompt, borrowed from the Emacs editor, simply means that \key{Esc X}
  was pressed.  Now type \var{in} and hit the space bar or the \key{Tab}
  key. In this context (completion context) the space bar and the \key{Tab}
  will expand the string in the Mini-Buffer up until it is no longer unique.
  In this case, \verb|insert_file| and \verb|insert_buffer| are only the two
  functions that start with \var{in}.  Hence, \var{in} will expand to
  \verb|insert_| at which point it becomes necessary to enter more
  information to uniquely specify the desired function.  However, in a
  completion context, the space bar also has a special property that enables
  the user to cycle among the possible completions.  For this example,
  hitting the space bar twice consecutively will produce the string
  \verb|insert_file| and hitting it again produces the desired string
  \verb|insert_buffer|.

  The role of the space bar in completion is a point where Emacs and \jed{}
  differ. Emacs will pop up a buffer of possible completions but \jed{} expects
  the user to press the space bar to cycle among them.  Both have there pros
  and cons.  Frequently, one sees messages on the Usenet newsgroup
  \verb|gnu.emacs.help| from Emacs users asking for the kind of completion
  \jed{} employs.

\subsection{File Names}

  \jed{} takes every file name and ``expands it'' according to a set of rules
  which vary according to the Operating System.  For concreteness, consider
  \jed{} running under MS-DOS.  Suppose the user reads a new file into the
  editor via the \verb|find_file| command which emacs binds to \key{Ctrl-X}
  \key{Ctrl-F}. Then the following might be displayed in the mini-buffer:

\begin{verbatim}
      Find File: C:\JED\SLANG\
\end{verbatim}


  Here \jed{} is prompting for a file name in the directory \verb|\JED\SLANG|
  on disk \var{C:}. However, suppose the user wants to get the file
  \verb|C:\JED\SRC\VIDEO.C|. Then the following responses produce equivalent
  filenames when \jed{} expands them internally:
\begin{verbatim}
      Find File: C:\JED\src\video.c
      Find File: C:\JED\SLANG\..\src\video.c
      Find File: C:\JED\SLANG\../src/video.c
\end{verbatim}

  Note that the on MS-DOS systems, \jed{} replaces the \var{/} with a \verb|\|
  and that case is not important.  Now suppose you wish to get the file
  \var{VIDEO.C} from disk \var{A:}.  The following are also valid:

\begin{verbatim}
      Find File: A:\video.c
      Find File: A:video.c
      Find File: C:\JED\SLANG\a:\video.c
\end{verbatim}

  In the last case, \jed{} is smart enough to figure out what is really meant.
  Although the above examples are for MS-DOS systems, the rules also apply to
  Unix and VMS systems as well.  The only change is the file name syntax.
  For example, on VMS
\begin{verbatim}
      sys$manager:[misc]dev$user:[davis.jed]vms.c
      dev$user:[davis.jed]vms.c
\end{verbatim}
  become equivalent filenames upon expansion.  For unix, the following are
  equivalent:

\begin{verbatim}
      /user1/users/davis/jed/unix.c
      /usr/local/src//user1/users/davis/jed/unix.c
      /usr/local/src/~/jed/unix.c
\end{verbatim}

  Note the last example: the tilde character \verb|~| always expands into
  the users \verb|HOME| directory, in this case to \verb|/user1/users/davis|.

  When \jed{} writes a buffer out to a file, it usually prompts for a file name
  in the minibuffer displaying the directory associated with the current
  buffer.  At this point a name can be appended to the directory string to
  form a valid file name or the user may simply hit the \key{RET} key.  If
  the latter alternative is chosen, \jed{} simply writes the buffer to the file
  already associated with the buffer.  Once the buffer is written to a file,
  the buffer becomes attached to that file.

\subsection{Buffer Name and File Name Completion}

  When \jed{} prompts for a file name or a buffer name, the space bar and
  the \key{Tab} keys are special.  Hitting the \key{Tab} key will complete
  the name that is currently in the minibuffer up until it is no longer
  unique. At that point, you can either enter more characters to complete
  the name or hit the space bar to cycle among the possible completions.
  The spacebar must be pressed at least twice to cycle among the
  completions.

  On MSDOS and VMS, it is possible to use wildcard characters in the file
  name for completion purposes.  For example, entering \key{*.c} and hitting
  the space bar will cycle among file names matching \key{*.c}.
  Unfortunately, this feature is not available on unix systems.

\section{Basic Editing}

  Editing with \jed{} is pretty easy--- most keys simply insert themselves.
  Movement around the buffer is usually done using the arrow keys or page up
  and page down keys.  If \verb|edt.sl| is loaded, the keypads on VTxxx
  terminals function as well.  Here, only the highlights are touched upon
  (cut/paste operations are not considered ``highlights''). In the
  following, any character prefixed by the \verb|^| character denotes a
  Control character. On keyboards without an explicit Escape key,
  \key{Ctrl-[} will most likely generate and Escape character.

  A ``prefix argument'' to a command may be generated by first hitting the
  \key{Esc} key, then entering the number followed by pressing the desired
  key.  Normally, the prefix argument is used simply for repetition.  For
  example, to move to the right 40 characters, one would press
  \key{Esc 4 0} followed immediately by the right arrow key. This
  illustrates the use of the repeat argument for repetition.  However, the
  prefix argument may be used in other ways as well.  For example, to begin
  defining a region, one would press the \key{Ctrl-@} key.  This sets the
  mark and begins highlighting.  Pressing the \key{Ctrl-@} key with a prefix
  argument will abort the act of defining the region and to pop the mark.

  The following list of useful keybindings assumes that \verb|emacs.sl| has
  been loaded.

   \key{Ctrl-L}\\
     Redraw screen.

   \key{Ctrl-\_}\\
     Undo  (Control-underscore, also \key{Ctrl-X u}).

   \key{Esc q}\\
     Reformat paragraph (wrap mode).  Used with a prefix
     argument. will justify the paragraph as well.

    \key{Esc n}\\
                narrow paragraph (wrap mode).  Used with a prefix
                argument will justify the paragraph as well.

    \key{Esc ;}\\
                Make Language comment (Fortran and C)
                
    \key{Esc} \verb|\|\\
                Trim whitespace around point
                
    \key{Esc !}\\
                Execute shell command
                
    \key{Esc \$}\\
                Ispell word (unix)
                
    \key{Ctrl-X ?}\\
                Show line/column information.
                
      \key{`}\\
                \verb|quoted_insert| --- insert next char as is (backquote key)
                
    \key{Esc s}\\
                Center line.
                
    \key{Esc u}\\
                Upcase word.
                
    \key{Esc d}\\
                Downcase word.
                
    \key{Esc c}\\
                Capitalize word.
                
    \key{Esc x}\\
                Get M-x minibuffer prompt with command completion
                
    \key{Ctrl-X Ctrl-B}\\
                pop up a list of buffers
                
    \key{Ctrl-X Ctrl-C}\\
                exit \jed{}
                
    \key{Ctrl-X 0}\\
                Delete Current Window
                
    \key{Ctrl-X 1}\\
                One Window.
                
    \key{Ctrl-X 2}\\
                Split Window.
                
    \key{Ctrl-X o}\\
                Other window.
                
    \key{Ctrl-X b}\\
                switch to buffer
                
    \key{Ctrl-X k}\\
                kill buffer
                
    \key{Ctrl-X s}\\
                save some buffers
                
    \key{Ctrl-X Esc}\\
                Get \var{S-Lang>} prompt for interface to the \slang{}
                interpreter. 
                
    \key{Esc .}\\
                Find tag (unix ctags compatible)

    \key{Ctrl-@}\\
                Set Mark (Begin defining a region).  Used with a
                prefix argument aborts the act of defining the region and
                pops the Mark.


\subsection{Undo}

   One of \jed{}'s nicest features is the ability to undo nearly any change
   that occurs within a buffer at the touch of a key.  If you delete a word,
   you can undo it.  If you delete 10 words in the middle of the buffer,
   move to the top of the buffer and randomly make changes, you can undo all
   of that too.

   By default, the \verb|undo| function is bound to the key \key{Ctrl-\_}
   (Ascii 31). Since some terminals are not capable of generating this
   character, it is also bound to the key sequence \key{Ctrl-X} u.

   Due to the lack of virtual memory support on IBMPC systems, the
   \verb|undo| function is not enabled on every buffer.  In particular, it
   is not enabled for the \verb|*scratch*| buffer.  However, it is enabled
   for any buffer which is associated with a file.  A ``plus'' character on
   the left hand side of the status line indicates that undo is enabled for
   the buffer. It is possible to enable undo for any buffer by using the
   \verb|toggle_undo| function.

\subsection{Marking Text (Point and Mark)}

   Many commands work on certain regions of text.  A region is defined by
   the \verb|Point| and the \verb|Mark| The \verb|Point| is the location of
   the current editing point or cursor position.  The \verb|Mark| is the
   location of a mark. The mark is set using the \verb|set_mark_cmd| which
   is bound to \key{Ctrl-@} (Control-2 or Control-Space on some keyboards).
   When the mark is set, the \var{m} mark indicator will appear on the
   status line.  This indicates that a region is being defined.  Moving the
   cursor (\verb|Point|) defines the other end of a region.  If the variable
   \var{HIGHLIGHT} is non-zero, \jed{} will highlight the region as it is
   defined.

Even without highlighting, it is easy to see where the location of the mark
is by using the \verb|exchange| command which is bound to \key{Ctrl-X} \key{Ctrl-X}.
This simply exchanges the \verb|Point| and the \verb|Mark|.  The region is
still intact since it is defined only by the \verb|Point| and \verb|Mark|.
Pressing \key{Ctrl-X} \key{Ctrl-X} again restores the mark and Point back to their
original locations.  Try it.

\subsection{Tab Issues.}

  Strictly speaking, \jed{} uses only fixed column tabs whose size is
  determined by the value of the \verb|TAB| variable. Setting the \verb|TAB|
  variable to 0 causes \jed{} to not use tabs as whitespace and to display
  tabs as \key{Ctrl-I}. Please note that changing the tab settings on the
  terminal will have no effect as far as \jed{} is concerned.  The
  \verb|TAB| variable is local to each buffer allowing every buffer to have
  its own tab setting.  The variable \verb|TAB_DEFAULT| is the tab setting
  that is given to all newly created buffers.  The default value for this
  variable is 8 which corresponds to eight column tabs.

  \jed{} is also able to ``simulate'' arbitrary tabs as well through the use
  of user defined tab stops.  Calling the function \verb|edit_tab_stops|
  allows the user to interactively set the tab stops.  That is, one simply
  presses \key{Esc X} to get the \verb|M-x| prompt and enters
  \verb|edit_tab_stops|. A window will pop open displaying the current tab
  settings.  To add a tab stop, simply place a \key{T} in the appropriate
  column.  Use the space bar to remove a tab stop.

  Here an argument is presented in favor of simulated tabs over real tab
  stops. First, consider what a ``tab'' really is.  A ``tab'' in a file is
  nothing more than a character whose ASCII value is 9.  For this reason,
  one also denotes a tab as \verb|^I| (\key{Ctrl-I}).  Unlike most other
  ASCII characters, the effect of the tab character is device dependent and
  is controlled through the device tab settings.  Hence, a file which
  displays one way on one device may look totally different on another
  device if the tab settings do not correspond. For this reason, many people
  avoid tabs altogether and others the adopt ``standard'' of eight column
  tabs.  Even though people always argue about what the correct tab settings
  should be, it must be kept in mind that this is primarily a human issue
  and not a machine issue.

  On a device employing tab stops, a tab will cause the cursor to jump to
  the position of the next tab stop.  Now consider the effect of changing
  the tab settings.  Assume that in one part of a document, text was entered
  using the first setting and in another part, the second setting was used.
  When moving from the part of the document where the current tab setting is
  appropriate to the part where the other tab setting was used will cause
  the document to look unformatted unless the appropriate tab settings are
  restored. Wordprocessors store the tab settings in the file with the text
  so that the tabs may be dynamically changed to eliminate such unwanted
  behavior. However, text editors such as \jed{}, vi, Emacs, EDT, EVE (TPU),
  etc, do not store this information in the file. \jed{} avoids this problem
  by using simulated tabs.  When using simulated tabs, tabs are not really
  used at all. Rather \jed{} inserts the appropriate number of spaces to
  achieve the desired effect.  This also has the advantage of one being able
  to cut and paste from the part of a document using one tab setting to
  another part with a different tab setting.  This simple operation may lead
  to unwanted results on some wordprocessors as well as those text editors
  using real tab stops.

\subsection{Searching}

  \jed{} currently has two kinds of searches: ordinary searches and
  incremental searches.  Both types of searches have forward and backward
  versions.  The actual functions for binding purposes are:

\begin{verbatim}
      search_forward
      search_backward
      isearch_forward
      isearch_backward
\end{verbatim}

  There is also the \verb|occur| function which finds all occurrences of a
  single word (string).  This function has no backwards version. By default
  it is not bound to any keys, so to use it, \verb|occur| must be entered at
  the \verb|M-x| prompt (\key{Esc X}) or one is always free to bind it
  to a key.

  In the following only the incremental search is discussed.

  The default type of search in Emacs in the incremental search. However,
  since this type of search is confusing to the uninitiated, the ordinary
  type of search has been made the default in \jed{}'s Emacs emulation. For the
  traditional emacs keybinding, it is up to the user to provide the
  keybinding.

  As the name suggests, an incremental search performs a search
  incrementally. That is, as you enter the search string, the editor begins
  searching right away.  For example, suppose you wish to search for the
  string \var{apple}. As soon as the letter \key{a} is entered into the
  incremental search prompt, \jed{} will search for the first occurrence of
  \var{a}. Then as soon as the \key{p} is entered, \jed{} will search from the
  current point for the string \var{ap}, etc. This way, one is able to
  quickly locate the desired string with only a minimal amount of
  information.

  Unlike the ``ordinary'' search, the incremental search is not terminated
  with the \key{Enter} key.  Hitting the \key{Enter} key causes \jed{} to
  search for the next occurrence of the string based on the data currently
  entered at the prompt.  The search is terminated with the \key{Esc} key.

  Finally, the \key{DEL} key (\key{Ctrl-?}) is used to erase the last
  character entered at the search prompt.  In addition to erasing the last
  character of the search string, \jed{} will return back to the location of
  the previous match. Erasing all characters will cause the editor to return
  to the place where the search began.  Like many things, this is one of
  those that is easier to do than explain.  Feel free to play around with
  it.

\subsection{Rectangles}

  \jed{} has built-in support for the editing of rectangular regions of
  text. One corner of rectangle is defined by setting the mark somewhere in
  the text.  The Point (cursor location) defines the opposite corner of the
  rectangle.

  Once a rectangle is defined, one may use the following functions:

      \verb|kill_rect|\\
         Delete text inside the rectangle saving the rectangle
         in the internal rectangle buffer.
         
      \verb|n_rect|\\
         Push all text in the rectangle to the right outside the
         rectangle.
         
      \verb|copy_rect|\\
         Copy text inside the rectangle to the internal rectangle
         buffer.
         
      \verb|blank_rect|\\
         Replace all text inside the rectangle by spaces.


  The function \verb|insert_rect| inserts a previously killed or copied
  rectangle into the text at the Point.

  These functions have no default binding and must be entered into the
  MiniBuffer by pressing \key{Esc X} to produce the \verb|M-x| prompt.

\subsection{Sorting}

  \jed{} is capable of sorting a region of lines using the heapsort
  algorithm. The region is sorted alphabetically based upon the ASCII values
  of the characters located within a user defined rectangle in the region.
  That is, the rectangle simply defines the characters upon what the sort is
  based. Simply move to the top line of the region and set the mark on the
  top left corner of the rectangle.  Move to the bottom line and place the
  point at the position which defines the lower right corner of the
  rectangle. Press \key{Esc X} to get the \verb|M-x| prompt and enter
  \verb|sort| As as example, consider the following data:

\begin{verbatim}
      Fruit:             Quantity:
      lemons                3
      pears                37
      peaches             175
      apples              200
      oranges              56
\end{verbatim}

  To sort the data based upon the name, move the Point to the top left
  corner of the sorting rectangle.  In this case, the Point should be moved
  to the \var{l} in the word \var{lemons}.  Set the mark.  Now move to the
  lower right corner of the rectangle which is immediately after the \var{s}
  in \var{oranges}.  Pressing \key{Esc X} and entering \verb|sort| yields:

\begin{verbatim}
      Fruit:             Quantity:
      apples              200
      lemons                3
      oranges              56
      peaches             175
      pears                37
\end{verbatim}

  Suppose that it is desired to sort by quantity instead.  Looking at the
  original (unsorted) data, move the Point to two spaces before the \var{3}
  on the line containing \var{lemons}.  The cursor should be right under the
  \var{u} in \var{Quantity}.  Set the mark.  Now move the Point to immediately after \var{56} on
  the \var{oranges} line and again press \key{Esc X} and enter \verb|sort|.
  This yields the desired sort:

\begin{verbatim}
      Fruit:             Quantity:
      lemons                3
      pears                37
      oranges              56
      peaches             175
      apples              200
\end{verbatim}

\section{Modes}

  \jed{} supports two internal modes as well as user defined modes.  The two
  internal modes consist of a ``C'' mode for C Language programming and a
  ``Wrap'' mode for ordinary text editing.  Examples of user defined modes
  are Fortran mode and DCL mode.

  Online documentation is provided for nearly every mode \jed{} defines.
  For help on the current mode, press \key{Esc X} and enter
  \verb|describe_mode|.  A window will appear with a short description of the
  special features of the mode as well as a description of the variables
  affecting the mode.

\subsection{Wrap Mode}

  In this mode, text is wrapped at the column given by the \verb|WRAP|
  variable. The default is 78.  The text does not wrap until the cursor goes
  beyond the wrap column and a space is inserted.

\subsubsection{Formatting paragraphs}

  Paragraph delimiters are: blank lines, lines that begin with either a
  percent character, \var{\%}, or a backslash character \verb|\|.  This
  definition is ideally suited for editing \LaTeX{} documents. However, it
  is possible for the user to change this definition.  See the discussion of
  the hook, \verb|is_paragraph_separator|, in the section on hooks for
  explicit details on how to do this.

  The paragraph is formatted according to the indentation of the current
  line. If the current line is indented, the paragraph will be given the
  same indentation.  The default binding for this function is \key{Esc q}.

  In addition, a paragraph may be ``narrowed'' by the
  \verb|narrow_paragraph| function which is bound to \key{Esc N} by default.
  This differs from the ordinary \verb|format_paragraph| function described
  above in that the right margin is reduced by an amount equal to the
  indentation of the current line. For example:
\begin{verbatim}
      This paragraph is the result of using the
      function ``narrow_paragraph''.  Note how the
      right margin is less here than in the above
      paragraph.
\end{verbatim}

  Finally, if either of these functions is called from the keyboard with a
  prefix argument, the paragraph will be justified as well.  For example,
  pressing \key{Esc 1 Esc N} on the previous paragraph yields:
\begin{verbatim}
      This paragraph  is  the  result   of   using  the
      function   ``narrow_paragraph''.    Note  how the
      right margin   is less here  than  in   the above
      paragraph.
\end{verbatim}

  See the discussion of \verb|format_paragraph_hook| in the section on hooks
  for details on how this is implemented.

\subsection{Smart Quotes}

  You have probably noticed that many key words in this document are quoted
  in double quotes like ``this is double quoted'' and `this is single
  quoted'. By default, the double quote key (") and single quote key (') are
  bound to the function \verb|text_smart_quote|.  With this binding and in
  wrap mode, the single quote key inserts a single quote with the ``proper''
  orientation and the double quote key inserts two single quotes of the
  ``proper'' direction. To turn this off, rebind the keys to
  \verb|self_insert_cmd|.  Some modes already do this (e.g., EDT).

  This brings up the question: if the double quote key is bound to
  \verb|text_smart_quote| then how does one insert the character (")?  The
  most common way is to use the \verb|quoted_insert| function which, by
  default, is bound to the single backquote (\key{`}) key.  This is the same
  mechanism that is used to insert control characters.  The other method is
  to use the fact that if the preceding character is a backslash, \verb|\|,
  the character simply self inserts.  Again, this is ideal for writing
  \TeX{} documents.

\subsection{C Mode}

  C Mode facilitates the editing of C files.  Much of the latter part of the
  development of the \jed{} editor was done using this mode.  This mode may be
  customized by a judicious choice of the variables \verb|C_INDENT| and
  \verb|C_BRACE| as well as the bindings of the curly brace keys \var{\{} 
  and \key{\}}.  Experiment to find what you like or write your own using the
  \slang{} interface.

  By default, the \key{Enter} key is bound to the function
  \verb|newline_and_indent|. This does what its name suggests: inserts a
  newline and indents.  Again, some modes may rebind this key.  In addition,
  the keys \var{\{}, \var{\}}, and \key{Tab} are also special in this mode.
  The \key{Tab} key indents the current line and the \var{\{} and \var{\}}
  keys insert themselves and reindent.  If you do not like any of these
  bindings, simply rebind the offending one to \verb|self_insert_cmd|.

  Finally, the key sequence \key{Esc ;} is bound to a function called
  \verb|c_make_comment|.  This function makes and indents a C comment to the
  column specified by the value of the variable \verb|C_Comment_Column|.  If
  a comment is already present on the line, it is indented.

\subsection{Fortran Mode}

  Fortran Mode is written entirely in \slang{} and is designed to facilitate
  the writing of Fortran programs.  It features automatic indentation of
  Fortran code as well as automatic placement of Fortran statement Labels.

  In this mode, the keys \key{0}-\key{9} are bound to a function
  \verb|for_elebel| which does the following:
\begin{itemize} 
\begin{enumerate}
\item Inserts the calling character (0-9) into the buffer.

\item If the character is preceded by only other digit characters, it
      assumes the character is for a label and moves it to the appropriate
      position.

\item Reindents the line.
\end{enumerate}
\end{itemize} 

  This function is very similar to the one Emacs uses for labels.

\section{Keyboard Macros}

  \jed{} is able to record a series of keystrokes from the terminal and replay
  them.  The saved series of keystrokes is known as a keyboard macro.  To
  begin a keyboard macro, simply enter the begin keyboard macro key sequence
  which is bound to \key{Ctrl-X (} if \verb|emacs.sl| is loaded.  To stop
  recording the keystrokes, enter \key{Ctrl-X )}.  Then to ``execute'' the
  macro, press \key{Ctrl-X e}. Please note that it is illegal to execute a
  macro while defining one and doing so generates an error.  A macro can be
  aborted at anytime by pressing the \key{Ctrl-G} key.

  One nice feature \jed{} includes is the \verb|macro_query| function.  That
  is, while defining a macro, the key sequence \key{Ctrl-X q} will cause
  \jed{} to issue the prompt \var{Enter String:} in the minibuffer.  Any
  string that is entered will be inserted into the buffer and the process of
  defining the macro continues. Every time the macro is executed, \jed{}
  will prompt for a NEW string to be inserted.

  Any time an error is generated, the process of defining the macro is
  aborted as well as execution of the macro.  This is very useful and may be
  exploited often.  For example, suppose you want to trim excess whitespace
  from the end of ALL lines in a buffer.  Let us also suppose that the
  number of lines in the buffer is less than 32000.  Then consider the
  following keystrokes:
\begin{verbatim}
      Ctrl-X (                      (begin macro)
      Ctrl-E                        (goto end of line)
      ESC                           (trim whitespace)
      Down Arrow                    (go down one line)
      Ctrl-X )                      (end macro)
\end{verbatim} 

Now the macro has been defined.  So move to the top of the buffer and
execute it 32000 times:
\begin{verbatim}
      ESC <                         (top of buffer)
      ESC 3 2 0 0 0                 (repeat next command 32000 times
      Ctrl-X e                      (execute macro)
\end{verbatim}
If the buffer has less than 32000 lines, the end of the buffer will be
reached and an error will be generated aborting the execution of the macro.

\section{Shells and Shell Commands}

  The default binding to execute a shell command and pump the output to a
  buffer is \key{Esc !}.  \jed{} will prompt for a command line and spawn a
  subprocess for its execution.

  Strictly speaking, \jed{} does not support interactive subprocesses.
  However, \jed{} includes \slang{} code that ``emulates'' such a
  subprocess.  It may invoked by typing \verb|shell| at the \verb|M-x|
  minibuffer prompt.  A window will be created with a buffer named
  \var{*shell*} attached to it. Any text entered at the system dependent
  shell prompt will be executed in a subprocess and the result stuffed back
  in the shell buffer. Don't try to execute any commands which try to take
  over the keyboard or the screen or something undesirable may happen.
  Examples of types of stupid commands are spawning other editors, logging
  in to remote systems, et cetera.  Even \var{chdir} is stupid since its
  effect is not permanent.  That is,

\begin{verbatim}
      > cd ..
      > dir
\end{verbatim}


  will not do what might naively be expected.  That is, the two commands
  above are not equivalent to the single command \var{dir ..}.

\section{Getting Help}

  \jed{}'s help functions are bound to \key{Ctrl-H} by default.  For example,
  \key{\key{Ctrl-H} C} will show what function a key carries out,
  \key{\key{Ctrl-H} i} will run \jed{}'s info reader, \key{Ctrl-H f} will give
  help on a particular \slang{} function, etc.  However, some modes may use
  the \key{Ctrl-H} key for something else.  For example, if EDT mode is in
  effect, then \key{Ctrl-H} may be bound to \verb|bol| which causes the
  cursor to move to the beginning of the line. See the section on EDT for
  more information.

  If \jed{} is properly installed, this entire document is accessable from
  within the editor using \jed{}'s info reader.  \key{Ctrl-H i} will load
  \verb|info_mode| allowing the user to browse the document as well as other
  ``info'' documents.

\section{Editing Binary Files}

  \jed{} may edit binary files as long as the proper precautions are taken. On
  IBMPC systems, this involves calling the \slang{} function
  \verb|set_file_translation| with an integer argument.  If the argument is
  0, files are opened as text files; otherwise, they are opened in binary
  mode. There is no need to call this function for other systems.  However,
  beware of the user variable \verb|ADD_NEWLINE| which if non zero, a
  newline character will be appended to the file if the last character is
  not a newline character.  If you are going to edit binary files, it is
  probably a good idea to set this variable to zero.

\section{Dired--- the Directory editor}

  In addition to editing files, \jed{} is also able to rename and delete
  them as well.  \jed{}'s Dired mode allows one to do just this is a simple
  and safe manner.

  To run dired, simply press \key{Esc X} and enter \verb|dired| at the
  prompt.  \jed{} will load \verb|dired.sl| and prompt for a directory name.
  Once the directory is given, \jed{} will display a list files in the
  directory in a buffer named \var{*dired*}.  One may use normal buffer
  movement keys to move around this buffer.  To delete one or more files,
  use the \key{d} key to ``tag'' the files.  This in itself does not delete
  them; rather, it simply marks them for deleting.  A capital `D' will
  appear in the left margin to indicate that a file has been tagged.  Simply
  hit the \key{u} key to untag a file.  The delete key will also untag the
  previously tagged file.

  To actually delete the tagged files, press the `x' key.  This action
  causes \jed{} to display a list of the tagged files in a separate window
  and prompt the user for confirmation.  Only when the proper confirmation
  is given, will the file be deleted.

  Renaming a file is just as simple.  Simply move to the line containg the
  name of the file that you wish to rename and hit the `r' key.  \jed{} will
  prompt for a filename or a directory name.  If a directory is given, the
  file will be moved to the new directory but will keep the name.  However,
  for the operation to succeed, the file must be one the same file system.
  To rename tagged files to a different directory residing on the same file
  system, use the \key{m} key.  This has the effect of moving the tagged
  file out of the current directory to the new one.

  One may also use the \key{f} key to read the file indicated by the cursor
  position into a buffer for editing.  If the file is a directory, the
  directory will be used for dired operations. In addition, one may also use
  the \key{v} to simply ``view'' a file.

  Finally, the \key{g} key will re-read the current directory and the
  \key{h} and \key{?} keys provide some help.

\section{Mail}

  This section applies to Unix and VMS systems only.  On these systems, it
  is possible to compose and send mail directly using \jed{}.  This assumes
  that the Unix system has \verb|/usr/ucb/mail|.  It is trivial to modify
  \verb|mail.sl| to support another Unix mailer.  For VMS, \jed{} uses the
  callable mail interface present on VMS versions 5.0 and later.

  The default binding for the mail is \key{Ctrl-X m}.  Alternatively, one
  may press \key{Esc X} and enter \verb|mail| at the \var{M-x} prompt.
  The mail function will cause a window to open with a buffer called
  \verb|*mail*| which contains the three lines:
\begin{verbatim}
      To: 
      Subject: 
      ---text follows this line---
\end{verbatim}
  Simply enter the email address of the person that you want to send the
  mail to on the line containing \var{To:} and put the subject of the
  message on the next line labeled \var{Subject:}.  The text that you wish
  to mail follows the line labeled \var{---text follows this line---} which
  is used by \jed{} as a marker. After you have composed the mail message,
  press \key{Esc X} and enter \verb|send| at the \verb|M-x| prompt.  For
  example, the following is an email requesting to be put on the \jed{} mailing
  list:

\begin{verbatim}
      To: davis@space.mit.edu
      Subject: jed mailing list
      ---text follows this line---
      Hi,
      
         Please add me to the JED mailing list so that I may be notified
      of upcoming releases of JED.

      --Maria
\end{verbatim}

  For VMS systems, the above example will probably fail because an internet
  address has been used for the example.  For systems using a TCP/IP
  package, it may be necessary to change \var{davis@space.mit.edu} to
  something
  like \var{smtp\%"davis@space.mit.edu"}.

  The mail function looks for a user defined hook called \verb|mail_hook| and
  execute it if it exists.  This hook may be used to bind certain keys in
  the keymap associated with the \verb|*mail*| buffer.  For example,

\begin{verbatim}
      define mail_hook ()
      {
        local_unsetkey ("^C");
        local_setkey ("send", "^C^C");
      }
\end{verbatim}

  defines the key \key{Ctrl-C Ctrl-C} in the mail keymap to perform the
  \verb|send| function.  Other possibilities include binding a key sequence,
  say \key{Ctrl-C Ctrl-W}, to a function that inserts the contents of a
  signature file.

\section{Customization}

  To extend \jed{}, it is necessary to become familiar with the \slang{}
  programming language.  \slang{} not a standalone programming language like
  C, Pascal, etc. Rather it is meant to be embedded into a C program. The
  \slang{} programming language itself provides only arithmetic, looping, and
  branching constructs.  In addition, it defines a few other primitive
  operations on its data structures.  It is up to the application to define
  other built-in operations tailored to the application.  That is what has
  been done for the \jed{} editor.  See the document \verb|slang.txt| for
  \slang{} basics as well as the \jed{} Programmer's Manual for functions \jed{} has
  added to the language.  In any case, look at the \verb|*.sl| files for
  explicit examples.

  For the most part, the average user will simply want to rebind some keys
  and change some variables (e.g., tab width).  Here I discuss setting keys
  and the predefined global variables.

\subsection{Setting Keys}

  Defining a key to invoke a certain function is accomplished using the
  \verb|setkey| function. This function takes two arguments: the function to
  be executed and the key binding.  For example, suppose that you want to
  bind the key \key{Ctrl-A} to cause the cursor to go to the beginning of
  the current line.  The \jed{} function that causes this is \verb|bol| (See
  the \jed{} Programmer's Manual for a complete list of functions).  Putting
  the line:
\begin{verbatim}
      setkey ("bol", "^A");
\end{verbatim}
  in the startup file \verb|jed.rc| (\verb|.jedrc|) file will perform the
  binding.  Here \verb|^A| consists of the two characters \verb|^| and
  \var{A} which \jed{} will interpret as the single character \verb|Ctrl-A|.
  For more examples, see either of the \slang{} files \verb|emacs.sl| or
  \verb|edt.sl|.
  
  The first argument to the \verb|setkey| function may be {\em any} \slang
  expression.  Well, almost any.  The only restriction is that the newline
  character cannot appear in the expression.  For example, the line
\begin{verbatim} 
      setkey ("bol();skip_white ();",  "^A");
\end{verbatim}     
  defines the \verb|Ctrl-A| key such that when it is pressed, the editing
  point will move the beginning of the line and then skip whitespace up to
  the first non-whitespace character on the line.
  
  In addition to being able to define keys to execute functions, it is also
  possible to define a key to directly insert a string of characters.  For
  example, suppose that you want to define a key to insert the string
  \var{int main(int argc, char **argv)} whenever you press the key 
  \key{Esc m}.  This may be accomplished as follows:
\begin{verbatim}
      setkey (" int main(int argc, char **argv)", "\em");
\end{verbatim}

  Notice two things.  First of all, the key sequence \key{Esc m} has
  been written as \verb|"\em"| where \verb|\e| will be interpreted by \jed{} as
  \key{Esc}. The other salient feature is that the first argument to
  \verb|setkey|, the ``function'' argument, begins with a space.  This tells
  \jed{} that it is not be interpreted as the name of a function; rather, the
  characters following the space are to be inserted into the buffer.
  Omitting the space character would cause \jed{} to execute a function called
  \var{int main(int argc, char **argv)} which would fail and generate an
  error.

  Finally, it is possible to define a key to execute a series of keystrokes
  similar to a keyboard macro.  This is done by prefixing the ``function''
  name with the \var{@} character.  This instructs \jed{} to interpret the
  characters following the \var{@} character as characters entered from the
  keyboard and execute any function that they are bound to.  For example,
  consider the following key definition which will generate a C language
  comment to comment out the current line of text.  In C, this may be
  achieved by inserting symbol \var{"/*"} at the beginning of the line and
  inserting \var{"*/"} at the end of the line.  Hence, the sequence is clear
  (Emacs keybindings):

\begin{itemize} 
\begin{enumerate}
\item Goto the beginning of the line: \key{Ctrl-A} or decimal \verb|"\001"|.

\item Insert \var{/*}.

\item Goto end of the line: \key{Ctrl-E} or decimal \verb|\005|.

\item Insert \var{*/}
\end{enumerate}
\end{itemize} 

  To bind this sequence of steps to the key sequence \key{Esc ;}, simply use
\begin{verbatim}
     setkey("@\001/*\005*/", "\e;");
\end{verbatim}

  Again, the prefix \var{@} lets \jed{} know that the remaining characters will
  carry out the functions they are currently bound to. Also pay particular
  attention to the way \key{Ctrl-A} and \key{Ctrl-E} have been written. Do
  not attempt to use the \verb|^| to represent ``\key{Ctrl}''.  It does not
  have the same meaning in the first argument to the \verb|setkey| function
  as it does in the second argument.  To have control characters in the
  first argument, you must enter them as \verb|\|{\em xyz} where {\em xyz} is
  a three digit decimal number coinciding with the ASCII value of the
  character. In this notation, the \key{Esc} character could have been
  written as \verb|\027|.  See the \slang{} Programmer's Reference Manual for
  further discussion of this notation.

  The \verb|setkey| function sets a key in the \verb|global| keymap from
  which all others are derived.  It is also possible to use the function
  \verb|local_setkey| which operates only upon the current keymap which may
  or may not be the \verb|global| map.

\subsection{Predefined Variables}

  \jed{} includes some predefined variables which the user may change.  By
  convention, predefined variables are in uppercase.  The variables which
  effect all modes include:

      \verb|BLINK|\\
       (1) if non-zero, blink matching parenthesis.
       
      \verb|TAB_DEFAULT|\\
       (8) sets default tab setting for newly created buffers
       to specified number of columns.
       
      \verb|TAB|\\
       Value of tab setting for current buffer.

      \verb|ADD_NEWLINE|\\
       (1) adds newline to end of file if needed when writing it out to the
            disk.
            
      \verb|META_CHAR|\\
       (-1) prefix for chars with high bit set (see section on eight bit
            clean issues for details)
            
      \verb|DISPLAY_EIGHT_BIT|\\
            see section on eight bit clean issues.

      \verb|COLOR|\\
       (23) IBMPC background color (see \verb|jed.rc| for meaning)

      \verb|LINENUMBERS|\\
       (0) if 1, show current line number on status line

      \verb|WANT_EOB|\\
       (0) if 1, [EOB] denotes end of buffer.

      \verb|TERM_CANNOT_INSERT|\\
       (0) if 1, do not put the terminal in insert mode when writing to the
       screen.
              
      \verb|IGNORE_BEEP|\\
       (0) do not beep the terminal when signalling errors


  In addition to the above, there are variables which affect only certain
  modes.  See the section on modes for details.

\subsection{Hooks}

  A hook is a user defined function that \jed{} calls under certain conditions
  which allow the user to modify default actions.  For example, when \jed{}
  starts up it looks for the existence of a user defined function
  \verb|command_line_hook|.  If this function exists, \jed{} calls the
  function. What the function does is completely arbitrary and is left to
  the discretion of the user.  The startup file, \verb|site.sl|, defines
  such a function which reads in the files listed on the command line. It is
  also this function which loads the \verb|jed.rc| startup file.  Unlike the
  other hooks, this one must be present in the file \verb|site.sl| since it
  is the only file loaded before calling the hook.

  After the startup files are loaded, \jed{} calls the hook
  \verb|jed_startup_hook| immediately before entering the main editor loop.
  This hook is useful to modify certain data structures which may not have
  existed when the startup files were loaded.

  In addition to the above hooks, \jed{} currently also looks for:

      \verb|suspend_hook|\\
        function to be executed before suspending

      \verb|resume_hook|\\
        function that gets carried out after suspension

      \verb|exit_hook|\\
        gets executed before exiting \jed{}

      \verb|mode_hook|\\
        sets buffer mode based on filename extension

      \verb|find_file_hook|\\
         called before file is read into a buffer.  It currently
         checks for presence of autosave file and warns user if
         it is more recent than file.
         
  See \verb|site.sl| for explicit examples of the above hooks.

  Another useful hook is \verb|is_paragraph_separator|.  This hook is called
  when \jed{} searches for the beginning or end of a paragraph.  This search
  is performed by all paragraph formatting functions as well as the forward
  and backward paragraph movement commands. As \jed{} performs the search,
  it moves from one line to another testing the line to see if it separates
  a paragraph.  The function of the hook is to make this decision and return
  zero if the line does not separate paragraphs or return one if it does.
  The default value of this hook may be written in \slang{} as

\begin{verbatim}
     define is_paragraph_separator ()
     {
       bol ();
       if (looking_at ("\\")) return 1;
       if (looking_at ("%")) return 1;
       skip_white(); eolp ();
     }
\end{verbatim}

  A related hook called after a paragraph is formatted is
  \verb|format_paragraph_hook|. This hook is only called if either
  \verb|format_paragraph| or \verb|narrow_paragraph| is called with a prefix
  digit argument.  For example, \verb|format_paragraph| is bound to 
  \key{Esc q}.  Simply pressing this key sequence will call 
  \verb|format_paragraph| but \verb|format_paragraph_hook| will not be called. 
  However, pressing \key{Esc 1} followed by \key{Esc q} will
  result in a call to \verb|format_paragraph_hook|.  Currently, this hook
  simply justifies the paragraph.  That is, it fills each line in the
  paragraph such that the line ends at the right margin, which is defined by
  the \verb|WRAP| variable.

\subsection{S-Lang Programming Hints (Debugging)}

  This section assumes some knowledge about \slang{} and is designed to
  explain how to debug \slang{} routines quickly.  For information about
  \slang{}, read \verb|slang.txt|.

  There are two ways of loading a file of \slang{} code into \jed{}.  The
  most common way is through the function \verb|evalfile|.  If an error
  occurs while loading a file, \jed{} will give some indication of where the
  problem lies by displaying the line number and the offending bit of
  \slang{} code in the minibuffer.  In practice though, this can be quite
  inefficient. The \verb|evalfile| function is primarily designed to load
  debugged and tested \slang{} code.

  The best way to develop and test \slang{} code with \jed{} is to use the
  function \verb|evalbuffer|.  Simply load the piece of code into \jed{} as an
  ordinary file, press \key{Esc X} and enter the function
  \verb|evalbuffer| If the piece of code in the buffer has any syntax
  errors, \jed{} will put the cursor on the error.  This is the best way to
  spot compile time errors such as syntax errors.  However, this will not
  catch runtime errors.

  When a runtime error occurs, \jed{} will put the cursor on the top level
  function where the original call was made and NOT the actual location of
  the function.  To aid in determining where an error occurs, \jed{} can be
  made to give a symbolic traceback.  As the \slang{} runtime stack unwinds,
  \slang{} will simply print the name of function at that particular level.
  If the function includes local variables, their values will be dumped as
  well.  Hence, it is easy to quickly narrow the location of an error down
  to function where the error occurs.  By default, the traceback is
  disabled.  The traceback is enabled by setting the \slang{} variable
  \verb|_traceback| to a non-zero value.  It is simpliest to just press
  \key{Ctrl-X Esc} and enter \verb|_traceback = 1| at the \var{S-Lang}
  prompt.  This is one of those times where one needs access to the
  \var{S-Lang>} prompt and not the \var{M-x} prompt. For example, consider
  the following piece of code:
\begin{verbatim}
      define fun_two () {forever {}}   % loops forever
      define fun_one () {fun_two ()}   % calls fun_two-- never returns
\end{verbatim}
  Simply enter the above into an empty \jed{} \verb|*scratch*| buffer, then
  press \key{Ctrl-X Esc} and enter:
\begin{verbatim} 
      _traceback = 1; () = evalbuffer (); fun_one ();
\end{verbatim} 
  This will turn on tracebacks, evaluate the buffer and call the
  function \verb|fun_one|.  \jed{} will then be put into an infinite loop which
  can only be stopped by pressing the abort character which by default is
  \key{Ctrl-G}. Doing so, will produce the traceback messages
\begin{verbatim}
       S-Lang Traceback: fun_two
       S-Lang Traceback: fun_one
\end{verbatim}
  in addition to the error message \var{User Break!}.  Of course, this
  technique only narrows down the source of an error to a particular
  function. To proceed further, it may necessary to put ``print'' statements
  at suitable places in the function.  There are several ways to do this:
\begin{itemize}
\begin{enumerate}
\item Use the \verb|insert| function to insert the contents of a variable
      into the current buffer.

\item Use the \verb|error| function to abort the function and display the
      value of a variable in the minibuffer.

\item Use the \verb|message| function to display the value of a variable in
      the minibuffer.  Unlike \verb|error|, the \verb|message| function does
      not abort the execution of the function.
\end{enumerate}
\end{itemize} 

  Since each of these functions require a string argument, it is usually
  best to call the \verb|string| function first for the conversion followed
  by the output function.  This has to be done anyway if it is desired to
  get the contents of an integer variable.  Although the second approach is
  prehaps the most useful in practice, it is somtimes appropriate to use a
  combination of these techniques.

  Finally, to print the entire stack, one can use the \verb|print_stack|
  function.  This function dumps the \slang{} runtime stack into the
  \var{*traceback*} buffer.

  Since \slang{} is an interpreted language, judicious application of the
  above techniques should lead very quickly to the source of any errors.


\section{Eight Bit Clean Issues}


\subsection{Displaying Characters with the High Bit Set}

  There are several issues to consider here. The most important issue is how
  to get \jed{} to display 8 bit characters in a ``clean'' way.  By
  ``clean'' I mean any character with the high bit set is sent to the
  display device as is.  This is achieved by putting the line:

\begin{verbatim}
      DISPLAY_EIGHT_BIT = 1;
\end{verbatim}


  in the \verb|jed.rc| (\verb|.jedrc|) startup file.  European systems might
  want to put this in the file \verb|site.sl| for all users.  The default is
  1 so unless its value has been changed, this step may not be necessary.

  There is another issue. Suppose you want to display 8 bit characters with
  extended Ascii codes greater than or equal to some value, say 160.  This
  is done by putting \verb|DISPLAY_EIGHT_BIT = 160;|.  I believe that ISO
  Latin character sets assume this.  This is the default value for Unix and
  VMS systems.

\subsection{Inputting Characters with the hight bit Set}
  Inputting characters with the high bit set into \jed{} is another issue.
  How \jed{} interprets this bit is controlled by the variable
  \verb|META_CHAR|. What happens is this: When \jed{} reads a character from
  the input device with the high bit set, it:
\begin{itemize}
\begin{enumerate}
\item Checks the value of \verb|META_CHAR|.  If this value is -1, \jed{} simply
      inserts the character into the buffer.

\item For any other value of \verb|META_CHAR| in the range 0 to 255, \jed{}
      returns two 7-bit characters.  The first character returned is
      \verb|META_CHAR| itself.  The next character returned is the original
      character but with the high bit stripped.
\end{enumerate}
\end{itemize} 

  The default value of \verb|META_CHAR| is -1 which means that when \jed{} sees
  a character with the high bit set, \jed{} leaves it as is.  Please note that
  a character with the high bit set it {\em cannot} be the prefix character
  of a keymap. It can be a part of the keymap but not the prefix.

  Some systems only handle 7-bit character sequences and as a result, \jed{}
  will only see 7-bit characters.  \jed{} is still able to insert {\em any}
  character in the range 0-255 on a 7-bit system.  This is done through the
  use of the \verb|quoted_insert| function which, by default, is bound to
  the backquote key \var{`}.  If the \verb|quoted_insert| function is called
  with a digit argument (repeat argument), the character with the value of
  the argument is inserted into the buffer.  Operationally, one hits
  \key{Esc}, enters the extended Ascii code and hits the backquote key.  For
  example, to insert character 255 into the buffer, simply press the
  following five keys: \key{Esc 2 5 5 `}.

\subsection{Upper Case - Lower Case Conversions}

  The above discussion centers around input and output of characters with
  the high bit set.  How \jed{} treats them internally is another issue and new
  questions arise.  For example, what is the uppercase equivalent of a
  character with ASCII code 231?  This may vary from language to language.
  Some languages even have characters whose uppercase equivalent correspond
  to multiple characters.  For \jed{}, the following assumptions have been
  made:
\begin{itemize}
\item Each character is only 8 bits.
\item Each character has a unique uppercase equivalent.
\item Each character has a unique lowercase equivalent.
\end{itemize} 
  It would be nice if a fourth assumption could be made:
\begin{itemize} 
\item The value of the lowercase of a character is greater than or equal to
      its uppercase counterpart.
\end{itemize} 
  However, apparently this is not possible since most IBMPC character sets
  violate this assumption.  Hence, \jed{} does not assume it.  Suppose \var{X}
  is the upper case value of some character and suppose \var{Y} is its lower
  case value.  Then to make \jed{} aware of this fact and use it case
  conversions, it may be necessary to put a statement of the form:
\begin{verbatim}
     define_case (X, Y);
\end{verbatim}
  in the startup file.  For example, suppose 211 is the uppercase of 244.
  Then, the line
\begin{verbatim}
      define_case (211, 244);
\end{verbatim}
  will make \jed{} use this fact in operations involving the case of a character.

  This has already been done for the ISO Latin 1 character set. See the file
  \verb|iso-latin.sl| for details. For MSDOS, this will not work. Instead
  use the files \verb|dos437.sl| and \verb|dos850.sl|.  By default, \jed{}'s
  internal lookup tables are initialized to the ISO Latin set for Unix and
  VMS systems and to the DOS 437 code page for the IBMPC.  To change the
  defaults, it is only necessary to load the appropriate file. For example,
  to load \verb|dos850.sl| definitions, put
\begin{verbatim}
      evalfile ("dos850"); pop ();
\end{verbatim}
  in the startup file (e.g., \verb|site.sl|).  In addition to
  uppercase/lowercase information, these files also contain word
  definitions, i.e., which characters constitute a ``word''.

\section{Miscellaneous}

\subsection{Abort Character}
  The abort character (\key{Ctrl-G} by default) is special and should not be
  rebound.  On the IBMPC, the keyboard interrupt 0x09 is hooked and a quit
  condition is signaled when it is pressed.  For this reason, it should not
  be used in any keybindings. A similar statement holds for the other
  systems.

  This character may be changed using the function \verb|set_abort_char|
  Using this function affects all keymaps.  For example, putting the line
\begin{verbatim}
      set_abort_char (30);
\end{verbatim}
  in your \verb|jed.rc| file will change the abort character from its
  current value to 30 which is \key{Ctrl-}\verb|^|.

\subsection{Input Translation}

  By using the function \verb|map_input| the user is able to remap
  characters input from the terminal before \jed{}'s keymap routines have a
  chance to act upon them.  This is useful when it is difficult to get \jed{}
  to see certain characters.  For example, consider the \key{Ctrl-S}
  character.  This character is especially notorious because many systems
  use it and \key{Ctrl-Q} for flow control. Nevertheless Emacs uses
  \key{Ctrl-S} for searching.  Short of rebinding all keys which involve a
  \key{Ctrl-S} how does one work with functions that are bound to key
  sequences using \key{Ctrl-S}?  This is where \verb|map_input| comes into
  play.  The \verb|map_input| function requires two integer arguments which
  define how a given ascii character is to be mapped.  Suppose that you wish
  to substitute \key{Ctrl-}\verb|\| for \key{Ctrl-S} everywhere. The line
\begin{verbatim}
      map_input (28, 19);
\end{verbatim}
  will do the trick.  Here 28 is the ascii character of \key{Ctrl-}\verb|\|
  and 19 is the ascii character for the \key{Ctrl-S}.

  As another example, consider the case where the backspace key sends out a
  \key{Ctrl-H} instead of the \key{DEL} character (\key{Ctrl-?}).
\begin{verbatim}
      map_input (8, 127);
\end{verbatim}
  will map the \key{Ctrl-H} (ascii 8) to the delete character (ascii 127).

\subsection{Display Sizes}

  On VMS and unix systems, the screen size may be changed to either 80 or
  132 columns by using the functions \verb|w80| and \verb|w132|
  respectively. Simply enter the appropriate function name at the \verb|M-x|
  prompt in the minibuffer.  The default binding for access to the
  minibuffer is \key{Esc X}.  Most window systems, e.g., DECWindows, allow
  the window size to be changed.  When this is done, \jed{} should
  automatically adapt to the new size.

  On the PC, at this time the screen size cannot be changed while \jed{} is
  running.  Instead it is necessary to exit \jed{} first then set the display
  size and rerun \jed{}.

\end{document}
}
----------------------------------------------------------------------
query_replace_match:

"\( [a-z]+\)\1 "    "\1 "       finds and replaces repeated words.
"/\*\(.*\)\*/.*$"   "//\1"      convert /* c comment */ at eol to // ...
"//.*$"             "/*\1 */"   convert c++ to C

