%$Id: s.tex,v 1.1 96/03/24 11:12:42 al Exp $
% man circuit s .
%------------------------------------------------------------------------
\section{{\tt S}: Voltage Controlled Switch}
%------------------------------------------------------------------------
\subsection{Syntax}
\begin{verse}
{\tt S}{\it xxxxxxx n+ n-- nc+ nc-- mname} \{{\it ic}\}
\end{verse}
%------------------------------------------------------------------------
\subsection{Purpose}

Voltage controlled switch.
%------------------------------------------------------------------------
\subsection{Comments}

{\it N+} and {\it n--} are the positive and negative element nodes,
respectively.  {\it Nc+} and {\it nc--} are the controlling nodes.
{\it Mname} is the model name.  A switch is a resistor between {\it
n+} and {\it n--}.  The value of the resistor is determined by the
state of the switch.

The resistance between {\it n+} and {\it n--} will be {\it RON}
when the controlling voltage (between {\it nc+} and {\it nc--}) is
above {\it VT} + {\it VH}.  The resistance will be {\it ROFF} when
the controlling voltage is below {\it VT} - {\it VH}.  When the
controlling voltage is between {\it VT} - {\it VH} and {\it VT} +
{\it VH}, the resistance will retain its prior value.

You may specify {\tt ON} or {\tt OFF} to indicate the initial state
of the switch when the controlling voltage is in the hysteresis
region.

{\tt RON} and {\tt ROFF} must have finite positive values.
%------------------------------------------------------------------------
\subsection{Model Parameters}

\begin{description}

\item[{\tt VT} = {\it x}] Threshold voltage.  (Default = 0.)

\item[{\tt VH} = {\it x}] Hysteresis voltage.  (Default = 0.)

\item[{\tt RON} = {\it x}] On resistance.  (Default = 1.)

\item[{\tt ROFF} = {\it x}] Off resistance.  (Default = 1e12)

\end{description}
%------------------------------------------------------------------------
%------------------------------------------------------------------------
