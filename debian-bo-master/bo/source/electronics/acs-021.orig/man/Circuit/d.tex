%$Id: d.tex,v 1.1 96/03/24 11:12:35 al Exp $
% man circuit d .
%------------------------------------------------------------------------
\section{{\tt D}: Diode}
%------------------------------------------------------------------------
\subsection{Syntax}
\begin{verse}
{\tt D}{\it xxxxxxx n+ n-- mname} \{{\it area}\} \{{\it args}\}
\end{verse}
%------------------------------------------------------------------------
\subsection{Purpose}

Junction diode.
%------------------------------------------------------------------------
\subsection{Comments}

{\it N+} and {\it n--} are the positive and negative element nodes,
respectively.  {\it Mname} is the model name.  {\it Area} is the area
factor.  If the area factor is omitted, a value of 1.0 is assumed.  {\it
Args} is a list of additional arguments.  The parameters available are a
superset of those available in SPICE.

A diode can also use a MOSFET model (type {\tt NMOS} or {\tt PMOS})
to represent the equivalent of the source-bulk or drain-bulk diodes.

When the element is printed out, by a {\tt list} or {\tt save} command, the
the computed values of {\tt IS}, {\tt RS}, {\tt CJ}, and {\tt CJSW} are
printed as a comment if they were not explicitly entered.
%------------------------------------------------------------------------
\subsection{Element Parameters}

\begin{description}

\item[{\tt Area} = {\it x}] Area factor.  (Default = 1.0) If optional
parameters {\tt IS}, {\tt RS}, and {\tt CJO} are not specified, the {\tt
.model} value is multiplied by {\tt area} to get the actual value.

\item[{\tt Perim} = {\it x}] Perimeter factor.  (Default = 1.0) If optional
parameter {\tt CJSW} is not specified, the {\tt .model} value is multiplied
by {\tt perim} to get the actual value.

\item[{\tt IC} = {\it x}] Initial condition.  The initial voltage to use in
transient analysis, if the {\tt UIC} option is specified.  Default: don't
use initial condition.  This is presently ignored, but accepted for
compatibility.

\item[{\tt OFF}] Start iterating with this diode off, in DC analysis.

\item[{\tt STIFF}] Use a stiffly stable integration method and
ignore truncation error.

\item[{\tt IS} = {\it x}] Saturation current.  This overrides {\tt IS} in
the {\tt .model}, and is not affected by {\tt area}.  Default: use {\tt IS}
from {\tt .model} * {\tt area}.

\item[{\tt RS} = {\it x}] Ohmic (series) resistance.  This overrides {\tt
RS} in the {\tt .model}, and is not affected by {\tt area}.  Default: use
{\tt RS} from {\tt .model} * {\tt area}.  This is presently ignored, but
accepted for compatibility.

\item[{\tt CJ} = {\it x}] Zero-bias junction capacitance.  This overrides
{\tt CJ} in the {\tt .model}, and is not affected by {\tt area}.  Default:
use {\tt CJ} from {\tt .model} * {\tt area}.

\item[{\tt CJSW} = {\it x}] Zero-bias sidewall capacitance.  This overrides
{\tt CJSW} in the {\tt .model}, and is not affected by {\tt perim}.  Default:
use {\tt CJSW} from {\tt .model} * {\tt perim}.

\end{description}
%------------------------------------------------------------------------
\subsection{Model Parameters}

\begin{description}

\item[{\tt IS} = {\it x}] Normalized saturation current. (Amperes).
(Default = 1.0e-14) {\tt IS} is multiplied by the {\it area} in the element
statement to get the actual saturation current.  It may be overridden by
specifying {\tt IS} in the element statement.

\item[{\tt RS} = {\it x}] Normalized ohmic resistance. (Ohms) (Default = 0.)
{\tt RS} is multiplied by the {\it area} in the element statement to get the
actual ohmic resistance.  It may be overridden by specifying {\tt RS} in the
element statement.  {\tt RS} is accepted, and silently ignored, for
compatibility, but not implemented.

\item[{\tt N} = {\it x}] Emission coefficient. (Default = 1.0)
In ECA-2 the default value was 2.

\item[{\tt TT} = {\it x}] Transit time. (Default = 0.)  The diffusion
capacitance is given by: $c_d = TT g_d$ where $g_d$ is the diode
conductance.

\item[{\tt VJ} = {\it x}] Junction potential.  (Default = 1.0) Used in
computation of capacitance.  For compatibility with older versions of SPICE,
{\tt PB} is accepted as an alias for {\tt VJ}.

\item[{\tt CJo} = {\it x}] Normalized zero-bias depletion capacitance.
(Default = 0.)  {\tt CJo} is multiplied by the {\it area} in the element
statement to get the actual zero-bias capacitance.  It may be overridden by
specifying {\tt CJ} in the element statement.

\item[{\tt Mj} = {\it x}] Grading coefficient.  (Default = 0.5)

\item[{\tt CJSw} = {\it x}] Normalized zero-bias sidewall capacitance.
(Default = 0.)  {\tt CJSw} is multiplied by the {\it perimeter} in the
element statement to get the actual zero-bias capacitance.  It may be
overridden by specifying {\tt CJSW} in the element statement.

\item[{\tt MJSw} = {\it x}] Sidewall grading coefficient.  (Default = 0.33)

\item[{\tt EG} = {\it x}] Activation energy. (electron Volts) (Default =
1.11, silicon.)  For other types of diodes, use:
\begin{verse}
1.11 ev.  Silicon (default value)\\
0.69 ev.  Schottky barrier\\
0.67 ev.  Germanium\\
1.43 ev.  GaAs\\
2.26 ev.  GaP
\end{verse}

\item[{\tt XTI} = {\it x}] Saturation current temperature exponent.
(Default = 3.0)  For Schottky barrier, use 2.0.

\item[{\tt KF} = {\it x}] Flicker noise coefficient.  (Default = 0.)  SPICE
parameter accepted but not implemented.

\item[{\tt AF} = {\it x}] Flicker noise exponent.  (Default = 1.0)  SPICE
parameter accepted but not implemented.

\item[{\tt FC} = {\it x}] Coefficient for forward bias depletion capacitance
formula. (Default = 0.5)

\item[{\tt BV} = {\it x}] Reverse breakdown voltage.  (Default = $\infty$.)
SPICE parameter accepted but not implemented.

\item[{\tt IBV} = {\it x}] Current at breakdown voltage. (Default = 1 ma.)
SPICE parameter accepted but not implemented.

\item[{\tt STIFF}] Use a stiffly stable integration method and
ignore truncation error.

\end{description}
%------------------------------------------------------------------------
\subsection{Probes}

\begin{description}

\item[{\tt Vd}] Voltage.  The first node (anode) is assumed positive.

\item[{\tt Id}] Total current.  It flows into the first node (anode), out of
the second (cathode).  I(Dxxxx) is the same as IJ(Dxxxx) + IC(Dxxxx).

\item[{\tt IJ}] Junction current.  The current through the junction.  
IJ(Dxxxx) is the same as I(Yj.Dxxxx).

\item[{\tt IC}] Capacitor current.  The current through the parallel 
capacitor.  IC(Dxxxx) is the same as I(Cj.Dxxxx).

\item[{\tt P}] Power.  P(Dxxxx) is the same as PJ(Dxxxx) + PC(Dxxxx).

\item[{\tt PD}] Power dissipated.  The power dissipated as heat.  
It is always positive and does not include power sourced.
It should be the same as P because the diode is passive.

\item[{\tt PS}] Power sourced.  The power sourced by the part.  
It is always positive and does not consider its own dissipation.
It should be 0 because the diode is passive.

\item[{\tt PJ}] Junction power.  PJ(Dxxxx) is the same as P(Yj.Dxxxx).

\item[{\tt PC}] Capacitor power.  PC(Dxxxx) is the same as P(Cj.Dxxxx).

\item[{\tt Cap}] Effective capacitance.  C(Dxxxx) is the same as EV(Cj.Dxxxx).

\item[{\tt Req}] Effective resistance.  R(Dxxxx) is the same as R(Yj.Dxxxx).

\item[{\tt REgion}] Region code.  A numeric code that represents the region
it is operating in.  +1 = forward, -1 = reversed, 0 = unknown.

\end{description}

All parameters of the internal elements Yj and Cj are available.  To access
them, concatenate the labels for the internal element with the diode,
separated by a dot.  Yj.D6 is the admittance (Yj) element of the diode D6.

In this release, there are no probes available in AC analysis except for the
internal elements.
%------------------------------------------------------------------------
%------------------------------------------------------------------------
