%$Id: x.tex,v 1.1 96/03/24 11:12:44 al Exp $
% man circuit x .
%------------------------------------------------------------------------
\section{{\tt X}: Subcircuit Call}
%------------------------------------------------------------------------
\subsection{Syntax}
\begin{verse}
{\tt X}{\it xxxxxxx n1} \{{\it n2 n3 ...}\} {\it subname}
\end{verse}
%------------------------------------------------------------------------
\subsection{Purpose}

Subcircuit call
%------------------------------------------------------------------------
\subsection{Comments}

Subcircuits are used by specifying pseudo-elements beginning with {\tt X},
followed by the connection nodes.
%------------------------------------------------------------------------
\subsection{Probes}

\begin{description}

\item[{\tt V}{\it x}]  Port (terminal node) voltage.  {\it x} is
which port to probe.  1 is the first node in the "X" statement, 2
is the second, and so on.

\item[{\tt P}] Power.  The sum of the power probes for all the internal elements.

\item[{\tt PD}] Power dissipated.  The total power dissipated as heat.

\item[{\tt PS}] Power sourced.  The total power generated.

\end{description}

In this release, there are no probes available in AC analysis except for the
internal elements.  More parameters will be added.  Internal elements can be
probed by concatenating the internal part label with the subcircuit label.
R5.X7 is R5 inside X7.
%------------------------------------------------------------------------
%------------------------------------------------------------------------
