%$Id: unmark.tex,v 1.2 96/03/30 18:03:44 al Exp $
% man commands unmark .
%------------------------------------------------------------------------
\section{{\tt UNMARK} command}
\index{unmark command}
\index{transient reruns}
%------------------------------------------------------------------------
\subsection{Syntax}
\begin{verse}
{\tt UNMark}
\end{verse}
%------------------------------------------------------------------------
\subsection{Purpose}

Forget remembered circuit voltages and currents.  Undo the `{\tt mark}'
command.
%------------------------------------------------------------------------
\subsection{Comments}

Allow time to proceed.  It has been held back by the `{\tt mark}' command.
%------------------------------------------------------------------------
\subsection{Examples}

\begin{description}
\item[{\tt transient 0 1 .01}] A transient analysis starting at zero,
running until 1 second, with step size .01 seconds.  After this run, the
clock is at 1 second.

\item[{\tt mark}] Remember the time, voltages, currents, etc.

\item[{\tt transient}] Another transient analysis.  It continues from 1
second, to 2 seconds.  (It spans 1 second, as before.)  This command was not
affected by the {\tt mark} command.

\item[{\tt transient}] This will do exactly the same as the last one.  From
1 second to 2 seconds.  If it were not for {\tt mark}, it would have started
from 2 seconds.

\item[{\tt transient 1.5 .001}] Try again with smaller steps.  Again, it
starts at 1 second.

\item[{\tt unmark}] Release the effect of {\tt mark}.

\item[{\tt transient}] Exactly the same as the last time, as if we didn't
{\tt unmark}.  (1 to 1.5 seconds.)

\item[{\tt transient}] This one continues from where the last one left off:
at 1.5 seconds.  From now on, time will move forward.

\end{description}

%------------------------------------------------------------------------
%------------------------------------------------------------------------
