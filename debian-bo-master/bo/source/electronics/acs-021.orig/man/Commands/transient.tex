%$Id: transient.tex,v 1.2 96/03/30 18:03:42 al Exp $
% man commands transient .
%------------------------------------------------------------------------
\section{{\tt TRANSIENT} command}
\index{transient command}
\index{time domain}
\index{nonlinear transient analysis}
%------------------------------------------------------------------------
\subsection{Syntax}
\begin{verse}
{\tt Transient} {\it start stop stepsize} \{{\it options} ...\}\\
{\tt Transient} {\it stepsize stop start} \{{\it options} ...\}
\end{verse}
%------------------------------------------------------------------------
\subsection{Purpose}

Performs a nonlinear time domain (transient) analysis.
%------------------------------------------------------------------------
\subsection{Comments}

The nodes to look at must have been previously selected by the {\tt Print} or
{\tt Plot} command.

Three parameters are normally needed for a Transient analysis: start time,
stop time and step size, in this order.  The SPICE order (step size, stop,
start) is also acceptable.  An optional fourth parameter is the maximum internal time step.

If all of these are omitted, the simulation will continue from where the most
recent one left off, with the same step size, unless the circuit topology has
been changed.  It will run for the same length of time as the previous run.

Do not use a step size too large as this will result in errors in the
results.  If you suspect that the results are not accurate, try a larger
argument to `Skip'.  This will force a smaller internal step size.  If the
results are close to the same, they can be trusted.  If not, try a still
larger `Skip' argument until they appear to match close enough.
\index{aliasing: transient}
\index{accuracy: transient}
\index{stability: transient}

The most obvious error of this type is aliasing.  You must select sample
frequency at least twice the highest signal frequency that exists anywhere in
the circuit.  This frequency can be very high, when you use the default step
function as input.  The signal generator does {\bf not} have any filtering.
%------------------------------------------------------------------------
\subsection{Options}

\begin{description}

%%%\item[{\tt <} {\it file}] Get circuit input from {\it file}.

\item[{\tt >} {\it file}] Send results of analysis to {\it file}.

\item[{\tt >>} {\it file}] Append results to {\it file}.

%%%\item[{\tt ACMAx}] Use worst case max values, per last {\tt AC}
%%%analysis.

%%%\item[{\tt ACMIn}] Use worst case min values, per last {\tt AC}
%%%analysis.

\item[{\tt ALL}] Show all internal time steps including extra steps
requested by {\tt Skip}, events, and by automatic time step control.

\item[{\tt Cold}] Zero initial conditions.  Cold start from
power-up.

%%%\item[{\tt DCMAx}] Use worst case max values, per last {\tt DC}
%%%or {\tt OP} analysis.

%%%\item[{\tt DCMIn}] Use worst case min values, per last {\tt DC}
%%%or {\tt OP} analysis.

\item[{\tt DTMIn} = {\it x}] Minimum time step.  (Default = from
{\tt options})  The smallest internal time step in transient
analysis.  The {\tt transient} command {\tt dtmin} option and the
{\tt dtratio} option override it if it is bigger.

\item[{\tt DTRatio} = {\it x}] The ratio between minimum and maximum
time step.  (Default = from {\tt options}).

%%%\item[{\tt Echo}] Echo disk reads to console, when input is
%%%from a file.

%%%\item[{\tt LAg}] Use worst case values, for lagging phase, per
%%%{\tt AC} analysis.

%%%\item[{\tt LEad}] Use worst case values, for leading phase, per
%%%{\tt AC} analysis.

\item[{\tt NOPlot}] Suppress plotting.

\item[{\tt PLot}] Graphic output, when plotting is otherwise
off.

\item[{\tt Quiet}] Suppress console output.

\item[{\tt Skip} {\it count}] Force at least {\it count} simulation
steps for each one displayed.  If the output is a table or ASCII plot, the
extra steps are hidden.

%%%\item[{\tt Table}] Tabular output. Override default plot.

\item[{\tt TEmperature} {\it degrees}] Temperature, degrees C.

\item[{\tt TRace} {\it n}] Show extended information during solution.
\begin{description}
\item[2] Show all accepted internal time steps.
\item[3] Show all internal time steps including rejected steps.
\item[4] Show every iteration.
\item[5] Show extended diagnostics.
\end{description}

\item[{\tt UIC}] Use initial conditions.  ACS will use the values
specified with the {\tt IC =} options on the various elements.

\end{description}
%------------------------------------------------------------------------
\subsection{Examples}

\begin{description}

\item[{\tt transient 0 100u 10n}] Start at time 0, stop after
100 micro-seconds.  Simulate using 10 nanosecond steps.

\item[{\tt transient}] No parameters mean to continue from the last run.  In
this case it means to step from 100 us to 200 us in 10 ns steps.  (The same
step size and run length, but offset to start where the last one stopped.

\item[{\tt transient skip 10}] Do 10 extra steps internally for every step
that would be done otherwise.  In this case it means to internally step at 1
nanosecond.  If the output is in tabular form, the extra steps are hidden.

\item[{\tt transient 0}] Start over at time = 0.  Keep the same
step size and run length.

\item[{\tt transient cold}] Zero initial conditions.  This will
show the power-on transient.

\item[{\tt transient >arun}] Save the results of this run in
the file {\tt arun}.

%%%\item[{\tt transient < aninput}] Use the file {\tt aninput}
%%%as a user defined input.  It substitutes for the signal
%%%generator.

\end{description}
%------------------------------------------------------------------------
%------------------------------------------------------------------------
