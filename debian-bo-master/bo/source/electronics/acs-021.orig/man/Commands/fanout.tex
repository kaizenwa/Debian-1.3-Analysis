%$Id: fanout.tex,v 1.2 96/03/30 18:03:16 al Exp $
% man commands fanout .
%------------------------------------------------------------------------
\section{{\tt FANOUT} command}
\index{fanout command}
\index{view circuit}
\index{connection list}
\index{node list}
\index{list by node}
%------------------------------------------------------------------------
\subsection{Syntax}
\begin{verse}
{\tt FANout} \{{\it nodes}\}
\end{verse}
%------------------------------------------------------------------------
\subsection{Purpose}

Lists connections to each node.
%------------------------------------------------------------------------
\subsection{Comments}

{\tt Fanout} lists the line number and label of each part connected to each
node.  If both ends of a part are connected the same place, it is listed
twice.

For a partial list, just specify the numbers.  A number alone ({\tt 17}) says
this branch alone.  A trailing dash ({\tt 23-}) says from here to the end.  A
leading dash ({\tt -33}) says from the start to here.  Two numbers ({\tt 9
13}) specify a range.
%------------------------------------------------------------------------
\subsection{Examples}

\begin{description}

\item[{\tt fanout}] Lists all the nodes in the circuit, with their
connections.

\item[{\tt fanout 99}] List parts connecting to node 99.

\item[{\tt fanout 0}] List the connections to node 0.  (There must be at
least one, unless you are editing a model.)

\item[{\tt fanout 78-}] List connections to nodes 78 and up.

\item[{\tt fanout 124 127}] List connections to nodes 124, 125, 126, 127.

\end{description}
%------------------------------------------------------------------------
%------------------------------------------------------------------------
