%$Id: sweep.tex,v 1.2 96/03/30 18:03:39 al Exp $
% man commands sweep .
%------------------------------------------------------------------------
\section{{\tt SWEEP} command}
\index{sweep command}
\index{sweep component values}
\index{step component values}
%------------------------------------------------------------------------
\subsection{Syntax}
\begin{verse}
{\tt SWeep} \{{\it stepcount}\}  {\it partlabel=range} ...
\end{verse}
%------------------------------------------------------------------------
\subsection{Purpose}

Sweep a component (or group of components) over a range.  Set up a loop for
iteration.
%------------------------------------------------------------------------
\subsection{Comments}

This command begins a loop which will sweep a component or group of
components.

When this command is given, the only apparent actions will be a change in the
prompt from `{\tt -->}' to `{\tt >>>}', and some disk action.

The different prompt means that commands are not executed immediately, but
are stored in a temporary file.

The bare command will repeat the same command sequence as the last time {\tt
SWeep} was run, and not prompt for anything else.

Additional components can be swept at the same time by entering a `{\tt
FAult}' command at the `{\tt >>>}' prompt.  The `{\tt FAult}' behaves
differently here:  It accepts a range, which is the sweep limits.

The `{\tt GO}' command will end the entry sequence, and make it all happen.
After this, the values are restored.  (Also, all {\tt FAult}s are restored,
as if by the `{\tt Restore}' command.)

All commands can be used in this mode.  Of course, some of them are not
really useful ({\tt Quit}) because they work as usual.

Only linear, ordinary parts can be swept.  (No semiconductor devices, or
elements using behavioral modeling.)  The tolerance remains unchanged.  If
you attempt to sweep a nonlinear or otherwise strange part, it becomes
ordinary and linear during the sweep.
%------------------------------------------------------------------------
\subsection{Example}

\begin{verbatim}
-->sweep  5   R14=1,100k   R15=100k,1
>>>list
>>>ac 500 2k oct
>>>go
\end{verbatim}

This sequence of commands says to simultaneously sweep R14 and R15 in 5 steps,
in opposite directions, list the circuit and do an AC analysis for each step.

Assuming the circuit was:
\begin{verbatim}
     R14  1   0   50k
     R15  2   0   50k
\end{verbatim}

The result of this sequence would be:
\begin{verbatim}
     R14  1   0   1
     R15  2   0   100k
\end{verbatim}
{\rm {\it an AC analysis}}
\begin{verbatim}
     R14  1   0   25.75k
     R15  2   0   75.25k
\end{verbatim}
{\rm {\it an AC analysis}}
\begin{verbatim}
     R14  1   0   50.5k
     R15  2   0   50.5k
\end{verbatim}
{\rm {\it an AC analysis}}
\begin{verbatim}
     R14  1   0   75.25k
     R15  2   0   25.75k
\end{verbatim}
{\rm {\it an AC analysis}}
\begin{verbatim}
     R14  1   0   100k
     R15  2   0   1
\end{verbatim}
{\rm {\it an AC analysis}}

After all this is done, the circuit is restored, so {\tt list} would show:
\begin{verbatim}
     R14  1   0   50k
     R15  2   0   50k
\end{verbatim}

You could accomplish the same thing by entering {\tt fault} commands at the
`{\tt >>>}' prompt.

\begin{verbatim}
-->sweep  5
>>>fault  R14=1, 100k
>>>fault  R15=100k, 1
>>>list
>>>ac 500 2k oct
>>>go
\end{verbatim}
%------------------------------------------------------------------------
%------------------------------------------------------------------------
