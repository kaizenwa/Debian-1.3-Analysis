%$Id: help.tex,v 1.2 96/03/30 18:03:21 al Exp $
% man commands help .
%------------------------------------------------------------------------
\section{{\tt HELP} command}
\index{help}
\index{on-line documentation}
%------------------------------------------------------------------------
\subsection{Syntax}
\begin{verse}
{\tt Help} \{{\it subject}\}
\end{verse}
%------------------------------------------------------------------------
\subsection{Purpose}

Provides on line documentation.
%------------------------------------------------------------------------
\subsection{Comments}

{\tt Help} lists the commands.

{\tt Help Help} lists the other subjects on which help is available.

{\tt Help ERrors} lists the error messages.  The word in the error message
that is partially capitalized is a keyword.  {\tt Help} followed by this
word will tell you more about that message.

Types of subjects available include commands, command options, error
messages, and circuit description.

In order for the Help command to work, the file {\tt acs.hlp} must be
available somewhere in the executables search path.  Usually the best
place for it is the same directory as {\tt acs}.
%------------------------------------------------------------------------
\subsection{Examples}

\begin{description}

\item[{\tt help build}] Tells you about the `{\tt Build}' command.

\item[{\tt help mosfet}] Tells you about the MOSFET model usage.

\item[{\tt help errors}] Lists the error messages.  Strange looking
capitalization identifies keywords that will tell you how to get more about
a particular error message.

\item[{\tt help converge}] Gives details about the `{\tt did not converge}'
error message.

\end{description}
%------------------------------------------------------------------------
%------------------------------------------------------------------------
