%$Id: fault.tex,v 1.2 96/03/30 18:03:17 al Exp $
% man commands fault .
%------------------------------------------------------------------------
\section{{\tt FAULT} command}
\index{fault command}
\index{change values: temporary}
%------------------------------------------------------------------------
\subsection{Syntax}
\begin{verse}
{\tt FAult} {\it partlabel}={\it value} ...
\end{verse}
%------------------------------------------------------------------------
\subsection{Purpose}

Temporarily change a component value.
%------------------------------------------------------------------------
\subsection{Comments}

This command quickly changes the value of a component, usually with the
intention that you will not want to save it.

If you apply this command to a nonlinear or otherwise strange part, it becomes
ordinary and linear until the fault is removed.

It is an error to {\tt fault} a model call.

If several components have the same label, the fault value applies to all of
them.  (They will all have the same value.)

The {\tt unfault} command restores the old values.
%------------------------------------------------------------------------
\subsection{Example}

\begin{description}

\item[{\tt fault R66=1k}] R66 now has a value of 1k, regardless of what it
was before.

\item[{\tt fault C12=220p L1=1u}] C12 is 220 pf and L1 is 1 uH, for now.

\item[{\tt unfault}] Clears all faults.  It is back to what it was before.

\end{description}
%------------------------------------------------------------------------
%------------------------------------------------------------------------
