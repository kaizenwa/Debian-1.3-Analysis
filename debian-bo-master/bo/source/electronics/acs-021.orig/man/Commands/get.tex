%$Id: get.tex,v 1.1 96/03/24 11:12:57 al Exp $
% man commands get .
%------------------------------------------------------------------------
\section{{\tt GET} command}
\index{get command}
\index{load circuit from file}
\index{read circuit from file}
\index{retrieve circuit from file}
\index{file: get}
\index{file: read}
%------------------------------------------------------------------------
\subsection{Syntax}
\begin{verse}
{\tt GET} {\it filename}
\end{verse}
%------------------------------------------------------------------------
\subsection{Purpose}

Gets an existing circuit file, after clearing memory.
%------------------------------------------------------------------------
\subsection{Comments}

The first comment line of the file being read is taken as the `title'.  See
the {\tt title} command.

Comments in the circuit file are stored, unless they start with {\tt *+} in
which case they are thrown away.

`Dot cards' are set up, but not executed.  This means that variables and
options are changed, but simulation commands are not actually done.  As
an example, the {\tt options} command is actually performed, since it only
sets up variables.  The {\tt ac} card is not performed, but its parameters
are stored, so that a plain {\tt ac} command will perform the analysis
specified in the file.

Any circuit already in memory will be erased before loading the new circuit.
%------------------------------------------------------------------------
\subsection{Examples}

\begin{description}

\item[{\tt get amp.ckt}] Get the circuit file {\tt amp.ckt} from
the current directory.

\item[{\tt get /usr/foo/ckt/amp.ckt}] Get the file {\tt amp.ckt}
from the {\tt /usr/foo/ckt} directory.

\item[{\tt get npn.mod}] Get the file {\tt npn.mod}.
\end{description}
%------------------------------------------------------------------------
%------------------------------------------------------------------------
