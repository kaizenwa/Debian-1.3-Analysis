%$Id: alarm.tex,v 1.2 96/03/30 18:03:06 al Exp $
% man commands alarm .
%------------------------------------------------------------------------
\section{{\tt ALARM} command}
\index{alarm command}
\index{output selection}
\index{safe operating area}
%------------------------------------------------------------------------
\subsection{Syntax}
\begin{verse}
{\tt ALArm}\\
{\tt ALArm} {\it mode points ...} ...\\
{\tt ALArm} {\it mode + points ...} ...\\
{\tt ALArm} {\it mode - points ...} ...\\
{\tt ALArm} {\it mode} CLEAR
\end{verse}
%------------------------------------------------------------------------
\subsection{Purpose}

Select points in the circuit to check against user defined limits.
%------------------------------------------------------------------------
\subsection{Comments}

The `{\tt alarm}' command selects points in the circuit to check
against limits.  There is no output unless the limits are exceeded.
If the limits are exceeded a the value is printed.

There are separate lists of probe points for each type of analysis.

To list the points, use the bare command `{\tt alarm}'.

The syntax for each point is always {\it parameter(node)(limits)},
{\it parameter(componentlabel)(limits)}, or {\it parameter(index)(limits)}.
Some require a dummy index.

For more information on the data available see the {\tt print}
command.

You can add to or delete from an existing list by prefixing with
{\tt +} or {\tt -}.  {\tt alarm ac + v(3)} adds v(3) to the existing
set of AC probes.  {\tt alarm ac - q(c5)} removes q(c5) from the
list.  You can use the wildcard characters {\tt *} and {\tt ?} when
deleting.
%------------------------------------------------------------------------
\subsection{Examples}

\begin{description}

\item[{\tt alarm ac vm(12)(0,5) vm(13)(-5,5)}] Check magnitude of
the voltage at node 12 against a range of 0 to 5, and node 13
against a range of -5 to 5 for AC analysis.  Print a warning when
the limits are exceeded.

\item[{\tt alarm dc v(r26)}] The voltage across {\tt R26} for DC
analysis.

\item[{\tt alarm tran v(r83)(0,5) p(r83)(0,1u)}] Check the voltage
and power of {\tt R83} in the next transient analysis.  The voltage
range is 0 to 5.  The power range is 0 to 1 microwatt.  Print a
warning when the range is exceeded.

\item[{\tt alarm}] List all the probes for all modes.

\item[{\tt alarm dc}] Display the DC alarm list.

\item[{\tt alarm ac CLear}] Clear the AC list.

\end{description}
%------------------------------------------------------------------------
%------------------------------------------------------------------------
