%$Id: 2batch.tex,v 1.1 96/03/24 11:12:47 al Exp $
% man commands batch .
%------------------------------------------------------------------------
\section{{\tt <} command}
\index{< command}
\index{batch mode}
\index{files}
\index{disk files}
\index{input redirection}
\index{i-o redirection}
\index{redirection: i-o}
%------------------------------------------------------------------------
\subsection{Syntax}
\begin{verse}
{\tt <} {{\it filename}}\\
{\tt <<} {{\it filename}}
\end{verse}
%------------------------------------------------------------------------
\subsection{Purpose}

Run a simulation in batch mode.  Gets the commands and circuit from a disk
file.  {\tt <<} clears the old circuit, first.
%------------------------------------------------------------------------
\subsection{Comments}

You can invoke the batch mode directly from the command that starts the
program.  The first command line argument is considered to be an argument
for this command.

The file format is almost as you would type it on the keyboard.  Commands
must be prefixed with a dot, and circuit elements can be entered directly,
as if in {\em build} mode.  This is compatible with Spice.  The {\tt log}
command can make this file as you work the program, or you can make it with
any editor.  

Any line that starts with {\tt *} a comment line.

Any line that starts with {\tt .} (dot) is a command.

Any line that starts with a letter is a component to be added or changed.

A {\tt <} command in the file transfers control to a new file.  Files can be
nested.

A bare {\tt <} in the file (or the end of the file) gives it back to the
console.

Unlike SPICE, commands are executed in order.  This can result in some
surprises when using some SPICE files.  SPICE queues up commands, then
executes them in a predetermined order.
%------------------------------------------------------------------------
\subsection{Examples}

\begin{description}

\item[{\tt < thisone.ckt}] Activates batch mode, from the file {\tt
thisone.ckt}, in the current directory.

\item[{\tt < runit.bat}] Use the file {\tt runit.bat}.


From the shell: on start up:

\item[{\tt acs afile}] Start up the program.  Start using the file {\tt
afile.ckt} in batch mode, as if you entered {\tt < afile} as the first
command.

\end{description}
%------------------------------------------------------------------------
%------------------------------------------------------------------------
