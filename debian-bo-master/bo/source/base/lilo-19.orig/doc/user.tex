%
% Copyright 1992-1996 Werner Almesberger.
% All rights reserved.
%
% See file COPYING for details.
%

%%def%:=
%:\begin{verbatim}
%:LILO - Generic Boot Loader for Linux ("LInux LOader") by Werner Almesberger
%:===========================================================================
%:
%:Version 19
%:
%:Important: The file INCOMPAT contains vital (in)compatibility information
%:           for this release of LILO. Read it before proceeding.
%:
%:Installing boot loaders is inherently dangerous. Be sure to have some means
%:to boot your system from a different media if you install LILO on your hard
%:disk.
%:
%:
%:There is also a LaTeX version of this document in the "doc" directory.
%:It is much easier nicer to read than pure ASCII.
%:
%:
%:Installation
%:------------
%:
%:Please read the file INCOMPAT for compatibility notes.
%:
%:The installation procedure is described in the section "Normal first-time
%:installation". Please read "Booting basics" for the whole story.
%:
%:*** QUICK INSTALLATION ***
%:
%:    If you want to install LILO on your hard disk and if you don't want
%:    to use all its features, you can use the quick installation script.
%:    Read QuickInst for details.
%:
%:
%:\end{verbatim}

%%beginskip
\documentstyle[fullpage]{article}

\parindent=0pt
\parskip=4pt
\tolerance=9000

\hyphenation{hexa-decimal}

% Okay, what follows is more TeX than LaTeX ...

\def\hditem#1{\hbox to 1.2in{#1\hfil}}
\def\boottwo#1{$$
  \bf
  \begin{tabular}{|ll|}
    \hline
    \hditem{\rm Master Boot Record} & \hditem{\rm Operating system} \\
    \hline
    #1 \\
    \hline
  \end{tabular}
  $$}

\def\bootthree#1{$$
  \bf
  \begin{tabular}{|lll|}
    \hline
    \hditem{\rm Master Boot Record} & \hditem{\rm Boot sector} &
      \hditem{\rm Operating system} \\
    \hline
    #1 \\
    \hline
  \end{tabular}
  $$}

\def\bootfour#1{$$
  \bf
  \begin{tabular}{|llll|}
    \hline
    \hditem{\rm Master Boot Record} & \hbox to 1.1in{\rm Boot sector\hfil} &
      \hditem{\rm Operating systems} & \hbox to 0.4in{\hfil} \\
    \hline
    #1 \\
    \hline
  \end{tabular}
  $$}
\def\sep{\rightarrowfill &}
\def\empty{&}
\def\branch{\hfill$\longrightarrow$ &}
\def\cont{---\,$\cdots$}
\def\key#1{$[$#1$]$}
\def\LILO{LILO}
\def\meta#1{{\it #1\/}}	% ... <blah> ...
\def\path#1{{\tt #1}}   % ... /foo/bar ...
\def\raw#1{{\tt #1}}	% ...  raw output  ...
\def\craw#1{{\tt #1}}	% ...  raw output  ...
\def\name#1{{\sf #1}}	% ... FooBar ...
\def\emphasize#1{{\bf #1}} % ... _don't_ ...
\def\unit#1{\hbox{\tt #1}} % ... \unit{-x foo} ...
%%def\\key{([^{}]*)}=[$1]
%%def\\LILO=LILO
%%def\\meta{([^{}]*)}=<$1>
%%def\\path{([^{}]*)}=$1
%%def\\raw{([^{}]*)}=~$1~
%%cmd\\craw{([^{}]*)}=((($x = $1) =~ tr/a-z/A-Z/), $x)
%%def\\name{={
%%def\\emphasize{([^{}]*)}=\\_$1\\_
%%def\\begin{command}=\\raw{$SI$SI$SI$SI
%%def\\end{command}=$SO$SO$SO$SO}
%%def\\addtocounter{([^{}]*)}{([^{}]*)}=~
%%def\\hbox{([^{}]*)}=$1
%%def(\\unit{[^{} ]*) ([^{}]*})=$1~$2
%%def\\unit{([^{} ]*)}=$1
%%def{\$([^{}])\$}=$1
%%def%4=    
%	^^^^
%       four invisible spaces here
%%def%;=        
%	^^^^^^^^
%       eight invisible spaces here

\newenvironment{command}{\def\[{$\bigl[$}\def\]{$\bigr]$}\def\|{$\big\vert$}%
  \parindent=-2em\advance\leftskip by -\parindent\vskip -\parskip~\par
  \begingroup\tt\textfont0=\font}{%
  ~\endgroup\par\advance\hoffset by \parindent}

\begin{document}

\title{\LILO \\
  \Large
  Generic boot loader for Linux \\
  Version 19 \\ ~\\
  \bf User's guide}
\author{Werner Almesberger \\
  {\tt werner.almesberger@lrc.di.epfl.ch}}
\date{May 21, 1996}

\maketitle
{
  \parskip=-1pt
  \setcounter{tocdepth}{2}
  \tableofcontents
}
%%endskip

~\\
\LILO\ is a versatile boot loader for Linux. It does not depend on a specific
file system, can boot Linux kernel images from
floppy disks and from hard disks and can even boot other operating
systems.\footnote{PC/MS-DOS, DR DOS, OS/2, 386BSD, SCO UNIX, Unixware,
$\ldots$}

One of up to sixteen different images can be selected at boot time.
Various parameters, such as the root
device, can be set independently for each kernel. \LILO\ can even be
used as the master boot record.

This document introduces the basics of disk organization and booting,
continues with an overview of common boot techniques and finally describes
installation and use of \LILO\ in greater detail. The troubleshooting
section at the end describes diagnostic messages and contains suggestions
for most problems that have been observed in the past.

Please read the sections about installation and configuration if you're
already using an older version of \LILO. This distribution is accompanied
by a file named \path{INCOMPAT} that describes further incompatibilities
to older versions.

For the impatient: there is a quick-installation script to create
a simple but quite usable installation. See section \ref{quickinst} for
details.

But wait $\ldots$ here are a few easy rules that will help you avoiding
most problems people experience with \LILO:

\begin{itemize}
  \item \emphasize{Don't panic.} If something doesn't work, try to find
    out what is wrong, try to verify your assumption and only then attempt
    to fix it.
  \item Read the documentation. Especially if what the system does doesn't
    correspond to what you think it should do.
  \item Make sure you have an emergency boot disk, that you know how to
    use it, and that it is always kept up to date.
  \item Run \path{/sbin/lilo} \emphasize{whenever} the kernel or any part
    of \LILO, including its configuration file, has changed.
  \item If using \LILO\ as the MBR, de-install it \emphasize{before}
    performing a destructive upgrade and/or erasing your Linux partitions.
  \item Don't trust setup scripts. Always verify the \path{/etc/lilo.conf}
    they create before booting.
  \item If using a big disk, be prepared for inconveniences: you may have
    to use the \craw{linear} option.
\end{itemize}


\newpage
\subsection*{System overview}

\LILO\ is a collection of several programs and other files:

\begin{description}
  \item[The map installer] is the program you run under Linux to put all
    files belonging to \LILO\ at the appropriate places and to record
    information about the location of data needed at boot time. This program
    normally resides in \path{/sbin/lilo}. It has to be run to refresh that
    information whenever any part of the system that \LILO\ knows about
    changes, e.g. after installing a new kernel.
  \item[Various files] contain data \LILO\ needs at boot time, e.g. the
    boot loader. Those files normally reside in \path{/boot}. The most
    important files are the boot loader (see below) and the map file
    (\path{/boot/map}), where the map installer records the location of
    the kernel(s).\footnote{\LILO\ does not know how to read a file system.
    Instead, the map installer asks the kernel for the physical location of
    files (e.g. the kernel image(s)) and records that information. This
    allows \LILO\ to work with most file systems that are supported by
    Linux.} Another important file is the configuration file, which is
    normally called \path{/etc/lilo.conf}
  \item[The boot loader] is the part of \LILO\ that is loaded by the BIOS
    and that loads kernels or the boot sectors of other operating systems.
    It also provides a simple command-line interface to interactively
    select the item to boot and to add boot options.
\end{description}

\LILO\ primarily accesses the following parts of the system:

\begin{description}
  \item[The root file system partition] is important for two reasons: first,
    \LILO\ sometimes has to tell the kernel where to look for it. Second, it
    is frequently a convenient place for many other items \LILO\ uses, such
    as the boot sector, the \path{/boot} directory, and the kernels.
  \item[The boot sector] contains the first part of \LILO's boot loader. It
    loads the much larger second-stage loader. Both loaders are typically
    stored in the file \path{/boot/boot.b}
  \item[The kernel] is loaded and started by the boot loader. Kernels
    typically reside in the root directory or in \path{/boot}.
\end{description}

Note that many of the files \LILO\ needs at boot time have to be accessible
with the BIOS. This creates certain restrictions, see section \ref{broken}.


\newpage
\section{Introduction}

The following sections describe how PCs
boot in general and what has to be known when booting Linux and using
\LILO\ in particular.


\subsection{Disk organization}
\label{diskorg}

When designing a boot concept, it is important to understand some of the
subtleties of how PCs typically organize disks. The most simple case are
floppy disks. They consist of a boot sector, some administrative
data (FAT or super block, etc.) and the data area. Because that
administrative data is irrelevant as far as booting is concerned, it is
regarded as part of the data area for simplicity.

%%beginskip
$$
\begin{tabular}{|c|c|}
  \hline
  Boot sector & \hbox to 1.5in{\hfil} \\
  \cline{1-1}
  \multicolumn{2}{|c|}{} \\
  \multicolumn{2}{|c|}{Data area} \\
  \multicolumn{2}{|c|}{} \\
  \hline
  \end{tabular}
$$
%%endskip
%:\begin{verbatim}
%;    +---------------------------+
%;    |Boot sector|               |
%;    |-----------+               |
%;    |                           |
%;    |         Data area         |
%;    |                           |
%;    |                           |
%;    +---------------------------+
%:\end{verbatim}

The entire disk appears as one device (e.g. \path{/dev/fd0}) on Linux.

The MS-DOS boot sector has the following structure:

%%beginskip
$$
\begin{tabular}{r|c|}
  \cline{2-2}
  \tt 0x000 & Jump to the program code\\
  \cline{2-2}
  \tt 0x003 & \\
  & Disk parameters \\
  & \\
  \cline{2-2}
  \tt 0x02C/0x03E & \\
  & Program code \\
  & \\
  & \\
  \cline{2-2}
  \tt 0x1FE & Magic number (0xAA55)\\
  \cline{2-2}
\end{tabular}
$$
%%endskip
%:\begin{verbatim}
%;            +------------------------+
%;      0x000 |Jump to the program code|
%;            |------------------------|
%;      0x003 |                        |
%;            |    Disk parameters     |
%;            |                        |
%;            |------------------------|
%;0x02C/0x03E |                        |
%;            |      Program code      |
%;            |                        |
%;            |                        |
%;            |------------------------|
%;      0x1FE | Magic number (0xAA55)  |
%;            +------------------------+
%:\end{verbatim}

\LILO\ uses a similar boot sector, but it does not contain the disk
parameters part. This is no problem for Minix, Ext2 or similar file systems,
because
they don't look at the boot sector, but putting a \LILO\ boot sector on an
MS-DOS file system would make it inaccessible for MS-DOS.

Hard disks are organized in a more complex way than floppy disks. They
contain several data areas called partitions. Up to four so-called
primary partitions can exist on an MS-DOS hard disk. If more partitions
are needed, one primary partition is used as an extended partition that
contains several logical partitions.

The first sector of each hard disk contains a partition table, and an
extended partition and \emphasize{each} logical partition contains a partition
table too.

%%beginskip
$$
\begin{tabular}{|l|l|l|}
  \hline
  \multicolumn{3}{|l|}{Partition table\hbox to 2in{\hfil\tt /dev/hda~}} \\
  \cline{2-3}
  & \multicolumn{2}{l|}{Partition 1\hfill {\tt /dev/hda1}} \\
  & \multicolumn{2}{l|}{} \\
  \cline{2-3}
  & \multicolumn{2}{l|}{Partition 2\hfill {\tt /dev/hda2}} \\
  & \multicolumn{2}{l|}{} \\
  \hline
\end{tabular}
$$
%%endskip
%:\begin{verbatim}
%;+--------------------------------------------+
%;| Partition table                  /dev/hda  |
%;| +------------------------------------------|
%;| | Partition 1                    /dev/hda1 |
%;| |                                          |
%;| |------------------------------------------|
%;| | Partition 2                    /dev/hda2 |
%;| |                                          |
%;+--------------------------------------------+
%:\end{verbatim}

The entire disk can be accessed as \path{/dev/hda}, \path{/dev/hdb},
\path{/dev/sda}, etc. The primary partitions are \path{/dev/hda1 $\ldots$
/dev/hda4}.

%%beginskip
$$
\begin{tabular}{|l|l|l|}
  \hline
  \multicolumn{3}{|l|}{Partition table\hbox to 2in{\hfil\tt /dev/hda~}} \\
  \cline{2-3}
  & \multicolumn{2}{l|}{Partition 1\hfill {\tt /dev/hda1}} \\
  & \multicolumn{2}{l|}{} \\
  \cline{2-3}
  & \multicolumn{2}{l|}{Partition 2\hfill {\tt /dev/hda2}} \\
  & \multicolumn{2}{l|}{} \\
  \cline{2-3}
  & \multicolumn{2}{l|}{Extended partition\hfill {\tt /dev/hda3}} \\
  \cline{3-3}
  & & Extended partition table \\
  \cline{3-3}
  & & Partition 3\hfill {\tt /dev/hda5}\\
  & & \\
  \cline{3-3}
  & & Extended partition table \\
  \cline{3-3}
& & Partition 4\hfill {\tt /dev/hda6}\\
  & & \\
  \hline
\end{tabular}
$$
%%endskip
%:\begin{verbatim}
%;+--------------------------------------------+
%;| Partition table                  /dev/hda  |
%;| +------------------------------------------|
%;| | Partition 1                    /dev/hda1 |
%;| |                                          |
%;| |------------------------------------------|
%;| | Partition 2                    /dev/hda2 |
%;| |                                          |
%;| |------------------------------------------|
%;| | Extended partition             /dev/hda3 |
%;| | +----------------------------------------|
%;| | | Extended partition table               |
%;| | |----------------------------------------|
%;| | | Partition 3                  /dev/hda5 |
%;| | |                                        |
%;| | |----------------------------------------|
%;| | | Extended partition table               |
%;| | |----------------------------------------|
%;| | | Partition 4                  /dev/hda6 |
%;| | |                                        |
%;+--------------------------------------------+
%:\end{verbatim}

This hard disk has two primary partitions and an extended partition
that contains two logical partitions. They are accessed as
\path{/dev/hda5 $\ldots$}

Note that the partition tables of logical partitions are not accessible
as the first blocks of some devices, while the main partition table,
all boot sectors and the partition tables of extended partitions are.

Partition tables are stored in partition boot sectors. Normally, only the
partition boot sector of the entire disk is used as a boot sector. It is
also frequently called the master boot record (MBR).

%%beginskip
$$
\begin{tabular}{r|c|}
  \cline{2-2}
  \tt 0x000 & \\
  & Program code \\
  & \\
  & \\
  \cline{2-2}
  \tt 0x1BE & Partition table \\
  & \\
  \cline{2-2}
  \tt 0x1FE & Magic number (0xAA55) \\
  \cline{2-2}
\end{tabular}
$$
%%endskip
%:\begin{verbatim}
%;            +------------------------+
%;      0x000 |                        |
%;            |      Program code      |
%;            |                        |
%;            |                        |
%;            |------------------------|
%;      0x1BE |    Partition table     |
%;            |                        |
%;            |------------------------|
%;      0x1FE | Magic number (0xAA55)  |
%;            +------------------------+
%:\end{verbatim}

The \LILO\ boot sector is designed to be usable as a partition boot sector.
(I.e. there is room for the partition table.)
Therefore, the \LILO\ boot sector can be stored at the following locations:

\begin{itemize}
  \item boot sector of a Linux floppy disk. (\path{/dev/fd0}, $\ldots$)
  \item MBR of the first hard disk. (\path{/dev/hda}, $\ldots$)
  \item boot sector of a primary Linux file system partition on the first hard
    disk. (\path{/dev/hda1}, $\ldots$)
  \item partition boot sector of an extended partition on the first hard disk.
    (\path{/dev/hda1}, $\ldots$)\footnote{Most FDISK-type programs don't
      believe in booting from an extended partition and refuse to
      activate it. \LILO\ is accompanied by a simple program
      (\name{activate}) that doesn't have this restriction. Linux \name{fdisk}
      also supports activating extended partitions.}
\end{itemize}

It \emphasize{can't} be stored at any of the following locations:

\begin{itemize}
  \item boot sector of a non-Linux floppy disk or primary partition.
  \item a Linux swap partition.
  \item boot sector of a logical partition in an extended partition.%
\footnote{\LILO\ can be forced to put the boot sector on such a
 partition by using the \raw{-b} option or the \craw{boot} variable.
 However, only few programs that operate as master boot records
 support booting from a logical partition.}
  \item on the second hard disk. (Unless for backup installations,
    if the current first disk will be removed or disabled, or if some
    other boot loader is used, that is capable of loading boot sectors from
    other drives.)
\end{itemize}

\LILO\ typically doesn't detect attempts to put its boot sector at an
invalid location.


\subsection{Booting basics}

When booting from a floppy disk, the first sector of the disk, the so-called
boot sector, is loaded. That boot sector contains a small program that loads
the respective operating system. MS-DOS boot sectors also contain 
a data area, where disk and file system parameters (cluster size, number of
sectors, number of heads, etc.) are stored.

When booting from a hard disk, the very first sector of that disk, the
so-called master boot record (MBR) is loaded. This sector contains a
loader program and the partition table of the disk. The loader program
usually loads the boot sector, as if the system was booting from a floppy.

Note that there is no functional difference between the MBR and the boot
sector other than that the MBR contains the partition information but
doesn't contain any file system-specific information (e.g. MS-DOS disk
parameters).

The first 446 (0x1BE) bytes of the MBR are used by the loader program.
They are followed by the partition table, with a length of 64 (0x40)
bytes. The last two bytes contain a magic number that is sometimes used to
verify that a given sector really is a boot sector.

There is a large number of possible boot configurations. The most common
ones are described in the following sections.


\subsubsection{MS-DOS alone}

%%beginskip
\bootthree{DOS-MBR \sep MS-DOS \sep COMMAND.COM}
%%endskip
%:\begin{verbatim}
%;+-------------------------------------------------------+
%;| Master Boot Record    Boot sector    Operating system |
%;|-------------------------------------------------------|
%;| DOS-MBR ------------> MS-DOS ------> COMMAND.COM      |
%;+-------------------------------------------------------+
%:\end{verbatim}

This is what usually happens when MS-DOS boots from a hard disk: the DOS-MBR
determines the active partition and loads the MS-DOS boot sector. This boot
sector loads MS-DOS and finally passes control to \path{COMMAND.COM}. (This is
greatly simplified.)


\subsubsection{BOOTLIN\protect\footnotemark}
\footnotetext{A more recent program called \name{LOADLIN} uses
the same approach. Those who wish to use this method of booting are advised
to use \name{LOADLIN} instead of \name{BOOTLIN}.}

%%beginskip
\bootfour{DOS-MBR \sep MS-DOS \sep COMMAND.COM \empty \\
  \empty \branch BOOTLIN \sep Linux}
%%endskip
%:\begin{verbatim}
%4+------------------------------------------------------------+
%4| Master Boot Record    Boot sector    Operating system      |
%4|------------------------------------------------------------|
%4| DOS-MBR ------------> MS-DOS ------> COMMAND.COM           |
%4|                                 ---> BOOTLIN ------> Linux |
%4+------------------------------------------------------------+
%:\end{verbatim}

A typical \name{BOOTLIN} setup: everything happens like when booting MS-DOS,
but in \path{CONFIG.SYS} or \path{AUTOEXEC.BAT}, BOOTLIN is invoked. Typically,
a program like \path{BOOT.SYS} is used to choose among configuration sections
in \path{CONFIG.SYS} and \path{AUTOEXEC.BAT}.
This approach has the pleasant property that no boot
sectors have to be altered.

Please refer to the documentation accompanying the package for installation
instructions and further details.


\subsubsection{\LILO\ started by DOS-MBR}

%%beginskip
\bootthree{DOS-MBR \sep LILO \sep Linux \\
  \branch {\rm other OS} \empty}
%%endskip
%:\begin{verbatim}
%;+-------------------------------------------------------+
%;| Master Boot Record    Boot sector    Operating system |
%;|-------------------------------------------------------|
%;| DOS-MBR ------------> LILO --------> Linux            |
%;|                  ---> other OS                        |
%;+-------------------------------------------------------+
%:\end{verbatim}

This is a ``safe'' \LILO\ setup: \LILO\ is booted by the DOS-MBR. No other boot
sectors have to be touched. If the other OS (or one of them, if there are
several other operating systems being used) should be booted without
using \LILO, the other
partition has to be marked ``active'' with \name{fdisk} or \name{activate}.

Installation:
\begin{itemize}
  \item install \LILO\ with its boot sector on the Linux partition.
  \item use \name{fdisk} or \name{activate} to make that partition active.
  \item reboot.
\end{itemize}

Deinstallation:
\begin{itemize}
  \item make a different partition active.
  \item install whatever should replace \LILO\ or Linux.
\end{itemize}


\subsubsection{Several alternate branches}

%%beginskip
\bootfour{DOS-MBR \sep MS-DOS \sep COMMAND.COM \empty \\
  \empty \branch BOOTLIN \sep Linux \\
  \branch LILO \sep Linux \empty \\
  \empty \branch MS-DOS \cont \empty}
%%endskip
%:\begin{verbatim}
%4+------------------------------------------------------------+
%4| Master Boot Record    Boot sector    Operating system      |
%4|------------------------------------------------------------|
%4| DOS-MBR ------------> MS-DOS ------> COMMAND.COM           |
%4|                                 ---> BOOTLIN ------> Linux |
%4|                  ---> LILO --------> Linux                 |
%4|                                 ---> MS-DOS --- ...        |
%4+------------------------------------------------------------+
%:\end{verbatim}

An extended form of the above setup: the MBR is not changed and both branches
can either boot Linux or MS-DOS. (\LILO\ could also boot any other
operating system.)


\subsubsection{\LILO\ started by \name{BOOTACTV}\protect\footnotemark}
\footnotetext{Other, possibly better known boot switchers, e.g. \name{OS/2
Bootmanager} operate in a similar way. The installation procedures typically
vary.}

%%beginskip
\bootthree{BOOTACTV \sep LILO \sep Linux \\
  \branch {\rm other OS} \empty}
%%endskip
%:\begin{verbatim}
%;+-------------------------------------------------------+
%;| Master Boot Record    Boot sector    Operating system |
%;|-------------------------------------------------------|
%;| BOOTACTV -----------> LILO --------> Linux            |
%;|                  ---> other OS                        |
%;+-------------------------------------------------------+
%:\end{verbatim}

Here, the MBR is replaced by \name{BOOTACTV} (or any other interactive boot
partition selector) and the choice between Linux and the
other operating system(s) can be made at boot time. This approach should be
used if \LILO\ fails to boot the other operating system(s).\footnote{%
And the author would like to be notified if booting the other operating
system(s) doesn't work with \LILO, but if it works with an other boot partition
selector.}

Installation:
\begin{itemize}
  \item boot Linux.
  \item make a backup copy of your MBR on a floppy disk, e.g. \\
    \verb"dd if=/dev/hda of=/fd/MBR bs=512 count=1"
  \item install \LILO\ with the boot sector on the Linux partition.
  \item install \name{BOOTACTV} as the MBR, e.g. \\
    \verb"dd if=bootactv.bin of=/dev/hda bs=446 count=1"
  \item reboot.
\end{itemize}

Deinstallation:
\begin{itemize}
  \item boot Linux.
  \item restore the old MBR, e.g. \\
    \verb"dd if=/MBR of=/dev/hda bs=446 count=1" \\
    or \verb"FDISK /MBR" under MS-DOS.
\end{itemize}

If replacing the MBR appears undesirable and if a second Linux partition
exists (e.g. \path{/usr}, \emphasize{not} a swap partition), \name{BOOTACTV}
can be merged with
the partition table and stored as the ``boot sector'' of that partition.
Then, the partition can be marked active to be booted by the DOS-MBR.

Example:
\begin{verbatim}
# dd if=/dev/hda of=/dev/hda3 bs=512 count=1
# dd if=bootactv.bin of=/dev/hda3 bs=446 count=1
\end{verbatim}

\emphasize{WARNING:} Whenever the disk is re-partitioned, the merged boot
sector on that ``spare'' Linux partition has to be updated too.


\subsubsection{\LILO\ alone}

%%beginskip
\boottwo{LILO \sep Linux \\
  \branch {\rm other OS}}
%%endskip
%:\begin{verbatim}
%;    +----------------------------------------+
%;    | Master Boot Record    Operating system |
%;    |----------------------------------------|
%;    | LILO ---------------> Linux            |
%;    |                  ---> other OS         |
%;    +----------------------------------------+
%:\end{verbatim}

\LILO\ can also take over the entire boot procedure. If installed as the MBR,
\LILO\ is responsible for either booting Linux or any other OS. This approach
has the disadvantage, that the old MBR is overwritten and has to be restored
(either from a backup copy, with \verb"FDISK /MBR" on recent versions of
MS-DOS or by overwriting it with something like \name{BOOTACTV}) if Linux
should ever be removed from the system.

You should verify that \LILO\ is able to boot your other operating system(s)
before relying on this method.

Installation:
\begin{itemize}
  \item boot Linux.
  \item make a backup copy of your MBR on a floppy disk, e.g. \\
    \verb"dd if=/dev/hda of=/fd/MBR bs=512 count=1"
  \item install \LILO\ with its boot sector as the MBR.
  \item reboot.
\end{itemize}

Deinstallation:
\begin{itemize}
  \item boot Linux.
  \item restore the old MBR, e.g. \\
    \verb"dd if=/fd/MBR of=/dev/hda bs=446 count=1"
\end{itemize}

If you've installed \LILO\ as the master boot record, you have to
explicitly specify the boot sector (configuration variable \craw{boot=$\ldots$})
when updating the map. Otherwise, it
will try to use the boot sector of your current root partition, which
may even work, but will probably leave your system unbootable.


\subsubsection{Names}

The following names have been used to describe boot sectors or parts of
operating systems:

\begin{description}
  \item[``DOS-MBR''] is the original MS-DOS MBR. It scans the partition
    table for a partition that is marked ``active'' and loads the boot
    sector of that partition. Programs like MS-DOS' \name{FDISK}, Linux
    \name{fdisk} or \name{activate} (accompanies \LILO)
    can change the active marker in the partition table.
  \item[``MS-DOS''] denotes the MS-DOS boot sector that loads the other parts
    of the system (\path{IO.SYS}, etc.).
  \item[``COMMAND.COM''] is the standard command interpreter of MS-DOS.
  \item[``BOOTLIN'' and ``LOADLIN''] are programs that load a Linux
    kernel image from an
    MS-DOS partition into memory and execute it. They are usually invoked
    from \path{CONFIG.SYS} and used in combination with a \path{CONFIG.SYS}
    configuration switcher, like BOOT.SYS.\footnote{\name{BOOTLIN} is
      available for anonymous FTP from \\
      \path{ftp://nic.funet.fi/pub/Linux/tools/bootlin.zip} \\
      \name{LOADLIN} is available for anonymous FTP from \\
      \path{ftp://tsx-11.mit.edu/pub/linux/dos\_utils/lodlin\meta{n}.tar.gz} \\
      \path{ftp://sunsite.unc.edu/pub/Linux/system/Linux-boot/lodlin\meta{n}%
.tgz} \\
      \name{BOOT.SYS} is available for anonymous FTP from \\
      \path{ftp://sunsite.unc.edu/pub/Linux/system/Linux-boot/boot142.zip} or \\
      \path{ftp://nic.funet.fi/pub/Linux/tools/boot142.zip}}
  \item[``LILO''] can either load a Linux kernel or the boot sector of any
    other operating system. It has a first stage boot sector that
    loads the remaining parts of \LILO\ from various locations.\footnote{%
      \LILO\ can be found in \\
      \path{ftp://tsx-11.mit.edu/pub/linux/packages/lilo/lilo.\meta{n}.tar.gz}
	\\
      \path{ftp://sunsite.unc.edu/pub/Linux/system/Linux-boot/lilo/lilo.%
\meta{n}.tar.gz} \\
      \path{ftp://nic.funet.fi/pub/Linux/tools/lilo.\meta{n}.tar.gz} \\
      \path{ftp://lrcftp.epfl.ch/pub/linux/local/lilo/lilo.\meta{n}.tar.gz}}
  \item[``BOOTACTV''] permits interactive selection of the partition from
    which the boot sector should be read. If no key is pressed within a
    given interval, the partition marked active is booted. \name{BOOTACTV} is
    included in the \name{pfdisk} package. There are also several similar
    programs, like PBOOT and OS-BS.\footnote{\name{pfdisk} is available for
      anonymous FTP from \\
      \path{ftp://sunsite.unc.edu/pub/Linux/utils/disk-management/pfdisk.tar.Z}
	or \\
      \path{ftp://nic.funet.fi/pub/Linux/tools/pfdisk.tar.Z} \\
      PBOOT can be found in \\
      \path{ftp://nic.funet.fi/pub/Linux/tools/pboot.zip}}
\end{description}


\subsection{Choosing the ``right'' boot concept}

Although \LILO\ can be installed in many different ways, the choice is
usually limited by the present setup and therefore,
typically only a small number of configurations which fit naturally
into an existing system remains.

The configuration file \path{/etc/lilo.conf} for the examples
could look like this:

\begin{verbatim}
boot = /dev/hda2
compact
image = /vmlinuz
image = /vmlinuz.old
other = /dev/hda1
  table = /dev/hda
  label = msdos
\end{verbatim}

It installs a Linux kernel image (\path{/vmlinuz}),
an alternate Linux kernel image (\path{/vmlinuz.old}) and
a chain loader to boot MS-DOS from \path{/dev/hda1}. The option \craw{compact}
on the second line instructs the map installer to optimize loading.

In all examples, the names of the IDE-type hard disk devices
(\path{/dev/hda$\ldots$}) are used. Everything applies to other disk types
(e.g. SCSI disks; \path{/dev/sda$\ldots$}) too.


\subsubsection{BIOS restrictions}
\label{broken}

Nowadays, an increasing number of systems is equipped with comparably large
disks or even with multiple disks. At the time the disk interface of the
standard PC BIOS has been designed (about 15 years ago), such configurations
were apparently considered to be too unlikely to be worth supporting.

The most common BIOS restrictions that affect \LILO\ are the limitation to
two hard disks and the inability to access more than 1024 cylinders per
disk. \LILO\ can detect both conditions, but in order to solve the
underlying problems, manual intervention is necessary.

The drive limit does not exist in every BIOS. Some modern motherboards
and disk controllers are equipped with a BIOS that supports more (typically
four) disk drives. When attempting to access the third, fourth, etc. drive,
\LILO\ prints a warning message but continues. Unless the BIOS really
supports more than two drives, the system will \emphasize{not} be able to
boot in that case.\footnote{If only ``unimportant'' parts of the system
are located on the ``high'' drives, some functionality may be available.}

The cylinder limit is a very common problem with IDE disks. There, the
number of cylinders typically exceeds 1024 if the drive has a capacity
of more than 504 MB. Many SCSI driver BIOSes present the disk geometry
in a way that makes the limit occur near 1 GB. Modern disk controllers
may even push the limit up to about 8 GB. All cylinders beyond the
1024th are inaccessible for the BIOS. \LILO\ detects this problem and
aborts the installation (unless the \craw{linear} option is used, see
section \ref{cfgglo}).

Note that large partitions that only partially extend into the ``forbidden
zone'' are still in jeopardy even if they appear to work at first, because
the file system does not know about the restrictions and may allocate
disk space from the area beyond the 1024th cylinder when installing new
kernels. \LILO\ therefore prints a warning message but continues as long
as no imminent danger exists.

There are four approaches of how such problems can be solved:
\begin{itemize}
  \item use of a different partition which is on an accessible disk and
    which does not exceed the 1024 cylinder limit. If there is only a DOS
    partition which fulfills all the criteria, that partition can be
    used to store the relevant files. (See section \ref{viados}.)
  \item rearranging partitions and disks. This is typically a destructive
    operation, so care should be taken to make good backups.
  \item if the system is running DOS, \name{LOADLIN} can be used instead
    of \LILO.
  \item if all else fails, installation of a more capable BIOS, a different
    controller or a different disk configuration.
\end{itemize}

\LILO\ depends on the BIOS to load the following items:
\begin{itemize}
  \item \path{/boot/boot.b}
  \item \path{/boot/map} (created when running \path{/sbin/lilo})
  \item all kernels
  \item the boot sectors of all other operating systems it boots
  \item the startup message, if one has been defined
\end{itemize}

Normally, this implies that the Linux root file system should be in the
``safe'' area. However, it is already sufficient to put all kernels into
\path{/boot} and to either mount a ``good'' partition on \path{/boot} or
to let \path{/boot} be a symbolic link pointing to or into such a
partition.

See also \path{/usr/src/linux/drivers/block/README.ide} for a detailed
description of problems with large disks.


\subsubsection{One disk, Linux on a primary partition}

If at least one primary partition of the first hard disk is used as a
Linux file system (\path{/}, \path{/usr}, etc. but \emphasize{not} as a swap
partition), the \LILO\ boot sector should be stored on that partition
and it should be booted by the original master boot record or by a
program like \name{BOOTACTV}.

%%beginskip
$$
  \begin{tabular}{r|c|c|}
    \cline{2-3}
    & \multicolumn{2}{|l|}{MBR\hbox to 1.3in{\hfill\tt /dev/hda~}} \\
    \cline{3-3}
    & & MS-DOS\hfill\tt /dev/hda1 \\
    \cline{3-3}
    $\rightarrow$ & & Linux {\tt /}\hfill\tt /dev/hda2 \\
    \cline{2-3}
  \end{tabular}
$$
%%endskip
%:\begin{verbatim}
%;    +--------------------------+
%;    | MBR            /dev/hda  |
%;    | +------------------------|
%;    | | MS-DOS       /dev/hda1 |
%;    | |------------------------|
%;--> | | Linux /      /dev/hda2 |
%;    +--------------------------+
%:\end{verbatim}

In this example, the \craw{boot} variable could be omitted, because
the boot sector is on the root partition.


\subsubsection{One disk, Linux on a logical partition}

If no primary partition is available for Linux, but at least one logical
partition of an extended partition on the first hard disk contains a
Linux file system, the \LILO\ boot sector should be stored in the partition
sector of the extended partition and it should be booted by the original
master boot record or by a program like \name{BOOTACTV}.

%%beginskip
$$
  \begin{tabular}{r|c|c|c|}
    \cline{2-4}
    & \multicolumn{3}{|l|}{MBR\hbox to 1.3in{\hfill\tt /dev/hda~}} \\
    \cline{3-4}
    & & \multicolumn{2}{|l|}{MS-DOS\hfill\tt /dev/hda1} \\
    \cline{3-4}
    $\rightarrow$ & & \multicolumn{2}{|l|}{Extended\hfill\tt /dev/hda2} \\
    \cline{4-4}
    & & & Linux\hfill\tt /dev/hda5 \\
    \cline{4-4}
    & & & $\ldots$\hfill\tt /dev/hda6 \\
    \cline{2-4}
  \end{tabular}
$$
%%endskip
%:\begin{verbatim}
%;    +--------------------------+
%;    | MBR            /dev/hda  |
%;    | +------------------------|
%;    | | MS-DOS       /dev/hda1 |
%;    | |------------------------|
%;--> | | Extended     /dev/hda2 |
%;    | | +----------------------|
%;    | | | Linux      /dev/hda5 |
%;    | | |----------------------|
%;    | | | ...        /dev/hda6 |
%;    +--------------------------+
%:\end{verbatim}

Because many disk partitioning programs refuse to make an extended
partition (in our example \path{/dev/hda2}) active, you might have to
use \name{activate}, which comes with the \LILO\ distribution.

OS/2 BootManager should be able to boot \LILO\ boot sectors from logical
partitions. The installation on the extended partition itself is not
necessary in this case.


\subsubsection{Two disks, Linux (at least partially) on the first disk}

This case is equivalent to the configurations where only one disk
is in the system. The Linux boot sector resides on the first hard
disk and the second disk is used later in the boot process.

Only the location of the boot sector matters -- everything
else (\path{/boot/boot.b},
\path{/boot/map}, the root file system, a swap partition, other
Linux file systems, etc.) can be located anywhere on the second disk.


\subsubsection{Two disks, Linux on second disk, first disk has an extended
  partition}

If there is no Linux partition on the first disk, but there is an
extended partition, the \LILO\ boot sector can be stored in the partition
sector of the extended partition and it should be booted by the original
master boot record or by a program like \name{BOOTACTV}.

%%beginskip
$$
  \begin{tabular}{r|c|c|c|c|c|c|}
    \multicolumn{1}{r}{}
    & \multicolumn{3}{c}{\bf First disk} &
      \multicolumn{1}{r}{\qquad}
      & \multicolumn{2}{c}{\bf Second disk} \\
    \cline{2-4}\cline{6-7}
    & \multicolumn{3}{|l|}{MBR\hbox to 1.3in{\hfill\tt /dev/hda~}} &
      & \multicolumn{2}{|l|}{MBR\hbox to 1.3in{\hfill\tt /dev/hdb~}} \\
    \cline{3-4}\cline{7-7}
    & & \multicolumn{2}{|l|}{MS-DOS\hfill\tt /dev/hda1} &
      & & Linux\hfill\tt /dev/hdb1 \\
    \cline{3-4}\cline{7-7}
    $\rightarrow$ & & \multicolumn{2}{|l|}{Extended\hfill\tt /dev/hda2} &
      & & $\ldots$\hfill\tt /dev/hdb2 \\
    \cline{4-4}
    & & & $\ldots$\hfill\tt /dev/hda5 & & & \\
    \cline{4-4}
    & & & $\ldots$\hfill\tt /dev/hda6 & & & \\
    \cline{2-4}\cline{6-7}
  \end{tabular}
$$
%%endskip
%:\begin{verbatim}
%4             FIRST DISK                     SECOND DISK
%4    +--------------------------+    +--------------------------+
%4    | MBR            /dev/hda  |    | MBR            /dev/hdb  |
%4    | +------------------------|    | +------------------------|
%4    | | MS-DOS       /dev/hda1 |    | | Linux        /dev/hdb1 |
%4    | |------------------------|    | |------------------------|
%4--> | | Extended     /dev/hda2 |    | | ...          /dev/hdb2 |
%4    | | +----------------------|    | |                        |
%4    | | | ...        /dev/hda5 |    | |                        |
%4    | | |----------------------|    | |                        |
%4    | | | ...        /dev/hda6 |    | |                        |
%4    +--------------------------+    +--------------------------+
%:\end{verbatim}

The program \name{activate}, that accompanies \LILO, may have to be used to
set the active marker on an extended partition, because MS-DOS' \name{FDISK}
and some older version of Linux \name{fdisk} refuse to do that. (Which is
generally a good idea.)


\subsubsection{Two disks, Linux on second disk, first disk has no
  extended partition}

If there is neither a Linux partition nor an extended partition on the first
disk, then there's only one place left, where a \LILO\ boot sector could be
stored: the master boot record.

In this configuration, \LILO\ is responsible for booting all other operating
systems too.

%%beginskip
$$
  \begin{tabular}{r|c|c|c|c|c|}
    \multicolumn{1}{r}{}
    & \multicolumn{2}{c}{\bf First disk} &
      \multicolumn{1}{r}{\qquad}
      & \multicolumn{2}{c}{\bf Second disk} \\
    \cline{2-3}\cline{5-6}
    $\rightarrow$ & \multicolumn{2}{|l|}{MBR\hbox to 1.3in{
      \hfill\tt /dev/hda~}} &
      & \multicolumn{2}{|l|}{MBR\hbox to 1.3in{\hfill\tt /dev/hdb~}} \\
    \cline{3-3}\cline{6-6}
    & & MS-DOS\hfill\tt /dev/hda1 &
      & & Linux\hfill\tt /dev/hdb1 \\
    \cline{3-3}\cline{6-6}
    & & $\ldots$\hfill\tt /dev/hda2 &
      & & $\ldots$\hfill\tt /dev/hdb2 \\
    \cline{2-3}\cline{5-6}
  \end{tabular}
$$
%%endskip
%:\begin{verbatim}
%4             FIRST DISK                     SECOND DISK
%4    +--------------------------+    +--------------------------+
%4--> | MBR            /dev/hda  |    | MBR            /dev/hdb  |
%4    | +------------------------|    | +------------------------|
%4    | | MS-DOS       /dev/hda1 |    | | Linux        /dev/hdb1 |
%4    | |------------------------|    | |------------------------|
%4    | | ...          /dev/hda2 |    | | ...          /dev/hdb2 |
%4    +--------------------------+    +--------------------------+
%:\end{verbatim}

You should back up your old MBR before installing \LILO\ and verify that
\LILO\ is able to boot your other operating system(s) before relying on
this approach.

The line \verb"boot = /dev/hda2" in \path{/etc/lilo.conf} would have to be
changed to \verb"boot = /dev/hda" in this example.


\subsubsection{More than two disks}

On systems with more than two disks, typically only the first two can
be accessed. The configuration choices are therefore the same as with
two disks.

When attempting to access one of the extra disks, \LILO\ displays a
warning message (\raw{Warning: BIOS drive 0x\meta{number} may not be
accessible}) but does not abort. This is done in order to allow
the lucky few whose BIOS (or controller-BIOS) does support more than
two drives to make use of this feature. By all others, this warning
should be considered a fatal error.

Note that the two disks restriction is only imposed by the BIOS.
Linux normally has no problems using all disks once it is booted.


\subsubsection{\path{/boot} on a DOS partition}
\label{viados}

Recent kernels support all the functions \LILO\ needs to map files also
on MS-DOS (or UMSDOS) file systems. Since DOS partitions tend to occupy
exactly the places where BIOS restrictions (see section \ref{broken})
are invisible, they're an ideal location for \path{/boot} if the native
Linux file systems can't be used because of BIOS problems.

In order to accomplish this, the DOS partition is mounted read-write,
a directory (e.g. \path{/dos/linux}) is created, all files from
\path{/boot} are moved to that directory, \path{/boot} is replaced by
a symbolic link to it, the kernels are also moved to the new directory,
their new location is recorded in \path{/etc/lilo.conf}, and finally
\path{/sbin/lilo} is run.

From then on, new kernels must always be copied into that directory on
the DOS partition before running \path{/sbin/lilo}, e.g. when recompiling
a kernel, the standard procedure changes from

\begin{verbatim}
# make zlilo
\end{verbatim}

to

\begin{verbatim}
# make zImage
# mv /dos/linux/vmlinuz /dos/linux/vmlinuz.old
# mv arch/i386/boot/zImage /dos/linux/vmlinuz
# /sbin/lilo
\end{verbatim}

\emphasize{WARNING:} De-fragmenting such a DOS partition is likely to
make Linux or even the whole system unbootable. Therefore, the DOS
partition should either not be de-fragmented, or a Linux boot disk
should be prepared (and tested) to bring up Linux and to run
\path{/sbin/lilo} after the de-fragmentation.\footnote{Setting the
``system'' attribute from DOS on the critical files (e.g. everything
in \path{C:$\backslash$LINUX}) may help to protect them from being
rearranged. However, the boot floppy should still be ready, just in case.}


\newpage
\section{The boot prompt}
\label{cmdopt}

Immediately after it's loaded, \LILO\ checks whether one of the following
is happening:

\begin{itemize}
  \item any of the \key{Shift}, \key{Control} or \key{Alt} keys is being
    pressed.
  \item \key{CapsLock} or \key{ScrollLock} is set.
\end{itemize}

If this is the case, \LILO\ displays the \verb"boot:" prompt and waits for
the name of a boot image (i.e. Linux kernel or other operating system).
Otherwise, it boots the default boot image\footnote{%
The default boot image is either the first boot image, the image specified
with the \craw{default} variable, or the image that
has been selected at the boot prompt.}
or -- if a
delay has been specified -- waits for one of the listed activities
until that amount of time has passed.

At the boot prompt, the name of the image to boot can be entered. Typing
errors can be corrected with \key{BackSpace}, \key{Delete},
\key{Ctrl U} and \key{Ctrl X}. A list of known images can be obtained by
pressing \key{?} (on the US keyboard) or \key{Tab}.

If \key{Enter} is pressed and no file name has been entered, the default
image is booted.


\subsection{Boot command-line options}

\LILO\ is also able to pass command-line options to the kernel.
Command-line options are words that follow the name of the boot
image and that are separated by spaces.

Example:
\begin{verbatim}
boot: linux single root=200
\end{verbatim}

This document only gives an overview of boot options. Please consult Paul
Gortmaker's BootPrompt-HOWTO for a more complete and more up to date list.
You can get it from
\path{ftp://sunsite.unc.edu/pub/Linux/docs/HOWTO/BootPrompt-HOWTO.gz}
or from one of the many mirror sites.


\subsubsection{Standard options}

The 1.1.90 kernel
recognizes the options \raw{debug}, \raw{no387}, \raw{no-hlt},
\raw{ramdisk=\meta{size}}, \raw{reserve=\meta{base},\meta{size},$\ldots$},
\raw{root=\meta{device}}, \raw{ro}, and \raw{rw},
and all current \name{init} programs also recognize the option \raw{single}.
The options \raw{lock} and \raw{vga} are processed by the boot loader itself.
Boot command-line options are always case-sensitive.

\raw{single} boots the system in single-user mode. This bypasses most
system initialization procedures and directly starts a root shell on the
console. Multi-user mode can typically be entered by exiting the single-user
shell or by rebooting.

\raw{root=\meta{device}} changes the root device. This overrides
settings that may have been made in the boot image and on the
\LILO\ command line. \meta{device} is either the hexadecimal device number
%%beginskip
\footnote{%
This is a list of device numbers of some frequently used devices: \\
\begin{tabular}{lllllll}
\qquad
& {\tt /dev/fd0} & 200\qquad & \tt /dev/hda1 & 301\qquad & /dev/sda1 & 801 \\
& {\tt /dev/fd1} & 201 & \tt /dev/hda2 & 302 & /dev/sda2 & 802 \\
& \multicolumn{2}{c}{$\cdots$} & \multicolumn{2}{c}{$\cdots$} &
\multicolumn{2}{c}{$\cdots$} \\
& & & \tt /dev/hdb1 & 341 & /dev/sdb1 & 811 \\
& & & \tt /dev/hdb2 & 342 & /dev/sdb2 & 812 \\
& \multicolumn{2}{c}{} & \multicolumn{2}{c}{$\cdots$} &
\multicolumn{2}{c}{$\cdots$} \\
\end{tabular}}
%%endskip
or the full path name of the device, e.g. \verb"/dev/hda3".%
\footnote{The device names are hard-coded in the kernel. Therefore, only
the ``standard'' names are supported and some less common devices may not
be recognized. In those cases, only numbers can be used.}

\raw{reserve=\meta{base},\meta{size},$\ldots$} reserves IO port regions.
This can be used to prevent device drivers from auto-probing addresses
where other devices are located, which get confused by the probing.

\raw{ro} instructs the kernel to mount the root file system read-only.
\raw{rw} mounts it read-write. If neither \raw{ro} nor \raw{rw} is
specified, the setting from the boot image is used.

\raw{no-hlt} avoids executing a \raw{HLT} instructions whenever the system
is idle. \raw{HLT} normally significantly reduces power consumption and
therefore also heat
dissipation of the CPU, but may not work properly with some clone CPUs.
\raw{no387} disables using the hardware FPU even if one is present.

\raw{debug} enables more verbose console logging.

Recent kernels also accept the options \raw{init=\meta{name}} and
\raw{noinitrd}. \raw{init} specifies the name of the \name{init} program
to execute. Therefore, if single mode cannot be entered because \name{init}
is mis-configured, one may still be able to reach a shell using
\verb"init=/bin/sh". \raw{noinitrd} disables automatic loading of the initial
RAM disk. Instead, its contents are available on \path{/dev/initrd}.

\raw{vga=\meta{mode}} alters the VGA mode set at startup. The values
\raw{normal}, \raw{extended}, \raw{ask} or a decimal number are
recognized. (See also page \pageref{vga}.)

\raw{kbd=\meta{code},$\ldots$} preloads a sequence of keystrokes in the
BIOS keyboard buffer. The keystrokes have to be entered as 16 bit hexadecimal
numbers, with the upper byte containing the scan code and the lower byte
containing the ASCII code. Note that most programs only use the ASCII
code, so the scan code can frequently be omitted. Scan code tables can be
found in many books on PC hardware. Note that scan codes depend on the
keyboard layout.

Finally, \raw{lock} stores the current command-line as the default
command-line, so that \LILO\ boots the same image with the same options
(including \raw{lock}) when invoked the next time.


\subsubsection{Device-specific options}

There is also a large number of options to specify certain characteristics
(e.g. IO and memory addresses)
of devices. The 1.1.90 kernel understands the following device options (if
the corresponding devices are included):
\raw{ether}, \raw{max\_scsi\_luns}, \raw{hd}, \raw{hda}, \raw{hdb},\raw{hdc},
\raw{hdd}, \raw{st}, \raw{bmouse},
\raw{st0x}, \raw{tmc8xx}, \raw{t128}, \raw{pas16}, \raw{ncr5380},
\raw{aha152x}, \raw{aha1542}, \raw{aha274x}, \raw{buslogic}, \raw{xd},
\raw{mcd}, \raw{aztcd}, \raw{sound}, \raw{sbpcd}, and \raw{cdu31a}.
The usage of these options is \raw{\meta{option}=\meta{number},$\ldots$}.
Please consult the corresponding FAQs and HOWTOs for details.


\subsubsection{Other options}

Options of the type \raw{\meta{variable}=\meta{value}} which are neither
standard options nor device-specific options,
cause the respective variables to be set in the environment passed to
\name{init}. The case of the variable name is preserved. (I.e. it's not
automatically converted to upper case.)

Note that environment variables passed to \name{init} are typically
available in system initialization scripts (e.g. \path{/etc/rc.local}),
but they're not visible from ordinary login sessions, because the \name{login}
program removes them from the user's environment.


\subsubsection{Repeating options}

The effect of repeating boot command-line options depends on the options.%
\footnote{Options are frequently repeated when a string defined with
\craw{append} or \craw{literal} is prepended to the parameters typed in by
the user. Also, \LILO\ implicitly prepends the options \raw{ramdisk},
\raw{ro}, \raw{root}, or \raw{rw} when \craw{ramdisk}, \craw{read-only},
\craw{read-write}, or \craw{root}, respectively, are set in the configuration
file.
(\raw{lock} and \raw{vga} are handled by a different internal mechanism.)}
There are three possible behaviours:

Options that only enable or disable a certain functionality can be repeated
any number of times. \raw{debug}, \raw{lock}, \raw{no-hlt}, and \raw{no387}
fall into this category.

Other options change a global setting whenever they appear, so only the
value or presence of the last option matters. The antagonists \raw{ro} and
\raw{rw} are such options. Also, \raw{ramdisk}, \raw{root}, and \raw{vga}
work this way. Example: \raw{ro rw} would mount the root file system
read-write.

Finally, when \raw{reserve} and many device-specific options are repeated,
each occurrence has its own meaning, e.g. \raw{hd=$\ldots$ hd=$\ldots$} would
configure two hard disks, and \raw{reserve=0x300,8 reserve=0x5f0,16} would
reserve the ranges 0x300 to 0x307 and 0x5f0 to 0x5ff (which is equivalent to
writing \raw{reserve=0x300,8,0x5f0,16}).


\subsubsection{Implicit options}

\LILO\ always passes the string \raw{BOOT\_IMAGE=\meta{name}} to the
kernel, where \meta{name} is the name by which the kernel is identified
(e.g. the label). This variable can be used in \path{/etc/rc} to select
a different behaviour, depending on the kernel.

When booting automatically, the word \raw{auto} is also passed on
the command line. This can be used by \name{init} to suppress interactive
prompts in the boot phase.


\subsection{Boot image selection}
\label{bootsel}

The details of selecting the boot image are somewhat complicated. The
following tables illustrate them. First, if neither \craw{prompt} is
set nor a shift key is being pressed:

$$
\begin{tabular}{cc|cl}
  Externally & Command  & Auto- & Booted image \\
  provided   & line in  & matic & \\
  cmd. line\footnotemark
  & map file\footnotemark
  & boot\footnotemark
  & \\
  \hline
  No  & No  & Yes & Default image \\
  Yes & --- & Yes & Specified by external \\
      &     &     & command line \\
  No  & Yes & Yes & Specified by command line \\
      &     &     & in map file \\
\end{tabular}
\addtocounter{footnote}{-2}
\footnotetext{Externally provided command lines could be used to
  add front-ends to \LILO. They would pass the respective command
  string to LILO, which would then interpret it like keyboard input. This
  feature is currently not used.}
\addtocounter{footnote}{1}
\footnotetext{This command line is set by invoking the map installer
    with the \raw{-R} option, by using the boot command-line option
    \raw{lock}, or if a fallback command line is set (with \craw{fallback}).}
\addtocounter{footnote}{1}
\footnotetext{I.e. the keyword \raw{auto} is added.}
$$

If \craw{prompt} is not set and a shift key is being pressed:
$$
\begin{tabular}{cccc|cl}
  Input   & Empty  & Extern. & Cmd.l.  & Auto- & Booted image \\
  timeout & cmd.l. & cmd.l.  & in map  & matic & \\
          &        &         & file    & boot  & \\
  \hline
  No  & No  & --- & --- & No  & Specified by the user \\
  No  & Yes & --- & --- & No  & Default image \\
  Yes & n/a & --- & --- & Yes & Default image \\
\end{tabular}
$$

Finally, if the configuration variable \craw{prompt} is set:
$$
\begin{tabular}{cccc|cl}
  Input   & Empty  & Extern. & Cmd.l.  & Auto- & Booted image \\
  timeout & cmd.l. & cmd.l.  & in map  & matic & \\
          &        &         & file    & boot  & \\
  \hline
  No  & No  & No  & No  & No  & Specified by the user \\
  No  & Yes & No  & No  & No  & Default image \\
  Yes & n/a & No  & No  & Yes & Default image \\
  n/a & n/a & Yes & --- & Yes & Specified by external \\
      &     &     &     &     & command line \\
  n/a & n/a & No  & Yes & Yes & Specified by command \\
      &     &     &     &     & line in map file \\
\end{tabular}
$$

Note that \LILO\ pauses for the amount of time specified in \craw{delay}
when at the end of a default command line. The automatic boot can then
be interrupted by pressing a shifting key.

The default image is the first image in the map file or the image specified
with the \craw{default} variable. However,
after an unsuccessful boot attempt, the respective image becomes the
default image.


\newpage
\section{Map installer}

The map installer program \path{/sbin/lilo} updates the boot sector
and creates the map file. If the map installer detects an error, it terminates
immediately and does not touch the boot sector and the map file.

Whenever the map installer updates a boot sector, the original boot
sector is copied
to \path{/boot/boot.\meta{number}}, where \meta{number} is the
hexadecimal device number. If such a file already exists, no backup
copy is made. Similarly, a file \raw{/boot/part.\meta{number}} is
created if \LILO\ modifies the partition table. (See page \pageref{fix-table}.)


\subsection{Command-line options}

The \LILO\ map installer can be invoked in the following ways:


%%beginskip
\def\seealso{See also sections \ref{corropt} and \ref{cfgglo}.}
%%def\\seealso=See also sections \\ref{corropt} and \\ref{cfgglo}.
%%endskip

\subsubsection{Show current installation}
\label{invshow}

The currently mapped files are listed. With \raw{-v}, also many parameters
are shown.

\begin{command}
  /sbin/lilo
    \unit{\[ -C \meta{config\_file} \]}
    \unit{-q}
    \unit{\[ -m \meta{map\_file} \]}
    \unit{\[ -v $\ldots$ \]}
\end{command}

\begin{description}
  \item[\raw{-C \meta{config\_file}}]~\\
    Specifies the configuration file that is used by the map
    installer (see section \ref{config}). If \raw{-C} is omitted,
    \path{/etc/lilo.conf} is used.
  \item[\raw{-m \meta{map\_file}}]~\\
    Specifies an alternate map file. \seealso
  \item[\raw{-q}]~ \\
    Lists the currently mapped files.
  \item[\raw{-v $\ldots$}]~\\
    Increase verbosity. \seealso
\end{description}


\subsubsection{Create or update map}
\label{creupd}

A new map is created for the images described in the configuration file
\path{/etc/lilo.conf} and they are registered in the boot sector.

\begin{command}
  /sbin/lilo
    \unit{\[ -C \meta{config\_file} \]}
    \unit{\[ -b \meta{boot\_device} \]}
    \unit{\[ -c \]}
    \unit{\[ -l \]}
    \unit{\[ -i \meta{boot\_sector} \]}
    \unit{\[ -f \meta{disk\_tab} \]}
    \unit{\[ -m \meta{map\_file} \]}
    \unit{\[ -d \meta{delay\/} \]}
    \unit{\[ -v $\ldots$ \]}
    \unit{\[ -t \]}
    \unit{\[ -s \meta{save\_file} \|}
    \unit{-S \meta{save\_file} \]}
    \unit{\[ -P fix \|}
    \unit{-P ignore \]}
    \unit{\[ -r \meta{root\_dir} \]}
\end{command}

\begin{description}
  \item[\raw{-b \meta{boot\_device}}]~\\
    Specifies the boot device. \seealso
  \item[\raw{-c}]~\\
    Enables map compaction. \seealso
  \item[\raw{-C \meta{config\_file}}]~\\
    Specifies an alternate configuration file. See also section \ref{invshow}.
  \item[\raw{-d \meta{delay}}]~\\
    Sets the delay before \LILO\ boots the default image. Note that the
    delay is specified in \emphasize{tenths} of a second. \seealso
  \item[\raw{-D \meta{name}}]~\\
    Specifies the default image. \seealso
  \item[\raw{-f \meta{disk\_tab}}]~\\
    Specifies a disk parameter table file. \seealso
  \item[\raw{-i \meta{boot\_sector}}]~\\
    Specifies an alternate boot file. \seealso
  \item[\raw{-l}]~\\
    Enables linear sector addresses. \seealso
  \item[\raw{-m \meta{map\_file}}]~\\
    Specifies an alternate map file. \seealso
  \item[\raw{-P \meta{mode}}]~\\
    Specifies how invalid partition table entries should be handled. \seealso
  \item[\raw{-r \meta{root\_directory}}]~ \\
    Chroots to the specified directory before doing anything else. This is
    useful when running the map installer while the normal root file system
    is mounted somewhere else, e.g. when recovering from an installation
    failure with a recovery disk. The \raw{-r} option is implied if
    the environment variable \raw{ROOT} is set.\footnote{E.g. if your root
      partition is mounted on \path{/mnt}, you can update the map by simply
      running \raw{ROOT=/mnt /mnt/sbin/lilo}}
    The current directory is changed to the new root directory, so using
    relative paths may not work.
  \item[\raw{-s \meta{save\_file}}]~\\
    Specifies an alternate boot sector save file. \seealso
  \item[\raw{-S \meta{save\_file}}]~\\
    Like \raw{-s}, but overwrites old save file.
  \item[\raw{-t}]~ \\
    Test only. This performs the entire installation procedure except
    replacing the map file, writing the modified boot sector and fixing
    partition tables. This
    can be used in conjunction with the \raw{-v} option to verify that
    \LILO\ will use sane values.
  \item[\raw{-v $\ldots$}]~\\
    Increase verbosity. \seealso
\end{description}


\subsubsection{Change default command line}
\label{invcmd}

Changes \LILO's default command line. See also section \ref{bootsel}.

\begin{command}
  /sbin/lilo
    \unit{\[ -C \meta{config\_file} \]}
    \unit{\[ -m \meta{map\_file} \]}
    \unit{-R \[ \meta{word} $\ldots$ \]}
\end{command}

\begin{description}
  \item[\raw{-C \meta{config\_file}}]~\\
    Specifies an alternate configuration file. See also section \ref{invshow}.
  \item[\raw{-m \meta{map\_file}}]~\\
    Specifies an alternate map file. \seealso
  \item[\raw{-R \meta{word $\ldots$}}]~ \\
    Stores the specified words in the map file. The boot loader uses those
    words as the default command line when booting the next time. That command
    line is removed from the map file by the boot loader by overwriting the
    sector immediately after reading it. The first word has
    to be the name of a boot image. If \raw{-R} is not followed by any words,
    the current default command line in the map file is erased.\footnote{%
      \raw{-R} is typically used in reboot scripts, e.g. \\
%%beginskip
\tt
%%endskip
\#!/bin/sh \\
cd / \\
if /sbin/lilo -R "\$*"; then \\
%%beginskip
\hbox to 0pt{}%
%%endskip
~~~~echo $\vert$ shutdown -r now \\
fi}
    An error message is issued and a non-zero exit code is returned if the
    command line is not accepted.
\end{description}


\subsubsection{Kernel name translation}

Determines the path of the kernel.

\begin{command}
  /sbin/lilo
    \unit{\[ -C \meta{config\_file} \]}
    \unit{-I \meta{name}}
    \unit{\[ \meta{options} \]}
\end{command}

\begin{description}
  \item[\raw{-C \meta{config\_file}}]~\\
    Specifies an alternate configuration file. See also section \ref{invshow}.
  \item[\raw{-I \meta{name} {$[$} \meta{options} {$]$}}]~ \\
    Translates the specified label name to the path of the corresponding kernel
    image and prints that path on standard output. This can be used to
    synchronize
    files that depend on the kernel (e.g. the \name{ps} database). The image
    name can be obtained from the environment variable \raw{BOOT\_IMAGE}. An
    error message is issued and a non-zero exit code is returned if no
    matching label name can be found. The existence of the image file is
    verified if the option character \raw{v} is added.
\end{description}


\subsubsection{De-installation}

Restores the boot sector that was used before the installation of \LILO.
Note that this option only works properly if \LILO's directories (e.g.
\path{/boot}) have not been touched since the first installation. See also
section \ref{deinst}.

\begin{command}
  /sbin/lilo
    \unit{\[ -C \meta{config\_file} \]}
    \unit{\[ -s \meta{save\_file} \]}
    \unit{-u \| -U}
    \unit{\[ \meta{boot\_device} \]}
\end{command}

\begin{description}
  \item[\raw{-C \meta{config\_file}}]~\\
    Specifies an alternate configuration file. See also section \ref{invshow}.
  \item[\raw{-s \meta{save\_file}}]~\\
    Specifies an alternate boot sector save file. \seealso
  \item[\raw{-u {$[$} \meta{device\_name} {$]$}}]~ \\
    Restores the backup copy of the specified boot sector. If no device is
    specified, the value of the \verb"boot" variable is used. If this one
    is also unavailable, \LILO\ uses the current root device. The name of
    the backup copy is derived from the device name. The \verb"-s" option
    or the \verb"backup" variable can be used to override this.
    \LILO\ validates the backup copy by checking a time stamp.
  \item[\raw{-U {$[$} \meta{device\_name} {$]$}}]~ \\
    Like \verb"-u", but does not check the time stamp.
\end{description}


\subsubsection{Print version number}

\begin{command}
  /sbin/lilo -V
\end{command}

\begin{description}
  \item[\raw{-V}]~ \\
    Print the version number and exit.
\end{description}


\subsubsection{Options corresponding to configuration variables}
\label{corropt}

There are also many command-line options that correspond to configuration
variables. See section \ref{cfgglo} for a description.

$$
\begin{tabular}{l|l}
  Command-line option		& Configuration variable \\
  \hline
  \raw{-b \meta{boot\_device}}	& \raw{boot=\meta{boot\_device}} \\
  \raw{-c}			& \raw{compact} \\
  \raw{-d \meta{tsecs}}		& \raw{delay=\meta{tsecs}} \\
  \raw{-D \meta{name}}		& \raw{default=\meta{name}} \\
  \raw{-i \meta{boot\_sector}}	& \raw{install=\meta{boot\_sector}} \\
  \raw{-l}			& \raw{linear} \\
  \raw{-m \meta{map\_file}}	& \raw{map=\meta{map\_file}} \\
  \raw{-P fix}			& \raw{fix-table} \\
  \raw{-P ignore}		& \raw{ignore-table} \\
  \raw{-s \meta{backup\_file}}	& \raw{backup=\meta{backup\_file}} \\
  \raw{-S \meta{backup\_file}}	& \raw{force-backup=\meta{backup\_file}} \\
  \raw{-v $\ldots$}		& \raw{verbose=\meta{level}} \\
\end{tabular}
$$


\subsection{Configuration}
\label{config}

The configuration information is stored in the file \path{/etc/lilo.conf}
and consists of variable assignments.


\subsubsection{Syntax}

The following syntax rules apply:

\begin{itemize}
  \item flag variables consist of a single word and are followed by
    whitespace or the end of the file.
  \item string variables consist of the variable name, optional whitespace,
    an equal sign, optional whitespace, the value and required whitespace,
    or the end of the file.
  \item a non-empty sequence of blanks, tabs, newlines and comments counts
    as whitespace.
  \item variable names are case-insensitive. Values are usually case-sensitive,
    but there are a few exceptions. (See below.)
  \item tabs and newlines are special characters and may not be part of a
    variable name or a value. The use of other control characters and
    non-ASCII characters is discouraged.
  \item blanks and equal signs may only be part of a variable name or a
    value if they are escaped by a backslash or if the value is embedded in
    double quotes. An equal sign may not be the only character in a name or
    value.
  \item an escaped tab is converted to an escaped blank. An escaped newline
    is removed from the input stream. An escaped backslash (i.e. two
    backslashes) is converted to a backslash. Inside quoted strings, only
    double quotes, backslashes, dollar signs, and newlines can be escaped.
  \item quoted strings can be continued over several lines by ending each
    incomplete line with a backslash. A single space is inserted in the
    string for the line end and all spaces or tabs that follow immediately
    on the next line.
  \item environment variables can be used by specifying them in the form
    \verb"$"\meta{name} or \verb"${"\meta{name}\verb"}". Dollar signs can
    be escaped.
  \item comments begin with a number sign and end with the next newline.
    All characters (including backslashes) until the newline are ignored.
\end{itemize}

Example:
\begin{verbatim}
boot = $FD
install = $MNT/boot.b
map = $MNT/map
compact
read-only
append = "nfsroot=/home/linux-install/root \
  nfsaddrs=128.178.156.28:128.178.156.24::255.255.255.0:lrcinst"
image = $MNT/zImage
\end{verbatim}


\subsubsection{Global options}
\label{cfgglo}

\path{/etc/lilo.conf} begins with a possibly empty global options section.
Many global options can also be set from the command line, but storing
permanent options in the configuration file is more convenient.

The following global options are recognized:

\begin{description}
  \item[\craw{backup=}\meta{backup\_file}] Copy the original boot sector to
    \meta{backup\_file} (which may also be a device, e.g. \path{/dev/null})
    instead of \path{/boot/boot.\meta{number}}
  \item[\craw{boot=}\meta{boot\_device}] Sets the name of the device (e.g.
    a hard disk partition) that contains
    the boot sector. If \craw{boot} is omitted, the boot sector is read from
    (and possibly written to) the device that is currently mounted as root.
  \item[\craw{compact}] Tries to merge read requests for adjacent sectors into
    a single read request. This drastically reduces load time and keeps the
    map smaller. Using \craw{compact} is especially recommended when booting
    from a floppy disk.
    \craw{compact} may conflict with \craw{linear}, see section \ref{othpro}.
  \item[\craw{default=}\meta{name}] Uses the specified image as the default
    boot image. If \craw{default} is omitted, the image appearing first in the
    configuration file is used.
  \item[\craw{delay=}\meta{tsecs}] Specifies the number of tenths of a second
    \LILO\ should wait before booting the first image. This is useful on
    systems that immediately boot from the hard disk after enabling the
    keyboard. \LILO\ doesn't wait if \craw{delay} is omitted or if \craw{delay}
    is set to zero.
  \item[\craw{disk=}\meta{device\_name}] Defines non-standard parameters
    for the specified disk. See section \ref{diskgeo} for details.
  \item[\craw{fallback=}\meta{command\_line}] Specifies a string that is stored
    as the default command line if the current image is booted. This is useful
    when experimenting with kernels which may crash before allowing
    interaction with the system. If using the \craw{fallback} option, the
    next reboot (e.g. triggered by a manual reset or by a watchdog timer)
    will load a different (supposedly stable) kernel. The fallback option is
    cleared by removing or changing the default command line with the \raw{-R}
    option, see \ref{invcmd}.
  \item[\craw{fix-table}]\label{fix-table}allows \LILO\ to adjust 3D addresses
    in partition
    tables. Each partition entry contains a 3D (sector/head/cylinder) and a
    linear address of the first and the last sector of the partition. If a
    partition is not track-aligned and if certain other operating systems
    (e.g. PC/MS-DOS or OS/2) are using the same disk, they may change the 3D
    address. \LILO\ can store its boot sector only on partitions where both
    address types correspond. \LILO\ re-adjusts incorrect 3D start addresses
    if \craw{fix-table} is set.

    \emphasize{WARNING:} This does not guarantee that other operating systems
    may not attempt to reset the address later. It is also possible that this
    change has other, unexpected side-effects. The correct fix is to
    re-partition the drive with a program that does align partitions to tracks.
    Also, with some disks (e.g. some large EIDE disks with address
    translation enabled), under some circumstances, it may even be unavoidable
    to have conflicting partition table entries.
  \item[\craw{force-backup=}\meta{backup\_file}] Like \craw{backup}, but
    overwrite an old backup copy if it exists.
    \craw{backup=}\meta{backup\_file}
    is ignored if \craw{force-backup} appears in the same configuration file.
  \item[\craw{ignore-table}] tells LILO to ignore corrupt partition tables and
    to put the boot sector even on partitions that appear to be unsuitable for
    that.
  \item[\craw{install=}\meta{boot\_sector}] Install the specified file as the
    new boot sector. If \craw{install} is omitted, \path{/boot/boot.b} is used
    as the default.
  \item[\craw{linear}] Generate linear sector addresses instead of
    sector/head/cylinder addresses. Linear addresses are translated at run
    time and do not depend on disk geometry. Note that boot disks
    may not be portable if \craw{linear} is used,
    because the BIOS service to
    determine the disk geometry does not work reliably for floppy disks.
    When using \craw{linear} with large disks, \path{/sbin/lilo} may
    generate references to inaccessible disk areas (see section \ref{broken}),
    because 3D sector addresses are not known before boot time.
    \craw{linear} may conflict with \craw{compact}, see section \ref{othpro}.
  \item[\craw{lock}] Enables automatic recording of boot command lines as the
    defaults for the following boots. This way, \LILO\ ``locks'' on a
    choice until it is manually overridden.
  \item[\craw{map=}\meta{map\_file}] Specifies the location of the map file.
    If \craw{map} is omitted, a file \path{/boot/map} is used.
  \item[\craw{message=}\meta{message\_file}] specifies a file containing a
    message that is displayed before the boot prompt. No message is displayed
    while waiting for a shifting key after printing ``LILO ''. In the message,
    the \raw{FF}
    character (\hbox{\key{Ctrl L}}) clears the local screen. The size of the
    message file is limited to 65535 bytes. The map file has to be rebuilt
    if the message file is changed or moved.
  \item[\craw{nowarn}] Disables warnings about possible future dangers.
  \item[\craw{optional}] makes all images optional. (See below.)
  \item[\craw{password=}\meta{password}] sets a password for all images.
    (See below.)
  \item[\craw{prompt}] forces entering the boot prompt without expecting any
    prior key-presses. Unattended reboots are impossible if \craw{prompt} is
    set and \craw{timeout} isn't.
  \item[\craw{restricted}] relaxes the password protection. (See below.)
  \item[\craw{serial=}\meta{parameters}] enables control from a serial line.
    The specified serial port is initialized and \LILO\ is accepting input from
    it and from the PC's keyboard. Sending a break on the serial line
    corresponds to pressing a shift key on the console in order to get
    \LILO's attention. All boot images should be password-protected if the
    serial access is less secure than access to the console, e.g. if the
    line is connected to a modem. The parameter string has the following
    syntax: \\
    \raw{\meta{port},\meta{bps}\,\meta{parity}\,\meta{bits}} \\
    The components \meta{bps}, \meta{parity} and \meta{bits} can be
    omitted. If a component is omitted, all following components have to
    be omitted too. Additionally, the comma has to be omitted if only the
    port number is specified.
    \begin{description}
      \item[\meta{port}] the number of the serial port, zero-based. 0
	corresponds to \path{COM1} alias \path{/dev/ttyS0}, etc. All four
	ports can be used (if present).
      \item[\meta{bps}] the baud rate of the serial port. The following
	baud rates are supported: 110, 150, 300, 600, 1200, 2400, 4800 and
	9600 bps. Default is 2400 bps.
      \item[\meta{parity}] the parity used on the serial line. \LILO\ ignores
	input parity and strips the 8th bit. The following (upper or lower
	case) characters are used to describe the parity: \raw{n} for no
	parity, \raw{e} for even parity and \raw{o} for odd parity.
      \item[\meta{bits}] the number of bits in a character. Only 7 and 8
	bits are supported. Default is 8 if parity is ``none'', 7 if parity
	is ``even'' or ``odd''.
    \end{description}
    If \craw{serial} is set, the value of \craw{delay} is automatically raised
    to 20.

    Example: \verb"serial=0,2400n8" initializes \path{COM1} with the
    default parameters.
  \item[\craw{timeout=}\meta{tsecs}] sets a timeout (in tenths of a second) for
    keyboard input. If no key is pressed for the specified time, the first
    image is automatically booted. Similarly, password input is aborted if
    the user is idle for too long. The default timeout is infinite.
  \item[\craw{verbose=}\meta{level}] Turns on lots of progress reporting.
    Higher numbers give more verbose output. If \raw{-v} is additionally
    specified on the command line, \meta{level} is increased accordingly.
    The following verbosity levels exist:
    \begin{description}
      \item[$<$0] only warnings and errors are shown
      \item[0] prints one line for each added or skipped image
      \item[1] mentions names of important files and devices and why they
	are accessed. Also displays informational messages for exceptional
	but harmless conditions and prints the version number.
      \item[2] displays statistics and processing of temporary files and
        devices
      \item[3] displays disk geometry information
      \item[4] lists sector mappings as they are written into the map file
	(i.e. after compaction, in a format suitable to pass it to the BIOS)
      \item[5] lists the mapping of each sector (i.e. before compaction, raw)
    \end{description}
    When using the \raw{-q} option, the levels have a slightly different
    meaning:
    \begin{description}
      \item[0] displays only image names
      \item[1] also displays all global and per-image settings
      \item[2] displays the address of the first map sector
    \end{description}
\end{description}

Additionally, the kernel configuration parameters \craw{append}, \craw{ramdisk},
\craw{read-only}, \craw{read-write}, \craw{root} and \craw{vga} can be set in
the global options section. They are used as defaults if they aren't specified
in the configuration sections of the respective kernel images. See below for a
description.

The plethora of options may be intimidating at first, but in ``normal''
configurations, hardly any options but \craw{boot}, \craw{compact},
\craw{delay}, \craw{root}, and \craw{vga} are used.


\subsubsection{General per-image options}
\label{cfggen}

\LILO\ uses the main file name (without its path) of each image specification
to identify that image. A different name can be used by setting the variable
\craw{label} (see the example above). A second name for the same entry
can be used by specifying the \craw{alias=}\meta{name} option.

Images are protected by a password if the variable \craw{password} is set.
If the variable \craw{restricted} is set in addition to \craw{password},
a password is only required to boot the respective image if parameters
are specified on the command line (e.g. \verb"single").
\craw{password} and \craw{restricted} can also be set in the options section
to be the default password and password protection mode for all images.
Because the configuration file contains unencrypted passwords when using
\craw{password}, it should only be readable for the super-user.

If an image section contains the variable \craw{optional} (or if that variable
is set in the options section), the respective image is omitted if its main
file is not available at map creation time. This is useful to specify test
kernels that are not always present.


\subsubsection{Per-image options for kernels}
\label{cfgkern}

Each (kernel or non-kernel)
image description begins with a special variable (see section \ref{bootimg})
which is followed by optional
variables. The following variables can be used for all image descriptions
that describe a Linux kernel:

\begin{description}
  \item[\craw{append=}\meta{string}] appends the options specified in
    {\meta{string}} to the parameter line passed to the kernel. This is
    typically used to specify parameters of hardware that can't be entirely
    auto-detected, e.g.\\
    \verb'append = "hd=64,32,202"'
  \item[\craw{initrd=}\meta{name}] specifies the file that will be loaded at
    boot time as the initial RAM disk. 
  \item[\craw{literal=}\meta{string}] like \craw{append}, but removes all other
    options (e.g. setting of the root device). Because vital options can be
    removed unintentionally with \craw{literal}, this option cannot be set in
    the global options section.
  \item[\craw{ramdisk=}\meta{size}] specifies the size of the optional RAM disk.
    A value of zero indicates that no RAM disk should be created. If this
    variable is omitted, the RAM disk size configured into the boot
    image is used.
  \item[\craw{read-only}] specifies that the root file system should be
    mounted read-only. Typically, the system startup procedure re-mounts
    the root file system read-write later (e.g. after fsck'ing it).
  \item[\craw{read-write}] specifies that the root file system should be
    mounted read-write.
  \item[\craw{root=}\meta{root\_device}] specifies the device that should be
    mounted as root. If the special name \craw{current} is used, the root
    device is set to the device on which the root file system is currently
    mounted. If the root has been changed with \raw{-r}, the respective
    device is used. If the variable \craw{root} is omitted, the root device
    setting contained in the kernel image is used. It can be changed
    with the \name{rdev} program.
  \item[\craw{vga=}\meta{mode}]\label{vga} specifies the VGA text mode that
    should be selected when booting. The following values are recognized
    (case is ignored):
    \begin{description}
      \item[\craw{normal}] select normal 80x25 text mode.
      \item[\craw{extended}] select 80x50 text mode. The word \craw{extended}
	can be abbreviated to \craw{ext}.
      \item[\craw{ask}] stop and ask for user input (at boot time).
      \item[\meta{number}] use the corresponding text mode. A list of available
	modes can be obtained by booting with \raw{vga=ask} and pressing
	\key{Enter}.
    \end{description}
    If this variable is omitted, the VGA mode setting contained in the kernel
    image is used. \name{rdev} supports manipulation of the VGA text mode
    setting in the kernel image.
\end{description}

If one of \craw{ramdisk}, \craw{read-only}, \craw{read-write}, \craw{root},
or \craw{vga} is omitted in the configuration file and the
corresponding value in the kernel image is changed, \LILO\ or the kernel
will use the new value.

It is perfectly valid to use different settings for the same
image, because \LILO\ stores them in the image descriptors and not in the
images themselves.

Example:
\begin{verbatim}
image = /vmlinuz
  label = lin-hd
  root = /dev/hda2
image = /vmlinuz
  label = lin-fd
  root = /dev/fd0
\end{verbatim}


\subsection{Boot image types}
\label{bootimg}

\LILO\ can boot the following types of images:
\begin{itemize}
  \item kernel images from a file.
  \item kernel images from a block device. (E.g. a floppy disk.)
  \item the boot sector of some other operating system.
\end{itemize}

The image type is determined by the name of the initial
variable of the configuration section.

The image files can reside on any media that is accessible at boot time.
There's no need to put them on the root device, although this certainly
doesn't hurt.

In the configuration sections of all boot images, the following variables
are recognized: \craw{alias}, \craw{label}, \craw{lock}, \craw{optional},
\craw{password} and \craw{restricted}.

In the configuration sections of all kernels (i.e. every boot image type
except ``other operating system''), the following variables are recognized:
\craw{append}, \craw{literal}, \craw{ramdisk}, \craw{read-only},
\craw{read-write}, \craw{root} and \craw{vga}.


\subsubsection{Booting kernel images from a file}

The image is specified as follows: \craw{image=}\meta{name}

Example:
\begin{verbatim}
image = /linux
\end{verbatim}


\subsubsection{Booting kernel images from a device}

The range of sectors
that should be mapped, has to be specified. Either a range
(\raw{\meta{start}-\meta{end}}) or a start and a distance
(\raw{\meta{start}+\meta{number}}) have to be specified. \meta{start}
and \meta{end} are zero-based. If only
the start is specified, only that sector is mapped.

The image is specified as follows: \craw{image=}\meta{device\_name}\quad
Additionally, the \craw{range} variable must be set.

Example:
\begin{verbatim}
image = /dev/fd0
  range = 1+512
\end{verbatim}


\subsubsection{Booting a foreign operating system}
\label{bootfor}

\LILO\ can even boot other operating systems, i.e. MS-DOS.
To boot an other operating
system, the name of a loader program, the device or file that contains the boot
sector and the device that contains the partition table have to be
specified.

The boot sector is merged with the partition table and stored in the map file.

Currently, the loaders \path{chain.b}, \path{os2\_d.b}, \path{any\_b.b} and
\path{any\_d.b} exist.
\path{chain.b} simply starts the specified boot sector.\footnote{The boot
sector is loaded by \LILO's secondary boot loader before control is passed
to the code of \path{chain.b}.} \path{os2\_d.b} can boot OS/2 from the second
hard disk. \path{any\_b.b} and \path{any\_d.b} install resident drivers that
swap the first and the second floppy or hard disk drive. They can boot any
operating system from the second hard disk, as long as that operating system
uses \emphasize{only} the BIOS
to access that hard disk.\footnote{So you should be very suspicious if the
operating system requires any specific configuration or even drivers to use
the disk it is booted from. Since there is a general trend to use optimized
drivers to fully exploit the hardware capabilities (e.g. non-blocking disk
access), booting systems from the second disk may become increasingly
difficult.} This is known to work for PC/MS-DOS.

The image is specified as follows: \craw{other=}\meta{device\_name} or
\craw{other=}\meta{file\_name}

The following additional variables are recognized: \craw{loader}, \craw{table}
and \craw{unsafe}.

\begin{description}
  \item[\craw{loader=}\meta{chain\_loader}] specifies the chain loader that
    should be used. If it is omitted, \path{/boot/chain.b} is used. The
    chain loader must be specified if booting from a device other than the
    first hard or floppy disk.
  \item[\craw{table=}\meta{device}] specifies the device that contains the
    partition
    table. \LILO\ does not pass partition information to the booted operating
    system if this variable is omitted. (Some operating systems have other
    means to determine from which partition they have been booted. E.g. MS-DOS
    usually stores the geometry of the boot disk or partition in its boot
    sector.) Note that \path{/sbin/lilo} must be re-run if a partition table
    mapped referenced with \craw{table} is modified.
  \item[\craw{rewrite-table}] rewrite the partition table with the active
    flag set to the primary partition the operating system is loaded from.
    This is done immediately before executing that boot sector. This option
    is intended for booting systems which determine their boot partition by
    examining the active flag in the partition table.\footnote{Note that
    the use of \craw{rewrite-table} is only necessary if more than one such
    system is used. For a single such system, installing \LILO\ on the MBR
    and activating that system's partition is sufficient.} In order to use
    \craw{rewrite-table}, the following requirements have to be fulfilled:
    \begin{itemize}
      \item the build-time option \raw{REWRITE\_TABLE} must be set
      \item \LILO's boot sector must be installed on the MBR (otherwise, it
	would make itself unbootable)
      \item the boot sector of the other system must be loaded from a
	primary partition (i.e. the partition that system is on)
      \item the \craw{table} option must be present and specify a device
        that could actually contain a partition table, e.g. \path{/dev/hda}
	or \path{/dev/hdb}\footnote{This is only a paranoia check -- the
	partition table specified with \craw{table} is not used (but it is
	loaded) and the table of the first disk (see also the following
	remark) is loaded instead.}
    \end{itemize}
    Note that, if using any of the drive swapping chain loaders (e.g.
    \path{/boot/any\_d.b}), the partition table is loaded, modified, and
    written after the swapping code is installed. It is therefore also
    possible to boot systems requiring \craw{rewrite-table} from the second
    disk.
  \item[\craw{unsafe}] do not access the boot sector at map creation time. This
    disables some sanity checks, including a partition table check. If the boot
    sector is on a fixed-format floppy disk device, using \craw{unsafe} avoids
    the need to put a readable disk into the drive when running the map
    installer. \craw{unsafe} and \craw{table} are mutually incompatible.
\end{description}

Examples:
\begin{verbatim}
other = /dev/hda2
  label = os2
  table = /dev/hda
\end{verbatim}

\begin{verbatim}
other = /dev/hdb1
  label = dos
  loader = /boot/any_d.b
  table = /dev/hdb
\end{verbatim}


\subsection{Disk geometry}
\label{diskgeo}

For floppies and most hard disks, \LILO\ can obtain the
disk geometry information from the kernel. Unfortunately, there are some
exotic disks or adapters which may either not supply this information or
which may even return incorrect information.

If no geometry information is available, \LILO\ reports either the error \\
\raw{geo\_query\_dev HDIO\_GETGEO (dev 0x\meta{number})} \\
or \\
\raw{Device 0x\meta{number}: Got bad geometry \meta{sec}/\meta{hd}/%
\meta{cyl}}

If incorrect information is returned, booting may fail in several ways,
typically with a partial ``LILO'' banner message. In this document, that
is called a ``geometry mismatch''.

The next step should be to attempt setting the \craw{linear} configuration
variable or the \raw{-l} command-line option. If
this doesn't help, the entire disk geometry has to be specified explicitly.
Note that \craw{linear} doesn't always work with floppy disks.


\subsubsection{Obtaining the geometry}

The disk geometry parameters can be obtained by booting MS-DOS and
running the program \path{DPARAM.COM} with the hexadecimal BIOS code of
the drive as its argument, e.g. \verb"dparam 0x80" for the first hard
disk. It displays the number of sectors per
track, the number of heads per cylinder and the number of cylinders.
All three numbers are one-based.

Alternatively, the geometry may also be determined by reading the information
presented by the ``setup'' section of the ROM-BIOS or by using certain disk
utilities under operating systems accessing the disk through the BIOS.


\subsubsection{Specifying the geometry}
\label{diskspec}

Disk geometry parameters are specified in the options section of the
configuration file. Each disk parameter sub-section begins with
\craw{disk=}\meta{disk\_device}, similar to the way how boot images are
specified.
It is suggested to group disk parameter sections together, preferably
at the beginning or the end of the options section.

For each disk, the following variables can be specified:
\begin{description}
  \item[\craw{bios=}\meta{bios\_device\_code}] Is the number the BIOS uses
    to refer to that device. Normally, it's \raw{0x80} for the first hard
    disk and \raw{0x81} for the second hard disk. Note that hexadecimal
    numbers have to begin with ``0x''. If \craw{bios} is omitted,
    \LILO\ tries to ``guess'' that number.
  \item[\craw{sectors=}\meta{sectors}] and
  \item[\craw{heads=}\meta{heads}] specify the number of sectors per
    track and the number of heads, i.e. the number of tracks per
    cylinder. Both parameters have to be either specified together or
    they have to be entirely omitted. If omitted, \LILO\ tries to obtain
    that geometry information from the kernel.
  \item[\craw{cylinders=}\meta{cylinders}] Specifies the number of cylinders.
    This value is only used for sanity checks. If \craw{cylinders} is omitted,
    \LILO\ uses the information obtained from the kernel if geometry
    information had to be requested in order to determine some other
    parameter. Otherwise,\footnote{I.e. if the BIOS device code, the number
    of sectors, the number of heads and the partition start are specified.
    Note that the number of cylinders may appear to vary if \craw{cylinders}
    is absent and only some of the partition starts are specified.} it just
    assumes the number of cylinders to be 1024, which is the cylinder limit
    imposed by the BIOS.
  \item[\craw{inaccessible}] Marks the device as inaccessible (for the BIOS).
    This is useful if some disks on the system can't be read by the BIOS,
    although \LILO\ ``thinks'' they can. If one accidentally tries to use
    files located on such disks for booting, the map installer won't notice
    and the system becomes unbootable. The most likely use of
    \craw{inaccessible} is to prevent repetition after experiencing such a
    situation once. No other variables may be specified if a device is
    configured as \craw{inaccessible}.
\end{description}

Additionally, partition sub-sub-sections can be added with
\craw{partition=}\meta{partition\_device}. Each partition section can
contain only one variable:
\begin{description}
  \item[\craw{start=}\meta{partition\_offset}] Specifies the zero-based
    number of the start sector of that partition. The whole disk always
    has a partition offset of zero. The partition offset is only necessary
    when using devices for which the kernel does not provide that
    information, e.g. CD-ROMs.
\end{description}

Example:
\begin{verbatim}
disk = /dev/sda
  bios = 0x80
  sectors = 32
  heads = 64
  cylinders = 632
  partition = /dev/sda1
    start = 2048
  partition = /dev/sda2
    start = 204800
  partition = /dev/sda3
    start = 500000
  partition = /dev/sda4
    start = 900000
\end{verbatim}

Because many SCSI controllers don't support more than 1 GB when using
the BIOS interface, \LILO\ can't access files that are located beyond the
1 GB limit of large SCSI disks on such controllers and reports errors in
these cases.


\newpage
\section{Installation and updates}


\subsection{Installation}

This section describes the installation of \LILO. See section \ref{deinst}
for how to uninstall \LILO.


\subsubsection{Compatibility}

The kernel header files have to be in
\path{/usr/include/linux} and the kernel usually has to be configured by
running \raw{make config} before \LILO\ can be compiled.

\path{/bin/sh} has to be a real Bourne shell. \name{bash} is sufficiently
compatible, but some \name{ksh} clones may cause problems.

Using command-line options that are handled by \name{init} usually works with
any current version of \name{init}. All currently available \name{init}
packages typically support new features of \LILO\ in the respective next
release.

A file named \path{INCOMPAT} is included in the distribution. It describes
incompatibilities to older versions of \LILO\ and may also contain further
compatibility notes.


\subsubsection{Quick installation}
\label{quickinst}

If you want to install \LILO\ on your hard disk and if you don't want
to use all its features, you can use the quick installation script.
Read \path{QuickInst} for details.

\name{QuickInst} can only be used for first-time installations or to
entirely replace an existing installation, \emphasize{not}
to update or modify an existing installation of \LILO. Be sure you've
extracted \LILO\ into a directory that doesn't contain any files of
other \LILO\ installations.


\subsubsection{Files}

Some of the files contained in \path{lilo.17.tar.gz}:

\begin{description}
  \item[\path{lilo/README}]~\\
    This documentation in plain ASCII format. Some sections containing
    complex tables are only included in the \LaTeX\ version in
    \path{doc/user.tex}
  \item[\path{lilo/INCOMPAT}]~\\
    List of incompatibilities to previous versions of \LILO.
  \item[\path{lilo/CHANGES}]~\\
    Change history.
  \item[\path{lilo/QuickInst}]~\\
    Quick installation script.
  \item[\path{lilo/Makefile}]~\\
    Makefile to generate everything else.
  \item[\path{lilo/*.c}, \path{lilo/*.h}]~\\
    LILO map installer C source.
  \item[\path{lilo/*.S}]~\\
    LILO boot loader assembler source.
  \item[\path{lilo/activate.c}]~\\
    C source of a simple boot partition setter.
  \item[\path{lilo/dparam.s}]~\\
    Assembler source of a disk parameter dumper.
  \item[\path{lilo/mkdist}]~\\
    Shell script used to create the current \LILO\ distribution.
  \item[\path{lilo/doc/README}]~\\
    Description of how the documentation is generated.
  \item[\path{lilo/doc/Makefile}]~\\
    Makefile used to convert the \LaTeX\ source into either DVI output or
    the plain ASCII README file.
  \item[\path{lilo/doc/user.tex}]~\\
    \LaTeX\ source of \LILO's user's guide (this document).
  \item[\path{lilo/doc/tech.tex}]~\\
    \LaTeX\ source of \LILO's technical overview.
  \item[\path{lilo/doc/*.fig}]~\\
    Various \name{xfig} pictures used in the technical overview.
  \item[\path{lilo/doc/fullpage.sty}]~\\
    Style file to save a few square miles of forest.
  \item[\path{lilo/doc/rlatex}]~\\
    Shell script that invokes \LaTeX\ repeatedly until all references have
    settled.
  \item[\path{lilo/doc/t2a.pl}]~\\
    \name{Perl} script to convert the \LaTeX\ source of the user's guide to
    plain ASCII.
\end{description}

Files created after \raw{make} in \path{lilo/} (among others):

\begin{description}
  \item[\path{lilo/any\_b.b}]~\\
    Chain loader that swaps the first two floppy drives (i.e. ``A:'' and
    ``B:''). \raw{make install} puts this file into \path{/boot}
  \item[\path{lilo/any\_d.b}]~\\
    Chain loader that swaps the first two hard disk devices (i.e. ``C:'' and
    ``D:''). \raw{make install} puts this file into \path{/boot}
  \item[\path{lilo/boot.b}]~\\
    Combined boot sector. \raw{make install} puts this file into \path{/boot}
  \item[\path{lilo/chain.b}]~\\
    Generic chain loader. \raw{make install} puts this file into \path{/boot}
  \item[\path{lilo/os2\_d.b}]~\\
    Chain loader to load \name{OS/2} from the second hard disk.
    \raw{make install} puts this file into \path{/boot}
  \item[\path{lilo/lilo}]~\\
    \LILO\ (map) installer. \raw{make install} puts this file into \path{/sbin}
  \item[\path{lilo/activate}]~\\
    Simple boot partition setter.
  \item[\path{lilo/dparam.com}]~\\
    MS-DOS executable of the disk parameter dumper.
\end{description}


\subsubsection{Normal first-time installation}
\label{install}

First, you have to install the \LILO\ files:

\begin{itemize}
  \item extract all files from \path{lilo.\meta{version}.tar.gz} in a new
    directory.\footnote{E.g. \path{/usr/src/lilo}}
  \item configure the \path{Makefile} (see section \ref{cfgmf})
  \item run\quad\raw{make}\quad to compile and assemble all parts.
  \item run\quad\raw{make install}\quad to copy all \LILO\ files to
     the directories where they're installed. \path{/sbin} should now
     contain the file \path{lilo}, 
     and \path{/boot} should contain the following files: \path{any\_b.b},
     \path{any\_d.b}, \path{boot.b}, \path{chain.b} and \path{os2\_d.b}.
\end{itemize}

If you want to use \LILO\ on a non-standard disk, you might have to determine
the parameters of your disk(s) and specify them in the configuration file.
See section \ref{diskgeo} for details.
If you're using such a non-standard system, the next step is to test
\LILO\ with the boot sector on a floppy disk:

\begin{itemize}
  \item insert a blank (but low-level formatted) floppy disk into
    \path{/dev/fd0}.
  \item run \raw{echo image=\meta{kernel\_image}}\verb" |"\\
    \raw{/sbin/lilo -C - -b /dev/fd0 -v -v -v} \\
    If you've already installed \LILO\ on your system, you might not want
    to overwrite your old map file.
    Use the \raw{-m} option to specify an alternate map file name.
  \item reboot. \LILO\ should now load its boot loaders from the floppy disk
    and then continue loading the kernel from the hard disk.
\end{itemize}

Now, you have to decide, which boot concept you want to use. Let's assume
you have a Linux partition on \path{/dev/hda2} and you want to install your
\LILO\ boot sector there. The DOS-MBR loads the \LILO\ boot sector.

\begin{itemize}
  \item get a working boot disk, e.g. an install or recovery disk. Verify that
    you can boot
    with this setup and that you can mount your Linux partition(s) with it.
  \item if the boot sector you want to overwrite with \LILO\ is of any value
    (e.g. it's the MBR or if it contains a boot loader you might want to use
    if you encounter problems with \LILO), you should mount your boot disk
    and make a backup copy of your boot sector to a file on that floppy,
    e.g. \verb"dd if=/dev/hda of=/fd/boot_sector bs=512 count=1"
  \item create the configuration file \path{/etc/lilo.conf}, e.g. \\
\meta{global settings} \\
\verb"  "\meta{image specifications} \\
\verb"  "$\ldots$ \\
    Be sure to use absolute paths for all files. Relative paths may cause
    unexpected behaviour when using the \raw{-r} option.
  \item now, you can check what \LILO\ would do if you were about to install
    it on your hard disk: \\
\verb"/sbin/lilo -v -v -v -t"
  \item if you need some additional boot utility (i.e. \name{BOOTACTV}), you
    should install that now
  \item run \path{/sbin/lilo} to install \LILO\ on your hard disk
  \item if you have to change the active partition, use \name{fdisk} or
    \name{activate} to do that
  \item reboot
\end{itemize}


\subsubsection{Build-time configuration}
\label{cfgmf}

Certain build-time parameters can be configured.
They can either be edited in the top-level \path{Makefile} or they can be
stored in a file \path{/etc/lilo.defines}. Settings in the \path{Makefile}
are ignored if that file exists.

The following items can be configured:
\begin{description}
  \item[\raw{BEEP}] Enables beeping after displaying ``LILO''. This is
    useful on machines without speaker and when working over a serial
    console. This option is disabled by default.
  \item[\raw{FIX\_TYPE}] Enables ``un-hiding'' of the DOS partition currently
    being booted which may have been hidden by OS/2 BootManager (partition
    types 1, 4, and 6). \raw{FIX\_TYPE} has no effect if booting anything but
    a DOS partition. This option is enabled by default but it is only effective
    if \raw{REWRITE\_TABLE} is also enabled.
  \item[\raw{HIDE\_DOS}] Enables hiding of all DOS partitions but the one
    being booted, when loading a foreign operating system.
    This option is enabled by default but it is only effective if
    \raw{REWRITE\_TABLE} and \raw{FIX\_TYPE} are also enabled.
  \item[\raw{IGNORECASE}] Makes image name matching case-insensitive, i.e.
    ``linux'' and ``Linux'' are identical. This option is enabled by
    default. Note that password matching is always case-sensitive.
  \item[\raw{NO1STDIAG}] Do not generate diagnostics on read errors in the
    first stage boot loader. This avoids possibly irritating error codes
    if the disk controller has transient read problems. This option is
    disabled by default.
  \item[\raw{NODRAIN}] The boot loader empties the keyboard buffer
    when starting, because it may contain garbage on some systems. Draining
    the keyboard buffer can be disabled by setting the \raw{NODRAIN} option.
    \raw{NODRAIN} is disabled by default.
  \item[\raw{NOINSTDEF}] If the option \craw{install} is omitted, don't install
    a new boot sector, but try to modify the old boot sector instead. This
    option is disabled by default.
  \item[\raw{ONE\_SHOT}] Disables the command-line timeout (configuration
    variable \craw{timeout}) if any key is pressed. This way, very short
    timeouts can be used if \craw{prompt} is set. \raw{ONE\_SHOT} is disabled
    by default.
  \item[\raw{READONLY}] Disallows overwriting the default command line sector
    of the map file. This way, command lines set with \raw{-R} stay in effect
    until they are explicitly removed. \raw{READONLY} also disabled \craw{lock}
    and \craw{fallback}. This option is disabled by default.
  \item[\raw{REWRITE\_TABLE}] Enables rewriting the partition table at boot
    time. This may be necessary to boot certain operating systems who
    expect the active flag to be set on their partition. See also section
    \ref{bootfor}. This option is \emphasize{dangerous} and it is disabled by
    default.
  \item[\raw{VARSETUP}] Enables the use of variable-size setup segments.
    This option is enabled by default and is only provided to fall back to
    fixed-size setup segments in the unlikely case of problems when using
    old kernels.
  \item[\raw{XL\_SECS=\meta{sectors}}] Enable support for extra large
    (non-standard) floppy disks. The number of sectors is set in the BIOS
    disk parameter table to the specified value. Note that this hack may
    yield incorrect behaviour on some systems. This option is disabled by
    default.
\end{description}

\path{/etc/lilo.defines} should be used if one wishes to make
permanent configuration changes. The usual installation procedures don't
touch that file. Example:

\begin{verbatim}
-DIGNORECASE -DONE_SHOT
\end{verbatim}

After changing the build-time configuration, \LILO\ has to be recompiled
with the following commands:

\begin{verbatim}
make spotless
make
\end{verbatim}


\subsubsection{Floppy disk installation}

In some cases\footnote{E.g. if no hard disk is accessible through the BIOS.},
it may be desirable to install \LILO\ on a floppy disk in a way that it can
boot a kernel without accessing the hard disk.

The basic procedure is quite straightforward (see also section \ref{broken}):
\begin{itemize}
  \item a file system has to be created on the file system
  \item the kernel and \path{boot.b} have to be copied to the floppy disk
  \item \path{/sbin/lilo} has to be run to create the map file
\end{itemize}

This can be as easy as
\begin{verbatim}
/sbin/mke2fs /dev/fd0
[ -d /fd ] || mkdir /fd
mount /dev/fd0 /fd
cp /boot/boot.b /fd
cp /zImage /fd
echo image=/fd/zImage label=linux |
  /sbin/lilo -C - -b /dev/fd0 -i /fd/boot.b -c -m /fd/map
umount /fd
\end{verbatim}

The command line of \path{/sbin/lilo} is a little tricky. \raw{-C -} takes
the configuration from standard input (naturally, one could also create a
configuration file), \raw{-b /dev/fd0} specifies that the boot sector is
written to the floppy disk, \raw{-i /fd/boot.b} takes the first and second
stage loaders from the floppy, \raw{-c} speeds up the load process, and
\raw{-m /fd/map} puts the map file on the floppy too.


\subsection{Updates}

\LILO\ is affected by updates of kernels, the whole system and (trivially)
of \LILO\ itself. Typically, only \path{/sbin/lilo} has to be run after
any of those updates and everything will be well again (at least as far
as \LILO\ is concerned).


\subsubsection{\LILO\ update}
\label{liloupd}

When updating to a new version of \LILO, the initial steps are the same as
for a first time installation: extract all files, configure the \path{Makefile},
run \raw{make} to build
the executables and run \raw{make install} to install the files.

The old versions of \path{boot.b}, \path{chain.b}, etc. are automatically
renamed to \path{boot.old}, \path{chain.old}, etc. This is done to ensure
that you can boot even if the installation procedure does not finish.
\path{boot.old}, \path{chain.old}, etc. can be deleted after the map file
is rebuilt.

Because the locations of \path{boot.b}, \path{chain.b}, etc. have changed
and because the map file format may be different too, you have to update
the boot sector and the map file. Run \path{/sbin/lilo} to do this.


\subsubsection{Kernel update}
\label{kernupd}

Whenever any of the kernel files that are accessed by \LILO\ is moved or
overwritten, the map has to be re-built.\footnote{It is advisable to
keep a second, stable, kernel image that can be booted if you forget
to update the map after a change to your usual kernel image.} Run
\path{/sbin/lilo} to do this.

The kernel has a make target ``zlilo'' that copies the kernel
to \path{/vmlinuz} and runs \path{/sbin/lilo}.


\subsubsection{System upgrade}

Normally, system upgrades (i.e. installation or removal of packages, possibly
replacement of large a part of the installed binaries) do not affect \LILO.
Of course, if a new kernel is installed in the process, the normal kernel
update procedure has to be followed (see section \ref{kernupd}). Also, if
kernels are removed or added, it may be necessary to update the configuration
file.

If \LILO\ is updated by this system upgrade, \path{/sbin/lilo} should be
run before booting the upgraded system. It is generally a good idea not to
rely on the upgrade procedure to perform this essential step automatically.

However, system upgrades which involve removal and re-creation of entire
partitions (e.g. \path{/}, \path{/usr}, etc.) are different. First, they
should be avoided, because they bear a high risk of losing other critical
files, e.g. the \path{/etc/Xconfig} you've spent the last week fiddling
with. If an upgrade really has to be performed in such a brute-force way,
this is equal with total removal of \LILO, followed by a new installation.
Therefore, the procedures described in the sections \ref{deinst} and
\ref{liloupd} have to be performed. If you've forgotten to make a backup
copy of \path{/etc/lilo.conf} before the destructive upgrade, you might
also have to go through section \ref{install} again.


\subsection{\LILO\ de-installation}
\label{deinst}

In order to stop \LILO\ from being invoked when the system boots, its
boot sector has to be either removed or disabled. All other files belonging
to \LILO\ can be deleted \emphasize{after} removing the boot sector,
if desired.%
\footnote{Backup copies of old boot sectors may be needed when removing
the boot sector. They are stored in \path{/boot}.}

Again, \emphasize{when removing Linux, \LILO\ must be de-installed before (!)
its files (\path{/boot}, etc.) are deleted.} This is
especially important if \LILO\ is operating as the MBR.

LILO 0.14 (and newer) can be de-installed with \verb"lilo -u". If LILO 0.14
or newer is currently installed, but the first version of LILO installed was
older than 0.14, \verb"lilo -U" may work. When using \verb"-U", the warning
at the end of this section applies.

If \LILO's boot sector has been installed on a primary partition and is
booted by the ``standard'' MBR or some partition switcher program, it can be
disabled by making a different partition active. MS-DOS' FDISK, Linux
\name{fdisk} or \LILO's \name{activate} can do that.

If \LILO's boot sector is the master boot record (MBR) of a disk, it has
to be replaced with a different MBR, typically MS-DOS' ``standard'' MBR.
When using MS-DOS 5.0 or above, the MS-DOS MBR can be restored with
\verb"FDISK /MBR". This only alters the boot loader code, not the
partition table.

\LILO\ automatically makes backup copies when it overwrites boot sectors.
They are named \path{/boot/boot.\meta{nnnn}}, with \meta{nnnn}
corresponding to the device number, e.g. \raw{0300} is \path{/dev/hda},
\raw{0800} is \path{/dev/sda}, etc. Those backups can be used to restore
the old MBR if no easier method is available. The commands are \\
\raw{dd if=/boot/boot.0300 of=/dev/hda bs=446 count=1} or \\
\raw{dd if=/boot/boot.0800 of=/dev/sda bs=446 count=1} \\
respectively.

\emphasize{WARNING:} Some Linux distributions install \path{boot.\meta{nnnn}}
files
from the system where the distribution was created. Using
those files may yield unpredictable results. Therefore, the file creation
date should be carefully checked.


\subsection{Installation of other operating systems}
\label{instoth}

Some other operating systems (e.g. MS-DOS 6.0) appear to modify the MBR in
their install procedures. It is therefore possible that \LILO\ will stop
to work after such an installation and Linux has to be booted from floppy
disk. The original state can be restored by either re-running
\path{/sbin/lilo } (if \LILO\ is installed as the MBR) or by making
\LILO's partition active (if it's installed on a primary partition).

It is generally a good idea to install \LILO\ after the other operating
systems have been installed. E.g. OS/2 is said to cause trouble when
attempting to add it to an existing Linux system. (However, booting
from floppy and running \path{/sbin/lilo} should get around most
interferences.)

Typically, the new operating system then has to be added to \LILO's
configuration (and \path{/sbin/lilo} has to be re-run) in order to boot
it.

See also section \ref{othpro} for a list of known problems with some
other operating systems.


\newpage
\section{Troubleshooting}

All parts of \LILO\ display some messages that can be used to diagnose
problems. 


\subsection{Map installer warnings and errors}

Most messages of the map installer (\path{/sbin/lilo}) should be
self-explanatory. Some messages that indicate common errors are
listed below. They are grouped
into fatal errors and warnings (non-fatal errors).


\subsubsection{Fatal errors}

\begin{description}
  \item[\raw{Boot sector of \meta{device\_name} doesn't have a boot %
    signature}]
  \item[\raw{Boot sector of \meta{device\_name} doesn't have a LILO %
    signature}]~\\
    The sector from which \LILO\ should be uninstalled doesn't appear to be a
    \LILO\ boot sector.
  \item[\raw{Can't put the boot sector on logical partition \meta{number}}]~\\
    An attempt has been made to put \LILO's boot sector on the current
    root file system partition which is on a logical partition. This usually
    doesn't have the desired effect, because common MBRs can only boot
    primary partitions. This check can be bypassed by explicitly
    specifying the boot partition with the \raw{-b} option or by setting
    the configuration variable \craw{boot}.
  \item[\raw{Checksum error}]~\\
    The descriptor table of the map file has an invalid checksum. Refresh
    the map file \emphasize{immediately} !
  \item[\raw{Device 0x\meta{number}: Configured as inaccessible.}]~\\
     There is a \craw{disk} section entry indicating that the device is
     inaccessible
     from the BIOS. You should check carefully that all files \LILO\ tries to
     access when booting are on the right device.
  \item[\raw{Device 0x\meta{number}: Got bad geometry \meta{sec}/\meta{hd}/%
\meta{cyl}}]~\\
    The device driver for your SCSI controller does not support geometry
    detection. You have to specify the geometry explicitly (see section
    \ref{diskgeo}).
  \item[\raw{Device 0x\meta{number}: Invalid partition table, entry %
    \meta{number}}]~\\
    The 3D and linear addresses of the first sector of the specified partition
    don't correspond. This is typically caused by partitioning a disk with
    a program that doesn't align partitions to tracks and later using
    PC/MS-DOS or OS/2 on that disk. \LILO\ can attempt to correct the problem,
  \item[\raw{Device 0x\meta{number}: Partition type 0x\meta{number} does not
    seem suitable for a LILO boot sector}]~\\
    The location where the \LILO\ boot sector should be placed does not seem
    to be suitable for that. (See also also section \ref{diskorg}). You should
    either adjust the partition type to reflect the actual use or put the
    boot sector on a different partition. This consistency check only yields
    a warning (i.e. \LILO\ continues) if the option \craw{ignore-table} is
    set.
  \item[\raw{\meta{device\_name} is not a valid partition device}]~\\
    The specified device is either not a device at all, a whole disk, or
    a partition on a different disk than the one in whose section its
    entry appears.
  \item[\raw{\meta{device\_name} is not a whole disk device}]~\\
    Only the geometry of whole disks (e.g. \path{/dev/hda}, \path{/dev/sdb},
    etc.) can be redefined when using \craw{disk} sections.
  \item[\raw{DISKTAB and DISK are mutually exclusive}]~\\
    You cannot use a \path{disktab} file and disk geometry definitions in the
    configuration file at the same time. Maybe \path{/etc/disktab} was
    accidentally used, because that's the default for backward-compatibility.
    You should delete \path{/etc/disktab} after completing the transition to
    \craw{disk} sections.
  \item[\raw{Duplicate entry in partition table}]~\\
    A partition table entry appears twice. The partition table has to be
    fixed with \name{fdisk}.
  \item[\raw{Duplicate geometry definition for \meta{device\_name}}]~\\
    A disk or partition geometry definition entry for the same device appears
    twice in the configuration file. Note that you mustn't write a partition
    section for the whole disk --- its start sector is always the first
    sector of the disk.
  \item[\raw{First sector of \meta{device} doesn't have a valid boot %
    signature}]~\\
    The first sector of the specified device does not appear to be a valid
    boot sector. You might have confused the device name.\footnote{Because
    different partition programs may display the partitions in a different
    order, it is possible that what you think is your first partition isn't
    \path{/dev/hda1}, etc. A good method to verify the content of a partition
    is to try to mount it.}
  \item[\raw{geo\_comp\_addr: Cylinder \meta{number} beyond end of media %
    (\meta{number})}]~\\
    A file block appears to be located beyond the last cylinder of the disk.
    This probably indicates an error in the disk geometry specification (see
    section \ref{diskgeo}) or a file system corruption.
  \item[\raw{geo\_comp\_addr: Cylinder number is too big (\meta{number} $>$
   1023)}]~\\
    Blocks of a file are located beyond the 1024th cylinder of a hard disk.
    \LILO\ can't access such files, because the BIOS limits cylinder numbers to
    the range 0$\ldots$1023. Try moving the file to a different place,
    preferably a
    partition that is entirely within the first 1024 cylinders of the disk.
  \item[\raw{Hole found in map file (\meta{location})}]~\\
    The map installer is
    confused about the disk organization. Please report this error.
  \item[\raw{\meta{item} doesn't have a valid LILO signature}]~\\
    The specified item has been located, but is not part of \LILO.
  \item[\raw{\meta{item} has an invalid stage code (\meta{number})}]~\\
    The specified item has probably been corrupted. Try re-building
    \LILO.
  \item[\raw{\meta{item} is version \meta{number}. Expecting version%
      \meta{number}.}]~\\
    The specified entity is either too old or too new. Make sure all parts
    of \LILO\ (map installer, boot loaders and chain loaders) are from the
    same distribution. \footnote{The expected version number may be different
    from the version number of the \LILO\ package, because file version
    numbers are only increased when the file formats change.}
  \item[\raw{Kernel \meta{name} is too big}]~\\
    The kernel image (without the setup
    code) is bigger than 512 kbytes. \LILO\ would overwrite itself when trying
    to load such a kernel. Try removing some unused drivers and compiling
    the kernel again. With recent (compressed) kernels, the reason for this
    error is most likely that the kernel image is damaged or that it
    contains trailing ``junk'', e.g. as the result of copying an entire boot
    floppy to the hard disk.
  \item[\raw{LOCK and FALLBACK are mutually exclusive}]~\\
    Since \craw{lock} and \craw{fallback} both change the default command line,
    they can't be reasonably used together.
  \item[\raw{Map \meta{path} is not a regular file.}]~\\
    This is probably the result of an attempt to omit writing a map file, e.g.
    with \raw{-m /dev/null}. The \raw{-t} option should be used to accomplish
    this.
  \item[\raw{Must specify LOADER for BIOS device \meta{number}}]~\\
    When booting an operating system from any device than the first hard or
    floppy disk, specifying the chain loader (option \craw{loader} in the
    image section) is now mandatory.
  \item[\raw{Must specify SECTORS and HEADS together}]~\\
    It is assumed that disks with a ``strange'' number of sectors will also
    have a ``strange'' number of heads. Therefore, it's all or nothing.
  \item[\raw{No geometry variables allowed if INACCESSIBLE}]~\\
    If a device is configured as \craw{inaccessible} (see section
    \ref{diskspec}),
    its \craw{disk} section must not contain any geometry variables.
  \item[\raw{No image \meta{image} is defined}]~\\
    The command line specified either with the \raw{-R} option or with
    \craw{fallback} does not contain the name of a valid image. Note that
    optional images which have not been included in the map file are not
    considered as valid.
  \item[\raw{Partition entry not found}]~\\
    The partition from which an other
    operating system should be booted isn't listed in the specified
    partition table. This either means that an incorrect partition
    table has been specified or that you're trying to boot from a
    logical partition. The latter usually doesn't work. You can bypass
    this check by omitting the partition table specification (e.g.
    omitting the variable \craw{table}).
  \item[\raw{REWRITE-TABLE requires $\ldots$}]~\\
    One of the many restrictions surrounding the \craw{rewrite-table}
    option was not met. Please read section \ref{bootfor} carefully.
  \item[\raw{Sorry, don't know how to handle device \meta{number}}]~\\
    \LILO\ uses files that are located on a device for which
    there is no easy way to determine the disk geometry. Such devices have
    to be explicitly described, see section \ref{diskgeo}.
  \item[\raw{Sorry, this version of LILO doesn't support REWRITE-TABLE}]~\\
    The current version of \LILO\ does not support \craw{rewrite-table} due
    to some restructuring (see \path{INCOMPAT} for details).
    \craw{rewrite-table} will be available again with
    added functionality in the next version.
  \item[\raw{This LILO is compiled READONLY and doesn't support $\ldots$}]~\\
    If \LILO\ is not allowed to write to the disk at boot time (see section
    \ref{cfgmf}), options like \craw{lock} and \craw{fallback} are unavailable.
  \item[\raw{Timestamp in boot sector of \meta{device} differs from date of %
    \meta{file}}]~\\
    The backup copy of the boot sector does not appear to be an ancestor of
    the current boot sector. If you are absolutely sure that the boot sector
    is indeed correct, you can bypass this check by using \raw{-U} instead
    of \raw{-u}.
  \item[\raw{Trying to map files from unnamed device 0x\meta{number}
    (NFS ?)}]~\\
    This is probably the same problem as described below, only with the
    root file system residing on NFS.
  \item[\raw{Trying to map files from your RAM disk.
    Please check -r option or ROOT environment variable.}]~\\
    Most likely, you or some installation script is trying to invoke
    \LILO\ in a way that some of the files is has to access reside on the
    RAM disk. Normally, the \raw{ROOT} environment variable should be set
    to the mount point of the effective root device if installing
    \LILO\ with a different root directory. See also sections \ref{creupd}
    and \ref{install}.
  \item[\raw{VGA mode presetting is not supported by your kernel.}]~\\
    Your kernel sources appear to be very old ('93 ?). \LILO\ may work on
    your system if you remove the \craw{vga} option.
  \item[\raw{write \meta{item}: \meta{error\_reason}}]~\\
    The disk is probably full or mounted read-only.
\end{description}


\subsubsection{Warnings}
\label{warnings}

Warnings labeled with ``Warning'' can be turned off with the \craw{nowarn}
option.

\begin{description}
  \item[\raw{FIGETBSZ \meta{file\_name}: \meta{ error\_reason}}]~\\
    The map installer is unable to determine the block size of a file system.
    It assumes a block size of two sectors (1kB).
  \item[\raw{Ignoring entry '\meta{variable\_name}'}]~\\
    The command-line option corresponding to the specified variable is set.
    Therefore, the configuration file entry is ignored.
  \item[\raw{Setting DELAY to 20 (2 seconds)}]~\\
    Because accidentally booting the wrong kernel or operating system may
    be very inconvenient on systems that are not run from a local display,
    the minimum delay is two seconds if the \craw{serial} variable is set.
  \item[\raw{(temp) \meta{item}: \meta{error\_reason}}]~\\
    Deleting a temporary file has failed for the specified reason.
  \item[\raw{Warning: BIOS drive 0x\meta{number} may not be accessible}]~\\
    Because most BIOS versions only support two floppies and two hard disks,
    files located on additional disks may be inaccessible. This warning
    indicates that some kernels or even the whole system may be unbootable.
  \item[\raw{Warning: COMPACT may conflict with LINEAR on some systems}]~\\
    Please see section \ref{othpro} for a description of this problem.
  \item[\raw{Warning: \meta{config\_file} should be owned by root}]~\\
    In order to prevent users from compromising system integrity, the
    configuration file should be owned by root and write access for all other
    users should be disabled.
  \item[\raw{Warning: \meta{config\_file} should be readable only for root if
    using PASSWORD}]~\\
    Users should not be allowed to read the configuration file when using the
    \craw{password} option, because then, it contains unencrypted passwords.
  \item[\raw{Warning: \meta{config\_file} should be writable only for root}]~\\
    See ``\raw{Warning: \meta{config\_file} should be owned by root}''.
  \item[\raw{Warning: device 0x\meta{number} exceeds 1024 cylinder limit}]~\\
    A disk or partition exceeds the 1024 cylinder limit imposed by the BIOS.
    This may result in a fatal error in the current installation run or in
    later installation runs. See ``\raw{geo\_comp\_addr:
    Cylinder number is too big (\meta{number} $>$ 1023)}'' for details.
  \item[\raw{Warning: \meta{device} is not on the first disk}]~\\
    The specified partition is probably not on the first disk. \LILO's boot
    sector can only be booted from the first disk unless some special boot
    manager is used.
  \item[\raw{WARNING: The system is unbootable !}]~\\
    One of the last installation steps has failed. This warning is typically
    followed by a fatal error describing the problem.
\end{description}


\subsection{Boot loader messages}

The boot loader generates three types of messages: progress and error messages
while it is loading, messages indicating disk access errors, and error
messages in response to invalid command-line input. Since messages of the
latter type are usually self-explanatory, only the two other categories
are explained.


\subsubsection{\LILO\ start message}

When \LILO\ loads itself, it displays the word ``LILO''. Each letter is
printed before or after performing some specific action. If \LILO\ fails
at some point, the letters printed so far can be used to identify the
problem. This is described in more detail in the technical overview.

Note that some hex digits may be inserted after the first ``L'' if a
transient disk problem occurs. Unless LILO stops at that point, generating
an endless stream of error codes, such hex digits do not indicate a
severe problem.

\begin{description}
  \item[\rm (\meta{nothing})] No part of \LILO\ has been loaded.
    \LILO\ either isn't installed or the partition on which its boot
    sector is located isn't active.
  \item[\raw{L \meta{error} $\ldots$}] The first stage boot loader has been
    loaded and started, but it can't load the second stage boot loader.
    The two-digit error codes indicate the type of problem. (See also section
    \ref{bioserr}.) This condition usually indicates a media failure or a
    geometry mismatch (e.g. bad disk parameters, see section \ref{diskgeo}).
  \item[\raw{LI}] The first stage boot loader was able to load the second
    stage boot loader, but has failed to execute it. This can either be
    caused by a geometry mismatch or by moving \path{/boot/boot.b} without
    running the map installer.
  \item[\raw{LIL}] The second stage boot loader has been started, but it
    can't load the descriptor table from the map file. This is typically
    caused by a media failure or by a geometry mismatch.
  \item[\raw{LIL?}] The second stage boot loader has been loaded at an
    incorrect address. This is typically caused by a subtle geometry
    mismatch or by moving \path{/boot/boot.b} without running the map
    installer.
  \item[\raw{LIL-}] The descriptor table is corrupt. This can either be caused
    by a geometry mismatch or by moving \path{/boot/map} without running
    the map installer.
  \item[\raw{LILO}] All parts of \LILO\ have been successfully loaded.
\end{description}


\subsubsection{Disk error codes}
\label{bioserr}

If the BIOS signals an error when \LILO\ is trying to load a boot
image, the respective error code is displayed. The following BIOS
error codes are known:

\begin{description}
  \item[\raw{0x00}] ``Internal error''. This code is generated by
    the sector read routine of the \LILO\ boot loader whenever an
    internal inconsistency is detected. This might be caused by corrupt
    files. Try re-building the map file. Another possible cause for this
    error are attempts to access cylinders beyond 1024 while using the
    \craw{linear} option. See section \ref{broken} for more details and
    for how to solve the problem.
  \item[\raw{0x01}] ``Illegal command''. This shouldn't happen, but if it
    does, it may indicate an attempt to access a disk which is not supported
    by the BIOS. See also ``Warning: BIOS drive 0x\meta{number} may not be
    accessible'' in section \ref{warnings}.
  \item[\raw{0x02}] ``Address mark not found''. This usually indicates
    a media problem. Try again several times.
  \item[\raw{0x03}] ``Write-protected disk''. This should only occur on
    write operations.
  \item[\raw{0x04}] ``Sector not found''. This typically indicates a
    geometry mismatch. If you're booting a raw-written disk image, verify
    whether it was created for disks with the same geometry as the one
    you're using. If you're booting from a SCSI disk or a large IDE disk,
    you should check,
    whether \LILO\ has obtained correct geometry data from the kernel or
    whether the geometry definition corresponds
    to the real disk geometry. (See section \ref{diskgeo}.) Removing
    \craw{compact} may help too. So may adding \craw{linear}.
  \item[\raw{0x06}] ``Change line active''. This should be a transient
    error. Try booting a second time.
  \item[\raw{0x07}] ``Invalid initialization''. The BIOS failed to properly
    initialize the disk controller. You should control the BIOS setup
    parameters. A warm boot might help too.
  \item[\raw{0x08}] ``DMA overrun''. This shouldn't happen. Try booting
    again.
  \item[\raw{0x09}] ``DMA attempt across 64k boundary''. This shouldn't
    happen. Try omitting the \craw{compact} option and report this problem to
    the author.
  \item[\raw{0x0C}] ``Invalid media''. This shouldn't happen and might
    be caused by a media error. Try booting again.
  \item[\raw{0x10}] ``CRC error''. A media error has been detected. Try
    booting several times,
    running the map installer a second time (to put the map file at some
    other physical location or to write ``good data'' over
    the bad spot), mapping out the bad sectors/tracks and, if all else
    fails, replacing the media.
  \item[\raw{0x11}] ``ECC correction successful''. A read error occurred,
    but was corrected. \LILO\ does not recognize this condition and
    aborts the load process anyway. A second load attempt should
    succeed.
  \item[\raw{0x20}] ``Controller error''. This shouldn't happen.
  \item[\raw{0x40}] ``Seek failure''. This might be a media problem. Try
    booting again.
  \item[\raw{0x80}] ``Disk timeout''. The disk or the drive isn't ready.
    Either the media is bad or the disk isn't spinning. If you're booting
    from a floppy, you might not have closed the drive door. Otherwise,
    trying to boot again might help.
  \item[\raw{0xBB}] ``BIOS error''. This shouldn't happen. Try booting
    again. If the problem persists, removing the \craw{compact} option or
    adding/removing \craw{linear} might help.
\end{description}

If the error occurred during a write operation, the error code (two hex
digits) is prefixed with a ``W''. Although write errors don't affect the
boot process, they might indicate a severe problem, because they usually
imply that \LILO\ has tried to write to an invalid location.
If spurious write errors occur on a system, it might be a good idea
to configure \LILO\ to run read-only (see section \ref{cfgmf}).

Generally, invalid geometry and attempts to use more than two disks without a
very modern BIOS may yield misleading error codes. Please check carefully
if \path{/sbin/lilo} doesn't emit any warnings. Then try using the
\craw{linear} option (see section \ref{cfgglo}).


\subsection{Other problems}
\label{othpro}

This section contains a collection of less common problems that have been
observed. See also section \ref{instoth} for general remarks on using
\LILO\ with other operating systems. Some of the problems are obscure and
so are the work-arounds.

\begin{itemize}
  \item If \LILO\ doesn't go away even if you erase its files, format your
    Linux partition, etc., you've probably installed \LILO\ as your MBR and
    you've forgotten to deinstall it before deleting its files. See section
    \ref{deinst} for what you can do now.
  \item For yet unknown reasons, \LILO\ may fail on some systems with
    AMI BIOS if the ``Hard Disk Type 47 RAM area'' is set to ``0:300''
    instead of ``DOS 1K''.
  \item Some disk controller BIOSes perform disk geometry/address translations
    that
    are incompatible with the way the device's geometry is seen from Linux,
    i.e. without going through the BIOS. Particularly, large IDE disks and
    some recent PCI SCSI
    controllers appear to have this problem. In such cases, either the
    translated geometry has to be specified in a \craw{disk} section or the
    sector address translation can be deferred by using the \craw{linear}
    option. In a setup where floppies are not normally used for booting,
    the \craw{linear} approach should be preferred, because this avoids
    the risk of specifying incorrect numbers.
  \item OS/2 is said to be bootable from a logical partition with
    \LILO\ acting as the primary boot selector if \LILO\ is installed on
    the MBR, the OS/2 BootManager is on an active primary partition and
    \LILO\ boots BootManager. Putting \LILO\ on an extended partition
    instead is said to crash the OS/2 FDISK in this scenario.

    Note that booting \LILO\ from BootManager (so BootManager is the
    primary selector) or booting OS/2 directly from a primary partition
    (without BootManager) should generally work. See also section
    \ref{instoth}.
  \item Windows NT is reported to be bootable with \LILO\ when \LILO\ acts as
    the MBR and the Windows NT boot loader is on the DOS partition. However,
    NT's disk manager complains about LILO's MBR when trying to edit the
    partition table.
  \item Some PC UNIX systems (SCO and Unixware have been reported to exhibit
    this problem) depend on their partition being active. Such a setup can
    currently only be obtained by installing \LILO\ as the MBR and making
    the respective partition active.\footnote{Future versions of \LILO\ may
    be able to change the active flag dynamically.}
  \item Future Domain TMC-1680 adapters with the BIOS versions 3.4 and 3.5
    assign BIOS device numbers in the wrong order, e.g. on a two-disk system,
    \path{/dev/sda} becomes \raw{0x81} and \path{/dev/sdb} becomes
    \raw{0x80}. This can be fixed with the following \craw{disk} section:\\
    \verb"disk=/dev/sda bios=0x81 disk=/dev/sdb bios=0x80"\\
    Note that this is only valid for a two-disk system. In three-disk
    systems, \path{/dev/sdc} would become \raw{0x80}, etc. Also, single-disk
    systems don't have this problem (and the ``fix'' would break them).
  \item Some BIOSes don't properly recognize disks with an unusual partition
    table (e.g. without any partition marked active) and refuse to boot from
    them. This can also affect the second hard disk and the problem may
    only occur if the system is booted in a particular way (e.g. only after a
    cold boot).
  \item On some systems, using \craw{linear} and \craw{compact} together
    leads to a boot failure. The exact circumstances under which this
    happens are still unknown.
\end{itemize}

%%beginskip
\end{document}
%%endskip
