\documentstyle[11pt]{report}
\def\thesection{\arabic{section}}
\setcounter{tocdepth}{3}
\setcounter{secnumdepth}{3}
\textwidth=6in
\textheight=8.5in
\hoffset=-.6in
\voffset=-.6in
\newcommand{\dent}{\hspace*{\parindent}}
\newcommand{\nh}[1]{\hyphenation{#1}}
%
%    The definition of \bigarray requires that @ be considered a letter.
%    Thus, the \catcode commands are necessary.
%
\catcode`\@=11
\def\thefigure{\@arabic\c@figure}
\def\thetable{\@arabic\c@table}
\def\bigarray{%
    \def\@arrayclassz{\ifcase \@lastchclass \@acolampacol \or \@ampacol \or
       \or \or \@addamp \or
       \@acolampacol \or \@firstampfalse \@acol \fi
    \edef\@preamble{\@preamble
      \ifcase \@chnum
         \hfil$\displaystyle\relax\@sharp$\hfil
          \or $\displaystyle\relax\@sharp$\hfil
     \or \hfil$\displaystyle\relax\@sharp$\fi}}
    }
\catcode`\@=12
%
\newcommand{\leftnorm}{\left\|}
\newcommand{\rightnorm}{\right\|}
\newcommand{\TS}{\begin{table}[tbh]\centering}
\newcommand{\TE}{\end{table}}
\newcommand{\DS}{\vspace{11pt}\newline\hspace*{\parindent}}
\newcommand{\DE}{\vspace{11pt}\newline}
\newcommand{\FTS}{\begin{table}[p]\centering}
\newcommand{\FTE}{\end{table}}
\newcommand{\bop}{\vspace{11pt}\begin{flushleft}\begin{tabular}{@{} p{4.5em} l l }}
\newcommand{\eop}{\end{tabular}\end{flushleft}}
\newcommand{\keep}[1]{\vspace{11pt}\parbox{\textwidth}{#1}\vspace{11pt}}
\newcommand{\up}[1]{\small\raisebox{1ex}{#1}}
\newcommand{\biblio}{\begin{thebibliography}}
\newcommand{\bibend}{\end{thebibliography}}
%\renewcommand{\footnoterule}{\rule{1.0in}{0.2mm}}

% psfigTeX macros
%
% All software, documentation, and related files in this distribution of
% psfig/tex are Copyright (c) 1987 Trevor J. Darrell
%
% Permission is granted for use and non-profit distribution of psfig/tex 
% providing that this notice be clearly maintained, but the right to
% distribute any portion of psfig/tex for profit or as part of any commercial
% product is specifically reserved for the author.
%
% Psfig/tex version 1.1
%
% file last modified: $Header: psfig.tex,v 1.8 87/07/25 13:21:09 trevor Exp $
%
\catcode`\@=11\relax
\newwrite\@unused
\def\typeout#1{{\let\protect\string\immediate\write\@unused{#1}}}
\typeout{psfig: version 1.1}
\def\psglobal#1{
\typeout{psfig: including #1 globally}
\immediate\special{ps:plotfile #1 global}}
\def\psfiginit{\typeout{psfiginit}
\immediate\psglobal{/usr/lib/ps/figtex.pro}}
%
% @psdo control structure -- similar to Latex @for.
% I redefined these with different names so that psfig can
% be used with TeX as well as LaTeX, and so that it will not 
% be vunerable to future changes in LaTeX's internal
% control structure,
%
\def\@nnil{\@nil}
\def\@empty{}
\def\@psdonoop#1\@@#2#3{}
\def\@psdo#1:=#2\do#3{\edef\@psdotmp{#2}\ifx\@psdotmp\@empty \else
    \expandafter\@psdoloop#2,\@nil,\@nil\@@#1{#3}\fi}
\def\@psdoloop#1,#2,#3\@@#4#5{\def#4{#1}\ifx #4\@nnil \else
       #5\def#4{#2}\ifx #4\@nnil \else#5\@ipsdoloop #3\@@#4{#5}\fi\fi}
\def\@ipsdoloop#1,#2\@@#3#4{\def#3{#1}\ifx #3\@nnil 
       \let\@nextwhile=\@psdonoop \else
      #4\relax\let\@nextwhile=\@ipsdoloop\fi\@nextwhile#2\@@#3{#4}}
\def\@tpsdo#1:=#2\do#3{\xdef\@psdotmp{#2}\ifx\@psdotmp\@empty \else
    \@tpsdoloop#2\@nil\@nil\@@#1{#3}\fi}
\def\@tpsdoloop#1#2\@@#3#4{\def#3{#1}\ifx #3\@nnil 
       \let\@nextwhile=\@psdonoop \else
      #4\relax\let\@nextwhile=\@tpsdoloop\fi\@nextwhile#2\@@#3{#4}}
% 
%
\def\psdraft{
	\def\@psdraft{0}
	%\typeout{draft level now is \@psdraft \space . }
}
\def\psfull{
	\def\@psdraft{100}
	%\typeout{draft level now is \@psdraft \space . }
}
\psfull
\newif\if@prologfile
\newif\if@postlogfile
%%% These are for the option list.
%%% A specification of the form a = b maps to calling \@p@@sa{b}
\newif\if@bbllx
\newif\if@bblly
\newif\if@bburx
\newif\if@bbury
\newif\if@height
\newif\if@width
\newif\if@rheight
\newif\if@rwidth
\newif\if@clip
\def\@p@@sclip#1{\@cliptrue}
\def\@p@@sfile#1{%\typeout{file is #1}
		   \def\@p@sfile{#1}
}
\def\@p@@sfigure#1{\def\@p@sfile{#1}}
\def\@p@@sbbllx#1{
		%\typeout{bbllx is #1}
		\@bbllxtrue
		\dimen100=#1
		\edef\@p@sbbllx{\number\dimen100}
}
\def\@p@@sbblly#1{
		%\typeout{bblly is #1}
		\@bbllytrue
		\dimen100=#1
		\edef\@p@sbblly{\number\dimen100}
}
\def\@p@@sbburx#1{
		%\typeout{bburx is #1}
		\@bburxtrue
		\dimen100=#1
		\edef\@p@sbburx{\number\dimen100}
}
\def\@p@@sbbury#1{
		%\typeout{bbury is #1}
		\@bburytrue
		\dimen100=#1
		\edef\@p@sbbury{\number\dimen100}
}
\def\@p@@sheight#1{
		\@heighttrue
		\dimen100=#1
   		\edef\@p@sheight{\number\dimen100}
		%\typeout{Height is \@p@sheight}
}
\def\@p@@swidth#1{
		%\typeout{Width is #1}
		\@widthtrue
		\dimen100=#1
		\edef\@p@swidth{\number\dimen100}
}
\def\@p@@srheight#1{
		%\typeout{Reserved height is #1}
		\@rheighttrue
		\dimen100=#1
		\edef\@p@srheight{\number\dimen100}
}
\def\@p@@srwidth#1{
		%\typeout{Reserved width is #1}
		\@rwidthtrue
		\dimen100=#1
		\edef\@p@srwidth{\number\dimen100}
}
\def\@p@@sprolog#1{\@prologfiletrue\def\@prologfileval{#1}}
\def\@p@@spostlog#1{\@postlogfiletrue\def\@postlogfileval{#1}}
\def\@cs@name#1{\csname #1\endcsname}
\def\@setparms#1=#2,{\@cs@name{@p@@s#1}{#2}}
%
% initialize the defaults (size the size of the figure)
%
\def\ps@init@parms{
		\@bbllxfalse \@bbllyfalse
		\@bburxfalse \@bburyfalse
		\@heightfalse \@widthfalse
		\@rheightfalse \@rwidthfalse
		\def\@p@sbbllx{}\def\@p@sbblly{}
		\def\@p@sbburx{}\def\@p@sbbury{}
		\def\@p@sheight{}\def\@p@swidth{}
		\def\@p@srheight{}\def\@p@srwidth{}
		\def\@p@sfile{}
		\def\@p@scost{10}
		\def\@sc{}
		\@prologfilefalse
		\@postlogfilefalse
		\@clipfalse
}
%
% Go through the options setting things up.
%
\def\parse@ps@parms#1{
	 	\@psdo\@psfiga:=#1\do
		   {\expandafter\@setparms\@psfiga,}}
%
% Compute bb height and width
%
\newif\ifno@bb
\newif\ifnot@eof
\newread\ps@stream
\def\bb@missing{
	\typeout{psfig: searching \@p@sfile \space  for bounding box}
	\openin\ps@stream=\@p@sfile
	\no@bbtrue
	\not@eoftrue
	\catcode`\%=12
	\loop
		\read\ps@stream to \line@in
		\global\toks200=\expandafter{\line@in}
		\ifeof\ps@stream \not@eoffalse \fi
		%\typeout{ looking at :: \the\toks200 }
		\@bbtest{\toks200}
		\if@bbmatch\not@eoffalse\expandafter\bb@cull\the\toks200\fi
	\ifnot@eof \repeat
	\catcode`\%=14
}	
\catcode`\%=12
\newif\if@bbmatch
\def\@bbtest#1{\expandafter\@a@\the#1%%BoundingBox:\@bbtest\@a@}
\long\def\@a@#1%%BoundingBox:#2#3\@a@{\ifx\@bbtest#2\@bbmatchfalse\else\@bbmatchtrue\fi}
\long\def\bb@cull#1 #2 #3 #4 #5 {
	\dimen100=#2 bp\edef\@p@sbbllx{\number\dimen100}
	\dimen100=#3 bp\edef\@p@sbblly{\number\dimen100}
	\dimen100=#4 bp\edef\@p@sbburx{\number\dimen100}
	\dimen100=#5 bp\edef\@p@sbbury{\number\dimen100}
	\no@bbfalse
}
\catcode`\%=14
%
\def\compute@bb{
		\no@bbfalse
		\if@bbllx \else \no@bbtrue \fi
		\if@bblly \else \no@bbtrue \fi
		\if@bburx \else \no@bbtrue \fi
		\if@bbury \else \no@bbtrue \fi
		\ifno@bb \bb@missing \fi
		\ifno@bb \typeout{FATAL ERROR: no bb supplied or found}
			\no-bb-error
		\fi
		%
		\count203=\@p@sbburx
		\count204=\@p@sbbury
		\advance\count203 by -\@p@sbbllx
		\advance\count204 by -\@p@sbblly
		\edef\@bbw{\number\count203}
		\edef\@bbh{\number\count204}
		%\typeout{ bbh = \@bbh, bbw = \@bbw }
}
%
% \in@hundreds performs #1 * (#2 / #3) correct to the hundreds,
%	then leaves the result in @result
%
\def\in@hundreds#1#2#3{\count240=#2 \count241=#3
		     \count100=\count240	% 100 is first digit #2/#3
		     \divide\count100 by \count241
		     \count101=\count100
		     \multiply\count101 by \count241
		     \advance\count240 by -\count101
		     \multiply\count240 by 10
		     \count101=\count240	%101 is second digit of #2/#3
		     \divide\count101 by \count241
		     \count102=\count101
		     \multiply\count102 by \count241
		     \advance\count240 by -\count102
		     \multiply\count240 by 10
		     \count102=\count240	% 102 is the third digit
		     \divide\count102 by \count241
		     \count200=#1\count205=0
		     \count201=\count200
			\multiply\count201 by \count100
		 	\advance\count205 by \count201
		     \count201=\count200
			\divide\count201 by 10
			\multiply\count201 by \count101
			\advance\count205 by \count201
			%
		     \count201=\count200
			\divide\count201 by 100
			\multiply\count201 by \count102
			\advance\count205 by \count201
			%
		     \edef\@result{\number\count205}
}
\def\compute@wfromh{
		% computing : width = height * (bbw / bbh)
		\in@hundreds{\@p@sheight}{\@bbw}{\@bbh}
		%\typeout{ \@p@sheight * \@bbw / \@bbh, = \@result }
		\edef\@p@swidth{\@result}
		%\typeout{w from h: width is \@p@swidth}
}
\def\compute@hfromw{
		% computing : height = width * (bbh / bbw)
		\in@hundreds{\@p@swidth}{\@bbh}{\@bbw}
		%\typeout{ \@p@swidth * \@bbh / \@bbw = \@result }
		\edef\@p@sheight{\@result}
		%\typeout{h from w : height is \@p@sheight}
}
\def\compute@handw{
		\if@height 
			\if@width
			\else
				\compute@wfromh
			\fi
		\else 
			\if@width
				\compute@hfromw
			\else
				\edef\@p@sheight{\@bbh}
				\edef\@p@swidth{\@bbw}
			\fi
		\fi
}
\def\compute@resv{
		\if@rheight \else \edef\@p@srheight{\@p@sheight} \fi
		\if@rwidth \else \edef\@p@srwidth{\@p@swidth} \fi
}
%		
% Compute any missing values
\def\compute@sizes{
	\compute@bb
	\compute@handw
	\compute@resv
}
%
% \psfig
% usage : \psfig{file=, height=, width=, bbllx=, bblly=, bburx=, bbury=,
%			rheight=, rwidth=, clip=}
%
% "clip=" is a switch and takes no value, but the `=' must be preset.
\def\psfig#1{\vbox {
	% do a zero width hard space so that a single
	% \psfig in a centering enviornment will behave nicely
	%{\setbox0=\hbox{\ }\ \hskip-\wd0}
	%
	\ps@init@parms
	\parse@ps@parms{#1}
	\compute@sizes
	%
	\ifnum\@p@scost<\@psdraft{
		\typeout{psfig: including \@p@sfile \space }
		%
		\special{ps::[begin] 	\@p@swidth \space \@p@sheight \space
				\@p@sbbllx \space \@p@sbblly \space
				\@p@sbburx \space \@p@sbbury \space
				startTexFig \space }
		\if@clip{
			\typeout{(clip)}
			\special{ps:: \@p@sbbllx \space \@p@sbblly \space
				\@p@sbburx \space \@p@sbbury \space
				doclip \space }
		}\fi
		\if@prologfile
		    \special{ps: plotfile \@prologfileval \space } \fi
		\special{ps: plotfile \@p@sfile \space }
		\if@postlogfile
		    \special{ps: plotfile \@postlogfileval \space } \fi
		\special{ps::[end] endTexFig \space }
		% Create the vbox to reserve the space for the figure
		\vbox to \@p@srheight true sp{
			\hbox to \@p@srwidth true sp{
				\hfil
			}
		\vfil
		}
	}\else{
		% draft figure, just reserve the space and print the
		% path name.
		\vbox to \@p@srheight true sp{
		\vss
			\hbox to \@p@srwidth true sp{
				\hss
				\@p@sfile
				\hss
			}
		\vss
		}
	}\fi
}}
\catcode`\@=12\relax



\newcounter{titem}
\newcommand{\titem}{\stepcounter{titem}\thetitem.}



\begin{document}
\begin{center}
  {\Large LAPACK Working Note 81\\
  Quick Installation Guide for LAPACK on Unix Systems\footnote{This work was
 supported by NSF Grant No. ASC-8715728.}}
\end{center}
\begin{center}
%  Edward Anderson\footnote{Current address:  Cray Research Inc.,
%                           655F Lone Oak Drive, Eagan, MN  55121},
  Jack Dongarra and Susan Ostrouchov\\
  Department of Computer Science \\
  University of Tennessee \\
  Knoxville, Tennessee  37996-1301 \\
\end{center}
\begin{center}
  REVISED:  VERSION 2.0, September 30, 1994
\end{center}

\begin{center}
Abstract
\end{center}
This working note describes how to install, test, and time version 2.0
of LAPACK, a linear algebra package for high-performance
computers, on a Unix System.  Non-Unix installation instructions and
further details of the testing and timing suites are only contained in
LAPACK Working Note 41, and not in this abbreviated version.
%Separate instructions are provided for the Unix and non-Unix
%versions of the test package.
%Further details are also given on the design of the test and timing
%programs. 
\newpage

\tableofcontents

\newpage
% Introduction to Implementation Guide

\section{Introduction}

\dent
LAPACK is a linear algebra library for high-performance
computers.
The library includes Fortran 77 subroutines for 
the analysis and solution of systems of simultaneous linear algebraic
equations, linear least-squares problems, and matrix eigenvalue
problems.
Our approach to achieving high efficiency is based on the use of
a standard set of Basic Linear Algebra Subprograms (the BLAS),
which can be optimized for each computing environment.
By confining most of the computational work to the BLAS,
the subroutines should be 
transportable and efficient across a wide range of computers.

This working note describes how to install, test, and time this
release of LAPACK on a Unix System.

The instructions for installing, testing, and timing are designed for a 
person whose
responsibility is the maintenance of a mathematical software library.
We assume the installer has experience in compiling and running 
Fortran programs and in creating object libraries.
The installation process involves reading the tape or tarring the file, creating a set of
libraries, and compiling and running the test and timing programs.

%This guide combines the instructions for the Unix and non-Unix
%versions of the LAPACK test package (the non-Unix version is in Appendix
%~\ref{appendixe}).
%At this time, the non-Unix version of LAPACK can only be obtained
%after first untarring the Unix tar tape and then following the instructions in
%Appendix ~\ref{appendixe}.

Section~\ref{fileformat} describes how the files are organized on the tape
or file, and
Section~\ref{overview} gives a general overview of the parts of the test package.
Step-by-step instructions appear in Section~\ref{installation}.
%for the Unix version and in the appendix for the non-Unix version.

For users desiring additional information, please refer to LAPACK
Working Note 41.
% Sections~\ref{moretesting}
%and ~\ref{moretiming} give
%details of the test and timing programs and their input files.
%Appendices ~\ref{appendixa} and ~\ref{appendixb} briefly describe 
%the LAPACK routines and auxiliary routines provided
%in this release.  
%Appendix ~\ref{appendixc} lists the operation counts we have computed 
%for the BLAS and for some of the LAPACK routines.
%Appendix ~\ref{appendixd}, entitled ``Caveats'', is a compendium of the known 
%problems from our own experiences, with suggestions on how to 
%overcome them.  {\em It is strongly advised that the user read Appendix
%D before proceeding with the installation process.}
%Appendix E contains the execution times of the different test
%and timing runs on two sample machines.
%Appendix ~\ref{appendixe} contains the instructions to install LAPACK on a non-Unix
%system.

\section{Revisions Since the First Public Release}

\dent
Since its first public release in February, 1992, LAPACK has had
several updates, which have encompassed the introduction of new routines as
well as extending the functionality of existing routines.  The first update,
June 30, 1992, was version 1.0a; the second update, October 31, 1992,
was version 1.0b; the third update, March 31, 1993, was version 1.1; and
finally, September 30, 1994, version 2.0.  All LAPACK routines reflect the current
version number with the date 
on the routine indicating when it was last modified.
For more information on
revisions please refer to the LAPACK release\_notes file
on netlib, or the second edition of the LAPACK Users' Guide.

We plan to have only one or two updates a year, and provide a PRERELEASE
directory on netlib to contain new software that is being considered for
inclusion.  Users can then provide input and experimentation with these
prerelease routines.  The {\tt tar} file {\tt lapack.tar.z} that is
available on netlib is always the most up-to-date.

On-line manpages (troff files) for LAPACK driver and computational
routines, as well as most of the BLAS routines, are available via 
the {\tt lapack} index on netlib.

%The new routines that have been introduced are
%driver and computational routines for
%the generalized nonsymmetric eigenproblem,
%generalized linear least squares problems,
%the generalized singular value decomposition,
%a generalized banded symmetric-definite eigenproblem,
%and divide-and-conquer methods for symmetric eigenproblems.
%Additional computational routines include
%the generalized QR and RQ factorizations, and
%reduction of a band matrix to bidiagonal form.

%Added functionality has been incorporated into the expert driver
%routines that involve equilibration (xGESVX, xGBSVX, xPOSVX, xPPSVX,
%and xPBSVX).  The option FACT = 'F' now permits the user to input a
%prefactored, pre-equilibrated matrix.  The expert drivers xGESVX and
%xGBSVX
%now return the reciprocal of the pivot growth from Gaussian
%elimination.  xBDSQR has been modified to compute singular values of
%bidiagonal matrices much more quickly than before, provided singular
%vectors are not also wanted.
%The least squares driver
%routines xGELS, xGELSS, and xGELSX now make available the residual
%root-sum-squares for each right hand side.
%
%\section*{Bug Fixes and Code Modifications:}
%\begin{itemize}
%\item CHEGST, CHSEQR, SSTEBZ (and their double precision equivalents)
%had errors corrected;
%\item CLANST $\rightarrow$ CLANHT name change to avoid notation
%confusion;
%\item Special version of ILAENV was missing from the TESTING and TIMING
%directories;
%\item All -EVX drivers and xGESVX had incorrect workspace requirements
%(WORK for the real routines, and RWORK in the complex routines)
%specified in their Argument sections;
%\item xGTCON also had incorrect workspace requirements specified in
%its Argument section;
%\item Added functionality has been incorporated into the expert driver
%routines that involve equilibration (xGESVX, xGBSVX, xPOSVX, xPPSVX,
%xPBSVX).  The option FACT = 'F' now permits the user to input a
%prefactored, pre-equilibrated matrix.
%\item comments added to SLAEBZ and SSTEBZ to enforce SCALAR mode in
%certain loops; in addition, in front of each such "DO" statement
%a series of compiler directives has been added which should
%enforce SCALAR mode on most popular vector computers;
%\end{itemize}
%\section*{New Routines:}
%\begin{itemize}
%\item xGBBRD, computation routine for the banded reduction to bidiagonal
%form;
%\item xGEGS and xGEGV, driver routines for the Generalized Nonsymmetric
%Eigenvalue Problem with accompanying computational and auxiliary
%routines xGGBAK, xGGBAL, xGGHRD, xHGEQZ, xTGEVC, and SLAG2;
%\item xGGQRF and xGGRQF, computational routines for the generalized QR
%and RQ factorizations;
%\item xGGSVD, driver routine for the generalized singular value
%decomposition with accompanying computational and auxiliary routines
%xGGSVP, xTGSJA, xLAGS2, xLAPLL, and xLAPMT;
%\item xGGGLM, driver routine for the solution of the generalized linear
%regression model;
%\item xGGLSE, driver routine for the solution of the constrained linear
%least squares problem;
%\item SSBGV/CHBGV, driver routines for the generalized banded
%symmetric-definite eigenproblem with accompanying computational routines
%SSBGST/CHBGST, and xPBSTF;
%\item xSBEVD, xSPEVD, SSTEVD, and SSYEVD/CHEEVD, divide and conquer
%routines for symmetric eigenproblem with accompanying auxiliary routines
%CLACRM, xLAED0, SLAED1, SLAED2, SLAED3, SLAED4, SLAED5, SLAED6, xLAED7,
%xLAED8, SLAED9, SLAEDA, SLAMRG, SLASQ1, SLASQ2, SLASQ3, SLASQ4, and SLASRT.
%\end{itemize}

\section{File Format}\label{fileformat}

\dent
The software for LAPACK is distributed in the form of a
compressed tar file (via xnetlib, anonymous ftp, or the World Wide Web)
or a Unix {\tt tar} tape from NAG (Numerical Algorithms Group, Inc.),
which contains the Fortran source for LAPACK,
the Basic Linear Algebra Subprograms
(the Level 1, 2, and 3 BLAS) needed by LAPACK, the testing programs,
and the timing programs.
Users who wish to have a non-Unix installation should refer to LAPACK
Working Note 41,
although the overview in section~\ref{overview} applies to both the Unix and non-Unix
versions.
%Users who wish to have a non-Unix installation should go to Appendix ~\ref{appendixe},
%although the overview in section ~\ref{overview} applies to both the Unix and non-Unix
%versions.

%Xnetlib is an X-version of netlib developed at the University
%of Tennessee and Oak Ridge National Laboratory.  Unlike netlib, which
%uses electronic mail to process requests for software and other text,
%xnetlib uses an X Window graphical user interface and a socket-based
%connection between the user's machine and the xnetlib server machine to
%process software requests.
%To receive a copy of xnetlib send the message "send xnetlib.shar from
%xnetlib" to netlib@ornl.gov.
%When you receive the shar file, remove the mail header, save it to a
%file, type 'sh filename' and follow the instructions in the README
%file.
%
%After installing and invoking xnetlib, select the \fbox{LIBRARY} button
%in the header menu with your mouse, click your mouse on the word {\tt lapack}
%in the listing, and the current LAPACK index will appear.
%At this point you can download any files (for example, the compressed tar
%file {\tt lapack.tar.z}) by clicking the mouse on the desired file(s),
%selecting the \fbox{DOWNLOAD} button from the menu, and finally
%clicking on the \fbox{GET FILES NOW} button.
%
%Alternatively, you can retrieve the file via anonymous ftp at netlib.
%Your instructions are:
%
%\begin{verbatim}
%     ftp netlib2.cs.utk.edu
%     login:  anonymous
%     password:  <your email address>
%     cd lapack
%     binary
%     get lapack.tar.z
%     quit
%\end{verbatim}
%
%And finally, the package may be accessed via the World Wide Web through
%the URL address:
%\begin{quote}
%{\tt http://netlib2.cs.utk.edu/lapack/index.html}
%\end{quote}

The software on the {\tt tar} tape or {\tt tar} file
is organized in a number of essential directories as shown
in Figure 1.  Please note that this figure does not reflect everything
that is contained in the {\tt LAPACK} directory.  Input and instructional
files are also located at various levels.
\begin{figure}
\vspace{11pt}
\centerline{\psfig{figure=org2.ps,width=6.5in}}
\caption{Unix organization of LAPACK}
\vspace{11pt}
\end{figure}
Libraries are created in the LAPACK directory and 
executable files are created in one of the directories BLAS, TESTING,
or TIMING.  Input files for the test and timing programs are also
found in these three directories so that testing may be carried out
in the directories LAPACK/BLAS, LAPACK/TESTING, and LAPACK/TIMING.
A top-level makefile in the LAPACK directory is provided to perform the 
entire installation procedure.

\section{Overview of Tape Contents}\label{overview}

\dent
Most routines in LAPACK occur in four versions: REAL,
DOUBLE PRECISION, COMPLEX, and COMPLEX*16.
The first three versions (REAL, DOUBLE PRECISION, and COMPLEX)
are written in standard Fortran 77 and are completely portable;
the COMPLEX*16 version is provided for
those compilers which allow this data type.  
For convenience, we often refer to routines by their single precision
names; the leading `S' can be replaced by a `D' for double precision,
a `C' for complex, or a `Z' for complex*16.  
For LAPACK use and testing you must decide which version(s)
of the package you intend to install at your site (for example,
REAL and COMPLEX on a Cray computer or DOUBLE PRECISION and
COMPLEX*16 on an IBM computer).

\subsection{LAPACK Routines}

\hspace{\parindent}
There are three classes of LAPACK routines:
\begin{itemize}

\item {\bf driver} routines solve a complete problem, such as solving
a system of linear equations or computing the eigenvalues of a real
symmetric matrix.  Users are encouraged to use a driver routine if there
is one that meets their requirements.  The driver routines are listed
in LAPACK Working Note 41~\cite{WN41} and the LAPACK Users' Guide~\cite{LUG}.
%in Appendix ~\ref{appendixa}.

\item {\bf computational} routines, also called simply LAPACK routines,
perform a distinct computational task, such as computing
the $LU$ decomposition of an $m$-by-$n$ matrix or finding the
eigenvalues and eigenvectors of a symmetric tridiagonal matrix using
the $QR$ algorithm. 
The LAPACK routines are listed in LAPACK Working Note 41~\cite{WN41}
and the LAPACK Users' Guide~\cite{LUG}.
%The LAPACK routines are listed in Appendix ~\ref{appendixa}; see also LAPACK
%Working Note \#5 \cite{WN5}.

\item {\bf auxiliary} routines are all the other subroutines called
by the driver routines and computational routines.  
%Among them are subroutines to perform subtasks of block algorithms,
%in particular, the unblocked versions of the block algorithms;
%extensions to the BLAS, such as matrix-vector operations involving
%complex symmetric matrices;
%the special routines LSAME and XERBLA which first appeared with the
%BLAS;
%and a number of routines to perform common low-level computations,
%such as computing a matrix norm, generating an elementary Householder
%transformation, and applying a sequence of plane rotations.
%Many of the auxiliary routines may be of use to numerical analysts
%or software developers, so we have documented the Fortran source for
%these routines with the same level of detail used for the LAPACK
%routines and driver routines.
The auxiliary routines are listed in LAPACK Working Note 41~\cite{WN41}
and the LAPACK Users' Guide~\cite{LUG}.
%The auxiliary routines are listed in Appendix ~\ref{appendixb}.
\end{itemize}

\subsection{Level 1, 2, and 3 BLAS}

\hspace{\parindent}
The BLAS are a set of Basic Linear Algebra Subprograms that perform
vector-vector, matrix-vector, and matrix-matrix operations. 
LAPACK is designed around the Level 1, 2, and 3 BLAS, and nearly all
of the parallelism in the LAPACK routines is contained in the BLAS.
Therefore,
the key to getting good performance from LAPACK lies in having an
efficient version of the BLAS optimized for your particular machine. 
If you have access to a library containing optimized versions of some
or all of the BLAS, you should certainly use it (but be sure to run
the BLAS test programs).
If an optimized library of the BLAS is not available,
Fortran source code for the Level 1, 2, and 3 BLAS
is provided on the tape.
Users should not expect too much from the Fortran BLAS; these versions
were written to define the basic operations and do not employ the
standard tricks for optimizing Fortran code.

The formal definitions of the Level 1, 2, and 3 BLAS
are in \cite{BLAS1}, \cite{BLAS2}, and \cite{BLAS3}. 
Copies of the BLAS Quick Reference card are available from the authors
or netlib.

\subsection{LAPACK Test Routines}

\hspace{\parindent}
This release contains two distinct test programs for LAPACK routines
in each data type.  One test program tests the routines for solving
linear equations and linear least squares problems,
and the other tests routines for the matrix eigenvalue problem.
The routines for generating test matrices are used by both test
programs and are compiled into a library for use by both test programs.

\subsection{LAPACK Timing Routines}

\hspace{\parindent}
This release also contains two distinct timing programs for the
LAPACK routines in each data type. 
The linear equation timing program gathers performance data in
megaflops on the factor, solve, and inverse routines for solving
linear systems, the routines to generate or apply an orthogonal matrix
given as a sequence of elementary transformations, and the reductions
to bidiagonal, tridiagonal, or Hessenberg form for eigenvalue
computations.
The operation counts used in computing the megaflop rates are computed
from a formula;
see LAPACK Working Note 41~\cite{WN41}.
% see Appendix ~\ref{appendixc}.
The eigenvalue timing program is used with the eigensystem routines
and returns the execution time, number of floating point operations, and
megaflop rate for each of the requested subroutines.
In this program, the number of operations is computed while the
code is executing using special instrumented versions of the LAPACK
subroutines.

\section{Installing LAPACK on a Unix System}\label{installation}

Installing, testing, and timing the Unix version of LAPACK
involves the following steps: 
\begin{enumerate}
\item Read the tape or uncompress and tar the file.

\item Edit the file {\tt LAPACK/make.inc}.
 
\item Edit the file {\tt LAPACK/Makefile} and type {\tt make}.

%\item Test and Install the Machine-Dependent Routines \\
%{\em (WARNING:  You may need to supply a correct version of second.f and
%dsecnd.f for your machine)}
%{\tt
%\begin{list}{}{}
%\item cd LAPACK
%\item make install
%\end{list} }
%
%\item Create the BLAS Library, {\em if necessary} \\
%{\em (NOTE:  For best performance, it is recommended you use the manufacturers' BLAS)}
%{\tt
%\begin{list}{}{}
%\item {\tt cd LAPACK}
%\item {\tt make blaslib}
%\end{list} }
%
%\item Run the Level 1, 2, and 3 BLAS Test Programs
%\begin{list}{}{}
%\item {\tt cd LAPACK}
%\item {\tt make blas\_testing}
%\end{list}
%
%\item Create the LAPACK Library
%\begin{list}{}{}
%\item {\tt cd LAPACK}
%\item {\tt make lapacklib}
%\end{list}
%
%\item Create the Library of Test Matrix Generators
%\begin{list}{}{}
%\item {\tt cd LAPACK}
%\item {\tt make tmglib}
%\end{list}
%
%\item Run the LAPACK Test Programs
%\begin{list}{}{}
%\item {\tt cd LAPACK}
%\item {\tt make testing}
%\end{list}
%
%\item Run the LAPACK Timing Programs
%\begin{list}{}{}
%\item {\tt cd LAPACK}
%\item {\tt make timing}
%\end{list}
%
%\item Run the BLAS Timing Programs
%\begin{list}{}{}
%\item {\tt cd LAPACK}
%\item {\tt make blas\_timing}
%\end{list}
\end{enumerate}
 
\subsection{Read the Tape or Untar the File}
\dent
If you received a tar tape of LAPACK, type one of the following commands
to unload the tape
(the device name may be different at your site):

\begin{list}{}{}
\item{{\tt tar  xvf  /dev/rst0 }}    (cartridge tape),  or

\item{{\tt tar  xvf  /dev/rmt8 }}    (9-track tape)
\end{list}

Alternatively, if you received a tar file of LAPACK via xnetlib, anonymous
ftp, or the World Wide Web, enter the following two commands to untar the file:

\begin{list}{}
\item{{\tt uncompress {\em file} (where {\em file} is the name of the
compressed tar file)}}
\item{{\tt tar xvf {\em file} (where {\em file} is the name of the tar
file)}}
\end{list}

\noindent
This will create a top-level directory called {\tt LAPACK}, which
requires approximately 33 Mbytes of disk space.
The total space requirements including the object files and executables
is approximately 80 Mbytes for all four data types.

\subsection{Edit the file {\tt LAPACK/make.inc}}

\dent
Before the libraries can be built, or the testing and timing programs
run, you must define all machine-specific parameters for the
architecture to which you are installing LAPACK.  All machine-specific
parameters are contained in the file {\tt LAPACK/make.inc}.  First, you will
need to modify the {\tt PLAT} definition, which is appended to all
library names, to specify the architecture to which you are installing
LAPACK.  This features avoids confusion in library names when you are 
installing LAPACK on more than one architecture.  Next, you will need
to modify {\tt FORTRAN}, {\tt OPTS}, {\tt NOOPT}, {\tt LOADER},
{\tt LOADOPTS}, {\tt ARCH}, {\tt ARCHFLAGS}, and {\tt RANLIB} to specify 
the compiler, compiler options, loader, loader options, archiver, 
archiver options, and ranlib for your machine.  If your architecture
does not require {\tt ranlib} to be run after each archive command (as
is the case with CRAY computers running UNICOS, or Hewlett Packard
computers running HP-UX), set {\tt ranlib=echo}.  And finally, you must
modify the {\tt BLASLIB} definition to specify the BLAS library to which
you will be linking.  If an optimized version of the BLAS is available
on your machine, you are highly recommended to link to that library.
Otherwise, by default, {\tt BLASLIB} is set to the Fortran 77 version.

\subsection{Edit the file {\tt LAPACK/Makefile}}\label{toplevelmakefile}

\dent
This {\tt Makefile} can be modified to perform as much of the
installation process as the user desires.  Ideally, this is the ONLY
makefile the user must modify.  However, modification of lower-level
makefiles may be necessary if a specific routine needs to be compiled
with a different level of optimization.  

First, edit the definitions of {\tt blaslib}, {\tt lapacklib},
{\tt tmglib}, {\tt testing}, and {\tt timing} in the file {\tt LAPACK/Makefile}
to specify the data types desired.  For example,
if you only wish to compile the single precision real version of the
LAPACK library, you would modify the {\tt lapacklib} definition to be:

\begin{verbatim}
lapacklib:
        ( cd SRC; $(MAKE) single )
\end{verbatim}

Likewise, you could specify {\tt double, complex, or complex16} to
build the double precision real, single precision complex, or double
precision complex libraries, respectively.  By default, the presence of
no arguments following the {\tt make} command will result in the
building of all four data types.
The make command can be run more than once to add another
data type to the library if necessary.

If you are installing LAPACK on a Silicon Graphics machine, you must
modify the respective definitions of {\tt testing} and {\tt timing} to be
\begin{verbatim}
testing:
        ( cd TESTING; $(MAKE) -f Makefile.sgi )
\end{verbatim}
and
\begin{verbatim}
timing:
        ( cd TIMING; $(MAKE) -f Makefile.sgi )
\end{verbatim}
      
Next, if you will be using a locally available BLAS library, you will need
to remove {\tt blaslib} from the {\tt lib} definition.  And finally,
if you do not wish to build all of the libraries individually and
likewise run all of the testing and timing separately, you can
modify the {\tt all} definition to specify the amount of the
installation process that you want performed.  By default,
the {\tt all} definition is set to
\begin{verbatim}
all: install lib blas_testing testing timing blas_timing
\end{verbatim}
which will perform all phases of the installation
process -- testing of machine-dependent routines, building the libraries,
BLAS testing, LAPACK testing, LAPACK timing, and BLAS timing.

The entire installation process will then be performed by typing {\tt
make}.

Questions and/or comments can be directed to the
authors as described in Section~\ref{sendresults}.  If test failures
occur, please refer to the appropriate subsection in
Section~\ref{furtherdetails}.

If disk space is limited, I would suggest building each data type separately
and/or deleting all object files after building the libraries.  Likewise, all
testing and timing executables can be deleted after the testing and timing
process is completed.  The removal of all object files and executables
can be accomplished by the following:

\begin{list}{}{}
\item {\tt cd LAPACK}
\item {\tt make clean}
\end{list}

\section{Further Details of the Installation Process}\label{furtherdetails}

Alternatively, you can choose to run each of the phases of the
installation process separately.  The following sections give details
on how this may be achieved.

\subsection{Test and Install the Machine-Dependent Routines.}
\dent
There are five machine-dependent functions in the test and timing
package, at least three of which must be installed.  They are

\begin{tabbing}
MONOMO  \=  DOUBLE PRECYSION  \=  \kill
LSAME   \>  LOGICAL      \> Test if two characters are the same regardless of case \\
SLAMCH  \>  REAL  \> Determine machine-dependent parameters \\
DLAMCH  \>  DOUBLE PRECISION \> Determine machine-dependent parameters \\
SECOND  \>  REAL  \> Return time in seconds from a fixed starting time \\
DSECND  \>  DOUBLE PRECISION  \> Return time in seconds from a fixed starting time
\end{tabbing}

\noindent
If you are working only in single precision, you do not need to install
DLAMCH and DSECND, and if you are working only in double precision,
you do not need to install SLAMCH and SECOND.

These five subroutines are provided in {\tt LAPACK/INSTALL},
along with five test programs.
To compile the five test programs and run the tests, go to {\tt LAPACK} and
type {\tt make install}.  The test programs are called
{\tt testlsame, testslamch, testdlamch, testsecond,} and
{\tt testdsecnd}. 
If you do not wish to run all tests, you will need to modify the 
{\tt install} definition in the {\tt LAPACK/Makefile} to only include the
tests you wish to run.  Otherwise, all tests will be performed.
The expected results of each test program are described below.

\subsubsection{Installing LSAME}
\dent
LSAME is a logical function with two character parameters, A and B.
It returns .TRUE. if A and B are the same regardless of case, or .FALSE.
if they are different. 
For example, the expression

\begin{list}{}{} 
\item {\tt LSAME( UPLO, 'U' )}
\end{list}
\noindent 
is equivalent to
\begin{list}{}{} 
\item {\tt ( UPLO.EQ.'U' ).OR.( UPLO.EQ.'u' )}
\end{list} 
 
The test program in {\tt lsametst.f} tests all combinations of
the same character in upper and lower case for A and B, and two
cases where A and B are different characters.

Run the test program by typing {\tt testlsame}.
If LSAME works correctly, the only message you should see after the
execution of {\tt testlsame} is
\begin{verbatim}
 ASCII character set
 Tests completed
\end{verbatim}
The file {\tt lsame.f} is automatically copied to
{\tt LAPACK/BLAS/SRC/} and {\tt LAPACK/SRC/}. 
The function LSAME is needed by both the BLAS and LAPACK, so it is safer
to have it in both libraries as long as this does not cause trouble
in the link phase when both libraries are used.

\subsubsection{Installing SLAMCH and DLAMCH}
\dent
SLAMCH and DLAMCH are real functions with a single character parameter
that indicates the machine parameter to be returned.  The test 
program in {\tt slamchtst.f}
simply prints out the different values computed by SLAMCH,
so you need to know something about what the values should be. 
For example, the output of the test program executable {\tt testslamch}
for SLAMCH on a Sun SPARCstation is
\begin{verbatim}
 Epsilon                      =     5.96046E-08
 Safe minimum                 =     1.17549E-38
 Base                         =     2.00000
 Precision                    =     1.19209E-07
 Number of digits in mantissa =     24.0000
 Rounding mode                =     1.00000
 Minimum exponent             =    -125.000
 Underflow threshold          =     1.17549E-38
 Largest exponent             =     128.000
 Overflow threshold           =     3.40282E+38
 Reciprocal of safe minimum   =     8.50706E+37
\end{verbatim}
On a Cray machine, the safe minimum underflows its output
representation and the overflow threshold overflows its output
representation, so the safe minimum is printed as 0.00000 and overflow
is printed as R.  This is normal.
If you would prefer to print a representable number, you can modify
the test program to print SFMIN*100. and RMAX/100. for the safe
minimum and overflow thresholds.

Likewise, the test executable {\tt testdlamch} is run for DLAMCH.

The files {\tt slamch.f} and {\tt dlamch.f} are automatically copied to
to {\tt LAPACK/SRC/}.
If both tests were successful, go to Section~\ref{second}.

If SLAMCH (or DLAMCH) returns an invalid value, you will have to create
your own version of this function.  The following options are used in
LAPACK and must be set:

\begin{list}{}{}
\item {`B': }  Base of the machine
\item {`E': }  Epsilon (relative machine precision)
\item {`O': }  Overflow threshold
\item {`P': }  Precision = Epsilon*Base
\item {`S': }  Safe minimum (often same as underflow threshold)
\item {`U': }  Underflow threshold
\end{list}

Some people may be familiar with R1MACH (D1MACH), a primitive
routine for setting machine parameters in which the user must
comment out the appropriate assignment statements for the target
machine.  If a version of R1MACH is on hand, the assignments in
SLAMCH can be made to refer to R1MACH using the correspondence

\begin{list}{}{}
\item {SLAMCH( `U' )}  $=$ R1MACH( 1 )
\item {SLAMCH( `O' )}  $=$ R1MACH( 2 )
\item {SLAMCH( `E' )}  $=$ R1MACH( 3 )
\item {SLAMCH( `B' )}  $=$ R1MACH( 5 )
\end{list}

\noindent
The safe minimum returned by SLAMCH( 'S' ) is initially set to the
underflow value, but if $1/({\rm overflow}) \geq ({\rm underflow})$
it is recomputed as $(1/({\rm overflow})) * ( 1 + \varepsilon )$,
where $\varepsilon$ is the machine precision.

BE AWARE that the initial call to SLAMCH or DLAMCH is expensive.  
We suggest that installers run it once, save the results, and hard-code
the constants in the version they put in their library.

\subsubsection{Installing SECOND and DSECND}\label{second}
\dent
Both the timing routines and the test routines call SECOND
(DSECND), a real function with no arguments that returns the time
in seconds from some fixed starting time.
Our version of this routine returns only ``user time'', and
not ``user time $+$ system time''. 
The version of SECOND in {\tt second.f} calls ETIME, a Fortran library
routine available on some computer systems.
If ETIME is not available or a better local timing function exists,
you will have to provide the correct interface to SECOND and DSECND
on your machine.

The test program in {\tt secondtst.f}
performs a million operations using 5000 iterations of 
the SAXPY operation $y := y + \alpha x$ on a vector of length 100.
The total time and megaflops for this test is reported, then
the operation is repeated including a call to SECOND on each of
the 5000 iterations to determine the overhead due to calling SECOND.
The test program executable is called {\tt testsecond} (or {\tt testdsecnd}).
There is no single right answer, but the times
in seconds should be positive and the megaflop ratios should be 
appropriate for your machine.
The files {\tt second.f} and {\tt dsecnd.f} are automatically copied to
{\tt LAPACK/SRC/} for inclusion in the LAPACK library.

\subsection{Create the BLAS Library} 
\dent
Ideally, a highly optimized version of the BLAS library already
exists on your machine. 
In this case you can go directly to Section~\ref{testblas} to
make the BLAS test programs.  
You may already have a library containing some of the BLAS,
but not all (Level 1 and 2, but not Level 3, for example).
If so, you should use your local version of the BLAS wherever
possible.

\begin{itemize}
\item[a)]
Go to {\tt LAPACK} and edit the definition of {\tt blaslib} in the
file {\tt Makefile} to specify the data types desired, as in the example
in Section~\ref{toplevelmakefile}.

If you already have some of the BLAS, you will need to edit the file
{\tt LAPACK/BLAS/SRC/Makefile} to comment out the lines 
defining the BLAS you have.  

\item[b)]
Type {\tt make blaslib}.
The make command can be run more than once to add another
data type to the library if necessary.  
\end{itemize}

\noindent
The BLAS library is created in {\tt LAPACK/blas\_PLAT.a}, where {\tt
PLAT} is the user-defined architecture suffix specified in the file {\tt
LAPACK/make.inc}.

\subsection{Run the BLAS Test Programs}\label{testblas}

\dent
Test programs for the Level 1, 2, and 3 BLAS are in the directory 
{\tt LAPACK/BLAS/TESTING}.

To compile and run the Level 1, 2, and 3 BLAS test programs,
go to {\tt LAPACK} and type {\tt make blas\_testing}.  The executable
files are called {\tt xblat\_s}, {\tt xblat\_d}, {\tt xblat\_c}, and {\tt
xblat\_z}, where the \_ (underscore) is replaced by 1, 2, or 3,
depending upon the level of BLAS that it is testing.  All executable and
output files are created in {\tt LAPACK/BLAS/}.
For the Level 1 BLAS tests, the output file names are {\tt sblat1.out},
{\tt dblat1.out}, {\tt cblat1.out}, and {\tt zblat1.out}.  For the Level
2 and 3 BLAS, the name of the output file is indicated on the first line of the
input file and is currently defined to be {\tt SBLAT2.SUMM} for
the Level 2 REAL version, and {\tt SBLAT3.SUMM} for the Level 3 REAL
version, with similar names for the other data types.

If the tests using the supplied data files were completed successfully,
consider whether the tests were sufficiently thorough.
For example, on a machine with vector registers, at least one value
of $N$ greater than the length of the vector registers should be used;
otherwise, important parts of the compiled code may not be
exercised by the tests. 
If the tests were not successful, either because the program did not
finish or the test ratios did not pass the threshold, you will
probably have to find and correct the problem before continuing. 
If you have been testing a system-specific
BLAS library, try using the Fortran BLAS for the routines that
did not pass the tests.
For more details on the BLAS test programs, 
see \cite{BLAS2-test} and \cite{BLAS3-test}.

\subsection{Create the LAPACK Library}

\begin{itemize}
\item[a)]
Go to the directory {\tt LAPACK} and edit the definition of
{\tt lapacklib} in the file {\tt Makefile} to specify the data types desired,
as in the example in Section~\ref{toplevelmakefile}.

\item[b)]
Type {\tt make lapacklib}.
The make command can be run more than once to add another
data type to the library if necessary.  

\end{itemize}

\noindent
The LAPACK library is created in {\tt LAPACK/lapack\_PLAT.a}, where {\tt
PLAT} is the user-defined architecture suffix specified in the file {\tt
LAPACK/make.inc}.

\subsection{Create the Test Matrix Generator Library}

\begin{itemize}
\item[a)]
Go to the directory {\tt LAPACK} and edit the definition of {\tt tmglib}
in the file {\tt Makefile} to specify the data types desired, as in the
example in Section~\ref{toplevelmakefile}.

\item[b)]
Type {\tt make tmglib}.
The make command can be run more than once to add another
data type to the library if necessary.  

\end{itemize}

\noindent
The test matrix generator library is created in {\tt LAPACK/tmglib\_PLAT.a},
where {\tt PLAT} is the user-defined architecture suffix specified in the 
file {\tt LAPACK/make.inc}.

\subsection{Run the LAPACK Test Programs}

\dent
There are two distinct test programs for LAPACK routines
in each data type, one for the linear equation routines and
one for the eigensystem routines.
In each data type, there is one input file for testing the linear
equation routines and seventeen input files for testing the eigenvalue
routines.
The input files reside in {\tt LAPACK/TESTING}.
For more information on the test programs and how to modify the
input files, please refer to LAPACK Working Note 41~\cite{WN41}.
% see Section~\ref{moretesting}. 

If you do not wish to run each of the tests individually, you can
go to {\tt LAPACK}, edit the definition {\tt testing} in the file {\tt
Makefile} to specify the data types desired, and type {\tt make
testing}.  This will
compile and run the tests as described in sections~\ref{testlin} 
and ~\ref{testeig}.

If you are installing LAPACK on a Silicon Graphics machine, you must
modify the definition of {\tt testing} to be
\begin{verbatim}
testing:
        ( cd TESTING; $(MAKE) -f Makefile.sgi )
\end{verbatim}

\subsubsection{Testing the Linear Equations Routines}\label{testlin}

\begin{itemize}

\item[a)]
Go to {\tt LAPACK/TESTING/LIN} and type {\tt make} followed by the data types
desired.  The executable files are called {\tt xlintsts, xlintstc,
xlintstd}, or {\tt xlintstz} and are created in {\tt LAPACK/TESTING}.

\item[b)]
Go to {\tt LAPACK/TESTING} and run the tests for each data type.
For the REAL version, the command is
\begin{list}{}{}
\item{} {\tt xlintsts  < stest.in > stest.out}
\end{list}

\noindent
The tests using {\tt xlintstd}, {\tt xlintstc}, and {\tt xlintstz} are similar
with the leading `s' in the input and output file names replaced
by `d', `c', or `z'.

\end{itemize}

If you encountered failures in this phase of the testing process, please
refer to Section~\ref{sendresults}.

\subsubsection{Testing the Eigensystem Routines}\label{testeig}

\begin{itemize}

\item[a)]
Go to {\tt LAPACK/TESTING/EIG} and type {\tt make} followed by the data types
desired.  The executable files are called {\tt xeigtsts,
xeigtstc, xeigtstd}, and {\tt xeigtstz} and are created
in {\tt LAPACK/TESTING}.

\item[b)]
Go to {\tt LAPACK/TESTING} and run the tests for each data type.
The tests for the eigensystem routines use seventeen separate input files
for testing the nonsymmetric eigenvalue problem,
the symmetric eigenvalue problem, the banded symmetric eigenvalue
problem, the generalized symmetric eigenvalue
problem, the generalized nonsymmetric eigenvalue problem, the 
singular value decomposition, the banded singular value decomposition,
the generalized singular value
decomposition, the generalized QR and RQ factorizations, the generalized
linear regression model, and the constrained linear least squares
problem.
The tests for the REAL version are as follows:
\begin{list}{}{}
\item {\tt xeigtsts  < nep.in > snep.out}
\item {\tt xeigtsts  < sep.in > ssep.out}
\item {\tt xeigtsts  < svd.in > ssvd.out}
\item {\tt xeigtsts  < sec.in > sec.out}
\item {\tt xeigtsts  < sed.in > sed.out}
\item {\tt xeigtsts  < sgg.in > sgg.out}
\item {\tt xeigtsts  < ssg.in > ssg.out}
\item {\tt xeigtsts  < ssb.in > ssb.out}
\item {\tt xeigtsts  < sbb.in > sbb.out}
\item {\tt xeigtsts  < sbal.in > sbal.out}
\item {\tt xeigtsts  < sbak.in > sbak.out}
\item {\tt xeigtsts  < sgbal.in > sgbal.out}
\item {\tt xeigtsts  < sgbak.in > sgbak.out}
\item {\tt xeigtsts  < glm.in > sglm.out}
\item {\tt xeigtsts  < gqr.in > sgqr.out}
\item {\tt xeigtsts  < gsv.in > sgsv.out}
\item {\tt xeigtsts  < lse.in > slse.out}
\end{list}
The tests using {\tt xeigtstc}, {\tt xeigtstd}, and {\tt xeigtstz} also
use the input files {\tt nep.in}, {\tt sep.in}, {\tt svd.in},
{\tt glm.in}, {\tt gqr.in}, {\tt gsv.in}, and {\tt lse.in},
but the leading `s' in the other input file names must be changed
to `c', `d', or `z'.
\end{itemize}

If you encountered failures in this phase of the testing process, please
refer to Section~\ref{sendresults}.

\subsection{Run the LAPACK Timing Programs}

\dent
There are two distinct timing programs for LAPACK routines
in each data type, one for the linear equation routines and
one for the eigensystem routines.  The timing program for the
linear equation routines is also used to time the BLAS.
We encourage you to conduct these timing experiments
in REAL and COMPLEX or in DOUBLE PRECISION and COMPLEX*16; it is
not necessary to send timing results in all four data types.

Two sets of input files are provided, a small set and a large set.
The small data sets are appropriate for a standard workstation or
other non-vector machine.
The large data sets are appropriate for supercomputers, vector
computers, and high-performance workstations.
We are mainly interested in results from the large data sets, and
it is not necessary to run both the large and small sets.
The values of N in the large data sets are about five times larger
than those in the small data set,
and the large data sets use additional values for parameters such as the
block size NB and the leading array dimension LDA.
Small data sets are indicated by lower case names, such as
{\tt stime.in}, and large data sets are indicated by upper case names,
such as {\tt STIME.in}.
Except as noted, the leading `s' (or `S') in the input file name must be
replaced by `d', `c', or `z' (`D', `C', or `Z') for the other data types.

We encourage you to obtain timing results with the large data sets,
as this allows us to compare different machines.
If this would take too much time, suggestions for paring back the large
data sets are given in the instructions below.
We also encourage you to experiment with these timing
programs and send us any interesting results, such as results for
larger problems or for a wider range of block sizes.
The main programs are dimensioned for the large data sets,
so the parameters in the main program may have to be reduced in order
to run the small data sets on a small machine, or increased to run
experiments with larger problems.

The minimum time each subroutine will be timed is set to 0.0 in
the large data files and to 0.05 in the small data files, and on
many machines this value should be increased.
If the timing interval is not long
enough, the time for the subroutine after subtracting the overhead
may be very small or zero, resulting in megaflop rates that are
very large or zero. (To avoid division by zero, the megaflop rate is
set to zero if the time is less than or equal to zero.)
The minimum time that should be used depends on the machine and the
resolution of the clock.

For more information on the timing programs and how to modify the
input files, please refer to LAPACK Working Note 41~\cite{WN41}.
% see Section~\ref{moretiming}.

If you do not wish to run each of the timings individually, you can
go to {\tt LAPACK}, edit the definition {\tt timing} in the file {\tt
Makefile} to specify the data types desired, and type {\tt make
timing}.  This will compile
and run the timings for the linear equation routines and the eigensystem
routines (see Sections~\ref{timelin} and ~\ref{timeeig}). 

If you are installing LAPACK on a Silicon Graphics machine, you must
modify the definition of {\tt timing} to be
\begin{verbatim}
timing:
        ( cd TIMING; $(MAKE) -f Makefile.sgi )
\end{verbatim}

If you encounter failures in any phase of the timing process, please
feel free to contact the authors as directed in Section~\ref{sendresults}.
Tell us the 
type of machine on which the tests were run, the version of the operating
system, the compiler and compiler options that were used,
and details of the BLAS library or libraries that you used.  You should
also include a copy of the output file in which the failure occurs.

Please note that the BLAS
timing runs will still need to be run as instructed in ~\ref{timeblas}.

\subsubsection{Timing the Linear Equations Routines}\label{timelin}

\dent
The linear equation timing program is found in {\tt LAPACK/TIMING/LIN}
and the input files are in {\tt LAPACK/TIMING}.
Three input files are provided in each data type for timing the
linear equation routines, one for square matrices, one for band
matrices, and one for rectangular matrices.  The small data sets for the REAL version
are {\tt stime.in}, {\tt sband.in}, and {\tt stime2.in}, respectively,
and the large data sets are
{\tt STIME.in}, {\tt SBAND.in}, and {\tt STIME2.in}.

\begin{itemize}
\item[a)]
To make the linear equation timing programs,
go to {\tt LAPACK/TIMING/LIN} and type {\tt make} followed by the data
types desired, as in the examples in Section~\ref{toplevelmakefile}.
The executable files are called {\tt xlintims},
{\tt xlintimc}, {\tt xlintimd}, and {\tt xlintimz} and are created
in {\tt LAPACK/TIMING}.

\item[b)]
Go to {\tt LAPACK/TIMING} and
make any necessary modifications to the input files.
You may need to set the minimum time a subroutine will
be timed to a positive value, or to restrict the size of the tests
if you are using a computer with performance in between that of a
workstation and that of a supercomputer.
The computational requirements can be cut in half by using only one
value of LDA.
If it is necessary to also reduce the matrix sizes or the values of
the blocksize, corresponding changes should be made to the 
BLAS input files (see Section~\ref{timeblas}).

\item[c)]
Run the programs for each data type you are using. 
For the REAL version, the commands for the small data sets are

\begin{list}{}{}
\item{} {\tt xlintims < stime.in > stime.out }
\item{} {\tt xlintims < sband.in > sband.out }
\item{} {\tt xlintims < stime2.in > stime2.out }
\end{list}
or the commands for the large data sets are
\begin{list}{}{}
\item{} {\tt xlintims < STIME.in > STIME.out }
\item{} {\tt xlintims < SBAND.in > SBAND.out }
\item{} {\tt xlintims < STIME2.in > STIME2.out }
\end{list}

\noindent
Similar commands should be used for the other data types.
\end{itemize}

\subsubsection{Timing the BLAS}\label{timeblas}

\dent
The linear equation timing program is also used to time the BLAS.
Three input files are provided in each data type for timing the Level
2 and 3 BLAS. 
These input files time the BLAS using the matrix shapes encountered
in the LAPACK routines, and we will use the results to analyze the
performance of the LAPACK routines. 
For the REAL version, the small data files are
{\tt sblasa.in}, {\tt sblasb.in}, and {\tt sblasc.in}
and the large data files are
{\tt SBLASA.in}, {\tt SBLASB.in}, and {\tt SBLASC.in}.
There are three sets of inputs because there are three
parameters in the Level 3 BLAS, M, N, and K, and
in most applications one of these parameters is small (on the order
of the blocksize) while the other two are large (on the order of the
matrix size).  
In {\tt sblasa.in}, M and N are large but K is
small, while in {\tt sblasb.in} the small parameter is M, and
in {\tt sblasc.in} the small parameter is N.  
The Level 2 BLAS are timed only in the first data set, where K
is also used as the bandwidth for the banded routines.

\begin{itemize}

\item[a)]
Go to {\tt LAPACK/TIMING} and
make any necessary modifications to the input files.
You may need to set the minimum time a subroutine will
be timed to a positive value.
If you modified the values of N or NB 
in Section~\ref{timelin}, set M, N, and K accordingly.
The large parameters among M, N, and K
should be the same as the matrix sizes used in timing the linear
equation routines,
and the small parameter should be the same as the
blocksizes used in timing the linear equation routines.
If necessary, the large data set can be simplified by using only one
value of LDA.

\item[b)]
Run the programs for each data type you are using. 
For the REAL version, the commands for the small data sets are

\begin{list}{}{}
\item{} {\tt xlintims < sblasa.in > sblasa.out }
\item{} {\tt xlintims < sblasb.in > sblasb.out }
\item{} {\tt xlintims < sblasc.in > sblasc.out }
\end{list}
or the commands for the large data sets are
\begin{list}{}{}
\item{} {\tt xlintims < SBLASA.in > SBLASA.out }
\item{} {\tt xlintims < SBLASB.in > SBLASB.out }
\item{} {\tt xlintims < SBLASC.in > SBLASC.out }
\end{list}

\noindent
Similar commands should be used for the other data types.
\end{itemize}

\subsubsection{Timing the Eigensystem Routines}\label{timeeig}

\dent
The eigensystem timing program is found in {\tt LAPACK/TIMING/EIG}
and the input files are in {\tt LAPACK/TIMING}.
Four input files are provided in each data type for timing the
eigensystem routines,
one for the generalized nonsymmetric eigenvalue problem, 
one for the nonsymmetric eigenvalue problem, 
one for the symmetric and generalized symmetric eigenvalue problem,
and one for the singular value decomposition.
For the REAL version, the small data sets are called {\tt sgeptim.in},
{\tt sneptim.in}, {\tt sseptim.in}, and {\tt ssvdtim.in}, respectively.
and the large data sets are called {\tt SGEPTIM.in}, {\tt SNEPTIM.in},
{\tt SSEPTIM.in}, and {\tt SSVDTIM.in}.
Each of the four input files reads a different set of parameters,
and the format of the input is indicated by a 3-character code
on the first line.

The timing program for eigenvalue/singular value routines accumulates
the operation count as the routines are executing using special
instrumented versions of the LAPACK routines.  The first step in
compiling the timing program is therefore to make a library of the
instrumented routines.

\begin{itemize}
\item[a)]
\begin{sloppypar}
To make a library of the instrumented LAPACK routines, first
go to {\tt LAPACK/TIMING/EIG/EIGSRC} and type {\tt make} followed
by the data types desired, as in the examples of Section~\ref{toplevelmakefile}. 
The library of instrumented code is created in
{\tt LAPACK/TIMING/EIG/eigsrc\_PLAT.a},
where {\tt PLAT} is the user-defined architecture suffix specified in the
file {\tt LAPACK/make.inc}.
\end{sloppypar}

\item[b)]
To make the eigensystem timing programs, 
go to {\tt LAPACK/TIMING/EIG} and
type {\tt make} followed by the data types desired, as in the examples
of Section~\ref{toplevelmakefile}.  The executable files are called
{\tt xeigtims}, {\tt xeigtimc}, {\tt xeigtimd}, and {\tt xeigtimz}
and are created in {\tt LAPACK/TIMING}.

\item[c)]
Go to {\tt LAPACK/TIMING} and
make any necessary modifications to the input files.
You may need to set the minimum time a subroutine will
be timed to a positive value, or to restrict the number of tests
if you are using a computer with performance in between that of a
workstation and that of a supercomputer.
Instead of decreasing the matrix dimensions to reduce the time,
it would be better to reduce the number of matrix types to be timed,
since the performance varies more with the matrix size than with the
type.  For example, for the nonsymmetric eigenvalue routines,
you could use only one matrix of type 4 instead of four matrices of
types 1, 3, 4, and 6.
Refer to LAPACK Working Note 41~\cite{WN41} for further details.
%  See Section~\ref{moretiming} for further details.

\item[d)]
Run the programs for each data type you are using. 
For the REAL version, the commands for the small data sets are

\begin{list}{}{}
\item{} {\tt xeigtims < sgeptim.in > sgeptim.out }
\item{} {\tt xeigtims < sneptim.in > sneptim.out }
\item{} {\tt xeigtims < sseptim.in > sseptim.out }
\item{} {\tt xeigtims < ssvdtim.in > ssvdtim.out }
\end{list}
or the commands for the large data sets are
\begin{list}{}{}
\item{} {\tt xeigtims < SGEPTIM.in > SGEPTIM.out }
\item{} {\tt xeigtims < SNEPTIM.in > SNEPTIM.out }
\item{} {\tt xeigtims < SSEPTIM.in > SSEPTIM.out }
\item{} {\tt xeigtims < SSVDTIM.in > SSVDTIM.out }
\end{list}

\noindent
Similar commands should be used for the other data types.
\end{itemize}

\subsection{Send the Results to Tennessee}\label{sendresults}

\dent
Congratulations!  You have now finished installing, testing, and
timing LAPACK.  If you encountered failures in any phase of the 
testing or timing process, please
consult our release\_notes file on netlib (send email to netlib@ornl.gov
and in the message type "send release\_notes from lapack").  This file
contains machine-dependent installation clues which hopefully will 
alleviate your difficulties or at least let you know that other users
have had similar difficulties on that machine.  If there is not an entry
for your machine or the suggestions do not fix your problem, please feel
free to contact the authors at
\begin{list}{}{}
\item {\tt lapack@cs.utk.edu}.
\end{list}
Tell us the 
type of machine on which the tests were run, the version of the operating
system, the compiler and compiler options that were used,
and details of the BLAS library or libraries that you used.  You should
also include a copy of the output file in which the failure occurs.

We would like to keep our release\_notes file as up-to-date as possible.
Therefore, if you do not see an entry for your machine, please contact us
with your testing results.

Comments and suggestions are also welcome.

We encourage you to make the LAPACK library available to your
users and provide us with feedback from their experiences.
This release of LAPACK is not guaranteed to be compatible
with any previous test release.

\section*{Acknowledgments}

Ed Anderson of Cray Research Inc. contributed to previous versions of
this report.

\newpage
\addcontentsline{toc}{section}{Bibliography}
\biblio{9}

\bibitem{LUG}
E. Anderson, Z. Bai, C. Bischof, J. Demmel, J. Dongarra,
J. Du Croz, A. Greenbaum, S. Hammarling, A. McKenney,
S. Ostrouchov, and D. Sorensen, 
{\em LAPACK Users' Guide}, Second Edition,
{SIAM}, Philadelphia, PA, 1994.

\bibitem{WN16}
E. Anderson and J. Dongarra,
{\em LAPACK Working Note 16: 
Results from the Initial Release of LAPACK},
University of Tennessee, CS-89-89, November 1989.

\bibitem{WN41}
E. Anderson, J. Dongarra, and S. Ostrouchov,
{\em LAPACK Working Note 41: 
Installation Guide for LAPACK},
University of Tennessee, CS-92-151, February 1992 (revised October 1994).

\bibitem{WN5}
C. Bischof, J. Demmel, J. Dongarra, J. Du Croz, A. Greenbaum,
S. Hammarling, and D. Sorensen,
{\em LAPACK Working Note \#5:  Provisional Contents},
Argonne National Laboratory, ANL-88-38, September 1988.

\bibitem{WN13}
Z. Bai, J. Demmel, and A. McKenney,
{\em LAPACK Working Note \#13: On the Conditioning of the Nonsymmetric
Eigenvalue Problem:  Theory and Software}, 
University of Tennessee, CS-89-86, October 1989.
 
\bibitem{BLAS3}
J. Dongarra, J. Du Croz, I. Duff, and S. Hammarling,
``A Set of Level 3 Basic Linear Algebra Subprograms,''
{\em ACM Trans. Math. Soft.}, 16, 1:1-17, March 1990
%Argonne National Laboratory, ANL-MCS-P88-1, August 1988.

\bibitem{BLAS3-test}
J. Dongarra, J. Du Croz, I. Duff, and S. Hammarling,
``A Set of Level 3 Basic Linear Algebra Subprograms:
Model Implementation and Test Programs,''
{\em ACM Trans. Math. Soft.}, 16, 1:18-28, March 1990
%Argonne National Laboratory, ANL-MCS-TM-119, June 1988.

\bibitem{BLAS2}
J. Dongarra, J. Du Croz, S. Hammarling, and R. Hanson,
``An Extended Set of Fortran Basic Linear Algebra Subprograms,''
{\em ACM Trans. Math. Soft.}, 14, 1:1-17, March 1988.

\bibitem{BLAS2-test}
J. Dongarra, J. Du Croz, S. Hammarling, and R. Hanson,
``An Extended Set of Fortran Basic Linear Algebra Subprograms:
Model Implementation and Test Programs,''
{\em ACM Trans. Math. Soft.}, 14, 1:18-32, March 1988.

\bibitem{BLAS1}
C. L. Lawson, R. J. Hanson, D. R. Kincaid, and F. T. Krogh,
``Basic Linear Algebra Subprograms for Fortran Usage,''
{\em ACM Trans. Math. Soft.}, 5, 3:308-323, September 1979.

\bibend

\end{document}
