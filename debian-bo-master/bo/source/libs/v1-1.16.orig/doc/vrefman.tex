%\subsection* {Constructor} %------------------------------------
%\subsection* {Methods to Override} %----------------------------
%\subsection* {Utility Methods} %--------------------------------
%\subsection* {Other Methods} %----------------------------------
%\subsection* {Inherited Methods} %------------------------------
%
%
\documentclass[letterpaper,draft]{book}
% Try to fill up the page!
% This was from another example - we will ignore it for now
\normalsize%\setlength{\voffset}{-.5in}
\setlength{\topmargin}{0in}
\setlength{\oddsidemargin}{1in}
\setlength{\evensidemargin}{.05in}
\setlength{\marginparwidth}{0in}
\addtolength{\textheight}{1in}
\addtolength{\textwidth}{.7in}
% \code is a piece of code
\newcommand{\code}{\texttt}
% \V is V
\newcommand{\V}{\emph{\textbf{V}}}
% special math chars I need: -> ~ ^
\newcommand{\rta}{$\rightarrow$\ }
\newcommand{\tild}{\~{ }}
%\newcommand{\vrbsize}{\small}
\newcommand{\verbfont}{\footnotesize}
\newcommand{\regfont}{\normalfont\normalsize}
% \Class is used to start a new class description at page begin

%----old \newcommand{\Class}[1]{\newpage \markboth{#1}{#1} \section* {#1}}
\newcommand{\Class}[1]{\section {#1}}

%\Indextt for tt index

\newcommand{\Indextt}[1]{\index{#1@\texttt{#1}}}
% \Meth is for a method
\newcommand{\Meth}[1]{\subsubsection* {#1}}

% \Cmd is for a dialog command
\newcommand{\Cmd}[1]{\subsubsection* {#1}}

% \Param is a parameter description
\newcommand{\Param}[1]{\subparagraph {#1}}

% kill off the figfont
\newcommand{\SetFigFont}[3]{}%

% \Sect is used to set typeface of chapter and section titles
\newcommand{\Sect}[1]{\emph{#1}}

% Method environment that allows parameters to be entered with
% the \Param command between the \begin and \end.
\newenvironment{Method}[1]{\Meth{#1}
\begin{description}
\begin{description}}
{\end{description}
\end{description}}

% An environment for a definition
\newenvironment{Def}[1]{\begin{description} \item[#1]}{\end{description}}

\newenvironment{Point}{\begin{itemize} \item }{\end{itemize}}

%\usepackage{verbatim}
%\usepackage{eepic}
%\usepackage{pictex}
\usepackage{makeidx}

../texinputs/html.sty
%--------
\makeatletter
\def\@evenhead{
	\textbf{\textrm{\thepage \hfill \leftmark}}}
\def\@oddhead{
	\textbf{\textrm{\rightmark \hfill \thepage}}}
\makeatother

\renewcommand{\chaptermark}[1]{\markboth{#1}{#1}}
\renewcommand{\sectionmark}[1]{\markright{\thesection~#1}}

\newfont{\REALLYHUGE}{cminch}
\makeindex
%==========================================================================
\begin{document}

\vspace*{1in} \thispagestyle{empty}
\begin{center}

\REALLYHUGE V \Huge \\ ~ \\ \textbf{A C++ GUI \\ Framework}

\vspace{1in}

\LARGE Bruce E. Wampler, Ph.D. \\
\large ~ \\
 \emph{P.O. Box 140} \\ 
 \emph{35 High Country Dr.} \\ 
 \emph{Cedar Crest, NM 87008} \\ ~ \\
 \texttt{wampler@cs.unm.edu} \\
 \texttt{http://www.cs.unm.edu/\~{}wampler}

\vspace{1in}
\normalsize
Version 1.16 \\
January 20, 1997
\end{center}

\newpage
\vspace*{1in} \thispagestyle{empty}

\textbf{V - A C++ GUI Framework}

Copyright \copyright 1995,1996,1997, Bruce E. Wampler

All rights reserved.

\texttt{wampler@cs.unm.edu}

\texttt{http://www.cs.unm.edu/\~{}wampler}

\vspace{1in}

\emph{V - A C++ GUI Framework}, Version 1.16,
may be reproduced and distributed, in whole or in part, subject
to the following conditions:

\begin{enumerate}
\item The copyright notice above and this permission notice must
be preserved complete on all complete or partial copies.

\item You may not translate or create a derivative of this work
without the author's written permission.

\item If you distribute this manual in part, you must provide
instructions and a means for obtaining a complete version.

\item You may make a profit on copies of this work only if it
is included as part of an electronic distribution of other free software
works (e.g., Linux or GNU).

\item Small portions may be reproduced as illustrations for reviews
or quotations in other works without this permission notice if proper
citation is given.

\end{enumerate}

My goal is to get as many people as can be helped using \V\@. If
the terms of this documentation copyright are unsatisfactory, please
contact me and we can probably work something out.

The complete source for \V\ (X Athena and Microsoft Windows versions)
is available via anonymous ftp at \code{ftp.cs.unm.edu/pub/wampler}, or
via my www home page (see above). This manual
is also available as an html document on my web page.

I would like to acknowledge the Computer Science Department of the
University of New Mexico for providing e-mail, ftp, and WWW service.

\pagenumbering{roman} \setcounter{page}{2} \tableofcontents

\chapter {What is V?}

\pagenumbering{arabic} \setcounter{page}{1}

\V\ is a C++ Graphical User Interface Framework designed
to provide an easy to use and program system for building GUI
applications. The framework is small, elegant, and provides the
tools required for building all but the most specialized
applications.

The \V\ framework has also been designed to be portable.
Currently, versions for the X Windowing System (using a
customized 3D Athena widget set), Microsoft Windows 3.1, and
Microsoft WIN32 (Windows 95 and NT) are available. A version for
OS/2 is in progress. A Macintosh port is under investigation. The
\V\ system is freely available for use by anyone under the terms
of the GNU Library General Public License.

Why did I write \V, and why did I put it under the GNU license? I
have been programming for over 20 years now, and building
interactive applications for most of that time. During that time,
I got tired of complicated, difficult to learn and use libraries
for building interfaces, and wanted something easier.

I've also been successful in the software business, having founded
two different software companies, Aspen Software and Reference
Software International. I was the principle designer and author
of the widely known and used grammar checker,
Grammatik\footnote{Grammatik is a trademark of Novell, Inc.}.
Basically, I see \V\ as something of a public service; a way
to give something back to the software industry that has been
good to me. The concept of a portable GUI library is not
original, but I think some of the design goals of \V\ are
significantly different than other similar libraries I've seen.

\begin{itemize}

\item The main design goal is for for \V\ is ease of programming.
I don't think that building the GUI part of an application should
be the hardest part of the job as it is with most native
GUI toolkits. \V\ is
small, easy to learn, easy to use, and provides the essentials of
a good graphical user interface.

I have some evidence that I have succeeded in this goal. \V\ has
been used for several semesters for large team projects in the
software engineering class I taught at the University of New
Mexico.  While I get many questions from my students related to
the projects they are doing, I got virtually no questions about
using \V\ itself. The small number of questions about \V\ has
been both startling and rewarding, and is good evidence that this
design goal has been met. \V\ has also been used successfully
for a Junior level programming class. Previously, the high
overhead of learning to write applications for X has prevented
the students from writing small programs with interesting user
interfaces. The simplicity of \V\ has allowed them to do this for
the first time.

\item \V\ is designed to be portable. Over the years, I've programmed
on a wide variety of interactive platforms. The main GUI
platforms widely used today include the X Window System,
Microsoft Windows (3.1, 95, and NT), OS/2, and the Macintosh. \V\
has been designed to work on all those platforms, and present a
look and feel that is consistent with native applications.

\item \V\ is not too big. It has less than 15 C++ classes that you
will have to interact with. This is unlike many other frameworks
that provide dozens and dozens of classes that you must learn and
understand. The \V\ framework only supports GUIs. It does not
have templates, containers, and bunches of other C++ classes. If
you need a good list class, use your favorite one from another
class library. Use \V\ for your interface.

\item \V\ has very good associated documentation. It is likely that
part of the reason that \V\ is easy to use is that it is accompanied
by a better than average programming manual. I've tried to not only
give a useful explanation of each \V\ class and function, but to
accompany each description with a short example that shows how to
use the \V\ feature in a useful way. There are also several examples
provided with the \V\ distribution to help you get started with a
basic \V\ application.

\item \V\ is an alternative for building \emph{compiled} GUI applications.
While interpretive solutions such as Tk/Tcl for building GUI
applications are becoming popular, they don't allow fully
compiled code on multiple platforms. As machines get faster and
faster, I don't think the advantage of an edit/interpret cycle
versus an edit/compile/run cycle is significant.

\item The \V\ library is free software; you can redistribute it and/or
modify it under the terms of the GNU Library General Public
License as published by the Free Software Foundation; either
version 2 of the License, or any later version.

\V\ is distributed in the hope that it will be useful,
but \emph{without any warranty}; without even the implied
warranty of \emph{merchantability} or \emph{fitness for a
particular purpose}.  See the GNU Library General Public License
for more details.

\item The source code for \V\ is of commercial quality, and I hope
some of the easiest to read and understand code you will ever
encounter (if you decide to look at the \V\ source code).

\end{itemize}

There is, of course, a price to pay for the ease of programming
with \V\@. The main constraint is that you are somewhat
restricted to following \V's (and thus my own) view of the
world. The \V\ model does not exactly conform to the native
models of X, Windows, and the Mac, but it is a very good
compromise.  For the most part applications developed with \V\
will in fact conform to the host look and feel, but may be
lacking some of the bells and whistles of the most sophisticated
commercial applications available for a given platform. For the
vast majority of applications, this will not matter. You will end
up with applications that look pretty good, and are likely to
have a much cleaner and better interface than they would have
otherwise.

If you are a C programmer, then the fact \V\ is a C++ library
might be a problem. While it is a fully object-oriented C++
framework, it can be used with C code if you know a bit about
C++. Also, \V\ does not allow you to do everything you could if
you programmed in the native windowing library. You won't have
every single conceivable control, and some controls are slightly
restricted in how you can use them.

And finally, why the name \V\@? \index{Why named V?}\index{name V}
First of all, it is a simple
name. It follows the tradition of \emph{C} and \emph{X}. It makes
naming the classes easier. And, my son's name is Van, which
starts with a V. So \V\ it is.

%***********************************************************************
%***********************************************************************
%***********************************************************************
\chapter{The V View of the World}

Before getting into the details of \V, you might find it useful to
read this overview of how the \V\ view of the world was developed.
If you are new at writing GUI applications, you should find this
chapter especially useful.

\section{A Generalized GUI Model}

\index{generalized GUI model}\index{GUI}
If you examine a large number of applications available on the major
GUI platforms, you will find the interfaces typically have a
great deal in common. While the visual details may differ,
most applications have windows that show views of the data being
manipulated, and use menus and dialogs for control interaction
with the user. The user interacts with the program using a
pointing device, usually a mouse, and the keyboard.

\subsection*{Windows}
\index{window}
\begin{rawhtml}
<IMG BORDER=0 ALIGN=BOTTOM ALT="" SRC="../fig/protoapp.gif">
\end{rawhtml}

\begin{figure}[htb]

\begin{center}
\small

\begin{latexonly}
\setlength{\unitlength}{0.012500in}%
\begin{picture}(330,245)(50,550)
\thicklines
\put( 50,580){\framebox(330,215){}}
\put( 50,775){\line( 1, 0){330}}
\put( 50,755){\line( 1, 0){330}}
\put( 60,770){\framebox(0,0){}}
\put(150,755){\line( 0,-1){ 65}}
\put(150,690){\line( 1, 0){ 90}}
\put(240,690){\line( 0, 1){ 65}}
\put( 50,610){\line( 1, 0){330}}
\put( 50,630){\line( 1, 0){310}}
\put(360,630){\line( 0, 1){125}}
\put(360,630){\line( 0,-1){ 20}}
\put(360,630){\line( 1, 0){ 20}}
\put(125,630){\line( 0,-1){ 20}}
\put(210,630){\line( 0,-1){ 20}}
\put(360,710){\line( 1, 0){ 20}}
\put(360,670){\line( 1, 0){ 20}}
\put(125,625){\line( 1, 0){ 85}}
\put(125,620){\line( 1, 0){ 85}}
\put(125,615){\line( 1, 0){ 85}}
\put(360,705){\line( 1, 0){ 20}}
\put(360,700){\line( 1, 0){ 20}}
\put(360,695){\line( 1, 0){ 20}}
\put(360,690){\makebox(0.4444,0.6667){\SetFigFont{10}{12}{rm}.}}
\put(360,690){\line( 1, 0){ 20}}
\put(360,685){\line( 1, 0){ 20}}
\put(360,680){\line( 1, 0){ 20}}
\put(360,675){\line( 1, 0){ 20}}
\put(130,630){\line( 0,-1){ 20}}
\put(135,630){\line( 0,-1){ 20}}
\put(140,630){\line( 0,-1){ 20}}
\put(145,630){\line( 0,-1){ 20}}
\put(150,630){\line( 0,-1){ 20}}
\put(155,630){\line( 0,-1){ 20}}
\put(160,630){\line( 0,-1){ 20}}
\put(165,630){\line( 0,-1){ 20}}
\put(170,630){\line( 0,-1){ 20}}
\put(175,630){\line( 0,-1){ 20}}
\put(180,630){\line( 0,-1){ 20}}
\put(185,630){\line( 0,-1){ 20}}
\put(190,630){\line( 0,-1){ 20}}
\put(195,630){\line( 0,-1){ 20}}
\put(200,630){\line( 0,-1){ 20}}
\put(205,630){\line( 0,-1){ 20}}
\put(365,710){\line( 0,-1){ 40}}
\put(370,710){\line( 0,-1){ 40}}
\put(375,710){\line( 0,-1){ 40}}
\put(180,660){\line( 1, 1){ 20}}
\put(200,680){\line( 1,-1){ 20}}
\put(220,660){\line( 1, 1){ 20}}
\put(240,680){\line( 1,-1){ 20}}
\put(260,660){\line( 1, 1){ 40}}
\put(300,700){\line( 0,-1){ 40}}
\put(300,660){\line(-1, 0){ 40}}
\put( 80,720){\line( 0,-1){ 60}}
\put( 80,660){\line( 1, 0){ 60}}
\put(140,660){\line(-1, 1){ 60}}
\put( 80,720){\line(-1,-1){ 30}}
\put(300,700){\line( 0, 1){ 20}}
\put(300,720){\line( 1,-1){ 20}}
\put(320,700){\line(-1, 0){ 20}}
\put(155,585){\framebox(75,20){}}
\put(235,585){\framebox(35,20){}}
\put(275,585){\framebox(40,20){}}
\put( 55,585){\framebox(45,20){}}
\put(105,585){\framebox(45,20){}}
\multiput( 70,765)(-0.33333,0.44444){10}{\makebox(0.4444,0.6667){\SetFigFont{7}{8.4}{rm}.}}
\put( 67,769){\line(-1, 0){  4}}
\multiput( 63,769)(-0.33333,-0.44444){10}{\makebox(0.4444,0.6667){\SetFigFont{7}{8.4}{rm}.}}
\multiput( 60,765)(0.33333,-0.44444){10}{\makebox(0.4444,0.6667){\SetFigFont{7}{8.4}{rm}.}}
\put( 63,761){\line( 1, 0){  4}}
\multiput( 67,761)(0.33333,0.44444){10}{\makebox(0.4444,0.6667){\SetFigFont{7}{8.4}{rm}.}}
\put( 60,760){\framebox(10,10){}}
\put( 50,580){\line( 0,-1){ 30}}
\put( 50,550){\line( 1, 0){330}}
\put(380,550){\line( 0, 1){ 30}}
\put(380,580){\line( 0, 1){  5}}
\put(160,555){\framebox(20,20){}}
\put(270,555){\framebox(25,20){}}
\put( 80,760){\makebox(0,0)[lb]{\smash{\SetFigFont{12}{14.4}{rm}File}}}
\put(150,760){\makebox(0,0)[lb]{\smash{\SetFigFont{12}{14.4}{rm}Test}}}
\put(160,740){\makebox(0,0)[lb]{\smash{\SetFigFont{12}{14.4}{rm}CheckMe}}}
\put(160,725){\makebox(0,0)[lb]{\smash{\SetFigFont{12}{14.4}{rm}Copy Sensitive}}}
\put(160,710){\makebox(0,0)[lb]{\smash{\SetFigFont{12}{14.4}{rm}Dialog}}}
\put(160,695){\makebox(0,0)[lb]{\smash{\SetFigFont{12}{14.4}{rm}Modal Dialog}}}
\put(115,760){\makebox(0,0)[lb]{\smash{\SetFigFont{12}{14.4}{rm}Edit}}}
\put(155,780){\makebox(0,0)[lb]{\smash{\SetFigFont{12}{14.4}{rm}Prototype V Example}}}
\put( 65,590){\makebox(0,0)[lb]{\smash{\SetFigFont{12}{14.4}{rm}Copy}}}
\put(110,590){\makebox(0,0)[lb]{\smash{\SetFigFont{12}{14.4}{rm}Dialog}}}
\put(165,590){\makebox(0,0)[lb]{\smash{\SetFigFont{12}{14.4}{rm}TestButton}}}
\put(240,590){\makebox(0,0)[lb]{\smash{\SetFigFont{12}{14.4}{rm}Clear}}}
\put(285,590){\makebox(0,0)[lb]{\smash{\SetFigFont{12}{14.4}{rm}Exit}}}
\put(120,780){\makebox(0,0)[lb]{\smash{\SetFigFont{10}{12.0}{rm}(1)}}}
\put( 55,745){\makebox(0,0)[lb]{\smash{\SetFigFont{10}{12.0}{rm}(2)}}}
\put(180,760){\makebox(0,0)[lb]{\smash{\SetFigFont{10}{12.0}{rm}(3)}}}
\put(245,720){\makebox(0,0)[lb]{\smash{\SetFigFont{10}{12.0}{rm}(4)}}}
\put(160,635){\makebox(0,0)[lb]{\smash{\SetFigFont{10}{12.0}{rm}(5)}}}
\put(340,680){\makebox(0,0)[lb]{\smash{\SetFigFont{10}{12.0}{rm}(5)}}}
\put(320,590){\makebox(0,0)[lb]{\smash{\SetFigFont{10}{12.0}{rm}(7)}}}
\put( 90,680){\makebox(0,0)[lb]{\smash{\SetFigFont{10}{12.0}{rm}(6)}}}
\put( 60,560){\makebox(0,0)[lb]{\smash{\SetFigFont{12}{14.4}{rm}Commands issued:}}}
\put(165,560){\makebox(0,0)[lb]{\smash{\SetFigFont{12}{14.4}{rm}12}}}
\put(190,560){\makebox(0,0)[lb]{\smash{\SetFigFont{12}{14.4}{rm}Last keypress:}}}
\put(275,560){\makebox(0,0)[lb]{\smash{\SetFigFont{12}{14.4}{rm}X}}}
\put(310,560){\makebox(0,0)[lb]{\smash{\SetFigFont{10}{12.0}{rm}(8)}}}
\end{picture}

\end{latexonly}

\normalfont\normalsize

\end{center}

\caption{This top level consists of: (1)~The title bar. (2)~The close
button. (3)~The menu bar. (4)~A pulldown menu. (5)~Vertical and horizontal
scroll bars. (6)~ The drawing
canvas. (7)~The command bar. (8)~The status bar.} \label{fig:protoapp}
\end{figure}

The \emph{window} is usually the main interface object used by an
application. The data being manipulated by the user (e.g.,
text, graphics, spreadsheet cells) is displayed in the window. Often,
several windows may be open at the same time, each giving
a different view of the data. There is usually a menu associated
with the window for entering commands to manipulate data or
to bring up dialogs.

The top level interface object
used by \V\ is a \emph{Command Window}. Each command window
consists of a \emph{menu bar}, placed at the
top of the window;  a \emph{canvas}
area, used to draw text and graphics to display the data; 
and optional \emph{command bars},
which include commands buttons and objects; and optional
\emph{status bars} to display state information.

Figure~\ref{fig:protoapp} represents, more or less, a typical
top-level \V\ window. 

\subsection*{Dialog Boxes}
\index{dialog}

\begin{rawhtml}
<IMG BORDER=0 ALIGN=BOTTOM ALT="" SRC="../fig/mydialog.gif">
\end{rawhtml}

\begin{figure}[htb]
\begin{center}
\small

\begin{latexonly}
\setlength{\unitlength}{0.012500in}%
\begin{picture}(315,170)(60,630)
\thicklines
\put( 75,645){\framebox(95,20){}}
\put( 80,650){\makebox(0,0)[lb]{\smash{\SetFigFont{12}{14.4}{rm}Toggle Sensitive}}}
\put(180,645){\framebox(80,20){}}
\put(185,650){\makebox(0,0)[lb]{\smash{\SetFigFont{12}{14.4}{rm}Change Me A}}}
\put(270,645){\framebox(45,20){}}
\put(275,650){\makebox(0,0)[lb]{\smash{\SetFigFont{12}{14.4}{rm}Cancel}}}
\put(215,690){\circle{10}}
\put(215,745){\circle*{10}}
\put(216,719){\circle{10}}
\put( 85,740){\framebox(10,10){}}
\put(145,740){\framebox(10,10){}}
\put( 85,720){\framebox(10,10){}}
\put( 75,715){\framebox(125,60){}}
\put(280,735){\framebox(55,20){}}
\put(280,710){\framebox(55,20){}}
\put(280,685){\framebox(55,20){}}
\put(205,675){\framebox(60,100){}}
\put(270,675){\framebox(75,100){}}
\put( 60,630){\framebox(315,170){}}
\put(325,645){\framebox(40,20){}}
\put(330,665){\line( 0,-1){ 20}}
\put(360,665){\line( 0,-1){ 20}}
\put( 85,760){\makebox(0,0)[lb]{\smash{\SetFigFont{12}{14.4}{rm}CheckBox}}}
\put(100,740){\makebox(0,0)[lb]{\smash{\SetFigFont{12}{14.4}{rm}Test A}}}
\put(160,740){\makebox(0,0)[lb]{\smash{\SetFigFont{12}{14.4}{rm}Test B}}}
\put(100,720){\makebox(0,0)[lb]{\smash{\SetFigFont{12}{14.4}{rm}Test C}}}
\put(220,760){\makebox(0,0)[lb]{\smash{\SetFigFont{12}{14.4}{rm}Radios}}}
\put(280,760){\makebox(0,0)[lb]{\smash{\SetFigFont{12}{14.4}{rm}Buttons}}}
\put(285,740){\makebox(0,0)[lb]{\smash{\SetFigFont{12}{14.4}{rm}Button 1}}}
\put(285,715){\makebox(0,0)[lb]{\smash{\SetFigFont{12}{14.4}{rm}Button 2}}}
\put(285,690){\makebox(0,0)[lb]{\smash{\SetFigFont{12}{14.4}{rm}Button 3}}}
\put(225,685){\makebox(0,0)[lb]{\smash{\SetFigFont{12}{14.4}{rm}KRQE}}}
\put(225,740){\makebox(0,0)[lb]{\smash{\SetFigFont{12}{14.4}{rm}KOB}}}
\put(225,715){\makebox(0,0)[lb]{\smash{\SetFigFont{12}{14.4}{rm}KOAT}}}
\put(210,785){\makebox(0,0)[lb]{\smash{\SetFigFont{10}{12.0}{rm}(1)}}}
\put(150,760){\makebox(0,0)[lb]{\smash{\SetFigFont{10}{12.0}{rm}(2)}}}
\put(190,690){\makebox(0,0)[lb]{\smash{\SetFigFont{10}{12.0}{rm}(3)}}}
\put(325,760){\makebox(0,0)[lb]{\smash{\SetFigFont{10}{12.0}{rm}(4)}}}
\put(110,670){\makebox(0,0)[lb]{\smash{\SetFigFont{10}{12.0}{rm}(5)}}}
\put(350,670){\makebox(0,0)[lb]{\smash{\SetFigFont{10}{12.0}{rm}(6)}}}
\put(335,650){\makebox(0,0)[lb]{\smash{\SetFigFont{12}{14.4}{rm}OK}}}
\put( 80,785){\makebox(0,0)[lb]{\smash{\SetFigFont{12}{14.4}{rm}Sample Modeless Dialog}}}
\end{picture}

\end{latexonly}

\normalfont\normalsize
\end{center}

\caption{This dialog consists of: (1)~Dialog label. (2)~Three check boxes
in a frame. (3)~Three radio buttons in a frame. (4)~Three buttons
in a frame. (5)~Four buttons, including (6)~ the default OK
button.} \label{fig:mydialog}
\end{figure}

Much control interaction with \V applications takes place
through one of two dialog objects: \emph{modal}
and \emph{modeless} dialogs. In a modal dialog, interaction with
any other window or dialog is locked out until the user
interacts with it. In a modeless dialog, the user can continue to
interact with other parts of the application while the dialog remains
displayed. Modal dialogs will go away once the user enters a
command. Modeless dialogs may or may not go away, depending on
their purpose.

\V\ supports a comprehensive set of controls for dialogs. These
include command buttons, text labels, text input, list selection
boxes, combo boxes, radio buttons, check boxes, spinners for
value entry, sliders, and progress bars. These controls may be
grouped into boxes. Layout of controls in a dialog is defined in
the dialog definition list in the source code. Controls may be
used in window command bars as well as dialogs.

Figure~\ref{fig:mydialog} represents, more or less, a typical
\V\ dialog.

\subsection*{Events}
\index{event}

The structure of the code for user command processing in GUI
applications is quite different from traditional C programs. The
user input control model of traditional C programs is rather
simple, usually using \code{printf} and \code{getc} or some
variant for interaction. Logically, the program reaches a point
where it needs input, and then waits for that input.

GUI applications deal with user input much differently.
Interaction with an application from the user's viewpoint
consists of a series of mouse movements and clicks, and text and
command input through the keyboard. From the programmer's
viewpoint, each of these is an event. The important thing about
an event is that it can occur at any time, and the program cannot
simply stop and wait for the event to happen.

Interaction with an application by the user can generate several
different kinds of events. Consider mouse events.
If the mouse is in the drawing area, each movement generates a
\emph{mouse movement} event. If the user clicks a mouse button, a
\emph{mouse button} event is generated. A keystroke from the
keyboard will generate a \emph{keyboard} event.

If the mouse pointer is in a dialog, or over a menu or command
button, then movement events are not generated. Instead, button
clicks generate \emph{command} events. 

Sometimes an application needs to track the passage of time. The
application can call a system function that will generate a
\emph{timer} event at a given interval.

In a GUI environment, windows are usually not displayed alone. Often,
other applications are running, each with its own windows. The
host windowing system typically displays windows with various decorations
that let the user manipulate the windows. Sometimes, these
manipulations will generate events that require a response
from the application code. For example, the user can use the
mouse to change the size of a window causing a \emph{resize}
event. When multiple windows are displayed, some can be
completely or partially covered by other windows. If the user
moves a window so that a different part of the window is
displayed, then an \emph{expose} event is generated, which
requires the program to redraw part of the canvas area.

All these events require a response from the application --
to carry out the command, to draw something in the canvas area,
or to redraw the canvas after a resize or expose event. Some
events, however, are handled by the system, and not the
application. This includes drawing menus and handling dialogs.
For example, when a dialog is displayed, the system tracks mouse
movements within the dialog, and handles redrawing the dialog for
expose events. In general, the application is responsible for
resize and expose events only for the canvas area.

All these events are asynchronous, and the application must be able
to respond immediately to any of these events. Traditionally, 
handling events has been rather complicated. For each possible
event, the program registers an \emph{event handler} with
the system. Then, the program runs in an \emph{event loop}.
The event loop receives an event, and then calls a
function to dispatch the event to the proper event handler.

C++ makes dealing with events much easier. Each event can be considered
a message, and the message is central to object-oriented
programming. In \V, each object, such as a command window, has
methods\footnote{I use the general object-oriented term \emph{method}
to refer to what are called \emph{member functions} in C++
terminology.} that the system sends event messages to. For
example, there is a \code{WindowCommand} method that responds to
command events from the system. The application overrides the
default \V\ \code{WindowCommand} method to handle commands as
needed by the application. All the details of the event loop and
event handlers are hidden in the \V\ implementation. If you have
ever programmed with event handlers and loops, you will find the
simplicity of overriding default methods incredibly easy in
comparison!

%---------------------------------------------------------------
\section{Easy to program}

One of the main goals of the design of \V\ was to make it easy to
use to write real programs. Some of the factors that help \V\ meet
this goal are described in the following sections.

\subsection*{Hide the dirty details}

One of the problems with using most native GUI libraries such
as Xt or Windows is the huge amount of overhead and detail required
to perform even the simplest tasks. You are typically coding at a
very low level. While part of this complexity may be necessary to
allow total flexibility in what you can do, the vast majority of
applications just do not need total flexibility. \V\ was designed
to hide most of the details of the underlying GUI library. Things
such as library initialization, specific window handles, and
calls required to build low level controls are all hidden.
Instead, you work at the much higher level of objects needed to
build a typical GUI.

\subsection*{Easy to define GUI objects}

It has always seemed to me that a GUI object such as a menu could
most simply be thought of as a single object consisting of a list
of items on that menu with their associated attributes. Rather than
requiring a set of complicated calls to build that menu list, in
\V\ you can simply define a menu as a static C++ struct array --
a list in other words. The same applies to dialogs. A dialog is a
list of control objects with associated attributes, including
how the controls are positioned in the dialog. This philosophy leads
to very easy to maintain code. Menu and dialog lists are well
defined in a single place in your code, and it is very easy to
modify and change the list definitions. Actions for each menu or
dialog command are defined in a single C++ method that responds
to command events.

\subsection*{No resource editors}
\index{resource editor}

One data object used by most, but not all, native GUI libraries
is what is usually called a resource file. A resource file is
most often used to specify layout of dialogs and menus. One
reason resource files are used is that specifying the layout of
dialogs and menus directly in the code is often very difficult
for the native libraries.

The combination of the way \V\ lets you specify menus and dialogs,
and the way C++ makes responding to event messages so easy has really
removed the need for resource files. This in turn eliminates one
of the more complicated aspects of portability across platforms.

\subsection*{Look and Feel}

One of the limitations of \V\ is that it has its own look and feel.
While this may be a limitation, it is not necessarily bad. First,
the look and feel is constrained so that applications will be portable
across platforms and look like native applications on each platform.
This means some things that are possible on one GUI platform, but
not another, are not included in \V\@.

\V\ also incorporates much of my own experience. I really like
simplicity, and believe that just because you can do something,
it is not necessarily a good idea to do so. Thus, for example,
there are limitations on the number of menu items per menu, and
how deeply you can nest pull down menus. These limits in fact
help enforce good interface design.

%---------------------------------------------------------------
\section{Good Example of OO}

While \V\ has been designed to develop real and useful GUI
applications, it also has been designed to be useful in a
learning environment. Thus, \V\ represents a good example of
object-oriented design. 

GUI systems are a natural for object orientation (OO). It is easy
to understand the nature of each object -- a window, a dialog, a
command button, a menu bar, a canvas, and so on. Inheritance and
aggregation of these objects is also very natural. Events are
messages, and sending messages to methods is pure OO.

\index{GNU license}
Since \V\ is licensed under the terms of the GNU Library General
Public License, the source code will always be available for
study. It was written using the guidelines of Appendix B, and is
very readable and easy to understand. Not only is the \V\ source
code a good example of OO programming, you may also find it
interesting if you want to learn things about how the
underlying GUI toolkits work. While good examples of freeware X
source code are readily available, good examples of non-trivial
Windows source code are nearly impossible to come by. I hope the
\V\ Windows source code will help fill this void.


%---------------------------------------------------------------
\section{The V Object Hierarchy}
\index{V Object Hierarchy}

This manual contains several object hierarchy diagrams of the
\V\ framework, and of \V\ applications. There are many graphical
notations in varying degrees of widespread use, but I have found
the Coad-Yourdon\footnote{Peter Coad and Edward Yourdon, 
\emph{Object-Oriented Analysis}, 2nd ed. (Yourdon
Press/Prentice Hall, 1990); and Edward Yourdon, \emph{Object-Oriented
System Design, An Integrated Approach} (Yourdon Press/Prentice Hall,
1994, ISBN 0-13-636325-3).} notation one of the
\index{Coad-Yourdon}
easiest to learn and simplest to use. The basic graphical
elements of the notation are shown in  Figure~\ref{fig:oonotation}.

\begin{rawhtml}
<IMG BORDER=0 ALIGN=BOTTOM ALT="" SRC="../fig/oonotation.gif">
\end{rawhtml}

\begin{figure}[htb]

\begin{center}
\small
\begin{latexonly}
\setlength{\unitlength}{0.012500in}%
\begin{picture}(324,150)(19,670)
\thicklines
\put( 30,695){\oval( 20, 20)[tr]}
\put( 30,695){\oval( 20, 20)[tl]}
\put( 20,695){\line( 1, 0){ 20}}
\put( 30,695){\line( 0,-1){ 15}}
\put( 30,720){\line( 0,-1){ 15}}
\put( 50,695){\makebox(0,0)[lb]{\smash{\SetFigFont{10}{12.0}{rm}inheritance (is-a)}}}
\multiput(195,695)(0.40000,0.40000){26}{\makebox(0.4444,0.6667){\SetFigFont{7}{8.4}{rm}.}}
\multiput(205,705)(0.40000,-0.40000){26}{\makebox(0.4444,0.6667){\SetFigFont{7}{8.4}{rm}.}}
\put(215,695){\line(-1, 0){ 20}}
\put(220,710){\makebox(0,0)[lb]{\smash{\SetFigFont{10}{12.0}{rm}1,N}}}
\put(220,685){\makebox(0,0)[lb]{\smash{\SetFigFont{10}{12.0}{rm}1,N}}}
\put(205,695){\line( 0,-1){ 15}}
\put(205,680){\line( 0, 1){  5}}
\put(250,695){\makebox(0,0)[lb]{\smash{\SetFigFont{10}{12.0}{rm}aggregation (has-a)}}}
\put(230,670){\makebox(0,0)[lb]{\smash{\SetFigFont{10}{12.0}{rm}(Part Of 1 to N objects)}}}
\put(230,725){\makebox(0,0)[lb]{\smash{\SetFigFont{10}{12.0}{rm}(Has 1 to N objects)}}}
\put(205,720){\line( 0,-1){ 15}}
\put(195,760){\framebox(90,60){}}
\put(200,765){\framebox(80,50){}}
\put(200,800){\line( 1, 0){ 80}}
\put(200,780){\line( 1, 0){ 80}}
\put(295,800){\makebox(0,0)[lb]{\smash{\SetFigFont{10}{12.0}{rm}Class with}}}
\put(295,785){\makebox(0,0)[lb]{\smash{\SetFigFont{10}{12.0}{rm}instances of}}}
\put(295,770){\makebox(0,0)[lb]{\smash{\SetFigFont{10}{12.0}{rm}objects}}}
\put(210,805){\makebox(0,0)[lb]{\smash{\SetFigFont{10}{12.0}{rm}class name}}}
\put(210,770){\makebox(0,0)[lb]{\smash{\SetFigFont{10}{12.0}{rm}methods}}}
\put(210,785){\makebox(0,0)[lb]{\smash{\SetFigFont{10}{12.0}{rm}attributes}}}
\put( 20,765){\framebox(80,50){}}
\put( 20,800){\line( 1, 0){ 80}}
\put( 20,780){\line( 1, 0){ 80}}
\put(110,800){\makebox(0,0)[lb]{\smash{\SetFigFont{10}{12.0}{rm}Class with}}}
\put(110,785){\makebox(0,0)[lb]{\smash{\SetFigFont{10}{12.0}{rm}no instances}}}
\put( 30,805){\makebox(0,0)[lb]{\smash{\SetFigFont{10}{12.0}{rm}class name}}}
\put( 30,770){\makebox(0,0)[lb]{\smash{\SetFigFont{10}{12.0}{rm}methods}}}
\put( 30,785){\makebox(0,0)[lb]{\smash{\SetFigFont{10}{12.0}{rm}attributes}}}
\put(110,770){\makebox(0,0)[lb]{\smash{\SetFigFont{10}{12.0}{rm}of objects}}}
\end{picture}

\end{latexonly}
\normalfont\normalsize
\end{center}

\caption{Coad-Yourdon OO Notation} \label{fig:oonotation}
\end{figure}

An object is shown in a rectangular box. A single border indicates
a generalized base class that will not have any instances,
while a double border indicates that the named object can have
instances. Generalization/specialization (inheritance, or is-a)
relationships are shown with half circles. Whole/part (aggregation,
or has-a) relationships are shown with triangles\footnote{Hint:
It is sometimes hard to remember which shape is which. A triangle
looks like a capital letter A as in Aggregation. The half circle
shape is then inheritance.}.

The ``1,N'' notation at the top of the aggregation triangle
indicates that the object above can contain from 1 to N instances
of the object below. The lower ``1,N'' indicates the lower object
can be a part of 1 to N objects. The values can be changed to
reflect reality. Thus, it is common to have ``1,N'' at the top,
indicating that an object may contain many instances of the lower
object, and just a ``1'' for the lower value, indication that an
object is a part of exactly one of the upper objects.

When discussing a design at a high level, the attributes and
methods boxes are often left blank. This leads to hierarchies
such as the one for \V\ in Figure~\ref{fig:vproghier} that shows
the programming view of the \V\ framework. In this case, there
are no generalized base objects, and most of the relationships
are whole/part.

\begin{rawhtml}
<IMG BORDER=0 ALIGN=BOTTOM ALT="" SRC="../fig/vproghier.gif">
\end{rawhtml}

\begin{figure}[htb]
\begin{center}
\small
\begin{latexonly}
\setlength{\unitlength}{0.012500in}%
\begin{picture}(360,420)(5,415)
\thicklines
\multiput(255,765)(0.40000,0.40000){26}{\makebox(0.4444,0.6667){\SetFigFont{7}{8.4}{rm}.}}
\multiput(265,775)(0.40000,-0.40000){26}{\makebox(0.4444,0.6667){\SetFigFont{7}{8.4}{rm}.}}
\put(275,765){\line(-1, 0){ 20}}
\put(265,785){\oval( 20, 20)[tr]}
\put(265,785){\oval( 20, 20)[tl]}
\put(255,785){\line( 1, 0){ 20}}
\put( 10,535){\framebox(80,50){}}
\put(  5,530){\framebox(90,60){}}
\put( 10,570){\line( 1, 0){ 80}}
\put( 10,550){\line( 1, 0){ 80}}
\put( 25,575){\makebox(0,0)[lb]{\smash{\SetFigFont{10}{12.0}{rm}vMenuPane}}}
\multiput( 40,605)(0.40000,0.40000){26}{\makebox(0.4444,0.6667){\SetFigFont{7}{8.4}{rm}.}}
\multiput( 50,615)(0.40000,-0.40000){26}{\makebox(0.4444,0.6667){\SetFigFont{7}{8.4}{rm}.}}
\put( 60,605){\line(-1, 0){ 20}}
\put( 65,595){\makebox(0,0)[lb]{\smash{\SetFigFont{10}{12.0}{rm}1}}}
\put( 65,620){\makebox(0,0)[lb]{\smash{\SetFigFont{10}{12.0}{rm}1}}}
\put( 50,605){\line( 0,-1){ 15}}
\put(280,535){\framebox(80,50){}}
\put(275,530){\framebox(90,60){}}
\put(280,570){\line( 1, 0){ 80}}
\put(280,550){\line( 1, 0){ 80}}
\put(290,575){\makebox(0,0)[lb]{\smash{\SetFigFont{10}{12.0}{rm}vModalDialog}}}
\multiput(310,500)(0.40000,0.40000){26}{\makebox(0.4444,0.6667){\SetFigFont{7}{8.4}{rm}.}}
\multiput(320,510)(0.40000,-0.40000){26}{\makebox(0.4444,0.6667){\SetFigFont{7}{8.4}{rm}.}}
\put(330,500){\line(-1, 0){ 20}}
\put(335,490){\makebox(0,0)[lb]{\smash{\SetFigFont{10}{12.0}{rm}1}}}
\put(335,515){\makebox(0,0)[lb]{\smash{\SetFigFont{10}{12.0}{rm}1,N}}}
\multiput(245,500)(0.40000,0.40000){26}{\makebox(0.4444,0.6667){\SetFigFont{7}{8.4}{rm}.}}
\multiput(255,510)(0.40000,-0.40000){26}{\makebox(0.4444,0.6667){\SetFigFont{7}{8.4}{rm}.}}
\put(265,500){\line(-1, 0){ 20}}
\put(270,490){\makebox(0,0)[lb]{\smash{\SetFigFont{10}{12.0}{rm}1}}}
\put(270,515){\makebox(0,0)[lb]{\smash{\SetFigFont{10}{12.0}{rm}1,N}}}
\put(250,420){\framebox(80,50){}}
\put(245,415){\framebox(90,60){}}
\put(250,455){\line( 1, 0){ 80}}
\put(250,435){\line( 1, 0){ 80}}
\put(251,460){\makebox(0,0)[lb]{\smash{\SetFigFont{10}{12.0}{rm}CommandObject}}}
\put(130,535){\framebox(80,50){}}
\put(125,530){\framebox(90,60){}}
\put(130,570){\line( 1, 0){ 80}}
\put(130,550){\line( 1, 0){ 80}}
\put(145,575){\makebox(0,0)[lb]{\smash{\SetFigFont{10}{12.0}{rm}vStatusPane}}}
\multiput(160,605)(0.40000,0.40000){26}{\makebox(0.4444,0.6667){\SetFigFont{7}{8.4}{rm}.}}
\multiput(170,615)(0.40000,-0.40000){26}{\makebox(0.4444,0.6667){\SetFigFont{7}{8.4}{rm}.}}
\put(180,605){\line(-1, 0){ 20}}
\put(170,605){\line( 0,-1){ 15}}
\put(185,595){\makebox(0,0)[lb]{\smash{\SetFigFont{10}{12.0}{rm}1}}}
\put(185,620){\makebox(0,0)[lb]{\smash{\SetFigFont{10}{12.0}{rm}0,N}}}
\put(130,420){\framebox(80,50){}}
\put(125,415){\framebox(90,60){}}
\put(130,455){\line( 1, 0){ 80}}
\put(130,435){\line( 1, 0){ 80}}
\put(133,460){\makebox(0,0)[lb]{\smash{\SetFigFont{10}{12.0}{rm}vCommandPane}}}
\multiput(160,490)(0.40000,0.40000){26}{\makebox(0.4444,0.6667){\SetFigFont{7}{8.4}{rm}.}}
\multiput(170,500)(0.40000,-0.40000){26}{\makebox(0.4444,0.6667){\SetFigFont{7}{8.4}{rm}.}}
\put(180,490){\line(-1, 0){ 20}}
\put(170,490){\line( 0,-1){ 15}}
\put(185,480){\makebox(0,0)[lb]{\smash{\SetFigFont{10}{12.0}{rm}1}}}
\put(185,505){\makebox(0,0)[lb]{\smash{\SetFigFont{10}{12.0}{rm}0,N}}}
\put( 10,420){\framebox(80,50){}}
\put(  5,415){\framebox(90,60){}}
\put( 10,455){\line( 1, 0){ 80}}
\put( 10,435){\line( 1, 0){ 80}}
\put( 25,460){\makebox(0,0)[lb]{\smash{\SetFigFont{10}{12.0}{rm}vCanvasPane}}}
\multiput( 40,490)(0.40000,0.40000){26}{\makebox(0.4444,0.6667){\SetFigFont{7}{8.4}{rm}.}}
\multiput( 50,500)(0.40000,-0.40000){26}{\makebox(0.4444,0.6667){\SetFigFont{7}{8.4}{rm}.}}
\put( 60,490){\line(-1, 0){ 20}}
\put( 50,490){\line( 0,-1){ 15}}
\put( 65,480){\makebox(0,0)[lb]{\smash{\SetFigFont{10}{12.0}{rm}1}}}
\put( 65,505){\makebox(0,0)[lb]{\smash{\SetFigFont{10}{12.0}{rm}1}}}
\put( 10,780){\framebox(80,50){}}
\put(  5,775){\framebox(90,60){}}
\put( 10,815){\line( 1, 0){ 80}}
\put( 10,795){\line( 1, 0){ 80}}
\put( 35,820){\makebox(0,0)[lb]{\smash{\SetFigFont{10}{12.0}{rm}vApp}}}
\multiput( 40,750)(0.40000,0.40000){26}{\makebox(0.4444,0.6667){\SetFigFont{7}{8.4}{rm}.}}
\multiput( 50,760)(0.40000,-0.40000){26}{\makebox(0.4444,0.6667){\SetFigFont{7}{8.4}{rm}.}}
\put( 60,750){\line(-1, 0){ 20}}
\put( 65,740){\makebox(0,0)[lb]{\smash{\SetFigFont{10}{12.0}{rm}1}}}
\put( 65,765){\makebox(0,0)[lb]{\smash{\SetFigFont{10}{12.0}{rm}1,N}}}
\put( 10,680){\framebox(80,50){}}
\put(  5,675){\framebox(90,60){}}
\put( 10,715){\line( 1, 0){ 80}}
\put( 10,695){\line( 1, 0){ 80}}
\put( 20,720){\makebox(0,0)[lb]{\smash{\SetFigFont{10}{12.0}{rm}vCmdWindow}}}
\put(320,595){\oval( 20, 20)[tr]}
\put(320,595){\oval( 20, 20)[tl]}
\put(310,595){\line( 1, 0){ 20}}
\put(245,620){\framebox(80,50){}}
\put(240,615){\framebox(90,60){}}
\put(245,655){\line( 1, 0){ 80}}
\put(245,635){\line( 1, 0){ 80}}
\put(265,660){\makebox(0,0)[lb]{\smash{\SetFigFont{10}{12.0}{rm}vDialog}}}
\multiput(275,690)(0.40000,0.40000){26}{\makebox(0.4444,0.6667){\SetFigFont{7}{8.4}{rm}.}}
\multiput(285,700)(0.40000,-0.40000){26}{\makebox(0.4444,0.6667){\SetFigFont{7}{8.4}{rm}.}}
\put(295,690){\line(-1, 0){ 20}}
\put(300,680){\makebox(0,0)[lb]{\smash{\SetFigFont{10}{12.0}{rm}1}}}
\put(300,705){\makebox(0,0)[lb]{\smash{\SetFigFont{10}{12.0}{rm}0,N}}}
\put(320,530){\line( 0,-1){ 20}}
\put(255,500){\line( 0,-1){ 15}}
\put(255,485){\line( 1, 0){ 25}}
\put(280,485){\line( 0,-1){ 10}}
\put(320,500){\line( 0,-1){ 15}}
\put(320,485){\line(-1, 0){ 20}}
\put(300,485){\line( 0,-1){ 10}}
\put(300,475){\line( 0, 1){  5}}
\put( 20,675){\line( 0,-1){ 50}}
\put( 20,625){\line( 1, 0){ 30}}
\put( 50,625){\line( 0,-1){ 10}}
\put( 80,675){\line( 0,-1){  5}}
\put( 80,670){\line( 1, 0){140}}
\put(220,670){\line( 0, 1){ 50}}
\put(220,720){\line( 1, 0){ 65}}
\put(285,720){\line( 0,-1){ 20}}
\put(285,690){\line( 0,-1){ 15}}
\put(255,510){\line( 0, 1){105}}
\put(310,615){\line( 0,-1){  5}}
\put(310,610){\line( 1, 0){ 10}}
\put(320,610){\line( 0,-1){  5}}
\put(320,595){\line( 0,-1){  5}}
\put( 35,675){\line( 0,-1){ 40}}
\put( 35,635){\line( 1, 0){ 70}}
\put(105,635){\line( 0,-1){115}}
\put(105,520){\line(-1, 0){ 55}}
\put( 50,520){\line( 0,-1){ 20}}
\put( 50,675){\line( 0,-1){ 30}}
\put( 50,645){\line( 1, 0){ 65}}
\put(115,645){\line( 0,-1){125}}
\put(115,520){\line( 1, 0){ 55}}
\put(170,520){\line( 0,-1){ 20}}
\put( 65,675){\line( 0,-1){ 20}}
\put( 65,655){\line( 1, 0){105}}
\put(170,655){\line( 0,-1){ 40}}
\put( 50,775){\line( 0,-1){ 15}}
\put( 50,750){\line( 0,-1){ 15}}
\put(280,785){\makebox(0,0)[lb]{\smash{\SetFigFont{10}{12.0}{rm}= inheritance}}}
\put(280,765){\makebox(0,0)[lb]{\smash{\SetFigFont{10}{12.0}{rm}= aggregation}}}
\end{picture}

\end{latexonly}
\normalfont\normalsize
\end{center}
\caption{Programming View of V Classes} \label{fig:vproghier}
\end{figure}

Figure~\ref{fig:vproghier} reveals some interesting things about
\V's look and feel. Note that a \code{vApp} class has 1 to N
\code{vCmdWindows}, indicating that there will be at least one
window. Each window, in turn, has exactly one menu and canvas,
but zero to many command panes, status panes, and dialogs.

The version of the \V\ hierarchy in the Appendix shows an
implementation view of the hierarchy. Some of the classes that
are never seen or used by the programmer are shown in that
hierarchy. 

%***********************************************************************
%***********************************************************************
%***********************************************************************
\chapter{A V Tutorial}

%------------------------------------------------------------------------

This chapter is intended to cover the elements that make up a
\V\ application. The first section covers the general
organization of a ``Standard \V\ Application''. Read this section
to get an overview of a \V\ application. Don't worry about the details
yet -- just the the main idea. Then read Section~\ref{sec:tutexamp}
and Appendix A, which has the source code of a small, complete
\V\ application, to get the details.

\section{Getting Started with Your Own V Application}
\index{getting started}

As with any new system, \V\ has a learning curve before you can
write applications of your own. \V's learning curve is actually
pretty short. The experience of the students using \V\ has shown
the best way to get started with \V\ is to first read the first
part of this reference manual, including this chapter. Then begin
with an example \V\ application. 

The \V application generator, \code{vgen}, included with the
\V distribution is the easiest way to begin building a \V
application. Run \code{vgen}, select the basic options you
want to include in your application, select the directory
to save the generated code in, and then generate the basic
skeleton application. From the skeleton app, it is relatively
easy to add your own functionality.

The tutorial application described in this chapter is also an
excellent \V example. Start by getting the example to compile.
Then modify the code to add or remove features. Before long, you
will have a good feel for \V, and be able to add all the features
you need.

There are several other example programs provided with the \V\
distribution. This tutorial is found in \code{\~{}/v/tutor}.
The VDraw program is found in \code{\~{}/v/draw}. The program
used to test all \V\ functionality is found in \code{\~{}/v/test}.
It will have an example of how to use every \V\ feature,
although it is not as well structured as the other examples.

\section{A Standard V Application}
\index{standard V application}

While the \V\ framework is flexible enough to allow many different
approaches to building an application, you should find it easier
to base your applications on a model \emph{Standard \V\ Application}.
The software organization described by a Standard \V\ Application
can support MVC (Model-View-Controller) object-oriented architecture paradigm.

Figure~\ref{fig:stdvapp} shows the hierarchy of a standard \V\ application.
A standard \V\ application consists of the parts described below.
Each part consists of a pair of \code{.cpp} (or \code{.cxx}) and
\code{.h} files (except the \code{makefile}).

\begin{rawhtml}
<IMG BORDER=0 ALIGN=BOTTOM ALT="" SRC="../fig/stdvapp.gif">
\end{rawhtml}

\begin{figure}[htb]
\begin{center}
\vspace{.1in}

\small
\begin{latexonly}
\setlength{\unitlength}{0.012500in}%
\begin{picture}(430,405)(10,420)
\thicklines
\put(160,760){\framebox(80,60){}}
\put(155,755){\framebox(90,70){}}
\put(160,805){\line( 1, 0){ 80}}
\put(160,775){\line( 1, 0){ 80}}
\put(185,810){\makebox(0,0)[lb]{\smash{\SetFigFont{10}{12.0}{rm}myApp}}}
\put(160,645){\framebox(80,60){}}
\put(155,640){\framebox(90,70){}}
\put(160,690){\line( 1, 0){ 80}}
\put(160,660){\line( 1, 0){ 80}}
\put(165,695){\makebox(0,0)[lb]{\smash{\SetFigFont{10}{12.0}{rm}myCmdWindow}}}
\multiput(190,725)(0.40000,0.40000){26}{\makebox(0.4444,0.6667){\SetFigFont{7}{8.4}{rm}.}}
\multiput(200,735)(0.40000,-0.40000){26}{\makebox(0.4444,0.6667){\SetFigFont{7}{8.4}{rm}.}}
\put(210,725){\line(-1, 0){ 20}}
\put(215,715){\makebox(0,0)[lb]{\smash{\SetFigFont{10}{12.0}{rm}1}}}
\put(215,730){\makebox(0,0)[lb]{\smash{\SetFigFont{10}{12.0}{rm}1,N}}}
\put( 40,645){\framebox(80,60){}}
\put( 35,640){\framebox(90,70){}}
\put( 40,690){\line( 1, 0){ 80}}
\put( 40,660){\line( 1, 0){ 80}}
\put( 45,695){\makebox(0,0)[lb]{\smash{\SetFigFont{10}{12.0}{rm}myAppWinInfo}}}
\multiput( 70,725)(0.40000,0.40000){26}{\makebox(0.4444,0.6667){\SetFigFont{7}{8.4}{rm}.}}
\multiput( 80,735)(0.40000,-0.40000){26}{\makebox(0.4444,0.6667){\SetFigFont{7}{8.4}{rm}.}}
\put( 90,725){\line(-1, 0){ 20}}
\put( 95,715){\makebox(0,0)[lb]{\smash{\SetFigFont{10}{12.0}{rm}1}}}
\put( 95,730){\makebox(0,0)[lb]{\smash{\SetFigFont{10}{12.0}{rm}1,N}}}
\multiput(310,725)(0.40000,0.40000){26}{\makebox(0.4444,0.6667){\SetFigFont{7}{8.4}{rm}.}}
\multiput(320,735)(0.40000,-0.40000){26}{\makebox(0.4444,0.6667){\SetFigFont{7}{8.4}{rm}.}}
\put(330,725){\line(-1, 0){ 20}}
\put(335,715){\makebox(0,0)[lb]{\smash{\SetFigFont{10}{12.0}{rm}1}}}
\put(335,730){\makebox(0,0)[lb]{\smash{\SetFigFont{10}{12.0}{rm}1}}}
\put( 50,800){\makebox(0,0)[lb]{\smash{\SetFigFont{12}{14.4}{rm}A Standard}}}
\put( 50,780){\makebox(0,0)[lb]{\smash{\SetFigFont{12}{14.4}{rm}V Application}}}
\put( 15,525){\framebox(80,60){}}
\put( 10,520){\framebox(90,70){}}
\put( 15,570){\line( 1, 0){ 80}}
\put( 15,540){\line( 1, 0){ 80}}
\put( 30,575){\makebox(0,0)[lb]{\smash{\SetFigFont{10}{12.0}{rm}vMenuPane}}}
\put( 55,605){\line( 0,-1){ 15}}
\multiput( 45,605)(0.40000,0.40000){26}{\makebox(0.4444,0.6667){\SetFigFont{7}{8.4}{rm}.}}
\multiput( 55,615)(0.40000,-0.40000){26}{\makebox(0.4444,0.6667){\SetFigFont{7}{8.4}{rm}.}}
\put( 65,605){\line(-1, 0){ 20}}
\put( 70,595){\makebox(0,0)[lb]{\smash{\SetFigFont{10}{12.0}{rm}1}}}
\put( 70,610){\makebox(0,0)[lb]{\smash{\SetFigFont{10}{12.0}{rm}1}}}
\put(110,525){\framebox(80,60){}}
\put(105,520){\framebox(90,70){}}
\put(110,570){\line( 1, 0){ 80}}
\put(110,540){\line( 1, 0){ 80}}
\put(115,575){\makebox(0,0)[lb]{\smash{\SetFigFont{10}{12.0}{rm}myCanvasPane}}}
\put(150,605){\line( 0,-1){ 15}}
\multiput(140,605)(0.40000,0.40000){26}{\makebox(0.4444,0.6667){\SetFigFont{7}{8.4}{rm}.}}
\multiput(150,615)(0.40000,-0.40000){26}{\makebox(0.4444,0.6667){\SetFigFont{7}{8.4}{rm}.}}
\put(160,605){\line(-1, 0){ 20}}
\put(165,595){\makebox(0,0)[lb]{\smash{\SetFigFont{10}{12.0}{rm}1}}}
\put(165,610){\makebox(0,0)[lb]{\smash{\SetFigFont{10}{12.0}{rm}1}}}
\multiput(235,605)(0.40000,0.40000){26}{\makebox(0.4444,0.6667){\SetFigFont{7}{8.4}{rm}.}}
\multiput(245,615)(0.40000,-0.40000){26}{\makebox(0.4444,0.6667){\SetFigFont{7}{8.4}{rm}.}}
\put(255,605){\line(-1, 0){ 20}}
\put(260,595){\makebox(0,0)[lb]{\smash{\SetFigFont{10}{12.0}{rm}1}}}
\put(260,610){\makebox(0,0)[lb]{\smash{\SetFigFont{10}{12.0}{rm}0,N}}}
\put(200,520){\framebox(90,70){}}
\put(210,575){\makebox(0,0)[lb]{\smash{\SetFigFont{10}{12.0}{rm}vCommandPane}}}
\put(205,525){\framebox(80,60){}}
\put(205,540){\line( 1, 0){ 80}}
\put(205,570){\line( 1, 0){ 80}}
\put(245,605){\line( 0,-1){ 15}}
\multiput(330,605)(0.40000,0.40000){26}{\makebox(0.4444,0.6667){\SetFigFont{7}{8.4}{rm}.}}
\multiput(340,615)(0.40000,-0.40000){26}{\makebox(0.4444,0.6667){\SetFigFont{7}{8.4}{rm}.}}
\put(350,605){\line(-1, 0){ 20}}
\put(355,595){\makebox(0,0)[lb]{\smash{\SetFigFont{10}{12.0}{rm}1}}}
\put(355,610){\makebox(0,0)[lb]{\smash{\SetFigFont{10}{12.0}{rm}0,N}}}
\put(295,520){\framebox(90,70){}}
\put(300,525){\framebox(80,60){}}
\put(316,575){\makebox(0,0)[lb]{\smash{\SetFigFont{10}{12.0}{rm}vStatusPane}}}
\put(300,540){\line( 1, 0){ 80}}
\put(300,570){\line( 1, 0){ 80}}
\put(340,605){\line( 0,-1){ 15}}
\multiput(385,505)(0.40000,0.40000){26}{\makebox(0.4444,0.6667){\SetFigFont{7}{8.4}{rm}.}}
\multiput(395,515)(0.40000,-0.40000){26}{\makebox(0.4444,0.6667){\SetFigFont{7}{8.4}{rm}.}}
\put(405,505){\line(-1, 0){ 20}}
\put(410,495){\makebox(0,0)[lb]{\smash{\SetFigFont{10}{12.0}{rm}1}}}
\put(410,510){\makebox(0,0)[lb]{\smash{\SetFigFont{10}{12.0}{rm}0,N}}}
\put(355,425){\framebox(80,60){}}
\put(350,420){\framebox(90,70){}}
\put(355,470){\line( 1, 0){ 80}}
\put(355,440){\line( 1, 0){ 80}}
\put(375,475){\makebox(0,0)[lb]{\smash{\SetFigFont{10}{12.0}{rm}myDialog}}}
\put(395,505){\line( 0,-1){ 15}}
\put( 55,615){\line( 0, 1){ 15}}
\put( 55,630){\line( 1, 0){125}}
\put(180,630){\line( 0, 1){ 10}}
\put(190,640){\line( 0,-1){ 20}}
\put(190,620){\line(-1, 0){ 40}}
\put(150,620){\line( 0,-1){  5}}
\put(200,725){\line( 0,-1){ 15}}
\put(280,645){\framebox(80,60){}}
\put(275,640){\framebox(90,70){}}
\put(280,690){\line( 1, 0){ 80}}
\put(280,660){\line( 1, 0){ 80}}
\put(200,755){\line( 0,-1){ 20}}
\put(180,755){\line( 0,-1){ 10}}
\put(180,745){\line(-1, 0){100}}
\put( 80,745){\line( 0,-1){ 10}}
\put(220,755){\line( 0,-1){ 10}}
\put(220,745){\line( 1, 0){100}}
\put(320,745){\line( 0,-1){ 10}}
\put(320,725){\line( 0,-1){ 15}}
\put( 80,725){\line( 0,-1){ 15}}
\multiput(260,800)(0.40000,0.40000){26}{\makebox(0.4444,0.6667){\SetFigFont{7}{8.4}{rm}.}}
\multiput(270,810)(0.40000,-0.40000){26}{\makebox(0.4444,0.6667){\SetFigFont{7}{8.4}{rm}.}}
\put(280,800){\line(-1, 0){ 20}}
\put(200,640){\line( 0,-1){ 20}}
\put(200,620){\line( 1, 0){ 45}}
\put(245,620){\line( 0,-1){  5}}
\put(210,640){\line( 0,-1){ 15}}
\put(210,625){\line( 1, 0){130}}
\put(340,625){\line( 0,-1){ 10}}
\put(220,640){\line( 0,-1){ 10}}
\put(220,630){\line( 1, 0){175}}
\put(395,630){\line( 0,-1){115}}
\put(290,695){\makebox(0,0)[lb]{\smash{\SetFigFont{10}{12.0}{rm}myAppModel}}}
\put(285,800){\makebox(0,0)[lb]{\smash{\SetFigFont{10}{12.0}{rm}= Aggregration}}}
\put(285,785){\makebox(0,0)[lb]{\smash{\SetFigFont{10}{12.0}{rm}     (has a)}}}
\end{picture}

\end{latexonly}
\normalfont\normalsize
\end{center}

\caption{Standard V Application} \label{fig:stdvapp}

\end{figure}

\begin{description}

\item [The Application]

In many ways, the heart of a Standard \V\ Application is the
application class derived from the \code{vApp} class. By
convention, this derived class is called \code{myApp} (but you
can use a different name if you want.) There will always be
exactly one instance of the \code{myApp} class.  The \code{myApp}
class acts as a coordinator between the windows that implement
the user interface (the views) and the objects and algorithms
that actually make up the application (the model).  The \code{myApp}
class will contain in a whole/part (or aggregation) relationship
the windows defined by the application, as well as any classes
needed to implement the application.

The \code{vApp} class has several utility methods that are
usually used unmodified, plus several methods that are usually
overridden by the \code{myApp} class.  These are described in the
section covering \code{vApp}.  In addition, your \code{myApp}
class will usually have several other programmer defined methods
used to interface the command windows with the application model.

\item [Windows and Canvases]

Each Standard \V\ Application will have at least one top level
window, and possible subwindows. These will usually be
command windows derived from the \code{vCmdWindow} class. Your
main derived class should be called \code{myCmdWindow}, and
include a constructor that defines a menu bar, a canvas, and
possible command and status bars. Of course, there will be a
corresponding destructor. The \code{.cpp} file will contain the
static definitions of the menu and any command and status bars.
It will also override the \code{WindowCommand} method of \code{vCmdWindow}
superclass. In your \code{WindowCommand} method, you will have a
\code{switch} with a \code{case} for each menu item and button
defined for the window.

Since a \code{vCmdWindow} contains different panes such as
\code{vMenuPanes}, \code{vCanvasPanes}, \code{vCommandPanes}, and
\code{vStatusPanes}, your top level command window object will
usually define the appropriate pointers to each of these objects
as required by the specific application. The \code{myCmdWindow}
constructor will then have a \code{new} for each pane used.

Each instance of a window will be built using a call to the
\code{vApp::NewAppWin} method. This allows the app object
to track windows, and control interaction between the app
model and the views represented by each window. 

Some applications need to open subwindows. These windows may or
may not use the same menu, command bar, and canvas as the top
level window. If they do, then they can use the same static
definitions used by the top level window. Subwindows may also
have their own menu, button, and canvas definitions.

\item [Canvases for Windows]

Since each window usually needs a canvas, you will usually derive
a canvas object from the \code{vCanvasPane} class. At
this point in the life of \V, there are only two possible kinds
of canvas.  The first is for graphics drawing, and is
derived directly from the \code{vCanvasPane} class.  The other
kind is a text canvas derived from the \code{vTextCanvasPane}
class. The derived class will define override methods required
for the user to interact with the canvas.

\item [Optional Dialogs]

Most applications will need dialogs -- either modeless or modal.
A Standard \V\ dialog consists of a \code{.cpp} file with the
static definition of the dialog commands, and the definitions
of methods derived from the \code{vDialog} class. These will
include a constructor and destructor, and a \code{DialogCommand}
override with a \code{switch} with a \code{case} for each command
defined for the dialog. Each \code{case} will have the code
required to carry out useful work.

The top level window (or the subwindow that
defines and uses the dialog) will create an instance of each
dialog it needs (via \code{new}). The constructor for the
dialog sets up the commands used for the dialog.

Typically, the top level window defines menu and button commands
that result in the creation of a dialog. The top level window
is thus usually responsible for invoking dialogs.

\item [Optional Modal Dialogs]

Modal dialogs are almost identical to modeless dialogs. The main
difference is how the dialog is invoked from the defining window.

\item [Menu, Command and Status Bars]

By definition, the look and feel of a \V\ application requires
a menu bar on the command window. A \V\ application also typically
has a command bar and a status bar, but these are not required.

\item [The Application Model]

Each application will need code to implement its data structures
and algorithms. The design of the application model is beyond the
scope of \V, but will usually be defined as a relatively
independent hierarchy contained by the \code{myApp} object.
Interaction between the application model and the various
views represented by \code{myCmdWindows} can be coordinated with
the \code{myAppWinInfo} class.

\item [The Makefile]

Each \V\ Standard Application should have an associated \code{makefile}
that can be used to compile and link the application.

\end{description}

Please note that while \V\ is object-oriented, the objects
represent real screen windows. Thus, it makes no sense for most
\index{copy constructors}
\index{object assignment (=)}
\V\ objects to support copy constructors or object assignment. If
you use one of these \V\ objects in a way requiring a copy
constructor or an assignment (fortunately, it is difficult to
contrive such an example), the code will generate a run time
error.

\section{Special V Applications}

\subsection*{Windows MDI/SDI}
\index{MDI model}\index{SDI model}

The basic standard \V application includes a command window
with a menu, a command bar, a canvas, and a status bar. While
this model suits most applications, there are some special
cases that \V supports.

First, on Windows, \V supports the standard Windows MDI model
(Multiple Document Interface)
by default. The MDI model consists of a parent window that
can contain several children canvases, each with a different
menu that changes in the main parent window when a child
gets focus. In practice, the menus are usually the same for
all children windows, and each window is used to hold a
new document or data object. One of the main advantages
of the MDI model is that each application has a main window
to distinguish it from other Windows applications, and
as many child windows as it needs to manipulate its data.

On X versions, there is no need for a special parent window.
Each time you open a new command window, you get a new window
on the X display.

The Windows MDI model forces some screen decorations that
are not appropriate for all applications. Thus, \V also supports
the standard Windows SDI model. The SDI model allows only one
canvas/command window combination. There is a parameter to
the \code{vApp} constructor that tells \V to use the SDI model.
This parameter is not used on the X version.

\subsection*{Canvasless, menuless V Application}

Sometimes an application needs just a command bar with no menu or
canvas. By setting the \code{simSDI} parameter to 1, and
supplying a width and height value to the \code{vApp}
constructor, \V allows this kind of simple interface. Instead of
adding a menu and a canvas as is done for normal \V apps, a
menuless and canvasless app just defines a command pane for the
command bar. The height and width are used to specify the height
and width of the application, and require different values for
Windows or X.

\section{A Tutorial Example V Application}
\index{tutorial example}
\label{sec:tutexamp}

Now that you've read about the parts of a standard \V\
application, it might be useful to go over a simple example of a
\V\ application. Appendix A contains the source code for a simple
\V\ application. The code is tutorial, and well commented. You
can read the code directly and get a good understanding of what
elements are required for a \V\ application. This section will
give a higher level overview of the code in Appendix A.

You should read this code, paying special attention to the comments.
Most of the information you need to build a typical \V\
application is explained in this code. This sample code is also
available on line under the \code{\tild/v/tutor} directory.
The source code of a slightly different standard \V\ application
is included the \code{\tild/v/examp} directory of the \V\
distribution.

The previous section suggested using \code{myApp} for names. This
tutorial uses a \code{t} prefix instead of \code{my}. You really
can use whatever names you want. It will help to be consistent,
however.

The code is broken down into five sections, corresponding to the
main application, the main window, a simple canvas, and  modal
and modeless dialogs. The source code for each of these parts is
included in Appendix A. The source code is extensively commented,
and the comments contain much detail on how you should structure
a \V\ application, so please read them carefully. The following
sections give a brief overview of each source file included in
the tutorial example.

\subsection*{The Base Application Class}
\Indextt{vApp}\index{base application class}

The file \code{tutapp.cpp} contains the overridden definitions of
the classes \code{NewAppWin}, \code{Exit}, \code{Close\-App\-Win},
\code{App\-Command}, and \code{KeyIn} methods. These examples
don't do much work, but are provided as a template for building
complete applications.

The single definition of the application (\code{static} \code{tutApp}
\code{tut\_App("TutorApp")}, and the \code{AppMain} main program
are also in this file. The initial window is created in \code{AppMain}
by calling \code{NewAppWin}.

One thing that can be difficult to grasp when using a
framework such as \V\ is understanding where the program starts,
and how you get things rolling. This happens in \code{tutapp.cpp},
so it is especially important to understand this piece of code.
The essential thing to understand is that C++ will invoke the
constructors of static objects before beginning execution of the
program proper. Thus, you declare a static instance of the
\code{vApp} object, and its constructor is used to initialize the
native GUI library and get things going. Your program will
\emph{not} have a \code{main} function (see \code{AppMain} in the
description of the \code{vApp} class for more details).

As with all files in the tutorial, each has a \code{.cpp} source
file, and its associated \code{.h} header file. All \V\ code has
been written using the coding guidelines given in Appendix B.
This includes the order of the declarations included in header
files.


\subsection*{The Command Window}
\index{window}\index{command window}\Indextt{vCmdWindow}

The file \code{tcmdwin.cpp} contains the code for the main
command window. Of particular interest are the definitions of
the main menu, command pane, and status pane. These panes are
defined and added to the window in the constructor.

There is also code to demonstrate handling keyboard and window
command events in the \code{KeyIn} and \code{WindowCommand} methods.
There is also a simple example of using the \code{vFileSelect}
utility class, as well as invoking modeless and modal dialogs.

\subsection*{The Canvas}
\index{canvas}\Indextt{vCanvas}

The file \code{tcanvas.cpp} contains the code for the canvas.
This is a really simple canvas example which supports
drawing a few lines. This class handles redrawing after
expose events very simply, but demonstrates what must be done
in general.


\subsection*{A Modeless Dialog}
\index{dialog}

The file \code{tdialog.cpp} contains the code for a modeless
dialog. There are just a few example buttons, check boxes, and
radio buttons. The \code{DialogCommand} methods demonstrates how
to handle commands from a dialog.


\subsection*{A Modal Dialog}

The file \code{tmodal.cpp} contains the code for a modal dialog.
The definition of a modal dialog is nearly identical to a modeless
dialog. The main difference is how they are invoked, which is
shown in the \code{tcmdwin.cpp} code.


\subsection*{The Makefile}
\index{makefile}

The file \code{makefile} contains a sample Unix-style make file.
This version is for Gnu make, which has features different
than some other flavors of make. It should still serve as a
decent example.

%***********************************************************************
%***********************************************************************
%***********************************************************************

\chapter {The Application}

This chapter covers the top level classes used to build an application.

The classes covered in this chapter include:

\begin{description}
	\item[vApp] The base class for applications.
	\item[vAppWinInfo] A utility class to interface views to models.
\end{description}

%------------------------------------------------------------------------
\Class{vApp}
\Indextt{vApp}

The base class for building applications.

\subsection* {Synopsis}

\begin{description}
	\item [Header:] \code{<v/vapp.h>}
	\item [Class name:] vApp
	\item [Contains:] vCmdWindow, vAppWinInfo
\end{description}

\subsection* {Description}

The \code{vApp} class serves as the base class for building
applications. There must be exactly one instance of an object
derived from the \code{vApp} class.  The base class contains and
hides the code for interacting with the host windowing system,
and serves to simplify using the windowing system.

You will usually derive a class based on \code{vApp} that will
serve as the main control center of the application, as well as
containing the window objects needed for the user interface.
The single instance of the application class is defined
in the body of the derived application class code.

The \code{vApp} class has several utility methods of general
usefulness, as well as several methods that are normally
overridden to provide the control interface from the application
to the command windows. The derived class will
also usually have other methods used to interface with the
application.

In order to simplify the control interface between the application
and the windows, the \code{vAppWinInfo} class has been provided.
The application can extend that class to keep track of
relevant information for each window. When the \code{NewAppWin}
method is used to create a window, it will create an
appropriate instance of a \code{vAppWinInfo} object, and return
a pointer to the new object. The
base \code{vApp} then provides the method \code{getAppWinInfo}
to retrieve the information associated with a given window.

\subsection* {Constructor} %------------------------------------

\Meth{vApp(char* appName)}
\Meth{vApp(char* appName, simSDI = 0, int fh = 0, int fw = 0)}
\Indextt{vApp}

\Param {appName} Default name for the application. This name
will be used by default when names are not provided for windows.
The name also appears on the ``main window'' for some platforms,
including Microsoft Windows, but not X.
The constructor also initializes some internal state information.
There must be exactly one instance of the \code{vApp} object, and
will usually represent your derived \code{myApp} object. See the
code below with \code{AppMain} for an example of creating the
single app instance.

\Param {simSDI} This \emph{optional} parameter is used to specify
that \V should start as a Windows SDI application if it is set
to 1. This parameter has no effect for the X version.

\Param{fw, fh} These are used to specify the size of a
menuless and canvasless \V application, and are optional.

\subsection* {Methods to Override} %------------------------

%............................................................
\Meth{void AppCommand(vWindow* win, ItemVal val)}
\Indextt{AppCommand}

Any window commands not processed by the \code{vWindow} object
are passed to \code{AppCommand}. You can override this method
to handle any commands not processed in windows.

%............................................................
\Meth{int AppMain(int argc, char** argv)}
\Indextt{AppMain}

This is a global function (not a class member!) that is called
once by the system at start up time with the standard command
line arguments \code{argc} and \code{argv}. You provide this
function in your code.

Your program will not have a C \code{main} function. The main
reason for this is portability. While you would usually have
a \code{main} in a Unix based program, MS-Windows does not
use \code{main}, but rather \code{PASCAL WinMain}. By handling
how the program gets started and providing the \code{AppMain}
mechanism, \V\ allows you to ignore the differences. You will
still have all the capability to access the command line arguments
and do whatever else you would do in \code{main} without having
to know about getting the host windowing system up and running.

The windowing system will have been initialized before \code{AppMain}
is called. You can process the command line arguments, and
perform other required initializations. The top level command
window should also created in \code{AppMain} by calling \code{NewAppWin}.

Before \code{AppMain} is called, the single instance of your
derived \code{vApp} object must also be constructed, usually by
instantiating a static instance with a statement such as \code{static
myApp* MyApp = new myApp("ProtoApp")}. As part of the
construction of the \code{myApp} object, the global pointer
\code{vApp* theApp} is also pointed to the single instance of the
\code{vApp} or derived \code{myApp} object. You can then use
\code{theApp} anywhere in your code to access methods provided by
the \code{vApp} class.

Your \code{AppMain} should return a 0 if it was successful. A
nonzero return value will cause the \V\ system to terminate with
an exit code corresponding to the value you returned.

\subsection* {Example}

\footnotesize
\begin{verbatim}

// EVERY V application needs the equivalent of the following line

  static myApp* MyApp = new myApp("ProtoApp");  // Construct the app.

//==========================>>> AppMain <<<===========================
  int AppMain(int argc, char** argv)
  {
    // Use AppMain to perform special app initialization, and
    // to create the main window.  This example assumes that
    // NewAppWin knows how to create the proper window.

    (void) theApp->NewAppWin(0, "Prototype V Example", 350, 100, 0);
    return 0;
  }
\end{verbatim}

\normalfont\normalsize

%............................................................
\Meth{void CloseAppWin(vWindow* win)}
\Indextt{CloseAppWin}

This is the normal way to close a window. Your derived \code{CloseAppWin}
should first handle all housekeeping details, such as saving the
contents of a file, and then call the default \code{vApp::CloseAppWin}
method. The default method calls the window's \code{CloseWin}
method and removes the window.

The \code{CloseAppWin} method is also called when the user clicks
the close button of the window.  This close button will
correspond to the standard close window button depending on the
native windowing system. On the X Window System, each window will
have a close button in the menu bar. On Windows, this corresponds
to a double click on the upper left box of the title bar, or
the ``X'' box in Windows 95.

%............................................................
\Meth{void Exit(void)}
\Indextt{Exit}

This is the normal way to exit from a standard \V\ application. The
overridden method can perform any special processing (e.g.,
asking ``Are you sure?'') required. The default \code{Exit} will
call \code{CloseAppWin} for each window created with \code{NewAppWin},
and then exit from the windowing system.

%............................................................
\Meth{void KeyIn(vWindow* win, vKey key, unsigned int shift)}
\Indextt{KeyIn}

Any input key events not handled by the \code{vWindow} object are
passed to \code{VApp::KeyIn}. See \code{KeyIn} in the \code{vWindow}
section for details of using keys.

%............................................................
\Meth{vWindow* NewAppWin(vWindow* win, char* name, int w, int h,
  vAppWinInfo* winInfo)}
\Indextt{NewAppWin}

The purpose of the \code{NewAppWin} method is to create a
new instance of a window. Most likely, you will override
\code{NewAppWin} with your own version, but you still \emph{must}
call the base \code{vApp::NewAppWin} method \emph{after} your
derived method has completed its initializations.

The default behavior of the base \code{NewAppWin} class is to set
the window title to \code{name}, and the width \code{w} and
height \code{h}. Note that the height and width are of the
\emph{canvas}, and not necessarily the whole app window. If you
don't add a canvas to the command window, the results are not
specified. Usually, your derived \code{NewAppWin} will create an
instance of your derived \code{vCmdWindow} class, and you will
pass its pointer in the \code{win} parameter. If the the \code{win}
parameter is null, then a standard \code{vCmdWindow} will be
created automatically, although that window won't be particularly
useful to anyone.

Your \code{NewAppWin} class may also create an instance of your
derived \code{vAppWinInfo} class. You would pass its pointer to the
\code{winInfo} parameter. If you pass a null, then the base \code{NewAppWin}
method also  creates an instance of the standard \code{vAppWinInfo}
class.

The real work done by the base \code{NewAppWin} is to register
the instance of the window with the internal \V\ run time system.
This is why you must call the base \code{NewAppWin} method.

\code{NewAppWin} returns a pointer to the object just created. Your
derived code can return the value returned by the base
\code{vApp::NewAppWin}, or the pointer it created itself.

\subsection* {Example}

The following shows a minimal example of deriving a \code{NewAppWin}
method.

\footnotesize

\begin{verbatim}
vWindow* myApp::NewAppWin(vWindow* win, char* name, int w, int h,
  vAppWinInfo* winInfo)
  {
    // Create and register a window. Usually this derived method
    // knows about the windows that need to be created, but
    // it is also possible to create the window instance outside.

    vWindow* thisWin = win;
    vAppWinInfo* theWinInfo = winInfo;

    if (!thisWin)   // Normal case: we will create the new window
        thisWin = new myCmdWindow(myname, w, h);  // create window

    // Now the application would do whatever it needed to create
    // a new view -- opening a file, tracking information, etc.
    // This information can be kept in the vAppWinInfo object.

    if (!theWinInfo)               // Create if not supplied
        vAppWinInfo* theWinInfo = new myAppWinInfo(name);

    // Now carry out the default actions
    return vApp::NewAppWin(thisWin, name, w, h, theWinInfo);
  }
\end{verbatim}
\normalfont\normalsize

\subsection* {Utility Methods} %--------------------------------------

%............................................................
\Meth{int DefaultHeight()}
\Indextt{DefaultHeight}

Returns a default window canvas height value in pixels corresponding to 24 lines
of text in the default font.

%............................................................
\Meth{int DefaultWidth()}
\Indextt{DefaultWidth}

Returns a default window canvas width value in pixels corresponding to 80 columns
of text in the default font.

%............................................................
\Meth{vFont GetDefaultFont(void)}
\Indextt{GetDefaultFont}

This method returns a \code{vFont} object representing the default
system font. It is a convenience method, and probably not overly useful
to application programs.

%............................................................
\Meth{vFont GetVVersion(int\& major, int\& minor)}
\Indextt{GetVVersion}

Returns the current major and minor version of \V.

%............................................................
\Meth{int IsRunning()}
\Indextt{IsRunning}

This method returns true if the windowing system is active and
running. A false return means the program was started from
a nonwindowing environment.

%............................................................
\Meth{int ScreenHeight()}
\Indextt{ScreenHeight}

Returns the overall height of the physical display screen in pixels.
Note that this value may or may not be overly useful. On X,
the \code{vCommandWindows} are drawn on the full display. On
the Windows MDI version, the command windows all fall inside the
MDI frame, and thus knowing the size of the whole screen is less
useful.

%............................................................
\Meth{int ScreenWidth()}
\Indextt{ScreenWidth}

Returns the overall width of the physical display screen in pixels.
See \code{ScreenHeight}.

%............................................................
\Meth{void SendWindowCommandAll(ItemVal id, int val, CmdType ctype)}
\Indextt{SendWindowCommandAll}

This method can be used to send a message to the \code{WindowCommand} method
of \emph{ALL} currently active windows. This method is most useful for
sending messages to windows from modeless dialogs. While messages to
the \code{WindowCommand} method usually originate with the system
in response to menu picks or command object selection, it can be
useful to send the messages directly under program control. The
\code{vDraw} sample program contains a good example of using
\code{SendWindowCommandAll} (and \code{SetValueAll}) in \code{vdrwdlg.cpp}.
There is no way to send a message to a specific window. The message
is sent to all active windows.

%............................................................
\Meth{void SetAppTitle(char* title)}
\Indextt{SetAppTitle}

This method is used to set the title of the main application
window. This currently only applies to the Microsoft Windows
MDI version of \V. It is a no-op for the X version. It is still
important that you choose a good title for your main window,
and set it either with this method, or by providing a good
name to the \code{vApp} initializer.

%............................................................
\Meth{void SetValueAll(ItemVal itemId, int Val,
ItemSetType what)}
\Indextt{SetValueAll}

This method is similar to \code{vWindow::SetValue}, except that
the control with the given \code{itemId} in \emph{ALL} currently
active windows is set. This is useful to keep control values
in different windows in sync. The only difference between
\code{vApp :: SetValueAll} and \code{vWindow :: SetValueAll} is
that the \code{vApp} version can be easily called from dialogs
as well as windows.

%............................................................
\Meth{void SetStringAll(ItemVal itemId, char* title)}
\Indextt{SetStringAll}

This method is similar to \code{vWindow::SetString}, except that
the string with the given \code{itemId} in \emph{ALL} currently
active windows is set. This is useful to keep control strings
in different windows in sync. The only difference between the
\code{vApp::SetStringAll} version and the \code{vWindow::SetStringAll}
version is that the \code{vApp} version can be easily called from
dialogs as well as windows.

%............................................................
\Meth{int ShowList()}
\Indextt{ShowList}

This method is intended mostly for debugging, and will print
on \code{stderr} the list of currently registered windows.

%............................................................
\Meth{vAppWinInfo *getAppWinInfo(vWindow* win)}
\Indextt{getAppWinInfo}

This method provides an easy way to retrieve the \code{vAppWinInfo}
(or more typically, a derived class) object that is associated
with a window.  By convention, when a window is first created,
it and its associated \code{vAppWinInfo} object are tracked by
\code{NewAppWin}. When a user action in a window causes a method
in \code{vApp} to be invoked, the \code{this} of that window is
usually sent to the \code{vApp} method.  You then use that
\code{vWindow} pointer to call \code{getAppWinInfo} to get a
pointer to the associated \code{vAppWinInfo} object. It will be
up to you to determine what information that object has, and how
to use it.

\subsection* {Tasking} %--------------------------------------
\index{multitasking}\index{tasking}\index{background processing}
\index{compute bound applications}

Some applications may have extensive computation requirements. In
traditional programming environments, this is usually no problem.
However, for GUI based applications, the code cannot simply
perform extensive computation in response to some command event
(such as a "Begin Computation" menu command). GUIs make a basic
assumption that the application will process events relatively
quickly. While computation is in process, the application will
not receive additional events, and may appear to hang if the
computation is too long.

\V\ provides two different approaches to handling compute bound
applications. The most straight forward approach is to have
the computation periodically call the \V\ method \code{vApp::CheckEvents}.
\code{CheckEvents} will process events, and pass the messages to the
appropriate \V\ method. This method may be the most appropriate
for applications such as simulations.
\index{simulations}

The second technique is to have the \V\ system call a work procedure
periodically to allow some computation to be performed. This technique
may be most appropriate for applications that have short computations
that should be performed even if the user is not entering commands
or interacting with the application. The technique is supported by
the \code{WorkSlice} method.

\Meth{CheckEvents()}
\Indextt{CheckEvents}

Most \V\ applications will \emph{not} need this utility. However, it is possible
for some compute bound applications to lock out system response to the
events needed to update the screen. If you notice that your application
stops responding to input, or fails to consistently update items
in your window, then place calls to \code{vApp::CheckEvents()} in
your code somewhere. You may have to experiment how often you need
to call it. It does have some overhead, so you don't want it to
slow down your app. But it does need to get called enough so the
system can keep up with the screen updates. This function needs
no parameters, and returns no value.

\Meth{EnableWorkSlice(long slice)}
\Indextt{EnableWorkSlice}

For applications that need computations to be performed continuously
or periodically, even while the user is not interacting with the program,
\V\ provides \code{EnableWorkSlice} and \code{WorkSlice}. After
\code{EnableWorkSlice} has been called, \V\ will call the app's
\code{WorkSlice} method every \code{slice} milliseconds.
The \code{WorkSlice} method of every open \code{vCommandWindow}
will also be called. Calling \code{EnableWorkSlice} with a zero
value will stop the calls to the \code{WorkSlice} methods.

\V\ uses a standard \V\ \code{vTimer} object to implement this
behavior. Thus, all of the information about actual time intervals
and limits on the number of timers discussed in the \code{vTimer}
description apply to \code{EnableWorkSlice} and \code{WorkSlice}.

\Meth{WorkSlice()}
\Indextt{WorkSlice}

When a \code{EnableWorkSlice} has been called with a positive value,
\V\ calls \code{vApp::WorkSlice} at approximately the specified
interval (or more
likely, the overridden method in your app), as well as the
\code{vWindow::WorkSlice} method of each open \code{vCommandWindow}.
Your application can override the appropriate \code{WorkSlice}
method to perform short, periodic computations. Theses
computations should be shorter than the time interval specified for
\code{EnableWorkSlice}. This may be difficult to ensure since
different processors will work at different speeds. One simple
way to be sure you don't get multiple calls to the \code{WorkSlice}
method is to set a static variable on entry to the code. Note
that \code{vCommandWindow} also has a \code{WorkSlice} method.
The \code{WorkSlice} for the \code{vApp} is called first,
followed by a call to each open \code{vCommandWindow}
sequentially in no specific order.

%............................................................

\subsection* {See Also}

vWindow, vAppWinInfo

%-------------------------------------------------------------------------

\Class{vAppWinInfo}
\Indextt{vAppWinInfo}

A utility class to interface views to models.
\subsection* {Synopsis}

\begin{description}
	\item [Header:] \code{<v/vawinfo.h>}
	\item [Class name:] vAppWinInfo
\end{description}

\subsection* {Description}

This class is intended to be used as a base class for deriving
your own \code{myAppWinInfo} class. The purpose of such a class is
to serve as a controller data base for the MVC architecture. Typically,
you would keep an \code{AppWinInfo} record for each view (open window)
of the file or whatever represented by the model. For example, if
your application were a text file viewer, the \code{AppWinInfo} record
could track the file name, current line number, and viewing mode
for each view of a file.

\V\ makes using a \code{AppWinInfo} object easier by
automatically tracking it when you create each new window with
\code{NewAppWin}. You can then easily retrieve the \code{AppWinInfo}
object associated with each window by using the \code{vApp::getAppWinInfo}
method.

\subsection* {Constructor} %------------------------------------

\Meth{vAppWinInfo(char* infoName = "", void* ptr = 0)}
\Indextt{vAppWinInfo}

You can provide two values for the \code{vAppWinInfo} constructor.
The first is a pointer to a character string which you can use
to store some name meaningful to you application. The second is a
\code{void *} pointer, and can be used to point to anything you
want. The constructor makes a copy of the name string, but just
copies the void pointer and does not copy the object pointed to.

\subsection* {Utility Methods}

%............................................................
\Meth{virtual char* infoName()}
\Indextt{infoName}

Returns a pointer to the name supplied to the constructor.

%............................................................
\Meth{virtual void* getPtr()}
\Indextt{getPtr}

Returns the value of the pointer name supplied to the constructor.


\subsection* {See Also}

vApp

%***********************************************************************
%***********************************************************************
%***********************************************************************

\chapter {Command Windows}
\index{command window}\index{window}

This chapter covers the classes used to build windows and command
windows.

The classes covered in this chapter include:

\begin{description}
	\item[vCmdWindow] A class to show a window with various command panes.
	\item[vCommandPane] Used to define commands on a command bar.
	\item[vMenu] Used to define pull down menus.
	\item[vPane] Base class for various window panes.
	\item[vStatus] Used to define label fields on a status bar.
	\item[vWindow] A class to show a window on the display.
\end{description}

%------------------------------------------------------------------------

\Class{vCmdWindow}
\Indextt{vCmdWindow}

A class to show a window with various command panes.

\subsection* {Synopsis}

\begin{description}
	\item [Header:] \code{<v/vcmdwin.h>}
	\item [Class name:] vCmdWindow
 	\item [Hierarchy:] vBaseWindow \rta vWindow \rta vCmdWindow
	\item [Contains:] vDialog, vPane
\end{description}

\subsection* {Description}

The \code{vCmdWindow} class is derived from the \code{vWindow}
class. This class is intended as a class that serves as
a main control window containing various \code{vPane} objects
such as menu bars, canvases, and command bars. The main 
difference between the \code{vCmdWindow} class and the
\code{vWindow} class is how they are treated by the host
windowing system. You will normally derive your windows from
the \code{vCmdWindow} class.

\subsection* {Constructor} %------------------------------------

%............................................................
\Meth{vCmdWindow(char* title)}
\Indextt{vCmdWindow}

\Meth{vCmdWindow(char* title, int h, int w)}

These construct a \code{vCmdWindow} with a title and a size specified
in pixels. You can use \code{theApp->DefaultHeight()} and
\code{theApp->DefaultWidth()} in the call to the constructor to
create a ``standard'' size window. Note that the height and width are
of the canvas area, and not the entire window.

\subsection* {Inherited Methods} %------------------------------

See the section \code{vWindow} for details of the following methods.

\Meth{virtual void KeyIn(vKey key, unsigned int shift)}
\Indextt{KeyIn}
\Meth{virtual void MenuCommand(ItemVal itemId)}
\Indextt{MenuCommand}
\Meth{virtual void WindowCommand(ItemVal Id, ItemVal Val, CmdType Type)}
\Indextt{WindowCommand}

\Meth{virtual void AddPane(vPane* pane)}
\Indextt{AddPane}
\Meth{virtual void GetPosition(int\& left, int\& top, int\& width, int\& height)}
\Indextt{GetPosition}
\Meth{virtual int GetValue(ItemVal itemId)}
\Indextt{GetValue}
\Meth{virtual void RaiseWindow(void)}
\Indextt{RaiseWindow}
\Meth{virtual void ShowPane(vPane* wpane, int OnOrOff)}
\Indextt{ShowPane}
\Meth{virtual void SetValue(ItemVal itemId, int Val, ItemSetType what)}
\Indextt{SetValue}
\Meth{virtual void SetString(ItemVal itemId, char* title)}
\Indextt{SetString}
\Meth{virtual void SetTitle(char* title)}
\Indextt{SetTitle}

\Meth{virtual void CloseWin()}
\Indextt{CloseWin}

\subsection* {See Also}

vWindow

%------------------------------------------------------------------------

\Class{vCommandPane}
\Indextt{vCommandPane}

Used to define commands on a command bar.

\subsection* {Synopsis}

\begin{description}
	\item [Header:] \code{<v/vcmdpane.h>}
	\item [Class name:] vCommandPane
	\item [Used by:] vCmdWindow
\end{description}

\subsection* {Description}

A command pane is a horizontal bar in a command window that
holds \code{CommandObjects}.  You can use any of the
\code{CommandObjects}, although they all might not make sense
to use on a command bar (a List, for example, is a bit large for
the visual paradigm, but it would work). The layout
is left to right, so you don't need to fill in the RightOf
and Below fields. You can include Frames in a command bar,
and commands contained in that frame do use the RightOf
and Below attributes. 

You define the commands on a command bar using a
\code{CommandObject} array. You first create the command pane
with \code{myCmdPane = new vCommandPane(CommandBar)}, and
then add it to the window with \code{AddPane(myCmdPane)}.

You then handle the command objects in a command bar pretty much
like the same way as in a dialog. The main difference is that you
use the \code{vWindow} versions of \code{SetValue} and \code{WindowCommand}
instead of the corresponding methods of the \code{vDialog} class.
Other than the left to right ordering, things are pretty much the
same.

\subsection* {Example}

The discussion of \code{CommandObject} and \code{vDialog} contains
several examples of defining command objects.

See the section \code{vPane} for a general description of panes.

\subsection* {See Also}

vWindow, vStatus, CommandObject, vDialog, vPane

%------------------------------------------------------------------------
\Class{vMenu}
\Indextt{vMenu}

Used to define pull down menus.

\subsection* {Synopsis}

\begin{description}
	\item [Header:] \code{<v/v\_menu.h>}
	\item [Type name:] vMenu
\end{description}

\subsection* {Description}

The \code{vMenu} structure is used to define pulldown menus,
which includes the top level item on the menu bar, as well as the
items contained in the pulldown menu. These are passed to the
constructor of a \code{vCmdWindow} type object.

See the section \code{vPane} for a general description of panes.

\subsection* {Definition}

\footnotesize
\begin{verbatim}
typedef struct vMenu
  {
    char* label;       // The label on the menu
    ItemVal menuId;    // A User assigned unique id
    unsigned
     sensitive : 1,    // If item is sensitive or not
     checked : 1;      // If item is checked or not (*)
    char* keyLabel;    // Label for an accelerator key (*)
    vKey accel;        // Value of accelerator key
    vMenu* SubMenu;    // Ptr to a submenu 
    unsigned int kShift; // Shift state of accelerator
  } MenuItem;
\end{verbatim}
\normalfont\normalsize

Note that the items marked with an asterisk (\code{checked} 
and \code{keyLabel}) are not used when defining the top 
level menu bar items.

\subsection* {Structure Members}

\Param{char* label} The label on the menu.
See the description of the \code{vWindow} class for
information on setting the label of menu bar items.

For some platforms (Windows, but not Athena X),
\index{menu shortcut}
you can add a \& to indicate a shortcut for the command.
For example, specifying a label \code{\&File} allows
Windows users to pull down the \code{File} menu
by pressing \code{Alt-F}, and specifying a submenu label as
\code{\&New} allows the user to use \code{Alt-N} to select
the \code{New} command. The Athena version of \V\ strips
the \&, so you can (and probably should) denote shortcuts
for menu items even in Athena versions.

\Param{ItemVal MenuId} A user assigned unique id. This id is
\index{ItemVal}
passed to the \code{MenuCommand} (or \code{WindowCommand}) method
when a menu item is selected.  If a menu item with a submenu is
selected, \V\ will not return the id, but will cause the submenu
to be displayed.

It will be common practice to use the same id for menu items and
command objects defined on a command bar, and the same id value would then
be passed to \code{WindowCommand} for either the menu selection
or the equivalent button selection. Similarly, using the id to
set the item's sensitivity will change both the menu and the
button.

The values you use for your id in menus and controls should
be limited to being less than 30,000. The predefined
\V\ values are all above 30,000, and are reserved. \emph{
There is no enforcement of this policy.} It is up to you
to pick reasonable values.

If you want a separator line on a pulldown menu, you must use
the predefined value \code{M\_Line} for the \code{MenuId}.

\Param{int sensitive} Controls if item is initially sensitive or
not. Insensitive items are displayed grayed out. The predefined
symbols \code{notSens} and \code{isSens} can be used to define
the \code{MenuItem}. Note that \V\ uses the static definition of
the \code{MenuItem} to store the current sensitive state, and all
menus (or windows) sharing the same static definition will have
the same sensitive state. See the description of the \code{vWindow}
class for information on setting the sensitivity of menu bar
items.

\Param{int checked} The user can put a check mark in front of the
label of a menu item. This convention is often used to show a
given setting is in effect. Like the sensitive member, this
statically tracks the checked state. The predefined values
\code{isChk} and \code{notChk} can be used to specify this value.
This value is not used when defining the top level menu bar, and
you can use the predefined value \code{notUsed} for that case.
See the description of the \code{vWindow} class for information
on setting checked state of menu items.

\Param{char* keyLabel} Label for an accelerator key. The
predefined symbol \code{noKeyLbl} can be used to indicate there
is no \code{keyLabel}. This value is not used when defining the
top level menu bar, and you can use the predefined value \code{notUsed}
accelerator key.

\Param{vKey accel} This is the value of the keystroke that
is the accelerator for this menu item. When the user presses
this key, the \code{vWindow::MenuCommand} method will be
called just as though the user had used the mouse to select
the menu item. This value may be used in combination with the
\code{kShift} and \code{keyLabel} parameters. See the
explanation of \code{vWindow::KeyIn} for a complete
explanation of key codes.

Note that the Windows version really doesn't support
\code{Alt} key codes. The Windows system intercepts
Alt keys and tries to interpret them as menu accelerators.
Unfortunately, there is no simple way to override this
behavior, so Alt keys are essentially unsupported on Windows.
Using functions keys with combinations of Shift and Control
is supported, as are regular control keys.

\Param{MenuItem* SubMenu} Pointer to another \code{MenuItem}
definition of a submenu. \V\ will cause submenus to be shown
automatically when selected. The predefined symbol \code{noSub}
can be used to indicate there is no submenu.

\Param{unsigned int kShift} This is the shift value to
be used with the \code{accel} key definition. To use
\code{Ctrl-D} as the accelerator key, you would specify the
value for Control-D (easily specified as \code{'D'-'@'}) for
\code{accel}, and leave \code{kShift} set to zero. If you use a
Ctrl code, you must specify both the control code, and the
\code{VKM\_Ctrl} shift code. Note that this value is at the end
of the \code{vMenu} structure because of it was forgotten in
early implementations of \V. By placing it at the end, earlier
versions of \V code are compatible with no changes to the source.
Sigh, I didn't get this one right.

\subsection* {Example}

This example defines a menu bar with the items \emph{File} and
\emph{Edit}. The \code{MenuBar} definition would be passed to the
constructor of the appropriate \code{vCmdWindow} derived object.

\vspace{.1in}
\small

\begin{rawhtml}
<IMG BORDER=0 ALIGN=BOTTOM ALT="" SRC="../fig/menubar.gif">
\end{rawhtml}

\begin{latexonly}
\setlength{\unitlength}{0.012500in}%
\begin{picture}(230,105)(5,730)
\thicklines
\put(  5,815){\framebox(230,20){}}
\put( 15,815){\line( 0,-1){ 85}}
\put( 15,730){\line( 1, 0){ 55}}
\put( 70,730){\line( 0, 1){ 85}}
\put( 15,750){\line( 1, 0){ 55}}
\put( 25,800){\makebox(0,0)[lb]{\smash{\SetFigFont{10}{12.0}{rm}New}}}
\put( 25,785){\makebox(0,0)[lb]{\smash{\SetFigFont{10}{12.0}{rm}Open}}}
\put( 25,770){\makebox(0,0)[lb]{\smash{\SetFigFont{10}{12.0}{rm}Save}}}
\put( 25,755){\makebox(0,0)[lb]{\smash{\SetFigFont{10}{12.0}{rm}Save As}}}
\put( 25,820){\makebox(0,0)[lb]{\smash{\SetFigFont{10}{12.0}{rm}File}}}
\put( 55,820){\makebox(0,0)[lb]{\smash{\SetFigFont{10}{12.0}{rm}Edit}}}
\put( 25,735){\makebox(0,0)[lb]{\smash{\SetFigFont{10}{12.0}{rm}Exit}}}
\end{picture}

\end{latexonly}

\normalfont\normalsize
\vspace{.1in}

Only the File submenu is shown here, and is an example of the
menu as it might be included in a standard File menu. Note that this
example menu includes items that can all be specified by using
standard predefined values (see \emph{Predefined ItemVals}). It
also includes an optionally defined \code{Debug} item. A
definition like this might be used for the \code{FileMenu} in
the \code{Menu} example. Note that \& is used to denote shortcuts
for menu items.

\footnotesize
\begin{verbatim}
static vMenu FileMenu[] =
  {
    {"&New", M_New, isSens,notChk,noKeyLbl,noKey,noSub},
    {"&Open", M_Open, isSens,notChk,noKeyLbl, noKey, noSub},
    {"&Save", M_Save, isSens,notChk,noKeyLbl,noKey,noSub},
    {"Save &As", M_SaveAs, isSens,notChk,noKeyLbl,noKey,noSub},
#ifdef vDEBUG
    {"-", M_Line, notSens,notChk,noKeyLbl,noKey,noSub},
    {"&Debug", M_SetDebug,isSens,notChk,noKeyLbl,noKey,noSub},
#endif
    {"-", M_Line, notSens,notChk,noKeyLbl,noKey,noSub},
    {"E&xit", M_Exit, isSens,notChk,noKeyLbl,noKey,noSub},
    {0}
  };

static vMenu EditMenu[] = {...};  // Define Edit pulldown

// Define menu bar, which includes the File and Edit pulldown
static vMenu MenuBar[] =
  {
    {"&File",M_File,isSens,notUsed,notUsed,noKey,&FileMenu[0]},
    {"&Edit",M_Edit,isSens,notUsed,notUsed,noKey,&EditMenu[0]},
    {0,0}                         // end of menubar
  };

  ...

  vMenuPane myMenuPane = new vMenuPane(MenuBar);  // construct pane
  AddPane(myMenuPane);
\end{verbatim}
\normalfont\normalsize

\subsection* {See Also}

vWindow, vPane

%------------------------------------------------------------------------

\Class{vPane}
\Indextt{vPane}

The \code{vPane} class serves as a base class for various pane
objects contained by the \code{vWindow} class. There are no
methods or services provided by the \code{vPane} class that
you need to use directly, but the class is used extensively by
\V\ internally.

There are four types of panes used by \V\ in a \code{vCmdWindow},
including menu panes, canvas panes, command panes and status
panes. To add a pane to a window, you will first define the
contents of the pane (menu, commands, status info) using static
arrays, then construct an instance of the pane with \code{new
vWhateverPane}. Then you add the instance to the window using
\code{AddPane}.

Note that the canvas panes are described in the \Sect{Drawing}
chapter. The commands used with a command pane are described
in the \Sect{Dialogs} chapter, while menus and status bars
are covered in \code{vMenu} and \code{vStatus} in this chapter. 

\subsection* {Canvas Pane}
\index{canvas pane}
\begin{description}
        \item [Header:] \code{<v/vcanvas.h>}
        \item [Class name:] vCanvasPane
	\item [Constructor:] \code{userCanvasPane()}
\end{description}

\subsection* {Command Pane}
\index{command pane}
\begin{description}
        \item [Header:] \code{<v/vcmdpane.h>}
        \item [Class name:] vCommandPane
	\item [Constructor:] \code{vCommandPane(CommandObject* cmdbar)}
\end{description}

\subsection* {Menu Pane}
\index{menu pane}
\begin{description}
        \item [Header:] \code{<v/vmenu.h>}
        \item [Class name:] vMenuPane
	\item [Constructor:] \code{vMenuPane(vMenu* menubar)}
\end{description}

\subsection* {Status Pane}
\index{status pane}
\begin{description}
        \item [Header:] \code{<v/vstatusp.h>}
        \item [Class name:] vStatusPane
	\item [Constructor:] \code{vStatusPane(vStatus* sbar)}
\end{description}

\subsection* {See Also}

CommandObject, vCanvasPane, vCmdWindow, vCommandPane, vMenu,
vStatus

%--------------------------------------------------------------------

\Class{vStatus}
\Indextt{vStatus}

Used to define label fields on a status bar.

\subsection* {Synopsis}

\begin{description}
	\item [Header:] \code{<v/v\_defs.h>}
	\item [Type name:] vStatus
	\item [Used by:] vWindow
\end{description}

\subsection* {Description}

The \code{vStatus} structure is used to define the top level
status bar included on a \code{vCmdWindow}, and the labels it
contains. The \code{vStatus} array is usually passed to the
\code{vStatusPane} constructor. See the section \code{vPane} for
a general description of panes.

\subsection* {Definition}

\footnotesize
\begin{verbatim}
typedef struct vStatus      // for status bars
  {
    char* label;            // text label
    ItemVal statId;         // id
    CmdAttribute attrs;     // attributes - CA_NoBorder
    unsigned sensitive : 1; // if button is sensitive or not
    int width;              // to specify width (0 for default)
  } vButton;
\end{verbatim}
\normalfont\normalsize

\subsection* {Structure Members}

\Param{char* label} Text of label field. See the description of
the \code{vWindow} class for information on changing the text of
a label.

\Param{ItemVal id} Id for the label. Use this value when changing
value with \code{SetString} or \code{SetValue}.

\Param{CmdAttribute attrs} The current implementation only uses
\index{CA\_NoBorder}
the \code{CA\_NoBorder} attribute. If \code{CA\_NoBorder} is
supplied, the label will be drawn on the command bar without a border
or box around it.  Not supplying \code{CA\_NoBorder} (e.g.,
\code{CA\_None}) will result in a label with a border or box
around it. In general, unbordered labels don't change, and
bordered labels are used to show changing status.

\Param{int sensitive} If label is sensitive or not. Use
predefined symbols \code{isSens} and \code{notSens} to specify
the initial state. On some implementations, the label will be
grayed if it is insensitive. The sensitivity can be changed
using \code{vWindow::SetValue} as described in the section
\code{vWindow}.

\Param{int width} This can be used to specify a fixed width for
a label. Normally, the label will be sized to fit the length of the
text. If you provide a non-zero width, then the label field will
remain constant size.

\subsection* {Example}

This shows a sample status bar with two fields. It is added
to a \code{vCmdWindow} using \code{AddPane}. The value of the
file name would be changed by calling    
\code{SetString(m\_curFile, filename)} somewhere in your program.

\vspace{.1in}
\small
\begin{rawhtml}
<IMG BORDER=0 ALIGN=BOTTOM ALT="" SRC="../fig/statbar.gif">
\end{rawhtml}
\begin{latexonly}
\setlength{\unitlength}{0.012500in}%
\begin{picture}(260,25)(5,810)
\thicklines
\put(  5,810){\framebox(260,25){}}
\put( 80,815){\framebox(100,15){}}
\put( 15,820){\makebox(0,0)[lb]{\smash{\SetFigFont{10}{12.0}{rm}Current file:}}}
\put( 90,820){\makebox(0,0)[lb]{\smash{\SetFigFont{10}{12.0}{rm}sample.txt}}}
\end{picture}

\end{latexonly}
\normalfont\normalsize

\footnotesize
\begin{verbatim}
static vStatus sbar[] =
  {
    {"Current file:", m_curMsg,CA_NoBorder,isSens,0},
    {" ", m_curFile,CA_None,isSens,100},
    {0,0,0,0,0}
  };
  ...
  vStatusPane myStatusPane = new vStatusPane(sbar); // construct
  AddPane(myStatusPane);
\end{verbatim}
\normalfont\normalsize

\subsection* {See Also}

vWindow, vPane

%--------------------------------------------------------------------

\Class{vWindow}
\Indextt{vWindow}

A class to show a window on the display.

\subsection* {Synopsis}

\begin{description}
	\item [Header:] \code{<v/vwindow.h>}
	\item [Class name:] vWindow
 	\item [Hierarchy:] vBaseWindow \rta vWindow
	\item [Contains:] vDialog, vPane
\end{description}

\subsection* {Description}

The \code{vWindow} class is an aggregate class that usually has
associated \code{vPane} objects -- window panes, in other
words. There several kinds of panes, including menu panes,
command bar panes, status panes, and drawing canvas panes. As you
would expect, classes derived from \code{vWindow} also include
panes.

The \code{vWindow} class will probably never be used by your
application - it serves primarily as a superclass for the
\code{vCmdWindow} class. This class may be more useful in
future versions of \V, but for now it is not really useful
by itself. You will typically derive your own class from
\code{vCmdWindow}, and override several of the methods
defined by \code{vWindow} and \code{vCmdWindow}.

Menus and commands in the panes send messages to the \code{Window\-Command}
and \code{Menu\-Command} methods when the user clicks on a command
or menu item contained in the window. The application program can
also change attributes of the various menu items and commands
associated with a window. Canvas panes are designed to handle
their own interaction with the user (mouse events, etc.).

\subsection* {Constructor} %------------------------------------

%............................................................
\Meth{vWindow()}
\Indextt{vWindow}
\Meth{vWindow(char* title)}
\Meth{vWindow(char* title, int h, int w)}
\Meth{vWindow(char* title, int h, int, WindowType wintype)}
\Param {title} Title to place in title bar.
\Param {h,w} The height and width of the window.
\Param {wintype} CMDWINDOW or WINDOW type for window.
\newline
\newline

The constructor for \code{vWindow} is normally called with a
name, size, and possibly a window type. The name will be
displayed in the window's title bar by default. The size is the
initial size of the window's \emph{canvas} work area in pixels.
The type may be \code{CMDWINDOW} or \code{WINDOW}. The constructor
for \code{vCmdWindow} invokes the proper \code{vWindow} constructor.

\subsection* {Methods to Override} %----------------------------

%............................................................
\Meth{virtual void KeyIn(vKey key, unsigned int shift)}
\Indextt{KeyIn}
\index{keyboard input}

\code{KeyIn} is invoked when a key is pressed while a window has
focus. The \code{key} value is the \code{vKey} value of the key
pressed, and \code{shift} indicates the shift state of the key.

Handling the keystroke is not necessarily trivial. Regular ASCII
characters in the range from a Space (0x40) up to a tilde (\tild)
are passed to \code{KeyIn} directly, and shift will be 0, even
for upper case letters. The current version of \V\ does not have
explicit support for international characters, so values between 0x80 and
0xFF are undefined, and correspond to whatever might be the local
convention for the character set.  (This will be one thing for
X and another for Windows - but you can count on the values
for each platform. Thus, you can use non-English characters
on each platform, even though they won't be the same values on X
and Windows. I would like a portable solution for this. If any
non-English users of \V have any ideas about this problem, I'd
like to hear. The choice seems to be between the standard
MS-DOS code page solution and the ANSI character set used
on X platforms. I'm not ready to support multibyte characters
for some time yet.) Values between 0xFF00 and 0xFFFF correspond to the
various function keys and keypad keys found on a typical
keyboard. The standard set by IBM PCs has determined what function keys
are supported by \V\@. The file \code{<v/vkeys.h>} has the
definitions for the key codes supported. 

Besides getting a keycode for the non-ASCII keys, \code{KeyIn}
also gives a shift code corresponding to the Control, Shift, and
Alt modifier keys. (These are defined as \code{VKM\_Ctrl},
\code{VKM\_Shift}, and \code{VKM\_Alt}.) Pressing the F4 key
would return the code for F4 (vk\_F4), while the keystroke Alt-F4
will return the code for the F4 key, and the shift code set to
\code{VKM\_Alt}\@. More than one bit of the shift code can be
set -- the shift values are really bit values. Control keys from
the normal character set (Ctrl-A, etc.) are passed as their true
control code, but \emph{not} the \code{VKM\_Ctrl} shift set.

In addition, you also need to check for the \code{VKM\_Alt}
modifier applied to regular Ascii keys.  The keystroke Alt-K will
be mapped to a \emph{lower case} Ascii 'k' with the \code{VKM\_Alt}
bit set in \code{shift}.  The top row keys (1,2, etc.) can also
be pressed with the  \code{VKM\_Ctrl} bit set in \code{shift},
and your program will need to deal with these.  It will quite
often be the case that your program simply ignores many of these
values.

\code{KeyIn} will also return a value when only a modifier key is
pressed. For example, pressing the Alt key returns a key value of
\code{vk\_Alt}. A macro defined in \code{<v/vkeys.h>}
called \code{vk\_IsModifer(x)} can be used to determine if a key
\code{x} is a modifier. Your program can usually ignore modifier
keys.

If you have defined any keystroke combinations to be accelerators
for menu commands, your program will never see those keystrokes
in \code{KeyIn}. Instead, they are intercepted by the system and
mapped to the appropriate command to pass to the \code{MenuCommand}
method.

Note that the keystrokes are not displayed by the system. It is
up to your program to handle keystrokes and to do something
useful with them.

You should call \code{vWindow::KeyIn} from your derived method
with any keystrokes you don't handle. The \code{vWindow::KeyIn}
method passes these unhandled keystrokes up to the \code{vApp::KeyIn}
method. Thus, you will have the choice of either handling
keystrokes in the window or in the app class.

%............................................................
\Meth{virtual void MenuCommand(ItemVal itemId)}
\Indextt{MenuCommand}
\index{menu commands}

\code{MenuCommand} is called when a menu command is selected.
This virtual function allows menu commands to be distinguished
from other commands in a window, although it is not usually
necessary to do so. The default method simply passes the menu
command along to the \code{WindowCommand} method, so you don't
need to override this method if you don't distinguish between
menu and command events.

%............................................................
\Meth{virtual void WindowCommand(ItemVal Id, ItemVal Val, CmdType Type)}
\Indextt{WindowCommand}
\index{window commands}

This method is invoked when a user activates a command object in
a command pane. The \code{Id} of the command object is passed in
in the \code{Id} field, and the value and type (e.g., C\_Button
or C\_CheckBox) of the command are passed in in the \code{Val}
and \code{Type} parameters. Note that command objects in a
command pane are really no different than the command objects in a
dialog. Most of the discussion for handling these commands is
covered in the sections on dialogs. See \code{vCommandPane} and 
\code{vDialog::DialogCommand} for more details about the values
passed to \code{WindowCommand}.

\code{WindowCommand} is also called by the default \code{MenuCommand}
in response to menu picks. The \code{Id} is the id of the item
that generated the call.

The default behavior of \code{WindowCommand} is to call the
\code{AppCommand} method. However, you will almost always
override the default \code{WindowCommand} method.

\Meth{virtual void WorkSlice()}

See \code{vApp::WorkSlice} for a description of this method.

\subsection* {Utility Methods} %--------------------------------

%............................................................
\Meth{virtual void AddPane(vPane* pane)}
\Indextt{AddPane}

This method is used to add the pane \code{pane} to a window.
Panes will be displayed in the order they are added. You can add
exactly one menu pane, plus canvas, command, and status panes.
You typically first create a given pane (e.g., \code{myPane = new
XPane(PaneDef))}, and then add the pane to the window with
\code{AddPane(myPane)}.

%............................................................
\Meth{void GetPosition(int\& left, int\& top, int\& width, int\& height)}
\Indextt{GetPosition}

Returns the position and size of \code{this} window. These values
reflect the actual position and size on the screen of the window.
On X, this is the whole \code{vCommandWindow} frame. On the
Windows MDI version, it is the size and position of just the
drawing canvas and its scroll bars. The intent of this method
is to allow you to find out where the active window is so
you can move a window, or position a dialog so that it
doesn't cover a window.
It is most useful when used in
conjunction with \code{SetDialogPosition}.

%............................................................
\Meth{virtual int GetValue(ItemVal itemId)}
\Indextt{GetValue}

This method is used to retrieve the value of a menu or command
object in a menu or command pane.  The \code{itemId} is the id of
the item as defined in the menu or command bar definition.
For menu items, this will return the menu checked state.
For other command objects, the value returned will be appropriate
as described in the \Sect{Dialog Commands} section.

%............................................................
\Meth{virtual void RaiseWindow(void)}
\Indextt{RaiseWindow}
\index{top window}\index{window focus}\index{focus}

This method will raise the window to top of all windows on the
display.  Raising a window is often a result of mouse actions of
the user, but this method allows a buried window to be moved to
the top under program control. You will need to track which
window instance you want raised, possibly through the \code{vAppWinInfo}
object.

%............................................................
\Meth{virtual void SetValue(ItemVal itemId, int Val,
ItemSetType what)}
\Indextt{SetValue}

This method is used to change the state of command window items.
The item with \code{itemId} is set to \code{Val} using the
\code{ItemSetType} parameter to control what is set. Not all
command items can use all types of settings. See \code{vWindow::GetValue}
and \code{vDialog::SetValue} for a more complete description.

If a menu item and a command item in the same window share the
same id, they will both be set to the same value (this usually
applies to sensitivity). Only the controls in the window that
sent this message are changed.

%............................................................
\Meth{virtual void SetValueAll(ItemVal itemId, int Val,
ItemSetType what)}
\Indextt{SetValueAll}

This method is similar to \code{SetValue}, except that
the control with the given \code{itemId} in \emph{ALL} currently
active windows is set. This is useful to keep control values
in different windows in sync.

%............................................................
\Meth{virtual void SetPosition(int left, int top)}
\Indextt{SetPosition}

Moves \code{this} window to the location \code{left} and
\code{top}. This function is of limited usefulness.
\code{SetDialogPosition} is more useful.

%............................................................
\Meth{virtual void SetString(ItemVal itemId, char* title)}
\Indextt{SetString}

This can be used to change the label on a command bar button,
status bar label, or menu item. The item identified by \code{itemId}
will have its label changed to \code{title}.

%............................................................
\Meth{virtual void SetStringAll(ItemVal itemId, char* title)}
\Indextt{SetStringAll}

This method is similar to \code{SetString}, except that
the string with the given \code{itemId} in \emph{ALL} currently
active windows is set. This is useful to keep control strings
in different windows in sync.

%............................................................
\Meth{virtual void SetTitle(char* title)}
\Indextt{SetTitle}

Set the name of the window shown on its title bar to \code{title}.

%............................................................
\Meth{virtual void ShowPane(vPane* wpane, int OnOrOff)}
\Indextt{ShowPane}

You can show or hide a command, status, or canvas pane with this
method. The pane must first be defined, created, and added
to the command window (which will show the pane). You can then
hide the pane later by calling this method with the pointer to
the pane and \code{OnOrOff} set to 0. A 1 will show the pane.
Note that in some environments (e.g., X), the window may show up
again in a different position in the window. For example, if you
had a command bar above a status bar, and then hide the command
bar, it will be placed under the status bar when you show it
again. This is a ``feature'' of X.

%............................................................
\Meth{virtual void ShowWindow(void)}
\Indextt{ShowWindow}

You \emph{must} call \code{ShowWindow()} after you have added all the
panes to the window. You usually call \code{ShowWindow()} in the
constructor to your \code{vCmdWindow} class after you have created
all the panes and have used \code{AddPane} to add them to the window.

\subsection* {Other Methods} %----------------------------------

%............................................................
\Meth{virtual void CloseWin()}
\Indextt{CloseWin}

This method is called by the \code{vApp::CloseAppWin} method
as part of closing down a window. The default \code{vWindow::CloseWin()}
method's behavior is to take care of some critical housekeeping
chores. You will normally never override this method. However, it
is remotely conceivable that there will be an occasion you need
to do something really low level after a window has been
destroyed by the host GUI environment. In that case, your method
\emph{must} call the immediate superclass \code{vWindow::CloseWin()}, and
then do whatever it has to do. Normally, you handle such details
in your class's \code{CloseAppWin} method.

\subsection* {See Also}

vCmdWindow

%***********************************************************************
%***********************************************************************
%***********************************************************************

\chapter {Dialogs}

This chapter covers the classes used to build dialogs, and the various
kinds of command objects that can be included in a dialog.


The classes and command objects covered in this chapter include:

\begin{description}
	\item[CmdAttribute] A type describing attributes of various command objects.
	\item[CommandObject] Main type used to define commands to dialogs and command panes.
	\item[Commands] Command items used in building dialogs.
	\item[vIcon] Used to define \V\ icons.
	\item[vDialog] Class to build a modeless dialog.
	\item[vModalDialog] Used to show modal dialogs.
\end{description}

%--------------------------------------------------------------
\Class{CmdAttribute}
\Indextt{CmdAttribute}
\label{sec:cmdattribute}

A type describing attributes of various command objects.

\subsection* {Synopsis}

\begin{description}
	\item [Header:] \code{<v/v\_defs.h>}
	\item [Type name:] CmdAttribute
\end{description}

\subsection* {Description}

These attributes are used when defining command items.
They are used to modify default behavior. These attributes are
bit values, and some can be combined with an \emph{OR} operation.
Note that not all attributes can be used with all commands.

\subsection* {Attributes}

\Param{CA\_DefaultButton} Used with a \code{C\_Button} to
\index{CA\_DefaultButton}
indicate that this button will be the default button. The user
can activate the default button by pressing the Enter key as well
as using the mouse. It will most often be associated with the OK
button.

\Param{CA\_Hidden} Sometimes you may find it useful to have a
\index{CA\_Hidden}
\index{hiding controls}
\index{invisible controls}
command object that is not displayed at first. By using the
\code{CA\_Hidden} attribute, the command object will not be
displayed. The space it will require in the dialog or dialog pane
will still be allocated, but the command will not be displayed.
You can then unhide (or hide) the command using the \code{SetValue}
method: \code{SetValue(CmdID, TrueOrFalse, Hidden)}.

\Param{CA\_Horizontal} Command will have horizontal orientation.
\index{CA\_Horizontal}
This attribute is used with Sliders and Progress Bars.

\Param{CA\_Large} The object should be larger than usual. It can
\index{CA\_Large}
be used with Lists, Progress Bars, Sliders, Text Ins, and Value
Boxes.

\Param{CA\_MainMsg} Used with a \code{C\_Label} to indicate that
\index{CA\_MainMsg}
its string will be replaced with the message supplied to the
\code{ShowDialog} method.

\Param{CA\_NoBorder} Used for frames and status bar labels,
\index{CA\_NoBorder}
\code{CA\_NoBorder} specifies that the object is to be displayed
with no border.

\Param{CA\_NoLabel} Used for progress bars to suppress display of
\index{CA\_NoLabel}
the value label.

\Param{CA\_NoNotify} Used for combo boxes and lists. When
\index{CA\_NoNotify}
specified, the program will not be notified for each selection of
a combo box item or a list item. When specified, the program is
notified only when the combo box button is pressed, and must then
use \code{GetValue} to retrieve the item selected in the combo
box list. For lists, you will need another command button in the
dialog to indicate list selection is done.

\Param{CA\_NoSpace} Used for frames, this attribute causes the
\index{CA\_NoSpace}
command objects within the frame to be spaced together as tightly
as possible. Normally, command objects have a space of several
pixels between them when laid out in a dialog. The \code{CA\_NoSpace}
attribute is especially useful for producing a tightly spaced
set of command buttons.

\Param{CA\_None} No special attributes. Used as a symbolic
filler when defining items, and is really zero.

\Param{CA\_Percent} Used with progress bars to add a \% to the
\index{CA\_Percent}
value label.

\Param{CA\_Small} The object should be smaller than usual. It can
\index{CA\_Small}
be used with Progress Bars and Text Ins. On Progress Bars,
\code{CA\_Small} means that the text value box will not be shown.

\Param{CA\_Text} Used for Spinners to specify that a text list
\index{CA\_Text}
of possible values has been supplied.

\Param{CA\_Vertical} Command will have vertical orientation.
\index{CA\_Vertical}
This attribute is used with Sliders and Progress Bars.

%------------------------------------------------------------------------
\Class{CommandObject}
\Indextt{CommandObject}
\index{command objects}

Used to define commands to dialogs and command panes.

\subsection* {Synopsis}

\begin{description}
	\item [Header:] \code{<v/v\_defs.h>}
	\item [Type name:] CommandObject
	\item [Part of:] vDialog, vCommandPane
\end{description}

\subsection* {Description}

This structure is used to define command items in dialogs and
command panes. You will define a static array of \code{CommandObject}
items. This array is then passed to the \code{AddDialogCmds}
method of a dialog class such as \code{vDialog} or \code{vModalDialog},
or the constructor of a \code{vCommandPane} object, or more
typically, a class derived from one of those.

\subsection* {Definition}

\footnotesize
\begin{verbatim}
typedef struct CommandObject
  {
    CmdType cmdType;    // what kind of item is this
    ItemVal cmdId;      // unique id for the item
    ItemVal retVal;     // initial value of object
    char* title;        // string
    void* itemList;     // used when cmd needs a list
    CmdAttribute attrs; // list of attributes
    int Sensitive;      // if item is sensitive or not
    ItemVal cFrame;     // Frame used for an item
    ItemVal cRightOf;   // Item placed left of this id
    ItemVal cBelow;     // Item placed below this one
    int size;           // Used for size information
  } CommandObject;
\end{verbatim}
\normalfont\normalsize

\subsection* {Structure Members}

\Param{CmdType cmdType} This value determines what kind of command
item this is. The types of commands are explained in the
section \emph{Commands}.

\Param{ItemVal cmdId} This unique id for the command defined by
\index{ItemVal}
the programmer. Each command item belonging to a dialog should
have a unique id, and it is advisable to use some scheme to be
sure the ids are unique. The \V\ system does not do anything to
check for duplicate ids, and the behavior is undefined for
duplicate ids. The id for a command is passed to the
\code{DialogCommand} method of the dialog, as well as being used
for calls to the various \code{SetX} and \code{GetX} methods.
There are many predefined values that can be used for ids as
described in the chapter \Sect{Standard V Values}.

The values you use for your id in menus and controls should
be limited to being less than 30,000. The predefined
\V\ values are all above 30,000, and are reserved. \emph{
There is no enforcement of this policy.} It is up to you
to pick reasonable values.

The type \code{ItemVal} exists for historical reasons, and
is equivalent to an int, and will remain so. Thus, the easiest
way to assign and maintain unique ids for your controls
is to use a C++ \code{enum}. As many as possible examples
in this manual will use \code{enums}, but examples using the old style
\code{const} code{ItemVal} declarations may
continue to exist. There is more discussion of assigning ids
in the following example.

\Param{int retVal} The use of this value depends on the type
of command. For buttons, for example, this value will be passed
(along with the \code{cmdId}) to the \code{DialogCommand} method.
The \code{retVal} is also used for the initial on/off state of check
boxes and radio buttons. For some commands, \code{retVal} is
unused. Note that the static storage provided in the declaration
is \emph{not} used to hold the value internally. You should
use \code{GetValue} to retrieve the current value of a
command object.

\Param{char* title} This is used for the label or text string
used for command items.

\Param{void* itemList} This is used to pass values to commands
that need lists or strings. The ListCmd is an example. Note the
\code{void *} to allow arbitrary lists.

\Param{CmdAttribute attrs} Some command items use attributes
to describe their behavior.  These attributes are summarized
in Section~\ref{sec:cmdattribute}, \Sect{CmdAttribute}.

\Param{int Sensitive} This is used to determine if an item is
sensitive or not. Note that the static storage provided in the
declaration is used by the \V\ system to track the value, and
should be changed by the \code{SetValue} method rather than
directly. Thus dialogs sharing the same static declaration will
all have the same value. This is usually desired behavior.

\Param{ItemVal cFrame} Command items may be placed within a frame.
If this value is 0 (or better, the symbol \code{NoFrame}), the
command will be placed in the main dialog area. If a value is
supplied, then the command will be placed within the frame with
the id \code{cFrame}.

\index{dialog layout}
\Param{ItemVal cRightOf, ItemVal cBelow} These are used to describe
the placement of a command within a dialog.  Ids of other commands
in the same dialog are used to determine placement. The current
command will be placed to the right of the command \code{cRightOf},
and below the command \code{cBelow}. The commands left and above
don't necessarily have to be adjacent. By careful use of these
values, you can design very attractive dialogs. You can control
the width of command objects by padding the label with blanks.
Thus, for example, you can design a dialog with all buttons the
same size.

You can also use the \code{CA\_Hidden} attribute to selectively
\index{CA\_Hidden}
hide command objects that occupy the same location in the
dialog. Thus, you might have a button labeled \code{Hide}
right of and below the same command object as another button
labeled \code{UnHide}. By giving one of the two buttons
the \code{CA\_Hidden} attribute, only one will be displayed.
Then you can use \code{SetValue} at runtime to switch which
button is displayed in the same location. The bigger of the
two command objects will control the spacing.

\Param{int size}

The size parameter can be used for some command objects to
specify size. For example, for labeled Button commands,
the \code{size} specifies the minimum width in pixels of the
button. It is also used in various other command objects as
needed. A value of zero for \code{size} always means use the
default size. Thus, you can take advantage of how C++ handles
declarations and write \code{CommandObject} declarations that
leave off the \code{size} values, which
default to zero. Many of the examples in this reference do not
specify these values.

\subsection* {Example}

The following example defines a simple dialog with a message
label on the top row, a check box on the second row, two buttons
in a horizontally organized frame on the third row, and an OK
button on the bottom row. The ids in this example are
defined using an \code{enum}. Remember that your ids must be
less than 30,000, and using 0 is not a good idea.
Thus, the \code{enum} in this example gives the ids
values from 101 to 106.
An alternative used in \V\ code prior to release 1.13 was
to provide \code{const}
declarations to define meaningful symbolic values for the ids.
Many examples of this type of id declaration will likely
persist.

It also helps to use a consistent naming convention for ids.
The quick reference appendix lists suggested prefixes for
each control type under the \code{CmdType} section. For
example, use an id of the form \code{btnXXX} for buttons.
Predefined ids follow the form \code{M\_XXX}.

\vspace{.1in}

\small
\begin{rawhtml}
<IMG BORDER=0 ALIGN=BOTTOM ALT="" SRC="../fig/dlgcmd.gif">
\end{rawhtml}

\begin{latexonly}
\setlength{\unitlength}{0.012500in}%
\begin{picture}(125,80)(55,715)
\thicklines
\put( 60,720){\framebox(35,15){}}
\put( 65,735){\line( 0,-1){ 15}}
\put( 90,735){\line( 0,-1){ 15}}
\put( 70,725){\makebox(0,0)[lb]{\smash{\SetFigFont{10}{12.0}{rm}OK}}}
\put( 65,750){\framebox(50,15){}}
\put(120,750){\framebox(50,15){}}
\put( 60,745){\framebox(115,25){}}
\put( 55,715){\framebox(125,80){}}
\put( 60,780){\makebox(0,0)[lb]{\smash{\SetFigFont{10}{12.0}{rm}Sample}}}
\put( 70,755){\makebox(0,0)[lb]{\smash{\SetFigFont{10}{12.0}{rm}Button 1}}}
\put(125,755){\makebox(0,0)[lb]{\smash{\SetFigFont{10}{12.0}{rm}Button 2}}}
\end{picture}

\end{latexonly}
\normalfont\normalsize

\footnotesize
\begin{verbatim}
enum {lbl1 = 101, frm1, btn1, btn2,
static CommandObject Sample[] =
  {
    {C_Label, lbl1, 0,"Sample",NoList,CA_MainMsg,isSens,NoFrame,0,0},
    {C_Frame, frm1, 0, "", NoList,CA_None,isSens,NoFrame,0,lbl1},
    {C_Button, btn1, 0, "Button 1", NoList, CA_None, isSens,frm1,0,0},
    {C_Button, btn2, 0, "Button 2", NoList, CA_None, isSens,frm1,btn1,0},
    {C_Button, M_OK, M_OK, " OK ", NoList, CA_DefaultButton, 
        isSens, NoFrame,0,frm1},
    {C_EndOfList,0,0,0,0,CA_None,0,0,0}
  };
\end{verbatim}
\normalfont\normalsize

\subsection* {See Also}

vWindow, Predefined ItemVals, CmdAttribute, Commands

%------------------------------------------------------------------------
\Class{Commands}
\Indextt{command objects}

This section describes how each of the command objects available
in \V\ is used to build dialogs.

\subsection* {Synopsis}

\begin{description}
	\item [Header:] \code{<v/v\_defs.h>}
	\item [Type name:] CmdType
\end{description}

\subsection* {Description}

\V\ provides several different kinds of command items that are used
in dialogs. The kind of command is specified in the \code{cmdType}
field of the \code{CommandObject} structure when defining a
dialog. This section describes current dialog commands available
with \V\@. They will be constructed by \V\ to conform to the
conventions of the host windowing system. Each command is named
by the value used to define it in the \code{CommandObject}
structure.

\subsection* {Commands}

\Cmd{C\_Blank}
\Indextt{C\_Blank}
\index{command!blank}

A Blank can help you control the layout of your dialogs.
The Blank object will occupy the space it would take
if it were a \code{C\_Label}, but nothing will be displayed. This
is especially useful for leaving space between other command
objects, and getting nice layouts with RightOfs and Belows. You
control the size of the Blank by providing a string with an
appropriate number of blanks for the \code{title} field.

%-----------------------------------------------------------------
\Cmd{C\_BoxedLabel}
\Indextt{C\_BoxedLabel}
\index{command!boxed label}

\small
\begin{rawhtml}
<IMG BORDER=0 ALIGN=BOTTOM ALT="" SRC="../fig/boxlabel.gif">
\end{rawhtml}
\begin{latexonly}
\setlength{\unitlength}{0.012500in}%
\begin{picture}(120,20)(35,800)
\thicklines
\put( 35,800){\framebox(120,20){}}
\put( 40,805){\makebox(0,0)[lb]{\smash{\SetFigFont{10}{12.0}{rm}Special Label Information}}}
\end{picture}

\end{latexonly}
\normalfont\normalsize
\vspace{.1in}

This command object is just like a \code{C\_Label}, but drawn
with a surrounding box. See \code{C\_Label}.

%-----------------------------------------------------------------
\Cmd{C\_Button}
\Indextt{C\_Button}
\index{command!button}

\small
\begin{rawhtml}
<IMG BORDER=0 ALIGN=BOTTOM ALT="" SRC="../fig/button.gif">
\end{rawhtml}
\begin{latexonly}
\setlength{\unitlength}{0.012500in}%
\begin{picture}(50,15)(120,750)
\thicklines
\put(120,750){\framebox(50,15){}}
\put(130,755){\makebox(0,0)[lb]{\smash{\SetFigFont{10}{12.0}{rm}Save}}}
\end{picture}

\end{latexonly}
\normalfont\normalsize
\vspace{.1in}

A Button is one of the primary command input items used in dialog
boxes. When the user clicks on a Button, the values set in the
\code{cmdId} and \code{retVal} fields are passed to the \code{DialogCommand}
method. In practice, the \code{retVal} field is not really used
for buttons -- the \code{cmdId} field is used in the
\code{switch} statement of the \code{DialogCommand} method.

A button is defined in a \code{CommandObject} array. This is a
typical definition:

\footnotesize
\begin{verbatim}
 {C_Button, btnId, 0,"Save",NoList,CA_None,isSens,NoFrame,0,0}
\end{verbatim}
\normalfont\normalsize

The \code{retVal} field can be used to hold any value you wish.
For example, the predefined color button frame (see \code{vColor})
uses the \code{cmdId} field to identify each color button, and
uses the \code{retVal} field to hold the index into the standard
\V\ color array. If you don't need to use the \code{retVal},
a safe convention is to a 0 for
the \code{retVal}. You can put any label you
wish in the \code{title} field.

If you provide the attribute \code{CA\_DefaultButton} to the
\code{CmdAttribute} field, then this button will be considered
the default button for the dialog. The default button will be
visually different than other buttons (usually a different
border), and pressing the Return key is the same as clicking on
the button.

You can change the label of a button with:
\code{SetString(btnId,} \code{"New Label")}. You can change the
sensitivity of a button with \code{SetValue(btnID, OnOrOff,
Sensitive)}.

\Cmd{C\_CheckBox}
\Indextt{C\_CheckBox}
\index{command!check box}

\small
\begin{rawhtml}
<IMG BORDER=0 ALIGN=BOTTOM ALT="" SRC="../fig/chkbox.gif">
\end{rawhtml}
\begin{latexonly}
\setlength{\unitlength}{0.012500in}%
\begin{picture}(20,10)(60,760)
\thicklines
\put( 60,760){\framebox(10,10){}}
\multiput( 60,760)(0.40000,0.40000){26}{\makebox(0.4444,0.6667){\SetFigFont{7}{8.4}{rm}.}}
\multiput( 60,770)(0.40000,-0.40000){26}{\makebox(0.4444,0.6667){\SetFigFont{7}{8.4}{rm}.}}
\put( 80,760){\makebox(0,0)[lb]{\smash{\SetFigFont{10}{12.0}{rm}Show Details}}}
\end{picture}

\end{latexonly}
\normalfont\normalsize

\vspace{.1in}

A CheckBox is usually used to set some option on or off. A
CheckBox command item consists of a check box and an associated
label. When the user clicks on the check box, the \code{DialogCommand}
method is invoked with the \code{Id} set to the \code{cmdId} and
the \code{Val} set to the current state of the CheckBox. The
system takes care of checking and unchecking the displayed check
box -- the user code tracks the logical state of the check box.

A CheckBox is defined in a \code{CommandObject} array. This is a
typical definition:

\footnotesize
\begin{verbatim}
 {C_CheckBox, chkId, 1,"Show Details",NoList,CA_None,isSens,NoFrame,0,0}
\end{verbatim}
\normalfont\normalsize

The \code{retVal} is used to indicate the initial state of the
check box. You should use the \code{GetValue} method to get the
current state of a check box. You can also track the state
dynamically in the \code{DialogCommand} method. You can put any
label you wish in the \code{title} field.

You can change the label of a check box with: \code{SetString(chkId,}
\code{"New Label")}. You can change the sensitivity of a check
box with \code{SetValue(chkID, OnOrOff,Sensitive)}. You can
change the checked state with \code{SetValue(chkID, OnOrOff,
Checked)}.

If the user clicks the Cancel button and your code calls the
default \code{DialogCommand} method, \V\ will automatically reset
any check boxes back to their original state, and call the
\code{DialogCommand} method an additional time with the original
value if the state has changed.  Thus, your code can track the
state of check boxes as the user checks them, yet rely on the
behavior of the Cancel button to reset changed check boxes to the
original state.

The source code for the \V\ \code{vDebugDialog} class provides a
good example of using check boxes (at least for the X version).
It is found in \code{v/src/vdebug.cxx}.

\Cmd{C\_ColorButton}
\Indextt{C\_ColorButton}
\index{command!color button}

\small
\begin{rawhtml}
<IMG BORDER=0 ALIGN=BOTTOM ALT="" SRC="../fig/color.gif">
\end{rawhtml}
\begin{latexonly}
\setlength{\unitlength}{0.012500in}%
\begin{picture}(40,20)(20,800)
\thicklines
\put( 20,800){\framebox(40,20){}}
\multiput( 20,815)(0.41667,0.41667){13}{\makebox(0.4444,0.6667){\SetFigFont{7}{8.4}{rm}.}}
\multiput( 20,810)(0.40000,0.40000){26}{\makebox(0.4444,0.6667){\SetFigFont{7}{8.4}{rm}.}}
\put( 20,805){\line( 1, 1){ 15}}
\put( 20,800){\line( 1, 1){ 20}}
\put( 25,800){\line( 1, 1){ 20}}
\put( 30,800){\line( 1, 1){ 20}}
\put( 35,800){\line( 1, 1){ 20}}
\put( 40,800){\line( 1, 1){ 20}}
\put( 45,800){\line( 1, 1){ 15}}
\multiput( 50,800)(0.40000,0.40000){26}{\makebox(0.4444,0.6667){\SetFigFont{7}{8.4}{rm}.}}
\multiput( 55,800)(0.41667,0.41667){13}{\makebox(0.4444,0.6667){\SetFigFont{7}{8.4}{rm}.}}
\put( 20,815){\line( 1, 0){ 40}}
\put( 20,810){\line( 1, 0){ 40}}
\put( 20,805){\line( 1, 0){ 40}}
\end{picture}

\end{latexonly}
\normalfont\normalsize
\vspace{.1in}

A color command button. This works exactly the same as a \code{C\_Button}
except that the button may be colored. You use \code{C\_ColorButton}
for the \code{cmdType} field, and provide a pointer to a \code{vColor}
structure in the \code{itemList} field using a \code{(void*)}
cast. The label is optional.

The \code{retVal} field of a color button is not used. You can
generate a square color button of a specified size by specifying
an empty label (\verb+""+) \emph{and} a \code{size} value greater
than 0. When you specify the \code{size} field,  the color button
will be a colored square \code{size} pixels per side. When used
within a \code{CA\_NoSpace} frame, this feature would allow you
to build a palette of small, tightly spaced color buttons. In
fact, \V\ provides a couple of such palettes in 
\code{v/vcb2x4.h} and \code{v/vcb2x8.h}. These
include files, as well as the other details of the \code{vColor}
class are described in the section \code{vColor} in the \Sect{Drawing}
chapter.

There are two ways to change to color of a button. The most direct
way is to change each of the RGB values in three successive calls
to \code{SetValue} using \code{Red}, \code{Green}, and finally
\code{Blue} as the \code{ItemSetType} to change the RGB values. 
The call with \code{Blue} causes the color to be updated. I know
this isn't the most elegant way to do this, but it fits with the
\code{SetValue} model.

An alternate way is to change the value of the original \code{vColor}
used to define the initial color of the control, and then call
\code{SetValue} with the \code{ChangeColor} set type.

This is a short example of defining a red button, and then changing it.

\footnotesize
\begin{verbatim}
    static vColor btncolor(255,0,0};  // define red
    ...

    // part of a CommandObject definition
    {C_ColorButton, cbt1, 0, "", (void*)&btncolor,
        CA_None, isSens, NoFrame, 0, btnXXX},

    ...
    // Code to change the color by some arbitrary values
    btncolor.Set(btncolor.r()+127, btncolor.g()+63, btncolor.b()+31);
#ifdef ByColor    // by vColor after changing btncolor
    SetValue(cbt1,0,btncolor);
#else          // by individual colors
    SetValue(cbt1,(ItemVal)btncolor.r(),Red);
    SetValue(cbt1,(ItemVal)btncolor.g(),Green);
    // This final call with Blue causes color to update in dialog
    SetValue(cbt1,(ItemVal)btncolor.b(),Blue);
#endif
    ...
\end{verbatim}
\normalfont\normalsize

\Cmd{C\_ComboBox}
\Indextt{C\_ComboBox}
\index{command!combo box}

\small
\begin{rawhtml}
<IMG BORDER=0 ALIGN=BOTTOM ALT="" SRC="../fig/combobox.gif">
\end{rawhtml}
\begin{latexonly}
\setlength{\unitlength}{0.012500in}%
\begin{picture}(70,90)(50,685)
\put( 60,735){\makebox(0,0)[lb]{\smash{\SetFigFont{10}{12.0}{rm}Bruce}}}
\put( 60,720){\makebox(0,0)[lb]{\smash{\SetFigFont{10}{12.0}{rm}Katrina}}}
\put( 60,705){\makebox(0,0)[lb]{\smash{\SetFigFont{10}{12.0}{rm}Risa}}}
\put( 60,690){\makebox(0,0)[lb]{\smash{\SetFigFont{10}{12.0}{rm}Van}}}
\thicklines
\put( 55,755){\framebox(45,15){}}
\put(105,755){\framebox(10,15){}}
\put( 50,750){\framebox(70,25){}}
\put( 55,750){\line( 0,-1){ 65}}
\put( 55,685){\line( 1, 0){ 45}}
\put(100,685){\line( 0, 1){ 65}}
\put(105,765){\line( 1, 0){ 10}}
\multiput(115,765)(-0.41667,-0.41667){13}{\makebox(0.4444,0.6667){\SetFigFont{7}{8.4}{rm}.}}
\multiput(110,760)(-0.41667,0.41667){13}{\makebox(0.4444,0.6667){\SetFigFont{7}{8.4}{rm}.}}
\put(105,760){\line( 1, 0){ 10}}
\put( 60,760){\makebox(0,0)[lb]{\smash{\SetFigFont{10}{12.0}{rm}Risa}}}
\end{picture}

\end{latexonly}
\normalfont\normalsize
\vspace{.1in}

A combo box is a drop-down list. It normally
appears as box with text accompanied by some kind of down arrow
button. You pass a list of alternative text values in the \code{itemList}
field of the \code{CommandObject} structure. You also must set
the \code{retVal} field to the index (starting at 0) of the item
in the list that is the default value for the combo box text
title.

If the user clicks the arrow, a list pops up with a set of
alternative text values for the combo box label. If the user
picks one of the alternatives, the popup closes and the new value
fills the text part of the combo box. \V\ supports up to 32
items in the combo box list. You need to use a \code{C\_List} if
you need more than 32 items.

With default attributes, a combo box will send a message to
\code{DialogCommand} whenever a user picks a selection from the
combo box dialog. This can be useful for monitoring the item
selected. If you define the combo box with the attribute
\code{CA\_NoNotify}, the dialog in not notified on each pick.
You can use \code{GetValue} to retrieve the index of the
item shown in the combo box text field.

You can preselect the value by using \code{SetValue}.
You can change the contents of the combo list by using 
\code{vDialog::SetValue} with either  \code{ChangeList} or
\code{ChangeListPtr}. See \code{vDialog::SetValue} for more
details.

\subsubsection* {Example}

The following is a simple example of using a combo box in a modal
dialog.  This example does not process items as they are clicked,
and does not show code that would likely be in an overridden
\code{DialogCommand} method. The code interface to a list and a
combo box is very similar -- the interaction with the user is
different. This example will initially fill the combo box label
with the text of  \code{comboList[2]}.

\footnotesize
\begin{verbatim}
enum { cbxId = 300 };
char* comboList[] =
  {
    "First 0",   // The first item in the list
     ...
    "Item N",    // The last item in the list
    0            // 0 terminates the list
  };
  ...
CommandObject ComboList[] =
  {
    {C_ComboBox, cbxId, 2, "A Combo Box", (void*)comboList,
       CA_NoNotify,isSens,NoFrame,0,0},
    {C_Button, M_OK, M_OK, " OK ", NoList,
       CA_DefaultButton, isSens, NoFrame, 0, ListId},
    {C_EndOfList,0,0,0,0,CA_None,0,0,0}
  };
    ...
    vModalDialog cd(this);    // create list dialog
    int cid, cval;
    ...
    cd.AddDialogCmds(comboList);   // Add commands to dialog
    cid = ld.ShowModalDialog("",cval);  // Wait for OK
    cval = ld.GetValue(cbxId);  // Retrieve the item selected
\end{verbatim}
\normalfont\normalsize

\Cmd{C\_EndOfList}
\Indextt{C\_EndOfList}
\index{command!end of list}

This is not really a command, but is used to denote end of the
command list when defining a \code{CommandObject} structure.

\Cmd{C\_Frame} 
\Indextt{C\_Frame} 
\index{command!frame}

\small
\begin{rawhtml}
<IMG BORDER=0 ALIGN=BOTTOM ALT="" SRC="../fig/frame.gif">
\end{rawhtml}
\begin{latexonly}
\setlength{\unitlength}{0.012500in}%
\begin{picture}(125,50)(75,735)
\thicklines
\put( 80,760){\framebox(10,10){}}
\put( 80,750){\framebox(0,0){}}
\put( 80,740){\framebox(10,10){}}
\put(140,760){\framebox(10,10){}}
\put( 75,735){\framebox(125,50){}}
\put( 95,760){\makebox(0,0)[lb]{\smash{\SetFigFont{10}{12.0}{rm}Option 1}}}
\put(155,760){\makebox(0,0)[lb]{\smash{\SetFigFont{10}{12.0}{rm}Option 2}}}
\put( 95,740){\makebox(0,0)[lb]{\smash{\SetFigFont{10}{12.0}{rm}Option 3}}}
\put( 80,775){\makebox(0,0)[lb]{\smash{\SetFigFont{10}{12.0}{rm}Set Options}}}
\end{picture}

\end{latexonly}
\normalfont\normalsize
\vspace{.1in}

The frame is a line around a related group of dialog command
items. The dialog window itself can be considered to be the
outermost frame. Just as the placement of commands within the
dialog can be controlled with the \code{cRightOf} and \code{cBelow}
fields, the placement of controls within the frame use the same
fields. You then specify the id of the frame with the \code{cFrame}
field, and then relative position within that frame.

The \code{title} field of a frame is not used.

You may supply the \code{CA\_NoBorder} attribute to any frame,
which will cause the frame to be drawn without a border. This can
be used as a layout tool, and is especially useful to force
buttons to line up in vertical columns.

See the section \Sect{CommandObject} for an example of defining a
frame.

%@@@@@@@@@@@@@@@
\Cmd{C\_Icon}
\Indextt{C\_Icon}
\index{command!icon}

\small
\begin{rawhtml}
<IMG BORDER=0 ALIGN=BOTTOM ALT="" SRC="../fig/icon.gif">
\end{rawhtml}
\begin{latexonly}
\setlength{\unitlength}{0.012500in}%
\begin{picture}(30,30)(5,805)
\thicklines
\put( 10,805){\framebox(20,20){}}
\put(  5,820){\line( 1, 1){ 15}}
\put( 20,835){\line( 1,-1){ 15}}
\put( 15,805){\line( 0, 1){ 15}}
\put( 15,820){\line( 1, 0){  5}}
\put( 20,820){\line( 0,-1){ 15}}
\put( 30,815){\line(-1, 0){  5}}
\put( 25,815){\line( 0,-1){  5}}
\put( 25,810){\line( 1, 0){  5}}
\end{picture}

\end{latexonly}
\normalfont\normalsize
\vspace{.1in}

A display only icon. This works exactly the same as a \code{C\_Label}
except that an icon is displayed instead of text. You use \code{C\_Icon}
for the \code{cmdType} field, and provide a pointer to the
\code{vIcon} object in the \code{itemList} field using a
\code{(void*)} cast. You should also provide a meaningful label
for the \code{title} field since some versions of \V\ may not
support icons.

You can't dynamically change the icon. 

\Cmd{C\_IconButton}
\Indextt{C\_IconButton}
\index{command!icon button}

\small
\begin{rawhtml}
<IMG BORDER=0 ALIGN=BOTTOM ALT="" SRC="../fig/iconbtn.gif">
\end{rawhtml}
\begin{latexonly}
\setlength{\unitlength}{0.012500in}%
\begin{picture}(40,40)(5,795)
\thicklines
\multiput( 35,815)(-0.29412,0.49019){19}{\makebox(0.4444,0.6667){\SetFigFont{7}{8.4}{rm}.}}
\put( 30,824){\line(-1, 0){ 10}}
\multiput( 20,824)(-0.29412,-0.49019){19}{\makebox(0.4444,0.6667){\SetFigFont{7}{8.4}{rm}.}}
\multiput( 15,815)(0.29412,-0.49019){19}{\makebox(0.4444,0.6667){\SetFigFont{7}{8.4}{rm}.}}
\put( 20,806){\line( 1, 0){ 10}}
\multiput( 30,806)(0.29412,0.49019){19}{\makebox(0.4444,0.6667){\SetFigFont{7}{8.4}{rm}.}}
\put( 20,815){\line( 1, 0){ 10}}
\put( 25,820){\line( 0,-1){ 10}}
\put( 10,800){\framebox(30,30){}}
\put(  5,795){\framebox(40,40){}}
\end{picture}

\end{latexonly}
\normalfont\normalsize
\vspace{.1in}

A command button Icon. This works exactly the same as a \code{C\_Button}
except that an icon is displayed for the button instead of text.
You use \code{C\_IconButton} for the \code{cmdType} field, and
provide a pointer to the \code{vIcon} object in the \code{itemList}
field using a \code{(void*)} cast. You should also provide a
meaningful label for the \code{title} field since some versions
of \V\ may not support icons.

You can't dynamically change the icon. The button will be sized to
fit the icon. Note that the \code{v/icons} directory contains
quite a few icons suitable for using on command bars.

\Cmd{C\_Label}
\Cmd{C\_ColorLabel}
\Indextt{C\_Label}
\Indextt{C\_ColorLabel}
\index{command!label}

{\Large Select Options}
\vspace{.1in}

This places a label in a dialog. A label is defined in
a \code{CommandObject} array. This is a typical definition:

\footnotesize
\begin{verbatim}
 {C_Label, lblId,0,"Select Options",NoList,CA_None,isSens,NoFrame,0,0, 0,0}
\end{verbatim}
\normalfont\normalsize

While the value of a label can be changed with 
\code{SetString(lblId,} \code{"New Label")}, they are usually static
items. If the label is defined with the \code{CA\_MainMsg}
attribute, then that label position will be used to fill the the
message provided to the \code{ShowDialog} method.

A \code{C\_ColorLabel} is a label that uses the
List parameter of the \code{CommandObject} array to
specify a \code{vColor}. You can
specify the color and change the color in the same fashion as
described in the \code{C\_ColorButton} command.

\Cmd{C\_List}
\Indextt{C\_List}
\index{command!list}

\small
\begin{rawhtml}
<IMG BORDER=0 ALIGN=BOTTOM ALT="" SRC="../fig/list.gif">
\end{rawhtml}
\begin{latexonly}
% Makes a listing of one or more files
% Typical usage:
% tex list *.c \\end

\def\grabfile#1 {\setbox0=\lastbox\endgraf\doit{#1}}
\everypar{\grabfile}

\font\filenamefont= cmtt8 scaled\magstep3
\font\headlinefont= cmr8
\font\listingfont= cmtex10
\font\eoffont= cmti8

\def\today{\ifcase\month\or
  January\or February\or March\or April\or May\or June\or
  July\or August\or September\or October\or November\or December\fi
  \space\number\day, \number\year}
\newcount\m \newcount\n
\n=\time \divide\n 60 \m=-\n \multiply\m 60 \advance\m \time
\def\hours{\twodigits\n\twodigits\m}
\def\twodigits#1{\ifnum #1<10 0\fi \number#1}

\newlinechar=`@
\message{@\today\space at \hours}

\raggedbottom
\nopagenumbers

\chardef\other=12
\def\doit#1{\message{@Listing #1@}
  \begingroup \everypar{} \frenchspacing
  \headline{\filenamefont#1\quad\headlinefont \today\ at \hours
      \hfill Page \folio}
  \def\do##1{\catcode`##1=\other}\dospecials
  \catcode127=\other \catcode9=\other \catcode12=\other
  \parindent 0pt \parfillskip=0pt plus 1fil minus 1in
  \everypar{\hangindent 1in} \rightskip=0pt plus 2in
  \def\par{\ifvmode\penalty-500\medskip\else\endgraf\fi}
  \listingfont \obeylines \obeyspaces \global\pageno=1
  \input #1 {\eoffont(end of\/ file)}\endgraf\vfill\eject\endgroup}
{\obeyspaces\global\let =\ }
\catcode`\_=\other % allow _ in file names

% A tab (^^I) prints as lowercase gamma.
% Character ^^M could be made visible, with a bit of work;
% at present, it's indistinguishable from newline (^^J).

% You can get up to 103 characters on a line without an overfull box.

\end{latexonly}
\normalfont\normalsize
\vspace{.1in}

A list is a scrollable window of text items. The list can be made
up of any number of items, but only a limited number are
displayed in the list scroll box.  Most implementations will show
eight items at a time.

The user uses the scroll bar to show various parts of the list.
Normally, when the user clicks on a list item, the \code{DialogCommand}
is invoked with the id of the List command in the \code{Id}
parameter, and the index into the list of the item selected in
the \code{Val} parameter.  This value may be less than zero,
which means the user has unselected an item, and your code
should properly handle this situation. This only means the user
has selected the given item, but not that the selection is final.
There usually must be a command Button such as OK to indicate
final selection of the list item.

If the List is defined with the attribute \code{CA\_NoNotify},
\code{DialogCommand} is not called with each pick. You must then
use \code{GetValue} to get which item in the list was selected.

It is possible to preselect a given list item with the
\code{SetValue} method. Use the \code{GetValue} to
retrieve the selected item's index after the OK button is selected.
A value less than zero means no item was selected.

Change the contents of the list with
\code{vDialog::SetValue} using either \code{ChangeList} or
\code{ChangeListPtr}. See \code{vDialog::SetValue} for more
details.

\subsubsection* {Example}

The following is a simple example of using a list box in a modal
dialog.  This example does not process items as they are clicked.

\footnotesize
\begin{verbatim}
enum {lstId = 200 };
char* testList[] =
  {
    "First 0",   // The first item in the list
     ...
    "Item N",    // The last item in the list
    0            // 0 terminates the list
  };
  ...
CommandObject ListList[] =
  {
    {C_List, lstId, 0, "A List", (void*)testList,
       CA_NoNotify,isSens,NoFrame,0,0},
    {C_Button, M_OK, M_OK, " OK ", NoList,
       CA_DefaultButton, isSens, NoFrame, 0, lstId},
    {C_EndOfList,0,0,0,0,CA_None,0,0,0}
  };
    ...
    vModalDialog ld(this);    // create list dialog
    int lid, lval;
    ...
    ld.AddDialogCmds(ListList);   // Add commands to dialog
    ld.SetValue(lstId,8,Value);  // pre-select 8th item
    lid = ld.ShowModalDialog("",lval);  // Wait for OK
    lval = ld.GetValue(lstId);  // Retrieve the item selected
\end{verbatim}
\normalfont\normalsize


\Cmd{C\_ProgressBar}
\Indextt{C\_ProgressBar}
\index{command!progress bar}

\small
\begin{rawhtml}
<IMG BORDER=0 ALIGN=BOTTOM ALT="" SRC="../fig/progress.gif">
\end{rawhtml}
\begin{latexonly}
\documentstyle[12pt]{article}
\title{\bf Xspread - A project progress report}
\author{Rama Devi Puvvada}
\date{9 July 1994}

\begin{document}
\maketitle
\begin {abstract}
Xspread is a public domain spreadsheet program on the X window system similar 
to Lotus 1-2-3. At present Xspread is already useable and has no major defects 
except for a few bugs/deficiencies. My project as a part of the course work of 
the Software Engineering course, summer '94 is to make some improvements and 
enhancements to the existing software. This project helps in understanding 
some of the qualitys of the Software Engineering such as the extendability, 
maintainability, userfriendliness, understandability.
\end{abstract}

\section{Introduction}

Xspread is a spreadsheet program whose structure and operation is similar to standard spreadsheets. 
Like other spreadsheets the workplace is arranged in rows and columns of cells. 
Each cell can contain a number, a label or a formula which evaluates to a number or label. 
One can start the xspread program with an empty workplace or by giving a file name whose contents are placed in the work place. 
The original spreadsheet program supports many standard spreadsheet features like cell entry and editing, row and column insertion and deletions, specification of range names, function references, manual and automatic recalculation etc \ldots. 
A postscript manual is found in the ``/usr/proj/se/summer94'' directory under the name `xspread.dvi'. 
The later versions improved it by adding other features like graphing capability. 
The main goal of my project is to add utilities like ``sort'', ``search'' \ldots math functions. 
The other subgoals are to correct some errors in the existing software.

\section{To Do}
The following are some of the improvements I plan to make.
    \begin{itemize}
     \item Add utilities such as ``sort'', ``search'', \ldots math functions.
     \item Make the changes/fixes suggested by Richard Lloyd.
     \item Run lint on all the c programs to remove unreasonable constructs.
     \item Correct the errors in plotting graphs when the size of the ranges 
do not match.
     \item Add colors to the graphs.
     \item Improve the efficiency of the matrix functions. 
    \end{itemize}

\section{So Far}
I have been reading a lot of books on X-windows and writing some simple 
programs to get a deeper understanding of the concepts. I have gone through 
almost the entire existing code comprising of approximately 22,000 lines to 
have an idea of what each routine is doing. Going through the code I found 
that the matrix functions are not efficient and they need to be fixed. Running 
the xspread program I myself found some errors in the graph plotting which I 
would like to correct. I have already made the changes suggested by Richard 
Lloyd. Presently I am running lint on all c programs to remove unreasonable 
constructs. There are some more improvements to be made to the software which 
I haven't mentioned in the list of things to do and which I would love to do 
but may not be able to because of the shortage of time.

\end{document}


\end{latexonly}
\normalfont\normalsize
\vspace{.1in}

Bar to show progress. Used with \code{CA\_Vertical}
or \code{CA\_Horizontal} attributes to control orientation.
You change the value of the progress bar with
\code{SetValue(ProgID, val, Value)}, where \code{val} is
a value between 0 and 100, inclusive. Normally, the
progress bar will show both a graphical indication of the value,
and a text indication of the value between 0 and 100.

If you don't want the text value (for example, your value
represents something other than 0 to 100), then define the
progress bar with the \code{CA\_NoLabel} attribute. Use
the \code{CA\_Percent} attribute to have a \% added to the
displayed value. You can also use \code{CA\_Small} or \code{CA\_Large}
to make the progress bar smaller or larger than normal. If you
need a text value display for ranges other than 0 to 100, you can
build a \code{CA\_NoSpace} frame with a progress bar and a text
label that you modify yourself.

\subsection* {Example}

The following shows how to define a progress bar, and how to
set its value.

\footnotesize
\begin{verbatim}
enum{frm1 = 200, lbl1, pbrH, pbrV, ... };
  static CommandObject Cmds[] =
  {
    ...
    // Progress Bar in a frame
    {C_Frame, frm1, 0, "",NoList,CA_None,isSens,NoFrame, 0,0},
    {C_Label, lbl1, 0, "Progress",NoList,CA_None,isSens,frm1,0,0},
    {C_ProgressBar, pbrH, 50, "", NoList,
        CA_Horizontal,isSens,frm1, 0, lbl1},  // Horiz, with label

    {C_ProgressBar, pbrV, 50, "", NoList,  // Vertical, no value
      CA_Vertical | CA_Small, isSens,NoFrame, 0, frm1},
    ...
  };
  ...
  // Set the values of both bars to same
  SetValue(pbrH,retval,Value);    // The horizontal bar
  SetValue(pbrV,retval,Value);    // The vertical bar

\end{verbatim}
\normalfont\normalsize

\Cmd{C\_RadioButton} 
\Indextt{C\_RadioButton} 
\index{command!radio button}

\small
\begin{rawhtml}
<IMG BORDER=0 ALIGN=BOTTOM ALT="" SRC="../fig/radiob.gif">
\end{rawhtml}
\begin{latexonly}
\setlength{\unitlength}{0.012500in}%
\begin{picture}(200,20)(5,815)
\thicklines
\put( 15,825){\circle{10}}
\put( 60,825){\circle*{10}}
\put(110,825){\circle{10}}
\put(160,825){\circle{10}}
\put(  5,815){\framebox(200,20){}}
\put( 25,820){\makebox(0,0)[lb]{\smash{\SetFigFont{10}{12.0}{rm}KOB}}}
\put( 70,820){\makebox(0,0)[lb]{\smash{\SetFigFont{10}{12.0}{rm}KOAT}}}
\put(120,820){\makebox(0,0)[lb]{\smash{\SetFigFont{10}{12.0}{rm}KRQE}}}
\put(170,820){\makebox(0,0)[lb]{\smash{\SetFigFont{10}{12.0}{rm}KASA}}}
\end{picture}

\end{latexonly}
\normalfont\normalsize
\vspace{.1in}

Radio buttons are used to select one and only one item from a
group. When the user clicks on one button of the group, the
currently set button is turned off, and the new button is turned
on. Note that for each radio button press, \emph{two} events are
generated. One a call to \code{DialogCommand} with the
id of the button being turned off, and the other a call with the
id of the button being turned on. The order of these two events is
not guaranteed. The \code{retVal} field indicates the initial on
or off state, and only one radio button in a group should be on.

Radio buttons are grouped by frame. You will typically put
a group of radio buttons together in a frame. Any buttons
not in a frame (in other words, those just in the dialog
window) are grouped together.

Radio buttons are handled very much like check boxes. Your code
should dynamically monitor the state of each radio button with
the \code{DialogCommand} method. Selecting Cancel will
automatically generate calls to \code{DialogCommand} to restore
the each of the buttons to the original state.

You can use \code{SetValue} with a \code{Value} parameter to
change the settings of the buttons at runtime. \code{SetValue}
will enforce a single button on at a time.

\subsection* {Example}

The following example of defining and using radio buttons was
extracted from the sample file \code{v/examp/mydialog.cpp}. It
starts with the button \code{RB1} pushed.

\footnotesize
\begin{verbatim}
enum {
    frmV1 = 200, rdb1, rdb2, rdb3, ...
...
  };
...
static CommandObject DefaultCmds[] =
  {
    {C_Frame, frmV1, 0,"Radios",NoList,CA_Vertical,isSens,NoFrame,0,0},
    {C_RadioButton, rdb1, 1, "KOB",  NoList,CA_None,isSens, fmV1,0,0},
    {C_RadioButton, rdb2, 0, "KOAT", NoList,CA_None, isSens,frmV1,0,0},
    {C_RadioButton, rdb3, 0, "KRQE", NoList,CA_None, isSens,frmV1,0,0},
    {C_Button, M_Cancel,M_Cancel,"Cancel",NoList,CA_None,
        isSens, NoFrame, 0, frmV1},
    {C_Button, M_OK, M_OK, " OK ", NoList, CA_DefaultButton, 
        isSens, NoFrame, M_Cancel, frmV1},
    {C_EndOfList,0,0,0,0,CA_None,0,0,0}
  };
...
void myDialog::DialogCommand(ItemVal Id, ItemVal Val, CmdType Ctype)
  {
    switch (Id)              // switch on command id
      {
        case rdb1:            // Radio Button KOB
            // do something useful - current state is in retval
            break;
        ...
        // cases for other radio buttons

      }
    // let the super class handle M_Cancel and M_OK
    vDialog::DialogCommand(id,retval,ctype);
  }
\end{verbatim}
\normalfont\normalsize

\Cmd{C\_Slider}
\Indextt{C\_Slider}
\index{command!slider}

\small
\begin{rawhtml}
<IMG BORDER=0 ALIGN=BOTTOM ALT="" SRC="../fig/slider.gif">
\end{rawhtml}
\begin{latexonly}
\setlength{\unitlength}{0.012500in}%
\begin{picture}(65,15)(20,810)
\thicklines
\put( 20,815){\framebox(20,5){}}
\put( 40,810){\framebox(5,15){}}
\put( 45,815){\framebox(40,5){}}
\end{picture}

\end{latexonly}
\normalfont\normalsize
\vspace{.1in}

Used to enter a value with a slider handle. The slider will provide
your program with a value between 0 and 100, inclusive. Your program
can then scale that value to whatever it needs.

\V\ will draw sliders in one of three sizes. Use \code{CA\_Small}
for a small slider (which may not be big enough to return all
values between 0 and 100 on all platforms), \code{CA\_Large} to
get a larger than normal slider, and no attribute to get a standard
size slider that will return all values between 0 and 100. Use
the \code{CA\_Vertical} and \code{CA\_Horizontal} attributes to
specify orientation of the slider.

When the user changes the value of the slider, the \code{DialogCommand}
method is called with the id of the slider for the \code{Id} value,
and the current value of the slider for the \code{Retval} value.
You can use \code{SetVal} to set a value for the slider.

\subsection* {Example}

The following example shows the definition line of a slider, and
a code fragment from an overridden \code{DialogCommand} method
to get the value of the dialog and update a \code{C\_Text} item
with the current value of the slider. The slider starts with a
value of 50.

\footnotesize
\begin{verbatim}
enum { frm1 = 80, sld1, txt1 };
CommandObject Commands[] =
  {
    ...
    {C_Frame, frm1, 0, "",NoList,CA_None,isSens,NoFrame,0,0},
    {C_Slider, sld1, 50, "",NoList,CA_Horizontal,isSens,frm1,0,0},
    {C_Text, txt1, 0, "", "50",CA_None,isSens, frm1, sld1, 0},
    ...
  };
  ...
void testDialog::DialogCommand(ItemVal id,
  ItemVal retval, CmdType ctype)
  { 
    ...
    switch (id)     // Which dialog command item?
      {
        ...
        case sld1:    // The slider
          {
            char buff[20];
            sprintf(buff,"%d",retval);  // To string
            SetString(txt1,buff);      // Show value
          }
        ...
      }
    ...
  }

\end{verbatim}
\normalfont\normalsize

\Cmd{C\_Spinner}
\Indextt{C\_Spinner}
\index{command!spinner}

\small
\begin{rawhtml}
<IMG BORDER=0 ALIGN=BOTTOM ALT="" SRC="../fig/spinner.gif">
\end{rawhtml}
\begin{latexonly}
\setlength{\unitlength}{0.012500in}%
\begin{picture}(90,30)(20,795)
\thicklines
\put( 20,800){\framebox(70,20){}}
\put( 90,810){\framebox(20,15){}}
\put( 90,795){\framebox(20,15){}}
\put( 90,805){\line( 1, 0){ 20}}
\multiput(110,805)(-0.40000,-0.40000){26}{\makebox(0.4444,0.6667){\SetFigFont{7}{8.4}{rm}.}}
\multiput(100,795)(-0.40000,0.40000){26}{\makebox(0.4444,0.6667){\SetFigFont{7}{8.4}{rm}.}}
\put( 90,815){\line( 1, 0){ 20}}
\multiput(110,815)(-0.40000,0.40000){26}{\makebox(0.4444,0.6667){\SetFigFont{7}{8.4}{rm}.}}
\multiput(100,825)(-0.40000,-0.40000){26}{\makebox(0.4444,0.6667){\SetFigFont{7}{8.4}{rm}.}}
\put( 25,805){\makebox(0,0)[lb]{\smash{\SetFigFont{10}{12.0}{rm}Value List}}}
\end{picture}

\end{latexonly}
\normalfont\normalsize
\vspace{.1in}

This command item is used to provide an easy way for the user to
enter a value from a list of possible values, or in a range of values.
Depending on the attributes supplied to the \code{CommandObject}
definition, the user will be able to select from a short list of
text values, from a range of integers, or starting with some
initial integer value. As the user presses either the up or down
arrow, the value changes to the next permissible value. The
\code{retVal} field specifies the initial value of the integer,
or the index of the initial item of the text list. You use the
\code{GetValue} method to retrieve the final value from the
\code{C\_Spinner}.

You can change the contents of the spinner list by using 
\code{vDialog::SetValue} with either  \code{ChangeList} or
\code{ChangeListPtr}. See \code{vDialog::SetValue} for more
details.

\subsubsection* {Example}

This example shows how to setup the \code{C\_Spinner} to select
a value from a text list (when supplied with a list and the
\code{CA\_Text} attribute), from a range of integers (when
supplied a range list), or from a starting value (when no list is
provided). The definitions of the rest of the dialog are not
included. 

\footnotesize
\begin{verbatim}
  static char* spinList[] =    // a list of colors
    {
      "Red","Green","Blue", 0
    };
  static int minMaxStep[3] =  // specify range of
    {                         // -10 to 10
      -10, 10, 2              // in steps of 2
    };
  enum { spnColor = 300, spnMinMax, spnInt, ... };
  CommandObject SpinDialog[] =
    {
      ...
      {C_Spinner,spnColor,0,"Vbox", // A text list.
        (void*)spinList,CA_Text,     // the list is CA_Text
        isSens,NoFrame, 0,0},
      {C_Spinner,spnMinMax,0,"Vbox", // a range -10 to 10
        (void*)minMaxStep,CA_None,  // by 2's starting at 0
        isSens,NoFrame, 0,0},
      {C_Spinner,spnInt,32,"Vbox",  // int values step by 1
        NoList,CA_None,             // starting at 32
        isSens,NoFrame, 0,0},
      ...
    };

\end{verbatim}
\normalfont\normalsize

%-----------------------------------------------------------------
\Cmd{C\_Text}
\Indextt{C\_Text}
\index{command!text}

\small
\begin{rawhtml}
<IMG BORDER=0 ALIGN=BOTTOM ALT="" SRC="../fig/textbox.gif">
\end{rawhtml}
\begin{latexonly}
\setlength{\unitlength}{0.012500in}%
\begin{picture}(100,30)(10,805)
\thicklines
\put( 10,805){\framebox(100,30){}}
\put( 15,825){\makebox(0,0)[lb]{\smash{\SetFigFont{10}{12.0}{rm}This is an example}}}
\put( 15,810){\makebox(0,0)[lb]{\smash{\SetFigFont{10}{12.0}{rm}of a two line text.}}}
\end{picture}

\end{latexonly}
\normalfont\normalsize
\vspace{.1in}

This draws boxed text. It is intended for displaying information
that might be changed, unlike a label, which is usually constant.
The text may be multi-line by using a \code{'$\backslash$n`}. The
\code{retVal} and \code{title} fields are not used. The text to
display is passed in the \code{itemList} field.

You can use the \code{CA\_NoBorder} attribute to suppress the border.

A definition of a \code{C\_Text} item in a \code{CommandObject}
definition would look like:

\footnotesize
\begin{verbatim}
 {C_Text, txtId, 0, "", "This is an example\nof a two line text.",
          CA_None,isSens,NoFrame, 0, 0, 0,0}, 
\end{verbatim}
\normalfont\normalsize

You can change the label of text box with:
\code{SetString(txtId,} \code{"New text} \code{to show.")}.

%-----------------------------------------------------------
\Cmd{C\_TextIn}
\Indextt{C\_TextIn}
\index{command!text in}

\small
\begin{rawhtml}
<IMG BORDER=0 ALIGN=BOTTOM ALT="" SRC="../fig/textin.gif">
\end{rawhtml}
\begin{latexonly}
\setlength{\unitlength}{0.012500in}%
\begin{picture}(185,20)(35,800)
\thicklines
\put( 35,800){\framebox(185,20){}}
\put( 40,805){\makebox(0,0)[lb]{\smash{\SetFigFont{10}{12.0}{rm}Editable input text \_}}}
\end{picture}

\end{latexonly}
\normalfont\normalsize
\vspace{.1in}

This command is used for text entry from the
user. The text input command item will typically be boxed
field that the user can use to enter text.

The strategy for using a TextIn command item is similar to
the List command item. You need an OK button, and then
retrieve the text after the dialog has been closed.

You can provide a default string in the \code{title} field
which will be displayed in the TextIn field.  The user will
be able to edit the default string. Use an empty string
to get a blank text entry field. The \code{retVal} field is
not used.

There are two ways to control the size of the TextIn control.
If you specify \code{CA\_None}, you will get a TextIn
useful form most simple input commands. Using \code{CA\_Large}
gets a wider TextIn, while \code{CA\_Small} gets a smaller
TextIn. You can also use the \code{size} field of the
\code{CommandObject} to explicitly specify a width in
characters. When you specify a size, that number of
characters will fit in the TextIn, but the control
does \emph{not} enforce that size as a limit.

\subsubsection* {Example}

The following example demonstrates how to use a TextIn.

\footnotesize
\begin{verbatim}
CommandObject textInList[] =
  {
    ...
    {C_TextIn, txiId,0,"",NoList,CA_None,isSens,NoFrame,0,0},
    ...
    {C_EndOfList,0,0,0,0,CA_None,0,0,0}
  };
 ...
    vModalDialog md(this);      /// make a dialog
    int ans, val;
    char text_buff[255];        // get text back to this buffer
 ...
    md.AddDialogCmds(textInList);  // add commands
    ans = md.ShowModalDialog("Enter text.", val);  // Show it
    text_buff[0] = 0;          // make an empty string
    (void) md.GetTextIn(txiId, text_buff, 254); // get the string
 ...
\end{verbatim}
\normalfont\normalsize

%------------------------------------------------------------------------
\Cmd{C\_ToggleButton}
\Indextt{C\_ToggleButton}
\index{command!toggle button}

\small
\begin{rawhtml}
<IMG BORDER=0 ALIGN=BOTTOM ALT="" SRC="../fig/button.gif">
\end{rawhtml}
\begin{latexonly}
\setlength{\unitlength}{0.012500in}%
\begin{picture}(50,15)(120,750)
\thicklines
\put(120,750){\framebox(50,15){}}
\put(130,755){\makebox(0,0)[lb]{\smash{\SetFigFont{10}{12.0}{rm}Save}}}
\end{picture}

\end{latexonly}
\normalfont\normalsize
\vspace{.1in}

A \code{C\_ToggleButton} is a combination of a
button and a checkbox. When the toggle button is pressed,
the \code{vCmdWindow::WindowCommand} method is called, just
as with a regular command button. However, the system will change
the look of the toggle button to indicate it has been
pressed. Each click on a \code{C\_ToggleButton} will cause
the button to appear pressed in or pressed out.

The \code{retVal} field of the \code{CommandObject}
definition is used to indicate the initial state of the
toggle.

The behavior of a toggle button is like a check box, and
not a radio button. This is more flexible, but if you need
exclusive radio button like selection, you will have to
enforce it yourself using \code{SetValue(toggleId,val,Value)}. 

\begin{verbatim}
 // Define a toggle button with id tbtToggle and
 // an initial state of 1, which means pressed in
 {C_ToggleButton,tbtToggle, 1,"", NoList,CA_None,
     isSens, NoFrame, 0, 0},
 ...

 // The case in WindowCommand should be like this:

    case tbtToggle:
      {
	// Always safest to retrieve current value
        ItemVal curval = GetValue(tbtToggle);
        // Now, do whatever you need to
        if (curval)
           ... it is pressed
        else
           ... it is not pressed
        break;
      }

\end{verbatim}


%------------------------------------------------------------------------
\Cmd{C\_ToggleFrame}
\Indextt{C\_ToggleFrame}
\index{command!toggle frame}
\index{tab controls}

\small
\begin{rawhtml}
<IMG BORDER=0 ALIGN=BOTTOM ALT="" SRC="../fig/frame.gif">
\end{rawhtml}
\begin{latexonly}
\setlength{\unitlength}{0.012500in}%
\begin{picture}(125,50)(75,735)
\thicklines
\put( 80,760){\framebox(10,10){}}
\put( 80,750){\framebox(0,0){}}
\put( 80,740){\framebox(10,10){}}
\put(140,760){\framebox(10,10){}}
\put( 75,735){\framebox(125,50){}}
\put( 95,760){\makebox(0,0)[lb]{\smash{\SetFigFont{10}{12.0}{rm}Option 1}}}
\put(155,760){\makebox(0,0)[lb]{\smash{\SetFigFont{10}{12.0}{rm}Option 2}}}
\put( 95,740){\makebox(0,0)[lb]{\smash{\SetFigFont{10}{12.0}{rm}Option 3}}}
\put( 80,775){\makebox(0,0)[lb]{\smash{\SetFigFont{10}{12.0}{rm}Set Options}}}
\end{picture}

\end{latexonly}
\normalfont\normalsize
\vspace{.1in}

A \code{C\_ToggleFrame} is \V's answer to the Windows Tab
control. While \V doesn't have real Tab controls, using
a combination of \code{C\_ToggleFrames} and either
radio buttons or toggle buttons, you can design very nice
multi-frame dialogs.

A Toggle Frame works just like a regular \code{C\_Frame} except
that you can use \code{SetValue} with a type \code{Value} to
hide or make visible all controls contained or nested in the
toggle frame. (Note: setting the \code{Value} of a toggle
frame is \emph{not} the same as setting its \code{Hidden}
attribute.)

The strategy for using toggle frames follows. First, you
will usually use two or more toggle frames together.
In the dialog \code{CommandObject} definition, you first
define one radio button or one toggle button for each
toggle frame used in the dialog. You then define a
regular bordered \code{C\_Frame} positioned below the radio/toggle
buttons. Then place \code{CA\_NoBorder} toggle frames
inside that outer frame. The outer frame will be the
border for all the toggle frames. Inside each toggle frame,
you define controls in the normal way.

You must select just \emph{one} of the toggle frames to
be initially visible. This will correspond to the checked
radio button or pressed toggle button. The remaining
toggle frames \emph{and} their controls should all be
defined using the \code{CA\_Hidden} attribute.

You then hide and unhide toggle frames by responding
to the \code{vDialog::DialogCommand} messages generated
when a radio button or toggle button is pressed. You
\code{SetValue(togID, 1, Value)} to show a toggle pane
and all its controls, and \code{SetValue(togID, 0, Value)}
to hide all its controls.

The following example shows how to define and control
toggle frames:

\begin{verbatim}
    enum {lbl1 = 400, tbt1, tbt2, tbt3, frm1, tfr1, tfr2,
          btnA1, btnB1, btnA2, btnB2 };
    static CommandObject DefaultCmds[] =
      {
        // A label, then 2 toggle buttons to select toggle frames
        {C_Label,lbl1,0,"Tab Frame Demo",NoList,CA_None,isSens,
                 NoFrame,0,0},
        {C_ToggleButton,tbt1,1,"Tab 1",NoList, CA_None, isSens, 
                 lbl1, 0, 0},
        {C_ToggleButton,tbt2,0,"Tab 2",NoList, CA_None, isSens, 
                 lbl1, tbt, 0},
        {C_ToggleButton,tbt3,0,"Tab 3",NoList, CA_None, isSens,
                 lbl1, tbt2 0},

        // A Master frame to give uniform border to toggle frames
        {C_Frame,frm1,0, "", NoList,CA_None,isSens,lbl1,0,tbt1},

        // Toggle Frame 1 - default frame on
        {C_ToggleFrame, tfr1,1,"",NoList, CA_NoBorder,isSens,frm1,0,0},
        {C_Button,btnA1,0,"Button A(1)",NoList,CA_None,isSens,tfr1,0,0},
        {C_Button,btnB1,0,"Button B(1)",NoList,CA_None,isSens,tfr1,
                  0,btnA1},

        // Toggle Frame 2 - default off (CA_Hidden!)
        {C_ToggleFrame,tfr2,0,"",NoList,CA_NoBorder | CA_Hidden,
                isSens,frm1,0,0},
        {C_Button,btnA2,0,"Button A(2)",NoList,CA_Hidden,isSens,tfr2,0,0},
        {C_Button,btnB2,0,"Button B(2)",NoList,CA_Hidden,isSens,tfr2,
                  btnA2,0},

        {C_EndOfList,0,0,0,0,CA_None,0,0,0}
      };


    ...

    // In the DialogCommand method:

    switch (id)         // We will do some things depending on value
      {
        case tbt1:       // For toggle buttons, assume toggle to ON
          {
            SetValue(id,1,Value);     // turn on toggle button
            SetValue(tbt2,0,Value);    // other one off
            SetValue(tfr2,0,Value);    // Toggle other frame off
            SetValue(tfr1,1,Value);    // and ours on
            break;
          }

        case tbt2:       // Toggle 2
          {
            SetValue(id,1,Value);     // turn on toggle button
            SetValue(tbt1,0,Value);    // other off
            SetValue(tfr1,0,Value);    // Toggle other off
	    SetValue(tfr2,1,Value);    // and ours on
            break;
          }

      }
    // All commands should also route through the parent handler
    vDialog::DialogCommand(id,retval,ctype);
  }
\end{verbatim}

%------------------------------------------------------------------------
\Cmd{C\_ToggleIconButton}
\Indextt{C\_ToggleIconButton}
\index{command!toggle icon button}

\small
\begin{rawhtml}
<IMG BORDER=0 ALIGN=BOTTOM ALT="" SRC="../fig/iconbtn.gif">
\end{rawhtml}
\begin{latexonly}
\setlength{\unitlength}{0.012500in}%
\begin{picture}(40,40)(5,795)
\thicklines
\multiput( 35,815)(-0.29412,0.49019){19}{\makebox(0.4444,0.6667){\SetFigFont{7}{8.4}{rm}.}}
\put( 30,824){\line(-1, 0){ 10}}
\multiput( 20,824)(-0.29412,-0.49019){19}{\makebox(0.4444,0.6667){\SetFigFont{7}{8.4}{rm}.}}
\multiput( 15,815)(0.29412,-0.49019){19}{\makebox(0.4444,0.6667){\SetFigFont{7}{8.4}{rm}.}}
\put( 20,806){\line( 1, 0){ 10}}
\multiput( 30,806)(0.29412,0.49019){19}{\makebox(0.4444,0.6667){\SetFigFont{7}{8.4}{rm}.}}
\put( 20,815){\line( 1, 0){ 10}}
\put( 25,820){\line( 0,-1){ 10}}
\put( 10,800){\framebox(30,30){}}
\put(  5,795){\framebox(40,40){}}
\end{picture}

\end{latexonly}
\normalfont\normalsize
\vspace{.1in}

A \code{C\_ToggleIconButton} is a combination of an icon
button and a checkbox. When the toggle icon button is pressed,
the \code{vCmdWindow::WindowCommand} method is called, just
as with a regular icon button. However, the system will change
the look of the toggle icon button to indicate it has been
pressed. This is useful for good looking icon based interfaces
to indicate to a user that some option has been selected.
An additional press will change the appearance back to a
normal icon button. The \code{retVal} field of the \code{CommandObject}
definition is used to indicate the initial state of the
toggle.

The behavior of a toggle icon button is like a check box, and
not a radio button. This is more flexible, but if you need
exclusive radio button like selection, you will have to
enforce it yourself using \code{SetValue(toggleId,val,Value)}. 

\begin{verbatim}
 // Define a toggle icon button with id tibToggle and
 // an initial state of 1, which means pressed
 {C_ToggleIconButton,tibToggle, 1,"", &anIcon,CA_None,
     isSens, NoFrame, 0, 0},
 ...

 // The case in WindowCommand should be like this:

    case tibToggle:
      {
        // Always safest to retrieve current value
        ItemVal curval = GetValue(tibToggle);
        // Now, do whatever you need to
        if (curval)
           ... it is pressed
        else
           ... it is not pressed
        break;
      }

\end{verbatim}

%------------------------------------------------------------------------
\Class{vIcon}
\Indextt{vIcon}
\index{icons}\index{bitmaps}

Used to define \V\ icons.

\subsection* {Synopsis}

\begin{description}
        \item [Header:] \code{<v/v\_icon.h>}
	\item [Class name:] vIcon
\end{description}

\subsection* {Description}

Icons may be used for simple graphical labels in dialogs,
as well as for graphical command buttons in dialogs and command bars.
See the sections \code{vButton} and \Sect{Dialog Commands} for
descriptions of using icons.

Presently, \V\ supports monochrome icons which allow an on or
off state for each pixel, and color icons of either 256 or $2^{24}$ colors.
The format of \V\ monochrome icons is identical to the X bitmap format. This
is a packed array of unsigned characters (or bytes), with each bit
representing one pixel. The size of the icon is specified
separately from the icon array. The \V\ color icon format is internally
defined, and allows easy conversion to various color file formats
used by X and Windows.

\subsection* {Definition}

\footnotesize
\begin{verbatim}
    class vIcon     // an icon
      {
      public:             //---------------------------------------- public
        vIcon(unsigned char* ic, int h, int w, int d = 1);
        ~vIcon();
        int height;             // height in pixels
        int width;              // width in pixels
        int depth;              // bits per pixel (1,8, or 24)
        unsigned char* icon;    // ptr to icon array

      protected:        //--------------------------------------- protected
      private:          //--------------------------------------- private
      };
\end{verbatim}
\normalfont\normalsize

\subsection* {Constructor} %------------------------------------

\Meth{vIcon(unsigned char* icon, int height, int width, int depth = 1)}
\Indextt{vIcon}

The constructor for a \code{vIcon} has been designed to allow you to
easily define an icon. The first parameter is a pointer to the static icon
array. (Note: \code{vIcon} does not make a copy of the icon - it
needs to be a static or persistent definition in your code.) The second and third
parameters specify the height and width of the icon. The last
parameter specifies depth.

\subsection* {Class Members}

\Param{int height} This is the height in pixels of the icon.

\Param{int width} This is the width in pixels of the icon. A icon
will thus require (height * width) pixels.  These bits are packed
into bytes, with 0's padding the final byte if needed.

\Param{int depth} For monochrome icons, this will be one.
For color icons, the value is either 8 (for $2^{8}$ or 256 colors) or 24
(for $2^{24}$ colors).

\Param{unsigned char* icon} This is a pointer to the array of
bytes that contain the icon. \V\ basically uses the format
defined by X (\code{.XBM}) bitmaps for monochrome bitmaps.
It uses an internal format consisting of a color map followed
by a one byte per pixel color icon description, or a three
bytes per pixel color icon description.

\subsection* {Defining Icons}

The easiest way to define an icon is to include the definition of
it in your code (either directly or by an \code{\#include}).
You then provide the address of the icon data plus its height and
width to the initializer of the \code{vIcon} object.

The \V distribution includes a simple icon editor that can
be used to create and edit icons in standard \code{.vbm} format,
as well as several other formats.
You can also generate monochrome icons is with the X
\code{bitmap} utility. That program allows you to
draw a bitmap, and then save the definition as C code. This code
can be included directly in your code and used in the initialization
of the \code{vIcon} object.  If you follow the example, you should
be able to modify and play with your icons very easily.

A simple converter that converts a Windows \code{.bmp} format file
to a \V \code{.vbm} \V bitmap format is also included in the
standard \V distribution. There are many utilities that let
you generate \code{.bmp} files on both Windows and X, so this
tool easily lets you add color icons of arbitrary size. 
Chapter 9 has more details on \code{bmp2vbm}.

The standard \V distribution also contains a directory 
(\code{v/icons}) with quite a few sample icons suitable for using
in a command bar.

Once you have a \code{.vbm} file, the easiest way to add an icon
to your program is to include code similar to this in your source:

\footnotesize
\begin{verbatim}

#include "bruce.vbm"    // Picture of Bruce
  static vIcon bruceIcon(&bruce_bits[0], bruce_height,
                          bruce_width,8);

\end{verbatim}
\normalfont\normalsize


The following sections describe the format of the
\code{unsigned char* icon} data for 1, 8, and 24 bit
\V\ icons.

\subsubsection*{1 Bit Icons}

Icon definitions are packed into bytes. A bit value of 1
represents Black, a 0 is White. The bytes are arranged by rows,
starting with the top row, with the bytes padded with leading
zeros to come out to whole bytes. The bytes are scanned in
ascending order (\code{icon[0], icon[1],} etc.). Within bytes,
the bits are scanned from LSB to MSB. A 12 bit row with the
pattern \code{BBBWWBBWBWBW} would be represented as \code{unsigned
char row[] =} \code{\{ 0x67, 0x05 \};}. This is the format
produced by the X \code{bitmap} program.

\subsubsection*{8 Bit Icons}

Eight bit icons support 256 colors. Each pixel of the icon is
represented by one byte. Bytes are arranged in row order,
starting with the top row. Each byte represents an index into a
color map. The color map consists of RGB byte entries.
While an 8 bit icon can only have 256 colors, it can map into
$2^{24}$ possible colors. Thus, each 8 bit icon must also include
the color map as part of its data.
The very first byte of the \code{icon} data is the number of
entries in the color map \emph{minus one}\footnote{This is
necessary keep things as \code{chars} and still allow a possible
256 entries, since 256 is $2^{8}+1$, and a color map with 0
entries doesn't make sense.} (you don't have to define all 256
colors), followed by the color map RGB bytes, followed by the
icon pixels. The following is a very simple example of an icon:

\footnotesize
\begin{verbatim}
//vbm8
#define color_width 16
#define color_height 12
#define color_depth 8
static unsigned char color_bits[] = {
       2,       // 3 colors in color map (2 == 3-1)
       255,0,0, // byte value 0 maps to red
       0,255,0, // 1 -> green
       0,0,255, // 2 -> blue
       // Now, the pixels: an rgb "flag", 3 16x4 rows
       0,0,0,0,0,0,0,0,0,0,0,0,0,0,0,0, // RRRRRRRRRRRRRRRR
       0,0,0,0,0,0,0,0,0,0,1,1,1,1,1,0, // RRRRRRRRRRBBBBBR
       0,0,0,0,0,0,0,0,0,0,1,1,1,1,1,0, // RRRRRRRRRRBBBBBR
       0,0,0,0,0,0,0,0,0,0,0,0,0,0,0,0, // RRRRRRRRRRRRRRRR
       1,1,1,1,1,1,1,1,1,1,1,1,1,1,1,1, // GGGGGGGGGGGGGGGG
       1,1,1,1,1,1,1,1,1,1,1,1,1,1,1,1, // GGGGGGGGGGGGGGGG
       1,1,1,1,1,1,1,1,1,1,1,1,1,1,1,1, // GGGGGGGGGGGGGGGG
       1,1,1,1,1,1,1,1,1,1,1,1,1,1,1,1, // GGGGGGGGGGGGGGGG
       2,2,2,2,2,2,2,2,2,2,2,2,2,2,2,2, // BBBBBBBBBBBBBBBB
       2,2,2,2,2,2,2,2,2,2,2,2,2,2,2,2, // BBBBBBBBBBBBBBBB
       2,2,2,2,2,2,2,2,2,2,2,2,2,2,2,2, // BBBBBBBBBBBBBBBB
       2,2,2,2,2,2,2,2,2,2,2,2,2,2,2,2  // BBBBBBBBBBBBBBBB
     };

static vIcon colorIcon(&color_bits[0], color_height, color_width,
     color_depth);
\end{verbatim}
\normalfont\normalsize

\subsubsection*{24 Bit Icons}

Twenty-four bit icons are arranged in rows, staring with the top
row, of three bytes per pixel. Each 3 byte pixel value represents
an RGB value. There is no color map, and the RGB pixel values
start immediately in the \code{unsigned char* icon} data array.
This is a simple example of a 24 bit icon.

\footnotesize
\begin{verbatim}
//vbm24
#define c24_height 9
#define c24_width 6
#define c24_depth 24
    static unsigned char c24_bits[] = {
     255,0,0,255,0,0,255,0,0,255,0,0,0,255,0,0,255,0, //RRRRGG
     255,0,0,255,0,0,255,0,0,255,0,0,0,255,0,0,255,0, //RRRRGG
     255,0,0,255,0,0,255,0,0,255,0,0,255,0,0,255,0,0, //RRRRRR
     0,255,0,0,255,0,0,255,0,0,255,0,0,255,0,0,255,0, //GGGGGG
     0,255,0,0,255,0,0,255,0,0,255,0,0,255,0,0,255,0, //GGGGGG
     0,255,0,0,255,0,0,255,0,0,255,0,0,255,0,0,255,0, //GGGGGG
     0,0,255,0,0,255,0,0,255,0,0,255,0,0,255,0,0,255, //BBBBBB
     0,0,255,0,0,255,0,0,255,0,0,255,0,0,255,0,0,255, //BBBBBB
     0,0,255,0,0,255,0,0,255,0,0,255,0,0,255,0,0,255  //BBBBBB
    };
    static vIcon c24Icon(&c24_bits[0], c24_height, c24_width,
        c24_depth);

\end{verbatim}
\normalfont\normalsize

\subsection* {Example}

This example uses the definition of the checked box used by the
Athena checkbox dialog command.

\footnotesize
\begin{verbatim}
// This code is generated by the V Icon Editor:
//vbm1
#define checkbox_width 12
#define checkbox_height 12
#define checkbox_depth 1
static unsigned char checkbox_bits[] = {
   0xff, 0x0f, 0x03, 0x0c, 0x05, 0x0a, 0x09, 0x09, 
   0x91, 0x08, 0x61,  0x08, 0x61, 0x08, 0x91, 0x08,
   0x09, 0x09, 0x05, 0x0a, 0x03, 0x0c, 0xff, 0x0f};

// This code uses the above definitions to define an icon
// in the initializer of checkIcon to vIcon.

static vIcon checkIcon(&checkbox_bits[0],
    checkbox_height, checkbox_width, checkbox_depth);

\end{verbatim}
\normalfont\normalsize

\subsection* {See Also}

vButton, Dialog Commands C\_Icon and C\_IconButton

%-------------------------------------------------------------------
\Class{vDialog}
\Indextt{vDialog}

Class to build a modeless dialog.

\subsection* {Synopsis}

\begin{description}
	\item [Header:] \code{<v/vdialog.h>}
	\item [Class name:] vDialog
 	\item [Hierarchy:] (vBaseWindow,vCmdParent) \rta vDialog
	\item [Contains:] CommandObject
\end{description}

\subsection* {Description}

The \code{vDialog} class is used to build modeless dialogs. Since
most dialogs will require a response to the commands they define,
you will almost always derive your own subclass based on \code{vDialog},
and override the \code{DialogCommand} method to handle those
commands. Note that \code{vDialog} is multiply derived from the
\code{vBaseWindow} and the \code{vCmdParent} classes.

\subsection* {Constructor} %------------------------------------

%............................................................
\Meth{vDialog(vBaseWindow* parent)}
\Indextt{vDialog}
\Meth{vDialog(vApp* parent)}
\Meth{vDialog(vBaseWindow* parent, int isModal = 0, char* title = "")}
\Meth{vDialog(vApp* parent, int isModal = 0, char* title = "")}

A dialog is constructed by calling it with a pointer to a
vBaseWindow or vApp, which is usually the 'this' of the object that
creates the \code{vDialog}. The \code{isModal} parameter
indicates if the dialog should be modal or modeless. You would
usually use the default of 0. The modal flag is used by the
derived \code{vModalDialog} class. The \code{title} parameter can
be used to set a title for your dialog (see \code{SetDialogTitle}
for information on titles). If you create a derived dialog class,
you might provide a \code{parent} and a \code{title} in your
constructor, and provide the 0 for the \code{isModal} flag in the
call to the \code{vDialog} constructor.

The constructor builds an empty dialog. The \code{AddDialogCmds}
method must be called in order to build a useful dialog, which
you would usually do from within the constructor of your derived
dialog class.

\emph{IMPORTANT!} When you derive your own \code{vDialog} objects,
you should write constructors for both the \code{vBaseWindow*} and
\code{vApp*} versions. These two different constructors allow
dialogs to be used both from windows directly, and from the
\code{vApp} code as well. Normally, you would construct a dialog
from a window. Occasionally, it will be useful to build a dialog
from the vApp that applies to all windows, and not just the window
that constructed it.

%............................................................
\Meth{void vDialog::AddDialogCmds(CommandObject* cList)}
\Indextt{AddDialogCmds}
   
This method is used to add a list of commands to a dialog.
It is called after the dialog object has been created.
You can usually do this in the constructor for your
derived Dialog class. This method is passed an array
of \code{CommandObject} structures.
    
%............................................................
\Meth{void vDialog::SetDialogTitle(char* title)}
\Indextt{SetDialogTitle}

This can be used to dynamically change the title of any object
derived from a \code{vDialog} object. Note that the title will
not always be displayed. This depends on the host system. For
example, the user can set up their X window manager to not show
decorations on transient windows, which is how dialogs
are implemented on X. You should write your applications to
provide a meaningful title as they are often helpful when
displayed.

\subsection* {Example}

This example shows the steps required to use a dialog object.
Note that the example uses the \code{vDialog} class directly,
and thus only uses the default behavior of responding to the
\code{OK} button.
    
\vspace{.1in}
\small
\begin{rawhtml}
<IMG BORDER=0 ALIGN=BOTTOM ALT="" SRC="../fig/dialog.gif">
\end{rawhtml}
\begin{latexonly}
\setlength{\unitlength}{0.012500in}%
\begin{picture}(130,55)(35,760)
\thicklines
\put( 40,765){\framebox(35,20){}}
\put( 45,785){\line( 0,-1){ 20}}
\put( 70,785){\line( 0,-1){ 20}}
\put( 35,760){\framebox(130,55){}}
\put( 40,800){\makebox(0,0)[lb]{\smash{\SetFigFont{10}{12.0}{rm}Sample modeless dialog.}}}
\put( 50,770){\makebox(0,0)[lb]{\smash{\SetFigFont{10}{12.0}{rm}OK}}}
\end{picture}

\end{latexonly}
\normalfont\normalsize

\footnotesize
\begin{verbatim}
#include <v/vdialog.h>
    CommandObject cmdList[] =           // list of the commands
      {
        {C_Label, lbl1, 0, "Label",NoList,CA_MainMsg,isSens,0,0},
        {C_Button, M_OK, M_OK, " OK ", NoList,
            CA_DefaultButton, isSens,lbl1,0},
        {C_EndOfList,0,0,0,0,CA_None,0,0}  // This ends list
      };
    ...
    vDialog curDialog(this,0,"Sample Dialog"); // create dialog instance

    curDialog.AddDialogCmds(cmdList);          // add the commands

    curDialog.ShowDialog("Sample modeless dialog."); // invoke
    ...

\end{verbatim}
\normalfont\normalsize

This example creates a simple modeless dialog with a label and
an OK button placed below the label (see the description
of layout control below). \code{ShowDialog} displays the dialog,
and the \code{vDialog::DialogCommand} method will be invoked with
the id (2) and value (\code{M\_OK}) of the OK button when it is
pressed.

Use the \code{vModalDialog} class to define modal
dialogs.

\vspace{.1in}

The \code{CommandObject} structure includes the following:

\footnotesize
\begin{verbatim}
    typedef struct CommandObject
      {
        CmdType cmdType;    // what kind of item is this
        ItemVal cmdId;      // unique id for the item
        ItemVal retVal;     // initial value
                            //  depends on type of command
        char* title;        // string for label or title
        void* itemList;     // a list of stuff to use for the cmd
                            //  depends on type of command
        CmdAttribute attrs; // list of attributes of command
        unsigned
             Sensitive:1;   // if item is sensitive or not
        ItemVal cFrame;     // if item part of a frame
        ItemVal cRightOf;   // Item placed left of this id
        ItemVal cBelow;     // Item placed below this one
        int size;           // Used for size information
      } CommandObject;
\end{verbatim}
\normalfont\normalsize

Placements of command objects within the dialog box are controlled
by the \code{cRightOf} and \code{cBelow} fields. By specifying
where an object goes in relation to other command objects in the
dialog, it is simple to get a very pleasing layout of the dialog.
The exact spacing of command objects is controlled by the \code{vDialog}
class, but the application can used \code{C\_Blank} command
objects to help control spacing.

The various types of command objects that can be added include
(with suggested id prefix in parens):

\footnotesize
\begin{verbatim}
    C_EndOfList:   Used to denote end of command list
    C_Blank:       filler to help RightOfs, Belows work (blk)
    C_BoxedLabel:  a label with a box (bxl)
    C_Button:      Button (btn)
    C_CheckBox:    Checked Item (chk)
    C_ColorButton: Colored button (cbt)
    C_ColorLabel:  Colored label (clb)
    C_ComboBox:    Popup combo list (cbx)
    C_Frame:       General purpose frame (frm)
    C_Icon:        a display only Icon (ico)
    C_IconButton:  a command button Icon (icb)
    C_Label:       Regular text label (lbl)
    C_List:        List of items (lst)
    C_ProgressBar: Bar to show progress (pbr)
    C_RadioButton: Radio button (rdb)
    C_Slider:      Slider to enter value (sld)
    C_Spinner:     Spinner value entry (spn)
    C_TextIn:      Text input field (txi)
    C_Text:        wrapping text out (txt)
    C_ToggleButton: a toggle button (tbt)
    C_ToggleFrame: a toggle frame (tfr)
    C_ToggleIconButton:  a toggle Icon button (tib)
\end{verbatim}
\normalfont\normalsize

The use of these commands is described in the \code{DialogCommand}
section.

%............................................................
\Meth{virtual void CancelDialog()}
\Indextt{CancelDialog}

This method is used to cancel any action that took place in the
dialog.  The values of any items in the dialog are reset to their
original values, and the  This method is automatically invoked
when the user selects a button with the value \code{M\_Cancel}
and the \code{DialogCommand} method invoked as appropriate to
reset values of check boxes and so on. \code{CancelDialog} can
also be invoked by the application code.

%............................................................
\Meth{virtual void CloseDialog()}
\Indextt{CloseDialog}

The \code{CloseDialog} is used to close the dialog. It can be
called by user code, and is automatically invoked if the user
selects the \code{M\_Done} or \code{M\_OK} buttons and the the
user either doesn't override the \code{DialogCommand} or calls
the default \code{DialogCommand} from any derived \code{DialogCommand}
methods.

%............................................................
\Meth{virtual void DialogCommand(ItemVal Id, ItemVal Val, CmdType Type)}
\Indextt{DialogCommand}

This method is invoked when a user selects some command item
of the dialog. The default \code{DialogCommand} method will
normally be overridden by a user derived class. It is useful to
call the default \code{DialogCommand} from the derived method for
default handling of the \code{M\_Cancel} and \code{M\_OK}
buttons.

The \code{Id} parameter is the value of the \code{cmdId} field of
the \code{CommandObject} structure. The \code{Val} parameter is
the \code{retVal} value, and the \code{Type} is the \code{cmdType}.

The user defined \code{DialogCommand} is where most of the work
defined by the dialog is done. Typically the derived
\code{DialogCommand} will have a \code{switch} statement with a
\code{case} for each of the command \code{cmdId} values defined
for items in the dialog.

%............................................................
\Meth{void DialogDisplayed()}
\Indextt{DialogDisplayed}

This method is called by the \V\ runtime system after a dialog
has actually been displayed on the screen. This method is especially
useful to override to set values of dialog controls with
\code{SetValue} and \code{SetString}.

It is important to understand that the dialog does not get
displayed until \code{ShowDialog} or \code{ShowModalDialog} has
been called. There is a very important practical limitation
implied by this, especially for modal dialogs. The values of
controls \emph{cannot} be changed until the dialog has been
displayed, even though the \code{vDialog} object may exist. Thus,
you can't call \code{SetValue} or \code{SetString} until after
you call \code{ShowDialog} for modeless dialogs, or \code{ShowModalDialog}
for modal dialogs. Since \code{ShowModalDialog} does not return
until the user has closed the dialog, you must override \code{DialogDisplayed}
if you want to change the values of controls in a modal dialog
dynamically.

For most applications, this is not a problem because the
static definitions of controls in the \code{CommandObject} definition
will be usually be what is needed. However, if you need to create
a dialog that has those values changed at runtime, then the
easiest way is to include the required \code{SetValue} and
\code{SetString} calls inside the overridden \code{DialogDisplayed}.

%............................................................
\Meth{void GetDialogPosition(int\& left, int\& top, int\& width, int\& height)}
\Indextt{GetDialogPosition}

Returns the position and size of \code{this} dialog. These values
reflect the actual position and size on the screen of the dialog.
The intent of this method
is to allow you to find out where a dialog is so
position it so that it
doesn't cover a window.

%............................................................
\Meth{virtual int GetTextIn(ItemVal Id, char* str, int maxlen)}
\Indextt{GetTextIn}

This method is called by the application to retrieve any text
entered into any \code{C\_TextIn} items included in the dialog
box. It will usually be called after the dialog is closed.
You call \code{GetTextIn} with the \code{Id} of the TextIn
command, the address of a buffer (\code{str}), and the
size of \code{str} in \code{maxlen}.

%............................................................
\Meth{virtual int GetValue(ItemVal Id)}
\Indextt{GetValue}

This method is called by the user code to retrieve values of
command items, usually after the dialog is closed.  The most
typical use is to get the index of any item selected by the
user in a \code{C\_List} or \code{C\_ComboBox}.

%............................................................
\Meth{int IsDisplayed()}
\Indextt{IsDisplayed}

This returns true if the dialog object is currently displayed,
and false if it isn't. Typically, it will make sense only to
have a single displayed instance of any dialog, and your code
will want to create only one instance of any dialog. Since
modal dialogs allow the user to continue to interact with the
parent window, you must prevent multiple calls to \code{ShowDialog}.
One way would be to make the command that displays the dialog to
be insensitive. \code{IsDisplayed()} is provided as an alternative
method. You can check the \code{IsDisplayed()} status before
calling \code{ShowDialog}.

%............................................................
\Meth{virtual void SetDialogPosition(int left, int top)}
\Indextt{SetDialogPosition}

Moves \code{this} dialog to the location \code{left} and
\code{top}. This function can be used to move dialogs so
they don't cover other windows.

%............................................................
\Meth{virtual void SetValue(ItemVal Id, ItemVal val, ItemSetType type)}
\Indextt{SetValue}

This method is used to change the state of dialog command items.
The \code{ItemSetType} parameter is used to control what is set.
Not all dialog command items can use all types of settings. The possibilities
include:

\paragraph*{Checked}
\Indextt{Checked}

The \code{Checked} type is used to change the checked status
of check boxes. \V\ will normally handle checkboxes, but if
you implement a command such as \emph{Check All}, you can
use \code{SetValue} to change the check state according to
\code{ItemVal val}.

\paragraph*{Sensitive}
\Indextt{Sensitive}

The \code{Sensitive} type is used to change the sensitivity of
a dialog command.

\paragraph*{Value}
\Indextt{Value}

The \code{Value} type is used primarily to preselect the item
specified by \code{ItemVal val} in a list or combo box list.

\paragraph*{ChangeList, ChangeListPtr}
\Indextt{ChangeList}
\Indextt{ChangeListPtr}
\index{dynamic lists}
\index{lists}

Lists, Combo Boxes, and Spinners use the \code{itemList}
field of the defining \code{CommandObject} to specify
an appropriate list. \code{SetValue} provides two ways
to change the list values associated with these controls.

The key to using \code{ChangeListPtr} and \code{ChangeList}
is an understanding of just how the controls use the list.
When a list type control is instantiated, it keeps a private
copy of the pointer to the original list as specified
in the \code{itemList} field of the defining \code{CommandObject}.

So if you want to change the original list, then
\code{ChangeList} is used. The original list may be
longer or shorter, but it must be in the same place.
Remember that a NULL entry marks the end of the list.
So you could allocate a 100 item array, for example,
and then reuse it to hold 0 to 100 items.

Call \code{SetValue} with \code{type} set to \code{ChangeList}.
This will cause the list to be updated. Note that you must not
change the \code{itemList} pointer used when you defined the list
or combo box. The contents of the list can change, but the
pointer must be the same. The \code{val} parameter is not used
for \code{ChangeList}. 

Sometimes, especially for regular list controls, a statically
sized list just won't work. Using \code{ChangeListPtr} allows
you to use dynamically created list, but with a small coding
penalty. To use \code{ChangeListPtr}, you must first modify
the contents of the \code{itemList} field of the original 
\code{CommandObject} definition to point the the new list.
Then call \code{SetValue} with \code{ChangeListPtr}. Note
that this will both update the pointer, and update the
contents of the list. You \emph{don't} need to call again with
\code{ChangeList}.

The following illustrates using both types of list change:

\footnotesize
\begin{verbatim}

  char* comboList[] = {
    "Bruce", "Katrina", "Risa", "Van", 0 };
  char* list1[] = {"1", "2", "3", 0};
  char* list2[] = {"A", "B", "C", "D", 0};

  // The definition of the dialog
  CommandObject ListExample[] = {
    {C_ComboBox,100,0,"",(void*)comboList,CA_None,isSens,0,0,0},
    {C_List,200,0,"",(void*)list1,CA_None,isSens,0,0,0},
    ...
    };
   ...

    // Change the contents of the combo list
    comboList[0] = "Wampler";  // Change Bruce to Wampler
    SetValue(200,0,ChangeList);
   ...
    // Change to a new list entirely for list
    // Note that we have to change ListExample[1], the
    // original definition of the list control.
    ListExample[1].itemList = (void*)list2;  // change to list2
    SetValue(100,0,ChangeListPtr);
   ...
\end{verbatim}
\normalfont\normalsize

Note that this example uses static definitions of lists. It is
perfectly fine to use completely dynamic lists: you just have
to dynamically fill in the appropriate \code{itemList} field
of the defining \code{CommandObject}.

Please see the description of \code{DialogDisplayed}
for an important discussion of setting dialog control values.

%............................................................
\Meth{virtual void SetString(ItemVal Id, char* str)}
\Indextt{SetString}

This method is called to set the string values of dialog items. This
can include the labels on check boxes and radio buttons and
labels, as well as the text value of a Text item.

Please see the description of \code{DialogDisplayed}
for an important discussion of setting dialog control values.


%............................................................
\Meth{virtual void ShowDialog(char* message)}
\Indextt{ShowDialog}

After the dialog has been defined, it must then be displayed by
calling the \code{Show\-Dialog} method. If a  \code{C\_Label} was
defined with a \code{CA\_MainMsg} attribute, then the message
provided to \code{ShowDialog} will be used for that label.

\code{ShowDialog} returns to the calling code as soon as the
dialog is displayed. It is up to the \code{DialogCommand} method
to then handle command input to the dialog, and to close the
dialog when done.

Please see the description of \code{DialogDisplayed}
for an important discussion of setting dialog control values.

\subsection* {Derived Methods}

None.

\subsection* {Inherited Methods}

None.

\subsection* {See Also}

vModalDialog

%------------------------------------------------------------------

\Class{vModalDialog}
\Indextt{vModalDialog}

Used to show modal dialogs.

\subsection* {Synopsis}
\begin{description}
	\item [Header:] \code{<v/vmodald.h>}
	\item [Class name:] vModalDialog
 	\item [Hierarchy:] (vBaseWindow,vCmdParent) \rta vDialog \rta vModalDialog
	\item [Contains:] CommandObject
\end{description}

\subsection* {Description}

This class is an implementation of a modal dialog.  This means
that the dialog grabs control, and waits for the user to select
an appropriate command from the dialog.  You can use any of
the methods defined by the \code{vDialog} class, as well as the
new \code{ShowModalDialog} method.

\subsection* {Constructor} %------------------------------------

%............................................................
\Meth{vModalDialog(vBaseWindow* parent, char* title)}
\Indextt{vModalDialog}
\Meth{vModalDialog(vApp* parent, char* title)}

There are two versions of the constructor, one for constructing
dialogs from windows, the other from the vApp object. See the
description of the \code{vDialog} constructor for more details.

The default value for the title is an empty string, so you
can declare instances of modal dialogs without the title
string if you wish. The dialog title will always show in
Windows, but in X is dependent on how the window manager
treats decorations on transient windows.

\subsection* {New Methods}

%............................................................
\Meth{virtual ItemVal ShowModalDialog(char* message, ItemVal\& retval)}
\Indextt{ShowModalDialog}

This method displays the dialog, and does not return until
the modal dialog is closed. It returns the id of the
button that caused the return, and in \code{retval}, the value of
the button causing the return as defined in the dialog
declaration.

Please see the description of \code{DialogDisplayed}
for an important discussion of setting dialog control values.

There are a couple of ways to close a modal dialog and make
\code{ShowModalDialog} return, all controlled by the \code{DialogCommand}
method. The default \code{DialogCommand} will close the modal
dialog automatically when the user clicks the \code{M\_Cancel},
\code{M\_Done}, or \code{M\_OK} buttons.

All command actions are still passed to the virtual \code{DialogCommand}
method, which is usually overridden in the derived class. By
first calling \code{vModalDialog::DialogCommand}
to handle the default operation, and then checking for the
other buttons that should close the dialog, you can also close
the dialog by calling the \code{CloseDialog} method, which will
cause the return.

The following code demonstrates this.

\footnotesize
\begin{verbatim}
    void myModal::DialogCommand(ItemVal id, ItemVal val,
        CmdType ctype)
      {
        // Call the parent for default processing
        vModalDialog::DialogCommand(id,val,ctype);
        if (id == M_Yes || id == M_No) // These close, too.
            CloseDialog();
      }
\end{verbatim}
\normalfont\normalsize

\subsection* {Derived Methods}

%............................................................
\Meth{virtual void DialogCommand(ItemVal Id, ItemVal val, CmdType type)}
\Indextt{DialogCommand}

Adds a little functionality for handling this modally.

\subsection* {Inherited Methods}

\Meth{vDialog(vBaseWindow* parent)}
\Indextt{vDialog}

\Meth{vDialog(vBaseWindow* parent, int modalflag)}
\Meth{vDialog(vApp* parent)}

\Meth{vDialog(vApp* parent, int modalflag)}
\Meth{void vDialog::AddDialogCmds(CommandObject* cList)}
\Indextt{AddDialogCmds}
\Meth{virtual void CancelDialog()}
\Indextt{CancelDialog}
\Meth{virtual void CloseDialog()}
\Indextt{CloseDialog}
\Meth{virtual int GetTextIn(ItemVal Id, char* str, int maxlen)}
\Indextt{GetTextIn}
\Meth{virtual int GetValue(ItemVal Id)}
\Indextt{GetValue}
\Meth{virtual void SetValue(ItemVal Id, ItemVal val, ItemSetType type)}
\Indextt{SetValue}
\Meth{virtual void SetString(ItemVal Id, char* str)}
\Indextt{SetString}
\Meth{virtual void ShowDialog(char* message)}
\Indextt{ShowDialog}

\subsection* {See Also}

vDialog

%***********************************************************************
%***********************************************************************
%***********************************************************************
\chapter {Drawing}

This chapter covers classes and utility functions needed to draw
text and graphics.

The classes and objects covered in this chapter include:

\begin{description}
	\item[Introduction to Drawing] Basic \V\ drawing model.
	\item[Fonts] Various screen fonts are available in \V\@.
	\item[vBrush] A brush for filling areas.
	\item[vCanvasPaneDC] The canvas pane drawing canvas.
	\item[vCanvasPane] A base class to build graphical and text canvas panes.
	\item[vBaseGLCanvasPane] A specialized class to support OpenGL.
	\item[vColor] A class for specifying colors.
	\item[vDC] A base class describing drawing canvas methods.
	\item[vMemoryDC] A memory drawing canvas.
	\item[vPen] A drawing pen.
	\item[vPrintDC] A printer drawing canvas.
	\item[vPrinter] A printer setup dialog.
	\item[vTextCanvasPane] A class for drawing text on a canvas. 
	\item[vTextEditor] A class for editing text.
	\item[vBaseGLCanvasPane] A class to support OpenGL.
\end{description}

%------------------------------------------------------------------------
\Class{Introduction to Drawing}
\index{drawing}

The basic \V\ model of drawing is a canvas. \V\ supports several kinds
of drawing canvases. The most obvious canvas is the screen drawing canvas.
This will often be the main or even only canvas you use. \V\ also
supports printing canvases. Each kind of canvas has identical drawing
methods, so you can write code to draw that is mostly independent of
which kind of canvas is being used.

There is also a specialized drawing canvas to support OpenGL. This
class differs somewhat from the other drawing canvases.
\index{OpenGL}

\subsection* {Drawing with the vDC Class}
\Indextt{vDC}

You draw to the various canvases using a \code{vDC} class, the
general \V\ Drawing Canvas Class (the OpenGL canvas does not use
the code{vDC} class). The \code{vDC} class for drawing to the
screen is \code{vCanvasPaneDC}. The class \code{vPrintDC} is the
platform independent class to draw to a printer. For X, \code{vPrintDC}
supports PostScript printing. The Windows version supports
standard Windows printers. (You can also use the PostScript DC
independently on Windows.) If you write your drawing code to use
a \code{vDC} pointer, you will be able to draw to several
canvases just by changing the value of the pointer.

Each \code{vDC} supports the methods described in the \code{vDC} section.
Because the \code{vCanvasPane} class is so central to most
applications, it duplicates
all the \code{vDC} methods so you can call them directly from your
\code{vCanvasPane} object. In fact, all the methods in \code{vCanvasPane}
are just calls to the corresponding \code{vDC} using the \code{vCanvasPaneDC}
of the canvas pane. You can get the \code{vCanvasPaneDC} pointer with
the \code{GetDC} method.

There are three kinds of drawing methods supported by \V\@. The simplest
methods draw lines of various widths and colors using the current
\code{vPen}. You change the color and width of the lines being drawn
by setting the current \code{vPen} with the \code{SetPen} method.

The second type of drawing includes filling the space surrounded
by a shape such as a polygon. The edges of the shape are drawn using
the current \code{vPen}. The filled area is drawn using the current
\code{vBrush}. You can set various attributes of the brush, and use
\code{SetBrush} to change how the shapes will be filled, as well as
changing the attributes of the \code{vPen} used to draw the surrounding
line. Both the pen and the brush can be transparent, allowing you to
draw unfilled outline shaped, or to fill a shape without an outline.

Finally, \V\ supports drawing of text on a canvas using various
\code{vFonts} and text attributes. The canvas pane will start out
using the default system font (\code{vfSystemDefault}). If you need
a different initial font, use \code{vFont::SetFontValues} to
select the font you want, then \code{vCanvasPane::SetFont} to set
the new font.

\subsection{Coordinates}

All \V\ drawing canvas classes use integer physical coordinates
appropriate to the canvas. All devices call the upper left corner
x,y coordinate of the drawing canvas 0,0. The x values increase
to the right, and y values increase down.

It it up to each application to provide appropriate mapping
from the coordinates used for the particular model being used
(often called the world coordinate system) to the physical
mapping used by each \V\ drawing canvas. Each drawing canvas
will have a physical limit for the maximum x and maximum y,
usually imposed by the particular canvas (a screen or a paper
size, for example). You can set a scale factor for each drawing
canvas which can be helpful for using different kinds of drawing
canvases. \V\ also supports setting an x,y translation. This will
allow you to more easily use the scroll bars and set margins
on printers. Your application can usually use the messages
received from the scroll bars to set the translation coordinates
to map your the canvas to a different drawing area. The system
will handle clipping.

However, the application is for the most part responsible
for determining all coordinate mapping -- translations of
a viewport of the drawing, determining the scaling for
various drawing canvases, and any mapping from the world
to the physical coordinates. The application will have to
map the mouse input accordingly, too.

%------------------------------------------------------------------------

\Class{Fonts}
\index{fonts}
\Indextt{vFont}

Various screen fonts are available in \V\@.

\subsection* {Synopsis}

\begin{description}
	\item [Class:] \code{vFont}
	\item [Header:] \code{<v/vfont.h>}
\end{description}

\subsection* {Description}

Fonts are difficult to make portable. \V\ has adopted a font
model that is somewhat portable, yet allows you to take advantage
of various fonts available on different platforms. In fact, it
is possible to write your programs to use the \code{vFontSelect}
dialog class, and pretty much ignore many of the details of
selecting fonts. The main characteristics of fonts your program
will have to deal with are the height and width of text displayed
on a canvas. These values are provided by 
\code{vDC::TextHeight} and \code{vDC::TextWidth}.

Fonts are associated with drawing canvases. For example, the
\code{vCanvasPane::SetFont} method is used to set the font used
by the canvas pane. The sizes of the actual fonts will probably
differ on different kinds of canvases. Specifically, your program
should not depend on getting the same \code{TextWidth} value
for screen and printer canvases for the same font.

The class \code{vFont} is used to define font objects, and the
characteristics of the font are set either by the class
constructor when the font is instantiated, or by using the  
\code{vFont :: SetFontValues} method. The utility class \code{vFontSelect}
can be used to interactively set font characteristics.  The
characteristics associated with a font are described in the
following sections. Remember, however, that \code{vFontSelect::FontSelect}
can be used to set these attributes.

\subsubsection*{Font Family}

Each font belongs to a font family. There are eight font families
defined by \V\@ with the \code{vFontID} attribute of the font object.
Font families typically correspond to some typeface name such
as \emph{Helvetica} or \emph{Times Roman}, but use more generic names.
There are three system fonts, \code{vfDefaultSystem}, \code{vfDefaultFixed},
and \code{vfDefaultVariable}. These default fonts are defined by
the specific platform. \code{vfDefaultSystem} will usually be a
fixed space font, and is often settable by the user. On X, for
example, the default system font can be changed by using a
\code{-fn fontname} switch when starting the application. The
\code{vfDefaultSystem} font will have fixed attributes, and will
not be changeable by the program. The \code{vfDefaultFixed}
(fixed spacing) and  \code{vfDefaultVariable} (variable spacing)
fonts are also system specified, but can usually have their
attributes, such as size and weight changed.

\V\ also supports five other font families. The \code{vfSerif} font
is a seriffed font such as \code{Times Roman}. The \code{vfSanSerif}
is a serifless font such as \code{Swiss} or \code{Lucidia}. Both
of these are variable spaced fonts. The \code{vfFixed} is a fixed
space font, often called \code{Courier} on the host platform.
The \code{vfDecorative} font usually contains symbols or other drawing
characters. It is not very portable across platforms. Finally,
\V\ supports a font family called \code{vfOther}. This is used when
the system supports other fonts that are selectable via the
\code{vFontSelect} dialog class. Windows supports a wide variety of fonts,
while X does not support any additional fonts.

\subsubsection*{Font Style}

\V\ supports two kinds of font styles: \code{vfNormal} for normal
fonts, and \code{vfItalic} for italic fonts.

\subsubsection*{Font Weight}

\V\ supports two kinds of font weights: \code{vfNormal} for
normal weight fonts, and \code{vfBold} for boldface fonts.

\subsubsection*{Point Size}

\V\ supports a wide range of point size, usually ranging from 8 point
to 40 or 72 point fonts. Not all point sizes are supported on each
platform. How each point size maps to space on the screen or page
also vary from platform to platform.

\subsubsection*{Underlining}

You can also specify that a font is underlined. Currently, underlining
does not work for X screens.

\subsection* {Methods}

\Meth{vFont(vFontID fam = vfDefaultFixed, int size = 10,
vFontID sty = vfNormal, vFontID wt = vfNormal, int und = 0)}
\Indextt{vFont}

The constructor is used to declare a font with the specified
family, size, style, weight, and underline.

\Meth{vFontID GetFamily()}
\Indextt{GetFamily}

Returns the family of the font object.

\Meth{int GetPointSize()}
\Indextt{GetPointSize}

Returns the point size of the font object.

\Meth{vFontID GetStyle()}
\Indextt{GetStyle}

Returns the style of the font object.

\Meth{vFontID GetWeight()}
\Indextt{GetWeight}

Returns the weight of the font object.

\Meth{int GetUnderlined()}
\Indextt{GetUnderlined}

Returns the underline setting of the font object.

\Meth{void SetFontValues(vFontID fam = vfDefaultFixed, int size =
10, vFontID sty = vfNormal, vFontID wt = vfNormal, int und = 0)}
\Indextt{SetFontValues}

Changes the attributes of the font object. For example, the font
selection dialog uses this method to change the font attributes.

%------------------------------------------------------------------------

\Class{vBrush}
\Indextt{vBrush}\index{brush}

A class to specify the brush used to fill shapes.

\subsection* {Synopsis}

\begin{description}
	\item [Header:] \code{<v/vbrush.h>}
	\item [Class name:] vBrush
\end{description}

\subsection* {Description}

Brushes are used to fill shapes. Brushes have
two attributes, including color and style.

\subsection* {Methods}

\Meth{vBrush(unsigned int r = 0, unsigned int g = 0, unsigned
int b = 0, int style = vSolid)}
\Indextt{vBrush}

The brush constructor allows you to set the initial color and
style of the brush. The default constructs a solid black brush.

\Meth{int operator == , !=}

You can use the operators \code{==} and \code{!=} for comparisons.

\Meth{vColor GetColor()}
\Indextt{GetColor}

This method returns the current color of the brush as a \code{vColor} object.

\Meth{int GetFillMode()}
\Indextt{GetFillMode}

This method returns the fill mode of the brush (either \code{vAlternate}
or \code{vWinding}).

\Meth{int GetStyle()}
\Indextt{GetStyle}

This method returns the current style of the brush.

\Meth{void SetColor(vColor\& c)}
\Indextt{SetColor}
\index{color!brush}

You can use this method to set the brush color by passing
in a \code{vColor} object.

\Meth{int SetFillMode(int fillMode)}
\Indextt{SetFillMode}

This method sets the fill mode of the brush.
The fillMode parameter specifies one of two alternative filling
algorithms, \code{vAlternate} or \code{vWinding}. These algorithms correspond
to the equivalent algorithms on the native platforms.

\Meth{void SetStyle(int style)}
\Indextt{SetStyle}

This method is used to set the style of the brush. Brush styles
include:

\begin{description}
	\item [vSolid] The brush fills with a solid color.
	\item [vTransparent] The brush is transparent, which allows
you to draw unfilled shapes.
	\item [vHorizontalHatch] The brush fills with a horizontal hatch pattern
in the current color.
	\item [vVerticleHatch] The brush fills with a vertical hatch pattern.
	\item [vLeftDiagonalHatch] The brush fills with a left leaning
diagonal hatch pattern.
	\item [vRightDiagonalHatch] The brush fills with a right
leaning diagonal hatch pattern.
	\item [vCrossHatch] The brush fills with a vertical and horizontal cross hatch
pattern.
	\item [vDiagonalCrossHatch] The brush fills with a diagonal
cross hatch pattern.
\end{description}

%------------------------------------------------------------------------

\Class{vCanvasPaneDC}
\Indextt{vCanvasPaneDC}

The drawing canvas class for CanvasPanes.

\subsection* {Synopsis}

\begin{description}
	\item [Header:] \code{<v/vcpdc.h>}
	\item [Class name:] vCanvasPaneDC
 	\item [Hierarchy:] vDC \rta vCanvasPaneDC
\end{description}

\subsection* {Description}

This class is normally automatically used by the \code{vCanvasPane}
class. It provides the actual implementation of the screen drawing
canvas class. 

%------------------------------------------------------------------------

\Class{vCanvasPane}
\Indextt{vCanvasPane}

A base class to build graphical and text canvas panes.

\subsection* {Synopsis}

\begin{description}
	\item [Header:] \code{<v/vcanvas.h>}
	\item [Class name:] vCanvasPane
 	\item [Hierarchy:] vPane \rta vCanvasPane
\end{description}

\subsection* {Description}

This is the base drawing class. You use it to build more complicated
drawing canvases, either for graphical drawing or text drawing.
The \code{vCanvasPane} class has all the basic methods needed to
interact with the drawing canvas. It does not, however, know
how to handle repainting the screen on \code{Redraw} or \code{Resize}
events. It provides utility methods for drawing on the canvas,
and several other methods that are normally overridden by your
application.

See the section \code{vPane} for a general description of panes.

\subsection* {Utility Methods}

The following methods provide useful service without modification. Sometimes
you will want to override some of these, but you will then usually
call these methods from your derived class.

%............................................................
\Meth{Drawing}
\index{drawing}

The \code{vCanvasPane} normally creates a \code{vCanvasPaneDC} to
use for drawing, and class provides direct support by including
direct calls for the drawing methods described in the \code{vDC}
section. If your drawing will only be to the screen, then you
can use the methods of the \code{vCanvasPane} class directly.
Each of these methods is really an inline function that expands
to \code{\_cpDC->DrawWhatever()}.

If your drawing code might want to draw to both a screen and
a printer, you might want to use a parameter to the appropriate
drawing canvas. You can get the \code{vDC} used by the \code{vCanvasPane}
by calling \code{GetDC()}.

%............................................................
\Meth{virtual void CreateDC(void)}
\Indextt{CreateDC}

This method is called when the \code{vCanvasPane} is initialized.
The default is to create a drawing canvas using \code{\_cpDC = new
vCanvasPaneDC(this);}. If you want to derive a different canvas
pane class from \code{vCanvasPane} perhaps using a more
sophisticated drawing canvas derived from the \code{vCanvasPaneDC}
class, you can override the \code{CreateDC} method and set the
protected \code{vDC* \_cpDC} pointer to an instance of your new
drawing canvas (e.g., \code{\_cpDC = new myCanvasPaneDC(this)}
instead.

%............................................................
\Meth{vDC* GetDC()}
\Indextt{GetDC}

Returns a pointer to the \code{vDC} of the current drawing canvas. The
\code{vDC} can be used for most of the drawing methods to achieve
drawing canvas independence. If your code draws via a \code{vDC} pointer,
then the same code can draw to the screen canvas or the printer canvas
depending on what the \code{vDC} points to.

%............................................................
\Meth{VCursor GetCursor()}
\Indextt{GetCursor}

Returns the id of the current cursor being used in the canvas.
See \code{SetCursor}.

%............................................................
\Meth{virtual int GetHeight()}
\Indextt{GetHeight}

Returns the height of the current drawing canvas in pixels.

%............................................................
\Meth{virtual int GetHScroll(int\& Shown, int\& Top)}
\Indextt{GetHScroll}

Get the status of the Horizontal Scroll bar. Returns 1 if the
scroll bar is displayed, 0 if not. Returns in \code{Shown} and
\code{Top} the current values of the scroll bar. See \code{SetVScroll}
for a description of the meanings of parameters.

%............................................................
\Meth{virtual int GetVScroll(int\& Shown, int\& Top)}
\Indextt{GetVScroll}

Get the status of the Vertical Scroll bar. See
\code{GetHScroll} for details.

%............................................................
\Meth{virtual int GetWidth()}
\Indextt{GetWidth}

Returns the width of the current drawing canvas in pixels.
This is either the initial size of the window, or the size
after the user has resized the window.

%............................................................
\Meth{void SetCursor(VCursor id)}
\Indextt{SetCursor}

This method sets the cursor displayed while the mouse in
in the current canvas area. The default cursor is the
standard arrow cursor used on most host platforms. You
can change the cursor displayed within the canvas area
only by calling this method.

The cursors currently supported include:

\begin{description}

\item[VC\_Arrow] The standard arrow cursor.
\item[VC\_CenterArrow] An upward point arrow.
\item[VC\_CrossHair] A cross hair cursor.
\item[VC\_EWArrows] Double ended horizontal arrows (EastWest).
\item[VC\_Hand] A hand with a pointing finger (NOT ON WINDOWS).
\item[VC\_IBar] An I bar cursor.
\item[VC\_Icon] A cursor representing an icon.
\item[VC\_NSArrows] Double ended vertical arrows (NorthSouth).
\item[VC\_Pencil] A pencil (NOT ON WINDOWS).
\item[VC\_Question] A question mark cursor (NOT ON WINDOWS).
\item[VC\_Sizer] The cursor used for sizing windows.
\item[VC\_Wait] A cursor that symbolizes waiting, usually an hour glass.
\item[VC\_X] An X shaped cursor (NOT ON WINDOWS).
\end{description}

%............................................................
\Meth{void SetWidthHeight(int width, int height)}
\Indextt{SetWidthHeight}

This will set the size of the drawing canvas to \code{height}
and \code{width} in pixels. It will also cause a \code{Resize}
event message to be sent to the window.

%............................................................
\Meth{virtual void SetHScroll(int Shown, int Top)}
\Indextt{SetHScroll}

Set the horizontal scroll bar. See \code{SetVScroll} for
a description of the parameters.

%............................................................
\Meth{virtual void SetVScroll(int Shown, int Top)}
\Indextt{SetVScroll}\index{scroll bars}

Set the vertical scroll bar. The \code{Shown} parameter
is a value from 0 to 100, and represents the percent
of the scroll bar shows of the view in the canvas. For
example, the canvas might be displaying text from a file.
If the file was 100 lines long, and the window could show
20 lines, then the value of \code{Shown} would be 20,
meaning that the canvas is showing 20 percent of the file.
As the size of the data viewed in the canvas changes, your
program should change the scroll bar to corresponding values.

The \code{Top} parameter represents where the top of
the scroll indicator should be placed.  For example,
if the first line displayed in the canvas of a 100 line file
was line 40, then  \code{Top} should be 40, representing
40 percent.

This model of a scroll bar can be mapped to all the underlying
windowing systems supported by \V, but the visual appearance
of the scroll bar will vary.

%............................................................
\Meth{virtual void ShowHScroll(int OnOrOff)}
\Indextt{ShowHScroll}
\Meth{virtual void ShowVScroll(int OnOrOff)}
\Indextt{ShowVScroll}\index{scroll bars}

When a canvas is first displayed, it will begin with both
horizontal and scroll bars not shown by default. \code{ShowHScroll}
and \code{ShowVScroll} can be used to selectively turn on and off
the canvas scroll bars. When a scroll bar is turned off or on, the size
of the canvas may changes, so you should also call \code{Resize} after
you have set the scroll bars.

You must not call either of these methods until the canvas has
actually been instantiated on the screen. This means if your
application needs to start with scroll bars, you should have
the calls to \code{ShowVScroll} and \code{ShowHScroll} in
the code of your \code{vCmdWindow} class constructor
(or other initialization code) \emph{after} calling
\code{vWindow::ShowWindow} in your class constructor.

\subsection* {Methods to Override} %------------------------

%............................................................
\Meth{virtual void FontChanged(int vf)}
\Indextt{FontChanged}

Called when the font is changed. This usually means your
application needs to resize the window and recalculate
the number of rows and columns of text that can be displayed.

%............................................................
\Meth{virtual void HPage(int Shown, int Top)}
\Indextt{HPage}

When the user moves the horizontal scroll bar, it generates an
\code{HPage} event. It is up to your program to intercept (override)
this method, and provide proper interpretation. This event usually
is used for large movements. The meaning of \code{Shown} and
\code{Top} represent the state of the scroll bar as set by
the user.  It is then up to your program to display the
correct portion of the data shown in the canvas to correspond
to these values.  Your program uses \code{SetHScroll} to
set appropriate values, and they are explained there.
The \code{Shown} value supplied here
will correspond to the value you program set for the scroll bar.
The \code{Top} value should indicate the meaningful change as input
by the user.

%............................................................
\Meth{virtual void HScroll(int step)}
\Indextt{HScroll}

This method is called when the user enters a single step command
to the scroll bar. The value of \code{step} will be positive for right or
negative for left scroll.  These scrolls
are usually interpreted as discreet steps - either a line or
screenful at a time. It is up to your application to give
an appropriate interpretation.

%............................................................
\Meth{virtual void MouseDown(int x, int y, int button)}
\Indextt{MouseDown}

This is called when the user clicks a button on the mouse. The
\code{x} and \code{y} indicates the position of the mouse in the
canvas when the button was clicked. Mouse events in vCanvasPane
are no-ops, and your subclass of \code{vCanvasPane} will need to
handle proper interpretation of mouse clicks.

Sorry, but thanks to the Macintosh, handling of buttons is a bit
nonportable. The \code{button} parameter will have a value of 1,
2, or 3. On X based systems, 1 is the left button, 2 is the
middle button, and 3 is the right button. On Windows, 1 is the
left button, and 3 is the right button. Thus, applications using
the left and right buttons are portable from X to Windows. The
single Macintosh button will return a value of 1. 

If you intend your applications to port to all three platforms,
you will have to account for the single Macintosh button. If you
ignore X's middle button, then your applications can be directly
portable from X to Windows.

%............................................................
\Meth{virtual void MouseMotion(int x, int y)}
\Indextt{MouseMotion}

This is called when the mouse moves while a button is \emph{not}
pressed, and gives the current \code{x} and \code{y} of the
mouse. Most applications will ignore this information.

%............................................................
\Meth{virtual void MouseMove(int x, int y, int button)}
\Indextt{MouseMove}

This is called when the mouse moves while a button is pressed,
and gives the new \code{x}, \code{y}, and \code{button} of the
mouse. Mouse events in vCanvasPane are no-ops, and your subclass
needs to interpret them.
Note that scaling applies to output only. The mouse
events will provide unscaled coordinates, and it is up
to your code to scale mouse coordinates appropriately.
Mouse coordinate \emph{do} have the translation added.

%............................................................
\Meth{virtual void MouseUp(int x, int y, int button)}
\Indextt{MouseUp}

This is called when the user releases the mouse button, and
gives the final location of the mouse.
Mouse events in vCanvasPane are no-ops, and your subclass
needs to interpret them.

%............................................................
\Meth{virtual void Redraw(int x, int y, int width, int height)}
\Indextt{Redraw}

\code{Redraw} is called when the canvas needs to be redrawn.
The first redraw is generated when the canvas is first created.
Other redraws are generated when the canvas is covered or uncovered
by another window, and means the contents of the canvas must
be repainted.  The \code{vCanvasPane} does not know how to repaint
the contents of the canvas, so you must override this method to
be able to keep the canvas painted.

The parameters of \code{Redraw} represent the rectangular area
that needs to be repainted. This areas is not always the whole
canvas, and it is possible that many \code{Redraw} events will
be generated in a row as the user drags a covering window off
the canvas.

The default \code{Redraw} in \code{vCanvasPane} is a
no-op, and your subclass needs to override \code{Redraw}.

%............................................................
\Meth{virtual void Resize(int newW, int newH)}
\Indextt{Resize}

A \code{Resize} event is generated when the user changes the
size of the canvas using the resize window command provided
by the host windowing system.

The default \code{Resize} in \code{vBaseGLCanvasPane} is a
no-op, and your subclass needs to override \code{Redraw}.

%............................................................
\Meth{virtual void VPage(int Shown, int Top)}
\Indextt{VPage}

See \code{HPage}.

%............................................................
\Meth{virtual void VScroll(int step)}
\Indextt{VScroll}

This method is called when the user enters a single step command
to the vertical scroll bar. The value of \code{step} will be
positive for down or negative for up scroll.  These scrolls are
usually interpreted as discreet steps - either a line or
screenful at a time. It is up to your application to give an
appropriate interpretation.

\subsection* {See Also}

vTextCanvasPane

%------------------------------------------------------------------------
\Class{vTextEditor}
\Indextt{vTextEditor}\index{Text Editor}

A complete text editing canvas pane.

\subsection* {Synopsis}

\begin{description}
	\item [Header:] \code{<v/vtexted.h>}
	\item [Class name:] vTextEditor
 	\item [Hierarchy:] vCanvasPane \rta vTextCanvasPane \rta vTextEditor
\end{description}

\subsection* {Description}

This class is a completely functional line oriented text editor.
It can edit any file with lines less than 300 characters wide that
use a linefeed, carriage return, or combination of those to mark
the end of each line.

While you need to create your own class derived from \code{vTextEditor},
your class can be very minimal. You will need to provide some
service methods for the parent \code{vCmdWindow}, such as methods
to open, read, save, and close files. Other than actually working
with the real text source and providing that source to \code{vTextEditor},
you can get a fully functional text editor with no additional work.

However, \code{vTextEditor} has been designed to allow you to extend
and add functionality to the editor if you need to. The \code{vTextEditor}
also sends messages that will allow you to place various status
messages on a status bar if you wish. The hard stuff is done for you.
You don't need to worry about mouse movements, scroll bars or scroll
messages, updating the screen, handling keystrokes, or anything else
associated with actual editing. The \code{vTextEditor} class takes
care of all those details, and provides a standard editing interface.

The following steps are required to use \code{vTextEditor}. First,
you create an instance of your derived class from your \code{vCmdWindow}
class, something like this:

\footnotesize
\begin{verbatim}
   ...

   // The Text Editor Canvas
    vedCanvas = new vedTextEditor(this);
    AddPane(vedCanvas);
   ...

   // Show Window

    ShowWindow();
    vedCanvas->ShowVScroll(1);  // Show Vert Scroll for vTextEditor

    ...
\end{verbatim}
\normalfont\normalsize

Your derived \code{vTextEditor} class should provide the methods
needed for opening and reading the text file you want to edit.
(Actually, you can edit any text source you wish.) \code{VTextEditor}
doesn't actually read or write any text itself. It maintains an
internal line buffer. (The default version of the internal buffer
is essentially limited by the amount of memory your system can
provide. The buffer methods can be overridden to provide totally
unlimited file size, if you wish.) The idea is to have your
application control where the text comes from, and then
add it a line at a time to the \code{vTextEditor} buffer.
You retrieve the text a line at a time when you want to save
the edited text. Thus, your if your code is working with
disk files, it can read the text a line at a time, and let
\code{vTextEditor} worry about the buffering.

The following code shows how to add the contents of a text file
to the \code{vTextEditor} buffer, and display it in the canvas
for the first time. Calls to \code{vTextEditor} methods are
marked with **.

\footnotesize
\begin{verbatim}
//===================>>> vedTextEditor::ReadFile <<<====================
  int vedTextEditor::ReadFile(char* name)
  {
    const int maxBuff = 300;    // Line length
    char buff[maxBuff];

    if (!name || !*name)
        return 0;
    ifstream inFile(name);      // Open the file

    if (!inFile)
        return 0;               // file not there

    resetBuff();                // ** Tell vTextEditor to init buffer

    while (inFile.getline(buff,maxBuff))  // read file
      {
        if (!addLine(buff))     // ** Add the line to the buffer
          {
            ERROR_MESSAGE("File too big -- only partially read.");
            break;
          }
      }
    inFile.close();             // Close the file
    displayBuff();              // ** Now, display the buffer
    return 1;
  }
\end{verbatim}
\normalfont\normalsize

To load text into the editor buffer,
you first call \code{resetBuff} to initialize the
buffer, then add a line at a time with calls to \code{addLine},
and finally display the text by calling \code{displayBuff}.

When your are editing (e.g., the user enters a Close command),
you retrieve the text from the \code{vTextEditor} buffer
with calls to \code{getLine}.

Then, to use the editor, you pass keystrokes from the
\code{KeyIn} method of your \code{vCmdWindow} to the \code{EditKeyIn}
method of the \code{vTextEditor}. \code{EditKeyIn} interprets
the conventional meanings of the arrow keys, etc., and lets
you edit the text in the buffer. You will also probably implement
other commands, such as Find, by using the \code{EditCommand}
method.

\code{VTextEditor} also calls several methods to notify of
text state changes, such as current line, insert or overtype,
etc. You can receive these messages by overriding the default
methods, and display appropriate information on a status bar.

While \code{vTextEditor} is very complete, there are some
things missing. The major hole is cut and paste support. This
will be added when cut and paste support is added to \V. There
is also no real undo support. Maybe someday.


\subsection* {Constructor} %------------------------------------

\Meth{vTextEditor(vBaseWindow* parent)}
\Indextt{vTextEditor!constructor}

The \code{vTextEditor} constructor requires that you specify
the parent \code{vCmdWindow}. Since you usually create the text editor object
in your \code{vCmdWindow} object, this is easy. You will probably need
to cast the \code{this} to a \code{vBaseWindow*}.

\subsection* {Utility Methods} %------------------------

\Meth{resetBuff()}
\Indextt{vTextEditor!resetBuff}

Before you load new text into the buffer, you must first
call this method. It initializes the internal state of
the text buffer.

\Meth{virtual int addLine(char* line)}
\Indextt{vTextEditor!addLine}

This method is called repeatedly to add lines to the
text buffer. The default method is limited by the amount
of memory available on the system, and this method
return 0 when it runs out of memory.

Note that the entire text buffer package can be overridden
if you need to provide unlimited file size handling. You
should examine the source code for \code{vTextEditor} to
determine the specifications of the methods you'd need
to override.

\Meth{virtual void displayBuff()}
\Indextt{vTextEditor!displayBuff}

After you have added the complete file, call \code{displayBuff}
to display the text in the window.

\Meth{virtual int getLine(char* line, int maxChars, long
lineNum)}
\Meth{virtual int getFirstLine(char* line, int maxChars)}
\Indextt{vTextEditor!getFirstLine}
\Meth{virtual int getNextLine(char* line, int maxChars)}
\Indextt{vTextEditor!getNextLine}

These are used to retrieve the edited text from the buffer.
You can use \code{getFirstLine} with \code{getNextLine} for
easy sequential retrieval, or \code{getLine} for specific
lines. These methods return -1 when all lines have been
recovered.

\Meth{virtual int EditCommand(int id, long val)}
\Indextt{vTextEditor!EditCommand}

This method provides a complete interface to the functions
provided by \code{vTextEditor}. While the basic
editing functions are also handled by \code{EditKeyIn},
\code{EditCommand} gives access to functions that typically
are either usually invoked from a menu command (such as
Find), or don't have a standard mapping to a functions
key (such as lineGoto). If you want the functionality of
these commands in your application, you will have to
provide an appropriate menu or command pane item to
support them.

Each function supported by \code{vTextEditor} has an
associated id (symbolically defined in \code{v/vtexted.h}), each
beginning with \code{ed}. Many of
the functions also take an associated value. Many editors
allow a repetition count to be specified with many commands.
For example, it is sometimes useful to be able to specify
a command to move right some specific number of characters.
The \code{val} parameter can be used to specify a value
as desired. The only function that really need a value
other than 1 (or -1 in the case of directional movement
commands) is \code{edLineGoto}.

\code{EditCommand} returns 1 if the command was executed
successfully, 0 if the command was recognized, but not
successful (the find fails, for example), and -1 if
the command was not recognized as valid.

At the time this manual was written, the following commands
are supported. Because \code{vTextEditor} is evolving,
it is likely more commands will be added. Check the
\code{v/vtexted.h} file for specification of new editor
commands. In the following descriptions, the note
``no val'' means that the \code{val} parameter is not
used. A notation of ``+/-'' means the sign of \code{val}
indicates direction.

\begin{description}
\item[edBalMatch] find matching paren (if val > 1, up to val
lines away, otherwise within a reasonable range)
\item[edBufferBottom] move to bottom of file (no val)
\item[edCharDelete] delete +/- val chars
\item[edCharFoldCase] swap case of +/- val letters
\item[edCharInsert] insert char val
\item[edCharRight] move +/- val chars right
\item[edFind] invoke TextEd's find dialog (no val)
\item[edFindNext] find next occurrence of prev (no val)
\item[edLineBeginning] move to line beginning (no val)
\item[edLineDown] move down +/- val lines in column
\item[edLineDownBeg] move down +/- val lines
\item[edLineDelete] delete +/- val lines
\item[edLineDeleteFront] delete to beginning of line (no val)
\item[edLineDeleteToEnd] delete to end of line (no val)
\item[edLineEnd] move to end of line (no val)
\item[edLineGoto] move cursor to line val
\item[edLineOpen] open val new blank lines
\item[edScrollDown] scroll +/- val lines without changing cursor
\item[edVerify] force repaint of screen (no val)
\item[edWordRight] move cursor +/- val words right

\end{description}

For a basic editor, the simplest way to use \code{EditCommand}
is to use the \code{ed*} id's to define the associated menu
items and controls, and then call \code{EditCommand} as the
default case of the \code{switch} in the \code{WindowCommand}
method of your \code{vCmdWindow}. Thus, you might have
code that looks like this:

\footnotesize
\begin{verbatim}
   ...
  static vMenu EditMenu[] = {
  ...
    {"Find", edFind, isSens,notChk,noKeyLbl,noKey,noSub},
    {"Find Next", edFindNext, isSens,notChk,noKeyLbl,noKey,noSub},
    {"Find Matching Paren", edBalMatch, isSens,notChk,
      noKeyLbl,noKey,noSub},
  ...
  };

   ...

//===========>>> vedCmdWindow::WindowCommand <<<====================
 void vedCmdWindow::WindowCommand(ItemVal id, ItemVal val,
      CmdType cType)
 {
   switch (id)
    {
     ...

      default:  // route unhandled commands through editor
       {
         if (vedCanvas->EditCommand(id, 1) < 0)
            vCmdWindow::WindowCommand(id, val, cType);
        break;
       }

    }
   ...
 }

//====================>>> vedCmdWindow::KeyIn <<<====================
  void vedCmdWindow::KeyIn(vKey keysym, unsigned int shift)
  {
    if (vedCanvas->EditKeyIn(keysym, shift) < 0)
        vCmdWindow::KeyIn(keysym, shift);
  }
\end{verbatim}
\normalfont\normalsize

\Meth{virtual int EditKeyIn(vKey key, unsigned int shift)}
\Indextt{vTextEditor!EditKeyIn}

This method is usually called from the \code{KeyIn} method
of your derived \code{vCmdWindow} class. See the above code
example. 

The default implementation of \code{EditKeyIn} handles
most of the standard keys, such as
the arrow keys, the page keys, backspace, home, delete,
insert, and end keys. It will also insert regular
character keys into the text. It ignores function keys
and non-printing control key values except tab and newline.

You can override this method to provide your own look and feel
to the editor.

\Meth{edState GetEdState()}
\Indextt{vTextEditor!GetEdState}
\Meth{void SetEdState()}
\Indextt{vTextEditor!SetEdState}

\code{VTextEditor} maintains a state structure with relevant state
information associated with various operating options of \code{vTextEditor}. It is
defined in \code{v/vtexted.h}, and has the following fields:

\footnotesize
\begin{verbatim}
    typedef struct edState
      {
        long changes,           // count of changes
             cmdCount;          // how many times to repeat command
        int
            findAtBeginning,    // leave find at beginning of pattern
            fixed_scroll,       // flag if using fixed scroll
            ins_mode,           // true if insert mode
            counter,            // counter for + insert
            echof,              // whether or not to echo action
            tabspc,             // tab spacing
            wraplm;             // right limit
      } edState;
\end{verbatim}
\normalfont\normalsize

You can query and set the state with \code{GetEdState} and
\code{SetEdState}.

\Meth{long GetLines()}
\Indextt{vTextEditor!GetLines}

Returns the number of lines in the current buffer.

\subsection* {Methods to Override} %------------------------

\Meth{virtual void ChangeLoc(long line, int col)}
\Indextt{vTextEditor!ChangeLoc}

This method is called by \code{vTextEditor} whenever the current line
or current column is changed. This information could be displayed
on a status bar, for example.

\Meth{virtual void ChangeInsMode(int IsInsMode)}
\Indextt{vTextEditor!ChangeInsMode}

This method is called by \code{vTextEditor} whenever the
insert mode is changed. If \code{IsInsMode} is true, then
the editor is in insert mode. Otherwise, it is in overtype
mode. The editor starts in insert mode. This information could be displayed
on a status bar, for example.

\Meth{virtual void StatusMessage(char* Msg)}
\Indextt{vTextEditor!StatusMessage}

The editor will call this message with a non-critical message
such as ``Pattern Not Found'' for certain operations.
This information could be displayed on a status bar, for example.

\Meth{virtual void ErrorMessage(char* Msg)}
\Indextt{vTextEditor!ErrorMessage}

The editor will call this message with a critical error message
such as ``Bad parameter value'' for certain operations.
This information could be displayed in a warning dialog, for example.

\subsection* {See Also} %---------------------------------------

vTextCanvasPane
%------------------------------------------------------------------------

\Class{vBaseGLCanvasPane}
\Indextt{vBaseGLCanvasPane}\index{OpenGL}

A specialized base class to support OpenGL graphics.

\subsection* {Synopsis}

\begin{description}
	\item [Header:] \code{<v/vbglcnv.h>}
	\item [Class name:] vBaseGLCanvasPane
 	\item [Hierarchy:] vPane \rta vBaseGLCanvasPane
\end{description}

\subsection* {Description}

This is a specialized class to provide very basic support
for the OpenGL graphics package. Unlike other \V\ canvas
panes, this class does not use a \code{vDC} class. Instead,
it has a few features designed to support OpenGL.

This is a basic class. It does not provide many convenience
methods to support OpenGL at a high level, but it does hide
all the messy details of interfacing with the host GUI
environment, and provides the first really easy way to
generate sophisticated interfaces for OpenGL applications.
A more sophisticated class
called \code{vGLCanvasPane} that will provide a number
of convenience operations is under development, but the
base class is still very useful.

By following a standard convention to structure V/OpenGL
code, it is relatively easy to generate applications.
The details of this convention are explained in the tutorial
section of this description.

See the section \code{vPane} for a general description of panes.

\subsection* {Constructor} %------------------------------------
\Meth{vBaseGLCanvasPane(unsigned int vGLmode)}
\Indextt{vBaseGLCanvasPane()}

The \code{vBaseGLCanvasPane} constructor allows you to specify
certain attributes of the visual used by OpenGL. The options,
which can be ORed together, include:

\begin{description}

\item[vGL\_Default] Use the default visual, which includes
\code{vGL\_RGB} and \code{vGL\_DoubleBuffer}. \V\ will
use this default if you don't provide a value to the constructor.

\item[vGL\_RGB] This is the standard RGBA mode used by
most OpenGL programs. The size of the RED, GREEN, and
BLUE planes are maximized according to the capabilities
of the host machine. An ALPHA plane is not included
unless the \code{vGL\_Alpha} property is also specified.

\item[vGL\_Alpha] Used to include an APLHA plane. Not all
machines support ALPHA planes.

\item[vGL\_Indexed] Use indexed rather than RGB mode. \V\
will attempt to maximize the usefulness of the palette.
You should not specify both RGB and Indexed.

\item[vGL\_DoubleBuffer] Use Double buffering if available.
Single buffering is assumed if \code{vGL\_DoubleBuffer} is not
specified.

\item[vGL\_Stereo] Use a Stereo buffer if available.

\item[vGL\_Stencil] Use Stencil mode if available.

\item[vGL\_Accum] Use accumulation buffers if available.

\item[vGL\_Depth] Use Depth mode if available.

\end{description}

Not all of these attributes are available on all OpenGL implementations,
and \V\ will attempt to get a reasonable visual based on your
specifications. For now, the \code{vGL\_Default} mode works well
for many OpenGL applications.

\V\ supports only one visual per application, and the first
\code{vBaseGLCanvasPane} created determines the attributes of
the visual used.

\subsection* {Utility Methods} %-------------------------------

The following methods provide useful service without modification. Sometimes
you will want to override some of these, but you will then usually
call these methods from your derived class. Most of these methods
are the equivalent of the normal \V\ \code{vCanvasPane} class.

%............................................................
\Meth{VCursor GetCursor()}
\Indextt{GetCursor}

Returns the id of the current cursor being used in the canvas.
See \code{SetCursor}.

%............................................................
\Meth{virtual int GetHeight()}
\Indextt{GetHeight}

Returns the height of the current drawing canvas in pixels.

%............................................................
\Meth{virtual int GetHScroll(int\& Shown, int\& Top)}
\Indextt{GetHScroll}

Get the status of the Horizontal Scroll bar. Returns 1 if the
scroll bar is displayed, 0 if not. Returns in \code{Shown} and
\code{Top} the current values of the scroll bar. See \code{SetVScroll}
for a description of the meanings of parameters.

%............................................................
\Meth{virtual int GetVScroll(int\& Shown, int\& Top)}
\Indextt{GetVScroll}

Get the status of the Vertical Scroll bar. See
\code{GetHScroll} for details.

%............................................................
\Meth{virtual int GetWidth()}
\Indextt{GetWidth}

Returns the width of the current drawing canvas in pixels.
This is either the initial size of the window, or the size
after the user has resized the window.

%............................................................
\Meth{void SetCursor(VCursor id)}
\Indextt{SetCursor}

This method sets the cursor displayed while the mouse in
in the current canvas area.
See the description of \code{vCanvasPane} for details.

%............................................................
\Meth{void SetWidthHeight(int width, int height)}
\Indextt{SetWidthHeight}

This will set the size of the drawing canvas to \code{height}
and \code{width} in pixels. It will also cause a \code{Resize}
event message to be sent to the window.

%............................................................
\Meth{virtual void SetHScroll(int Shown, int Top)}
\Indextt{SetHScroll}

Set the horizontal scroll bar
See the description of \code{vCanvasPane} for details.

%............................................................
\Meth{virtual void SetVScroll(int Shown, int Top)}
\Indextt{SetVScroll}\index{scroll bars}

Set the vertical scroll bar. 
See the description of \code{vCanvasPane} for details.

%............................................................
\Meth{virtual void ShowHScroll(int OnOrOff)}
\Indextt{ShowHScroll}
\Meth{virtual void ShowVScroll(int OnOrOff)}
\Indextt{ShowVScroll}\index{scroll bars}

See the description of \code{vCanvasPane} for details.

\subsection* {Methods to Override} %------------------------

%............................................................
\Meth{virtual void HPage(int Shown, int Top)}
\Indextt{HPage}

When the user moves the horizontal scroll bar, it generates an
\code{HPage} event. 
See the description of \code{vCanvasPane} for details.

%............................................................
\Meth{virtual void HScroll(int step)}
\Indextt{HScroll}

This method is called when the user enters a single step command
to the scroll bar. 
See the description of \code{vCanvasPane} for details.

%............................................................
\Meth{virtual void MouseDown(int x, int y, int button)}
\Indextt{MouseDown}

This is called when the user clicks a button on the mouse.

It is important to remember that all mouse coordinates are
in screen pixels, and use 0,0 as the upper left corner.
You will probably have to map them to the actual coordinates
in use by your OpenGL graphic.

See the description of \code{vCanvasPane} for details.

%............................................................
\Meth{virtual void MouseMotion(int x, int y)}
\Indextt{MouseMotion}

This is called when the mouse moves while a button is \emph{not}
pressed.
See the description of \code{vCanvasPane} for details.

%............................................................
\Meth{virtual void MouseMove(int x, int y, int button)}
\Indextt{MouseMove}

This is called when the mouse moves while a button is pressed.
See the description of \code{vCanvasPane} for details.

%............................................................
\Meth{virtual void MouseUp(int x, int y, int button)}
\Indextt{MouseUp}

This is called when the user releases the mouse button.
See the description of \code{vCanvasPane} for details.

%............................................................
\Meth{virtual void Redraw(int x, int y, int width, int height)}
\Indextt{Redraw}

\code{Redraw} is called when the canvas needs to be redrawn.
The first redraw is generated when the canvas is first created.
Other redraws are generated when the canvas is covered or uncovered
by another window, and means the contents of the canvas must
be repainted. Normally, you will put a call to the code that
redraws your OpenGl picture here.

The parameters of \code{Redraw} represent the rectangular area
that needs to be repainted. This areas is not always the whole
canvas, and it is possible that many \code{Redraw} events will
be generated in a row as the user drags a covering window off
the canvas.

The default \code{Redraw} in \code{vBaseGLCanvasPane} is a
no-op, and your subclass needs to override \code{Redraw}.

%............................................................
\Meth{virtual void Resize(int newW, int newH)}
\Indextt{Resize}

A \code{Resize} event is generated when the user changes the
size of the canvas using the resize window command provided
by the host windowing system.

The default \code{Resize} in \code{vBaseGLCanvasPane} is a
no-op, and your subclass needs to override \code{Redraw}.

%............................................................
\Meth{virtual void VPage(int Shown, int Top)}
\Indextt{VPage}

See the description of \code{vCanvasPane} for details.

%............................................................
\Meth{virtual void VScroll(int step)}
\Indextt{VScroll}

See the description of \code{vCanvasPane} for details.

\subsection* {Specific OpenGL methods} %----------------------

%............................................................
\Meth{virtual void graphicsInit(void)}
\Indextt{graphicsInit}

This method is called after the OpenGL drawing canvas has
been created, and \emph{must} be overridden by your code.
You use this method to set up whatever you would usually do
to initialize OpenGL. In practice, this is a very convenient
way to get things started. 

It is critical that you call the \code{graphicsInit} method
in the base \code{vBaseGLCanvasPane} class \emph{first},
then whatever OpenGL calls you need.
See the example in the OpenGL tutorial section for more details.

%............................................................
\Meth{void vglMakeCurrent(void)}
\Indextt{vglMakeCurrent}

This method should be called by your program before you
call any OpenGL drawing code. Normally, this is called
first thing in \code{Redraw}, or whatever code you use
to draw with. It is essential to call this, and since
it is cheap to call this for an already current drawing
canvas, it is better to be safe.

%............................................................
\Meth{virtual void vglFlush(void)}
\Indextt{vglFlush}

Call this method after you are finished calling OpenGL to
draw a picture. It automatically handles the details of
displaying your picture in the window, including double
buffering and synchronization. It is normally found in your
\code{Redraw} method.
        
%............................................................
\Meth{virtual XVisualInfo* GetXVisualInfo()}
\Indextt{GetXVisualInfo()}

This method is specific to X, and will return a pointer to the
\code{XVisualInfo} structure currently being used. There will
be an equivalent method available for MS-Windows.

\subsection* {Tutorial} %---------------------------------------

A minimal V/OpenGL application will consist of a class derived
from \code{vApp}, a class derived from \code{vCmdWindow}, and
a canvas pane class derived from \code{vBaseGLCanvasPane}.
Most of your drawing code will be in or called from your derived
canvas pane.

Within that class, you will minimally need to override the
\code{graphicsInit} method, and the \code{Redraw} method. The
following code fragment, adapted directly from the example code
in Mark J. Kilgard's book, \emph{OpenGL, Programming for the X
Window System}, shows how simple it can be to draw a picture.
The full code can be found in the \code{opengl/shapes} directory
in the \V\ distribution.

\footnotesize
\begin{verbatim}

  static int initDone = 0;

  ......

//==========>>> testGLCanvasPane::graphicsInit <<<=================
  void testGLCanvasPane::graphicsInit(void)
  {
    // Always call the superclass first!
    vBaseGLCanvasPane::graphicsInit();

    // Example from Mark Kilgard
    glEnable(GL_DEPTH_TEST);
    glClearDepth(1.0);
    glClearColor(0.0, 0.0, 0.0, 0.0);  /* clear to black */
    glMatrixMode(GL_PROJECTION);
    gluPerspective(40.0, 1.0, 10.0, 200.0);
    glMatrixMode(GL_MODELVIEW);
    glTranslatef(0.0, 0.0, -50.0);
    glRotatef(-58.0, 0.0, 1.0, 0.0);

    initDone = 1;
  }

//============>>> testGLCanvasPane::Spin <<<=======================
  void testGLCanvasPane::Spin()
  {
    // Called from the parent CmdWindow for animation
    vglMakeCurrent();              // Call this FIRST!
    glRotatef(2.5, 1.0, 0.0, 0.0);
    Redraw(0,0,0,0);
  }

//============>>> testGLCanvasPane::Redraw <<<=====================
  void testGLCanvasPane::Redraw(int x, int y, int w, int h)
  {
    static int inRedraw = 0;

    if (inRedraw || !initDone)  // Don't draw until initialized
 	return;

    inRedraw = 1;               // Don't allow recursive redraws.

    vglMakeCurrent();           // Call this to make current

    // Code taken directly from Mark J. Kilgard's example
    // Draws 3 intersecting triangular planes
    glClear(GL_COLOR_BUFFER_BIT | GL_DEPTH_BUFFER_BIT);

    glBegin(GL_POLYGON);
    glColor3f(0.0, 0.0, 0.0); glVertex3f(-10.0, -10.0, 0.0);
    glColor3f(0.7, 0.7, 0.7); glVertex3f(10.0, -10.0, 0.0);
    glColor3f(1.0, 1.0, 1.0); glVertex3f(-10.0, 10.0, 0.0);
    glEnd();

    glBegin(GL_POLYGON);
    glColor3f(1.0, 1.0, 0.0); glVertex3f(0.0, -10.0, -10.0);
    glColor3f(0.0, 1.0, 0.7); glVertex3f(0.0, -10.0, 10.0);
    glColor3f(0.0, 0.0, 1.0); glVertex3f(0.0, 5.0, -10.0);
    glEnd();

    glBegin(GL_POLYGON);
    glColor3f(1.0, 1.0, 0.0); glVertex3f(-10.0, 6.0, 4.0);
    glColor3f(1.0, 0.0, 1.0); glVertex3f(-10.0, 3.0, 4.0);
    glColor3f(0.0, 0.0, 1.0); glVertex3f(4.0, -9.0, -10.0);
    glColor3f(1.0, 0.0, 1.0); glVertex3f(4.0, -6.0, -10.0);
    glEnd();

    vglFlush();    // Call when done drawing to display

    inRedraw = 0;  // Not in here any more
  }

  ....
\end{verbatim}
\normalfont\normalsize

Note that this example includes a method called \code{Spin}.
It is used to animate the intersecting planes. In a \V\
OpenGL application, the easiest way to implement animation
is with the timer. Create a timer in the Command Window
class, and then call the animation code in the canvas
in response to timer events. You should keep code to
prevent recursive redraws if the timer events end up
occurring faster than the picture can be rendered, which
might happen for complex pictures or heavily loaded systems.
See the example code in the \code{v/opengl} directory for
a complete example of animation using the timer.

\subsection* {See Also} %---------------------------------------

vCanvasPane

%------------------------------------------------------------------------

\Class{vColor}
\Indextt{vColor}
\index{color!in V}

A class for handling and specifying colors.

\subsection* {Synopsis}

\begin{description}
	\item [Header:] \code{<v/vcolor.h>}
	\item [Class name:] vColor
\end{description}

\subsection* {Description}

The \V\ color model allows you to specify colors as an RGB value.
\index{color!RGB model}
The intensity of each primary color, red, green, and blue are
specified as a value between 0 and 255. This allows you to
specify up to $2^{24}$ colors. Just how many of all these
colors you can see and how they will look on your display will
depend on that display. Even so, you can probably count  on
(255,0,0) being something close to red on most displays. Given
this 24 bit model, the \code{vColor} class allows you to define
colors easily.

In order to make using colors somewhat easier, \V\ has defined
\index{color!standard}
a standard array of 16 basic colors that you can
access by including \code{v/vcolor.h>}. This array is called
\code{vStdColors}. You index the array using the symbols \code{vC\_Black},
\code{vC\_Red}, \code{vC\_DimRed}, \code{vC\_Green}, \code{vC\_DimGreen},
\code{vC\_Blue}, \code{vC\_DimBlue}, \code{vC\_Yellow},
\code{vC\_DimYellow}, \code{vC\_Magenta}, \code{vC\_DimMagenta},
\code{vC\_Cyan}, \code{vC\_DimCyan}, \code{vC\_DarkGray},
\code{vC\_MedGray}, and \code{vC\_White}. For example, use the standard color
\code{vStdColors[vC\_Green]} to represent green. You can also get a
\code{char} for the color by using the symbol to index the
\code{char* vColorName[16]} array.

The file \code{<v/vcb2x4.h>} contains definitions for
\index{color!selection buttons}
8 color buttons in a 2 high by 4 wide frame. The file
\code{<v/vcb2x8.h>} has a 2 by 8 frame of all 16 standard colors.
You can specify the size of each button in the frame by
defining \code{vC\_Size}. The default is 8. You can also specify
the location in a dialog of the color button frame by defining
the symbols \code{vC\_Frame, vC\_RightOf,} and \code{vC\_Below}.
The ids of each button in the frame correspond to the color
indexes, but with a \code{M} prefix (e.g., \code{M\_Red} for
\code{vC\_Red}). See the example in \code{v/examp} for and example
of using the standard color button frames.

Also note that unlike most other \V\ objects, it makes perfect
sense to assign and copy \code{vColor} values. Thus, assignment,
copy constructor, and equality comparison operators are provided.

\subsection* {Constructor} %------------------------------------

\Meth{vColor(unsigned int rd 0, unsigned int gr = 0, unsigned int
bl = 0)}
\Indextt{vColor}

The class has been defined so you can easily initialize a color
either by using its constructor directly, or indirectly via an
array declaration. Each color has a red, green, and blue value in
the range of 0 to 255.

\footnotesize
\begin{verbatim}
  // Declare Red via constructor
  vColor btncolor(255, 0 , 0);   // Red

  // Declare array with green and blue
  vColor GreenAndBlue[2] =
    {
      (0, 255, 0),              // Green
      (0, 0, 255)               // Blue
    };
\end{verbatim}
\normalfont\normalsize

\subsection* {Utility Methods} %--------------------------------

\Meth{BitsOfColor()}

This method returns the number of bits used by the machine
to display to represent color. A value of 8, for example,
means the computer is using 8 bits to show the color.

\Meth{ResetColor(unsigned int rd = 0, unsigned int gr = 0,
unsigned int bl = 0)}
\Meth{ResetColor(vColor\& c)}
\Indextt{ResetColor}

Like the \code{Set} method, this method will set all three values
of the color at once. However, \V\ tries to preserve entries in
the system color palette or color map with \code{ResetColor}.
You can also pass a \code{vColor} object.

Consider the following code excerpt:

\footnotesize
\begin{verbatim}
    vColor aColor;        // A V Color
    vBrush aBrush;
    int iy;

    ...

    for (iy = 0 ; iy < 128 ; ++iy)
      {
        aColor.Set(iy,iy,iy);     // Set to shade of gray
        aBrush.SetColor(aColor);  // Set brush
        canvas.DrawLine(10,iy+100,200,iy+100);  // Draw line
      }
    ...

\end{verbatim}
\normalfont\normalsize

This example will use up 128 color map entries on some systems
(X, for example). Once a system has run out of entries, \V\ will
draw in black or white. When these systems run out of new color
map entries, the color drawn for new colors will be black or
white. 

\footnotesize
\begin{verbatim}
    vColor aColor;        // A V Color
    vBrush aBrush;
    int iy;

    ...

    for (iy = 0 ; iy < 128 ; ++iy)
      {
        aColor.ResetColor(iy,iy,iy);     // Set to shade of gray
        aBrush.SetColor(aColor);  // Set brush
        canvas.DrawLine(10,iy+100,200,iy+100);  // Draw line
      }
    ...

\end{verbatim}
\normalfont\normalsize

This example accomplishes the same as the first, but does not use
up color map entries. Instead, the entry used for \code{aColor}
is reused to get better use of the color map. If your application
will be working with a large number of colors that will vary,
using \code{ResetColor} will minimize the number of color map accesses.

On some systems, and systems with a full 24 bits of color,
\code{ResetColor} and \code{Set} work identically.

\emph{WARNING}: If you intend to use \code{ResetColor} on a
\code{vColor} object, then \code{ResetColor} is the only way
you should change the color of that object. You should not
use the color assignment operator, or \code{Set}. \code{ResetColor}
needs to do some unconventional things internally to
preserve color palette entries, and these can be incompatible
with regular assignment or \code{Set}. You can, however,
safely use such a \code{vColor} object with any other \code{vColor}
object. For example:

\footnotesize
\begin{verbatim}
    vColor c1, c2;

    c1.ResetColor(100,100,100);    // You can use c1 with others.
    c2 = c1;                       // OK, but this = now makes c2
                                   // incompatible with ResetColor.
    c2.ResetColor(200,200,200);    // DON'T DO THIS
\end{verbatim}
\normalfont\normalsize

\Meth{Set(unsigned int rd = 0, unsigned int gr = 0, unsigned int bl = 0)}

Set all three values of the color at once.

\Meth{void SetR(unsigned int rd = 0)}
\Indextt{SetR}

Set the Red value.

\Meth{void SetG(unsigned int gr = 0)}
\Indextt{SetG}

Set the Green value.

\Meth{void SetB(unsigned int bl = 0)}
\Indextt{SetB}

Set the Blue value.

\Meth{unsigned int r()}

Get the Red value.

\Meth{unsigned int g()}

Get the Green value.

\Meth{unsigned int b()}

Get the Blue value.

\Meth{int operator ==}

Compare two color objects for equality.

\Meth{int operator !=}

Compare two color objects for inequality.

\subsection* {Notes about color}

The color model used by \V\ attempts to hide most of the
details for using color. However, for some applications
you may end up confronting some of the sticky issues of
color.

Most machines in use in 1996 will not support all $2^{24}$
colors that can be represented by the RGB color specification.
Typically, they devote 8 or 16 bits to each pixel. This
means that the 24-bit RGB colors must be mapped to the
smaller 8-bit or 16-bit range. This mapping is usually
accomplished by using a palette or colormap.

\V\ tries to use the default system color palette provided by the
machine it is running on. On some systems, such as X, it is
possible to run out of entries in the color map. Others, like
Windows, map colors not in the color palette to dithered colors.
\V\ provides two methods to help with this problem. First,
\code{vColor::BitsOfColor()} tells you how many bits are used by
the running system to represent color. The method  
\code{vColor::ResetColor(r,g,b)} can be used to change the value
of a color without using up another entry in the system color
map. For now, these methods should allow you to work with color
with pretty good flexibility. Eventually, \V\ may include more
direct support for color palettes.


\subsection* {See Also}

C\_ColorButton, vCanvas


%------------------------------------------------------------------------

\Class{vDC}
\Indextt{vDC}

This is the base class that defines all the drawing methods provided
by the various drawing canvases.

\subsection* {Synopsis}

\begin{description}
	\item [Header:] \code{<v/vdc.h>}
	\item [Class name:] vDC
\end{description}

\subsection* {Description}

All drawing classes such as \code{vCanvasPaneDC} and \code{vPostScriptDC}
are derived from this class. Each drawing class will support
these methods as needed. Not all drawing classes have the same
scale, and printer drawing canvases provide extra support for
paging. Your code will not normally need to include \code{vdc.h}.

See the specific sections for details of drawing classes.

\subsection* {Utility Methods}

%............................................................
\Meth{virtual void BeginPage()}
\Indextt{BeginPage}

Supported by printer canvases. Call to specify a page is
beginning. Bracket pages with \code{BeginPage} and \code{EndPage}
calls.

%............................................................
\Meth{virtual void BeginPrinting()}
\Indextt{BeginPrinting}

Required by printer canvases. Call to specify a document is
beginning. You \emph{must} bracket documents with \code{BeginPrinting}
and \code{EndPrinting} calls. \code{BeginPrinting} includes an
implicit call to \code{BeginPage}.

%............................................................
\Meth{virtual void Clear()}
\Indextt{Clear}

Clear the canvas to the background color. No op on printers.

%............................................................
\Meth{virtual void ClearRect(int x, int y, int width, int height)}
\Indextt{ClearRect}

Clear a rectangular area starting at x,y of height and width. No
op on printers.

%............................................................
\Meth{void CopyFromMemoryDC(vMemoryDC* memDC, int destX, int
destY, int srcX = 0, int srcY = 0, int srcW = 0, int srcH = 0)}
\Indextt{CopyFromMemoryDC}
 
This method is used to copy the image contained in a \code{vMemoryDC}
to another drawing canvas. The parameter \code{memDC} specifies
the \code{vMemoryDC} object, and \code{destX} and \code{destY}
specify where the image is to be copied into \code{this} drawing
canvas (which will usually be 0,0). If you use the default
values for \code{srcX=0}, \code{srcY=0}, \code{srcW=0}, and
\code{srcH=0}, the entire source canvas will be copied.

Beginning with \V release 1.13, \code{CopyFromMemoryDC} provides
the extra parameters to specify an area of the source to copy.
You can specify the source origin, and its width and height.
The default values for these allow backward call and behavior
compatibility.

One of the most useful uses of this is to draw both the canvas
pane drawing canvas, and to a memory drawing canvas, and then
use \code{CopyFromMemoryDC} to copy the memory canvas to the
canvas pane for \code{Redraw} events.

%............................................................
\Meth{virtual void DrawAttrText(int x, int y, char* text, const
ChrAttr attr)}
\Indextt{DrawAttrText}

Draw text using the current font with specified attributes at
given x, y.

\index{text drawing attributes}
\code{ChrAttr attr} is used to specify attributes to override
some of the text drawing characteristics normally determined by
the pen and font. Specifying \code{ChNormal} means the current
pen and font will be used.  \code{ChReverse} is used to specify
the text should be drawn reversed or highlighted, using the
current font and pen. You can also specify 16 different standard
colors to override the pen color. You use \code{ORed}
combinations the basic color attributes \code{ChRed}, \code{ChBlue},
and \code{ChGreen}. Most combinations are also provided as
\code{ChYellow}, \code{ChCyan}, \code{ChMagenta},  \code{ChWhite},
and \code{ChGray}. These colors can be combined with \code{ChDimColor}
can be used for half bright color combinations (or you can use
\code{ChDimRed}, etc.). You can combine color attributes with
\code{ChReverse}. Attributes such as boldface, size, and
underlining are attributes of the font.

%............................................................
\Meth{virtual void DrawColorPoints(int x, int y, int nPts, vColor*
pts)}
\Indextt{DrawColorPoints}

Draw an array of \code{nPts} \code{vColors} as points starting at
x,y. This method is useful for drawing graphical images, and
bypasses the need to set the pen or brush for each point.
Typically, \code{DrawColorPoints} will be significantly faster
than separate calls to \code{DrawPoint}.

%............................................................
\Meth{virtual void DrawEllipse(int x, int y, int width, int
height)}
\Indextt{DrawEllipse}

Draw an ellipse inside the bounding box specified by x, y, width,
and height.
The current Pen will be used to draw the shape, and the current
Brush will be used to fill the shape.

%............................................................
\Meth{virtual void DrawIcon(int x, int y, vIcon\& icon)}
\Indextt{DrawIcon}

A \code{vIcon} is drawn at x,y using the current Pen.
Note that only the location of an icon is scaled. The icon
will retain its original size.

%............................................................
\Meth{virtual void DrawLine(int x, int y, int xend, int yend)}
\Indextt{DrawLine}

Draw a line from x,y to xend,yend.
The current Pen will be used to draw the line.

%............................................................
\Meth{virtual void DrawLines(vLine* lineList, int count)}
\Indextt{DrawLines}

Draws the \code{count} lines contained in the list \code{lineList}.

The current Pen will be used to draw the lines.

The type \code{vLine} is defined in \code{v\_defs.h} as:
\Indextt{vLine}
\footnotesize
\begin{verbatim}
    typedef struct vLine
      {
        short x, y, xend, yend;
      } vLine;
\end{verbatim}
\normalfont\normalsize


%............................................................
\Meth{virtual void DrawPoint(int x, int y)}
\Indextt{DrawPoint}

Draw a point at x,y using the current Pen.

%............................................................
\Meth{virtual void DrawPoints(vPoint* pointList, int count)}
\Indextt{DrawLines}

Draws the \code{count} points contained in the list \code{pointList}.

The current Pen will be used to draw the points.

The type \code{vPoint} is defined in \code{v\_defs.h} as:
\Indextt{vLine}
\footnotesize
\begin{verbatim}
    typedef struct vPoint
      {
        short x, y;
      } vPoint;
\end{verbatim}
\normalfont\normalsize

%............................................................
\Meth{virtual void DrawPolygon(int n, vPoint points[],
int fillMode = vAlternate)}
\Indextt{DrawPolygon}

A closed polygon of n points is drawn. Note that the
first and last element of the point list must specify
the same point.
The current Pen will be used to draw the shape, and the current
Brush will be used to fill the shape.

The fillMode parameter specifies one of two alternative filling
algorithms, \code{vAlternate} or \code{vWinding}. These algorithms correspond
to the equivalent algorithms on the native platforms.

The type \code{vPoint} is defined in \code{v\_defs.h} as:
\Indextt{vPoint}
\footnotesize
\begin{verbatim}
    typedef struct vPoint       // a point
      {
        short x, y;             // X version
      } vPoint; 
\end{verbatim}
\normalfont\normalsize

%............................................................
\Meth{virtual void DrawRoundedRectangle(int x, int y, int width,
int height, int radius = 10)}
\Indextt{DrawRoundedRectangle}

Draw a rectangle with rounded corners at x,y of size width and
height. The radius specifies the radius of the circle used to
draw the corners. If a radius of less than 0 is specified, the
radius of the corners will be ((width+height)/-2*radius) which
gives a more or less reasonable look for various sized
rectangles. The current Pen will be used to draw the shape, and
the current Brush will be used to fill the shape.

%............................................................
\Meth{virtual void DrawRectangle(int x, int y, int width,
int height)}
\Indextt{DrawRectangle}

Draw a rectangle with square corners at x,y of size width and
height. The current Pen will be used to draw the shape, and the
current Brush will be used to fill the shape.

%............................................................
\Meth{virtual void DrawRectangles(vRect* rectList, int count)}
\Indextt{DrawRectangles}

Draw a list of \code{count} \code{vRect} rectangles pointed to by
the list \code{rectList}.
The current Pen will be used to draw the rectangles, and the
current Brush will be used to fill the rectangles.

The type \code{vRect} is defined in \code{v\_defs.h} as:
\Indextt{vRect}
\footnotesize
\begin{verbatim}
    typedef struct vRect
      {
        short x, y, w, h;
      } vRect;
\end{verbatim}
\normalfont\normalsize

%............................................................
\Meth{virtual void DrawRubberLine(int x, int y, int xend, int yend)}
\Indextt{DrawRubberLine}

Draw a rubber-band line from x, y to xend, yend. This method is most
useful for showing lines while the mouse is down. By first drawing
a rubber line, and then redrawing over the same line with \code{DrawRubberLine}
causes the line to be erased. Thus, pairs of rubber lines can track
mouse movement. The current Pen is used to determine line style.

%............................................................
\Meth{virtual void DrawRubberEllipse(int x, int y, int width,
int height)}
\Indextt{DrawRubberEllipse}

Draw a rubber-band Ellipse. See DrawRubberLine.

%............................................................
\Meth{virtual void DrawRubberPoint(int x, int y)}
\Indextt{DrawRubberPoint}

Draw a rubber-band point. See DrawRubberLine.

%............................................................
\Meth{virtual void DrawRubberRectangle(int x, int y, int width,
int height)}
\Indextt{DrawRubberRectangle}

Draw a rubber-band rectangle. See DrawRubberLine.

%............................................................
\Meth{virtual void DrawText(int x, int y, char* text)}
\Indextt{DrawText}

Simple draw text at given x, y using the current font and
current pen. Unlike icons and other \V\ drawing objects,
\code{x} and \code{y} represent the lower left corner of the
first letter of the text. Using a \code{vSolid} pen results
in the text being drawn in with the pen's color using
the current background color. Using a \code{vTransparent}
pen results in text in the current color, but just drawing
the text over the current canvas colors. (See \code{vPen::SetStyle}.)

%............................................................

\Meth{virtual void EndPage()}
\Indextt{EndPage}

Supported by printer canvases. Call to specify a page is
ending. Bracket pages with \code{BeginPage} and \code{EndPage}
calls.

%............................................................
\Meth{virtual void EndPrinting()}
\Indextt{EndPrinting}

Supported by printer canvases. Call to specify a document is
ending. Bracket documents with \code{BeginPrinting} and
\code{EndPrinting} calls. \code{EndPrinting} includes
an implicit call to \code{EndPage}.

%............................................................
\Meth{virtual vBrush GetBrush()}
\Indextt{GetBrush}

Returns a copy of the current brush being used by the canvas.

%............................................................
\Meth{virtual vFont GetFont()}
\Indextt{GetFont}

Returns a copy of the current font of the drawing canvas.

%............................................................
\Meth{virtual vBrush GetPen()}
\Indextt{GetPen}

Returns a copy of the current pen being used by the canvas.

%............................................................
\Meth{virtual int GetPhysHeight()}
\Indextt{GetPhysHeight}

Returns the maximum physical y value supported by the
drawing canvas. Especially useful for determining scaling
for printers.

%............................................................
\Meth{virtual int GetPhysWidth()}
\Indextt{GetPhysWidth}

Returns the maximum physical x value supported by the
drawing canvas. Especially useful for determining scaling
for printers.

%............................................................
\Meth{virtual void GetScale(int\& mult, int\& div)}
\Indextt{GetScale}

Returns the scaling factors for the
canvas. See \code{SetScale}.

%............................................................
\Meth{void GetTranslate(int\& x, int\& y)}
\Indextt{GetTranslate}
\Meth{int GetTransX()}
\Indextt{GetTransX}
\Meth{int GetTransY()}
\Indextt{GetTransY}

Returns the current x and y translation values.

%............................................................
\Meth{virtual void SetBackground(vColor\& color)}
\Indextt{SetBackground}
\index{color!background}

This sets the background of the drawing canvas to the
specified color.

%............................................................
\Meth{virtual void SetBrush(vBrush\& brush)}
\Indextt{SetBrush}

This sets the brush used by the drawing canvas. Brushes are used
for the filling methods such as \code{vDrawPolygon}. It is
important to call \code{SetBrush} whenever you change any
attributes of a brush used by a drawing canvas.

%............................................................
\Meth{virtual void SetFont(vFont\& vf)}
\Indextt{SetFont}

Change the font associated with this canvas. The default method
handles changing the font and calls the FontChanged method for
the canvas pane.

%............................................................
\Meth{virtual void SetPen(vPen\& pen)}
\Indextt{SetPen}

Sets the current pen of the canvas to pen. Pens are used
to draw lines and the outlines of shapes. It is important
to call \code{SetPen} whenever you change any attributes of
a pen used by a drawing canvas.

%............................................................
\Meth{virtual void SetScale(int mult, int div)}
\Indextt{SetScale}

Sets the scaling factor. Each coordinate passed to the 
drawing canvas is first multiplied by mult and then divided
by div. Thus, to scale by one third, set mult to 1 and div to 3.
Many applications will never have to worry about scaling.
Note that scaling applies to output only. The mouse
events will provide unscaled coordinates, and it is up
to your code to scale mouse coordinates appropriately.

%............................................................
\Meth{void SetTranslate(int x, int y)}
\Indextt{SetTranslate}
\Meth{void SetTransX(int x)}
\Indextt{SetTransX}
\Meth{void SetTransY(int y)}
\Indextt{SetTransY}

These methods set the internal translation used by the
drawing canvas. Each coordinate sent to the various
drawing methods (e.g., \code{DrawRectangle}) will be
translated by these coordinates. This can be most useful
when using the scroll bars to change which part of a
drawing is visible on the canvas. Your application will
have to handle proper mapping of mouse coordinates.

%............................................................
\Meth{int TextHeight(int\& ascent, int\& descent)}
\Indextt{TextHeight}

This function returns the total height of the font \code{fontId}.
The total height of the font is the sum of the \code{ascent} and
\code{descent} heights of the font \code{fontId}. Each character
ascends \code{ascent} pixels above the Y coordinate where it is
being drawn, and \code{descent} pixels below the Y coordinate.

%............................................................
\Meth{int TextWidth(char* str)}
\Indextt{TextWidth}

Returns the width in pixels or drawing points of the string
\code{str} using the currently set font of the canvas.

%------------------------------------------------------------------------
\Class{vMemoryDC}
\Indextt{vMemoryDC}

A memory drawing canvas.

\subsection* {Synopsis}

\begin{description}
	\item [Header:] \code{<v/vmemdc.h>}
	\item [Class name:] vMemoryDC
\end{description}

\subsection* {Description}

This drawing canvas can be used to draw to memory. Like
all drawing canvases, the available methods are described
in \code{vDC}. A very effective technique for using a memory
canvas is to draw to both the screen canvas pane and a memory
canvas during interactive drawing, and use the memory canvas to
update the screen for \code{Redraw} events. This is especially
useful if your application requires extensive computation to
draw a screen.

\subsection* {Methods}

\Meth{vMemoryDC(int width, int height)}
\Indextt{vMemoryDC}

The constructor is used to construct a memory DC of the
specified width and height. This can be anything you need.
If you are using the memory DC to update the screen for
\code{Redraw} events, then it should be initialized to be
big enough to repaint whatever you will be drawing on the
physical screen. The methods  \code{vApp::ScreenWidth()} and
\code{vApp::ScreenHeight()} can be used to obtain the maximum
size of the physical screen.

The method \code{CopyFromMemoryDC} is used to copy the contents
of a memory DC to another DC. This can be another memory DC, but
will usually be a canvas pane DC.

%------------------------------------------------------------------------
\Class{vPen}
\Indextt{vPen}\index{pen}

A class to specify the pen used to draw lines and shapes.

\subsection* {Synopsis}

\begin{description}
	\item [Header:] \code{<v/vpen.h>}
	\item [Class name:] vPen
\end{description}

\subsection* {Description}

Pens are used to draw lines and the outlines of shapes. Pens have
several attributes, including color, width, and style.

\subsection* {Methods}

\Meth{vPen(unsigned int r = 0, unsigned int g = 0, unsigned int b = 0, 
int width = 1, int style = vSolid)}
\Indextt{vPen}

The constructor for a pen allows you to specify the pen's color,
width, and style. The default will construct a solid black pen of
width 1.

\Meth{int operator ==, !=}

You can use the operators \code{==} and \code{!=} for comparisons.

\Meth{vColor GetColor()}
\Indextt{GetColor}

This method returns the current color of the pen as a \code{vColor} object.

\Meth{int GetStyle()}
\Indextt{GetStyle}

This method returns the current style of the pen.

\Meth{void GetWidth()}
\Indextt{GetWidth}

This gets the width of the line the pen will draw.

\Meth{void SetColor(vColor\& c)}
\Indextt{SetColor}
\index{color!pen}

You can use this method to set the pen color by passing
in a \code{vColor} object.

\Meth{void SetStyle(int style)}
\Indextt{SetStyle}

This method is used to change the style of a pen. Styles include:

\begin{description}
	\item [vSolid] The pen draws a solid line.
	\item [vTransparent] The pen is transparent. A transparent
pen can be used to avoid drawing borders around shapes. When drawing
text, a transparent pen draws the text over the existing background.
	\item [vDash] The pen draws a dashed line.
	\item [vDot] The pen draws a dotted line.
	\item [vDashDot] The pen draws an alternating dash and dotted line.
\end{description}

\Meth{void SetWidth(int width)}
\Indextt{SetWidth}

This sets the width of the line the pen will draw.

%------------------------------------------------------------------------
\Class{vPrintDC}
\Indextt{vPrintDC}\index{printing}

A printer drawing canvas.

\subsection* {Synopsis}

\begin{description}
	\item [Header:] \code{<v/vprintdc.h>}
	\item [Class name:] vPrintDC
\end{description}

\subsection* {Description}

This drawing canvas can be used to draw to a printer. Like
all drawing canvases, the available methods are described
in \code{vDC}. A very effective technique for combining a printer
DC and a screen DC is to pass a pointer to either a \code{vCanvasPaneDC}
or a \code{vPrintDC} to the code that draws the screen. The same
code can then be used to draw or print.

To successfully use a \code{vPrintDC}, your code must
obtain the physical size of the page in units using
\code{GetPhysWidth} and \code{GetPhysHeight}. On
paper, these represent 1/72 inch points, and correspond
very closely, but not exactly, to a pixel on the screen.

You must bracket the printing with calls to \code{BeginPrinting}
and \code{EndPrinting}. Use \code{BeginPage} and \code{EndPage}
to control paging. Note that the width of text will not
necessarily be the same on a \code{vCanvasPaneDC} and a \code{vPrintDC},
even for the same fonts. Also, the size of the paper represents
the entire page. Most printers cannot actually print all the way
to the edges of the paper, so you will usually use \code{vDC:SetTranslate}
to leave some margins. (Don't forget to account for margins when
you calculate what can fit on a page.)

The implementation of \code{vPrintDC} is somewhat platform
dependent. For X, \code{vPrintDC} represents a PostScript
printer, and is derived from the class \code{vPSPrintDC}. For
Windows, \code{vPrintDC} is derived from the \code{vWinPrintDC}
class. To get platform independent operation for your
application, use \code{vPrintDC}. On Windows, you can also use
the PostScript version directly if you want by using the \code{vPSPrintDC}
class, but the program will not conform to standard Windows
behavior.

\subsection* {Methods}

\Meth{void SetPrinter(vPrinter\& printer)}
\Indextt{SetPrinter}

This method is used to associate a \code{vPrinter} with
a \code{vPrintDC}.
By default, a \code{vPrintDC} represents standard
8.5x11 inch Letter paper printed in black and white in
portrait orientation. You can use \code{vPrinter::Setup} to allow
the user to change the attributes of the printer, then use
\code{SetPrinter} to associate those attributes with the \code{vPrintDC}.
Note: If you change the default printer attributes, you \emph{must}
call \code{SetPrinter} before doing any drawing to the DC.

\subsection* {Example}

This is a simple example taken from the \code{VDraw} demo program.
\code{Print} is called to print the current drawing. \code{Print}
calls \code{vPrinter::Setup} to set the printer characteristics,
and then calls \code{DrawShapes} with a pointer to the \code{vPrintDC}.
\code{DrawShapes} is also called to repaint the screen using the
\code{vCanvasPaneDC}. By carefully planning for both screen and
printer drawing, your program can often share drawing code in
this fashion.

\footnotesize
\begin{verbatim}
//===================>>> myCanvasPane::Print <<<=================
  void myCanvasPane::Print()
  {
    // Print current picture

    vPrintDC pdc;               // create a vPrintDC object
    vPrinter printer;           // and a printer to set attributes

    printer.Setup("test.ps");   // setup the printer
    pdc.SetPrinter(printer);    // change to the printer we setup

    if (!pdc.BeginPrinting())   // call BeginPrinting first
        return;

    pdc.SetTranslate(36,36);    // Add 1/2" (36 * 1/72") margins

    DrawShapes(&pdc);           // Now, call shared drawing method

    pdc.EndPrinting();          // Finish printing
  }

//===================>>> myCanvasPane::DrawShapes <<<=================
  void myCanvasPane::DrawShapes(vDC* cp)
  {
    // Common code for drawing both on Screen and Printer
    ...
  }
\end{verbatim}
\normalfont\normalsize

%------------------------------------------------------------------------
\Class{vPrinter}
\Indextt{vPrinter}\index{printer attributes}

A printer object, with a dialog to interactively set printer
attributes.

\subsection* {Synopsis}

\begin{description}
	\item [Header:] \code{<v/vprinter.h>}
	\item [Class name:] vPrinter
\end{description}

\subsection* {Description}

The \code{vPrintDC} class prints to a printer (or a file that
will eventually be printed). Printers have such attributes as
size of paper, page orientation, color capability, etc. By
calling the \code{vPrinter::Setup} dialog before printing, the
user will be given the option of setting various printer
attributes.

The exact functionality of the \code{Setup} dialog will be
platform dependent. By using the \code{vPrinter} class, you will
get the behavior appropriate for the platform. If you want to use
the \code{vPSPrintDC} class for PostScript support on Windows,
you can use \code{vPSPrinter} directly.

You can use the various methods associated with a \code{vPrinter}
to get printer attributes as needed to during drawing to
the \code{vPrintDC}.

\subsection* {Methods}

\Meth{int GetCopies()}
\Indextt{GetCopies}

\Meth{void SetCopies(int s)}
\Indextt{SetCopies}

Many printers support printing multiple copies of the same
document. This attributes controls the number of copies printed.
The \code{Setup} dialog will provide control of this \emph{if} it
is supported.

\Meth{char* GetDocName()}
\Indextt{GetDocName}

Printer output may be directed to a file rather than the printer.
If it is, this will return the name of the file the output will
be sent to.

\Meth{int GetPaper()}
\Indextt{GetPaper}

\Meth{char* GetPaperName()}
\Indextt{GetPaperName}

Printers can print a variety of papers. The user may be
able to select which paper from the \code{Setup} dialog.
The printers supported are defined in the \code{vprinter.h}
header file (or the base class used by \code{vPrinter}).

\Meth{int GetPortrait()}
\Indextt{GetPortrait}

\Meth{void SetPortrait(int p)}
\Indextt{SetPortrait}

Many printers can print in either Portrait or Landscape orientation.
This returns true if the printer will print in portrait.

\Meth{int GetToFile()}
\Indextt{GetToFile}

\Meth{void SetToFile(int f)}

Printer output may be directed to a file rather than the printer.
This returns true if the user selected the option to send output
to a file.

\Meth{int GetUseColors()}
\Indextt{GetUseColors}

\Meth{void SetUseColors(int c)}
\Indextt{SetUseColors}

Printers can be either black and white, or color. This
returns true if the printer supports colors. You can
make a color printer print black and white by setting
this to false.

\Meth{int Setup(char* fn = 0)}
\Indextt{Setup}

This displays a modal dialog for the user to select desired
printer characteristics. If a filename is supplied, that
name will be used if the user selects print to file.
If \code{Setup} returns false, you should abandon the
print job. After you call \code{Setup}, you can then
call \code{vPrintDC::SetPrinter} to associate the printer
with the \code{vPrintDC}.

\subsection* {Example}

See \code{vPrintDC} for an example of using \code{vPrinter::Setup}.

%-----------------------------------------------------------------
\Class{vTextCanvasPane}
\Indextt{vTextCanvasPane}

A class for drawing text on a canvas. 

\subsection* {Synopsis}

\begin{description}
	\item [Header:] \code{<v/vtextcnv.h>}
	\item [Class name:] vTextCanvasPane
 	\item [Hierarchy:] vPane \rta vCanvasPane \rta vTextCanvasPane
\end{description}

\subsection* {Description}

This class provides a complete scrolling text window. You
can send text line by line to the window, and it will scroll the
text up the screen in response to linefeed characters. You can
also position the cursor, and selectively clear areas of the text
screen or display text at specific locations. This class handles
repainting the screen on \code{Redraw} events. In essence, the
\code{vTextCanvasPane} class provides the functionality of a
typical simple-minded text terminal.

\subsection* {New Methods}

%............................................................
\Meth{void ClearRow(const int row, const int col)}
\Indextt{ClearRow}

This clears to blanks row \code{row} of the screen from
column \code{col} to the end of the line.

%............................................................
\Meth{void ClearToEnd(const int row, const int col)}
\Indextt{ClearToEnd}

This clears to blanks from row \code{row} and column \code{col}
to the end of the screen.

%............................................................
\Meth{int GetCols()}
\Indextt{GetCols}

Returns number of columns in current text canvas.

%............................................................
\Meth{int GetRows()}
\Indextt{GetRows}

Returns number of rows in current text canvas.

%............................................................
\Meth{void GetRC(int\& row, int\& col)}
\Indextt{GetRC}

Returns in \code{row} and \code{col} the current row and
column of the text cursor.

%............................................................
\Meth{void GotoRC(const int row ,const int row)}
\Indextt{GotoRC}

Moves the text cursor to \code{row,col}.

%............................................................
\Meth{void DrawAttrText(const char* text, const ChrAttr attr)}
\Indextt{DrawAttrText}

Draws \code{text} starting at the current cursor location using
text attribute \code{attr}. For more details, see \code{vDC::DrawAttrText}.

%............................................................
\Meth{void DrawChar(const char chr, const ChrAttr attr)}
\Indextt{DrawChar}

Draws a single character \code{chr} at the current
cursor location using text attribute \code{attr}. See
\code{DrawAttrText} for more details.

%............................................................
\Meth{void DrawText(const char* text)}
\Indextt{DrawText}

Draws \code{text} starting at the current cursor location.
The newline character \code{'$\backslash$n'} will
cause the cursor to move to the beginning of the next line,
and the text to scroll if the cursor was on the last line.

%............................................................
\Meth{void HideTextCursor(void)}
\Indextt{HideTextCursor}

This method will hide the text cursor.

%............................................................
\Meth{void ShowTextCursor(void)}
\Indextt{ShowTextCursor}

This method will redisplay the text cursor at the current
row and column.

%............................................................
\Meth{void ScrollText(const int count)}
\Indextt{ScrollText}

This will scroll the text in the text canvas up or down by
\code{count} lines.  There will be \code{count} blank lines
created at the bottom or top of the screen.

%............................................................
\Meth{void ResizeText(const int rows, const int cols)}
\Indextt{ResizeText}

This method handles resize events. You will want to override
this to track the new number of rows and columns.

%............................................................
\Meth{void TextMouseDown(int row, int col, int button)}
\Indextt{TextMouseDown}

This is called when the user clicks the mouse button down.
It is called with the text row and column, and the button number.

%............................................................
\Meth{void TextMouseUp(int row, int col, int button)}
\Indextt{TextMouseUp}

This is called when the user releases the mouse button.
It is called with the text row and column, and the button number.

%............................................................
\Meth{void TextMouseMove(int row, int col, int button)}
\Indextt{TextMouseMove}

This is called when the mouse moves.
It is called with the text row and column, and the button number.


\subsection* {Derived Methods}	

%............................................................
\Meth{virtual void Clear()}
\Indextt{Clear}

This clears the text canvas and resets the row and column
to 0,0.

%............................................................
\Meth{void FontChanged(int)}
\Indextt{FontChanged}

This is called when the font of the canvas changes.
\code{FontChanged} calls \code{ResizeText}, so you probably
won't have to deal with this event.

%............................................................
\Meth{void Redraw(int x, int y, int width, int height)}
\Indextt{Redraw}

Called when the screen needs to be redrawn. Normally, you won't
have to override this class since the \code{vTextCanvasPane}
superclass will handle redrawing what is in the window. Instead,
you will usually just have to respond to the \code{FontChanged}
and \code{ResizeText} events when the contents of the canvas will
actually change.

\subsection* {Inherited Methods}	%....................

\Meth{virtual void HPage(int Shown, int Top)}
\Indextt{HPage}

\Meth{virtual void HScroll(int step)}
\Indextt{HScroll}

\Meth{virtual void SetFont(int vf)}
\Indextt{SetFont}

\Meth{virtual void SetHScroll(int Shown, int Top)}
\Indextt{SetHScroll}

\Meth{virtual void SetVScroll(int Shown, int Top)}
\Indextt{SetVScroll}

\Meth{virtual void VPage(int Shown, int Top)}
\Indextt{VPage}

\Meth{virtual void VScroll(int step)}
\Indextt{VScroll}

\subsection* {See Also}

vCanvasPane, vWindow


%***********************************************************************
%***********************************************************************
%***********************************************************************

\chapter {Standard V Values}

This chapter covers standard predefined values.

The classes and objects covered in this chapter include:

\begin{description}
	\item[Predefined ItemVals] A useful collection of predefined values.
Most are useful for defining dialogs, buttons, and menus.
\end{description}
%----------------------------------------------------------------------

\Class{Predefined ItemVals}
\index{predefined values}

A useful collection of predefined values. Most are useful for defining
dialogs, buttons, and menus.

\subsection* {Synopsis}

\begin{description}
	\item [Header:] \code{<v/v\_defs.h>}
\end{description}

\subsection* {Description}

When defining dialogs, menus, and command bars, you are required
to provide an id for each item.  There are many common operations
used in GUI designs, and \V\ provides various predefined values
for building your programs. The natural interpretation of most
of these values should be obvious, and the descriptions are kept
to a minimum. Most of the definitions describe the accepted practice
for menu or button items with the given title. While these \code{ItemVal}s
can be used anywhere, some have ``standard'' usage.

\subsection* {Control Values}

\Param{M\_About} Shows an informative message about current application.

\Param{M\_All} Select all.

\Param{M\_Cancel} Cancel. Usually used with a dialog. \V\ will
automatically reset dialog commands to their original state when
a \code{M\_Cancel} is selected from a \code{vDialog} descended
object.

\Param{M\_Clear} Used to clear a screen.

\Param{M\_Close} Used to close a file. The user is usually
prompted to save or ignore changes if any were made to the file. This
is usually not used to close a menu.

\Param{M\_Copy} Copy the highlighted text or item, and save into
the clipboard.

\Param{M\_Cut} Cut the highlighted text or item from the file, and
usually save into the clipboard.

\Param{M\_Delete} Delete the selected item or text -- usually
does not copy into the clipboard.

\Param{M\_Done} Done with operation.

\Param{M\_Edit} Typically a menu bar button to pulldown an edit menu.

\Param{M\_Exit} Exit from the program -- checking to see if files
need to be saved, of course.

\Param{M\_File} Typically a menu bar button to pulldown a file menu.

\Param{M\_Find} Find a pattern.

\Param{M\_FindAgain} Find pattern again.

\Param{M\_Font} Typically a menu bar button to pulldown a font menu.

\Param{M\_FontSelect} Select a font. (This is different from the
\code{M\_Font} value in that \code{M\_Font} is intended as a main
menu bar item, while this one is for a pulldown menu.

\Param{M\_Format} Typically a menu bar button to pulldown a format menu, which
allows the user to select formatting options.

\Param{M\_Help} Show help.

\Param{M\_Insert} Typically a menu bar button to pulldown an insert menu.

\Param{M\_Line} \code{M\_Line} is one of a few of these values
that gets special treatment by the system.  It is required for
defining line separators in menus.

\Param{M\_New} Used to create a new file.

\Param{M\_No} Answer No.

\Param{M\_None} Select none.

\Param{M\_OK} OK, accept operation or information. Causes return
from dialog.

\Param{M\_Open} Used to open an existing file.

\Param{M\_Options} Typically a menu bar button to pulldown an options menu.

\Param{M\_Paste} Paste the contents of the clipboard into the insertion
point of the current file or item.

\Param{M\_Preferences} Set preferences.

\Param{M\_Print} Print current file.

\Param{M\_PrintPreview} On screen preview how the current file would look
if printed.

\Param{M\_Replace} Replace pattern.

\Param{M\_Save} Used to save current file in its current name.

\Param{M\_SaveAs} Save current file under new name.

\Param{M\_Search} Typically a menu bar button to pulldown a search menu.

\Param{M\_SetDebug} Set debug stuff.

\Param{M\_Test} Typically a menu bar button to pulldown a test menu.

\Param{M\_Tools} Typically a menu bar button to pulldown a tools menu.

\Param{M\_UnDo} Undo the last action.

\Param{M\_View} Typically a menu bar button to pulldown a view menu, which
allows the user to select different views of the document.

\Param{M\_Window} Typically a menu bar button to pulldown a window menu, which
lets the user select different windows.

\Param{M\_Yes} Answer Yes.

\Class{Version Values}
\index{version values}

A useful collection of predefined values to determine the
version of V and the platform.

\subsection* {Synopsis}

\V defines several values useful for determining the revision of
\V, and the platform \V is compiled on.

\begin{description}
	\item [Header:] \code{<v/v\_defs.h>}
\end{description}

\subsection* {Version Values}

\Param{V\_VersMajor} The major version of \V, such as 1.

\Param{V\_VersMinor} The minor release of \V, such as 12.

\Param{V\_Version} A text string describing the version of
\V, such as \emph{V 1.12 - 8/4/96}.

\Param{V\_VersionX} Defined if the is the standard X version of \V.

\Param{V\_VersionMotif} Defined if the Motif version of \V.

\Param{V\_VersionWindows} Defined if the Windows version of \V.

\Param{V\_VersionWin95} Defined if the Windows 95 version of \V.

\Param{V\_VersionOS2} Defined for the OS2 version of \V.

%***********************************************************************
%***********************************************************************
%***********************************************************************

\chapter {Utilities}
\index{utilities}

This chapter covers \V\ utility classes and functions. Whenever
possible, any corresponding native utility has been used to
implement these classes. For example, you will get the standard
file interface dialog for Windows.

Since the Athena implementation has no corresponding native
utilities, the \V\ utility classes have been implemented for
Athena using standard \V\ classes as much as possible. Thus, the
source code for the Athena version of these utilities provides an
excellent example of \V\ for study.

The classes and objects covered in this chapter include:

\begin{description}
	\item[vDebugDialog] Utility class to access debugging messages.
	\item[vFileSelect] A utility class to select or set a file name.
	\item[vFontSelect] A utility class to select or set a font object.
	\item[vNoticeDialog] A utility class to display a message.
	\item[vReplyDialog] A utility class to get a text reply from the user.
	\item[vTimer] A class for getting timer events.
	\item[vYNReplyDialog] A utility class to display a message, and get a Yes or No answer.
	\item[Utility Functions] Several useful functions.
\end{description}

%-------------------------------------------------------------------
\Class{vDebugDialog}
\Indextt{vDebugDialog}

Utility class to access debugging messages.

\subsection* {Synopsis}

\begin{description}
	\item [Header:] \code{<v/vdebug.h>}
	\item [Class name:] vDebugDialog
 	\item [Hierarchy:] vModalDialog \rta vDebugDialog
\end{description}

\subsection* {Description}

\V\ provides built in debugging features. Most of the \V\ classes
contain debugging messages that are displayed on \code{stderr} or a
special debugging information window. For Unix systems, \code{stderr}
is usually the xterm window used to launch the \V\ application.
For other environments, the debugging window is system dependent.

Several categories of debugging messages have been defined by \V,
and display of messages from different categories is controlled
by the \code{vDebugDialog} class.

\V\ provides several macros that can be used to insert debugging
messages into your code. These are of the form \code{SysDebugN} for
system code, and \code{UserDebugN} for your code. Display of
these messages is controlled by the \code{vDEBUG} symbol, and the
settings of the \code{vDebugDialog} class.

You define an error message using a \code{UserDebug} macro.
Your message is a format string using the conventions of \code{sprintf}.
You can have none to three values by using the corresponding
\code{UserDebug} through \code{UserDebug3} macros. Each macro
takes a debug type, a message, and any required values for the
message format string. For example,
\code{UserDebug(Misc,"myClass: \%d$\backslash$n", val)}
will print the message ``myClass:~xx'' when it is executed and the
\code{Misc} debug message type is enabled.

If \code{vDEBUG} is \emph{not} defined, your debugging messages will
be null macros, and not occupy any code space.  If \code{vDEBUG} is
defined, then your messages will be conditionally displayed depending
on their type.

By default, V starts with the \code{System} category \code{BadVals}
on, and all three \code{User} categories on. Unix versions of \V\
support a command line option that allows you to enable each
option using
the \code{-vDebug} command line switch. You include the switch
\code{-vDebug} on the command line, followed by a single argument
value made up of letters corresponding to the various debugging
categories. If \code{-vDebug} is specified, all debugging categories
except those specified in the value are turned off. The value
for each category is listed in its header. For example, using
the switch \code{-vDebug SUCDm} would enable debugging messages
for both \code{System} and \code{User} constructors and destructors,
as well as \code{System} mouse events.
Note that the values are case sensitive.


\subsection*{Debugging Categories}

Each of the following debug categories can be set or unset using
the \code{vDebugDialog} class. These category names are to be
used as the first argument to the \code{UserDebug} macro.

\paragraph*{System (-vDebug S)}

These are the messages defined using the \code{SysDebug} macro.
These messages can sometimes be useful to determine if you are
using the classes properly. The constructor, destructor, and
command events are often the most useful system debug messages.
Turning this off will disable all system messages.

\paragraph*{User (-vDebug U)}

These are the messages defined using the \code{UserDebug} macros. Turning
this off will disable all user messages, while turning it on enables those
user messages that have been enabled.

\paragraph*{CmdEvents (-vDebug c)}

This category corresponds to command events, which include menu
picks and dialog command actions.

\paragraph*{MouseEvents (-vDebug m)}

This category corresponds to mouse events, such as a button click
or a move.

\paragraph*{WindowEvents (-vDebug w)}

This category corresponds to window events, such as a resize or redraw.

\paragraph*{Build (-vDebug b)}

This category corresponds to actions taken to build a window, such
as adding commands to a dialog.

\paragraph*{Misc (-vDebug o)}

This is a catch all category used for miscellaneous system messages.
The \code{o} vDebug stands for other.
You should probably use a UserAppN category for your miscellaneous messages.

\paragraph*{Text (-vDebug t)}

These messages are primarily used by the \code{vTextCanvasPane} class,
and are useful for debugging text display.

\paragraph*{BadVals (-vDebug v)}

These messages are generated when a bad parameter or illegal value
is detected. These can be most useful.

\paragraph*{Constructor (-vDebug C)}

These messages are displayed whenever a constructor for an
object is called. These messages can be very useful for tracking
object creation bugs. You should try to have
\code{UserDebug(Constructor,"X::X constructor")} messages
for all of your constructors, and a corresponding Destructor message.

\paragraph*{Destructor (-vDebug D)}

Messages from an object destructor.

\paragraph*{UserApp1, UserApp2, UserApp3 (-vDebug 123)}

These are provided to allow you up to three categories of your
own debugging messages.

\subsection* {Example}

To use the \V\ debugging facilities, it is usually easiest to
add a Debug command to a menu item -- controlled by the \code{vDEBUG}
symbol.  Then add calls to \code{UserDebug} as needed in your code.
This example shows how to define a Debug menu item, and then invoke
the \code{vDebugDialog} to control debugging settings.

\footnotesize
\begin{verbatim}

#include <v/vdebug.h>

    vMenu FileMenu[] =
      {
        ...
#ifdef vDEBUG
        {"-", M_Line, notSens,notChk,noKeyLbl,noKey,noSub},
        {"Debug", M_SetDebug,isSens,notChk,noKeyLbl,noKey,noSub},
#endif
        ...
      };

    ...
    case M_SetDebug:
     {
        vDebugDialog debug(this);    // instantiate
        UserDebug(Misc,"About to show Debug dialog.\n");
        debug.SetDebug();            // show the dialog
        break;
     }
    ...

\end{verbatim}
\normalfont\normalsize

%--------------------------------------------------------------------

\Class{vFileSelect}
\Indextt{vFileSelect}

A utility class to select or set a file name.

\subsection* {Synopsis}

\begin{description}
        \item [Header:] \code{<v/vfilesel.h>}
        \item [Class name:] vFileSelect
        \item [Hierarchy:] vModalDialog \rta vFileSelect
\end{description}

\subsection* {Description}

This utility class provides a dialog interface for selecting
filenames. It can be used either to select an input file name,
or verify or change an output file name. This utility does not
open or alter files -- it simply constructs a legal file name for
use in opening a file.

\subsection* {Methods}

%............................................................
\Meth{vFileSelect(vBaseWindow* win)}
\Indextt{vFileSelect}
\Meth{vFileSelect(vApp* app)}

The \code{vFileSelect} constructor requires a pointer to a
\code{vBaseWindow}, which includes all \V\ windows and dialogs,
or a pointer to the \code{vApp} object.
You will usually pass the \code{this} to the constructor.

%............................................................
\Meth{int FileSelect(const char* prompt, char* filename, const
int maxLen, char** filterList, int\& filterIndex)}
\Indextt{FileSelect}

\Meth{int FileSelectSave(const char* prompt, char* filename, const int
maxLen, char** filterList, int\& filterIndex)}
\Indextt{FileSelectSave}

You provide a \code{prompt} for the user, such as ``Open File.'' The
user then uses the dialog to select or set a file name. \code{FileSelect}
returns \code{True} if the user picked the OK button, and \code{False}
if they used the Cancel button.

The filename will be filled in to the \code{filename} buffer of
maximum length \code{maxLen}. The full path of the file will be
included with the file name.

You can also provide a list of filter patterns to filter file
extensions. If you don't provide a filter list, the default filter
of ``*'' will be used. Each item in the filter list can include a list
of file extensions separated by blanks. You can provide several
filtering options. The first filter in the list will be the default.
Only leading ``*'' wild cards are supported.

The \code{filterIndex} reference parameter is used to track which
filter the user selected. After \code{FileSelect} returns, \code{filterIndex}
will be set to the index of the filter list that the user last
selected. For the best interface, you should remember this value for
the next time you call \code{FileSelect} with the same filter list so
that the user selected filter will be preserved.

You should use \code{FileSelect} to open a new or existing file. If
the user is being asked to save a file (usually after picking a
\emph{Save As} menu choice), use the \code{FileSelectSave} method. On
some platforms, there will be no difference between these two
methods (X, for example). On other platforms (Windows, for example),
different underlying system provided file dialogs are used. To your
program, there will be no difference in functionality.

\subsection*{Example}

The following is a simple example of using \code{vFileSelect}.

\vspace{.1in}
\small
\begin{rawhtml}
<IMG BORDER=0 ALIGN=BOTTOM ALT="" SRC="../fig/filesel.gif">
\end{rawhtml}
\begin{latexonly}
\setlength{\unitlength}{0.012500in}%
\begin{picture}(295,220)(30,600)
\thicklines
\put( 45,610){\framebox(60,130){}}
\put(110,610){\framebox(10,130){}}
\put( 40,605){\framebox(85,140){}}
\put( 50,725){\makebox(0,0)[lb]{\smash{\SetFigFont{12}{14.4}{rm}../}}}
\put( 50,707){\makebox(0,0)[lb]{\smash{\SetFigFont{12}{14.4}{rm}bin/}}}
\put( 50,689){\makebox(0,0)[lb]{\smash{\SetFigFont{12}{14.4}{rm}doc/}}}
\put( 50,671){\makebox(0,0)[lb]{\smash{\SetFigFont{12}{14.4}{rm}examp/}}}
\put( 50,653){\makebox(0,0)[lb]{\smash{\SetFigFont{12}{14.4}{rm}lib/}}}
\put( 50,635){\makebox(0,0)[lb]{\smash{\SetFigFont{12}{14.4}{rm}src/}}}
\put( 50,617){\makebox(0,0)[lb]{\smash{\SetFigFont{12}{14.4}{rm}makefile}}}
\put(110,700){\line( 1, 0){ 10}}
\put(115,740){\line( 0,-1){ 40}}
\put(110,735){\line( 1, 0){ 10}}
\put(110,730){\line( 1, 0){ 10}}
\put(110,725){\line( 1, 0){ 10}}
\put(110,720){\line( 1, 0){ 10}}
\put(110,715){\line( 1, 0){ 10}}
\put(110,710){\line( 1, 0){ 10}}
\put(110,705){\line( 1, 0){ 10}}
\put(120,705){\makebox(0.4444,0.6667){\SetFigFont{10}{12}{rm}.}}
\put(140,700){\framebox(120,25){}}
\put(145,705){\framebox(90,15){}}
\put(240,705){\framebox(15,15){}}
\multiput(247,706)(0.40000,0.40000){21}{\makebox(0.4444,0.6667){\SetFigFont{7}{8.4}{rm}.}}
\put(255,714){\line(-1, 0){ 15}}
\multiput(240,714)(0.40000,-0.40000){21}{\makebox(0.4444,0.6667){\SetFigFont{7}{8.4}{rm}.}}
\put(140,730){\makebox(0,0)[lb]{\smash{\SetFigFont{12}{14.4}{rm}Filter:}}}
\put(150,705){\makebox(0,0)[lb]{\smash{\SetFigFont{12}{14.4}{rm}*}}}
\put( 70,775){\framebox(240,20){}}
\put(180,605){\framebox(50,20){}}
\put(185,625){\line( 0,-1){ 20}}
\put(225,625){\line( 0,-1){ 20}}
\put(225,605){\line( 0, 1){  5}}
\put(180,630){\framebox(65,20){}}
\put(180,655){\framebox(65,20){}}
\put( 30,600){\framebox(295,220){}}
\put( 40,800){\makebox(0,0)[lb]{\smash{\SetFigFont{12}{14.4}{rm}Open file}}}
\put( 40,780){\makebox(0,0)[lb]{\smash{\SetFigFont{12}{14.4}{rm}File:}}}
\put( 45,755){\makebox(0,0)[lb]{\smash{\SetFigFont{12}{14.4}{rm}Dir:    /home/bruce/v}}}
\put(195,610){\makebox(0,0)[lb]{\smash{\SetFigFont{12}{14.4}{rm}OK}}}
\put(195,635){\makebox(0,0)[lb]{\smash{\SetFigFont{12}{14.4}{rm}Cancel}}}
\put(195,660){\makebox(0,0)[lb]{\smash{\SetFigFont{12}{14.4}{rm}Select}}}
\end{picture}

\end{latexonly}
\normalfont\normalsize

\footnotesize
\begin{verbatim}
    static char* filter[] =     // define a filter list
      {
        "*",                    // all files
        "*.txt",                // .txt files
        "*.c *.cpp *.h",        // C sources
        0
      };
    static int filterIndex = 0;    // to track filter picked
    char name[100];

    vFileSelect fsel(this);     // instantiate

    int oans = fsel.FileSelect("Open file",name,99,filter,filterIndex);

    vNoticeDialog fsnote(this); // make an instance

    if (oans && *name)
        (void)fsnote.Notice(name);
    else
        (void)fsnote.Notice("No file name input.");
\end{verbatim}
\normalfont\normalsize

%--------------------------------------------------------------------

\Class{vFontSelect}
\Indextt{vFontSelect}

A utility class to select or set a file name.

\subsection* {Synopsis}

\begin{description}
        \item [Header:] \code{<v/vfontsel.h>}
        \item [Class name:] vFontSelect
        \item [Hierarchy:] vModalDialog \rta vFontSelect
\end{description}

\subsection* {Description}

This class provides the \code{FontSelect} method to set
the font being used. This class provides a platform
independent way to change fonts. Depending on the platform,
the user will be able to select many or most of the fonts
available on the platform. On Windows, for example, the standard
Windows font selection dialog is be used. On X, a relatively
full set of fonts are available.

\subsection* {Methods}

\Meth{vFontSelect(vBaseWindow* win)}
\Indextt{vFontSelect}
\Meth{vFontSelect(vApp* app)}

The \code{vFontSelect} constructor requires a pointer to a
\code{vBaseWindow}, which includes all \V\ windows and dialogs,
or a pointer to the \code{vApp} object.
You will usually pass the \code{this} to the constructor.

\Meth{int FontSelect(vFont\& font, const char* msg = "Select Font")}
\Indextt{FontSelect}

This method displays a dialog that lets the user select font
characteristics. If possible, the native font selection dialog
will be used (e.g., Windows). The font dialog will display the
current characteristics of the \code{font} object, and change
them upon successful return. A \code{false} return means the user
selected Cancel, while a \code{true} return means the user
finished the selection with an OK.

%--------------------------------------------------------------------

\Class{vNoticeDialog}
\Indextt{vNoticeDialog}

A utility class to display a message.

\subsection* {Synopsis}

\begin{description}
        \item [Header:] \code{<v/vnotice.h>}
        \item [Class name:] vNoticeDialog
        \item [Hierarchy:] vModalDialog \rta vNoticeDialog
\end{description}

\subsection* {Description}

This simple utility class can be used to display a simple
message to the user. The utility displays the message, and then
waits for the user to enter to press OK.

\subsection* {New Methods}

%............................................................
\Meth{vNoticeDialog(vBaseWindow* win)}
\Indextt{vNoticeDialog}
\Meth{vNoticeDialog(vApp* app)}

The \code{vNoticeDialog} constructor requires a pointer to a
\code{vBaseWindow}, which includes all \V\ windows and dialogs,
or a pointer to the \code{vApp} object.
You will usually pass the \code{this} to the constructor.

%............................................................
\Meth{void Notice(const char* prompt)}
\Indextt{Notice}

You provide a \code{prompt} for the user. If the message
contains '$backslash$n' newlines, it will be shown on multiple
lines.

\subsection*{Example}

The following is a simple example of using \code{vNoticeDialog}.

\vspace{.1in}

\small
\begin{rawhtml}
<IMG BORDER=0 ALIGN=BOTTOM ALT="" SRC="../fig/notice.gif">
\end{rawhtml}
\begin{latexonly}
\setlength{\unitlength}{0.012500in}%
\begin{picture}(150,60)(30,750)
\thicklines
\put( 40,755){\framebox(35,20){}}
\put( 45,775){\line( 0,-1){ 20}}
\put( 70,775){\line( 0,-1){ 20}}
\put( 50,760){\makebox(0,0)[lb]{\smash{\SetFigFont{10}{12.0}{rm}OK}}}
\put( 50,790){\circle{20}}
\put( 30,750){\framebox(150,60){}}
\put( 50,785){\makebox(0,0)[lb]{\smash{\SetFigFont{12}{14.4}{rm}!}}}
\put( 70,785){\makebox(0,0)[lb]{\smash{\SetFigFont{12}{14.4}{rm}This is a notice.}}}
\end{picture}

\end{latexonly}
\normalfont\normalsize

\footnotesize
\begin{verbatim}
    #include <v/vnotice.h>
    ...
    vNoticeDialog note(this);   // instantiate a notice

    (void)note.Notice("This is a notice.");
\end{verbatim}
\normalfont\normalsize

%--------------------------------------------------------------------

\Class{vReplyDialog}
\Indextt{vReplyDialog}

A utility class to get a text reply from the user.

\subsection* {Synopsis}

\begin{description}
        \item [Header:] \code{<v/vreply.h>}
        \item [Class name:] vReplyDialog
        \item [Hierarchy:] vModalDialog \rta vReplyDialog
\end{description}

\subsection* {Description}

This simple utility class can be used to obtain a text reply from
the user. The utility displays a message, and then waits for the
user to enter a reply into the reply field. The user completes the
operation by pressing OK or Cancel.

\subsection* {New Methods}

%............................................................
\Meth{vReplyDialog(vBaseWindow* win)}
\Indextt{vReplyDialog}
\Meth{vReplyDialog(vApp* app)}

The \code{vReplyDialog} constructor requires a pointer to a
\code{vBaseWindow}, which includes all \V\ windows and dialogs,
or a pointer to the \code{vApp} object.
You will usually pass the \code{this} to the constructor.

%............................................................
\Meth{int Reply(const char* prompt, char* reply, const int maxLen)}
\Indextt{Reply}

You provide a \code{prompt} for the user. The text the user enters
will be returned to the buffer \code{reply} of maximum length \code{maxLen}.
\code{Reply} will return the value \code{M\_OK} or \code{M\_Cancel}.

\subsection*{Example}

The following is a simple example of using \code{vReplyDialog}.

\vspace{.1in}
\small
\begin{rawhtml}
<IMG BORDER=0 ALIGN=BOTTOM ALT="" SRC="../fig/reply.gif">
\end{rawhtml}
\begin{latexonly}
\setlength{\unitlength}{0.012500in}%
\begin{picture}(200,80)(30,735)
\thicklines
\put( 40,765){\framebox(180,20){}}
\put( 45,770){\line( 1, 0){  5}}
\put(105,740){\framebox(35,20){}}
\put(110,760){\line( 0,-1){ 20}}
\put(135,760){\line( 0,-1){ 20}}
\put(115,745){\makebox(0,0)[lb]{\smash{\SetFigFont{10}{12.0}{rm}OK}}}
\put( 40,740){\framebox(55,20){}}
\put( 50,745){\makebox(0,0)[lb]{\smash{\SetFigFont{12}{14.4}{rm}Cancel}}}
\put( 50,800){\circle{20}}
\put( 30,735){\framebox(200,80){}}
\put( 70,795){\makebox(0,0)[lb]{\smash{\SetFigFont{12}{14.4}{rm}Please enter some text.}}}
\put( 50,795){\makebox(0,0)[lb]{\smash{\SetFigFont{12}{14.4}{rm}?}}}
\end{picture}

\end{latexonly}
\normalfont\normalsize

\footnotesize
\begin{verbatim}
    #include <v/vreply.h>
    ...
    vReplyDialog rp(this);      // instantiate
    char r[100];                // a buffer for reply

    (void)rp.Reply("Please enter some text.",r,99);

    vNoticeDialog note(this);   // instantiate a notice

    if (*r)
        (void)note.Notice(r);
    else
        (void)note.Notice("No text input.");
\end{verbatim}
\normalfont\normalsize

%------------------------------------------------------------------------

\Class{vTimer}
\Indextt{vTimer}
\index{timer}

A class for getting timer events.

\subsection* {Synopsis}

\begin{description}
	\item [Header:] \code{<v/vtimer.h>}
	\item [Class name:] vTimer
 	\item [Hierarchy:] vTimer
\end{description}

\subsection* {Description}

This is a utility class that allows you to get events driven
by the system timer. The accuracy and resolution of timers on
various systems varies, so this should be used only to get
events on a more or less regular basis. Use the C library \code{time}
routines to get real clock time.

\subsection* {New Methods}

%............................................................
\Meth{vTimer}
\Indextt{vTimer}

This constructs a timer object. The timer doesn't run until
you start it with \code{TimerSet}. To make a timer useful, you
can override the constructor to add a pointer to a window, and
then use that pointer from within your \code{TimerTick} method
to do something in that window: \code{myTimer(vWindow* useWindow)}.

%............................................................
\Meth{int TimerSet(long interval)}
\Indextt{TimerSet}

This starts the timer going. The timer will call your overridden
\code{TimerTick} method approximately every \code{interval}
milliseconds until you stop the timer. Most systems don't support
an unlimited number of timers, and \code{TimerSet} will return 0
if it couldn't get a system timer.

%............................................................
\Meth{void TimerStop()}
\Indextt{TimerStop}

Calling this stops the timer, but does not destruct it.

%............................................................
\Meth{void TimerTick()}
\Indextt{TimerTick}

This method is called by the system every interval milliseconds
(more or less). The way to use the timer is to derive your own
class, and override the \code{TimerTick} method.  Your method
will be called according to the interval set for the timer. Note
that you can't count on the accuracy of the timer interval.

%--------------------------------------------------------------------
\Class{vYNReplyDialog}
\Indextt{vYNReplyDialog}

A utility class to display a message, and get a Yes or No answer.

\subsection* {Synopsis}

\begin{description}
        \item [Header:] \code{<v/vynreply.h>}
        \item [Class name:] vYNReplyDialog
        \item [Hierarchy:] vModalDialog \rta vYNReplyDialog
\end{description}

\subsection* {Description}

This simple utility class can be used to display a simple
message to the user. The utility displays the message, and then
waits for the user to enter to press Yes, No, or Cancel.

\subsection* {New Methods}

%............................................................
\Meth{vYNReplyDialog(vBaseWindow* win)}
\Indextt{vYNReplyDialog}
\Meth{vYNReplyDialog(vApp* app)}

The \code{vYNReplyDialog} constructor requires a pointer to a
\code{vBaseWindow}, which includes all \V\ windows and dialogs,
or a pointer to the \code{vApp} object.
You will usually pass the \code{this} to the constructor.

%............................................................
\Meth{int AskYN(const char* prompt)}
\Indextt{AskYN}

You provide a \code{prompt} for the user. The user will then press
the Yes, No, or Cancel buttons. \code{AskYN} returns a 1 if
the user selected Yes, a 0 if they selected No, and a -1 if
they selected Cancel.

\subsection*{Example}

The following is a simple example of using \code{vYNReplyDialog}.

\vspace{.1in}

\small
\begin{rawhtml}
<IMG BORDER=0 ALIGN=BOTTOM ALT="" SRC="../fig/ynreply.gif">
\end{rawhtml}
\begin{latexonly}
\setlength{\unitlength}{0.012500in}%
\begin{picture}(170,60)(30,755)
\thicklines
\put(135,760){\framebox(55,20){}}
\put(145,765){\makebox(0,0)[lb]{\smash{\SetFigFont{12}{14.4}{rm}Cancel}}}
\put( 50,800){\circle{20}}
\put( 40,760){\framebox(40,20){}}
\put( 45,780){\line( 0,-1){ 20}}
\put( 75,780){\line( 0,-1){ 20}}
\put( 90,760){\framebox(35,20){}}
\put( 30,755){\framebox(170,60){}}
\put( 50,795){\makebox(0,0)[lb]{\smash{\SetFigFont{12}{14.4}{rm}?}}}
\put( 70,795){\makebox(0,0)[lb]{\smash{\SetFigFont{12}{14.4}{rm}Exit. Are you sure?}}}
\put( 50,765){\makebox(0,0)[lb]{\smash{\SetFigFont{12}{14.4}{rm}Yes}}}
\put(100,765){\makebox(0,0)[lb]{\smash{\SetFigFont{12}{14.4}{rm}No}}}
\end{picture}

\end{latexonly}
\normalfont\normalsize

\footnotesize
\begin{verbatim}
    #include <v/vynreply.h>
    ...
    vYNReplyDialog ynd(this);   // instantiate a notice

    int ans = ynd.AskYN("Exit. Are you sure?);
    if (ans == 1)
      exit(0);
\end{verbatim}
\normalfont\normalsize

%------------------------------------------------------------------
\Class{Utility Functions}
\index{utility functions}

Several useful utility functions.

\subsection* {Synopsis}

\begin{description}
	\item [Header:] \code{<v/vutil.h>}
\end{description}

\subsection* {Description}

\V\ provides several utility functions that can often help with
software portability (and can just be useful). These are free
subprograms -- not a member of any specific class.

\Meth{void ByteToStr(unsigned char b, char* str)}
\Indextt{ByteToStr}

This will convert the unsigned char in \code{b} to a \emph{Hex} character
string in \code{str}. You need to make \code{str} big enough to
hold the string.

\Meth{void IntToStr(int intg, char* str)}
\Indextt{IntToStr}

This will convert the integer in \code{intg} to a character
string in \code{str}. You need to make \code{str} big enough
to hold the string.

\Meth{void LongToStr(long intg, char* str)}
\Indextt{LongToStr}

This will convert the long integer in \code{intg} to a character
string in \code{str}. You need to make \code{str} big enough
to hold the string.

\Meth{long StrToLong(char* str)}
\Indextt{StrToLong}

This will convert the character string in \code{str} into a long
integer. You can cast to get ints.

\Meth{void vGetLocalTime(char* tm)}
\Indextt{vGetLocalTime}
\index{time}

This will return a string representation of the current local time
to the string \code{tm}. The format will be ``HH:MM:SS AM''. If you
need a different format, you will need to use the C functions
\code{time}, \code{localtime}, and \code{strftime} directly.

\Meth{void vGetLocalDate(char* dt)}
\Indextt{vGetLocalDate}
\index{date}

This will return a string representation of the current local
date to the string \code{dt}. The format will be ``MM/DD/YY''.
If you need a different format, you will need to use the C
functions \code{time}, \code{localtime}, and \code{strftime}
directly.

%------------------------------------------------------------------
\Class{Utility Programs}
\index{utility programs}
\index{icons} \index{bitmap}

\subsection*{bmp2vbm}

The utility \code{bmp2vbm} converts a Window or OS/2 \code{.bmp}
format bitmap file into a \code{.vbm} \V icon bitmap format file.
The \code{.vbm} file is then used with a \code{vIcon} object
definition. The \code{bmp2vbm} utility will not convert all
\code{.bmp} files. Specifically, it can't handle old format
\code{.bmp} files, nor can it handle compressed \code{.bmp} files.

Windows has many tools to generate \code{.bmp} files. For X,
the widely available tool \code{xv} can generate \code{.bmp}
files from various source formats.

\code{Bmp2vbm} is a command line tool - run it from a Unix prompt,
or from an \code{MSDOS} box on Windows. The command line format is:
\code{bmp2vbm inputname outputname iconname}. You should specify only
the base file names: \code{bmp2vbm} will automatically supply
the \code{.bmp} and \code{.vbm} extension. The \code{iconname}
specifies the name used to generate the date (e.g., \code{iconname\_bits}).

%------------------------------------------------------------------
\Class{More Goodies}
\index{goodies}
\index{icons}

The directory \code{v/icons} includes over 30 different monochrome
icons in \code{.vbm} format suitable for building command pane
tool bars. Most of these icons were derived from various
Windows sources, and I would encourage their use for the
standard functions they define. Some of these include
creating a new file (new.vbm), opening an existing file (open.vbm),
cut, copy, and paste (*.vbm), printing (print.vbm), and so on.

There is a demo program in the \code{v/icons} directory that
can be compiled and used to see what all the icons look like.
All the icons are 16 by 16 bits, and
will match standard buttons in height on Windows. The height of 
standard buttons on X depends on the default system font.

As usual, contributions of other \V icons is more than welcome.
I hope to build up the icons directory to several hundred icons.


%***********************************************************************
%***********************************************************************
%***********************************************************************

\appendix


\chapter{Tutorial C++ Source}

%----------------------------------------------------------------------
The following is a complete tutorial \V\ application. The source for
this tutorial is available int the directory \~{}/v/tutor.

\section{The Application}
\index{tutorial source code}

\subsection*{tutapp.cpp}

%\input{../tutor/tutapp.cpp}

\footnotesize
\begin{verbatim}
//========================================================================
//  tutapp.cpp:     Source file for tutorial V application
//
//      Copyright 1995, Bruce E. Wampler. All rights reserved.
//========================================================================
//
// This file is used to define the main application object. There
// will be exactly one instance of the application object. You will
// usually derive you own app class from the vApp class. This file
// defines a sample tutApp class. The usual purpose of tutApp is to
// start the initial window, and act as a central controller for
// your application. Rather than reading this file sequentially,
// you should skip to the end and read the comments surrounding the
// AppMain function.


//  Files required for tutorial minimal application:
//      tutapp.h:       Header for the min app
//      tutapp.cpp:     Source code for min app
//      tcmdwin.h:      Header code for sample command window
//      tcmdwin.cpp:    Source code for sample command window
//      tdialog.h:      Header for sample modeless dialog
//      tdialog.cpp:    Source for sample modeless dialog
//      tmodal.h:       Header for sample modal dialog
//      tmodal.cpp:     Source for sample modal dialog
//

// First #include header files we need to use.

#include "tutapp.h"     // our header file

//=========================>>> tutApp::NewAppWin <<<======================
  vWindow* tutApp::NewAppWin(vWindow* win, char* name, int w, int h,
    vAppWinInfo* winInfo)
  {
    // This version of NewAppWin is provided with the information
    // required to name and size a window.
    // 
    // Typically, this method would get a file name or other information
    // needed to setup the AppWinInfo class  specifically for the
    // application. Thus, each open window usually represents a view of
    // a file or data object.

    vWindow* thisWin = win;             // local copy to use
    vAppWinInfo* awinfo = winInfo;
    char *myname = name;                // local copy

    if (!*name)
        myname = "Example";            // make up a name
        
    // The UserDebug macros are useful for tracking what is going on.
    // This shows we're building a window.

    UserDebug1(Build,"tutApp::NewAppWin(%s)\n",myname);

    // You may instantiate an instance of the window outside of
    // NewAppWin, or allow NewAppWin to create the instance.

    if (!thisWin)       // Didn't provide a window, so create one.
        thisWin = new tCmdWindow(myname, w, h);

    // The vAppWinInfo class is meant to serve as a database used by the
    // tutApp controller. If you use this feature, you will probably
    // derive your own myAppWinInfo class from vAppWinInfo. The instance
    // of vAppWinInfo created here will be automatically deleted when
    // this window instance is closed through CloseAppWin.

    if (!awinfo)        // Did caller provide an appinfo?
        awinfo = new vAppWinInfo(myname);

    // After you have created an instance of the window and an instance
    // of the AppWinInfo, you MUST call the base vApp::NewAppWin method.
    // You won't need to explicitly keep track of the pointer to
    // each new window -- unless it has to interact with other windows.
    // If that is the case, then you can use your derived vAppWinInfo
    // to coordinate the interaction.

    return vApp::NewAppWin(thisWin,name,w,h,awinfo);
  }

//===========================>>> tutApp::Exit <<<=========================
  void tutApp::Exit(void)
  {
    // This can be called to close all windows. If the app needs to do
    // something special, it can. Otherwise, it can call the general
    // vApp::Exit method, which will perform appropriate calls the the
    // specialized tutApp::CloseAppWin.

    UserDebug(Build,"tutApp::Exit()\n");

    vApp::Exit();       // easy default behavior
  }

//======================>>> tutApp::CloseAppWin <<<=======================
  void tutApp::CloseAppWin(vWindow* win)
  {
    // This will be called BEFORE a window has been unregistered or
    // closed. The app can do whatever it needs to to close down the
    // data associated with this window. (It is invoked explicitly by
    // you in response to a Close menu pick, for example, or when the
    // user clicks the close button. It can also be called by vApp::Exit().
    // After this method cleans up, it can then call the superclass
    // vApp::CloseAppWin to unregister and close this window. Note that
    // the win gives a handle that can be used with vApp::getAppWinInfo
    // to retrieve the AppWinInfo class.

    UserDebug(Build,"tutApp::CloseAppWin()\n");

    // Code to handle close of window (such as saving/closing
    // a file) would go here...

    vApp::CloseAppWin(win);   // Unregister and close the window.
  }

//=====================>>> tutApp::AppCommand <<<=========================
  void tutApp::AppCommand(vWindow* win, ItemVal id, ItemVal val,
    CmdType cType)
  {
    // Any commands not processed by the window WindowCommand
    // method will be passed to here for default treatment.

    UserDebug1(Build,"tutApp::AppCmd(ID: %d)\n",id);
    vApp::AppCommand(win, id, val, cType);
  }

//=======================>>> tutApp::KeyIn <<<============================
  void tutApp::KeyIn(vWindow* win, vKey key, unsigned int shift)
  {
    // Any key strokes not processed by the window will be passed
    // along to here for default treatment.

    vApp::KeyIn(win, key, shift);
  }

//========================================================================
// Remember that any static references to an object are constructed by
// the C++ startup code before main or any other functions are called.
// Thus, the constructor for tutApp (and thus vApp) is invoked before
// anything else happens. This enables V to perform whatever
// initializations are required by the host GUI system - and frees you
// from having to worry about the typical gory details. All this means
// that EVERY V application needs a static instance of the tutApp to
// get things rolling. Note that the global variable theApp is set to
// point to this instance, and is the easiest way to access the vApp
// and tutApp methods (e.g., theApp->Exit()).
//========================================================================

  static tutApp tut_App("TutorApp");  // The single instance of the app

//===========================>>> AppMain <<<==============================
  int AppMain(int argc, char** argv)
  {
    // The V framework defines the instance of main. After some
    // processing of command line arguments, AppMain is called with
    // cleaned up command line arguments. Note that at this time, no
    // windows have been defined. Normally, AppMain is the place to
    // start up the first window. You can perform any initialization you
    // need to do here.

    (void) theApp->NewAppWin(0, "Tutorial V Example", 350, 100, 0);

    // At this point, the window is up, and all events are being
    // routed through its methods.

    // We MUST return 0 if the status is OK at this point.
    return 0;
  }
\end{verbatim}
\normalfont\normalsize

\subsection*{tutapp.h}

%\input{../tutor/tutapp.h}
\footnotesize
\begin{verbatim}
//========================================================================
//      tutapp.h:  Header file for tutorial V application
//
//      Copyright 1995, Bruce E. Wampler. All rights reserved.
//========================================================================

#ifndef TUTAPP_H                // Standard technique for avoiding
#define TUTAPP_H                // problems with multiple #includes

#ifdef vDEBUG
#include <v/vdebug.h>
#endif

#include <v/vapp.h>     // We are derived from vApp
#include <v/vawinfo.h>  // Need for app info

#include "tcmdwin.h"            // we use our tCmdWindow class

    class tutApp : public vApp
      {
        friend int AppMain(int, char**); // allow AppMain access

      public:           //---------------------------------------- public

        tutApp(char* name) : vApp(name) {}      // just call vApp
        virtual ~tutApp() {}

        // Routines from vApp that are normally overridden

        virtual vWindow* NewAppWin(vWindow* win, char* name, int w, int h,
           vAppWinInfo* winInfo);
        virtual void Exit(void);
        virtual void CloseAppWin(vWindow* win);
        virtual void AppCommand(vWindow* win, ItemVal id,
                ItemVal val, CmdType cType);
        virtual void KeyIn(vWindow* win, vKey key,
            unsigned int shift);

        // New routines for this particular app go here

      protected:        //------------------------------------- protected

      private:          //--------------------------------------- private

      };
#endif

\end{verbatim}
\normalfont\normalsize

%----------------------------------------------------------------------
\section{The Command Window}

\subsection*{tcmdwin.cpp}

%\input{../tutor/tcmdwin.cpp}
\footnotesize
\begin{verbatim}
//========================================================================
//  tcmdwin.cpp:     Source file for tutorial cmdwin class
//
//      Copyright 1995, Bruce E. Wampler. All rights reserved.
//========================================================================

// This file contains the source code for a typical command window
// derived from the vCmdWindow class. It will contain the definitions
// of the menu bar and command and status bars. It represents the main
// interaction point with the user.
// 
// We start out with the #includes needed to define this class plus
// any V utility dialogs such as vNotice we use.

#include <v/vnotice.h>  // so we can use notice
#include <v/vkeys.h>    // to map keys
#include <v/vutil.h>    // for utilities
#include <v/vfilesel.h> // for file select

#include "tcmdwin.h"            // our header file

// Now, we define static arrays for the menus, command bars, and
// status bars used by this window. This consists of defining the
// constants needed for IDs, followed by the static declarations of
// the menu and command arrays. Note that V predefines quite a few
// standard IDs which you can use instead of defining your own.

  // Start ID defines for the main window at 100

  const ItemVal m_CheckMe = 100;  // for CheckMe command
  const ItemVal m_CopySens = 101; // for Set Copy Sensitive
  const ItemVal m_Dialog = 102;   // to pop up the dialog
  const ItemVal m_ModalDialog = 103; // for modal dialog
  const ItemVal m_Clear = 104;    // Clear screen

// Now, the static declarations of the menu arrays. You first define
// the pulldown menus, one for each main menu bar label.

    static vMenu FileMenu[] =    // Items for File menu
      {
        {"New",M_New,isSens,notChk,noKeyLbl,noKey,noSub},
        {"Open",M_Open,isSens,notChk,noKeyLbl,noKey,noSub},
        {"Save",M_Save,notSens,notChk,noKeyLbl,noKey,noSub},
        {"Save As",M_SaveAs,notSens,notChk,noKeyLbl,noKey,noSub},
#ifdef vDEBUG                   // Defines V debug code
        {"-",M_Line,notSens,notChk,noKeyLbl,noKey,noSub},
        {"Debug",M_SetDebug,isSens,notChk,noKeyLbl,noKey,noSub},
#endif
        {"-",M_Line,notSens,notChk,noKeyLbl,noKey,noSub},
        {"Exit",M_Exit,isSens,notChk,noKeyLbl,noKey,noSub},
        {NULL}
      };

    static vMenu EditMenu[] =    // Items for Edit menu
      {
        {"Cut",M_Cut,notSens,notChk,noKeyLbl,noKey,noSub},
        {"Copy",M_Copy,notSens,notChk,noKeyLbl,noKey,noSub},
        {"Paste",M_Paste,notSens,notChk,noKeyLbl,noKey,noSub},
        {NULL}
      };

    static vMenu TestMenu[] =   // Items for Test menu
      {
        {"CheckMe",m_CheckMe,isSens,notChk,noKeyLbl,
            noKey,noSub},
        {"Copy Sensitive",m_CopySens,isSens,notChk,noKeyLbl,
            noKey,noSub},
        {"Dialog",m_Dialog,isSens,notChk,noKeyLbl,
            noKey,noSub},
        {"Modal Dialog",m_ModalDialog,isSens,notChk,noKeyLbl,
            noKey,noSub},
        {NULL}
      };

    // Now, define the items on the menu bar

    vMenu StandardMenu[] =     // The menu bar with three items
      {
        {"File",M_File,isSens,notUsed,notUsed,noKey,&FileMenu[0]},
        {"Edit",M_Edit,isSens,notUsed,notUsed,noKey,&EditMenu[0]},
        {"Test",M_Test,isSens,notUsed,notUsed,noKey,&TestMenu[0]},
        {NULL}
      };

// We now define a command bar. Command bars are optional, and there
// may be more than one. You can place any CommandObject you want on a
// command bar.

    static CommandObject CommandBar[] =  // A simple command bar
      {
        {C_Label,999,0 ,"Command Bar",NoList,CA_None,
            isSens,NoFrame,0,0},
        {C_Button,M_Copy,M_Copy,"Copy",NoList,CA_None,
            notSens,NoFrame,0,0},
        {C_Button,m_Dialog,m_Dialog,"Dialog",NoList,CA_None,
            isSens,NoFrame,0,0},
        {C_Button,m_Clear,m_Clear,"Clear",NoList,CA_None,
            isSens,NoFrame,0,0},
        {C_Button,M_Exit,M_Exit,"Exit",NoList,CA_None,
            isSens,NoFrame,0,0},
        {C_EndOfList,0,0,0,0,CA_None,0,0,0}  // This ends list
      };

// Sometimes it is easier to define IDs near the definition of
// the dialog or status bar using them.

  const ItemVal m_cmdMsg = 110;
  const ItemVal m_cmdCount = 111;
  const ItemVal m_keyMsg = 112;
  const ItemVal m_keyVal = 113;

    static vStatus StatBar[] =    // Define a simple status bar
      {
        {"Commands issued: ",m_cmdMsg,CA_NoBorder,isSens,0},
        {" ",m_cmdCount,CA_None,isSens,0},
        {"Last keypress: ",m_keyMsg,CA_NoBorder,isSens,0},
        {"   ",m_keyVal,CA_None,isSens,0},
        {0,0,0,0,0}           // This ends list
      };

    static int copy_sens = 0;   // local for tracking sensitive

//======================>>> tCmdWindow::tCmdWindow <<<====================
  tCmdWindow::tCmdWindow(char* name, int width, int height) :
    vCmdWindow(name, width)
  {
    // This is the constructor for tCmdWindow. 
    
    UserDebug1(Constructor,"tCmdWindow::tCmdWindow(%s) Constructor\n",name)

    // The "Standard" window will consist of a menubar, a canvas, an
    // optional button bar, and an optional status bar.
    // 
    // First, create and add the proper panes to the CmdWindow
    // Note: there must be a corresponding delete in the destructor

    // Create and add the standard Menu Bar to this window
    myMenu = new vMenuPane(StandardMenu);
    AddPane(myMenu);

    // Create and add our Canvas pane to this window
    myCanvas = new tCanvasPane;
    AddPane(myCanvas);

    // Create and add the command pane to this window
    myCmdPane = new vCommandPane(CommandBar);
    AddPane(myCmdPane);

    // Create and add the Status Bar to this window
    myStatus = new vStatusPane(StatBar);
    AddPane(myStatus);

    // In the V model, a window may have dialogs. Each dialog used
    // by a window must have an instance pointer. The easiest way
    // to create dialogs is to construct each one using a new here
    // which only defines the dialog - you need to use its
    // ShowDialog method at the appropriate time to display it).
    // You delete dialogs in the destructor for this window.
    // 
    // Now, create whatever dialogs this app defines:
    // instances of tDialog and tModalDialog

    sampleDialog = new tDialog(this);
    sampleModalDialog = new tModalDialog(this);

    // FINALLY, after all the panes have been constructed and
    // added, we must show the window!

    ShowWindow();
  }

//=====================>>> tCmdWindow::~tCmdWindow <<<====================
  tCmdWindow::~tCmdWindow()
  {
    UserDebug(Destructor,"tCmdWindow::~tCmdWindow() destructor\n")

    // Now put a delete for each new in the constructor.

    delete myMenu;
    delete myCanvas;
    delete myStatus;
    delete myCmdPane;
    delete sampleDialog;
    delete sampleModalDialog;
  }

//========================>>> tCmdWindow::KeyIn <<<=======================
  void tCmdWindow::KeyIn(vKey keysym, unsigned int shift)
  {
    // Keystrokes are routed to this window. This example code shows very
    // simple processing of keystrokes, and how to update the m_keyVal
    // field of the status bar.

    static char ctrl[] = "^X ";
    static char chr[] = " X ";

    if (VK_IsModifier(keysym))
        SetString(m_keyVal, "mod");     // change status bar
    else if (keysym < ' ')              // ctrl char
      {
        ctrl[1] = keysym + '@';
        SetString(m_keyVal, ctrl);      // change status bar
      }
    else if (keysym < 128)              // normal printable char
      {
        chr[1] = keysym;
        SetString(m_keyVal, chr);       // change status bar
      }
    else 
        SetString(m_keyVal, "+++");     // change status bar
  }

//====================>>> tCmdWindow::WindowCommand <<<===================
  void tCmdWindow::WindowCommand(ItemVal id, ItemVal val, CmdType cType)
  {
    // All commands generated from this window's menus and dialog bars
    // are routed through here.  The easiest way to handle commands is to
    // use a single, sometimes large switch. Each time you add a command
    // to a menu or command bar, add a case to the switch here. In this
    // example, we use the V Notice dialog to display entered commands.

    static int cmdCount = 0;    // Used for sample status update
    vNoticeDialog note(this);   // Used for default actions
    char buff[20];              // buffer for status bar

    ++cmdCount;                 // count commands that have been issued
    IntToStr(cmdCount,buff);    // Use V utility routine to get string
    SetString(m_cmdCount, buff);        // change status bar

    UserDebug1(CmdEvents,"tCmdWindow:WindowCommand(%d)\n",id)

    switch (id)                 // The main switch to handle commands
      {
        // File Menu commands ------------------------------------

        case M_New:             // For this example, we will open a
          {                     // new window using our NewAppWin.
            (void) theApp->NewAppWin(0,"",250,100);
            break;
          }

        case M_Open:            // This demos vFileSelect dialog
          {
            char name[100] = "";        // start out with null name
            vFileSelect fsel(this);     // an instance of vFileSelect
            int fI;                     // Filter index
            static char* filter[] = {   // Filter for file select
                "*", "*.txt", "*.c *.cpp *.h", 0 };

            // Show the file select dialog
            int ans = fsel.FileSelect("Open file",name,99,filter,fI);

            if (ans && *name)   // User picked a file name
              {
                SetTitle(name); // Set title of window to name
                note.Notice(name);  // Show the name
              }
            else                // Notify no name selected
                note.Notice("No file name selected.");
          }

        case M_Save:            // This would usually save a file
          {
            note.Notice("Save");
            break;
          }

        case M_SaveAs:          // Save to a specified name
          {
            note.Notice("Save As");
            break;
          }

#ifdef vDEBUG                   // Include debugging like this
        case M_SetDebug:
          {
            vDebugDialog debug(this);   // an instance of debug 
            debug.SetDebug();           // dialog - let user set
            break;
          }
#endif

        case M_Exit:            // Standard exit command
          {                     // Invoke the standard app Exit
            theApp->Exit();     // to close all windows
            break;              // will never get here
          }

        // Edit Menu commands ------------------------------------
        case M_Cut:             // Standard items for Edit menu
          {
            note.Notice("Cut");
            break;
          }

        case M_Copy:
          {
            note.Notice("Copy");
            break;
          }

        case M_Paste:
          {
            note.Notice("Paste");
            break;
          }

        // Test Menu commands ------------------------------------
        case m_CheckMe:         // Demonstrate using a checked menu
          {
            ItemVal curval = GetValue(id); // Get current status
            SetValue(m_CheckMe,!curval,Checked); // Toggle check

            if (curval)                 // Change menu label
                SetString(m_CheckMe,"Check Me");
            else
                SetString(m_CheckMe,"UnChk Me");
            break;
          }

        case m_CopySens:        // Demo changing sensitivity
          {
            copy_sens = !copy_sens;     // toggle
            // This will change both menu and command button
            SetValue(M_Copy,copy_sens,Sensitive);
            break;
          }

        case m_Dialog:          // Invoke our dialog
          {
            if (!sampleDialog->IsDisplayed())   // not twice!
                sampleDialog->ShowDialog("Sample Modeless Dialog");
            break;
          }

        case m_ModalDialog:     // Invoke our modal dialog
          {
            ItemVal val, id;
            id = sampleModalDialog->ShowModalDialog("Sample Modal",val);
            // Now do something useful with id and val ...
            break;
          }

        case m_Clear:           // Clear the canvas
          {
            myCanvas->Clear();  // Invoke the canvas Clear
            break;
          }

        default:                // route unhandled commands up
          {                     // to superclass
            vCmdWindow::WindowCommand(id, val, cType);
            break;
          }
      }
  }
\end{verbatim}
\normalfont\normalsize

\subsection*{tcmdwin.h}
%\input{../tutor/tcmdwin.h}
\footnotesize
\begin{verbatim}
//========================================================================
//      tcmdwin.h:  Header file for tutorial V command window
//
//      Copyright 1995, Bruce E. Wampler. All rights reserved.
//========================================================================
//
// Derive a window from the vCmdWindow class

#ifndef TCMDWIN_H
#define TCMDWIN_H

#include <v/vcmdwin.h>  // So we can use vCmdWindow
#include <v/vmenu.h>    // For the menu pane
#include <v/vstatusp.h> // For the status pane
#include <v/vcmdpane.h> // command pane

#ifdef vDEBUG
#include <v/vdebug.h>
#endif

#include "tdialog.h"    // user defined: tDialog
#include "tmodal.h"     // user defined: tModalDialog
#include "tcanvas.h"    // user defined: tCanvasPane

    class tCmdWindow : public vCmdWindow
      {
        friend int AppMain(int, char**);        // allow AppMain access

      public:           //---------------------------------------- public
        tCmdWindow(char*, int, int);    // Constructor with size
        virtual ~tCmdWindow();          // Destructor
        virtual void WindowCommand(ItemVal id,ItemVal val,CmdType cType);
        virtual void KeyIn(vKey keysym, unsigned int shift);

      protected:        //------------------------------------- protected

      private:          //--------------------------------------- private

        // Each user CmdWindow should conform to a "Standard" window,
        // which includes a menu bar, a canvas, an optional command bar,
        // and an optional status bar.

        vMenuPane* myMenu;              // For the menu bar
        tCanvasPane* myCanvas;          // For the canvas
        vStatusPane* myStatus;          // For the status bar
        vCommandPane* myCmdPane;        // for the command pane

        // Each user CmdWindow will probably have some dialogs and
        // subwindows. Declare pointers to each instance here.

        tDialog* sampleDialog;
        tModalDialog* sampleModalDialog;
      };
#endif
\end{verbatim}
\normalfont\normalsize

%----------------------------------------------------------------------
\section{The Canvas}

\subsection*{tcanvas.cpp}

%\input{../tutor/tcanvas.cpp}
\footnotesize
\begin{verbatim}
//========================================================================
//      tcanvas.cpp - source for tutorial canvas
//
//      Copyright 1995, Bruce E. Wampler, All Rights Reserved.
//========================================================================
//
// Each V application usually needs a canvas. In order to handle
// various events: mouse, redraw, resize, and scroll, you will need to
// derive your own canvas class. The base V vCanvasPane class can only
// draw -- it does not have any memory of what has been drawn on the
// screen (the vTextCanvasPane does handle redrawing, but is still
// limited). Thus, your class will usually be responsible for handling
// redrawing. This example is very simple. It lets the user draw
// lines - up to 200 - and will redraw the screen when it has been
// exposed.

// The example does not handle scrolling.

#include "tcanvas.h"            // include our header file

//====================>>> tCanvasPane::tCanvasPane <<<====================
  tCanvasPane::tCanvasPane()
  {
    // The constructor initializes our simple data structure.

    _mouseDown = 0; _nextpt = 0;
    _begx = -1; _begy = -1; _curx = -1; _cury = -1;
    _pt = new point[200];       // allocate only 200 lines
  }

//-===================>>> tCanvasPane::tCanvasPane <<<====================
  tCanvasPane::~tCanvasPane()
  {
    delete [] _pt;              // free the point array
  }

//======================>>> tCanvasPane::Clear <<<========================
  void tCanvasPane::Clear()
  {
    vCanvasPane::Clear();       // clear the canvas
    _nextpt = 0;                // start over at 0
  }

// This example does not handle scrolling, but a derived canvas would
// be likely to. Thus, we've included the derived scrolling methods,
// but simply call the superclass method for default handling, which
// is essentially a no op.

//======================>>> tCanvasPane::HPage <<<========================
  void tCanvasPane::HPage(int shown, int top)
  {
    vCanvasPane::HPage(shown, top);
  }

//======================>>> tCanvasPane::VPage <<<========================
  void tCanvasPane::VPage(int shown, int top)
  {
    vCanvasPane::VPage(shown, top);
  }

//======================>>> tCanvasPane::HScroll <<<======================
  void tCanvasPane::HScroll(int step)
  {
    vCanvasPane::HScroll(step);
  }

//======================>>> tCanvasPane::VScroll <<<======================
  void tCanvasPane::VScroll(int step)
  {
    vCanvasPane::VScroll(step);
  }

//=====================>>> tCanvasPane::MouseDown <<<=====================
  void tCanvasPane::MouseDown(int X, int Y, int button)
  {
   // Mouse down means the user is starting a line. We don't care which
   // button was pressed. There is nothing to draw until the mouse moves.

    _mouseDown = 1;                     // track mouse button
    _pt[_nextpt].x = _begx = _curx = X; // starting point
    _pt[_nextpt].y = _begy = _cury = Y;
    if (++_nextpt >= 200)               // set next point and do a simple
        _nextpt = 0;                    // minded storage allocation
  }

//======================>>> tCanvasPane::MouseMove <<<====================
  void tCanvasPane::MouseMove(int x, int y, int button)
  {
    // Mouse move means the user is drawing a line, so we have to draw
    // it on the screen. By drawing a Rubber Line, we can easily track
    // the user motions, and undraw the previous line.

    if (_begx != _curx || _begy != _cury) // Was there a previous line?
        DrawRubberLine(_begx, _begy, _curx, _cury);  // Undraw old line

    if (_begx != x || _begy != y)       // If we moved, draw new line
        DrawRubberLine(_begx, _begy, x, y);
    
    _curx = x; _cury = y;               // update positions
  }

//========================>>> tCanvasPane::MouseUp <<<====================
  void tCanvasPane::MouseUp(int X, int Y, int button)
  {
    // Mouse up means the user has ended a line, so we need to draw
    // a permanent line and update the data base.

    _mouseDown = 0;                     // Mouse down now
    if (_begx != X || _begy != Y)       // We drew a line
        DrawLine(_begx, _begy, X, Y);   // So draw permanent version

    _pt[_nextpt].x = X; _pt[_nextpt].y = Y;  // End point

    if (++_nextpt >= 200)               // set next point and do a simple
        _nextpt = 0;                    // minded storage allocation

    _begx = -1; _begy = -1; _curx = -1; _cury = -1;  // for next line
  }

//========================>>> tCanvasPane::Redraw <<<=====================
  void tCanvasPane::Redraw(int x, int y, int w, int h)
  {
    // This is a simple Redraw that just redraws everything.
    // Often, that will be more than fast enough, but the input
    // parameters can be used to make a more intelligent redraw.

    for (int i = 0 ; i < _nextpt ; i += 2)
        DrawLine(_pt[i].x, _pt[i].y, _pt[i+1].x, _pt[i+1].y);
  }

//======================>>> tCanvasPane::Resize <<<=======================
  void tCanvasPane::Resize(int w, int h)
  {
    // We also don't handle resizing in this example.
    vCanvasPane::Resize(w,h);
  }
\end{verbatim}
\normalfont\normalsize

\subsection*{tcanvas.h}
%\input{../tutor/tcanvas.h}
\footnotesize
\begin{verbatim}
//========================================================================
//  tcanvas.h -- header file for tutorial canvas class
//
//      Copyright 1995, Bruce E. Wampler, All Rights Reserved.
//========================================================================

#ifndef TCANVAS_H
#define TCANVAS_H

#include <v/vcanvas.h>  // derive from vCanvasPane

    typedef struct point        // simple structure for points
      {
        int x; int y;
      } point;

    class tCanvasPane : public vCanvasPane
      {
      public:           //---------------------------------------- public
        tCanvasPane();
        virtual ~tCanvasPane();

        // Windows
        virtual void Clear();

        // Scrolling
        virtual void HPage(int, int);
        virtual void VPage(int, int);
        virtual void HScroll(int);
        virtual void VScroll(int);

        // Events
        virtual void MouseDown(int, int, int);
        virtual void MouseUp(int, int, int);
        virtual void MouseMove(int, int, int);
        virtual void Redraw(int, int, int, int); // Expose/redraw event
        virtual void Resize(int, int);          // Resize event

      protected:        //------------------------------------- protected

      private:          //--------------------------------------- private
        // Note that we try to use a leading underscore to indicate
        // private members. We aren't always consistent!
        int _mouseDown;         // track if mouse down
        int _begx; int _begy;   // starting point
        int _curx; int _cury;   // current point
        point *_pt;             // the array of points
        int _nextpt;            // where next point goes
      };
#endif
\end{verbatim}
\normalfont\normalsize

%----------------------------------------------------------------------
\section{A Modeless Dialog}

\subsection*{tdialog.cpp}

%\input{../tutor/tdialog.cpp}
\footnotesize
\begin{verbatim}
//========================================================================
//  tdialog.cpp - Source file for tutorial tDialog class
//
//  Copyright 1995, Bruce E. Wampler, All Rights Reserved
//========================================================================

// #include the headers we need
#include <v/vnotice.h>
#include "tdialog.h"

// The structure of a derived dialog class is very similar to the
// structure of a command window class. First we define IDs for the
// various command objects used in the dialog. Then we declare the
// static array that defines the dialog.

const ItemVal mdLbl1 = 200;

const ItemVal mdFrm1 = 201;  const ItemVal mdLbl2 = 202;
const ItemVal mdCB1 = 203;   const ItemVal mdCB2 = 204;
const ItemVal mdCB3 = 205;

const ItemVal mdFrmV1 = 206; const ItemVal mdLbl3 = 207;
const ItemVal mdRB1 = 208;   const ItemVal mdRB2 = 209;

const ItemVal mdFrmV2 = 210; const ItemVal mdLbl4 = 211;
const ItemVal mdBtn1 = 212;  const ItemVal mdBtn2 = 213;

const ItemVal mdBtnChange = 214;

    static char change_me[] = "Change Me A";    // a label to change

// This defines the dialog

    static DialogCmd DefaultCmds[] =
      {
        {C_Label, mdLbl1, 0,"X",NoList,CA_MainMsg,isSens,NoFrame, 0, 0},

        {C_Frame,mdFrmV2,0,"",NoList,CA_None,isSens,NoFrame,0,mdLbl1},
        {C_Label,mdLbl4,0,"Buttons",NoList,CA_None,isSens,mdFrmV2,0,0},
        {C_Button,mdBtn1,mdBtn1,"Button 1",NoList,CA_None,
                isSens,mdFrmV2,0,mdLbl4},
        {C_Button,mdBtn2,mdBtn2,"Button 2",NoList,CA_None,
                isSens,mdFrmV2,0,mdBtn1},

        {C_Frame,mdFrm1,0,"",NoList,CA_None,isSens,NoFrame,mdFrmV2,mdLbl1},
        {C_Label,mdLbl2,0,"CheckBox",NoList,CA_None,isSens,mdFrm1,0,0},
        {C_CheckBox,mdCB1,0,"Test A",NoList,CA_None,
                isSens,mdFrm1,0,mdLbl2},
        {C_CheckBox,mdCB2,0,"Test B",NoList,CA_None,
                isSens,mdFrm1,mdCB1,mdLbl2},
        {C_CheckBox,mdCB3,1,"Test C",NoList,CA_None,isSens,mdFrm1,0,mdCB1},

        {C_Frame,mdFrmV1,0,"",NoList,CA_None,isSens,NoFrame,mdFrm1,mdLbl1},
        {C_Label,mdLbl3,0,"Radios",NoList,CA_None,isSens,mdFrmV1,0,0},
        {C_RadioButton,mdRB1,1,"KOB",NoList,CA_None,
                isSens,mdFrmV1,0,mdLbl3},
        {C_RadioButton,mdRB2,0,"KOAT",NoList,CA_None,
                isSens,mdFrmV1,0,mdRB1},

        {C_Button,mdBtnChange,0,change_me,NoList,CA_None,
                isSens,NoFrame,0,mdFrmV1},
        {C_Button,M_Cancel,M_Cancel," Cancel ",NoList,CA_None,
            isSens,NoFrame,mdBtnChange,mdFrmV1},
        {C_Button,M_OK,M_OK," OK ",NoList,CA_DefaultButton,
            isSens,NoFrame,M_Cancel,mdFrmV1},

        {C_EndOfList,0,0,0,0,CA_None,0,0,0}
      };


//==========================>>> tDialog::tDialog <<<======================
  tDialog::tDialog(vBaseWindow* bw) :
    vDialog(bw)
  {
    // The constructor for a derived dialog calls the superclass
    // constructor, and then adds the command objects to the dialog
    // by calling AddDialogCmds.

    UserDebug(Constructor,"tDialog::tDialog()\n")
    AddDialogCmds(DefaultCmds);         // add the command objects
  }

//=========================>>> tDialog::~tDialog <<<======================
  tDialog::~tDialog()
  {
    // Destructor often doesn't need to do anything

    UserDebug(Destructor,"tDialog::~tDialog() destructor\n")
  }

//====================>>> tDialog::DialogCommand <<<======================
  void tDialog::DialogCommand(ItemVal id, ItemVal retval, CmdType ctype)
  {
    // After the user has selected a command from the dialog,
    // this routine is called with the value.

    vNoticeDialog note(this);   // an instance we can use

    UserDebug1(CmdEvents,"tDialog::DialogCommand(id:%d)\n",id)

    switch (id)         // We will do some things depending on value
      {
        case mdCB1:             // CheckBox
            note.Notice("Test A");
            break;

        case mdCB2:             // CheckBox
            note.Notice("Test B");
            break;

        case mdCB3:             // CheckBox
            note.Notice("Test C");
            break;

        case mdRB1:             // Radio Button
            note.Notice("KOB");
            break;

        case mdRB2:             // Radio Button
            note.Notice("KOAT");
            break;

        case mdBtn1:            // Button
            note.Notice("Button 1");
            break;

        case mdBtn2:            // Button
            note.Notice("Button 2");
            break;

        case mdBtnChange:       // Example: change my own label
            // We will change the label on this button
            change_me[10]++;            // change the "A"
            SetString(mdBtnChange, change_me);
            break;
      }
    // All commands should also route through the parent handler
    // which has useful default behaviors for Cancel and OK
    vDialog::DialogCommand(id,retval,ctype);
  }
\end{verbatim}
\normalfont\normalsize

\subsection*{tdialog.h}
%\input{../tutor/tdialog.h}
\footnotesize
\begin{verbatim}
//========================================================================
//
//  tdialog.h - Header file for tutorial tDialog class
//
//  Copyright 1995, Bruce E. Wampler, All Rights Reserved
//========================================================================

#ifndef TDIALOG_H
#define TDIALOG_H

#include <v/vdialog.h>  // we derive from vDialog

    class tDialog : public vDialog
      {
      public:           //---------------------------------------- public
        tDialog(vBaseWindow*);
        virtual ~tDialog();             // Destructor
        virtual void DialogCommand(ItemVal id, ItemVal retval,
                CmdType ctype);

      protected:        //------------------------------------- protected

      private:          //--------------------------------------- private
        int _toggleId;
      };
#endif
\end{verbatim}
\normalfont\normalsize

%----------------------------------------------------------------------
\section{A Modal Dialog}

\subsection*{tmodal.cpp}

%\input{../tutor/tmodal.cpp}
\footnotesize
\begin{verbatim}
//========================================================================
//  tmodal.cpp - Source file for tutorial tModalDialog class
//
//  Copyright 1995, Bruce E. Wampler, All Rights Reserved
//========================================================================
//

#include "tmodal.h"             // our header file
#include <v/vnotice.h>

const ItemVal mmLbl1 = 300;
const ItemVal mmBtn1 = 301;
const ItemVal mmBtn2 = 302;

    static CommandObject DefaultCmds[] =
      {
        {C_Label, mmLbl1, 0,"X",NoList,CA_MainMsg,isSens,NoFrame, 0, 0},
        
        {C_Button,mmBtn1,mmBtn1," Test 1 ",NoList,CA_None,
                isSens,NoFrame,0,mmLbl1},
        {C_Button,mmBtn2,mmBtn2," Test 2 ",NoList,CA_None,
                isSens,NoFrame, mmBtn1,mmLbl1},

        {C_Button,M_Cancel,M_Cancel," Cancel ",NoList,CA_None,
                isSens,NoFrame, 0,mmBtn1},
        {C_Button,M_OK,M_OK,"   OK   ",NoList,CA_DefaultButton,
                isSens,NoFrame,M_Cancel,mmBtn1},

        {C_EndOfList,0,0,0,0,CA_None,0,0,0}
      };


//======================>>> tModalDialog::tModalDialog <<<================
  tModalDialog::tModalDialog(vBaseWindow* bw) :
    vModalDialog(bw)
  {
    UserDebug(Constructor,"tModalDialog::tModalDialog()\n")
    AddDialogCmds(DefaultCmds);         // add the predefined commands
  }

//=================>>> tModalDialog::~tModalDialog <<<====================
  tModalDialog::~tModalDialog()
  {
    UserDebug(Destructor,"tModalDialog::~tModalDialog() destructor\n")
  }

//===================>>> tModalDialog::DialogCommand <<<==================
  void tModalDialog::DialogCommand(ItemVal id,ItemVal retval,CmdType ctype)
  {
    // After the user has selected a command from the dialog,
    // this routine is called with the id and retval.

    vNoticeDialog note(this);

    UserDebug1(CmdEvents,"tModalDialog::DialogCommand(id:%d)\n",id)

    switch (id)         // We will do some things depending on value
      {
        case mmBtn1:            // Button 1
            note.Notice(" Test 1 ");
            break;

        case mmBtn2:            // Button 2
            note.Notice(" Test 2 ");
            break;
      }

    // let default behavior handle Cancel and OK
    vModalDialog::DialogCommand(id,retval,ctype);
  }
\end{verbatim}
\normalfont\normalsize

%\input{../tutor/tmodal.h}
\subsection*{tmodal.h}
\footnotesize
\begin{verbatim}
//========================================================
//  tmodal.h - Header file for tModalDialog class
//
//  Copyright 1995, Bruce E. Wampler, All Rights Reserved
//========================================================

#ifndef TMODAL_H
#define TMODAL_H

#include <v/vmodald.h>  // derived from vModalDialog

    class tModalDialog : public vModalDialog
      {
      public:           //---------------------------------------- public
        tModalDialog(vBaseWindow*);
        virtual ~tModalDialog();                // Destructor
        virtual void DialogCommand(ItemVal id, ItemVal retval,
                CmdType ctype);

      protected:        //--------------------------------------- protected

      private:          //--------------------------------------- private

      };
#endif
\end{verbatim}
\normalfont\normalsize

%----------------------------------------------------------------------
\section{The Makefile}

\subsection*{makefile}

%# Makefile for Independent JPEG Group's software

# makefile.cfg is edited by configure to produce a custom Makefile.

# Read installation instructions before saying "make" !!

# For compiling with source and object files in different directories.
srcdir = @srcdir@
VPATH = @srcdir@

# Where to install the programs and man pages.
prefix = @prefix@
exec_prefix = @exec_prefix@
bindir = $(exec_prefix)/bin
libdir = $(exec_prefix)/lib
includedir = $(prefix)/include
binprefix =
manprefix =
manext = 1
mandir = $(prefix)/man/man$(manext)

# The name of your C compiler:
CC= @CC@

# You may need to adjust these cc options:
CFLAGS= @CFLAGS@ @CPPFLAGS@ @INCLUDEFLAGS@
# Generally, we recommend defining any configuration symbols in jconfig.h,
# NOT via -D switches here.
# However, any special defines for ansi2knr.c may be included here:
ANSI2KNRFLAGS= @ANSI2KNRFLAGS@

# Link-time cc options:
LDFLAGS= @LDFLAGS@

# To link any special libraries, add the necessary -l commands here.
LDLIBS= @LIBS@

# Put here the object file name for the correct system-dependent memory
# manager file.  For Unix this is usually jmemnobs.o, but you may want
# to use jmemansi.o or jmemname.o if you have limited swap space.
SYSDEPMEM= @MEMORYMGR@

# miscellaneous OS-dependent stuff
SHELL= /bin/sh
# linker
LN= $(CC)
# file deletion command
RM= rm -f
# file rename command
MV= mv
# library (.a) file creation command
AR= ar rc
# second step in .a creation (use "touch" if not needed)
AR2= @RANLIB@
# installation program
INSTALL= @INSTALL@
INSTALL_PROGRAM= @INSTALL_PROGRAM@
INSTALL_DATA= @INSTALL_DATA@

# End of configurable options.


# source files: JPEG library proper
LIBSOURCES= jcapimin.c jcapistd.c jccoefct.c jccolor.c jcdctmgr.c jchuff.c \
        jcinit.c jcmainct.c jcmarker.c jcmaster.c jcomapi.c jcparam.c \
        jcphuff.c jcprepct.c jcsample.c jctrans.c jdapimin.c jdapistd.c \
        jdatadst.c jdatasrc.c jdcoefct.c jdcolor.c jddctmgr.c jdhuff.c \
        jdinput.c jdmainct.c jdmarker.c jdmaster.c jdmerge.c jdphuff.c \
        jdpostct.c jdsample.c jdtrans.c jerror.c jfdctflt.c jfdctfst.c \
        jfdctint.c jidctflt.c jidctfst.c jidctint.c jidctred.c jquant1.c \
        jquant2.c jutils.c jmemmgr.c jmemansi.c jmemname.c jmemnobs.c \
        jmemdos.c
# source files: cjpeg/djpeg/jpegtran applications, also rdjpgcom/wrjpgcom
APPSOURCES= cjpeg.c djpeg.c jpegtran.c cdjpeg.c rdcolmap.c rdswitch.c \
        rdjpgcom.c wrjpgcom.c rdppm.c wrppm.c rdgif.c wrgif.c rdtarga.c \
        wrtarga.c rdbmp.c wrbmp.c rdrle.c wrrle.c
SOURCES= $(LIBSOURCES) $(APPSOURCES)
# files included by source files
INCLUDES= jchuff.h jdhuff.h jdct.h jerror.h jinclude.h jmemsys.h jmorecfg.h \
        jpegint.h jpeglib.h jversion.h cdjpeg.h cderror.h
# documentation, test, and support files
DOCS= README install.doc usage.doc cjpeg.1 djpeg.1 jpegtran.1 rdjpgcom.1 \
        wrjpgcom.1 wizard.doc example.c libjpeg.doc structure.doc \
        coderules.doc filelist.doc change.log
MKFILES= configure makefile.cfg makefile.ansi makefile.unix makefile.bcc \
        makefile.mc6 makefile.dj makefile.wat makcjpeg.st makdjpeg.st \
        makljpeg.st maktjpeg.st makefile.manx makefile.sas makefile.mms \
        makefile.vms makvms.opt
CONFIGFILES= jconfig.cfg jconfig.manx jconfig.sas jconfig.st jconfig.bcc \
        jconfig.mc6 jconfig.dj jconfig.wat jconfig.vms
OTHERFILES= jconfig.doc ckconfig.c ansi2knr.c ansi2knr.1 jmemdosa.asm
TESTFILES= testorig.jpg testimg.ppm testimg.gif testimg.jpg testprog.jpg \
        testimgp.jpg
DISTFILES= $(DOCS) $(MKFILES) $(CONFIGFILES) $(SOURCES) $(INCLUDES) \
        $(OTHERFILES) $(TESTFILES)
# library object files common to compression and decompression
COMOBJECTS= jcomapi.o jutils.o jerror.o jmemmgr.o $(SYSDEPMEM)
# compression library object files
CLIBOBJECTS= jcapimin.o jcapistd.o jctrans.o jcparam.o jdatadst.o jcinit.o \
        jcmaster.o jcmarker.o jcmainct.o jcprepct.o jccoefct.o jccolor.o \
        jcsample.o jchuff.o jcphuff.o jcdctmgr.o jfdctfst.o jfdctflt.o \
        jfdctint.o
# decompression library object files
DLIBOBJECTS= jdapimin.o jdapistd.o jdtrans.o jdatasrc.o jdmaster.o \
        jdinput.o jdmarker.o jdhuff.o jdphuff.o jdmainct.o jdcoefct.o \
        jdpostct.o jddctmgr.o jidctfst.o jidctflt.o jidctint.o jidctred.o \
        jdsample.o jdcolor.o jquant1.o jquant2.o jdmerge.o
# These objectfiles are included in libjpeg.a
LIBOBJECTS= $(CLIBOBJECTS) $(DLIBOBJECTS) $(COMOBJECTS)
# object files for sample applications (excluding library files)
COBJECTS= cjpeg.o rdppm.o rdgif.o rdtarga.o rdrle.o rdbmp.o rdswitch.o \
        cdjpeg.o
DOBJECTS= djpeg.o wrppm.o wrgif.o wrtarga.o wrrle.o wrbmp.o rdcolmap.o \
        cdjpeg.o
TROBJECTS= jpegtran.o rdswitch.o cdjpeg.o


all: @ANSI2KNR@ libjpeg.a libjpeg.so.6.0.0 cjpeg djpeg jpegtran rdjpgcom wrjpgcom

# This rule causes ansi2knr to be invoked.
.c.o:
	$(CC) -fPIC $(CFLAGS) -c $*.c
	$(MV) $*.o shared/
	$(CC) $(CFLAGS) -c $*.c

@ISANSICOM@	./ansi2knr $(srcdir)/$*.c T$*.c
@ISANSICOM@	$(CC) $(CFLAGS) -c T$*.c
@ISANSICOM@	$(RM) T$*.c $*.o
@ISANSICOM@	$(MV) T$*.o $*.o

ansi2knr: ansi2knr.c
	$(CC) $(CFLAGS) $(ANSI2KNRFLAGS) -o ansi2knr ansi2knr.c

libjpeg.a: @ANSI2KNR@ $(LIBOBJECTS)
	$(RM) libjpeg.a
	$(AR) libjpeg.a  $(LIBOBJECTS)
	$(AR2) libjpeg.a

libjpeg.so.6.0.0: $(LIBOBJECTS)
	(cd shared; $(CC) -shared -Wl,-soname,libjpeg.so.6 -o ../libjpeg.so.6.0.0 $(LIBOBJECTS) )
	ln -sf libjpeg.so.6.0.0 libjpeg.so.6

cjpeg: $(COBJECTS) libjpeg.a
	$(LN) $(LDFLAGS) -o cjpeg $(COBJECTS) -L. -ljpeg  $(LDLIBS)

djpeg: $(DOBJECTS) libjpeg.a
	$(LN) $(LDFLAGS) -o djpeg $(DOBJECTS) -L. -ljpeg  $(LDLIBS)

jpegtran: $(TROBJECTS) libjpeg.a
	$(LN) $(LDFLAGS) -o jpegtran $(TROBJECTS) -L. -ljpeg  $(LDLIBS)

rdjpgcom: rdjpgcom.o
	$(LN) $(LDFLAGS) -o rdjpgcom rdjpgcom.o $(LDLIBS)

wrjpgcom: wrjpgcom.o
	$(LN) $(LDFLAGS) -o wrjpgcom wrjpgcom.o $(LDLIBS)

jconfig.h: jconfig.doc
	echo You must prepare a system-dependent jconfig.h file.
	echo Please read the installation directions in install.doc.
	exit 1

install: cjpeg djpeg jpegtran rdjpgcom wrjpgcom
	$(INSTALL_PROGRAM) cjpeg $(bindir)/$(binprefix)cjpeg
	$(INSTALL_PROGRAM) djpeg $(bindir)/$(binprefix)djpeg
	$(INSTALL_PROGRAM) jpegtran $(bindir)/$(binprefix)jpegtran
	$(INSTALL_PROGRAM) rdjpgcom $(bindir)/$(binprefix)rdjpgcom
	$(INSTALL_PROGRAM) wrjpgcom $(bindir)/$(binprefix)wrjpgcom
	$(INSTALL_DATA) $(srcdir)/cjpeg.1 $(mandir)/$(manprefix)cjpeg.$(manext)
	$(INSTALL_DATA) $(srcdir)/djpeg.1 $(mandir)/$(manprefix)djpeg.$(manext)
	$(INSTALL_DATA) $(srcdir)/jpegtran.1 $(mandir)/$(manprefix)jpegtran.$(manext)
	$(INSTALL_DATA) $(srcdir)/rdjpgcom.1 $(mandir)/$(manprefix)rdjpgcom.$(manext)
	$(INSTALL_DATA) $(srcdir)/wrjpgcom.1 $(mandir)/$(manprefix)wrjpgcom.$(manext)

install-lib: libjpeg.a libjpeg.so.6.0.0 install-headers
	$(INSTALL_DATA) libjpeg.a $(libdir)/$(binprefix)libjpeg.a
	$(INSTALL_PROGRAM) libjpeg.so.6.0.0 $(libdir)/$(binprefix)libjpeg.so.6.0.0
	ldconfig -v

install-headers: jconfig.h
	$(INSTALL_DATA) jconfig.h $(includedir)/jconfig.h
	$(INSTALL_DATA) $(srcdir)/jpeglib.h $(includedir)/jpeglib.h
	$(INSTALL_DATA) $(srcdir)/jmorecfg.h $(includedir)/jmorecfg.h
	$(INSTALL_DATA) $(srcdir)/jerror.h $(includedir)/jerror.h

clean:
	$(RM) *.o shared/*.o cjpeg djpeg jpegtran libjpeg.a libjpeg.so.6.0.0 libjpeg.so.6 rdjpgcom wrjpgcom
	$(RM) ansi2knr core testout* config.log config.status

distribute:
	$(RM) jpegsrc.tar*
	tar cvf jpegsrc.tar $(DISTFILES)
	compress -v jpegsrc.tar

test: cjpeg djpeg jpegtran
	$(RM) testout*
	./djpeg -dct int -ppm -outfile testout.ppm  $(srcdir)/testorig.jpg
	./djpeg -dct int -gif -outfile testout.gif  $(srcdir)/testorig.jpg
	./cjpeg -dct int -outfile testout.jpg  $(srcdir)/testimg.ppm
	./djpeg -dct int -ppm -outfile testoutp.ppm $(srcdir)/testprog.jpg
	./cjpeg -dct int -progressive -opt -outfile testoutp.jpg $(srcdir)/testimg.ppm
	./jpegtran -outfile testoutt.jpg $(srcdir)/testprog.jpg
	cmp $(srcdir)/testimg.ppm testout.ppm
	cmp $(srcdir)/testimg.gif testout.gif
	cmp $(srcdir)/testimg.jpg testout.jpg
	cmp $(srcdir)/testimg.ppm testoutp.ppm
	cmp $(srcdir)/testimgp.jpg testoutp.jpg
	cmp $(srcdir)/testorig.jpg testoutt.jpg

check: test

# GNU Make likes to know which target names are not really files to be made:
.PHONY: all install install-lib install-headers clean distribute test check


jcapimin.o: jcapimin.c jinclude.h jconfig.h jpeglib.h jmorecfg.h jpegint.h jerror.h
jcapistd.o: jcapistd.c jinclude.h jconfig.h jpeglib.h jmorecfg.h jpegint.h jerror.h
jccoefct.o: jccoefct.c jinclude.h jconfig.h jpeglib.h jmorecfg.h jpegint.h jerror.h
jccolor.o: jccolor.c jinclude.h jconfig.h jpeglib.h jmorecfg.h jpegint.h jerror.h
jcdctmgr.o: jcdctmgr.c jinclude.h jconfig.h jpeglib.h jmorecfg.h jpegint.h jerror.h jdct.h
jchuff.o: jchuff.c jinclude.h jconfig.h jpeglib.h jmorecfg.h jpegint.h jerror.h jchuff.h
jcinit.o: jcinit.c jinclude.h jconfig.h jpeglib.h jmorecfg.h jpegint.h jerror.h
jcmainct.o: jcmainct.c jinclude.h jconfig.h jpeglib.h jmorecfg.h jpegint.h jerror.h
jcmarker.o: jcmarker.c jinclude.h jconfig.h jpeglib.h jmorecfg.h jpegint.h jerror.h
jcmaster.o: jcmaster.c jinclude.h jconfig.h jpeglib.h jmorecfg.h jpegint.h jerror.h
jcomapi.o: jcomapi.c jinclude.h jconfig.h jpeglib.h jmorecfg.h jpegint.h jerror.h
jcparam.o: jcparam.c jinclude.h jconfig.h jpeglib.h jmorecfg.h jpegint.h jerror.h
jcphuff.o: jcphuff.c jinclude.h jconfig.h jpeglib.h jmorecfg.h jpegint.h jerror.h jchuff.h
jcprepct.o: jcprepct.c jinclude.h jconfig.h jpeglib.h jmorecfg.h jpegint.h jerror.h
jcsample.o: jcsample.c jinclude.h jconfig.h jpeglib.h jmorecfg.h jpegint.h jerror.h
jctrans.o: jctrans.c jinclude.h jconfig.h jpeglib.h jmorecfg.h jpegint.h jerror.h
jdapimin.o: jdapimin.c jinclude.h jconfig.h jpeglib.h jmorecfg.h jpegint.h jerror.h
jdapistd.o: jdapistd.c jinclude.h jconfig.h jpeglib.h jmorecfg.h jpegint.h jerror.h
jdatadst.o: jdatadst.c jinclude.h jconfig.h jpeglib.h jmorecfg.h jerror.h
jdatasrc.o: jdatasrc.c jinclude.h jconfig.h jpeglib.h jmorecfg.h jerror.h
jdcoefct.o: jdcoefct.c jinclude.h jconfig.h jpeglib.h jmorecfg.h jpegint.h jerror.h
jdcolor.o: jdcolor.c jinclude.h jconfig.h jpeglib.h jmorecfg.h jpegint.h jerror.h
jddctmgr.o: jddctmgr.c jinclude.h jconfig.h jpeglib.h jmorecfg.h jpegint.h jerror.h jdct.h
jdhuff.o: jdhuff.c jinclude.h jconfig.h jpeglib.h jmorecfg.h jpegint.h jerror.h jdhuff.h
jdinput.o: jdinput.c jinclude.h jconfig.h jpeglib.h jmorecfg.h jpegint.h jerror.h
jdmainct.o: jdmainct.c jinclude.h jconfig.h jpeglib.h jmorecfg.h jpegint.h jerror.h
jdmarker.o: jdmarker.c jinclude.h jconfig.h jpeglib.h jmorecfg.h jpegint.h jerror.h
jdmaster.o: jdmaster.c jinclude.h jconfig.h jpeglib.h jmorecfg.h jpegint.h jerror.h
jdmerge.o: jdmerge.c jinclude.h jconfig.h jpeglib.h jmorecfg.h jpegint.h jerror.h
jdphuff.o: jdphuff.c jinclude.h jconfig.h jpeglib.h jmorecfg.h jpegint.h jerror.h jdhuff.h
jdpostct.o: jdpostct.c jinclude.h jconfig.h jpeglib.h jmorecfg.h jpegint.h jerror.h
jdsample.o: jdsample.c jinclude.h jconfig.h jpeglib.h jmorecfg.h jpegint.h jerror.h
jdtrans.o: jdtrans.c jinclude.h jconfig.h jpeglib.h jmorecfg.h jpegint.h jerror.h
jerror.o: jerror.c jinclude.h jconfig.h jpeglib.h jmorecfg.h jversion.h jerror.h
jfdctflt.o: jfdctflt.c jinclude.h jconfig.h jpeglib.h jmorecfg.h jpegint.h jerror.h jdct.h
jfdctfst.o: jfdctfst.c jinclude.h jconfig.h jpeglib.h jmorecfg.h jpegint.h jerror.h jdct.h
jfdctint.o: jfdctint.c jinclude.h jconfig.h jpeglib.h jmorecfg.h jpegint.h jerror.h jdct.h
jidctflt.o: jidctflt.c jinclude.h jconfig.h jpeglib.h jmorecfg.h jpegint.h jerror.h jdct.h
jidctfst.o: jidctfst.c jinclude.h jconfig.h jpeglib.h jmorecfg.h jpegint.h jerror.h jdct.h
jidctint.o: jidctint.c jinclude.h jconfig.h jpeglib.h jmorecfg.h jpegint.h jerror.h jdct.h
jidctred.o: jidctred.c jinclude.h jconfig.h jpeglib.h jmorecfg.h jpegint.h jerror.h jdct.h
jquant1.o: jquant1.c jinclude.h jconfig.h jpeglib.h jmorecfg.h jpegint.h jerror.h
jquant2.o: jquant2.c jinclude.h jconfig.h jpeglib.h jmorecfg.h jpegint.h jerror.h
jutils.o: jutils.c jinclude.h jconfig.h jpeglib.h jmorecfg.h jpegint.h jerror.h
jmemmgr.o: jmemmgr.c jinclude.h jconfig.h jpeglib.h jmorecfg.h jpegint.h jerror.h jmemsys.h
jmemansi.o: jmemansi.c jinclude.h jconfig.h jpeglib.h jmorecfg.h jpegint.h jerror.h jmemsys.h
jmemname.o: jmemname.c jinclude.h jconfig.h jpeglib.h jmorecfg.h jpegint.h jerror.h jmemsys.h
jmemnobs.o: jmemnobs.c jinclude.h jconfig.h jpeglib.h jmorecfg.h jpegint.h jerror.h jmemsys.h
jmemdos.o: jmemdos.c jinclude.h jconfig.h jpeglib.h jmorecfg.h jpegint.h jerror.h jmemsys.h
cjpeg.o: cjpeg.c cdjpeg.h jinclude.h jconfig.h jpeglib.h jmorecfg.h jerror.h cderror.h jversion.h
djpeg.o: djpeg.c cdjpeg.h jinclude.h jconfig.h jpeglib.h jmorecfg.h jerror.h cderror.h jversion.h
jpegtran.o: jpegtran.c cdjpeg.h jinclude.h jconfig.h jpeglib.h jmorecfg.h jerror.h cderror.h jversion.h
cdjpeg.o: cdjpeg.c cdjpeg.h jinclude.h jconfig.h jpeglib.h jmorecfg.h jerror.h cderror.h
rdcolmap.o: rdcolmap.c cdjpeg.h jinclude.h jconfig.h jpeglib.h jmorecfg.h jerror.h cderror.h
rdswitch.o: rdswitch.c cdjpeg.h jinclude.h jconfig.h jpeglib.h jmorecfg.h jerror.h cderror.h
rdjpgcom.o: rdjpgcom.c jinclude.h jconfig.h
wrjpgcom.o: wrjpgcom.c jinclude.h jconfig.h
rdppm.o: rdppm.c cdjpeg.h jinclude.h jconfig.h jpeglib.h jmorecfg.h jerror.h cderror.h
wrppm.o: wrppm.c cdjpeg.h jinclude.h jconfig.h jpeglib.h jmorecfg.h jerror.h cderror.h
rdgif.o: rdgif.c cdjpeg.h jinclude.h jconfig.h jpeglib.h jmorecfg.h jerror.h cderror.h
wrgif.o: wrgif.c cdjpeg.h jinclude.h jconfig.h jpeglib.h jmorecfg.h jerror.h cderror.h
rdtarga.o: rdtarga.c cdjpeg.h jinclude.h jconfig.h jpeglib.h jmorecfg.h jerror.h cderror.h
wrtarga.o: wrtarga.c cdjpeg.h jinclude.h jconfig.h jpeglib.h jmorecfg.h jerror.h cderror.h
rdbmp.o: rdbmp.c cdjpeg.h jinclude.h jconfig.h jpeglib.h jmorecfg.h jerror.h cderror.h
wrbmp.o: wrbmp.c cdjpeg.h jinclude.h jconfig.h jpeglib.h jmorecfg.h jerror.h cderror.h
rdrle.o: rdrle.c cdjpeg.h jinclude.h jconfig.h jpeglib.h jmorecfg.h jerror.h cderror.h
wrrle.o: wrrle.c cdjpeg.h jinclude.h jconfig.h jpeglib.h jmorecfg.h jerror.h cderror.h

\footnotesize
\begin{verbatim}
# Sample GNU make makefile for V tutorial application
CC      =       g++

# Note: Platform dependent for a Linux system

HOME    =       /home/bruce
X11INC  =       /usr/X11/include
X11LIB  =       /usr/X11R6/lib
Arch    =       intel
LIBS    =       -lV -lXaw -lXmu -lXt -lXext -lX11

VPATH   =       ../include

# Architecture dependent

VLibDir =       $(HOME)/v/lib/$(Arch)

oDir    =       ../obj/$(Arch)

LibDir  =       ../lib/$(Arch)

Bin     =       ../bin/$(Arch)

#--------------------------------------------------------------

# Typical flags for includes and libraries

CFLAGS  =       -O -I$(X11INC) -I$(HOME)

LFLAGS  =       -O -L$(X11LIB) -L$(VLibDir)

EXOBJS  =       $(oDir)/tutapp.o \
                $(oDir)/tdialog.o \
                $(oDir)/tmodal.o \
                $(oDir)/tcanvas.o \
                $(oDir)/tcmdwin.o

all:    $(Bin)/tutapp

$(Bin)/tutapp:  $(EXOBJS) $(VLibDir)/libV.a
        $(CC) -o $@ $(LFLAGS) $(EXOBJS) $(LIBS)

$(oDir)/tcanvas.o:      tcanvas.cpp v_defs.h tcanvas.h
        $(CC) -c $(CFLAGS) -o $@ $<                     

$(oDir)/tdialog.o:      tdialog.cpp v_defs.h tdialog.h
        $(CC) -c $(CFLAGS) -o $@ $<                     

$(oDir)/tmodal.o:       tmodal.cpp v_defs.h tmodal.h
        $(CC) -c $(CFLAGS) -o $@ $<                     

$(oDir)/tcmdwin.o:      tcmdwin.cpp v_defs.h tcmdwin.h
        $(CC) -c $(CFLAGS) -o $@ $<                     

$(oDir)/tutapp.o:       tutapp.cpp v_defs.h tdialog.h tmodal.h \
        tutapp.h tcmdwin.h
        $(CC) -c $(CFLAGS) -o $@ $<
\end{verbatim}
\normalfont\normalsize


%***********************************************************************
%***********************************************************************
%***********************************************************************

\chapter{C++ Coding Style Guidelines}
\index{coding style}

I have developed the following guidelines for writing C++ code
over my long career as a programmer. All of \V\ has been written
using these guidelines, and I believe that using them is a big
first step leading to readable, portable, and reliable code.
Of course, just following these guidelines won't automatically
give you that, but I think they are still necessary.


\section {Readability}
\index{readability}

The ultimate goal of style guidelines is to help you to write code
that is readable. While this means code that is readable by you,
it mostly means code that is readable by others. Remember,
\emph{code has a life of its own!} No matter how small the project
may seem, or how temporary, most code ends up being used and reused
much longer than you might think. The real cost of software
is often in the long term maintenance. While you may end up maintaining
your own code, often it will be someone else. Even if it is you,
after a few months, or even weeks, you will have likely forgotten
just exactly what you were doing when you wrote the code to begin
with.

The point of this is to emphasize the importance of producing readable
code. Generally, readable code is inviting to look at. It is visually
pleasing, just as a well designed book is well laid out and visually
pleasing to look at.  Your code should have plenty of visual attributes
that make it easy to read.  This means lots of whitespace, consistent
indentation, abundant, well formatted comments, and visual
separation of important sections of code. Much of the structure
of your code should be visually obvious without having to read it.
Many of the following guidelines are intended to help
you produce readable code.

\section{Naming}
\index{naming}\index{variable names}

It is critical to choose meaningful names for your variables and
functions. Avoid short, two or three letter names unless those
names are really meaningful. While you may want to use short,
abbreviated names to avoid typing, this habit will make your
code more difficult to read later. While you should avoid
short names, consistently using names that are too long can
present problems, too. This can lead to code that must be
split across multiple lines because the names are too long.
Even so, it is probably better to trend to overly long names
than short, abbreviated names.

Names should use both upper and lower case letters, using
a case change to indicate word breaks. For example, a name
like \code{maxLength} is more readable than \code{MAXLENGTH},
\code{maxlength}, or even \code{max\_length}. In general,
using mixed case is better than using underscores. Underscores are
better used to indicate special classes of variables (see Class
Definitions below).

\section{Files}
\index{file naming}

Each C++ module should be split into two files -- the \code{.h} header
file which contains \code{class} definitions and variable declarations,
and a \code{.c} or \code{.cpp}\footnote{The naming conventions for
C++ source files has not really been standardized yet. Common
alternatives for .cpp include .C and .cxx.} file that contains
source code for the functions.

Generally, each class will have its own \code{.h} and \code{.cpp}
files. Utility helper functions that go with a class can be included
in the same file as the class. Other functions that do not go with
a class should be collected into logical groups and kept in 
a separate file. In general, files should not be much larger than
twenty to forty thousand characters long.

\section{Include Files}

Include files or header files (\code{.h}) files must each have
a \code{\#define} statement that prevents problems caused by
multiple inclusion. The standard way to do this is:

\begin{verbatim}
//
//  myclass.h - header file for myclass class definition
//
#ifndef MYCLASS_H              // Check to prevent
#define MYCLASS_H              // multiple inclusion

    ... definitions go here

#endif                         // last line of file
\end{verbatim}

Files are included from the source file by placing the \#include
statement near the beginning of the source file, starting in
column one.

\begin{verbatim}

#include "myclass.h"      // includes start in column 1

\end{verbatim}

\section{Function Definitions}
\index{function style}

All functions should have a prototype definition for use
by others. For class methods, this will be part of the class
declaration. For other functions, this should also be in a
\code{.h} file. The parameters of all prototypes should include
both the \emph{type} and the \emph{name} of each parameter since
the name often conveys extra useful information.

The body of each method or function should use this convention:

\footnotesize
\begin{verbatim}
//==================>>> myClass::myMethod <<<===================
  void myClass::myMethod(const int size)
  {
    // An introductory block of comments explaining the purpose
    // and interface to this function. You can also include an
    // author and modification history here if appropriate.

    ... declare variables used throughout the function

    ... body of function
  }
\end{verbatim}
\normalfont\normalsize

Each function should include the separator line to visually
separate the body of the function from others in the same file.
The preferred indentation for the function name is two spaces,
with the enclosing \{ and \} braces on separate lines, also
indented two spaces. An acceptable alternative style is to
have these lines start in the first column.

Following the opening \{ should come an introduction to the
function. Variables required by the entire function follow
the opening comments.  Following that is the body of the function.
Make liberal use of whitespace for visual separation.

\section{Indentation}
\index{indentation}

The preferred indentation scheme is based on groups of four
spaces, with braces indented two additional spaces. It is
acceptable to keep braces lined up with the outer statement
rather than indenting two extra, but all braces \emph{must}
be on a line by themselves. This spacing works well with
standard eight character tab stops -- your code will either
be indented on even tab stops, or on tab stops plus four.

Except for the most trivial cases of short, related assignment
statements, each statement should be on a separate line.
The body of loops and conditional statements should always
use braces -- never use a simple statement. There are two reasons
for this. First, using braces on separate lines adds whitespace,
which adds to the readability. Second, code is inevitably modified,
and by always using braces, you will be more likely to add
a statement in the proper place. As a special case for
initializer loops with no code body, it is acceptable to use just
a semicolon rather than braces.

The old K\&R style of placing the opening brace at the end
of the line is not acceptable. Most importantly, you lose
the visual impact of lined up braces when you do this. It
also tends to compress the code, and extra whitespace really
helps make code more readable.

When calling functions that require long, complex argument
lists, it is often advisable to place each argument on a
separate line accompanied by an explanatory comment.

Use a blank between keywords and the associated left paren:
\code{if (test)}.  Don't put a space for function calls:
\code{function(param);}. Don't use parens for the returned
value of a \code{return} statement. This helps to visually
distinguish a \code{return} from a function call.

The following code demonstrates indentation for various
C++ statements:

\footnotesize
\begin{verbatim}
//=====================>>> sample <<<=====================
  int sample(int action)
  {
    // This meaningless sample demonstrates indentation.
    // The code should not considered to do anything useful
    // other than demonstrating indentation.

    char* name;                 // explain each variable
    int set;                    // with useful comment

    if (action)                 // indent 4, space after if
      {                         // the { in 2
        set = doSomething(action);
      }                         // always use {}'s
    else
      {
        set = SomethingElse(action);
      }

    switch (set)                // example for switch
      {
        case 1:                 // case in +4 from switch
          {                     // always use braces for cases!
            getName1(name); 
            break;
          }

        case 2:                 // Try to comment each case
          {
            int temp = len(name);  // try to declare as needed
            fixName(temp,name);
            break;
          }

        default:                // Good idea to have default
            break;
      }

    // Prefix some blocks with comments like this
    // to describe what a section of code does
    // Note that 'char* cp' is preferred to 'char *cp'.
    // Take advantage of C++ scope rules, and declare
    // variables (e.g., len) as close to use as possible.
    for (int len = 0, char* cp = name ; *cp ; ++cp)
      {
        ++len;                 // all loops use {}'s
        tryThat(set,cp);       // and meaningful comments!
      }

    while (IsStillOK(name))    // indent like for
      {
        Complex(name,          // a complex function call
            set,               // can explain each parameter
            len);              // for easier maintenance
      }

    int status = (checkName(name)) ? len    // sample ?:
                                   : len / 2;

    return status;             // no parens on return
  }
\end{verbatim}
\normalfont\normalsize

\section{Comments}
\index{comments}\index{code comments}

It is difficult to over comment your code. Comments are
one of the most helpful things you can do to make your code
easier to maintain. A 1 to 1 ratio of comments to code should
be considered a bare minimum, with a ratio of more comments than
code probably a better thing.

I claim it is almost impossible to have too many comments. A few
expert programmers may disagree with this philosophy, and say that
well written code can be self-commenting. The problem is that
this is not really true. Assume, for example, that you are using
a standard software library, such as \code{Xt} or \V\@. You may know the
library backwards and forwards, and it may seem perfectly clear
to you what some code is doing. But assume that someone else
will be maintaining the code later. They may not know the library
as well, and what is obvious and self-commenting to you will
be gibberish to them. A few well placed comments explaining what
you are doing will be very helpful.

In order to write really effective comments, you must
comment \emph{as you write the code}! Do not go back and
add comments after the code is written. You can go back and
improve and expand your comments, but you should comment as
you go. A few seconds taken to add a comment as you write the
code can save many minutes or even hours later.

Comments should be meaningful and correct. If you change code,
be sure you change the comments to correspond! If you are in the
habit of commenting as you write, this will not be so hard.

Make the layout of your comments visually pleasing. Use whitespace
to separate sections of code. Line up block comments near the
left, and try to keep short per line comments lined up on
the right without going too far right. Line them up on a tab
stop if possible.

Above all, remember that what seems obvious to you at the moment
is likely to be forgotten even a week or two later.  And keep in
mind that someone else is likely to modify or study your code
later.  Don't keep secrets. If you had to look something up, or
have other information that might make the code more
understandable, put that in a comment.  If you are doing
something tricky or obscure (which you should avoid, but
sometimes can't avoid), explain what is going on! You might be
teaching a valuable trick to whoever is working with your code
later!

My own code has more comments than almost any other code I've seen.
Time after time, when someone else has had to use or maintain my
code, I've gotten feedback that it is very easy to understand and
modify. I attribute much of this positive feedback to the abundant
comments found in my code.

\section{Class Definitions}

The standards for class definitions are based on keeping
braces on separate lines, and on not using implicit assumptions.
Thus, a class will have braces on separate lines, either indented
two (the preferred style), or lined up with the class statement.

There should always be all three \code{public}, \code{protected},
and \code{private} sections in that order, even if a section is
empty. This order assumes it is more useful to have the public
stuff at the top for easier readability. And even if a section
is empty, that conveys information about the definition of the class.
The prototypes for member functions should include both the type
and name (e.g., \code{int OnOrOff}).

There should almost never be public access to class variables.
Provide methods to access and set variables of the class. You
may find it helpful to prefix class variables (especially private
class variables) with an underscore (\_variableName) to indicate
the variable is private to the class.

The following example shows indentation and layout of a class
definition. Note the visual separator for public, protected, and
private, and the alignment with the braces.

\footnotesize
\begin{verbatim}
    class myClass : public superClass    // name here
      {
        friend int FriendFunction(int ival);  // friends at top

      public:    //-------------------------- public

        myClass();          // constructor and destructor
        virtual ~myClass(); // first

        // simple methods can be inline
        int getVal() { return _val; }
        virtual void service(int iParam);    // prototype

      protected:  //------------------------- protected

        // even an empty section conveys information

      private:    //------------------------- private

        int _val;           // _ for class variables
      };
\end{verbatim}
\normalfont\normalsize

\section{C++ Language Features}

With C++, it is preferred to use \code{const} definitions
of symbolic values rather than \code{\#defines}.

Use \code{const} parameters whenever possible.

Declare variables as you need them, preferably inside
a code block, rather than all at the top of a function.
This makes your code much more maintainable, and helps
avoid errors introduced by bad reuse of a variable,
especially in loops.

For each \code{new} operation, there \emph{must} be a
corresponding \code{delete} operation. These \code{new}
and \code{delete} pairs will often be found in the
constructor and destructor for your objects.

\emph{Always} define copy constructors and an assignment operator
for each class if they use pointers and dynamic memory allocation
using \code{new}. Some of the biggest problems in C++ code
involves objects with pointers to dynamically allocated
space. You should use either deep copy semantics or reference
counts to avoid creating objects with dangling pointers.

When using \V\, use the debug macros as much as possible.
It is especially helpful to use \code{UserDebug} statements
in constructors and destructors.

\section{Software Portability}
\index{portability}

Always remember that your code might someday be ported to a
different system. Keep this in mind when writing your code.
These guidelines will help to make your code more portable.

Don't use nonstandard or nonportable language features. For
example, templates are not yet universally portable. Avoid
using them.

Use restrictive names when naming files. The most conservative
approach is to use single case names limited to 8 characters
for the name part, and 3 for the extension. This should
get better as time goes by, but for now this is still
a pretty good idea.

If you must use system calls, abstract them and isolate them
in a single place.

Don't go behind the back of \V\ to access X directly.

Avoid conversions that are Big and Little Endian dependent.
If you need them, isolate them.


%***********************************************************************
%***********************************************************************
%***********************************************************************

\chapter {Quick Reference}

%----------------------------------------------------------------
\Class{CommandObject}

\footnotesize
\begin{verbatim}
//  CommandObject: "#include <v/v_defs.h>"

typedef struct CommandObject
  {
    CmdType cmdType;    // what kind of item is this
    ItemVal cmdId;      // unique id for the item
    ItemVal retVal;     // value returned when picked
    char* title;        // string
    void* itemList;     // used when cmd needs a list
    CmdAttribute attrs; // list of attributes
    int Sensitive;      // if item is sensitive or not
    ItemVal cFrame;     // Frame used for an item
    ItemVal cRightOf;   // Item placed left of this id
    ItemVal cBelow;     // Item placed below this one
    int size;           // Used for size information
  } CommandObject;
\end{verbatim}
\normalfont\normalsize

\subsection*{CmdType}

\footnotesize
\begin{verbatim}
                   (with suggested prefix for id names)
    C_EndOfList:   Used to denote end of command list
    C_Blank:       filler to help RightOfs, Belows work (blk)
    C_BoxedLabel:  a label with a box (bxl)
    C_Button:      Button (btn)
    C_CheckBox:    Checked Item (chk)
    C_ColorButton: Colored button (cbt)
    C_ColorLabel:  Colored label (clb)
    C_ComboBox:    Popup combo list (cbx)
    C_Frame:       General purpose frame (frm)
    C_Icon:        a display only Icon (ico)
    C_IconButton:  a command button Icon (icb)
    C_Label:       Regular text label (lbl)
    C_List:        List of items (lst)
    C_ProgressBar: Bar to show progress (pbr)
    C_RadioButton: Radio button (rdb)
    C_Slider:      Slider to enter value (sld)
    C_Spinner:     Spinner value entry (spn)
    C_TextIn:      Text input field (txi)
    C_Text:        wrapping text out (txt)
    C_ToggleButton: a toggle button (tbt)
    C_ToggleFrame: a toggle frame (tfr)
    C_ToggleIconButton:  a toggle Icon button (tib)
\end{verbatim}
\normalfont\normalsize

\subsection*{CmdAttribute}

\footnotesize
\begin{verbatim}
    CA_DefaultButton: Special Default Button
       [Used by: C_Button]
    CA_Hidden:        Starts out hidden
       [All controls]
    CA_Horizontal:    Horizontal orientation
       [C_ProgressBar, C_Slider]
    CA_Large:         Command larger than normal
       [C_List C_ProgressBar, C_Slider, C_Spinner, C_TextIn]
    CA_MainMsg:       Gets replacement message
       [C_Label]
    CA_NoBorder:      No border (frames, status bar)
       [C_Frame,C_ToggleFrame,C_Text,status bar]
    CA_NoLabel:       No label on progress bar
       [C_ProgressBar]
    CA_NoNotify:      Don't notify on all events
       [C_ComboBox,C_List,C_Spinner]
    CA_NoSpace:       No space between widgets
       [C_Frame,C_ToggleFrame]
    CA_None:          No special attributes
       [All controls]
    CA_Percent:       Use % on progress bar
       [C_ProgressBar]
    CA_Small:         Command smaller than normal
       [C_ProgressBar,C_TextIn]
    CA_Text:          A Text value box
       [C_Spinner]
    CA_Vertical:      Vertical orientation
       [C_ProgressBar, C_Slider]
\end{verbatim}
\normalfont\normalsize

\subsection*{Useful symbolic values}

\footnotesize
\begin{verbatim}
    NoList:           No list used
    NoFrame:          Not a member of a frame
    isSens:           Is sensitive
    notSens:          Not sensitive
    isChk:            Is checked
    notChk:           Not checked
    notUsed:          Not used
    noIcon:           No icon
\end{verbatim}
\normalfont\normalsize

%----------------------------------------------------------------
\Class{vApp}

\footnotesize
\begin{verbatim}
//  vApp:  "#include <v/vapp.h>"

    class vApp : public vBaseItem
      {
      public:  //---------------------------------------- public
        vApp(char* appName);
        virtual ~vApp();

        // Methods to override

        virtual void AppCommand(vWindow* win, ItemVal id,
                                ItemVal retval, CmdType ctype);
        virtual void CloseAppWin(vWindow* win);
        virtual void Exit(void);
        virtual void KeyIn(vWindow* win, vKey key, unsigned int shift);
        virtual vWindow* NewAppWin(vWindow* win, char* name, int w, int h, 
                               vAppWinInfo* winInfo = 0);
        // Utility methods

        int DefaultHeight();
        int DefaultWidth();
        int IsRunning();
        void SendWindowCommandAll(ItemVal id, int val, CmdType ctype);
        void SetValueAll(ItemVal id, int val, ItemSetType setType);
        void SetStringAll(ItemVal id, char* str);
        const char* name();

        vAppWinInfo *getAppWinInfo(vWindow* Win);

        // platform dependent

        Display* display();           // To get the X display
        XtAppContext appContext();    // To get the context
      };

    extern vApp *theApp;     // Pointer to single global instance
    extern int AppMain(int argc, char** argv);  // Pseudo main program

\end{verbatim}
\normalfont\normalsize

%----------------------------------------------------------------
\Class{vBrush}

\footnotesize
\begin{verbatim}
//  vBrush:  "#include <v/vbrush.h>"
    class vBrush
      {
      public:       //-------------------------------- public
        vBrush(unsigned int r = 0, unsigned int g = 0, unsigned int b = 0,
               int style = vSolid, int fillMode = vAlternate);
        ~vBrush();

        int operator ==(vBrush b2);
        int operator !=(vBrush b2);

        vColor GetColor();
        int GetFillMode();
        int GetStyle();
        void SetColor(vColor c);
        void SetFillMode(int fillMode);
        void SetStyle(int style);
      };
\end{verbatim}
\normalfont\normalsize

%----------------------------------------------------------------
\Class{vCanvasPane}

\footnotesize
\begin{verbatim}
//  vCanvasPane:  "#include <v/vcanvas.h>"

    class vCanvasPane
      {
      public:   //---------------------------------------- public

        vCanvasPane(PaneType pt = P_Canvas);
        virtual ~vCanvasPane();

        virtual void ShowPane(int OnOrOff);

        // Cursor
        void SetCursor(VCursor id);
        VCursor GetCursor();
        void UnSetCursor(void);

        // Scrolling
        virtual void HPage(int Shown, int Top);
        virtual void VPage(int Shown, int Top);
        virtual void HScroll(int step);
        virtual void VScroll(int step);
        virtual void SetHScroll(int Shown, int Top);
        virtual void SetVScroll(int Shown, int Top);
        virtual int GetHScroll(int& Shown, int& Top);
        virtual int GetVScroll(int& Shown, int& Top);
        virtual void ShowHScroll(int OnOff);
        virtual void ShowVScroll(int OnOff);

        // Change messages
        virtual void FontChanged(vFont vf);

        // Events
        virtual void MouseDown(int x, int y, int button);
        virtual void MouseUp(int x, int y, int button);
        virtual void MouseMove(int x, int y, int button);
        virtual void EnterFocus();
        virtual void LeaveFocus();

        // Expose/redraw events
        virtual void Redraw(int x, int y, int width , int height);
        virtual void Resize(int newW, int newH);

        // Information
        virtual int GetWidth();
        virtual int GetHeight();

        // Drawing
        void Clear(void);
        virtual void ClearRect(int left, int top, int width, int height);
        virtual void DrawAttrText(int x, int y, char* text, const ChrAttr attr);
        virtual void DrawColorPoints(int x, int y, int nPoints, vColor* pts);
        virtual void DrawText(int x, int y, char* text);
        virtual void DrawEllipse(int x, int y, int width, int height);
        virtual void DrawIcon(int x, int y, vIcon& icon);
        virtual void DrawLine(int x, int y, int xend , int yend);
        virtual void DrawPoint(int x, int y);
        virtual void DrawPolygon(int n, vPoint points[], int fillStyle = vAlternate);
        virtual void DrawRectangle(int x, int y, int width, int height);
        virtual void DrawRoundedRectangle(int x, int y, int width, int height, int radius);
        virtual void DrawRubberLine(int x, int y, int xend, int yend);
        virtual void DrawRubberEllipse(int x, int y, int width, int height);
        virtual void DrawRubberRectangle(int x, int y, int width, int height);
        virtual vBrush GetBrush(void);
        virtual void SetBrush(vBrush brush);
        virtual vFont GetFont(void);
        virtual void SetFont(vFont fnt);
        virtual int TextWidth(char* str);
        virtual int TextHeight(int& ascent, int& descent);
        vDC* GetDC();

        // Appearance
        virtual void SetScale(int mult, int div);
        virtual void GetScale(int& mult, int& div);
        virtual void SetBackground(vColor color);
        virtual void SetPen(vPen pen);
        vPen GetPen();
        void SetTranslate(int x, int y);
        void SetTransX(int x);
        void SetTransY(int y);
        void GetTranslate(int& x, int& y);
        int GetTransX();
        int GetTransY();

        // Platform dependent
        Widget DrawingWindow();
      };
\end{verbatim}
\normalfont\normalsize

%----------------------------------------------------------------
\Class{vCmdWindow}

\footnotesize
\begin{verbatim}
//  vCmdWindow:  "#include <v/vcmdwin.h>"

    class vCmdWindow : public vWindow
      {
      public:           //----------------------------------- public
        vCmdWindow(char* name = "+", int width = 0, int height = 0);
        virtual ~vCmdWindow();                         // Destructor

        virtual void CloseWin(void);    // callback for close button
      };
\end{verbatim}
\normalfont\normalsize

%----------------------------------------------------------------
\Class{vColor}

\footnotesize
\begin{verbatim}
//  vColor:  "#include <v/vcolor.h>"

// Message constants for use in Color buttons (for color buttons, etc.)

        M_Black, M_Red, M_DimRed, M_Green, M_DimGreen, M_Blue, M_DimBlue,
        M_Yellow, M_DimYellow, M_Magenta, M_DimMagenta, M_Cyan, M_DimCyan,
        M_DarkGray, M_MedGray, M_White,
        M_ColorFrame

// Index constants into V "standard" color arrays: vStdColors, vColorNames.

        vC_Black, vC_Red, vC_DimRed, vC_Green, vC_DimGreen, vC_Blue, vC_DimBlue,
        vC_Yellow, vC_DimYellow, vC_Magenta, vC_DimMagenta, vC_Cyan, vC_DimCyan,
        vC_DarkGray, vC_MedGray, vC_White

    class vColor
      {
      public:    //---------------------------------------- public
        vColor(unsigned int rd = 0, unsigned int gr = 0, unsigned int bl = 0);
        ~vColor();

        int operator ==(vColor c2);
        int operator !=(vColor c2);
        int BitsOfColor();
        ResetColor(unsigned int rd = 0, unsigned int gr = 0, unsigned int bl = 0);
        Set(unsigned int rd = 0, unsigned int gr = 0, unsigned int bl = 0);
        Set(vColor& c);
        void SetR(unsigned int rd = 0);
        void SetG(unsigned int gr = 0);
        void SetB(unsigned int bl = 0);

        unsigned int r();
        unsigned int g();
        unsigned int b();
      };

    extern vColor vStdColors[16];       // 16 "standard" colors
    extern char* vColorNames[16];       // and their names
\end{verbatim}
\normalfont\normalsize

%----------------------------------------------------------------
\Class{vDebugDialog}

\footnotesize
\begin{verbatim}
//  vDebugDialog:  "#include <v/vdebug.h>"

    class vDebugDialog : public vModalDialog
      {
      public:    //---------------------------------------- public
        vDebugDialog(vBaseWindow* bw,char* title = "Debugging Options");
        vDebugDialog(vApp* aw,char* title = "Debugging Options");
        ~vDebugDialog();
        void SetDebug();
      };
\end{verbatim}
\normalfont\normalsize

\subsection* {Command Line Switches}
\begin{description}
\item [-vDebug SU$<$list$>$] Turn on System or User (or both) 
debug messages in list.
\item [c] Command events (menu, dialog commands).
\item [m] Mouse events.
\item [w] Window events (resize, redraw).
\item [b] Build events.
\item [o] Other misc. events.
\item [v] Bad values.
\item [C] Constructors.
\item [D] Destructors.
\item [123] User items 1, 2, or 3.
\end{description}

%----------------------------------------------------------------
\Class{vDialog}

\footnotesize
\begin{verbatim}
//  vDialog:  "#include <v/vdialog.h>"

    class vDialog
      {
      public:             //------------------------------------ public

        vDialog(vBaseWindow* creator, int modal = 0, char* title = "");
        vDialog(vApp* creator, int modal = 0, char* title = "");
        ~vDialog();

        WindowType wType();

        virtual void AddDialogCmds(CommandObject* cList);
        virtual void CancelDialog(void);
        virtual void CloseDialog(void);
        virtual void SetDialogTitle(char * title);
        virtual void DialogCommand(ItemVal id, ItemVal retval, CmdType ctype);
        virtual int GetTextIn(ItemVal id, char* strout, int maxlen);
        virtual int GetValue(ItemVal id);
        virtual void SetValue(ItemVal id, ItemVal val, ItemSetType setType);
        virtual void SetString(ItemVal id, char* str);
        int IsDisplayed(void);
        virtual void ShowDialog(const char* msg);
      };
\end{verbatim}
\normalfont\normalsize

%----------------------------------------------------------------
\Class{vFileSelect}

\footnotesize
\begin{verbatim}
//  vFileSelect:  "#include <v/vfilesel.h>"

    class vFileSelect : protected vModalDialog
      {
      public:     //---------------------------------------- public
        vFileSelect(vBaseWindow* bw, char* title = "File Select");
        vFileSelect(vApp* aw, char* title = "File Select");
        ~vFileSelect();

        int FileSelect(const char* msg, char* filename, 
            const int maxlen, char** filter);
        int FileSelectSave(const char* msg, char* filename, 
            const int maxlen, char** filter);
      };
\end{verbatim}
\normalfont\normalsize

%----------------------------------------------------------------
\Class{vFont}

\footnotesize
\begin{verbatim}
//  vFont:  "#include <v/vfont.h>"
    enum vFontID                // various font related ids
      {
        vfDefaultSystem,        // the default system font
        vfDefaultFixed,         // the system default fixed font
        vfDefaultVariable,      // the system default variable font
        vfSerif,                // serifed font - TimesRoman
        vfSansSerif,            // SansSerif - Swiss or Helvetica
        vfFixed,                // fixed font - Courier
        vfDecorative,           // decorative - dingbat
        vfOtherFont,            // for all other fonts
        vfNormal,               // normal style, weight
        vfBold,                 // boldface
        vfItalic,               // italic style
        vfEndOfList
      };

    class vFont         // make the font stuff a class to make it portable
      {
      public:           //---------------------------------------- public
        vFont(vFontID fam = vfDefaultFixed, int size = 10,
           vFontID sty = vfNormal, vFontID wt = vfNormal, int und = 0);
        ~vFont();

        vFontID GetFamily() { return _family; }
        int GetPointSize() { return _pointSize; }
        vFontID GetStyle() { return _style; }
        vFontID GetWeight() { return _weight; }
        int GetUnderlined() { return _underlined; }
        void SetFontValues(vFontID fam = vfDefaultFixed, int size = 10,
           vFontID sty = vfNormal, vFontID wt = vfNormal, int und = 0);
      };
#endif
\end{verbatim}
\normalfont\normalsize


%----------------------------------------------------------------
\Class{vFontSelect}

\footnotesize
\begin{verbatim}
//  vFontSelect:  "#include <v/vfontsel.h>"

    class vFontSelect : protected vModalDialog
      {
      public:       //---------------------------------------- public
        vFontSelect(vBaseWindow* bw, char* title = "Select Font");
        vFontSelect(vApp* aw, char* title = "Select Font");
       ~vFontSelect();

       int FontSelect(vFont& font, const char* msg = "Select Font" );
      };
#endif
\end{verbatim}
\normalfont\normalsize

%----------------------------------------------------------------
\Class{vIcon}

\footnotesize
\begin{verbatim}
    // <v/vicon.h>

    class vIcon     // an icon
      {
      public:             //---------------------------------------- public
        vIcon(unsigned char* ic, int h, int w, int d = 1);
        ~vIcon();
        int height;             // height in pixels
        int width;              // width in pixels
        int depth;              // bits per pixel
        unsigned char* icon;    // ptr to icon array

      protected:        //--------------------------------------- protected
      private:          //--------------------------------------- private
      };
\end{verbatim}
\normalfont\normalsize

%----------------------------------------------------------------
\Class{vMenu}

\footnotesize
\begin{verbatim}
//  vMenu:  "#include <v/v_menu.h>"

    typedef struct vMenu
      {
        char* label;       // The label on the menu
        ItemVal menuId;    // A User assigned unique id
        unsigned
          sensitive : 1,   // If item is sensitive or not
          checked : 1;     // If item is checked or not
        char* keyLabel;    // Label for an accelerator key
        vKey accel;        // Value of accelerator key
        vMenu* SubMenu;    // Ptr to a submenu 
      } MenuItem;
\end{verbatim}
\normalfont\normalsize

\subsection*{Useful symbolic values}

\footnotesize
\begin{verbatim}
    isSens:           Is sensitive
    notSens:          Not sensitive
    noSub:            No submenu
    isChk:            Is checked
    notChk:           Not checked
    noKey:            No accelerator specified
    noKeyLbl:         No accelerator label
\end{verbatim}
\normalfont\normalsize

%----------------------------------------------------------------
\Class{vModalDialog}
\footnotesize
\begin{verbatim}
//  vModalDialog:  "#include <v/vmodald.h>"

    class vModalDialog : public vDialog
      {
      public:   //---------------------------------------- public

        vModalDialog(vBaseWindow* creator, char* title = "");
        vModalDialog(vApp* creator, char* title = "");
        virtual ~vModalDialog();

        virtual ItemVal ShowModalDialog(const char* msg, ItemVal& retval);

        // rest are inherited from vDialog
      };
\end{verbatim}
\normalfont\normalsize

%----------------------------------------------------------------
\Class{vNoticeDialog}

\footnotesize
\begin{verbatim}
//  vNoticeDialog:  "#include <v/vnotice.h>"

    class vNoticeDialog : protected vModalDialog
      {
      public:    //---------------------------------------- public
        vNoticeDialog(vBaseWindow* bw, char* title = "Notice");
        vNoticeDialog(vApp* aw, char* title = "Notice");
        ~vNoticeDialog();

        void Notice(char* msg);
      };
\end{verbatim}
\normalfont\normalsize

%----------------------------------------------------------------
\Class{vPen}

\footnotesize
\begin{verbatim}
//  vPen:  "#include <v/vpen.h>"

    class vPen
      {
      public:    //---------------------------------------- public
        vPen(unsigned int r = 0, unsigned int g = 0,
             unsigned int b = 0, int width = 1, int style = vSolid);
        ~vPen();

        int operator ==(vPen p2);
        int operator !=(vPen p2);

        void SetStyle(int style);
        int GetStyle(void);
        void SetWidth(int width);
        int GetWidth();
        void SetColor(vColor c);
        vColor GetColor();
      };
\end{verbatim}
\normalfont\normalsize

%----------------------------------------------------------------
\Class{vReplyDialog}

\footnotesize
\begin{verbatim}
//  vReplyDialog:  "#include <v/vreply.h>"

    class vReplyDialog : protected vModalDialog
      {
      public:   //---------------------------------------- public
        vReplyDialog(vBaseWindow* bw, char* title = "Reply");
        vReplyDialog(vApp *aw, char* title = "Reply");

        int Reply(const char* msg, char* reply, const int maxlen);
      };
\end{verbatim}
\normalfont\normalsize

%----------------------------------------------------------------
\Class{vStatus}

\footnotesize
\begin{verbatim}
//  vStatus:  "#include <v/v_defs.h>"

    typedef struct vStatus      // for status bars
      {
        char* label;            // text label
        ItemVal statId;         // id
        CmdAttribute attrs;     // attributes - CA_NoBorder, CA_None
        unsigned sensitive : 1; // if button is sensitive or not
        int width;              // to specify width (0 for default)
      } vButton;
\end{verbatim}
\normalfont\normalsize

\subsection*{Useful symbolic values}

\footnotesize
\begin{verbatim}
    CA_NoBorder:      No border (frames, status bar)
    CA_None:          No special attributes
    isSens:           Is sensitive
    notSens:          Not sensitive
    noIcon:           No icon
\end{verbatim}
\normalfont\normalsize

%----------------------------------------------------------------
\Class{vTextCanvasPane}

\footnotesize
\begin{verbatim}
//  vTextCanvasPane:  "#include <v/vtextcnv.h>"

    class vTextCanvasPane : public vCanvasPane
      {
      public:  //---------------------------------------- public
        vTextCanvasPane();
        virtual ~vTextCanvasPane();

        // Window management/drawing

        virtual void Clear(void);
        virtual void ClearRow(const int r, const int c);
        virtual void ClearToEnd(const int r, const int c);
        int GetCols();
        int GetRows();
        void GetRC(int& r, int& c);
        void GotoRC(const int r, const int c);
        virtual void DrawAttrText(const char* text, const ChrAttr attr);
        virtual void DrawChar(const char chr, const ChrAttr attr);
        virtual void DrawText(const char* text);
        void HideTextCursor(void);
        void ShowTextCursor(void);

        // Scrolling
        void ScrollText(const int lineCount);

        // Events
        virtual void ResizeText(const int rows, const int cols);
        virtual void TextMouseDown(int row, int col, int button);
        virtual void TextMouseUp(int row, int col, int button);
        virtual void TextMouseMove(int row, int col, int button);
      };
\end{verbatim}
\normalfont\normalsize

%----------------------------------------------------------------
\Class{vTimer}

\footnotesize
\begin{verbatim}
//  vTimer:  "#include <v/vtimer.h>"

    class vTimer
      {
      public:   //---------------------------------------- public
        vTimer();
        virtual ~vTimer();

        virtual int TimerSet(long interval);
        virtual void TimerStop(void);
        virtual void TimerTick(void);
      };
\end{verbatim}
\normalfont\normalsize

%----------------------------------------------------------------
\Class{V Utilities}

\footnotesize
\begin{verbatim}
//  V Utilities:  "#include <v/vutil.h>"

    extern void LongToStr(long intg, char* str);
    extern void IntToStr(int intg, char* str);
    extern void vGetLocalTime(char* tm);
    extern void vGetLocalDate(char* tm);
\end{verbatim}
\normalfont\normalsize

%----------------------------------------------------------------
\Class{vWindow}

\footnotesize
\begin{verbatim}
//  vWindow:  "#include <v/vwindow.h>"

    enum WindowType
      { WINDOW, CMDWINDOW, DIALOG };  // Types of windows

    class vWindow
      {
      public:           //-------------------------------- public
        vWindow(char *name = "+", int width = 0, int height = 0,
           WindowType wintype = WINDOW);    // Constructor

        virtual ~vWindow();                 // Destructor

        // Methods to Override

        virtual void KeyIn(vKey keysym, unsigned int shift);
        virtual void MenuCommand(ItemVal id);
        virtual void WindowCommand(ItemVal id, ItemVal retval, CmdType ctype);

        // Utility Methods

        const Widget vHandle();         // X only - native handle
        const char* name();             // the name set
        virtual void ShowWindow(void);
        WindowType wType();          // what kind of window we are

        virtual void AddPane(vPane* add_pane);
        virtual void CloseWin(void);
        virtual int GetValue(ItemVal id);
        virtual void RaiseWindow(void);
        virtual void SetString(ItemVal id, char* str);
        virtual void SetStringAll(ItemVal id, char* str);
        virtual void SetTitle(char* title);
        virtual void SetValue(ItemVal id, int val, ItemSetType setType);
        virtual void SetValueAll(ItemVal id, int val, ItemSetType setType);
        void ShowPane(vPane* wpane, int OnOrOff);
      };
\end{verbatim}
\normalfont\normalsize
%----------------------------------------------------------------
\Class{vYNReplyDialog}

\footnotesize
\begin{verbatim}
//  vYNReplyDialog:  "#include <v/vynreply.h>"

    class vYNReplyDialog : protected vModalDialog
      {
      public:     //---------------------------------------- public
        vYNReplyDialog(vBaseWindow* bw, char* title = "Yes or No");
        vYNReplyDialog(vApp* aw, char* title = "Yes or No");
        ~vYNReplyDialog();

        int AskYN(const char* msg);
      };
\end{verbatim}
\normalfont\normalsize


%***********************************************************************
%***********************************************************************
%***********************************************************************

\chapter{V Programming Tools}
\index{V Programming Tools}

\section{The V Application Generator}

The \V\ Application Generator will automatically generate
C++ code needed to build a simple \V\ application. It has
several options that let you specify the name of the
application, the name of your derived classes, and what
\V\ interface elements to include in the application.
The \V\ Application Generator, \code{vgen}, does not generate code that
does any real work, it just provides a very good starting
skeleton program to build your application.

In the past, the approach for beginning a new \V\ application
has been to copy one of the example programs, and modify
it. \code{Vgen} has the advantage of allowing a certain amount
of customization with names and interface elements included
in the basic skeleton program.

On Linux or other Unix platforms, \code{vgen} will generate
the skeleton code and a makefile compatible with GNU make.
On the Windows version, just the files will be generated,
and it is up to you to include them in the project file
of your compiler. This is usually a very trivial operation.

Once you have generated and compiled the skeleton application,
you can modify the code to build your own application. It
is highly recommend that you start every new \V\ application
this way to get a consistency in the structure of the code.

\subsection{Overview}

\code{Vgen} is a very simple program to use. You run it,
and then select if you are generating a standard \V\
application skeleton, or if you are generating an
extra dialog skeleton to add to an existing application.
You can also generate a skeleton for a menuless and
canvasless \V application. You can also specify the
directory where the generated code should be placed.
Once you have generated the skelton, you should get it
to compile, either by using the generated makefile, or
by building an appropriate project with a Windows C++
compiler.

The most common use of \code{vgen} is to generate a standard \V\
application skeleton. This consists of a derived \code{vCommandWindow}
class with a simple menu bar, a command pane with a sample
button, a derived \code{vCanvasPane}, and a status pane. The
standard \V\ skeleton also allows you to generate a modeless and
a modal dialog. You can specify the name of the classes
you want to use, as well as the file names to use for each
of the generated files. The standard files generated include
a file for the derived \code{vApp} class, a file for the
derived \code{vCmdWindow} class, a file for the derived
\code{vCanvasPane} class, and files for the dialogs.
The X version will also generate a GNU compatible makefile.

\code{Vgen} also will generate extra copies of dialogs.
You can specify the class name of the dialog, and then
generate a skeleton file. These dialogs must be added
manually to the basic skeleton application.

The remainder of the reference manual will explain each
menu item and each dialog.

\subsection{File Menu}

The file menu only has an Exit command, which closes
\code{vgen}.

\subsection{Generate Menu}

The \code{Generate} menu selects which type of code you
want to generate. The \code{vgen} tool bar includes
shortcuts for the most common operations.

\subsubsection*{Generate:Standard Application}

This option brings up a dialog that controls the
generation of a standard \V\ application. This section
will explain each option contained on the Standard
V App dialog.

When \code{vgen} generates a skeleton application, it
uses some fixed conventions for naming derived classes and
file names. The \emph{Application Base Name} input lets
you specify the base name of each class. The default
base name is \code{my}. Thus, \code{vgen} will generate
the derived class names \code{myApp}, \code{myCmdWindow},
\code{myCanvasPane} or \code{myTextCanvasPane}, 
\code{myDialog}, and \code{myModalDialog}.

The \emph{File Base Name} input lets you control
the base name of the generated code files. If you
intend to do development for Windows, it is recommended
that you specify a name that follows the 8 character
limit on file base names. Using the default \code{my}
file base name, \code{vgen} will generate the files
\code{myapp.cpp} and \code{myapp.h}, \code{mycmdw.cpp}
and \code{mycmdw.h}, \code{mycnv.cpp} and \code{mycnv.h},
\code{mydlg.cpp} and \code{mydlg.h}, and \code{mymdlg.cpp}
and \code{mymdlg.h}. If you generate a makefile, it
will be called \code{makefile.my}.

The generate dialog allows you to control which interface
elements are included in the generated code. You can
elect to include a tool bar and a status bar. You can
also include code that shows the date and time on
the status bar if you wish. You can control if the
code generates Windows MDI or SDI model code (this
has no effect on the X code). The command window
class includes a short, standard menu bar that you
can later modify to add your own menu items.
You can also specify a title that will appear in
the app's title bar.

You can generate a canvas pane derived
from a \code{vCanvasPane} or a \code{vTextCanvasPane}.
You can elect to show the vertical or horizontal
scroll bars by default.

You can also control generation of a modal and a modeless
dialog. If you include these, code to activate the dialogs
will be included in the menu bar. You will usually modify
that code to activate the dialogs in a manner needed by
your application. 

You also have the option of generating
a GUN make compatible makefile. The make file needs to
know where the \code{v/include} and \V\ library files
are found on your system. The default \code{vgen}
assumes that these will be located in \code{/usr/local/v}.
There is a variable, \code{HOMEV} in the make file that
sets this path. If \V\ is found in a different place, you
can change it in the generate dialog, or you can change
an \code{ifdef} in the source code and recompile \code{vgen}.

Finally, you can control where the generated files are written.
The \emph{Set Save Path} brings up the standard file selection
dialog for where to save the \code{myapp.cpp} file. That file
and the others will be saved in whatever directory you specify.
If you don't specify a save path, the files will be saved in
the startup directory.

When you have made all your selections, the \emph{Generate}
button will generate the skeleton application.

\subsubsection*{Generate:Extra Dialog}

Many applications need more than one modeless or one modal dialog.
\code{Vgen}'s solution to this is not super sophisticated, but
it is easier than modifying an existing dialog from scratch.
The \emph{Extra Dialog} generate command allows you to generate
extra dialog classes that you can then manually add to your
main application. The dialogs generated are just like the
dialogs that the generate standard app builds, but with
a different base name. The options in this dialog include
set the class and file base names, the title, modal or
modeless, and the save path.

\subsubsection*{Generate:Menuless/Canvasless App}

\V\ supports a menuless, canvasless application. This
menu option allows you to generate a simple application
of this type, which includes only a window with a
tool bar. This option does not generate a menu bar,
a canvas, a status bar, or any dialogs.

\subsection{Future Plans}

I hope that \code{vgen} is only the beginning. It seems
pretty useful as it stands, but a full dialog layout
editor is planned to complete this tool.


\section{The V Icon Editor}
\index{Icon Editor}

The \V\ Icon Editor is a tool used to create and modify
icons. It is intended chiefly to create icons for the various
\V\ controls that use icons. It has many editing features
found in other icon or bitmap editors, but because it is intended chiefly as
an icon editor, it is limited to icons with a maximum size
of 150 pixels square and will easily manipulate up to 64 colors,
although it will display icons with up to 256 colors.
Typically, however, icons tend to be less than 64 pixels square,
and use a limited number of colors.

The current version of the \V\ Icon Editor will handle the native
\V \code{VBM} icon format, as well as \code{XBM} and \code{XPM}
X Windows format files, and Windows \code{BMP} format files,
\index{VBM icon format}\index{XBM bitmap format}
\index{XMP pixmap format}\index{BMP bitmap format}
so files generated by the X and Windows host platforms can
be easily edited and converted to \code{VBM} format.

The \V\ Icon Editor was originally developed as a team project
for the Spring, 1996 Software Engineering class at the University
of New Mexico by Philip Eckenroth, Mike Tipping, Marilee Padilla,
and John Fredric Jr. Masciantoni.  It has been heavily enhanced
by Bruce Wampler. Although this program makes use of many \V\
features, as a largely student project, the quality of the code
is somewhat variable, especially in its use of objects.

\subsection {Overview}

The icon editor functions very much
like other similar programs, and should be easy to use.
This guide is not intended as a complete tutorial, but
more as a brief, but complete, reference.

The \V\ Icon Editor will usually be called \code{viconed}.
It may be started with the name of a file to edit on
the command line.

The interface to the \V\ Icon Editor consists of a standard
menu bar, two tool bars, a drawing canvas, and a status bar.
The most common operations are supported by the tool bars
(which, for the most part, duplicate menu commands).
The drawing canvas shows an enlarged view of the icon as
well as an actual size view of the icon. The enlarged view
may be zoomed to several sizes, and displayed with or without
a grid. The remainder of this guide will describe each
menu command, and other features that can be invoked from
the tool bars.

You draw an icon using one of three types of brushes:
the normal brush, the text brush, and the copy/paste brush.
The brush will draw in one of several shapes. Not all
shapes work with all three brushes, but you can get
some interesting effects using the text or copy/paste
brush to draw a line or rectangle, for example.
The normal brush also has a choice of several sizes and styles.
Drawing with the left mouse button uses the foreground color,
while drawing with the right mouse button selects the background
color. Colors are selected with the color selection dialog.

\subsection {File Menu}

The \emph{File} menu generally includes commands needed to
open, close, and manipulate icon files.

\subsubsection*{File:New}

This will create a new icon using the current canvas. If the
current icon has been changed, you will be asked if you want
to save it. Then you will be prompted for the size of the
new icon. A blank icon of the specified size will be created,
and the color palette initialized to a set of default colors.

\subsubsection*{File:Open}

This command is used to open an existing icon in one of
the supported formats. If the current drawing canvas has
been edited or had an icon loaded, a new canvas window
will be opened. The color palette for the canvas window
will be initialized to the colors used in the opened icon.

The format of the icon is determined by the file extension.
\code{VBM} is the native \V bitmap format, and is the format
required by the various \V icon controls.  The current version
only supports the 1 and 8 bit \code{VBM} formats. \code{Viconed}
also supports the X Windows \code{XBM} monochrome bitmap files,
and \code{XPM} color pixmap files (up to about 90 colors).
The Windows \code{BMP} bitmap format is supported for
8-bit bitmaps. All icons are limited to a maximum of
150 by 150 pixels.

\subsubsection*{File:Save}

This will save the current icon. If the icon was new, you
will be prompted for a file name.

\subsubsection*{File:Save as}

You will be prompted for a name to save the current icon.
The format of the saved icon is determined by the file
extension. \code{VBM} specifies the standard \V\ icon
format. \code{Viconed} will automatically save either
the monochrome 1-bit \code{VBM} format, or the 256
color mapped 8-bit \code{VBM} format. The 24-bit
\code{VBM} format is not supported. The other formats
supported include X \code{XBM} monochrome bitmaps, and
X \code{XPM} color pixmaps up to about 90 colors.
The Windows \code{BMP} bitmap format is supported
for 8-bit bitmaps.

When \code{viconed} saves an icon, it will minimize
the size of the color map used in the file.

\subsubsection*{File:Close}

This will close the current icon, asking you if
you want to save it if it has changed, and then clear
the drawing canvas, ready to create a new icon, or
open another icon.

\subsubsection*{File:About}

This displays a dialog with information about \code{viconed}.

\subsubsection*{File:Exit}

All open icons will be closed, with save prompts as needed,
and \code{viconed} will exit.

\subsection {Edit Menu}

The current version of \code{viconed} does not yet support
standard cut, copy, and paste operations. (It does have
copy/paste brush support described later.)

\subsubsection* {Edit:Undo}

This will undo the last operation that changed the icon.
Only one level of undo is supported.

\subsubsection* {Edit:Clear}

This will clear the current icon to the background color.
A clear is \emph{not} undoable!

\subsubsection* {Edit:Resize}

You can resize the existing icon to a new size. The upper
left corner of the current icon will remain constant.
If you specify a smaller icon, you will lose the
lower and right portions. If you specify a bigger size,
then the current icon will become the upper left corner
of the new icon.  You may find the copy/paste brush
useful when resizing an icon.

\subsection {Draw Menu}

The Draw menu is used to select the shape of the brush.
The normal brush will draw the selected shape using
the current normal brush style. The text brush will
draw the shape using the current text. The copy/paste
brush will draw the shape using the copied shape.

Using the left button selects the foreground color,
while the right button selects the background color.

The shape selections are duplicated on the tool bar
for easy interaction.

\subsubsection* {Draw:Point}

This draws a single point, or instance of the text
or copy/paste brush. Holding the mouse button down and
moving will draw a series of points.

\subsubsection* {Draw:Line}

The button press selects the starting point of the
line, and the release selects the end point.

\subsubsection* {Draw:Rectangle}

The button press selects the first corner of the rectangle,
and the release the opposite corner. If snap is on,
then this will draw a square.

\subsubsection* {Draw:Rounded Rectangle}

This is a rounded rectangle instead of a square cornered
rectangle.

\subsubsection* {Draw:Ellipse}

This draws an ellipse, or a circle with snap on.

\subsubsection* {Draw:Pick Color}

This lets you pick a color from the current icon. Pressing
the left button will make the color under the cursor
the current foreground color, while the right button
will pick the background color. Using the pick color
(a dropper icon on the tool bar) is often easier than
using the color selection dialog.

\subsubsection* {Draw:Fill}

This will fill the closed area with the foreground or
background color depending on the mouse button pressed.

\subsubsection* {Draw:Refresh Image}

Normally, this command should not be needed, but it will
cause the icon to be redrawn.

\subsubsection* {Draw:Show Grid}

This will turn on or off the display of the drawing grid.

\subsection {Brush Menu}

These select the type of brush to use. Brush selection
is duplicated on the tool bar.

All three brushes actually use the same mechanism -
a general brush that can hold a pattern to draw
onto the icon. A regular brush is usually a pattern
of a single pixel, but can be any of the patterns
supported by the brush style dialog. The text
brush uses text to make the pattern. The copy/paste
brush set the pattern based on a selection
from the current icon.

You can get some interesting effects by using different
brush shapes (point, rectangle, etc.) when drawing with
any of the brushes. Using the point shape and then dragging
with the mouse held pressed can yield shadow effects,
for example.

\subsubsection* {Brush:Regular Brush}

The regular brush draws the currently selected shape
using the current regular brush style. The styles
include a single pixel point, as well as square, line, and
circular shapes of various sizes. The regular brush
style is selected from the brush dialog, which is
toggled on and off from the tool bar.

\subsubsection* {Brush:Text Brush}

When you select the text brush, you will be prompted
for some text, which will then become the brush.
You can then position the text, and press the right or
left mouse to draw the text in the icon.

Currently, only upper and lower case alphanumerics are
supported, and some of the letters don't quite look right.

\subsubsection* {Brush:Copy/Paste Brush}

Right after you select the copy/paste brush, you will
need to select an area of the current icon to "copy".
This then becomes the brush, and you then draw the brush
into the icon by pressing the right or left mouse buttons.


\subsection{Zoom Menu}

\code{Vicone} will display the icon zoomed from two up to
32 times the size of the actual icon. Use the zoom menu to
select the zoom factor.

\subsection{Tool Bars}

There are two tool bars in the code{viconed} interface.
The first tool bar shows the current foreground and background
colors on the left side. The next icon on this tool bar is
the "snap" toggle. When pressed, drawing with the rectangle
brush shape will draw squares, and the ellipse shape will draw
circles. The next icon is the brush toggle, and will display
the brush style dialog. The next icon is the color selection
toggle, and will display the color selection dialog. These three
toggles do not have corresponding entries on the menus.
The right end of the first tool bar show three toggles to
select the regular, text, and copy/paste brushes.

The second tool bar contains buttons for clear and undo,
as well as toggles for selecting the brush shape. All these
are duplicates of menu commands.

\subsubsection*{Color Selection Dialog}

The color selection dialog is used to pick and select foreground
and background colors. Internally, \code{viconed} uses 256 colors
for each icon pixel. Depending on the color resolution of your
display, all 256 colors may or may not be available. Typically,
icons do not use very many colors, so this shouldn't matter.

The color selection dialog shows a large color square
showing the current selection color. Two buttons next to the
current color square are used to apply the current selection
color to the foreground or background color.

Below that is a palette of 64 small color buttons arranged in
4 rows of 16. Selecting one of these buttons make it the
current selection color. \code{Viconed} supplies 64 default
colors for new icons. Again, depending on the color resolution
of you color display, these may or may not show as 64 colors.
When a new icon is loaded, its colors are used to load the
color selection color palette. Most icons won't use 64 colors,
and unused colors are filled with black.

Below the palette are three sliders that can be used to
change the color. Select a palette button to change, then
use the sliders to adjust the red, green, and blue. You
can also press the small red, green, or blue button
next to the sliders to enter a specific value for that
color component. The reset button will reset the palette
entry back to its original color.

Note that even though the color selection dialog only
has 64 entries, the colors shown do not necessarily have
any relation to the colors used in the icon. The colors
in the icon are set by the foreground and background
colors. You can use the sliders to specify any color,
and then apply it to the foreground or background color.
The color selection dialog allows you to easily
pick any one of the 64 palette colors.

Use only standard, basic colors in icons (black, white, red,
green, blue, etc.) to minimize the impact on the color maps used
on systems with color resolutions of 256 or less.


%***********************************************************************
%***********************************************************************
%***********************************************************************

\chapter{Release Notes}
\index{release notes}

\V\ is usually found in one of several archives: \code{v-1.00.tar.gz},
the full distribution of all current versions of V; \code{vx-1.00.tar.gz},
the X only distribution; and \code{vwin100.zip}, the MS-Windows only distribution.
Note the version (1.00) will change as V is updated.

\section {X Window System}
\index{X Window System}

The current X implementation of \V\ uses the Athena widget set with
some modified versions of some widgets from the Xaw3d widget set. A Motif
port had been started, but with the current state of the 3D look of
\V, I've decided to postpone the Motif port. It is about three-fourths
complete.

\subsection*{Directories}

The \V\ directory structure has been designed to allow you to either
install \V\ in a personal directory, or at a higher system level.
By defining an appropriate search path in the \code{Makefile}, your
applications can find the required \V\ files.

You should download the standard distribution file
\code{vx-1.00.tar.gz}\footnote{Or whatever the latest version is...},
which is a \code{gzip} GNU Zipped tar file (or the full combined version,
\code{v-1.00.tar.gz}). After you unzip the file
(with \code{gunzip}, or the -z switch on some versions of \code{tar}),
you can extract the file to wherever you want it. It will build
a \code{/v} subdirectory from the directory containing the \code{vx-1.00.tar}
file. The file hierarchy is:

\begin{description}
\item[/v] The main \V\ directory.

\item[/v/appgen] V application generator program.

\item[/v/bccide] MS-Windows build files for Borland C++. (Not included
in X only distribution.)

\item[/bmp2vbm] Source for a simple MS-Windows and OS/2 \code{.bmp} bitmap
format to \code{.vbm} V bit map format converter.

\item[/v/bin] The \code{/bin} directory is used to hold the binaries
of \V\ sample programs. No binaries are included on the distribution,
but at least the subdirectories \code{/intel} for Linux and \code{/sun4} for
Suns are included. There may be other subdirectories for other architectures.

\item[/v/doc] The sources for \V\ documentation. The documentation
is in \LaTeX format. The standard distribution includes only the source
\LaTeX files. Other versions, including dvi and Postscript, are available from
the \V\ ftp site.

\item[/v/draw] Source for the VDraw example program. Examples that
are identical across platforms use a \code{.cpp} file extension.

\item[/v/drawex] Very simple V draw example from C/C++ Users Journal
article.

\item[/v/examp] Source for a simple \V\ example.

\item[/v/includex/v] Source for the X \code{*.h} \V\ header files.

\item[/v/includew/v] Source for the MS-Windows \code{*.h} \V\ header files.
(Not included on X only version.)

\item[/v/lib] Compiled version of the \V\ library will be placed
under appropriate subdirectories here. No precompiled libraries are
included in the standard distribution.

\item[/v/msvc] Project and make files for Microsoft VC++.

\item[/v/obj] Compile object code is saved under here.

\item[/v/srcx] The full C++ source for the X \V\ library. The files
use a \code{.cxx} extension.

\item[/v/srcwin] The full C++ source for the Ms-Windows \V\ library. The files
use a \code{.cxx} extension. (Not on X only version.)

\item[/v/test] The test program used to test \V\ functionality.

\item[/v/texted] Source for a simple editor based on vTextEdit class.

\item[/v/tutor] The source code for the tutorial example.

\item[/v/vxgl] Source for V canvas class that interfaces to X
version of OpenGL.

\item[/v/watcom] Project and make files for Watcom C++ compiler.

\end{description}

\subsection* {Compilers}

The \code{makefile} provided with \V\ uses the GNU C++ compiler,
\code{g++}. \V\ does not use templates or other C++ features that
can cause portability problems. The current version has been
built and tested using \code{g++} Version 2.7.2, although it
did work with Version 2.6.3, but
not earlier versions. There is no inherent reason that \V\ should
not compile with other C++ compilers.

\subsection*{The X Makefile}
\index{makefile}

The \code{Makefile} is the main way to build X versions of \V.
It has comments that should help you to build the X version of \V.
See the file README.TXT for more instructions for installing \V
on a *nix platform. All of the customizations for a given platform
have been isolated into the file \code[Config.mk].

To date, \V\ has successfully been compiled on several platforms,
including Linux, SunOS, Solaris, AIX, SGIs, and DEC Alphas. The standard distribution
includes a \code{Makefile}
that can be easily configured for several platforms. The makefile requires
GNU make! The secret is to examine \code{Config.mk} and add and modify
the definitions at the beginning as needed for your platform. (For Linux,
this will usually be a no op, since Linux is the default configuration.)
Examine the definitions already there, and then add a section with
the locations defined as needed for your platform. Then use an \code{ARCH=}
definition on the \code{make} line (or make your platform the default.)

\subsection*{X Resources}
\index{X Resources}

\V\ makes limited used of X resources. The main use is to define
the basic color schemes for controls and dialogs. The following
resources are used:

\begin{description}

\item[vDialogBG] The color used for the background of dialogs and
command bars.

\item[vStatusBarBG] The color used for the background of the status
bar.

\item[vMenuBarBG] The background color of the menu bar and menu drop
downs.

\item[vControlBG] The background color for some controls, such as
sliders and scroll bars.

\item[vControlFace] The color used for the faces of various
controls such as buttons.

\item[vLightControlShadow] The color used for the light shadow
on 3D controls.

\item[vDarkControlShadow] The color used for the dark shadow on
3D controls.

\end{description}

By varying just the above X resources, you can really change
the visual look of your \V\ app.
The \code{/v/srcx} directory contains several files of the
form \code{vRes*} that contain various color schemes.
The default color scheme is contained in \code{vResDefault}
(but you don't need to load it -- it is the default).
The file \code{vResBlueMtf} contains the color scheme similar
to Motif. This is the contents of \code{vResDefault}:

\begin{verbatim}
*vDialogBG: gray75
*vStatusBarBG: gray80
*vMenuBarBG: gray70
*vControlBG: gray80
*vControlFace: gray70
*vLightControlShadow: gray87
*vDarkControlShadow: gray50
\end{verbatim}

To use one of these, or your own, resource files, you can
use the command \code{xrdb -merge vResColorscheme}. You
can also add the lines to your \code{.Xresources} file.

The X program name is the name you supply to the \code{vApp} constructor.

\subsection* {X Bugs}

The PostScript print driver does not draw shapes with hatched
brushes.

The PostScript drawing canvas does not support \code{CopyFromMemoryDC}.

Source code uses two naming conventions - \code{.cxx} and
\code{.cpp}. Gnu g++ version 2.6 and later support both
file extensions. G++ version 2.5 doesn't like \code{.cpp},
so you might have to rename those files to \code{.cxx},

\section{Microsoft Windows}
\index{Microsoft Windows}\index{Windows}

The current implementation of \V\ for MS-Windows is for Windows 3.1
and WIN32 (Windows 95 and NT). We will refer to this version as \code{Vwin} in this
description. Eventually some native Windows 95 controls will
be used, but for now the WIN32 version works fine for Win95.
The Windows version of \V\ is available at as 
\code[vwin100.zip] (a MS-Windows Zip format file). on the \V\ ftp
site. You will need a version of \code{ZIP} or \code{gunzip} and
\code{tar} to extract \V. The MS-Windows version does not include
documentation. Documentation is available as both \LaTeX files in
the X distribution, or split into a pair of Postscript or
DVI files on the \V\ ftp directory.

\subsection* {Directories}

The directory structure of \V\ under MS-Windows is similar to the X version.
On the distribution, the MS-Windows hierarchy is found under the \code{/v}
directory. (We will use Unix / notation for files instead of the
usual MS-Windows backslash notation. Most MS-Windows compilers handle
the / correctly, and / is used throughout the \V\ source files.)
When you unzip the archive, a subdirectory \code{/v} will be
built.

Under \code{/v} are \code{/bin/win} for the example \V\ MS-Windows
binaries, \code{/draw} for the VDraw example program, \code{/examp}
for a simple example program, \code{/includew/v} for the \V\
\code{.h} header files, \code{/lib/win} for the MS-Windows compiled
library, \code{/obj/win*} for the object files, \code{/srcwin}
for the MS-Windows version of the source code, \code{/test} for the
test driver program, and \code{/tutor} for the source code to the
tutorial included in this reference manual.

For MS-Windows, the \V\ library source files use a \code{.cpp} extension.
The example programs also use \code{.cpp}. The source for most of the
example programs is identical for the MS-Windows and X versions! However,
the source for the library \code{.cpp} and \code{.h} files are different
for each platform, so you must be careful not to mix the X and MS-Windows
versions of source code and header files.

\subsection* {Compilers}

\V has been successfully been compiled using Borland C++ 4.5 for
Win3.1 and WIN32; Borland C++ 5.02 for WIN32; Watcom 10.6 for Win3.1 and WIN32;
the GNU-WIN32 gnu g++ compiler; and Microsoft VisualC++ under several versions.

Several Borland \code{.ide} files are included on the directory 
\code{/vwin/bccide}. The \code{.ide} files assume \V\ is built on
drive C:, so you may have to modify it if you want to build \V\
on your own system. If you are using another compiler, then you
need to compile \emph{every} \code{.cpp} file found on the
\code{/srcwin} directory.

Project files for compiling with Watcom C++ are included in
the directory \code{v/watcom}. Unlike the Borland versions,
the object code and libraries are built directly on these
\code{watcom} directories.

\subsection* {MDI/SDI Models}
\index{MDI model}

\V\ for MS-Windows supports both the MS-Windows MDI and SDI models. 
By default, \V\ uses MDI, and will bring up the main MDI window,
and open the first MDI child window. There currently is no way to
have a main MDI window with no active MDI child windows -- when
you exit the last window, the application closes. The menu,
command bar, and status bars will change to the ones defined by
each child window as each child window is activated.

\V\ will automatically append a \code{Window} menu item to the main
menu. The built in \code{Window} menu supports the standard cascade
and tile MDI operations, as well as showing a list of MDI children.

\index{SDI model}
You can also get MS-Windows applications to look like the standard
SDI model. If you want an SDI app, you control this in the
static declaration of the \code{vApp} object:

\begin{verbatim}
  static testApp* tApp = new testApp("Vtest",1);
\end{verbatim}

The second parameter controls MDI or SDI. A default parameter
is defined by \V as 0 to indicate the MDI model. If you specify
a 1, then \V will take an SDI look. It actually does this
by using the MDI code, but maximizing the canvas window,
removing the extra buttons from the menu bar, and not
adding the \code{Window} menu. It is impossible for the
user to tell that this is really an MDI application, but
\V does not strictly enforce this. If you create more than
a single \code{vCmdWindow} object, unpredictable things will
happen under the SDI simulation. It is up to you to not do that.

Since X doesn't have an MDI/SDI equivalent, it is harmless
to specify SDI to an X version of your app.

\subsection* {Icons}

As stated in the main part of this manual, \V\ does not use resource files.
This is true for the MS-Windows versions. However, there is one reason you
might want to include a \code{.RC} file with a \V\ MS-Windows application,
and that is to allow you to define the icons used with the application.
(These are MS-Windows icons, and are \emph{not} the same things as
\code{vIcons}.)

Typical MS-Windows MDI applications use two icons - one for when the whole application
is iconized, and one when each child window is iconized. If you don't supply a
\code{.RC} file, you will get the default MS-Windows icons. The \V\ distribution
supplies two default icons of its own, called \code{vapp.ico} and \code{vwindow.ico}.
By including the definitions \code{vAppIcon} \code{ICON} \code{vapp.ico} and
\code{vWindowIcon} \code{ICON} \code{vwindow.ico} in the \code{.RC} file,
\V\ will load and use those icons for the application and each child window
respectively. You can substitute whatever two icons you want for your application
by specifying different \code{.ico} files for
the \code{vAppIcon} and \code{vWindowIcon} names in the \code{.RC} file.

\subsection* {DEF File}

MS-Windows applications are typically compiled using a \code{.DEF} file. You
can modify any of the \code{.DEF} files included with \V\ sample programs.

\subsection* {Bugs on MS-Windows}

\section{V Versions}

\begin{description}
\item[Version 1.00]
This version was local to the University of New Mexico on January 10, 1996.
Versions 1.01, 1.02, and 1.03 were local maintenance releases.

\item[Version 1.04]
This was the first major public release of \V, and was announced
to the world on February 14, 1996.

\item[Version 1.05]
This version had several bug fixes obtained from feedback of the
public release.

\item[Version 1.06]

This was an X only release, and added 3D controls.

\item[Version 1.07]

This release was never formally announced, and included some
of the changes listed for version 1.07.

\item[Version 1.08]

The 4/15/96 release added several significant features to \V:

\begin{itemize}

\item The \code{vMemoryDC} drawing canvas, including new methods
\code{CopyFromMemoryDC} and \code{DrawColorPoints}.

\item Internal revisions for handling of color, including adding
\code{vColor::ResetColor} to allow reuse of color maps, and 
\code{vColor::BitsOfColor} to get color capability. These
revisions allow \V apps to make more effective use of default
color maps.

\item \code{vPen::SetColor(r,g,b)} and \code{vBrush::SetColor(r,g,b)}
are being dropped in favor of the \code{vColor} forms. These calls
break the hidden management of color maps, and while still included
in the code, should \emph{not} be used. Support for the \code{(r,g,b)}
form will be dropped entirely in future versions of \V.

\item \code{C\_ToggleIconButton} was added to allow a pressed in
button interface look in place of check boxes and radio buttons.

\item Documentation for \code{ChangeColor} and \code{C\_ColorLabel}
was added, although the functionality has been there for a while.

\item The WIN32 port was finished, and the X and MS-Windows versions
are now in sync.

\end{itemize}

\item[Version 1.09]
Added \code{C\_ToggleButton} and
\code{C\_ToggleFrame} controls. It also includes a large
number of \V icons suitable for building command pane
tool bars.

\item[Version 1.10]
The 5/29/96 release of \V\ includes the following enhancements
and changes:

\begin{itemize}

\item The \V\ Icon Editor - an icon editor to
create icons for various \V\ controls.

\item Inclusion of accelerator key support in menus.

\item Addition of the \code{ChangeListPtr} set type to allow
completely dynamic lists, combo boxes, and spinners.

\item Addition of \code{DrawLines}, \code{DrawPoints}, and
\code{DrawRectangles} to \code{vCanvas}.

\item Several bug fixes for both MS-Windows and X, some relatively
major.

\item The canvas page scroll messages were changed
on the X version to correspond to the behavior of the MS-Windows version.
A page scroll message is sent only at the completion of
a scroll, not continuously as before. It is usually rather
difficult to implement nice continuous scrolling, so this
approach seems more useful to more people. This is the only
known change that might affect compatibility with previous
\V applications.

\item Addition of a directory for outside contributions.

\end{itemize}

\item[Version 1.11]

The 7/4/96 release of \V\ has several minor bug fixes for the
MS-Windows and X versions. It also adds the \code{WorkSlice} methods
to support applications that require computations to continue
even if the user is not entering commands to the application.

\item[Version 1.12]

This was a bug fix release for MS-Windows. The X version was
unchanged, but renumbered for consistency.

\item[Version 1.13]

This 8/24/1996 release of \V\ is a major release with several
new features, and some significant bug fixes that can change
the behavior of existing \V applications. The following includes
a list of changes:

\begin{itemize}

\item The \V\ Application Generator, \code{vgen} is now included
with the standard distribution. It will generate a simple
\V\ application as a starting skeleton for new apps.

\item The values being passed by \V\ to \code{vCanvasPane::VPage} and
\code{HPage} were incorrect on the X version. The documentation states
that the values for \code{Top} should be in the range 0 to 100.
The MS-Windows version worked correctly, but the X version was
passing a range of 0 to (100-Shown). This bug actually has
been in the X version since the switch to 3D Controls. With
version 1.13, both MS-Windows and X work the same.

\item The MS-Windows version of \code{vDC::DrawText} was fixed to
work according to the documentation. It had been drawing text
with the x,y as the upper left corner of the text. Beginning
with 1.13, it now draws at the lower left corner as specified
in the documentation.

\item Two functions, \code{GetHScroll} and \code{GetVScroll},
were added to \code{vCanvasPane} to make dealing with
scroll bars easier.

\item A new standard using \code{enums} for generating IDs for
controls has been adopted beginning with 1.13.

\item \code{C\_TextIn} controls now allow you to specify the
width of the control in characters using the \code{size} field.
This is described in Chapter 6.

\item In \code{vCanvasPane}, new parameters (with default
values for backward compatibility) were added to
\code{CopyFromMemoryDC} to allow subregions to be copied.

\item Using a \code{vTransparent} pen when drawing text
now results in leaving the existing background when drawing, and
a \code{vSolid} pen overwrites with the current background
color.

\item There was a conflict on MS-Windows with using \code{VK\_} for
key names. The MS-Windows version was changed long ago, and now
the X version also uses lower case letters (e.g., \code{vk\_Tab}).


\end{itemize}

\item[Version 1.14]

The major addition to 10/6/96 \V Release 1.14 is the addition of
the \code{vTextEditor} class, which is a very good first pass at
a complete editing canvas. The editor is complete, can be
extended to support custom command sets or file management.
It is missing cut, copy, and paste, which will be implemented
as general support for these is added to \V. The code for
\code{vTextEditor} is based on \code{vTextCanvasPane}, and
is identical for the X and MS-Windows versions.

Also, for the X Version, support for OpenGL has been added.
This support is found in the distribution directory
\code{v/vxgl}. While the \V OpenGL canvas pane seems
very robust, it is still somewhat experimental. I would
like any feedback on its use and design.

Other changes, mostly bug fixes, include:

\begin{itemize}

\item X version: The little close button on the left of the
menu bar has been dropped by popular request. It seems most people
didn't like it. If you do, you can still get it by defining
the symbol USE\_CLOSE\_BUTTON. Instead, \V now supports the
X WM\_DELETE\_WINDOW protocol. This protocol is supported
slightly differently by different window managers, but
accomplishes the same thing as the old close button.

\item X version: There was a minor bug in how the
scroll bars worked when \code{top == 0}.

\item X version: The method used to get the size of a window
was changed, and should now give correct values.

\item X version: There was a bug in drawing radio buttons
that only showed up on some systems.

\item X version: There was a bug in changing the current
selection in combo boxes.

\item X version: There was a bug in setting colors for the
PostScript DC.

\item X and MS-Windows: There were several bugs in
\code{vTextCanvasPane} exposed by the implementation
of \code{vTextEditor}.

\item X version: There was a bug in the key mapping that would
cause a program to terminate if an unrecognized key was pressed.

\item MS-Windows: The method to determine the size of the MDI
frame and client windows was improved (I hope!).

\item MS-Windows: A bug with the work timer was fixed. The interaction
between the work timer, check events, and the MS-Windows message
loop was changed to work better.

\item MS-Windows: The argument order of ClearRect was fixed to correspond
to X and the documentation.

\item MS-Windows: There was a bug that didn't allow SetValue to work
correctly for some controls.

\item MS-Windows: A bug in handling the MS-Windows caret in text
canvases was fixed. This one was a bit subtle, but nasty in
possible side effects. Also, EnterFocus and LeaveFocus did not work correctly.

\item MS-Windows: A bug in setting text colors on NT and Windows 3.1 was fixed.
The bug did not manifest itself on Windows 95.

\end{itemize}

\item[Version 1.15]

Release of V Version 1.15 has some non-backward compatible
changes. In previous versions of V, there were inconsistencies in
the order of width and height parameters. These have all been now
changed to consistently use a width/height order. (Except for
vIcon, which still use height/width.) The decision to fix this
order came from a general consensus of the V mail list.

You will need to change your code to reflect the new changes.
The following things must be changed:

1. Any calls to the constructor of a base or derived vCmdWindow
will need the width and height order swapped.

2. Calls or overrides of vApp::NewAppWin need the
order of width and height swapped.

3. Calls to vCanvasPane::SetHeightWidth(h,w) need to be
changed to vCanvasPane::SetWidthHeight(w,h).

4. Calls or overrides of all versions of Redraw(x,y,h,w)
need to be changed to Redraw(x,y,w,h).

5. Calls or overrides of all versions of Resize(h,w)
need to be changed to Resize(w,h). (The vTextCanvas
row/column versions retain their row/column order.)

Also, the makefiles have been revised for more flexible
building on different *nix platforms.

A new method, vDialog::DialogDisplayed has been added to
allow dynamic setting of dialog control values.

\item[Version 1.16]

Version 1.16 has no significant changes in V functionality.
It mostly has some bug fixes. The only major change is
the release of a completely new set of Makefiles for
the Unix version. These new makefiles were contributed by
a V user, and are much cleaner than the old versions.

A summary of the changes:
\begin{itemize}
\item A small change to the code generated by vAppGen.
\item A fix to scrolling in the V Icon Editor.
\item Some changes to the v_defs.h file for MS-Windows, including
compatibility changes needed for Microsoft VC++. Project files were
added for MSVC++.
\item The == and != operators for brushes, fonts, pens, and colors were
changed to use reference parameters consistently.
\item Various minor changes to enhance compiler compatibility, both
on MS-Windows and X.
\item VReply was fixed to work over multiple shows.
\item A void* was added to vAppWinfo.
\item Vmemdc had height and width switched.
\item Sizing of buttons on MS-Windows was fixed for Windows 95.
\item A resource leak was fixed for MS-Windows.
\item A major bug that showed up only under Microsoft VC++ was
fixed.
\item Initialization of text in strings was fixed for MS-Windows.
\item Changing the values of radio buttons on MS-Windows now works.
\item Spinners now honor the size specification.
\item A tab keystroke now works correctly on MS-Windows.
\item A bug in 
\item Various new tests were added to the test program.
\item A couple of bugs were fixed in the X OpenGL V interface.
\end{itemize}

\end{description}

\section {Changes to previous version}

The file v15to16.patch has been supplied to update only the
source files differences. It does not include the new make files,
nor changes to the documentation. The old make files should work
if you want to upgrade to 1.16 this way.

\section {Future Plans}

\begin{itemize}

\item An OS/2 Port is underway by volunteers. No anticipated completion
date is known. This project is not making good progress at the moment.

\item A very nice, simple C++ Class browser is done, but
needs a little polishing before release. 

\item The Dialog Designer is on hold, but a great new
application generator is now included.

\item \V still needs a class to portably interact with the standard system
clipboards.

\end{itemize}

%***********************************************************************
%***********************************************************************
%***********************************************************************
\chapter{V Class Hierarchy}

The following figure is shows the internal organization of the \V\
class hierarchy. Note that boxes with a double line edge denote
classes that
have object instances, while boxes with single line edges are abstract
classes used to build subclasses.  Most of the time, you won't care
about these abstract classes.

Also note that the classes derived from \code{vCmd} represent the
classes used to implement command objects. Normally, you won't need
to use these classes directly. Instances of these objects are generated
by \V\ from your \code{CommandObject} declarations when you call
\code{AddDialogCmds}.
\newpage
\begin{center}
{\Large The V GUI Class Hierarchy}
\end{center}

\begin{center}
\vspace{.1in}
\small
\setlength{\unitlength}{0.012500in}%
\begin{picture}(405,590)(15,230)
\thicklines
\put( 80,765){\framebox(80,50){}}
\put( 75,760){\framebox(90,60){}}
\put( 80,800){\line( 1, 0){ 80}}
\put( 80,780){\line( 1, 0){ 80}}
\put(105,805){\makebox(0,0)[lb]{\smash{\SetFigFont{10}{12.0}{rm}vApp}}}
\multiput(190,735)(0.40000,0.40000){26}{\makebox(0.4444,0.6667){\SetFigFont{7}{8.4}{rm}.}}
\multiput(200,745)(0.40000,-0.40000){26}{\makebox(0.4444,0.6667){\SetFigFont{7}{8.4}{rm}.}}
\put(210,735){\line(-1, 0){ 20}}
\put(215,725){\makebox(0,0)[lb]{\smash{\SetFigFont{10}{12.0}{rm}1}}}
\put(215,750){\makebox(0,0)[lb]{\smash{\SetFigFont{10}{12.0}{rm}1,N}}}
\put( 40,670){\framebox(80,50){}}
\put( 35,665){\framebox(90,60){}}
\put( 40,705){\line( 1, 0){ 80}}
\put( 40,685){\line( 1, 0){ 80}}
\put( 65,710){\makebox(0,0)[lb]{\smash{\SetFigFont{10}{12.0}{rm}vFont}}}
\put(160,670){\framebox(80,50){}}
\put(160,705){\line( 1, 0){ 80}}
\put(160,685){\line( 1, 0){ 80}}
\put(170,710){\makebox(0,0)[lb]{\smash{\SetFigFont{10}{12.0}{rm}vBaseWindow}}}
\put(200,650){\oval( 20, 20)[tr]}
\put(200,650){\oval( 20, 20)[tl]}
\put(190,650){\line( 1, 0){ 20}}
\put( 80,560){\oval( 20, 20)[tr]}
\put( 80,560){\oval( 20, 20)[tl]}
\put( 70,560){\line( 1, 0){ 20}}
\multiput(330,555)(0.40000,0.40000){26}{\makebox(0.4444,0.6667){\SetFigFont{7}{8.4}{rm}.}}
\multiput(340,565)(0.40000,-0.40000){26}{\makebox(0.4444,0.6667){\SetFigFont{7}{8.4}{rm}.}}
\put(350,555){\line(-1, 0){ 20}}
\put(355,570){\makebox(0,0)[lb]{\smash{\SetFigFont{10}{12.0}{rm}1,N}}}
\put(355,545){\makebox(0,0)[lb]{\smash{\SetFigFont{10}{12.0}{rm}1}}}
\put( 40,500){\framebox(80,50){}}
\put( 35,495){\framebox(90,60){}}
\put( 40,535){\line( 1, 0){ 80}}
\put( 40,515){\line( 1, 0){ 80}}
\put( 50,540){\makebox(0,0)[lb]{\smash{\SetFigFont{10}{12.0}{rm}vCmdWindow}}}
\multiput(255,765)(0.40000,0.40000){26}{\makebox(0.4444,0.6667){\SetFigFont{7}{8.4}{rm}.}}
\multiput(265,775)(0.40000,-0.40000){26}{\makebox(0.4444,0.6667){\SetFigFont{7}{8.4}{rm}.}}
\put(275,765){\line(-1, 0){ 20}}
\put(265,785){\oval( 20, 20)[tr]}
\put(265,785){\oval( 20, 20)[tl]}
\put(255,785){\line( 1, 0){ 20}}
\put(205,295){\framebox(80,50){}}
\put(200,290){\framebox(90,60){}}
\put(205,330){\line( 1, 0){ 80}}
\put(205,310){\line( 1, 0){ 80}}
\put(220,335){\makebox(0,0)[lb]{\smash{\SetFigFont{10}{12.0}{rm}vStatusPane}}}
\put(295,585){\framebox(80,50){}}
\put(290,580){\framebox(90,60){}}
\put(295,620){\line( 1, 0){ 80}}
\put(295,600){\line( 1, 0){ 80}}
\put(315,625){\makebox(0,0)[lb]{\smash{\SetFigFont{10}{12.0}{rm}vDialog}}}
\put(145,580){\framebox(80,50){}}
\put(155,620){\makebox(0,0)[lb]{\smash{\SetFigFont{10}{12.0}{rm}vCmdParent}}}
\put(145,615){\line( 1, 0){ 80}}
\put(145,595){\line( 1, 0){ 80}}
\put(260,600){\oval( 20, 20)[tr]}
\put(260,600){\oval( 20, 20)[tl]}
\put(250,600){\line( 1, 0){ 20}}
\put( 40,395){\framebox(80,50){}}
\put( 40,430){\line( 1, 0){ 80}}
\put( 40,410){\line( 1, 0){ 80}}
\put( 65,435){\makebox(0,0)[lb]{\smash{\SetFigFont{10}{12.0}{rm}vPane}}}
\put(160,490){\framebox(80,50){}}
\put(155,485){\framebox(90,60){}}
\put(160,525){\line( 1, 0){ 80}}
\put(160,505){\line( 1, 0){ 80}}
\put(170,530){\makebox(0,0)[lb]{\smash{\SetFigFont{10}{12.0}{rm}vModalDialog}}}
\put(220,555){\oval( 20, 20)[tr]}
\put(220,555){\oval( 20, 20)[tl]}
\put(210,555){\line( 1, 0){ 20}}
\put(200,465){\oval( 20, 20)[tr]}
\put(200,465){\oval( 20, 20)[tl]}
\put(190,465){\line( 1, 0){ 20}}
\put(160,455){\line( 0, 1){  5}}
\put(160,460){\line( 1, 0){ 80}}
\put(240,460){\line( 0,-1){  5}}
\put(165,415){\makebox(0,0)[lb]{\smash{\SetFigFont{10}{12.0}{rm}vYNReplyDialog}}}
\put(165,445){\makebox(0,0)[lb]{\smash{\SetFigFont{10}{12.0}{rm}vFileSelect}}}
\put(165,435){\makebox(0,0)[lb]{\smash{\SetFigFont{10}{12.0}{rm}vNoticeDialog}}}
\put(165,425){\makebox(0,0)[lb]{\smash{\SetFigFont{10}{12.0}{rm}vReplyDialog}}}
\put(200,465){\line( 0,-1){  5}}
\put( 80,375){\oval( 20, 20)[tr]}
\put( 80,375){\oval( 20, 20)[tl]}
\put( 70,375){\line( 1, 0){ 20}}
\put(300,490){\framebox(80,50){}}
\put(300,525){\line( 1, 0){ 80}}
\put(300,505){\line( 1, 0){ 80}}
\put(325,530){\makebox(0,0)[lb]{\smash{\SetFigFont{10}{12.0}{rm}vCmd}}}
\multiput( 70,465)(0.40000,0.40000){26}{\makebox(0.4444,0.6667){\SetFigFont{7}{8.4}{rm}.}}
\multiput( 80,475)(0.40000,-0.40000){26}{\makebox(0.4444,0.6667){\SetFigFont{7}{8.4}{rm}.}}
\put( 90,465){\line(-1, 0){ 20}}
\put( 90,450){\makebox(0,0)[lb]{\smash{\SetFigFont{10}{12.0}{rm}1}}}
\put( 90,480){\makebox(0,0)[lb]{\smash{\SetFigFont{10}{12.0}{rm}1,N}}}
\put(260,460){\makebox(0,0)[lb]{\smash{\SetFigFont{10}{12.0}{rm}vButtonCmd}}}
\put(260,450){\makebox(0,0)[lb]{\smash{\SetFigFont{10}{12.0}{rm}vCheckBoxCmd}}}
\put(260,430){\makebox(0,0)[lb]{\smash{\SetFigFont{10}{12.0}{rm}vFrameCmd}}}
\put(260,420){\makebox(0,0)[lb]{\smash{\SetFigFont{10}{12.0}{rm}vLabelCmd}}}
\put(260,410){\makebox(0,0)[lb]{\smash{\SetFigFont{10}{12.0}{rm}vListCmd}}}
\put(340,460){\makebox(0,0)[lb]{\smash{\SetFigFont{10}{12.0}{rm}vProgressCmd}}}
\put(340,450){\makebox(0,0)[lb]{\smash{\SetFigFont{10}{12.0}{rm}vRadioButtonCmd}}}
\put(260,440){\makebox(0,0)[lb]{\smash{\SetFigFont{10}{12.0}{rm}vComboBoxCmd}}}
\put(340,440){\makebox(0,0)[lb]{\smash{\SetFigFont{10}{12.0}{rm}vSliderCmd}}}
\put(340,430){\makebox(0,0)[lb]{\smash{\SetFigFont{10}{12.0}{rm}vTextCmd}}}
\put(340,420){\makebox(0,0)[lb]{\smash{\SetFigFont{10}{12.0}{rm}vTextInCmd}}}
\put(340,410){\makebox(0,0)[lb]{\smash{\SetFigFont{10}{12.0}{rm}vValueBoxCmd}}}
\put(110,295){\framebox(80,50){}}
\put(105,290){\framebox(90,60){}}
\put(110,330){\line( 1, 0){ 80}}
\put(110,310){\line( 1, 0){ 80}}
\put(125,335){\makebox(0,0)[lb]{\smash{\SetFigFont{10}{12.0}{rm}vMenuPane}}}
\put(300,295){\framebox(80,50){}}
\put(295,290){\framebox(90,60){}}
\put(300,330){\line( 1, 0){ 80}}
\put(300,310){\line( 1, 0){ 80}}
\put(307,335){\makebox(0,0)[lb]{\smash{\SetFigFont{10}{12.0}{rm}vCommandPane}}}
\put( 15,295){\framebox(80,50){}}
\put( 15,330){\line( 1, 0){ 80}}
\put( 15,310){\line( 1, 0){ 80}}
\put( 30,335){\makebox(0,0)[lb]{\smash{\SetFigFont{10}{12.0}{rm}vCanvasPane}}}
\put( 55,280){\oval( 20, 20)[tr]}
\put( 55,280){\oval( 20, 20)[tl]}
\put( 45,280){\line( 1, 0){ 20}}
\put( 15,270){\line( 0, 1){  5}}
\put( 15,275){\line( 1, 0){ 80}}
\put( 95,275){\line( 0,-1){  5}}
\put( 30,260){\makebox(0,0)[lb]{\smash{\SetFigFont{10}{12.0}{rm}vTextCanvas}}}
\put( 55,280){\line( 0,-1){  5}}
\put( 55,290){\line( 0, 1){  5}}
\put(360,355){\oval( 20, 20)[tr]}
\put(360,355){\oval( 20, 20)[tl]}
\put(350,355){\line( 1, 0){ 20}}
\multiput(330,270)(0.40000,0.40000){26}{\makebox(0.4444,0.6667){\SetFigFont{7}{8.4}{rm}.}}
\multiput(340,280)(0.40000,-0.40000){26}{\makebox(0.4444,0.6667){\SetFigFont{7}{8.4}{rm}.}}
\put(350,270){\line(-1, 0){ 20}}
\put(315,280){\makebox(0,0)[lb]{\smash{\SetFigFont{10}{12.0}{rm}1,N}}}
\put(315,265){\makebox(0,0)[lb]{\smash{\SetFigFont{10}{12.0}{rm}1}}}
\put(200,735){\line( 0,-1){ 15}}
\put(140,760){\line( 0,-1){ 10}}
\put(140,750){\line( 1, 0){ 60}}
\put(200,750){\line( 0,-1){  5}}
\put(200,660){\line( 0, 1){ 10}}
\put( 80,560){\line( 0,-1){  5}}
\put(340,555){\line( 0,-1){ 15}}
\put(340,565){\line( 0, 1){ 15}}
\put(210,650){\line( 1, 0){125}}
\put(335,650){\line( 0,-1){ 10}}
\put(335,640){\line( 0, 1){  5}}
\put(225,620){\line( 1, 0){ 35}}
\put(260,620){\line( 0,-1){ 10}}
\put(260,600){\line( 0,-1){  5}}
\put(260,595){\line( 1, 0){ 30}}
\put(220,555){\line( 0,-1){ 10}}
\put(220,565){\line( 0, 1){  5}}
\put(220,570){\line( 1, 0){100}}
\put(320,570){\line( 0, 1){ 10}}
\put(200,485){\line( 0,-1){ 10}}
\put( 40,580){\framebox(80,50){}}
\put( 40,615){\line( 1, 0){ 80}}
\put( 40,595){\line( 1, 0){ 80}}
\put( 80,630){\line( 0, 1){ 20}}
\put( 80,650){\line( 1, 0){110}}
\put( 80,580){\line( 0,-1){ 10}}
\put( 80,570){\line( 0, 1){  5}}
\put( 80,395){\line( 0,-1){ 10}}
\put( 80,475){\line( 0, 1){ 20}}
\put( 80,465){\line( 0,-1){ 20}}
\put(255,465){\line( 0, 1){  5}}
\put(255,470){\line( 1, 0){165}}
\put(420,470){\line( 0,-1){  5}}
\put(340,490){\line( 0,-1){ 20}}
\put( 55,345){\line( 0, 1){ 15}}
\put( 55,360){\line( 1, 0){265}}
\put(320,360){\line( 0,-1){ 10}}
\put( 80,375){\line( 0,-1){ 15}}
\put( 80,360){\line( 0, 1){  5}}
\put(150,360){\line( 0,-1){ 10}}
\put(245,360){\line( 0,-1){ 10}}
\put(340,290){\line( 0,-1){ 10}}
\put(340,270){\line( 0,-1){  5}}
\put(340,265){\vector( 1, 0){ 20}}
\put(360,355){\line( 0,-1){  5}}
\put(185,580){\line( 0,-1){ 10}}
\put(185,570){\line(-1, 0){ 45}}
\put(140,570){\line( 0,-1){200}}
\put(140,370){\line( 1, 0){220}}
\put(360,370){\line( 0,-1){  5}}
\put(280,785){\makebox(0,0)[lb]{\smash{\SetFigFont{10}{12.0}{rm}= inheritance}}}
\put(280,765){\makebox(0,0)[lb]{\smash{\SetFigFont{10}{12.0}{rm}= aggregation}}}
\put( 60,620){\makebox(0,0)[lb]{\smash{\SetFigFont{10}{12.0}{rm}vWindow}}}
\put( 90,230){\makebox(0,0)[lb]{\smash{\SetFigFont{14}{16.8}{rm}V - A Portable C++ GUI Framework}}}
\put(365,260){\makebox(0,0)[lb]{\smash{\SetFigFont{10}{12.0}{rm}vCmd}}}
\end{picture}

\regfont
\end{center}
\printindex
\end{document}
