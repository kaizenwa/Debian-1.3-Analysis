%***********************************************************************
%***********************************************************************
%***********************************************************************
\chapter{The V View of the World}

Before getting into the details of \V, you might find it useful to
read this overview of how the \V\ view of the world was developed.
If you are new at writing GUI applications, you should find this
chapter especially useful.

\section{A Generalized GUI Model}

\index{generalized GUI model}\index{GUI}
If you examine a large number of applications available on the major
GUI platforms, you will find the interfaces typically have a
great deal in common. While the visual details may differ,
most applications have windows that show views of the data being
manipulated, and use menus and dialogs for control interaction
with the user. The user interacts with the program using a
pointing device, usually a mouse, and the keyboard.

\subsection*{Windows}
\index{window}
\begin{rawhtml}
<IMG BORDER=0 ALIGN=BOTTOM ALT="" SRC="../fig/protoapp.gif">
\end{rawhtml}

\begin{figure}[htb]

\begin{center}
\small

\begin{latexonly}
\setlength{\unitlength}{0.012500in}%
\begin{picture}(330,245)(50,550)
\thicklines
\put( 50,580){\framebox(330,215){}}
\put( 50,775){\line( 1, 0){330}}
\put( 50,755){\line( 1, 0){330}}
\put( 60,770){\framebox(0,0){}}
\put(150,755){\line( 0,-1){ 65}}
\put(150,690){\line( 1, 0){ 90}}
\put(240,690){\line( 0, 1){ 65}}
\put( 50,610){\line( 1, 0){330}}
\put( 50,630){\line( 1, 0){310}}
\put(360,630){\line( 0, 1){125}}
\put(360,630){\line( 0,-1){ 20}}
\put(360,630){\line( 1, 0){ 20}}
\put(125,630){\line( 0,-1){ 20}}
\put(210,630){\line( 0,-1){ 20}}
\put(360,710){\line( 1, 0){ 20}}
\put(360,670){\line( 1, 0){ 20}}
\put(125,625){\line( 1, 0){ 85}}
\put(125,620){\line( 1, 0){ 85}}
\put(125,615){\line( 1, 0){ 85}}
\put(360,705){\line( 1, 0){ 20}}
\put(360,700){\line( 1, 0){ 20}}
\put(360,695){\line( 1, 0){ 20}}
\put(360,690){\makebox(0.4444,0.6667){\SetFigFont{10}{12}{rm}.}}
\put(360,690){\line( 1, 0){ 20}}
\put(360,685){\line( 1, 0){ 20}}
\put(360,680){\line( 1, 0){ 20}}
\put(360,675){\line( 1, 0){ 20}}
\put(130,630){\line( 0,-1){ 20}}
\put(135,630){\line( 0,-1){ 20}}
\put(140,630){\line( 0,-1){ 20}}
\put(145,630){\line( 0,-1){ 20}}
\put(150,630){\line( 0,-1){ 20}}
\put(155,630){\line( 0,-1){ 20}}
\put(160,630){\line( 0,-1){ 20}}
\put(165,630){\line( 0,-1){ 20}}
\put(170,630){\line( 0,-1){ 20}}
\put(175,630){\line( 0,-1){ 20}}
\put(180,630){\line( 0,-1){ 20}}
\put(185,630){\line( 0,-1){ 20}}
\put(190,630){\line( 0,-1){ 20}}
\put(195,630){\line( 0,-1){ 20}}
\put(200,630){\line( 0,-1){ 20}}
\put(205,630){\line( 0,-1){ 20}}
\put(365,710){\line( 0,-1){ 40}}
\put(370,710){\line( 0,-1){ 40}}
\put(375,710){\line( 0,-1){ 40}}
\put(180,660){\line( 1, 1){ 20}}
\put(200,680){\line( 1,-1){ 20}}
\put(220,660){\line( 1, 1){ 20}}
\put(240,680){\line( 1,-1){ 20}}
\put(260,660){\line( 1, 1){ 40}}
\put(300,700){\line( 0,-1){ 40}}
\put(300,660){\line(-1, 0){ 40}}
\put( 80,720){\line( 0,-1){ 60}}
\put( 80,660){\line( 1, 0){ 60}}
\put(140,660){\line(-1, 1){ 60}}
\put( 80,720){\line(-1,-1){ 30}}
\put(300,700){\line( 0, 1){ 20}}
\put(300,720){\line( 1,-1){ 20}}
\put(320,700){\line(-1, 0){ 20}}
\put(155,585){\framebox(75,20){}}
\put(235,585){\framebox(35,20){}}
\put(275,585){\framebox(40,20){}}
\put( 55,585){\framebox(45,20){}}
\put(105,585){\framebox(45,20){}}
\multiput( 70,765)(-0.33333,0.44444){10}{\makebox(0.4444,0.6667){\SetFigFont{7}{8.4}{rm}.}}
\put( 67,769){\line(-1, 0){  4}}
\multiput( 63,769)(-0.33333,-0.44444){10}{\makebox(0.4444,0.6667){\SetFigFont{7}{8.4}{rm}.}}
\multiput( 60,765)(0.33333,-0.44444){10}{\makebox(0.4444,0.6667){\SetFigFont{7}{8.4}{rm}.}}
\put( 63,761){\line( 1, 0){  4}}
\multiput( 67,761)(0.33333,0.44444){10}{\makebox(0.4444,0.6667){\SetFigFont{7}{8.4}{rm}.}}
\put( 60,760){\framebox(10,10){}}
\put( 50,580){\line( 0,-1){ 30}}
\put( 50,550){\line( 1, 0){330}}
\put(380,550){\line( 0, 1){ 30}}
\put(380,580){\line( 0, 1){  5}}
\put(160,555){\framebox(20,20){}}
\put(270,555){\framebox(25,20){}}
\put( 80,760){\makebox(0,0)[lb]{\smash{\SetFigFont{12}{14.4}{rm}File}}}
\put(150,760){\makebox(0,0)[lb]{\smash{\SetFigFont{12}{14.4}{rm}Test}}}
\put(160,740){\makebox(0,0)[lb]{\smash{\SetFigFont{12}{14.4}{rm}CheckMe}}}
\put(160,725){\makebox(0,0)[lb]{\smash{\SetFigFont{12}{14.4}{rm}Copy Sensitive}}}
\put(160,710){\makebox(0,0)[lb]{\smash{\SetFigFont{12}{14.4}{rm}Dialog}}}
\put(160,695){\makebox(0,0)[lb]{\smash{\SetFigFont{12}{14.4}{rm}Modal Dialog}}}
\put(115,760){\makebox(0,0)[lb]{\smash{\SetFigFont{12}{14.4}{rm}Edit}}}
\put(155,780){\makebox(0,0)[lb]{\smash{\SetFigFont{12}{14.4}{rm}Prototype V Example}}}
\put( 65,590){\makebox(0,0)[lb]{\smash{\SetFigFont{12}{14.4}{rm}Copy}}}
\put(110,590){\makebox(0,0)[lb]{\smash{\SetFigFont{12}{14.4}{rm}Dialog}}}
\put(165,590){\makebox(0,0)[lb]{\smash{\SetFigFont{12}{14.4}{rm}TestButton}}}
\put(240,590){\makebox(0,0)[lb]{\smash{\SetFigFont{12}{14.4}{rm}Clear}}}
\put(285,590){\makebox(0,0)[lb]{\smash{\SetFigFont{12}{14.4}{rm}Exit}}}
\put(120,780){\makebox(0,0)[lb]{\smash{\SetFigFont{10}{12.0}{rm}(1)}}}
\put( 55,745){\makebox(0,0)[lb]{\smash{\SetFigFont{10}{12.0}{rm}(2)}}}
\put(180,760){\makebox(0,0)[lb]{\smash{\SetFigFont{10}{12.0}{rm}(3)}}}
\put(245,720){\makebox(0,0)[lb]{\smash{\SetFigFont{10}{12.0}{rm}(4)}}}
\put(160,635){\makebox(0,0)[lb]{\smash{\SetFigFont{10}{12.0}{rm}(5)}}}
\put(340,680){\makebox(0,0)[lb]{\smash{\SetFigFont{10}{12.0}{rm}(5)}}}
\put(320,590){\makebox(0,0)[lb]{\smash{\SetFigFont{10}{12.0}{rm}(7)}}}
\put( 90,680){\makebox(0,0)[lb]{\smash{\SetFigFont{10}{12.0}{rm}(6)}}}
\put( 60,560){\makebox(0,0)[lb]{\smash{\SetFigFont{12}{14.4}{rm}Commands issued:}}}
\put(165,560){\makebox(0,0)[lb]{\smash{\SetFigFont{12}{14.4}{rm}12}}}
\put(190,560){\makebox(0,0)[lb]{\smash{\SetFigFont{12}{14.4}{rm}Last keypress:}}}
\put(275,560){\makebox(0,0)[lb]{\smash{\SetFigFont{12}{14.4}{rm}X}}}
\put(310,560){\makebox(0,0)[lb]{\smash{\SetFigFont{10}{12.0}{rm}(8)}}}
\end{picture}

\end{latexonly}

\normalfont\normalsize

\end{center}

\caption{This top level consists of: (1)~The title bar. (2)~The close
button. (3)~The menu bar. (4)~A pulldown menu. (5)~Vertical and horizontal
scroll bars. (6)~ The drawing
canvas. (7)~The command bar. (8)~The status bar.} \label{fig:protoapp}
\end{figure}

The \emph{window} is usually the main interface object used by an
application. The data being manipulated by the user (e.g.,
text, graphics, spreadsheet cells) is displayed in the window. Often,
several windows may be open at the same time, each giving
a different view of the data. There is usually a menu associated
with the window for entering commands to manipulate data or
to bring up dialogs.

The top level interface object
used by \V\ is a \emph{Command Window}. Each command window
consists of a \emph{menu bar}, placed at the
top of the window;  a \emph{canvas}
area, used to draw text and graphics to display the data; 
and optional \emph{command bars},
which include commands buttons and objects; and optional
\emph{status bars} to display state information.

Figure~\ref{fig:protoapp} represents, more or less, a typical
top-level \V\ window. 

\subsection*{Dialog Boxes}
\index{dialog}

\begin{rawhtml}
<IMG BORDER=0 ALIGN=BOTTOM ALT="" SRC="../fig/mydialog.gif">
\end{rawhtml}

\begin{figure}[htb]
\begin{center}
\small

\begin{latexonly}
\setlength{\unitlength}{0.012500in}%
\begin{picture}(315,170)(60,630)
\thicklines
\put( 75,645){\framebox(95,20){}}
\put( 80,650){\makebox(0,0)[lb]{\smash{\SetFigFont{12}{14.4}{rm}Toggle Sensitive}}}
\put(180,645){\framebox(80,20){}}
\put(185,650){\makebox(0,0)[lb]{\smash{\SetFigFont{12}{14.4}{rm}Change Me A}}}
\put(270,645){\framebox(45,20){}}
\put(275,650){\makebox(0,0)[lb]{\smash{\SetFigFont{12}{14.4}{rm}Cancel}}}
\put(215,690){\circle{10}}
\put(215,745){\circle*{10}}
\put(216,719){\circle{10}}
\put( 85,740){\framebox(10,10){}}
\put(145,740){\framebox(10,10){}}
\put( 85,720){\framebox(10,10){}}
\put( 75,715){\framebox(125,60){}}
\put(280,735){\framebox(55,20){}}
\put(280,710){\framebox(55,20){}}
\put(280,685){\framebox(55,20){}}
\put(205,675){\framebox(60,100){}}
\put(270,675){\framebox(75,100){}}
\put( 60,630){\framebox(315,170){}}
\put(325,645){\framebox(40,20){}}
\put(330,665){\line( 0,-1){ 20}}
\put(360,665){\line( 0,-1){ 20}}
\put( 85,760){\makebox(0,0)[lb]{\smash{\SetFigFont{12}{14.4}{rm}CheckBox}}}
\put(100,740){\makebox(0,0)[lb]{\smash{\SetFigFont{12}{14.4}{rm}Test A}}}
\put(160,740){\makebox(0,0)[lb]{\smash{\SetFigFont{12}{14.4}{rm}Test B}}}
\put(100,720){\makebox(0,0)[lb]{\smash{\SetFigFont{12}{14.4}{rm}Test C}}}
\put(220,760){\makebox(0,0)[lb]{\smash{\SetFigFont{12}{14.4}{rm}Radios}}}
\put(280,760){\makebox(0,0)[lb]{\smash{\SetFigFont{12}{14.4}{rm}Buttons}}}
\put(285,740){\makebox(0,0)[lb]{\smash{\SetFigFont{12}{14.4}{rm}Button 1}}}
\put(285,715){\makebox(0,0)[lb]{\smash{\SetFigFont{12}{14.4}{rm}Button 2}}}
\put(285,690){\makebox(0,0)[lb]{\smash{\SetFigFont{12}{14.4}{rm}Button 3}}}
\put(225,685){\makebox(0,0)[lb]{\smash{\SetFigFont{12}{14.4}{rm}KRQE}}}
\put(225,740){\makebox(0,0)[lb]{\smash{\SetFigFont{12}{14.4}{rm}KOB}}}
\put(225,715){\makebox(0,0)[lb]{\smash{\SetFigFont{12}{14.4}{rm}KOAT}}}
\put(210,785){\makebox(0,0)[lb]{\smash{\SetFigFont{10}{12.0}{rm}(1)}}}
\put(150,760){\makebox(0,0)[lb]{\smash{\SetFigFont{10}{12.0}{rm}(2)}}}
\put(190,690){\makebox(0,0)[lb]{\smash{\SetFigFont{10}{12.0}{rm}(3)}}}
\put(325,760){\makebox(0,0)[lb]{\smash{\SetFigFont{10}{12.0}{rm}(4)}}}
\put(110,670){\makebox(0,0)[lb]{\smash{\SetFigFont{10}{12.0}{rm}(5)}}}
\put(350,670){\makebox(0,0)[lb]{\smash{\SetFigFont{10}{12.0}{rm}(6)}}}
\put(335,650){\makebox(0,0)[lb]{\smash{\SetFigFont{12}{14.4}{rm}OK}}}
\put( 80,785){\makebox(0,0)[lb]{\smash{\SetFigFont{12}{14.4}{rm}Sample Modeless Dialog}}}
\end{picture}

\end{latexonly}

\normalfont\normalsize
\end{center}

\caption{This dialog consists of: (1)~Dialog label. (2)~Three check boxes
in a frame. (3)~Three radio buttons in a frame. (4)~Three buttons
in a frame. (5)~Four buttons, including (6)~ the default OK
button.} \label{fig:mydialog}
\end{figure}

Much control interaction with \V applications takes place
through one of two dialog objects: \emph{modal}
and \emph{modeless} dialogs. In a modal dialog, interaction with
any other window or dialog is locked out until the user
interacts with it. In a modeless dialog, the user can continue to
interact with other parts of the application while the dialog remains
displayed. Modal dialogs will go away once the user enters a
command. Modeless dialogs may or may not go away, depending on
their purpose.

\V\ supports a comprehensive set of controls for dialogs. These
include command buttons, text labels, text input, list selection
boxes, combo boxes, radio buttons, check boxes, spinners for
value entry, sliders, and progress bars. These controls may be
grouped into boxes. Layout of controls in a dialog is defined in
the dialog definition list in the source code. Controls may be
used in window command bars as well as dialogs.

Figure~\ref{fig:mydialog} represents, more or less, a typical
\V\ dialog.

\subsection*{Events}
\index{event}

The structure of the code for user command processing in GUI
applications is quite different from traditional C programs. The
user input control model of traditional C programs is rather
simple, usually using \code{printf} and \code{getc} or some
variant for interaction. Logically, the program reaches a point
where it needs input, and then waits for that input.

GUI applications deal with user input much differently.
Interaction with an application from the user's viewpoint
consists of a series of mouse movements and clicks, and text and
command input through the keyboard. From the programmer's
viewpoint, each of these is an event. The important thing about
an event is that it can occur at any time, and the program cannot
simply stop and wait for the event to happen.

Interaction with an application by the user can generate several
different kinds of events. Consider mouse events.
If the mouse is in the drawing area, each movement generates a
\emph{mouse movement} event. If the user clicks a mouse button, a
\emph{mouse button} event is generated. A keystroke from the
keyboard will generate a \emph{keyboard} event.

If the mouse pointer is in a dialog, or over a menu or command
button, then movement events are not generated. Instead, button
clicks generate \emph{command} events. 

Sometimes an application needs to track the passage of time. The
application can call a system function that will generate a
\emph{timer} event at a given interval.

In a GUI environment, windows are usually not displayed alone. Often,
other applications are running, each with its own windows. The
host windowing system typically displays windows with various decorations
that let the user manipulate the windows. Sometimes, these
manipulations will generate events that require a response
from the application code. For example, the user can use the
mouse to change the size of a window causing a \emph{resize}
event. When multiple windows are displayed, some can be
completely or partially covered by other windows. If the user
moves a window so that a different part of the window is
displayed, then an \emph{expose} event is generated, which
requires the program to redraw part of the canvas area.

All these events require a response from the application --
to carry out the command, to draw something in the canvas area,
or to redraw the canvas after a resize or expose event. Some
events, however, are handled by the system, and not the
application. This includes drawing menus and handling dialogs.
For example, when a dialog is displayed, the system tracks mouse
movements within the dialog, and handles redrawing the dialog for
expose events. In general, the application is responsible for
resize and expose events only for the canvas area.

All these events are asynchronous, and the application must be able
to respond immediately to any of these events. Traditionally, 
handling events has been rather complicated. For each possible
event, the program registers an \emph{event handler} with
the system. Then, the program runs in an \emph{event loop}.
The event loop receives an event, and then calls a
function to dispatch the event to the proper event handler.

C++ makes dealing with events much easier. Each event can be considered
a message, and the message is central to object-oriented
programming. In \V, each object, such as a command window, has
methods\footnote{I use the general object-oriented term \emph{method}
to refer to what are called \emph{member functions} in C++
terminology.} that the system sends event messages to. For
example, there is a \code{WindowCommand} method that responds to
command events from the system. The application overrides the
default \V\ \code{WindowCommand} method to handle commands as
needed by the application. All the details of the event loop and
event handlers are hidden in the \V\ implementation. If you have
ever programmed with event handlers and loops, you will find the
simplicity of overriding default methods incredibly easy in
comparison!

%---------------------------------------------------------------
\section{Easy to program}

One of the main goals of the design of \V\ was to make it easy to
use to write real programs. Some of the factors that help \V\ meet
this goal are described in the following sections.

\subsection*{Hide the dirty details}

One of the problems with using most native GUI libraries such
as Xt or Windows is the huge amount of overhead and detail required
to perform even the simplest tasks. You are typically coding at a
very low level. While part of this complexity may be necessary to
allow total flexibility in what you can do, the vast majority of
applications just do not need total flexibility. \V\ was designed
to hide most of the details of the underlying GUI library. Things
such as library initialization, specific window handles, and
calls required to build low level controls are all hidden.
Instead, you work at the much higher level of objects needed to
build a typical GUI.

\subsection*{Easy to define GUI objects}

It has always seemed to me that a GUI object such as a menu could
most simply be thought of as a single object consisting of a list
of items on that menu with their associated attributes. Rather than
requiring a set of complicated calls to build that menu list, in
\V\ you can simply define a menu as a static C++ struct array --
a list in other words. The same applies to dialogs. A dialog is a
list of control objects with associated attributes, including
how the controls are positioned in the dialog. This philosophy leads
to very easy to maintain code. Menu and dialog lists are well
defined in a single place in your code, and it is very easy to
modify and change the list definitions. Actions for each menu or
dialog command are defined in a single C++ method that responds
to command events.

\subsection*{No resource editors}
\index{resource editor}

One data object used by most, but not all, native GUI libraries
is what is usually called a resource file. A resource file is
most often used to specify layout of dialogs and menus. One
reason resource files are used is that specifying the layout of
dialogs and menus directly in the code is often very difficult
for the native libraries.

The combination of the way \V\ lets you specify menus and dialogs,
and the way C++ makes responding to event messages so easy has really
removed the need for resource files. This in turn eliminates one
of the more complicated aspects of portability across platforms.

\subsection*{Look and Feel}

One of the limitations of \V\ is that it has its own look and feel.
While this may be a limitation, it is not necessarily bad. First,
the look and feel is constrained so that applications will be portable
across platforms and look like native applications on each platform.
This means some things that are possible on one GUI platform, but
not another, are not included in \V\@.

\V\ also incorporates much of my own experience. I really like
simplicity, and believe that just because you can do something,
it is not necessarily a good idea to do so. Thus, for example,
there are limitations on the number of menu items per menu, and
how deeply you can nest pull down menus. These limits in fact
help enforce good interface design.

%---------------------------------------------------------------
\section{Good Example of OO}

While \V\ has been designed to develop real and useful GUI
applications, it also has been designed to be useful in a
learning environment. Thus, \V\ represents a good example of
object-oriented design. 

GUI systems are a natural for object orientation (OO). It is easy
to understand the nature of each object -- a window, a dialog, a
command button, a menu bar, a canvas, and so on. Inheritance and
aggregation of these objects is also very natural. Events are
messages, and sending messages to methods is pure OO.

\index{GNU license}
Since \V\ is licensed under the terms of the GNU Library General
Public License, the source code will always be available for
study. It was written using the guidelines of Appendix B, and is
very readable and easy to understand. Not only is the \V\ source
code a good example of OO programming, you may also find it
interesting if you want to learn things about how the
underlying GUI toolkits work. While good examples of freeware X
source code are readily available, good examples of non-trivial
Windows source code are nearly impossible to come by. I hope the
\V\ Windows source code will help fill this void.


%---------------------------------------------------------------
\section{The V Object Hierarchy}
\index{V Object Hierarchy}

This manual contains several object hierarchy diagrams of the
\V\ framework, and of \V\ applications. There are many graphical
notations in varying degrees of widespread use, but I have found
the Coad-Yourdon\footnote{Peter Coad and Edward Yourdon, 
\emph{Object-Oriented Analysis}, 2nd ed. (Yourdon
Press/Prentice Hall, 1990); and Edward Yourdon, \emph{Object-Oriented
System Design, An Integrated Approach} (Yourdon Press/Prentice Hall,
1994, ISBN 0-13-636325-3).} notation one of the
\index{Coad-Yourdon}
easiest to learn and simplest to use. The basic graphical
elements of the notation are shown in  Figure~\ref{fig:oonotation}.

\begin{rawhtml}
<IMG BORDER=0 ALIGN=BOTTOM ALT="" SRC="../fig/oonotation.gif">
\end{rawhtml}

\begin{figure}[htb]

\begin{center}
\small
\begin{latexonly}
\setlength{\unitlength}{0.012500in}%
\begin{picture}(324,150)(19,670)
\thicklines
\put( 30,695){\oval( 20, 20)[tr]}
\put( 30,695){\oval( 20, 20)[tl]}
\put( 20,695){\line( 1, 0){ 20}}
\put( 30,695){\line( 0,-1){ 15}}
\put( 30,720){\line( 0,-1){ 15}}
\put( 50,695){\makebox(0,0)[lb]{\smash{\SetFigFont{10}{12.0}{rm}inheritance (is-a)}}}
\multiput(195,695)(0.40000,0.40000){26}{\makebox(0.4444,0.6667){\SetFigFont{7}{8.4}{rm}.}}
\multiput(205,705)(0.40000,-0.40000){26}{\makebox(0.4444,0.6667){\SetFigFont{7}{8.4}{rm}.}}
\put(215,695){\line(-1, 0){ 20}}
\put(220,710){\makebox(0,0)[lb]{\smash{\SetFigFont{10}{12.0}{rm}1,N}}}
\put(220,685){\makebox(0,0)[lb]{\smash{\SetFigFont{10}{12.0}{rm}1,N}}}
\put(205,695){\line( 0,-1){ 15}}
\put(205,680){\line( 0, 1){  5}}
\put(250,695){\makebox(0,0)[lb]{\smash{\SetFigFont{10}{12.0}{rm}aggregation (has-a)}}}
\put(230,670){\makebox(0,0)[lb]{\smash{\SetFigFont{10}{12.0}{rm}(Part Of 1 to N objects)}}}
\put(230,725){\makebox(0,0)[lb]{\smash{\SetFigFont{10}{12.0}{rm}(Has 1 to N objects)}}}
\put(205,720){\line( 0,-1){ 15}}
\put(195,760){\framebox(90,60){}}
\put(200,765){\framebox(80,50){}}
\put(200,800){\line( 1, 0){ 80}}
\put(200,780){\line( 1, 0){ 80}}
\put(295,800){\makebox(0,0)[lb]{\smash{\SetFigFont{10}{12.0}{rm}Class with}}}
\put(295,785){\makebox(0,0)[lb]{\smash{\SetFigFont{10}{12.0}{rm}instances of}}}
\put(295,770){\makebox(0,0)[lb]{\smash{\SetFigFont{10}{12.0}{rm}objects}}}
\put(210,805){\makebox(0,0)[lb]{\smash{\SetFigFont{10}{12.0}{rm}class name}}}
\put(210,770){\makebox(0,0)[lb]{\smash{\SetFigFont{10}{12.0}{rm}methods}}}
\put(210,785){\makebox(0,0)[lb]{\smash{\SetFigFont{10}{12.0}{rm}attributes}}}
\put( 20,765){\framebox(80,50){}}
\put( 20,800){\line( 1, 0){ 80}}
\put( 20,780){\line( 1, 0){ 80}}
\put(110,800){\makebox(0,0)[lb]{\smash{\SetFigFont{10}{12.0}{rm}Class with}}}
\put(110,785){\makebox(0,0)[lb]{\smash{\SetFigFont{10}{12.0}{rm}no instances}}}
\put( 30,805){\makebox(0,0)[lb]{\smash{\SetFigFont{10}{12.0}{rm}class name}}}
\put( 30,770){\makebox(0,0)[lb]{\smash{\SetFigFont{10}{12.0}{rm}methods}}}
\put( 30,785){\makebox(0,0)[lb]{\smash{\SetFigFont{10}{12.0}{rm}attributes}}}
\put(110,770){\makebox(0,0)[lb]{\smash{\SetFigFont{10}{12.0}{rm}of objects}}}
\end{picture}

\end{latexonly}
\normalfont\normalsize
\end{center}

\caption{Coad-Yourdon OO Notation} \label{fig:oonotation}
\end{figure}

An object is shown in a rectangular box. A single border indicates
a generalized base class that will not have any instances,
while a double border indicates that the named object can have
instances. Generalization/specialization (inheritance, or is-a)
relationships are shown with half circles. Whole/part (aggregation,
or has-a) relationships are shown with triangles\footnote{Hint:
It is sometimes hard to remember which shape is which. A triangle
looks like a capital letter A as in Aggregation. The half circle
shape is then inheritance.}.

The ``1,N'' notation at the top of the aggregation triangle
indicates that the object above can contain from 1 to N instances
of the object below. The lower ``1,N'' indicates the lower object
can be a part of 1 to N objects. The values can be changed to
reflect reality. Thus, it is common to have ``1,N'' at the top,
indicating that an object may contain many instances of the lower
object, and just a ``1'' for the lower value, indication that an
object is a part of exactly one of the upper objects.

When discussing a design at a high level, the attributes and
methods boxes are often left blank. This leads to hierarchies
such as the one for \V\ in Figure~\ref{fig:vproghier} that shows
the programming view of the \V\ framework. In this case, there
are no generalized base objects, and most of the relationships
are whole/part.

\begin{rawhtml}
<IMG BORDER=0 ALIGN=BOTTOM ALT="" SRC="../fig/vproghier.gif">
\end{rawhtml}

\begin{figure}[htb]
\begin{center}
\small
\begin{latexonly}
\setlength{\unitlength}{0.012500in}%
\begin{picture}(360,420)(5,415)
\thicklines
\multiput(255,765)(0.40000,0.40000){26}{\makebox(0.4444,0.6667){\SetFigFont{7}{8.4}{rm}.}}
\multiput(265,775)(0.40000,-0.40000){26}{\makebox(0.4444,0.6667){\SetFigFont{7}{8.4}{rm}.}}
\put(275,765){\line(-1, 0){ 20}}
\put(265,785){\oval( 20, 20)[tr]}
\put(265,785){\oval( 20, 20)[tl]}
\put(255,785){\line( 1, 0){ 20}}
\put( 10,535){\framebox(80,50){}}
\put(  5,530){\framebox(90,60){}}
\put( 10,570){\line( 1, 0){ 80}}
\put( 10,550){\line( 1, 0){ 80}}
\put( 25,575){\makebox(0,0)[lb]{\smash{\SetFigFont{10}{12.0}{rm}vMenuPane}}}
\multiput( 40,605)(0.40000,0.40000){26}{\makebox(0.4444,0.6667){\SetFigFont{7}{8.4}{rm}.}}
\multiput( 50,615)(0.40000,-0.40000){26}{\makebox(0.4444,0.6667){\SetFigFont{7}{8.4}{rm}.}}
\put( 60,605){\line(-1, 0){ 20}}
\put( 65,595){\makebox(0,0)[lb]{\smash{\SetFigFont{10}{12.0}{rm}1}}}
\put( 65,620){\makebox(0,0)[lb]{\smash{\SetFigFont{10}{12.0}{rm}1}}}
\put( 50,605){\line( 0,-1){ 15}}
\put(280,535){\framebox(80,50){}}
\put(275,530){\framebox(90,60){}}
\put(280,570){\line( 1, 0){ 80}}
\put(280,550){\line( 1, 0){ 80}}
\put(290,575){\makebox(0,0)[lb]{\smash{\SetFigFont{10}{12.0}{rm}vModalDialog}}}
\multiput(310,500)(0.40000,0.40000){26}{\makebox(0.4444,0.6667){\SetFigFont{7}{8.4}{rm}.}}
\multiput(320,510)(0.40000,-0.40000){26}{\makebox(0.4444,0.6667){\SetFigFont{7}{8.4}{rm}.}}
\put(330,500){\line(-1, 0){ 20}}
\put(335,490){\makebox(0,0)[lb]{\smash{\SetFigFont{10}{12.0}{rm}1}}}
\put(335,515){\makebox(0,0)[lb]{\smash{\SetFigFont{10}{12.0}{rm}1,N}}}
\multiput(245,500)(0.40000,0.40000){26}{\makebox(0.4444,0.6667){\SetFigFont{7}{8.4}{rm}.}}
\multiput(255,510)(0.40000,-0.40000){26}{\makebox(0.4444,0.6667){\SetFigFont{7}{8.4}{rm}.}}
\put(265,500){\line(-1, 0){ 20}}
\put(270,490){\makebox(0,0)[lb]{\smash{\SetFigFont{10}{12.0}{rm}1}}}
\put(270,515){\makebox(0,0)[lb]{\smash{\SetFigFont{10}{12.0}{rm}1,N}}}
\put(250,420){\framebox(80,50){}}
\put(245,415){\framebox(90,60){}}
\put(250,455){\line( 1, 0){ 80}}
\put(250,435){\line( 1, 0){ 80}}
\put(251,460){\makebox(0,0)[lb]{\smash{\SetFigFont{10}{12.0}{rm}CommandObject}}}
\put(130,535){\framebox(80,50){}}
\put(125,530){\framebox(90,60){}}
\put(130,570){\line( 1, 0){ 80}}
\put(130,550){\line( 1, 0){ 80}}
\put(145,575){\makebox(0,0)[lb]{\smash{\SetFigFont{10}{12.0}{rm}vStatusPane}}}
\multiput(160,605)(0.40000,0.40000){26}{\makebox(0.4444,0.6667){\SetFigFont{7}{8.4}{rm}.}}
\multiput(170,615)(0.40000,-0.40000){26}{\makebox(0.4444,0.6667){\SetFigFont{7}{8.4}{rm}.}}
\put(180,605){\line(-1, 0){ 20}}
\put(170,605){\line( 0,-1){ 15}}
\put(185,595){\makebox(0,0)[lb]{\smash{\SetFigFont{10}{12.0}{rm}1}}}
\put(185,620){\makebox(0,0)[lb]{\smash{\SetFigFont{10}{12.0}{rm}0,N}}}
\put(130,420){\framebox(80,50){}}
\put(125,415){\framebox(90,60){}}
\put(130,455){\line( 1, 0){ 80}}
\put(130,435){\line( 1, 0){ 80}}
\put(133,460){\makebox(0,0)[lb]{\smash{\SetFigFont{10}{12.0}{rm}vCommandPane}}}
\multiput(160,490)(0.40000,0.40000){26}{\makebox(0.4444,0.6667){\SetFigFont{7}{8.4}{rm}.}}
\multiput(170,500)(0.40000,-0.40000){26}{\makebox(0.4444,0.6667){\SetFigFont{7}{8.4}{rm}.}}
\put(180,490){\line(-1, 0){ 20}}
\put(170,490){\line( 0,-1){ 15}}
\put(185,480){\makebox(0,0)[lb]{\smash{\SetFigFont{10}{12.0}{rm}1}}}
\put(185,505){\makebox(0,0)[lb]{\smash{\SetFigFont{10}{12.0}{rm}0,N}}}
\put( 10,420){\framebox(80,50){}}
\put(  5,415){\framebox(90,60){}}
\put( 10,455){\line( 1, 0){ 80}}
\put( 10,435){\line( 1, 0){ 80}}
\put( 25,460){\makebox(0,0)[lb]{\smash{\SetFigFont{10}{12.0}{rm}vCanvasPane}}}
\multiput( 40,490)(0.40000,0.40000){26}{\makebox(0.4444,0.6667){\SetFigFont{7}{8.4}{rm}.}}
\multiput( 50,500)(0.40000,-0.40000){26}{\makebox(0.4444,0.6667){\SetFigFont{7}{8.4}{rm}.}}
\put( 60,490){\line(-1, 0){ 20}}
\put( 50,490){\line( 0,-1){ 15}}
\put( 65,480){\makebox(0,0)[lb]{\smash{\SetFigFont{10}{12.0}{rm}1}}}
\put( 65,505){\makebox(0,0)[lb]{\smash{\SetFigFont{10}{12.0}{rm}1}}}
\put( 10,780){\framebox(80,50){}}
\put(  5,775){\framebox(90,60){}}
\put( 10,815){\line( 1, 0){ 80}}
\put( 10,795){\line( 1, 0){ 80}}
\put( 35,820){\makebox(0,0)[lb]{\smash{\SetFigFont{10}{12.0}{rm}vApp}}}
\multiput( 40,750)(0.40000,0.40000){26}{\makebox(0.4444,0.6667){\SetFigFont{7}{8.4}{rm}.}}
\multiput( 50,760)(0.40000,-0.40000){26}{\makebox(0.4444,0.6667){\SetFigFont{7}{8.4}{rm}.}}
\put( 60,750){\line(-1, 0){ 20}}
\put( 65,740){\makebox(0,0)[lb]{\smash{\SetFigFont{10}{12.0}{rm}1}}}
\put( 65,765){\makebox(0,0)[lb]{\smash{\SetFigFont{10}{12.0}{rm}1,N}}}
\put( 10,680){\framebox(80,50){}}
\put(  5,675){\framebox(90,60){}}
\put( 10,715){\line( 1, 0){ 80}}
\put( 10,695){\line( 1, 0){ 80}}
\put( 20,720){\makebox(0,0)[lb]{\smash{\SetFigFont{10}{12.0}{rm}vCmdWindow}}}
\put(320,595){\oval( 20, 20)[tr]}
\put(320,595){\oval( 20, 20)[tl]}
\put(310,595){\line( 1, 0){ 20}}
\put(245,620){\framebox(80,50){}}
\put(240,615){\framebox(90,60){}}
\put(245,655){\line( 1, 0){ 80}}
\put(245,635){\line( 1, 0){ 80}}
\put(265,660){\makebox(0,0)[lb]{\smash{\SetFigFont{10}{12.0}{rm}vDialog}}}
\multiput(275,690)(0.40000,0.40000){26}{\makebox(0.4444,0.6667){\SetFigFont{7}{8.4}{rm}.}}
\multiput(285,700)(0.40000,-0.40000){26}{\makebox(0.4444,0.6667){\SetFigFont{7}{8.4}{rm}.}}
\put(295,690){\line(-1, 0){ 20}}
\put(300,680){\makebox(0,0)[lb]{\smash{\SetFigFont{10}{12.0}{rm}1}}}
\put(300,705){\makebox(0,0)[lb]{\smash{\SetFigFont{10}{12.0}{rm}0,N}}}
\put(320,530){\line( 0,-1){ 20}}
\put(255,500){\line( 0,-1){ 15}}
\put(255,485){\line( 1, 0){ 25}}
\put(280,485){\line( 0,-1){ 10}}
\put(320,500){\line( 0,-1){ 15}}
\put(320,485){\line(-1, 0){ 20}}
\put(300,485){\line( 0,-1){ 10}}
\put(300,475){\line( 0, 1){  5}}
\put( 20,675){\line( 0,-1){ 50}}
\put( 20,625){\line( 1, 0){ 30}}
\put( 50,625){\line( 0,-1){ 10}}
\put( 80,675){\line( 0,-1){  5}}
\put( 80,670){\line( 1, 0){140}}
\put(220,670){\line( 0, 1){ 50}}
\put(220,720){\line( 1, 0){ 65}}
\put(285,720){\line( 0,-1){ 20}}
\put(285,690){\line( 0,-1){ 15}}
\put(255,510){\line( 0, 1){105}}
\put(310,615){\line( 0,-1){  5}}
\put(310,610){\line( 1, 0){ 10}}
\put(320,610){\line( 0,-1){  5}}
\put(320,595){\line( 0,-1){  5}}
\put( 35,675){\line( 0,-1){ 40}}
\put( 35,635){\line( 1, 0){ 70}}
\put(105,635){\line( 0,-1){115}}
\put(105,520){\line(-1, 0){ 55}}
\put( 50,520){\line( 0,-1){ 20}}
\put( 50,675){\line( 0,-1){ 30}}
\put( 50,645){\line( 1, 0){ 65}}
\put(115,645){\line( 0,-1){125}}
\put(115,520){\line( 1, 0){ 55}}
\put(170,520){\line( 0,-1){ 20}}
\put( 65,675){\line( 0,-1){ 20}}
\put( 65,655){\line( 1, 0){105}}
\put(170,655){\line( 0,-1){ 40}}
\put( 50,775){\line( 0,-1){ 15}}
\put( 50,750){\line( 0,-1){ 15}}
\put(280,785){\makebox(0,0)[lb]{\smash{\SetFigFont{10}{12.0}{rm}= inheritance}}}
\put(280,765){\makebox(0,0)[lb]{\smash{\SetFigFont{10}{12.0}{rm}= aggregation}}}
\end{picture}

\end{latexonly}
\normalfont\normalsize
\end{center}
\caption{Programming View of V Classes} \label{fig:vproghier}
\end{figure}

Figure~\ref{fig:vproghier} reveals some interesting things about
\V's look and feel. Note that a \code{vApp} class has 1 to N
\code{vCmdWindows}, indicating that there will be at least one
window. Each window, in turn, has exactly one menu and canvas,
but zero to many command panes, status panes, and dialogs.

The version of the \V\ hierarchy in the Appendix shows an
implementation view of the hierarchy. Some of the classes that
are never seen or used by the programmer are shown in that
hierarchy. 
