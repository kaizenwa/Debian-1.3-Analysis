%***********************************************************************
%***********************************************************************
%***********************************************************************
\chapter{A V Tutorial}

%------------------------------------------------------------------------

This chapter is intended to cover the elements that make up a
\V\ application. The first section covers the general
organization of a ``Standard \V\ Application''. Read this section
to get an overview of a \V\ application. Don't worry about the details
yet -- just the the main idea. Then read Section~\ref{sec:tutexamp}
and Appendix A, which has the source code of a small, complete
\V\ application, to get the details.

\section{Getting Started with Your Own V Application}
\index{getting started}

As with any new system, \V\ has a learning curve before you can
write applications of your own. \V's learning curve is actually
pretty short. The experience of the students using \V\ has shown
the best way to get started with \V\ is to first read the first
part of this reference manual, including this chapter. Then begin
with an example \V\ application. 

The \V application generator, \code{vgen}, included with the
\V distribution is the easiest way to begin building a \V
application. Run \code{vgen}, select the basic options you
want to include in your application, select the directory
to save the generated code in, and then generate the basic
skeleton application. From the skeleton app, it is relatively
easy to add your own functionality.

The tutorial application described in this chapter is also an
excellent \V example. Start by getting the example to compile.
Then modify the code to add or remove features. Before long, you
will have a good feel for \V, and be able to add all the features
you need.

There are several other example programs provided with the \V\
distribution. This tutorial is found in \code{\~{}/v/tutor}.
The VDraw program is found in \code{\~{}/v/draw}. The program
used to test all \V\ functionality is found in \code{\~{}/v/test}.
It will have an example of how to use every \V\ feature,
although it is not as well structured as the other examples.

\section{A Standard V Application}
\index{standard V application}

While the \V\ framework is flexible enough to allow many different
approaches to building an application, you should find it easier
to base your applications on a model \emph{Standard \V\ Application}.
The software organization described by a Standard \V\ Application
can support MVC (Model-View-Controller) object-oriented architecture paradigm.

Figure~\ref{fig:stdvapp} shows the hierarchy of a standard \V\ application.
A standard \V\ application consists of the parts described below.
Each part consists of a pair of \code{.cpp} (or \code{.cxx}) and
\code{.h} files (except the \code{makefile}).

\begin{rawhtml}
<IMG BORDER=0 ALIGN=BOTTOM ALT="" SRC="../fig/stdvapp.gif">
\end{rawhtml}

\begin{figure}[htb]
\begin{center}
\vspace{.1in}

\small
\begin{latexonly}
\setlength{\unitlength}{0.012500in}%
\begin{picture}(430,405)(10,420)
\thicklines
\put(160,760){\framebox(80,60){}}
\put(155,755){\framebox(90,70){}}
\put(160,805){\line( 1, 0){ 80}}
\put(160,775){\line( 1, 0){ 80}}
\put(185,810){\makebox(0,0)[lb]{\smash{\SetFigFont{10}{12.0}{rm}myApp}}}
\put(160,645){\framebox(80,60){}}
\put(155,640){\framebox(90,70){}}
\put(160,690){\line( 1, 0){ 80}}
\put(160,660){\line( 1, 0){ 80}}
\put(165,695){\makebox(0,0)[lb]{\smash{\SetFigFont{10}{12.0}{rm}myCmdWindow}}}
\multiput(190,725)(0.40000,0.40000){26}{\makebox(0.4444,0.6667){\SetFigFont{7}{8.4}{rm}.}}
\multiput(200,735)(0.40000,-0.40000){26}{\makebox(0.4444,0.6667){\SetFigFont{7}{8.4}{rm}.}}
\put(210,725){\line(-1, 0){ 20}}
\put(215,715){\makebox(0,0)[lb]{\smash{\SetFigFont{10}{12.0}{rm}1}}}
\put(215,730){\makebox(0,0)[lb]{\smash{\SetFigFont{10}{12.0}{rm}1,N}}}
\put( 40,645){\framebox(80,60){}}
\put( 35,640){\framebox(90,70){}}
\put( 40,690){\line( 1, 0){ 80}}
\put( 40,660){\line( 1, 0){ 80}}
\put( 45,695){\makebox(0,0)[lb]{\smash{\SetFigFont{10}{12.0}{rm}myAppWinInfo}}}
\multiput( 70,725)(0.40000,0.40000){26}{\makebox(0.4444,0.6667){\SetFigFont{7}{8.4}{rm}.}}
\multiput( 80,735)(0.40000,-0.40000){26}{\makebox(0.4444,0.6667){\SetFigFont{7}{8.4}{rm}.}}
\put( 90,725){\line(-1, 0){ 20}}
\put( 95,715){\makebox(0,0)[lb]{\smash{\SetFigFont{10}{12.0}{rm}1}}}
\put( 95,730){\makebox(0,0)[lb]{\smash{\SetFigFont{10}{12.0}{rm}1,N}}}
\multiput(310,725)(0.40000,0.40000){26}{\makebox(0.4444,0.6667){\SetFigFont{7}{8.4}{rm}.}}
\multiput(320,735)(0.40000,-0.40000){26}{\makebox(0.4444,0.6667){\SetFigFont{7}{8.4}{rm}.}}
\put(330,725){\line(-1, 0){ 20}}
\put(335,715){\makebox(0,0)[lb]{\smash{\SetFigFont{10}{12.0}{rm}1}}}
\put(335,730){\makebox(0,0)[lb]{\smash{\SetFigFont{10}{12.0}{rm}1}}}
\put( 50,800){\makebox(0,0)[lb]{\smash{\SetFigFont{12}{14.4}{rm}A Standard}}}
\put( 50,780){\makebox(0,0)[lb]{\smash{\SetFigFont{12}{14.4}{rm}V Application}}}
\put( 15,525){\framebox(80,60){}}
\put( 10,520){\framebox(90,70){}}
\put( 15,570){\line( 1, 0){ 80}}
\put( 15,540){\line( 1, 0){ 80}}
\put( 30,575){\makebox(0,0)[lb]{\smash{\SetFigFont{10}{12.0}{rm}vMenuPane}}}
\put( 55,605){\line( 0,-1){ 15}}
\multiput( 45,605)(0.40000,0.40000){26}{\makebox(0.4444,0.6667){\SetFigFont{7}{8.4}{rm}.}}
\multiput( 55,615)(0.40000,-0.40000){26}{\makebox(0.4444,0.6667){\SetFigFont{7}{8.4}{rm}.}}
\put( 65,605){\line(-1, 0){ 20}}
\put( 70,595){\makebox(0,0)[lb]{\smash{\SetFigFont{10}{12.0}{rm}1}}}
\put( 70,610){\makebox(0,0)[lb]{\smash{\SetFigFont{10}{12.0}{rm}1}}}
\put(110,525){\framebox(80,60){}}
\put(105,520){\framebox(90,70){}}
\put(110,570){\line( 1, 0){ 80}}
\put(110,540){\line( 1, 0){ 80}}
\put(115,575){\makebox(0,0)[lb]{\smash{\SetFigFont{10}{12.0}{rm}myCanvasPane}}}
\put(150,605){\line( 0,-1){ 15}}
\multiput(140,605)(0.40000,0.40000){26}{\makebox(0.4444,0.6667){\SetFigFont{7}{8.4}{rm}.}}
\multiput(150,615)(0.40000,-0.40000){26}{\makebox(0.4444,0.6667){\SetFigFont{7}{8.4}{rm}.}}
\put(160,605){\line(-1, 0){ 20}}
\put(165,595){\makebox(0,0)[lb]{\smash{\SetFigFont{10}{12.0}{rm}1}}}
\put(165,610){\makebox(0,0)[lb]{\smash{\SetFigFont{10}{12.0}{rm}1}}}
\multiput(235,605)(0.40000,0.40000){26}{\makebox(0.4444,0.6667){\SetFigFont{7}{8.4}{rm}.}}
\multiput(245,615)(0.40000,-0.40000){26}{\makebox(0.4444,0.6667){\SetFigFont{7}{8.4}{rm}.}}
\put(255,605){\line(-1, 0){ 20}}
\put(260,595){\makebox(0,0)[lb]{\smash{\SetFigFont{10}{12.0}{rm}1}}}
\put(260,610){\makebox(0,0)[lb]{\smash{\SetFigFont{10}{12.0}{rm}0,N}}}
\put(200,520){\framebox(90,70){}}
\put(210,575){\makebox(0,0)[lb]{\smash{\SetFigFont{10}{12.0}{rm}vCommandPane}}}
\put(205,525){\framebox(80,60){}}
\put(205,540){\line( 1, 0){ 80}}
\put(205,570){\line( 1, 0){ 80}}
\put(245,605){\line( 0,-1){ 15}}
\multiput(330,605)(0.40000,0.40000){26}{\makebox(0.4444,0.6667){\SetFigFont{7}{8.4}{rm}.}}
\multiput(340,615)(0.40000,-0.40000){26}{\makebox(0.4444,0.6667){\SetFigFont{7}{8.4}{rm}.}}
\put(350,605){\line(-1, 0){ 20}}
\put(355,595){\makebox(0,0)[lb]{\smash{\SetFigFont{10}{12.0}{rm}1}}}
\put(355,610){\makebox(0,0)[lb]{\smash{\SetFigFont{10}{12.0}{rm}0,N}}}
\put(295,520){\framebox(90,70){}}
\put(300,525){\framebox(80,60){}}
\put(316,575){\makebox(0,0)[lb]{\smash{\SetFigFont{10}{12.0}{rm}vStatusPane}}}
\put(300,540){\line( 1, 0){ 80}}
\put(300,570){\line( 1, 0){ 80}}
\put(340,605){\line( 0,-1){ 15}}
\multiput(385,505)(0.40000,0.40000){26}{\makebox(0.4444,0.6667){\SetFigFont{7}{8.4}{rm}.}}
\multiput(395,515)(0.40000,-0.40000){26}{\makebox(0.4444,0.6667){\SetFigFont{7}{8.4}{rm}.}}
\put(405,505){\line(-1, 0){ 20}}
\put(410,495){\makebox(0,0)[lb]{\smash{\SetFigFont{10}{12.0}{rm}1}}}
\put(410,510){\makebox(0,0)[lb]{\smash{\SetFigFont{10}{12.0}{rm}0,N}}}
\put(355,425){\framebox(80,60){}}
\put(350,420){\framebox(90,70){}}
\put(355,470){\line( 1, 0){ 80}}
\put(355,440){\line( 1, 0){ 80}}
\put(375,475){\makebox(0,0)[lb]{\smash{\SetFigFont{10}{12.0}{rm}myDialog}}}
\put(395,505){\line( 0,-1){ 15}}
\put( 55,615){\line( 0, 1){ 15}}
\put( 55,630){\line( 1, 0){125}}
\put(180,630){\line( 0, 1){ 10}}
\put(190,640){\line( 0,-1){ 20}}
\put(190,620){\line(-1, 0){ 40}}
\put(150,620){\line( 0,-1){  5}}
\put(200,725){\line( 0,-1){ 15}}
\put(280,645){\framebox(80,60){}}
\put(275,640){\framebox(90,70){}}
\put(280,690){\line( 1, 0){ 80}}
\put(280,660){\line( 1, 0){ 80}}
\put(200,755){\line( 0,-1){ 20}}
\put(180,755){\line( 0,-1){ 10}}
\put(180,745){\line(-1, 0){100}}
\put( 80,745){\line( 0,-1){ 10}}
\put(220,755){\line( 0,-1){ 10}}
\put(220,745){\line( 1, 0){100}}
\put(320,745){\line( 0,-1){ 10}}
\put(320,725){\line( 0,-1){ 15}}
\put( 80,725){\line( 0,-1){ 15}}
\multiput(260,800)(0.40000,0.40000){26}{\makebox(0.4444,0.6667){\SetFigFont{7}{8.4}{rm}.}}
\multiput(270,810)(0.40000,-0.40000){26}{\makebox(0.4444,0.6667){\SetFigFont{7}{8.4}{rm}.}}
\put(280,800){\line(-1, 0){ 20}}
\put(200,640){\line( 0,-1){ 20}}
\put(200,620){\line( 1, 0){ 45}}
\put(245,620){\line( 0,-1){  5}}
\put(210,640){\line( 0,-1){ 15}}
\put(210,625){\line( 1, 0){130}}
\put(340,625){\line( 0,-1){ 10}}
\put(220,640){\line( 0,-1){ 10}}
\put(220,630){\line( 1, 0){175}}
\put(395,630){\line( 0,-1){115}}
\put(290,695){\makebox(0,0)[lb]{\smash{\SetFigFont{10}{12.0}{rm}myAppModel}}}
\put(285,800){\makebox(0,0)[lb]{\smash{\SetFigFont{10}{12.0}{rm}= Aggregration}}}
\put(285,785){\makebox(0,0)[lb]{\smash{\SetFigFont{10}{12.0}{rm}     (has a)}}}
\end{picture}

\end{latexonly}
\normalfont\normalsize
\end{center}

\caption{Standard V Application} \label{fig:stdvapp}

\end{figure}

\begin{description}

\item [The Application]

In many ways, the heart of a Standard \V\ Application is the
application class derived from the \code{vApp} class. By
convention, this derived class is called \code{myApp} (but you
can use a different name if you want.) There will always be
exactly one instance of the \code{myApp} class.  The \code{myApp}
class acts as a coordinator between the windows that implement
the user interface (the views) and the objects and algorithms
that actually make up the application (the model).  The \code{myApp}
class will contain in a whole/part (or aggregation) relationship
the windows defined by the application, as well as any classes
needed to implement the application.

The \code{vApp} class has several utility methods that are
usually used unmodified, plus several methods that are usually
overridden by the \code{myApp} class.  These are described in the
section covering \code{vApp}.  In addition, your \code{myApp}
class will usually have several other programmer defined methods
used to interface the command windows with the application model.

\item [Windows and Canvases]

Each Standard \V\ Application will have at least one top level
window, and possible subwindows. These will usually be
command windows derived from the \code{vCmdWindow} class. Your
main derived class should be called \code{myCmdWindow}, and
include a constructor that defines a menu bar, a canvas, and
possible command and status bars. Of course, there will be a
corresponding destructor. The \code{.cpp} file will contain the
static definitions of the menu and any command and status bars.
It will also override the \code{WindowCommand} method of \code{vCmdWindow}
superclass. In your \code{WindowCommand} method, you will have a
\code{switch} with a \code{case} for each menu item and button
defined for the window.

Since a \code{vCmdWindow} contains different panes such as
\code{vMenuPanes}, \code{vCanvasPanes}, \code{vCommandPanes}, and
\code{vStatusPanes}, your top level command window object will
usually define the appropriate pointers to each of these objects
as required by the specific application. The \code{myCmdWindow}
constructor will then have a \code{new} for each pane used.

Each instance of a window will be built using a call to the
\code{vApp::NewAppWin} method. This allows the app object
to track windows, and control interaction between the app
model and the views represented by each window. 

Some applications need to open subwindows. These windows may or
may not use the same menu, command bar, and canvas as the top
level window. If they do, then they can use the same static
definitions used by the top level window. Subwindows may also
have their own menu, button, and canvas definitions.

\item [Canvases for Windows]

Since each window usually needs a canvas, you will usually derive
a canvas object from the \code{vCanvasPane} class. At
this point in the life of \V, there are only two possible kinds
of canvas.  The first is for graphics drawing, and is
derived directly from the \code{vCanvasPane} class.  The other
kind is a text canvas derived from the \code{vTextCanvasPane}
class. The derived class will define override methods required
for the user to interact with the canvas.

\item [Optional Dialogs]

Most applications will need dialogs -- either modeless or modal.
A Standard \V\ dialog consists of a \code{.cpp} file with the
static definition of the dialog commands, and the definitions
of methods derived from the \code{vDialog} class. These will
include a constructor and destructor, and a \code{DialogCommand}
override with a \code{switch} with a \code{case} for each command
defined for the dialog. Each \code{case} will have the code
required to carry out useful work.

The top level window (or the subwindow that
defines and uses the dialog) will create an instance of each
dialog it needs (via \code{new}). The constructor for the
dialog sets up the commands used for the dialog.

Typically, the top level window defines menu and button commands
that result in the creation of a dialog. The top level window
is thus usually responsible for invoking dialogs.

\item [Optional Modal Dialogs]

Modal dialogs are almost identical to modeless dialogs. The main
difference is how the dialog is invoked from the defining window.

\item [Menu, Command and Status Bars]

By definition, the look and feel of a \V\ application requires
a menu bar on the command window. A \V\ application also typically
has a command bar and a status bar, but these are not required.

\item [The Application Model]

Each application will need code to implement its data structures
and algorithms. The design of the application model is beyond the
scope of \V, but will usually be defined as a relatively
independent hierarchy contained by the \code{myApp} object.
Interaction between the application model and the various
views represented by \code{myCmdWindows} can be coordinated with
the \code{myAppWinInfo} class.

\item [The Makefile]

Each \V\ Standard Application should have an associated \code{makefile}
that can be used to compile and link the application.

\end{description}

Please note that while \V\ is object-oriented, the objects
represent real screen windows. Thus, it makes no sense for most
\index{copy constructors}
\index{object assignment (=)}
\V\ objects to support copy constructors or object assignment. If
you use one of these \V\ objects in a way requiring a copy
constructor or an assignment (fortunately, it is difficult to
contrive such an example), the code will generate a run time
error.

\section{Special V Applications}

\subsection*{Windows MDI/SDI}
\index{MDI model}\index{SDI model}

The basic standard \V application includes a command window
with a menu, a command bar, a canvas, and a status bar. While
this model suits most applications, there are some special
cases that \V supports.

First, on Windows, \V supports the standard Windows MDI model
(Multiple Document Interface)
by default. The MDI model consists of a parent window that
can contain several children canvases, each with a different
menu that changes in the main parent window when a child
gets focus. In practice, the menus are usually the same for
all children windows, and each window is used to hold a
new document or data object. One of the main advantages
of the MDI model is that each application has a main window
to distinguish it from other Windows applications, and
as many child windows as it needs to manipulate its data.

On X versions, there is no need for a special parent window.
Each time you open a new command window, you get a new window
on the X display.

The Windows MDI model forces some screen decorations that
are not appropriate for all applications. Thus, \V also supports
the standard Windows SDI model. The SDI model allows only one
canvas/command window combination. There is a parameter to
the \code{vApp} constructor that tells \V to use the SDI model.
This parameter is not used on the X version.

\subsection*{Canvasless, menuless V Application}

Sometimes an application needs just a command bar with no menu or
canvas. By setting the \code{simSDI} parameter to 1, and
supplying a width and height value to the \code{vApp}
constructor, \V allows this kind of simple interface. Instead of
adding a menu and a canvas as is done for normal \V apps, a
menuless and canvasless app just defines a command pane for the
command bar. The height and width are used to specify the height
and width of the application, and require different values for
Windows or X.

\section{A Tutorial Example V Application}
\index{tutorial example}
\label{sec:tutexamp}

Now that you've read about the parts of a standard \V\
application, it might be useful to go over a simple example of a
\V\ application. Appendix A contains the source code for a simple
\V\ application. The code is tutorial, and well commented. You
can read the code directly and get a good understanding of what
elements are required for a \V\ application. This section will
give a higher level overview of the code in Appendix A.

You should read this code, paying special attention to the comments.
Most of the information you need to build a typical \V\
application is explained in this code. This sample code is also
available on line under the \code{\tild/v/tutor} directory.
The source code of a slightly different standard \V\ application
is included the \code{\tild/v/examp} directory of the \V\
distribution.

The previous section suggested using \code{myApp} for names. This
tutorial uses a \code{t} prefix instead of \code{my}. You really
can use whatever names you want. It will help to be consistent,
however.

The code is broken down into five sections, corresponding to the
main application, the main window, a simple canvas, and  modal
and modeless dialogs. The source code for each of these parts is
included in Appendix A. The source code is extensively commented,
and the comments contain much detail on how you should structure
a \V\ application, so please read them carefully. The following
sections give a brief overview of each source file included in
the tutorial example.

\subsection*{The Base Application Class}
\Indextt{vApp}\index{base application class}

The file \code{tutapp.cpp} contains the overridden definitions of
the classes \code{NewAppWin}, \code{Exit}, \code{Close\-App\-Win},
\code{App\-Command}, and \code{KeyIn} methods. These examples
don't do much work, but are provided as a template for building
complete applications.

The single definition of the application (\code{static} \code{tutApp}
\code{tut\_App("TutorApp")}, and the \code{AppMain} main program
are also in this file. The initial window is created in \code{AppMain}
by calling \code{NewAppWin}.

One thing that can be difficult to grasp when using a
framework such as \V\ is understanding where the program starts,
and how you get things rolling. This happens in \code{tutapp.cpp},
so it is especially important to understand this piece of code.
The essential thing to understand is that C++ will invoke the
constructors of static objects before beginning execution of the
program proper. Thus, you declare a static instance of the
\code{vApp} object, and its constructor is used to initialize the
native GUI library and get things going. Your program will
\emph{not} have a \code{main} function (see \code{AppMain} in the
description of the \code{vApp} class for more details).

As with all files in the tutorial, each has a \code{.cpp} source
file, and its associated \code{.h} header file. All \V\ code has
been written using the coding guidelines given in Appendix B.
This includes the order of the declarations included in header
files.


\subsection*{The Command Window}
\index{window}\index{command window}\Indextt{vCmdWindow}

The file \code{tcmdwin.cpp} contains the code for the main
command window. Of particular interest are the definitions of
the main menu, command pane, and status pane. These panes are
defined and added to the window in the constructor.

There is also code to demonstrate handling keyboard and window
command events in the \code{KeyIn} and \code{WindowCommand} methods.
There is also a simple example of using the \code{vFileSelect}
utility class, as well as invoking modeless and modal dialogs.

\subsection*{The Canvas}
\index{canvas}\Indextt{vCanvas}

The file \code{tcanvas.cpp} contains the code for the canvas.
This is a really simple canvas example which supports
drawing a few lines. This class handles redrawing after
expose events very simply, but demonstrates what must be done
in general.


\subsection*{A Modeless Dialog}
\index{dialog}

The file \code{tdialog.cpp} contains the code for a modeless
dialog. There are just a few example buttons, check boxes, and
radio buttons. The \code{DialogCommand} methods demonstrates how
to handle commands from a dialog.


\subsection*{A Modal Dialog}

The file \code{tmodal.cpp} contains the code for a modal dialog.
The definition of a modal dialog is nearly identical to a modeless
dialog. The main difference is how they are invoked, which is
shown in the \code{tcmdwin.cpp} code.


\subsection*{The Makefile}
\index{makefile}

The file \code{makefile} contains a sample Unix-style make file.
This version is for Gnu make, which has features different
than some other flavors of make. It should still serve as a
decent example.
