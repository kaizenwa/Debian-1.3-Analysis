
%***********************************************************************
%***********************************************************************
%***********************************************************************

\chapter {Standard V Values}

This chapter covers standard predefined values.

The classes and objects covered in this chapter include:

\begin{description}
	\item[Predefined ItemVals] A useful collection of predefined values.
Most are useful for defining dialogs, buttons, and menus.
\end{description}
%----------------------------------------------------------------------

\Class{Predefined ItemVals}
\index{predefined values}

A useful collection of predefined values. Most are useful for defining
dialogs, buttons, and menus.

\subsection* {Synopsis}

\begin{description}
	\item [Header:] \code{<v/v\_defs.h>}
\end{description}

\subsection* {Description}

When defining dialogs, menus, and command bars, you are required
to provide an id for each item.  There are many common operations
used in GUI designs, and \V\ provides various predefined values
for building your programs. The natural interpretation of most
of these values should be obvious, and the descriptions are kept
to a minimum. Most of the definitions describe the accepted practice
for menu or button items with the given title. While these \code{ItemVal}s
can be used anywhere, some have ``standard'' usage.

\subsection* {Control Values}

\Param{M\_About} Shows an informative message about current application.

\Param{M\_All} Select all.

\Param{M\_Cancel} Cancel. Usually used with a dialog. \V\ will
automatically reset dialog commands to their original state when
a \code{M\_Cancel} is selected from a \code{vDialog} descended
object.

\Param{M\_Clear} Used to clear a screen.

\Param{M\_Close} Used to close a file. The user is usually
prompted to save or ignore changes if any were made to the file. This
is usually not used to close a menu.

\Param{M\_Copy} Copy the highlighted text or item, and save into
the clipboard.

\Param{M\_Cut} Cut the highlighted text or item from the file, and
usually save into the clipboard.

\Param{M\_Delete} Delete the selected item or text -- usually
does not copy into the clipboard.

\Param{M\_Done} Done with operation.

\Param{M\_Edit} Typically a menu bar button to pulldown an edit menu.

\Param{M\_Exit} Exit from the program -- checking to see if files
need to be saved, of course.

\Param{M\_File} Typically a menu bar button to pulldown a file menu.

\Param{M\_Find} Find a pattern.

\Param{M\_FindAgain} Find pattern again.

\Param{M\_Font} Typically a menu bar button to pulldown a font menu.

\Param{M\_FontSelect} Select a font. (This is different from the
\code{M\_Font} value in that \code{M\_Font} is intended as a main
menu bar item, while this one is for a pulldown menu.

\Param{M\_Format} Typically a menu bar button to pulldown a format menu, which
allows the user to select formatting options.

\Param{M\_Help} Show help.

\Param{M\_Insert} Typically a menu bar button to pulldown an insert menu.

\Param{M\_Line} \code{M\_Line} is one of a few of these values
that gets special treatment by the system.  It is required for
defining line separators in menus.

\Param{M\_New} Used to create a new file.

\Param{M\_No} Answer No.

\Param{M\_None} Select none.

\Param{M\_OK} OK, accept operation or information. Causes return
from dialog.

\Param{M\_Open} Used to open an existing file.

\Param{M\_Options} Typically a menu bar button to pulldown an options menu.

\Param{M\_Paste} Paste the contents of the clipboard into the insertion
point of the current file or item.

\Param{M\_Preferences} Set preferences.

\Param{M\_Print} Print current file.

\Param{M\_PrintPreview} On screen preview how the current file would look
if printed.

\Param{M\_Replace} Replace pattern.

\Param{M\_Save} Used to save current file in its current name.

\Param{M\_SaveAs} Save current file under new name.

\Param{M\_Search} Typically a menu bar button to pulldown a search menu.

\Param{M\_SetDebug} Set debug stuff.

\Param{M\_Test} Typically a menu bar button to pulldown a test menu.

\Param{M\_Tools} Typically a menu bar button to pulldown a tools menu.

\Param{M\_UnDo} Undo the last action.

\Param{M\_View} Typically a menu bar button to pulldown a view menu, which
allows the user to select different views of the document.

\Param{M\_Window} Typically a menu bar button to pulldown a window menu, which
lets the user select different windows.

\Param{M\_Yes} Answer Yes.

\Class{Version Values}
\index{version values}

A useful collection of predefined values to determine the
version of V and the platform.

\subsection* {Synopsis}

\V defines several values useful for determining the revision of
\V, and the platform \V is compiled on.

\begin{description}
	\item [Header:] \code{<v/v\_defs.h>}
\end{description}

\subsection* {Version Values}

\Param{V\_VersMajor} The major version of \V, such as 1.

\Param{V\_VersMinor} The minor release of \V, such as 12.

\Param{V\_Version} A text string describing the version of
\V, such as \emph{V 1.12 - 8/4/96}.

\Param{V\_VersionX} Defined if the is the standard X version of \V.

\Param{V\_VersionMotif} Defined if the Motif version of \V.

\Param{V\_VersionWindows} Defined if the Windows version of \V.

\Param{V\_VersionWin95} Defined if the Windows 95 version of \V.

\Param{V\_VersionOS2} Defined for the OS2 version of \V.
