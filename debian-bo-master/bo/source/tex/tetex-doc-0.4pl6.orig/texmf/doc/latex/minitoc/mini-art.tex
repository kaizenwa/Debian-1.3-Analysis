%%%%%%%%%%%%%%%%%%%%% A example file (differs from previous versions)
% mini-art.tex
% This file contains a set of tests for the minitoc.sty version #27
% style option file. You can alter most of parameters to test.
% article (\section must be defined)
\documentclass[12pt,a4paper]{article}
\usepackage{minitoc}
\setcounter{secnumdepth}{5}   % depth of numbering of sectionning commands
\setcounter{tocdepth}{3}           % depth of table of contents
\setlength{\stcindent}{24pt}       % indentation of secttocs, default
\renewcommand{\stcfont}{\small\rm} % font for secttocs, default
%\renewcommand{\stcSSfont}{\small\sf} % font for secttocs, subsections
% you can make experiments with \stcSSSfont, \stcPfont and \stcSPfont
% but it is ``fontomania''...
\raggedbottom                   % or \flushbottom, at your choice
%\parskip=12pt                  % a bug about \parskip has been fixed
%%% TEST: uncomment the 3 next lines to test
%%% resetting section number in each part
%\makeatletter
%\@addtoreset{section}{part}
%\makeatother
%%% END TEST
\begin{document}
\dosecttoc
\dosectlof[c]                   % center title of sectlof's
\doparttoc                      % test of parttoc/partlof stuff
\dopartlof                      % added in version #15
\faketableofcontents            % or \tableofcontents
\fakelistoffigures              % to check compatibility
\part{First Part}
\parttoc
\partlof[r]
\twocolumn                      % the secttoc in twocolumn layout is ugly,
                                % but works. Ideas to make it better?
\section{AAAAA}                 % a section with a lot of sections
\secttoc[r]                     % secttoc title on the right
bla bla bla bla bla bla bla bla bla bla bla
bla bla bla bla bla bla bla bla bla bla bla
\subsection{S1}
bla bla bla bla bla bla bla bla bla bla bla
bla bla bla bla bla bla bla bla bla bla bla
\subsection{S2}
bla bla bla bla bla bla bla bla bla bla bla
bla bla bla bla bla bla bla bla bla bla bla
\subsection{S3}
bla bla bla bla bla bla bla bla bla bla bla
bla bla bla bla bla bla bla bla bla bla bla
\subsection*{S4}
\addcontentsline{toc}{starsubsection}{*S4*}
bla bla bla bla bla bla bla bla bla bla bla
bla bla bla bla bla bla bla bla bla bla bla
\subsection{S5}
bla bla bla bla bla bla bla bla bla bla bla
bla bla bla bla bla bla bla bla bla bla bla
\subsection{S6}
bla bla bla bla bla bla bla bla bla bla bla
bla bla bla bla bla bla bla bla bla bla bla
\subsection{S7}
bla bla bla bla bla bla bla bla bla bla bla
bla bla bla bla bla bla bla bla bla bla bla
\subsection{S8}
bla bla bla bla bla bla bla bla bla bla bla
bla bla bla bla bla bla bla bla bla bla bla
\subsection{S9}
bla bla bla bla bla bla bla bla bla bla bla
bla bla bla bla bla bla bla bla bla bla bla
\subsection{S10}
bla bla bla bla bla bla bla bla bla bla bla
bla bla bla bla bla bla bla bla bla bla bla
\subsection{S11}
bla bla bla bla bla bla bla bla bla bla bla
bla bla bla bla bla bla bla bla bla bla bla
\subsection{S12}
bla bla bla bla bla bla bla bla bla bla bla
bla bla bla bla bla bla bla bla bla bla bla
\subsection{S13}
bla bla bla bla bla bla bla bla bla bla bla
bla bla bla bla bla bla bla bla bla bla bla
\subsection{S14}
bla bla bla bla bla bla bla bla bla bla bla
bla bla bla bla bla bla bla bla bla bla bla
\subsection{S15}
bla bla bla bla bla bla bla bla bla bla bla
bla bla bla bla bla bla bla bla bla bla bla
\subsection{S16}
bla bla bla bla bla bla bla bla bla bla bla
bla bla bla bla bla bla bla bla bla bla bla
\subsection{S17}
bla bla bla bla bla bla bla bla bla bla bla
bla bla bla bla bla bla bla bla bla bla bla
\subsection{S18}
bla bla bla bla bla bla bla bla bla bla bla
bla bla bla bla bla bla bla bla bla bla bla
\subsection{S19}
bla bla bla bla bla bla bla bla bla bla bla
bla bla bla bla bla bla bla bla bla bla bla
\subsection{S20}
bla bla bla bla bla bla bla bla bla bla bla
bla bla bla bla bla bla bla bla bla bla bla
\subsection{S21}
bla bla bla bla bla bla bla bla bla bla bla
bla bla bla bla bla bla bla bla bla bla bla
\subsection{S22}
bla bla bla bla bla bla bla bla bla bla bla
bla bla bla bla bla bla bla bla bla bla bla
\subsection{S23}
bla bla bla bla bla bla bla bla bla bla bla
bla bla bla bla bla bla bla bla bla bla bla
\subsection{S24}
bla bla bla bla bla bla bla bla bla bla bla
bla bla bla bla bla bla bla bla bla bla bla
\subsection{S26}
bla bla bla bla bla bla bla bla bla bla bla
bla bla bla bla bla bla bla bla bla bla bla
\subsection{S27}
bla bla bla bla bla bla bla bla bla bla bla
bla bla bla bla bla bla bla bla bla bla bla
\subsection{S28}
bla bla bla bla bla bla bla bla bla bla bla
bla bla bla bla bla bla bla bla bla bla bla
\subsection{S29}
bla bla bla bla bla bla bla bla bla bla bla
bla bla bla bla bla bla bla bla bla bla bla
\subsection{S30}
bla bla bla bla bla bla bla bla bla bla bla
bla bla bla bla bla bla bla bla bla bla bla
\onecolumn                              % back to one column
\section{BBBBB}
\secttoc
\bigskip                                % put some skip here
\sectlof                                % a sectlof
bla bla bla bla bla bla bla bla bla bla bla
bla bla bla bla bla bla bla bla bla bla bla
\subsection{T1}
bla bla bla bla bla bla bla bla bla bla bla
bla bla bla bla bla bla bla bla bla bla bla
\begin{figure}[t]        % tests compatibility with floating bodies
\setlength{\unitlength}{1mm}
\begin{picture}(100,50)
\end{picture}
\caption{F1}             % (I have not tested tables, but it is similar)
\end{figure}
\clearpage
\subsubsection[tt1]{TT1}    % tests optionnal arg. of a sectionning command
bla bla bla bla bla bla bla bla bla bla bla
bla bla bla bla bla bla bla bla bla bla bla
\paragraph{TTT1}
bla bla bla bla bla bla bla bla bla bla bla
bla bla bla bla bla bla bla bla bla bla bla
\subparagraph{TTTT1}
bla bla bla bla bla bla bla bla bla bla bla
bla bla bla bla bla bla bla bla bla bla bla
\begin{figure}
\setlength{\unitlength}{1mm}
\begin{picture}(100,50)
\end{picture}
\caption[f2]{F2}         % tests optionnal arg. of a caption
\end{figure}
\subsection{T2}
bla bla bla bla bla bla bla bla bla bla bla
bla bla bla bla bla bla bla bla bla bla bla
\section*{CCCCC}         % tests a pseudo-section. should have no secttoc
bla bla bla bla bla bla bla bla bla bla bla
bla bla bla bla bla bla bla bla bla bla bla
\subsection{U1}
bla bla bla bla bla bla bla bla bla bla bla
bla bla bla bla bla bla bla bla bla bla bla
\subsubsection{UU1}
bla bla bla bla bla bla bla bla bla bla bla
bla bla bla bla bla bla bla bla bla bla bla
\paragraph{UUU1}
bla bla bla bla bla bla bla bla bla bla bla
bla bla bla bla bla bla bla bla bla bla bla
\subparagraph{UUUU1}
bla bla bla bla bla bla bla bla bla bla bla
bla bla bla bla bla bla bla bla bla bla bla
\subsection{U2}
bla bla bla bla bla bla bla bla bla bla bla
bla bla bla bla bla bla bla bla bla bla bla
\part{Second Part}
\parttoc
\partlof[c]
%                        % the following section should have no secttoc,
\section{DDDDD}          % but if you uncomment \secttoc,
%\secttoc
                         % the secttoc appears
bla bla bla bla bla bla bla bla bla bla bla
bla bla bla bla bla bla bla bla bla bla bla
\subsection{V1}
bla bla bla bla bla bla bla bla bla bla bla
bla bla bla bla bla bla bla bla bla bla bla
\subsubsection{VV1}
bla bla bla bla bla bla bla bla bla bla bla
bla bla bla bla bla bla bla bla bla bla bla
\paragraph{VVV1}
bla bla bla bla bla bla bla bla bla bla bla
bla bla bla bla bla bla bla bla bla bla bla
\subparagraph{VVVV1}
bla bla bla bla bla bla bla bla bla bla bla
\begin{figure}[t]        % tests compatibility with floating bodies
\setlength{\unitlength}{1mm}
\begin{picture}(100,50)
\end{picture}
\caption{F3}             % (I have not tested tables, but it is similar)
\end{figure}
bla bla bla bla bla bla bla bla bla bla bla
\subsection{V2}
bla bla bla bla bla bla bla bla bla bla bla
bla bla bla bla bla bla bla bla bla bla bla
\section{EEEEE}                 % this section should have a secttoc
{%                              % left brace, see below
\setcounter{secttocdepth}{3}    % depth of sect table of contents;
                                % try with different values.
\secttoc
}                               % right brace
% this pair of braces is used to keep local the change
% on secttocdepth.
bla bla bla bla bla bla bla bla bla bla bla
bla bla bla bla bla bla bla bla bla bla bla
\subsection{W1}                    % with the given depth
bla bla bla bla bla bla bla bla bla bla bla
bla bla bla bla bla bla bla bla bla bla bla
\subsubsection{WW1}
bla bla bla bla bla bla bla bla bla bla bla
bla bla bla bla bla bla bla bla bla bla bla
\paragraph{WWW1}
bla bla bla bla bla bla bla bla bla bla bla
\begin{figure}[t]        % tests compatibility with floating bodies
\setlength{\unitlength}{1mm}
\begin{picture}(100,50)
\end{picture}
\caption{F4}             % (I have not tested tables, but it is similar)
\end{figure}
bla bla bla bla bla bla bla bla bla bla bla
\subparagraph{WWWW1}
bla bla bla bla bla bla bla bla bla bla bla
bla bla bla bla bla bla bla bla bla bla bla
bla bla bla bla bla bla bla bla bla bla bla
bla bla bla bla bla bla bla bla bla bla bla
\subsection{W2}
bla bla bla bla bla bla bla bla bla bla bla
bla bla bla bla bla bla bla bla bla bla bla
\appendix
\section{Comments}
\secttoc
\subsection{C1}
\subsection{C2}
\subsection{C3}
\begin{figure}[t]        % tests compatibility with floating bodies
\setlength{\unitlength}{1mm}
\begin{picture}(100,50)
\end{picture}
\caption{F5}             % (I have not tested tables, but it is similar)
\end{figure}
\subsection{C4}
\section{Evolution}
\secttoc
\subsection{D1}
\subsection{D2}
\subsection{D3}
\subsection{D4}
\end{document}
