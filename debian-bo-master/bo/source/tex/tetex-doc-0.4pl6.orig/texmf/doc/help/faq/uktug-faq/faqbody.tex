\def\fileversion{2.0i}    \def\filedate{1997/01/02}
%
% The above line define the file version, and must remain the first
% line with any `assignment' in the file, or things will blow up
% nastily
%
\ifnotreadCTAN
%
% lists of CTAN labels
%
% ... directories
  \input dirctan.tex
%
% ... files
  \input filectan.tex
%
% don't need to read them again, though
  \notreadCTANfalse
\fi

\section{Introduction}

This article was prepared by the Committee of the \acro{UK} \TeX{} Users
Group (\acro{UK}~\acro{TUG})\begin{footnoteenv}
  For 1996--97: Peter Abbott, Kaveh Bazargan, David Carlisle, Malcolm
  Clark, Robin Fairbairns, Hewlett, Alan Jeffrey and Sebastian
  Rahtz
\end{footnoteenv}
as a development of a regular posting to the \emph{Usenet} newsgroup
\Newsgroup|comp.text.tex| that was maintained for some time by
Bobby Bodenheimer (\Email|bobby@hot.caltech.edu|).

Usenet is a mechanism for exchanging articles between people who share
interests or needs\begin{footnoteenv}
Usenet, as its name implies, is a means of using some sort of network;
in the earliest days the network was made by stringing together a
series of telephone lines, but nowadays Usenet is most often carried
over the Internet
\end{footnoteenv};
a newsgroup is an area within Usenet carrying a particular
class of articles.  Since a common sort of article asks for help,
advice or information, and since certain of these questions are
regularly repeated (often with monotonous regularity), some
public-spirited souls took to writing articles which listed
``Frequently Asked Questions'' and answers to them.  Many members
of \acro{UK}~\acro{TUG} do not have access to Usenet, but could be expected to value
the answers about \TeX{} that have accumulated over the years; so we
decided to update the list and publish it in \BV{}; we are
grateful to Bobby for his permission to use his article in this way.
As a \emph{quid pro quo}, we are making the source of the article
freely available, and it can be compiled by anyone who runs a
production
\htmlignore
\LaTeXe{} (\Qref{}{latex2e}),
\endhtmlignore
\begin{htmlversion}
  \Qref{LaTeXe}{latex2e},
\end{htmlversion}
and has the required fonts.
It is the committee's hope that it will also be possible for the
content of this article to feed back to the world-wide \TeX{}
community via Bobby's regular posting.

\htmlignore
In addition, a translation of the article is available on the
World-Wide Web, via URL
\URL|http://www.cogs.susx.ac.uk/cgi-bin/texfaq2html?introduction=yes|%
\begin{footnoteenv}
This is a temporary URL; a final home for the document is to be
provided in due course
\end{footnoteenv}
\endhtmlignore

We have rearranged Bobby's article quite a lot, and have added new
questions and answers on the basis of our experience of answering
questions about \TeX{}, writing documents in \TeX{}, and developing
macros for \TeX{}, over the years.  We have also pruned it to take
account of the changes that have happened in the world of \TeX{} since
Bobby first started.

The committee is grateful for help and advice, from the following
outside its number:
Barbara Beeton,
Karl Berry,
Damian Cugley,
Michael Downes,
John Hobby,
Berthold Horn,
Werner Icking,
Ted Nieland,
Pat Rau,
Piet van Oostrum,
Joachim Schrod,
Philip Taylor,
Ulrik Vieth,
Rick Zaccone and
Reinhard Zierke.

Further, Rosemary Bailey, Jonathan Fine and Chris Rowley were members
of the committee during the period 1993--95, during which the
entreprise of developing this FAQ was conceived and its first version
was published, and we are grateful to them for their contributions to
it.

\htmlignore
\subsection*{Finding the Files}

Unless otherwise specified, all files mentioned in this article are
available from a \acro{CTAN} archive, or from one of their mirrors.
\Qref[Question]{}{Q-archives} % this one doesn't need anchor text
gives details of the \acro{CTAN} archives, and how to retrieve files from
them.  If you don't have access to the Internet,
\Qref[question]{}{Q-CD} tells you of sources of \CDROM{}s that offer
snapshots of the archives.

The reader should also note that the first directory name of the path
name of every file on \acro{CTAN} has been elided from what follows, for the
simple reason that it's always the same (\path|tex-archive/|).

To avoid confusion, we've also elided the full
stop\begin{footnoteenv}
`Full stop' (British English)==`period' (American English)
\end{footnoteenv}
from the end of any sentence whose last item is a path name (note that
such sentences only occur at the end of paragraphs).  Though the path
names are set in a different font from running text, it's not easy to
distinguish the font of a single dot!
\endhtmlignore

%%%%%%%%%%%%%%%%%%%%%%%%%%%%%%%%%%%%%%%%%%%%%%%%%%%%%%%%%%%%%%%%%

\section{The Background}

\Question{What is \TeX{}?}

\TeX{} is a typesetting system written by Donald E.~Knuth, who
says in the Preface to his book on \TeX{}
(see \Qref[question]{TeX-related books}{Q-books})
that it is ``\emph{intended for the creation of beautiful books---and especially for books that contain a lot of mathematics}''.

Knuth developed a system of `literate programming' to write \TeX{},
and he provides the literate (\acro{WEB}) source of \TeX{} free of charge,
together with tools for processing the |web| source into something
that can be compiled and something that can be printed; there's never
any mystery about what \TeX{} does.  Furthermore, the \acro{WEB} system
provides mechanisms to port \TeX{} to new operating systems and
computers; in order that one may have some confidence in the ports,
Knuth supplied a test by means of which one may judge the fidelity of
a \TeX{} system.  \TeX{} and its documents are therefore highly
portable.

\TeX{} is a macro processor, and offers its users a powerful
programming capability.  For this reason, \TeX{} on its own is a
pretty difficult beast to deal with, so Knuth provided a package of
macros for use with \TeX{} called |plain| \TeX{}; |plain| \TeX{} is
effectively the minimum set of macros one can usefully employ with
\TeX{}, together with some demonstration versions of higher-level
commands (the latter are better regarded as models than used as-is).
When people say they're ``programming in \TeX{}'', they usually mean
they're programming in |plain| \TeX{}.

\Question[tex-pronounce]{How should I pronounce ``\TeX{}''?}

The `X' stands for the Greek letter
\htmlignore
Chi ($\chi$),
\endhtmlignore
\begin{htmlversion}
Chi,
\end{htmlversion}
and is pronounced by English-speakers either a bit like the `ch' in
`loch' ([x] in the IPA) or like `k'.  It definitely is not pronounced
`ks'.

\Question[Q-MF]{What is \MF{}?}

\MF{} was written by Knuth as a companion to \TeX{}; whereas \TeX{}
defines the layout of glyphs on a page, \MF{} defines the shapes of
the glyphs and the relations between them.  \MF{} details the sizes of
glyphs, for \TeX{}'s benefit, and details the rasters used to
represent the glyphs, for the benefit of programs that will produce
printed output as post processes after a run of \TeX{}.

\MF{}'s language for defining fonts permits the expression of several
classes of things: first (of course), the simple geometry of the
glyphs; second, the properties of the print engine for which the
output is intended; and third, `meta'-information which can
distinguish different design sizes of the same font, or the difference
between two fonts that belong to the same (or related) families.

Knuth (and others) have designed a fair range of fonts using \MF{},
but font design using \MF{} is much more of a minority skill than is
\TeX{} macro-writing.  The complete \TeX{}-user nevertheless needs to
be aware of \MF{}, and to be able to run \MF{} to generate personal
copies of new fonts.

\Question[Q-MP]{What is \MP{}?}

The \MP{} system implements a picture-drawing language very much like
that of \MF{} except that it outputs PostScript commands instead of
run-length-encoded bitmaps.  \MP{} is a powerful language for
producing figures for documents to be printed on PostScript printers.
It provides access to all the features of PostScript and it includes
facilities for integrating text and graphics.  (Knuth tells us that he
uses nothing else for diagrams in text that he is writing.)

Much of \MP{}'s source code was copied from \MF{}'s sources with
Knuth's permission.

\Question{What is \LaTeX{}?}

\LaTeX{} is a \TeX{} macro package, originally written by Leslie Lamport, that
provides a document processing system.  \LaTeX{} allows markup to
describe the structure of a document, so that the user
need not think about presentation. By using document classes and
add-on packages, the same document can be produced in a variety of
different layouts.

Lamport says that \LaTeX{}
``\emph{represents a balance between functionality and ease of use}''.
This shows itself as a continual conflict that leads to
the need for such as the present article: \LaTeX{} \emph{can} 
meet most user requirements, but finding out \emph{how} is often
tricky.

\htmlignore
\begingroup\boldmath
\endhtmlignore
\Question{How should I pronounce ``\LaTeX{}(\twee{})''?}
\htmlignore
\endgroup\par
\endhtmlignore

Lamport never recommended how one should pronounce \LaTeX{}, but a lot
of people pronounce it `Lay \TeX{}' or perhaps `Lah \TeX{}' (with
\TeX{} pronounced as the program itself; see
\Qref[question]{the rules for TeX}{tex-pronounce}).

The `epsilon' in `\LaTeXe{}' is supposed to be suggestive of a small
improvement over the old \LaTeXo{}.  Nevertheless, most people
pronounce the name as `\LaTeX{}-two-ee'.
%... whereas Damian Cugley suggested (and we retain for the amusement
%... of those who read the source ;-)
%The `e' in `\LaTeXe{}' might look like a lowered one-stroke Greek
%lower case epsilon (when typeset) to mere mortals such as you or me, but
%it is in fact a lower case `e', and is pronounced like the name
%of the letter.  Some people pronounce `2e' as `twee'.

\htmlignore
\par
\endhtmlignore
\Question{Should I use \texttt{plain} \TeX{} or \LaTeX{}?}

There's no straightforward answer to this question.  Many people swear
by |plain| \TeX{}, and produce highly respectable documents using it
(Knuth is an example of this, of course).  But equally, many people
are happy to let someone else take the design decisions for them,
accepting a small loss of flexibility in exchange for a saving of
brain power.

The arguments around this topic can provoke huge amounts of noise and
heat, without offering much by way of light; your best bet is to find
out what those around you are using, and to go with the crowd.  Later
on, you can always switch your allegiance; don't bother about it.

If you are preparing a manuscript for a publisher or journal, ask them
what markup they want before you
develop your own; many big publishers have developed their own
\LaTeX{} styles for journals and books, and insist that authors stick
closely to their markup.

\htmlignore
\begingroup\boldmath
\endhtmlignore
\Question{What are the \acro{AMS} packages (\AMSTeX{}, \emph{etc}.)?}
\htmlignore
\endgroup\par
\endhtmlignore

\AMSTeX{} is a \TeX{} macro package, originally written by Michael Spivak for
the American Mathematical Society (\acro{AMS}) during 1983--1985. It
is described in
``\emph{The Joy of \TeX{}}'' by Michael D.~Spivak (second edition,
\acro{AMS}, 1990, \ISBN{0-821-82997-1}).  It is based on |plain| \TeX{}, but
provides many
features for producing more professional-looking maths formulas with
less burden on authors.  It pays attention to the finer details of
sizing and positioning that mathematical publishers care about. The
aspects covered include multi-line displayed equations, equation
numbering, ellipsis dots, matrices, double accents, multi-line
subscripts, syntax checking (faster processing on initial
error-checking \TeX{} runs), and other things.

As \LaTeX{} increased in popularity, authors asked to submit papers to
the \acro{AMS} in \LaTeX{}, and so the \acro{AMS} developed
\AMSLaTeX{}, which is a
collection of \LaTeX{} packages and classes that offer authors most of
the functionality of \AMSTeX{}.
\checked{RAB}{1994/11/12} % edited by RF; input from Michael Downes, too

%\htmlignore
%\begingroup\boldmath
%\endhtmlignore
%\Question{What is \protect\LAMSTeX{}?}
%\htmlignore
%\endgroup\par
%\endhtmlignore

%\LAMSTeX{} was Michael Spivak's fusion of \AMSTeX{} and \LaTeX{}.  Its
%strong points are:
%\begin{itemize}
%\item a commutative diagram package that produces very pleasing
%output;
%\item a separate program \ProgName|dvipaste| for producing complex
%tables separately from the main document (thus avoiding problems from
%exceeding \TeX{}'s main memory capacity); and
%\item extensive control at the user level over the formatting of
%automatically-generated numbers.
%\end{itemize}

%However, \AMSLaTeX{} had come out by the time \LAMSTeX{} was released,
%so that \LAMSTeX{} never really caught on.
%\checked{RAB}{1994/11/12} % edited by RF; input from Michael Downes, too

\Question[Q-eplain]{What is \Eplain{}?}

The \Eplain{} macro package expands on and extends the
definitions in |plain| \TeX{}.  \Eplain{} is not intended to provide
``generic typesetting capabilities'', as do \LaTeX{} or
\htmlignore
Texinfo (\Qref{}{Q-texinfo}).
\endhtmlignore
\begin{htmlversion}
\Qref{Texinfo}{Q-texinfo}.
\end{htmlversion}
Instead, it provides definitions that are intended to be useful
regardless of the high-level commands that you use when you actually
prepare your manuscript.

For example, \Eplain{} does not have a command \cs|section|,
which would format section headings in an ``appropriate'' way, as
\LaTeX{}'s \cs|section|.  The philosophy of \Eplain{} is that
some people will always need or want to go beyond the macro designer's
idea of ``appropriate''.  Such canned macros are fine~--- as long as you
are willing to accept the resulting output.  If you don't like the
results, or if you are trying to match a different format, you are out
of luck.

On the other hand, almost everyone would like capabilities such as
cross-referencing by labels, so that you don't have to put actual page
numbers in the manuscript.  Karl Berry, the author of \Eplain{}, says
he is not aware of any generally available macro packages that do
not force their typographic style on an author, and yet provide
such capabilities.

\Question{What is Lollipop?}

Lollipop is a macro package written by Victor Eijkhout; it was used in
the production of his book ``\emph{\TeX{} by Topic}'' (see
\Qref[question]{Tex-related books}{Q-books}).  The manual says of
it:
\begin{quote}
  Lollipop is `\TeX{} made easy'. Lollipop is a macro package that
  functions as a toolbox for writing \TeX{} macros.  It was my
  intention to make macro writing so easy that implementing a fully
  new layout in \TeX{} would become a matter of less than an hour for
  an average document, and that it would be a task that could be
  accomplished by someone with only a very basic training in \TeX{}
  programming.

  Lollipop is an attempt to make structured text formatting available
  for environments where previously only \WYSIWYG{} packages could be
  used because adapting the layout is so much more easy with them than
  with traditional \TeX{} macro packages.
\end{quote}

The manual goes on to talk of ambitions to ``capture some of the
\LaTeX{} market share''; it's a very witty package, but little sign of
\htmlignore
it taking over from \LaTeX{} is detectable\dots\@
\endhtmlignore
\begin{htmlversion}
it taking over from \LaTeX{} is detectable\dots{}
\end{htmlversion}
An article about Lollipop appeared in \TUGboat{} 13(3).

\Question[Q-texinfo]{What is Texinfo?}

Texinfo is a documentation system that uses one source file to
produce both on-line information and printed output.  So
instead of writing two different documents, one for the on-line help
and the other for a typeset manual, you need write only one
document source file.  When the work
is revised, you need only revise one document.  You can read the
on-line information, known as an ``Info file'', with an Info
documentation-reading program.  By convention, Texinfo source file
names end with a |.texi| or |.texinfo| extension.
You can write and format Texinfo files into Info files within \acro{GNU}
\ProgName|emacs|, and read them using the \ProgName|emacs| Info
reader.  If you do not have \ProgName|emacs|, you can format Texinfo
files into Info files using \ProgName|makeinfo| and read them using
\ProgName|info|.

The Texinfo distribution, including a set of \TeX{} macros for
formatting Texinfo files is available as \CTANref{texinfo-dist} (also
available as a \texttt{.zip} file \CTANref{texinfo-dist-zip}).

\Question{If \TeX{} is so good, how come it's free?}

It's free because Knuth chose to make it so.  He is nevertheless
apparently happy
that others should earn money by selling \TeX{}-based services and
products. While several valuable \TeX{}-related tools and packages are
offered subject to restrictions imposed by the \acro{GNU} General Public
Licence (`Copyleft'), \TeX{} itself is not subject to Copyleft.

There are commercial versions of \TeX{} available; for some users,
it's reassuring to have paid support.  What is more, some of the
commercial implementations
have features that are not available in free versions.  (The
reverse is also true: some free implementations have features
not available commercially.)

Usually, this article does not describe commercial
\htmlignore
versions;  \Qref[Question]{}{Q-commercial} lists the major vendors.
\endhtmlignore
\begin{htmlversion}
versions; see \Qref{major vendors}{Q-commercial} for some details.
\end{htmlversion}

\Question[tex-future]{What is the future of \TeX{}?}

Knuth has declared that he will do no further development of \TeX{};
he will continue to fix any bugs that are reported to him (though
bugs are rare).  This decision was made soon after
\TeX{} version~3.0 was released; at each bug-fix release
the version number acquires one more digit, so that it tends to the
limit~$\pi$ (at the time of writing, Knuth's latest release is version
3.14159).  Knuth wants \TeX{} to be frozen at version~$\pi$ when he
dies; thereafter, no further changes may be made to Knuth's source.
(A similar rule is applied to \MF{}; its version number tends to the
limit~$e$, and currently stands at 2.718.)

There are projects (some of them long-term
projects: see, for example,
\Qref[question]{the LaTeX3 project}{LaTeX3})
%\Qref[and]{and the SGML work}{Q-SGML})
to build substantial
new macro packages based on \TeX{}.  For the even longer term, there
are various projects to build a \emph{successor} to \TeX{}; see
\Qref[questions]{the Omega project}{Q-omega} and \Qref[]{NTS}{Q-NTS}.

%\Question{Why isn't \TeX{} \WYSIWYG{}?}
%
% hmmm; had a request for this, but can't think what to write

\Question[Q-TUG]{What are \acro{TUG} and \textsl{TUGboat}?}

\acro{TUG} is the \TeX{} Users Group.  \textsl{TUGboat} is
\acro{TUG}'s main journal,
containing useful articles about \TeX{} and \MF{}. \acro{TUG} also produces a
newsletter for members (\TeX{} and \acro{TUG} News), organises a yearly
conference, runs training courses, sells almost all \TeX{}-related
books, and distributes \TeX{}-related microcomputer software on disk. 
\acro{TUG} has a Technical Council to  coordinate \TeX{}-related developments
(\Qref{TUG Technical Working Groups}{Q-TUGTC}).
Enquiries should be directed to:
\begin{quote}
  \TeX{} Users Group\\
  1850 Union Street, \#1637\\
  San Francisco CA 94123\\
  \acro{USA}\\[.25\baselineskip]
  Tel: (+1) 805-963-1338\\
  Fax: (+1) 805-963-8358\\
  Email: \Email|tug@tug.org|\\
  Web: \URL|http://www.tug.org/|\\
  \acro{CTAN} details: \CTANref{tug}
\end{quote}

\Question[Q-othergroups]{Are there nationally-based user groups, too?}

The following groups publish their membership (\emph{etc}.) information
electronically on \acro{CTAN} archives:

\begin{quote}
  \htmlignore
  \acro{DANTE}, Deutschsprachige Anwendervereinigung\\
  \hspace*{1em}\TeX{} e.V.\\
  \endhtmlignore
\begin{htmlversion}
  DANTE, Deutschsprachige Anwendervereinigung \TeX{} e.V.\\
\end{htmlversion}
  Postfach 10\,18\,40\\
  D-69008 Heidelberg\\
  Germany\\[.25\baselineskip]
  Tel: (+49) 06221 2\,97\,66\\
  Fax: (+49) 06221 16\,79\,06\\
  Email: \Email|dante@dante.de|\\
  Web: \URL|http://www.dante.de/|\\
  \acro{CTAN} details: \CTANref{dante}
\end{quote}

\begin{quote}
  Association \acro{GUT}enberg,\\
  BP 10,\\
  93220 Gagny principal,\\
  France\\[.25\baselineskip]
  Email: \Email|gut@irisa.fr|\\
  Web: \URL|http://www.ens.fr/gut/|\\
  \acro{CTAN} details: \CTANref{gut}
\end{quote}

\htmlignore
  \begin{list}{}{}\item\relax
\endhtmlignore
\begin{htmlversion}
  \begin{quote}
\end{htmlversion}
  \acro{NTG} \\
  Postbus 394, 1740AJ  Schagen,\\
  The Netherlands\\[.25\baselineskip]
  Email: \Email|ntg@nic.surfnet.nl|\\
  Web: \URL|http://ei0.ei.ele.tue.nl/ntg/ntg.html|\\
\htmlignore
  \hphantom{Web: }(note that this is a temporary address)\\
\endhtmlignore
  \acro{CTAN} details: \CTANref{ntg}
\htmlignore
  \end{list}
\endhtmlignore
\begin{htmlversion}
  \end{quote}
\end{htmlversion}

\begin{quote}
  \acro{UK} \TeX{} Users' Group,\\
\htmlignore
  \careof{} Peter Abbott,\\
\endhtmlignore
\begin{htmlversion}
  c/o Peter Abbott,\\
\end{htmlversion}
  1 Eymore Close,\\
  Selly Oak,\\
  Birmingham B29 4LB\\
  \acro{UK}\\[.25\baselineskip]
  Tel: (+44) 0121 476 2159\\
  Email: \Email|UKTuG-Enquiries@tex.ac.uk|\\
  Web: \URL|http://www.tex.ac.uk/UKTUG/home.html|\\
  \acro{CTAN} details: \CTANref{uktug}
\end{quote}

A listing of all known groups is available as \CTANref{usergrps-list}

\Question[Q-TUGTC]{\acro{TUG} Technical Working Groups}

\htmlignore
\acro{TUG} (\Qref{}{Q-TUG})
\endhtmlignore
\begin{htmlversion}
  \Qref{TUG}{Q-TUG}
\end{htmlversion}
has an autonomous Technical Council which oversees a number of working
groups on areas of common interest to the \TeX{} community. The
Council has three members (current chair is Michael Ferguson, assisted
by Yannis Haralambous and Sebastian Rahtz), who liaise with chair
people of each working group. Each group establishes its own working
methods and membership, and anyone interested in taking part should
contact the chair. Suggestions for new groups should be addressed to
Michael Ferguson (\Email|mike@inrs-telecom.uquebec.ca|).

A brief list of the active groups follows:

\begin{description}
\item[WG-92-00 (IRP-TWG)]\emph{Independent Research Project TWG.}\\
To recognise and report to the \TeX{} Board and the \TeX{}
Community on important projects which are independent of \acro{TUG} but are
of concern to the entire \TeX{} Community. 

Contact: Alan Hoenig (\Email|ajhjj@cunyvm.cuny.edu|)

\item[WG-92-01]\emph{\TeX{} Extended Mathematics Font Encoding.}\\
To create font encoding standards for Mathematical fonts
used in \TeX{} systems.

Contact: Barbara Beeton (\Email|bnb@math.ams.org|)

\item[WG-92-03]\emph{Multiple Language Coordination.}\\
The primary purpose of this working group is to obtain, for \TeX{}
systems, a consistent means for implementing, accessing, and describing, the
fonts, ligature rules, hyphenation patterns and other special requirements
for a given linguistic group. 

Contact: Yannis Haralambous (\Emaildot|Yannis.Haralambous@univ-lille1.fr|)

\item[WG-92-04]\emph{\TeX{} for the Disabled.}\\
  The primary purpose of this working group is as a forum for those
  people interested in using and/or enhancing \TeX{} to serve the
  needs of those with visual and other disabilities.

Contact: T.V. Raman (\Email|raman@adobe.com|)


\item[WG-92-05]\emph{\TeX{} Archive Guidelines.}\\
The purpose of this Technical Working Group is to develop
guidelines for the effective management and utilisation of major \TeX{}
archives, and to initiate communication among the maintainers of the
existing archives for the purpose of coordination and synchronisation.

Contact: %George Greenwade  (\Email|bed_gdg@shsu.edu|)
Sebastian Rahtz (\Email|s.rahtz@elsevier.co.uk|)

\item[WG-94-07]\emph{\TeX{} Directory Structures.}\\
The primary purpose of this TWG is to identify a universal directory
structure for macros, fonts and other related \TeX{} software so that
recommendations can be made to all suppliers of \TeX{} software.

The group's current set of proposals are to be found on \acro{CTAN} at \CTANref{tds}

Contact: %Norm Walsh (\Email|norm@ora.com|)
Karl Berry (\Email|kb@cs.umb.edu|)

\item[WG-94-08]\emph{DVI Driver Implementation and Standardisation Issues.}\\
The major objective shall be to study the issues in
the requirements of \acro{DVI} Drivers imposed by changing needs and
technologies, and to make recommendations for implementation and
standardisation of such drivers to enhance the uniformity of their use.
Work will include, but not be limited to, the examination of the use,
syntax, and semantics of \cs|special{..}| commands.

Contact: Michael Sofka (\Email|sofkam@rpi.edu|)

\item[WG-94-09]\emph{\TeX{} and \acro{SGML}.}\\
The major objective is to investigate the
requirements and  difficulties in developing an interface technology
for \TeX{} and \acro{SGML}.

Contact: Ken Dreyhaupt (\Email|kend@springer-ny.com|)


\item[WG-94-10]\emph{\TeX{} and Linguistics.}\\
The main goal is to study and discuss the
requirements for typesetting linguistics in \TeX{} and as a means of
identifying, examining, testing, and comparing macros, fonts, style files
and other aids for typesetting linguistics.

Contact: Christina Thiele (\Email|cthiele@ccs.carleton.ca|)
\end{description}

%%%%%%%%%%%%%%%%%%%%%%%%%%%%%%%%%%%%%%%%%%%%%%%%%%%%%%%%%%%%%%%%%

\section{Documentation and Help}

\Question[Q-books]{Books on \TeX{} and its relations}

While Knuth's book is the definitive reference for \TeX{}, there are
other books covering \TeX{}:
\begin{booklist}
\item[The \TeX{}book]by Donald Knuth (Addison-Wesley, 1984,
  \ISBN{0-201-13447-0}, paperback \ISBN{0-201-13448-9})
\item[A Beginner's Book of \TeX{}]by Raymond Seroul and Silvio Levy,
  (Springer Verlag, 1992, \ISBN{0-387-97562-4})
\item[Introduction to \TeX{}]by Norbert Schwarz (Addison-Wesley,
  1989, \ISBN{0-201-51141-X})
\item[A Plain \TeX{} Primer]by Malcolm Clark (Oxford University
  Press, 1993, ISBNs~0-198-53724-7 (hardback) and~0-198-53784-0
  (paperback))
\item[\TeX{} by Topic]by Victor Eijkhout (Addison-Wesley, 1992,
  \ISBN{0-201-56882-9})
\item[\TeX{} for the Beginner]by Wynter Snow (Addison-Wesley, 1992,
  \ISBN{0-201-54799-6})
\item[\TeX{} for the Impatient]by Paul W.~Abrahams, Karl Berry and
  Kathryn A.~Hargreaves (Addison-Wesley, 1990, \ISBN{0-201-51375-7})
\item[\TeX{} in Practice]by Stephan von Bechtolsheim (Springer
  Verlag, 1993, 4 volumes, \ISBN{3-540-97296-X} for the set, or
% nos in brackets are for German distribution (Springer Verlag, Berlin)
  Vol.~1: 0-387-97595-0, % (3-540-97595-0)
  Vol.~2: 0-387-97596-9, % (3-540-97596-9)
  Vol.~3: 0-387-97597-7, and % (3-540-97597-7)
  Vol.~4: 0-387-97598-5)% (3-540-97598-5)
\htmlignore
\item[\TeX{}: Starting from \sqfbox{1}\thinspace\footnotemark]%
\footnotetext{That's `Starting from Square One'}%
\endhtmlignore
\begin{htmlversion}
\item[\TeX{}: Starting from Square One]
\end{htmlversion}
  by Michael Doob (Springer
  Verlag, 1993, \ISBN{3-540-56441-1})
\item[The Advanced \TeX{}book]by David Salomon (Springer Verlag, 1995,
  \ISBN{0-387-94556-3})
\end{booklist}
For \LaTeX{}, see:
\begin{booklist}
\item[\LaTeX{}, a Document Preparation System]by Leslie Lamport
  (second edition, Addison Wesley, 1994, \ISBN{0-201-15790-X})
\item[A guide to \LaTeXe{}]Helmut Kopka and Patrick W.~Daly (second
  edition, Addison-Wesley, 1995, \ISBN{0-201-42777-X})
\item[The \LaTeX{} Companion]by Michel Goossens, Frank Mittelbach,
  and Alexander Samarin (Addison-Wesley, 1993, \ISBN{0-201-54199-8})
\item[\LaTeX{} Notes:]\emph{Practical Tips for Preparing Technical Documents}
  by J.~Kenneth Shultis (Prentice Hall, 1994, \ISBN{0-131-20973-6})
\item[\LaTeX{} Line by Line]by Antoni Diller (John Wiley \& Sons,
  1993, \ISBN{0-471-93471-2})
\item[\LaTeX{} for Scientists and Engineers]by David J.~Buerger
  (McGraw-Hill, 1990, \ISBN{0-070-08845-4})
\item[Math into \TeX{}:]\emph{A Simplified Introduction using \AMSLaTeX{}}
  by George Gr\"atzer (Birkh\"auser, 1993, \ISBN{0-817-63637-4}, or,
  in Germany, \ISBN{3-764-33637-4})
\item[Math into \LaTeX{}:]\emph{An Introduction to \LaTeX{} and \AMSLaTeX{}}
  by George Gr\"atzer (Birkh\"auser, 1996, \ISBN{0-817-63805-9})
\end{booklist}
Of the list, Lamport's, Goossens, Mittelbach and Samarin's, Kopka and
Daly's, and Gr\"atzer's ``\emph{Math into \LaTeX{}}'' cover \LaTeXe{}.
A sample
of the last, in Adobe Acrobat format, is also available
(\CTANref{mil}).

The list for \MF{} is rather short:
\begin{booklist}
\item[The \MF{}book]by Donald Knuth (Addison Wesley, 1986,
  \ISBN{0-201-13445-4})
\end{booklist}
A book covering a wide range of topics (including installation and
maintenance) is:
\begin{booklist}
\item[Making \TeX{} Work]by Norman Walsh (O'Reilly and Associates,
  Inc, 1994, \ISBN{1-56592-051-1})
\end{booklist}

This list only covers books in English: \acro{UK}~\acro{TUG} cannot hope to maintain
a list of books in languages other than our own.

\Question{Where to find this article}

Bodenheimer's article, from which the present one was developed, is
posted (nominally monthly) to newsgroup
\Newsgroup|comp.text.tex| and cross-posted to newsgroups
\Newsgroup|news.answers| and \Newsgroup|comp.answers|. The most
recently posted copy of Bodenheimer's article is kept on \acro{CTAN} in
\CTANref{TeX-FAQ}; it is also archived at any site that archives
\Newsgroup|news.answers|, such as \FTP|rtfm.mit.edu|
(\FTP|18.181.0.24|), and the article is available there
  \checked{RF}{1994/11/24}%
via anonymous |ftp| (in the directory
\File|pub/usenet/news.answers/tex-faq|). If you have access to email,
but not to |ftp|, use the \Package|ftpmail| server
(see \Qref[]{access to the archives}{Q-archives}).
%\htmlignore
%`\acro{\texttt{SENDME FAQ}}'
%\endhtmlignore
%\begin{htmlversion}
%  `|SENDME FAQ|'
%\end{htmlversion}
%to \Email|fileserv@shsu.edu|
% Another way to retrieve it via email is through the mailserver at
% |rtfm|: send a message containing the lines `|help|' and
% `|index|' to \Email|mail-server@rtfm.mit.edu| for information on
% how to obtain it.

\htmlignore
A version of the present article may be browsed via the World-Wide Web, at URL
\URL|http://www.cogs.susx.ac.uk/cgi-bin/texfaq2html?introduction=yes|%
\begin{footnoteenv}
This is a temporary URL; a final home for the document is to be
provided in due course
\end{footnoteenv}
\endhtmlignore


%Other \Newsgroup|news.answers|/FAQ archives are: \FTP|cnam.cnam.fr|
%(163.173.128.6) in the anonymous ftp directory \path|/pub/FAQ|;
%\FTP|ftp.uu.net| (192.48.96.2) in the anonymous ftp directory
%\path|/pub/usenet| (also available via mail server requests to
%\path|netlib@uunet.uu.net|, or via uunet's 1-900 anonymous UUCP phone
%number); and \FTP|ftp.cs.ruu.nl| (131.211.80.17) in the anonymous ftp
%directory \path|./pub/NEWS.ANSWERS| (also accessible via mail server
%requests to mail-server@cs.ruu.nl). Many of the archives mentioned
%below (\Qref{Q-archives}) also maintain current versions of this document.

\Question[Q-maillists]{Mailing lists about \TeX{} and its friends}

There are (still) people who can use networks but can't read Usenet
news; for them, not all is lost if they can send and receive email.

The \TeX{}hax digest is operated as a
mailing list.  Send a message `\texttt{subscribe texhax}' to
\Email|texhax-request@tex.ac.uk| to join it.

The mailing list |info-tex| offers a mail analogue of the Usenet group
\Newsgroup|comp.text.tex|; mail to the list us automatically submitted
to the newsgroup, and thus answers to questions may be given by people
who only read the newsgroup.  Subscribe to the list by sending a
message `\texttt{subscribe info-tex <your name>}' to \Email|listserv@shsu.edu|

The (rather high volume of) postings to \Newsgroup|comp.text.tex| may
be had in digested form through the mailing list |ctt-digest|.
Subscribe to the list by sending a message
`\texttt{subscribe ctt-Digest <your name>}' to \Email|listserv@shsu.edu|

Announcements of \TeX{}-related installations on the \acro{CTAN}
archives are sent to the mailing list |ctan-ann|.  Subscribe
to the list by sending a message `\texttt{subscribe ctan-ann <your name>}' to
\Email|listserv@urz.Uni-Heidelberg.de|

Issues related to \MF{} (and, increasingly, \MP{}) are discussed on
the |metafont| mailing list; subscribe by sending a message
`\texttt{subscribe metafont <your name>}' to \Email|listserv@ens.fr|

Several other \TeX{}-related lists may be accessed via
\Email|listserv@urz.uni-heidelberg.de|.  Send a message containing
the line `|help|' to this address.\checked{RF}{1994/11/18}
% \Q{This is going some time soon; when?}%

The 
\htmlignore
literate programming newsgroup (\Qref{}{Q-lit})
\endhtmlignore
\begin{htmlversion}
  \Qref{literate programming}{Q-lit} newsgroup
\end{htmlversion}
\Newsgroup|comp.programming.literate| is gatewayed to the |litprog|
mailing list; subscribe by sending a message
`\texttt{subscribe litprog <your name>}' to \Email|listserv@shsu.edu|.

\Question[BibTeXing]{\BibTeX{} Documentation}

\BibTeX{}, a program originally designed to produce bibliographies in
conjunction with \LaTeX{}, is explained in Section 4.3 and Appendix B
of Leslie Lamport's \LaTeX{} manual
(see \Qref[question]{TeX-related books}{Q-books}).
The document ``\BibTeX{}ing'', contained in the file |btxdoc.tex|,
gives a more complete description.  \emph{The \LaTeX{} Companion}
(see \Qref[question]{TeX-related books}{Q-books}) also
has information on \BibTeX{} and writing \BibTeX{} style files.

The document ``Designing \BibTeX{} Styles'', contained in the file
|btxhak.tex|, explains the postfix stack-based language used to write
\BibTeX{} styles (|.bst| files). The file |btxbst.doc| is the template
for the four standard styles (|plain|, |abbrv|, |alpha|, |unsrt|). It
also contains their documentation.
The complete \BibTeX{} documentation set (including the files above)
is in \CTANref{bibtex-doc}

There is a Unix \BibTeX{} \ProgName|man| page in the \ProgName|web2c|
package (see \Qref[question]{TeX systems}{TeX-systems}).  Any copy
you may find of a \ProgName|man| page written in 1985 (before
``\BibTeX{}ing'' and ``Designing \BibTeX{} Styles'' appeared) is
obsolete, and should be thrown away.

\Question{The \PiCTeX{} manual}

\PiCTeX{} is a set of macros for drawing diagrams and pictures. The
macros are freely available in \CTANref{pictex}; however, the
\PiCTeX{} manual itself is not free. It is available for \$30 (\$35
with the disk) from the \TeX{} Users Group (see
\Qref[question]{TUG}{Q-TUG}).
The proceeds from the sales go to Michael Wichura, the author of \PiCTeX{},
and to \acro{TUG}.

\Question[index]{Finding \AllTeX{} macro packages}

Before you ask for a \TeX{} macro or \LaTeX{} class or package file to do
something, try searching Graham Williams'
(\Email|Graham.Williams@dit.csiro.au|) catalogue, available as
\CTANref{catalogue}; this lists many macro packages together with
brief descriptive texts.

%An older (indeed, increasingly out-of-date) resource that is sometimes
%worth
%searching is the \TeX{} macro index written by David M.
%Jones (\Email|dmjones@theory.lcs.mit.edu|), available in
%\CTANref{TeX-index}; however, remember that 
%any location quoted in the index is almost bound to be incorrect.

Having learnt of a file
that seems interesting, search a \acro{CTAN} archive for it (see 
\Qref[question]{finding files on CTAN}{siteindex}). For packages
listed in 
\htmlignore
\emph{The \LaTeX{} Companion} (%
\endhtmlignore
\Qref{The LaTeX Companion}{Q-books}%
\htmlignore
),
\endhtmlignore
the file \CTANref{compan-ctan} may be consulted as an alternative to
searching the archive's index. It lists the current location in the archive of
such files.

An alternative procedure is to use
\URL|http://www.ora.com/homepages/CTAN-Web/|, which permits limited
`keyword' searching for files on the \acro{CTAN} sites.

\Question[siteindex]{Finding files in the \acro{CTAN} archives}

To find software at a \acro{CTAN} site, you can use anonymous |ftp| to
the host with the command `\texttt{quote site index <term>}', or the
searching script at \URL|http://www.dante.de/cgi-bin/ctan-index|

To get the best use out of the |ftp| facility you should remember that
\texttt{<term>} is a \emph{Regular Expression} and not a fixed string,
and also that many files are distributed
in source form with an extension different to the final file. (For
example \LaTeX{} packages are often distributed sources with extension
|dtx| rather than as package files with extension |sty|.)

One should make the regular expresion general enough to find the file
you are looking for, but not too general, as the |ftp| interface will
only return the first 20 lines that match your request.

The following examples illustrate these points.
To search for the \LaTeX{} package `|caption|',
you might use the command:
\begin{verbatim}
  quote site index caption.sty
\end{verbatim}
but it will fail to find the desired package (which is
distributed as |caption.dtx|) and does return unwanted `hits' (such as 
|hangcaption.sty|).  Also, although this example does not show it the
`|.|' in `|caption.sty|' is used as the regular expression that
matches \emph{any} character.
So
\begin{verbatim}
  quote site index doc.sty
\end{verbatim}
matches such unwanted files as
\path|language/swedish/slatex/doc2sty/makefile|

Of course if you \emph{know} the package is stored as |.dtx| you can
search for that name, but in general you may not know the extension
used on the archive.
The solution is to add `|/|' to the front of the package name and
`|\\.| to the end. This will then search for a file name that consists
solely of the package name between the directory separator and the
extension. The two commands:
\begin{verbatim}
  quote site index /caption\\.
  quote site index /doc\\.
\end{verbatim}
do narrow the search down sufficiently. (In the case of doc, a few
extra files are found, but the list returned is sufficiently small to
be easily inspected.)

If the search string is too wide and too many files would match, the
list will be truncated to the first 20 items found. Using some
knowledge of the \acro{CTAN} directory tree you can usually narrow the search
sufficiently. As an example suppose you wanted to find a copy of the
|dvips| driver for \MSDOS{}. You might use the command:
\begin{verbatim}
  quote site index dvips
\end{verbatim}
but the result would be a truncated list, not including the file you
want. (If this list were not truncated 412 items would be returned!)
However we can restrict the search to \MSDOS{} related drivers as
follows.
\begin{verbatim}
  quote site index msdos.*dvips
\end{verbatim}
Which just returns relevant lines such as
\path|systems/msdos/dviware/dvips/dvips5528.zip|

A basic introduction to searching with regular expressions is:
\begin{itemize}
\htmlignore
\itemsep=0.5\itemsep
\endhtmlignore
\item Most charcters match themselves, so |"a"| matches |"a"| etc.;
\item |"."|  matches any character;
\item |"[abcD-F]"| matches any single character from the set 
      \{|"a"|,|"b"|,|"c"|,|"D"|,|"E"|,|"F"|\};
\item |"*"| placed after an expression matches zero or more occurrences
          so |"a*"| matches |"a"| and |"aaaa"|, and |"[a-zA-Z]*"| matches a
      `word';
\item |"\"| `quotes' a special character such as |"."| so |"\."| just
      matches |"."|;
\item |"^"| matches the beginning of a line;
\item |"$"| matches the end of a line.
\end{itemize}

For technical reasons in the quote site index command, you need to
`double' any |\| hence the string |/caption\\.| in the above example.
The quote site command ignores the case of letters.  Searching for
|caption| or |CAPTION| would produce the same result.
%%%%%%%%%%%%%%%%%%%%%%%%%%%%%%%%%%%%%%%%%%%%%%%%%%%%%%%%%%%%%%%%%

\section{Bits and pieces of \TeX{}}

\Question[Q-dvi]{What is a \acro{DVI} file?}

A \acro{DVI} file (that is, a file with the type or extension |.dvi|) is
\TeX{}'s main output file, using \TeX{} in its broadest sense to
include \LaTeX{}, etc.  `\acro{DVI}' is supposed to be an acronym for
\acro{D}e\acro{V}ice-\acro{I}ndependent, meaning that the file can be
printed on almost any
kind of typographic output device.  The \acro{DVI} file is designed to be
read by a driver (\Qref{DVI drivers}{Q-driver}) to produce
further output designed specifically for a particular printer (e.g., a
LaserJet) or to be used as input to a previewer for display on a
computer screen.  \acro{DVI} files use \TeX{}'s internal coding; a \TeX{}
input file should produce the same \acro{DVI} file regardless of which
implementation of \TeX{} is used to produce it.

A \acro{DVI} file contains all the information that is needed for printing 
or previewing except for the actual bitmaps or outlines of fonts, and 
possibly material to be introduced by means of
\htmlignore
\cs|special| commands (\Qref{}{Q-specials}).
\endhtmlignore
\begin{htmlversion}
  \Qref{\cs|special| commands}{Q-specials}.
\end{htmlversion}

The canonical reference for the structure of a \acro{DVI} file is the
source of \ProgName|dvitype| (\CTANref{dvitype}).


\Question[Q-driver]{What is a driver?}

A driver is a program that takes as input a |dvi| file
(\Qref{\acro{DVI} files}{Q-dvi})  and
(usually) produces a file that can be sent to a typographic
output device, called a printer for short.

A driver will usually be specific to a particular printer,
although any PostScript printer ought to be able to print
the output from a PostScript driver.
% (these are also called PostScript conversion programs).

As well as the \acro{DVI} file, the driver needs font information.
Font information may be held as bitmaps or as outlines, or simply as a
set of pointers into the fonts that the printer itself `has'.
Each driver will expect the font information in
a particular form.  For more information on the forms of fonts,
see questions \Qref[]{|pk| files}{Q-pk},
% beware of fill-paragraph here!!!!
\Qref[]{|tfm| files}{Q-tfm},
\Qref[]{virtual fonts}{virtualfonts}
and \Qref[]{Using PostScript fonts with \TeX{}}{Q-usepsfont}.


\Question[Q-pk]{What are \acro{PK} files?}

\acro{PK} files (packed raster) contain font bitmaps. The output
from
\htmlignore
\MF{} (\Qref{}{Q-mf})
\endhtmlignore
\begin{htmlversion}
  \Qref{\MF{}}{Q-mf}
\end{htmlversion}
includes a generic font (\acro{GF}) file and the
utility \ProgName|gftopk| produces the \acro{PK} file from that.
There are a lot of \acro{PK} files, as one is needed for each font,
that is each magnification (size) of each design (point) size for each
weight for each family.  Further, since the \acro{PK} files for one printer
do not necessarily work well for another, the whole set needs to be
duplicated for each printer type at a site.  As a result, they are
often held in an elaborate directory structure, or in `font library
files', to regularise access.


\Question[Q-tfm]{What are \acro{TFM} files?}

\acro{TFM} stands for \TeX{} font metrics, and \acro{TFM} files hold
information about the sizes of the characters of the font in question,
and about ligatures and kerns within that font.  One \acro{TFM} file is
needed for each font used by \TeX{}, that is for each design (point)
size for each weight for each family; one \acro{TFM} file serves for all
magnifications, so that there are (typically) fewer \acro{TFM} files than
there are \acro{PK} files.  The important point is that \acro{TFM} files are
used by \TeX{} (\LaTeX{}, etc.), but are not, generally, needed by the
printer driver.


\Question[virtualfonts]{Virtual fonts}

Virtual fonts for \TeX{} were first implemented by David Fuchs in the
early days of \TeX{}, but for most people they started when Knuth
redefined the format, and wrote some support software, in 1989.
Virtual fonts provide a way of telling \TeX{} about something more
complicated than just a one-to-one character mapping. The entities you
define in a virtual font look like characters to \TeX{} (they appear
with their sizes in a font metric file), but the \acro{DVI} processor may
expand them to something quite different. You can use this facility
just to remap characters, to make a composite font with glyphs drawn
from several sources, or to build up an effect in arbitrarily
complicated ways~--- a virtual font may contain anything which is
legal in a \acro{DVI} file.
In practice, the most common use of virtual fonts is to remap
PostScript fonts (see \Qref[question]{font metrics}{Q-metrics}) or to
build `fake' maths fonts.

It is important to realise that \TeX{} itself does \emph{not} see
virtual fonts; for every virtual font read by the \acro{DVI} driver there
is a corresponding \acro{TFM} file read by \TeX{}. Virtual fonts are normally
created in a single \acro{ASCII} |vpl| (Virtual Property List) file, which
includes both sets of information. The \ProgName|vptovf| program is
then used to the create the binary \acro{TFM} and \acro{VF} files.  The
commonest way (nowadays) of generating |vpl| files is to use the
\ProgName|fontinst| package, which is described in detail
\htmlignore
\Qref[in question]{}{Q-metrics}.
\endhtmlignore
\begin{htmlversion}
together with the discussion of
\Qref{PostScript font metrics}{Q-metrics}.
\end{htmlversion}
\CTANref{qdtexvpl} is another utility for creating ad-hoc virtual
fonts.

\Question[Q-specials]{\cs|special| commands}

\TeX{} provides the means to express things that device drivers can
do, but about which \TeX{} itself knows nothing.  For example, \TeX{}
itself knows nothing about how to include PostScript figures into
documents, or how to set the colour of printed text; but some device
drivers do.

Such things are introduced to your document by means of |\special|
commands; all that \TeX{} does with these commands is to expand their
arguments and then pass the command to the \acro{DVI} file.  In most
cases, there are macro packages provided (often with the driver) that
provide a comprehensible interface to the |\special|; for example,
there's little point including a figure if you leave no gap for it in
your text, and changing colour proves to be a particularly fraught
operation that requires real wizardry. \LaTeXe{}
has standard graphics and colour packages that make file inclusion,
rotation, scaling and colour via |\special|s all easy.

The allowable arguments of |\special| depend on the device driver
you're using.  Apart from the examples above, there are |\special|
commands in the em\TeX{} drivers (e.g., \ProgName|dvihplj|, \ProgName|dviscr|,
\emph{etc}.)~that will draw lines at arbitrary orientations, and
commands in \ProgName|dvitoln03| that permit the page to be set in
landscape orientation.

\Question[Q-dtx]{Documented \LaTeX{} sources (\texttt{.dtx} files)}

\LaTeXe{}, and many support macro packages, are now written in a
\htmlignore
literate programming style (\Qref{}{Q-lit}),
\endhtmlignore
\begin{htmlversion}
  \Qref{literate programming}{Q-lit} style,
\end{htmlversion}
with source and documentation in the
same file.  This format, known as `doc', was originated by Frank
Mittelbach. The documented sources conventionally have the suffix
\texttt{.dtx}, and should normally be stripped of documentation
before use with \LaTeX{}.  Alternatively you can run \LaTeX{} on a
\texttt{.dtx} file to produce a nicely formatted version of the
documented code. An installation script (with suffix
\texttt{.ins}) is usually provided, which needs the standard \LaTeXe{}
\ProgName|docstrip| package (among other things, the installation
process strips all the comments that make up the documentation for
speed when loading the file into a running \LaTeX{} system).  Several
packages can be included in one \texttt{.dtx} file, with conditional
sections, and there facilities for indices of macros etc.  Anyone can
write \texttt{.dtx} files; the format is explained in
\emph{The \LaTeX{} Companion}
(see \Qref[question]{books on \TeX{}}{Q-books}). There are
no programs yet to assist in composition.

\texttt{.dtx} files are not used by \LaTeX{} after they have been
processed to produce \texttt{.sty} or \texttt{.cls} (or whatever)
files.  They need not be kept with the working system; however, for
many packages the \texttt{.dtx} file is the primary source of
documentation, so you may want to keep \texttt{.dtx} files elsewhere.

\Question[dc-fonts]{What are the \acro{DC} fonts?}

A font consists of a number of \emph{glyphs}.  In order that the
glyphs may be printed, there has to be some way of accessing them; in
\TeX{} they're arranged in a numerical order called an
\emph{encoding}, and their number in the encoding is used.  For
various reasons, Knuth chose rather eccentric encodings; in
particular, he chose different encodings for different fonts.

When \TeX{} version 3 arrived, some at least of the reasons for the
eccentricity of
Knuth's encodings went away, and at \acro{TUG}'s Cork meeting, an encoding for
a set of 256 glyphs, for use in \TeX{} text, was defined.  The
intention was that these glyphs should cover `most' European
languages, in the sense of including all accented letters needed.
(Knuth's \acro{CMR} fonts missed things necessary for Icelandic,
Polish and Sami, for example, but the Cork fonts have them.)
\htmlignore
\LaTeXe{} (\Qref{}{latex2e})
\endhtmlignore
\begin{htmlversion}
  \Qref{\LaTeXe{}}{latex2e}
\end{htmlversion}
refers to the Cork encoding as \acro{T}1, and provides the means to
use fonts thus encoded to avoid problems with the interaction of
accents and hyphenation 
% beware!!!  fill-paragraph!!!
(\Qref{hyphenation of accented words}{hyphenated-accents}).
% beware!!!  fill-paragraph!!!

The only \MF{}-fonts that conform to the Cork encoding are the
\acro{DC} fonts (available as \CTANref{dc}; ensure you have version
1.2, patch level 1, released in December 1995, or later).  They look
\acro{CM}-like,
and should be regarded as an interim version of a hypothetical set of
\acro{EC} fonts (which, it is hoped, will be available some time in
1996).  Their serious disadvantage for the casual user is
that they are large~--- each \acro{DC} font is roughly twice the size
of the corresponding \acro{CM} font; what's more until corresponding
fonts for mathematics are produced, the \acro{CM} fonts must be
retained.

The Cork encoding is also implemented by the
\htmlignore
\acro{PSNFSS} system (\Qref{}{Q-usepsfont}),
\endhtmlignore
\begin{htmlversion}
  \Qref{\acro{PSNFSS} system}{Q-usepsfont},
\end{htmlversion}
for PostScript fonts.

%%%%%%%%%%%%%%%%%%%%%%%%%%%%%%%%%%%%%%%%%%%%%%%%%%%%%%%%%%%%%%%%%

\section{Acquiring the Software}

\Question[Q-archives]{Repositories of \TeX{} material}

To aid the archiving and retrieval of of \TeX{}-related files, a
\acro{TUG} working group developed the Comprehensive \TeX{} Archive
Network (\acro{CTAN}).  Each \acro{CTAN} site has identical material,
and maintains authoritative versions of its material.  These
collections are extensive; in particular, almost everything mentioned
in this article is archived at the \acro{CTAN} sites, even if its
location isn't explicitly stated.

The \acro{CTAN} sites are currently \FTP|ftp.dante.de|
(\FTP|129.206.100.192|) and \FTP|ftp.tex.ac.uk| (\FTP|128.232.1.87|).
% , and \FTP|ftp.shsu.edu| (\FTP|192.92.115.10|).
The organisation of \TeX{} files on
% all
these sites is identical and starts at
\path|tex-archive/|.  To reduce network load, please use the \acro{CTAN} site
or mirror closest to you.  A complete and current list of \acro{CTAN} sites
and known mirrors can be obtained by using the \ProgName|finger| utility on
`user' \Email|ctan@ftp.tex.ac.uk| or \Email|ctan@ftp.dante.de|; it is
also available as file \CTANref{CTAN-sites}

To find software at a \acro{CTAN} site, use anonymous |ftp| to the host,
and then execute the command `\texttt{quote site index <term>}' (see
\Qref[question]{finding files}{siteindex} for details).

The email server
\Email|ftpmail@ftp.dante.de| provides an |ftp|-like interface through
mail.  Send a message containing just the line `|help|' to your
nearest server for details of use.

% \item \FTP|ftp.cs.ruu.nl| (131.211.80.17) also contains a substantial
%   \TeX{} archive with |ftp| access. To use it via email, send a message
%   containing the line `|help|' to \Email|mail-server@cs.ruu.nl|. This mail
%   server can send binary files in a variety of different formats.
%\item % this one actually still exists
Users on \acro{BITNET} may access anonymous |ftp| for some files 
indirectly by sending mail to \Email|BITFTP@PUCC.BITNET|.\@  Send a
message containing the line `|help|' to this address for more
information.%
  \checked{RF}{1994/10/28}

There is also the \acro{DECUS} \TeX{} collection of
material for \acro{VMS}, Unix, \MSDOS{}, and the Macintosh.
% The material for \acro{VMS} has not been kept up-to-date, but
% continues to run on Open\acro{VMS} on the \acro{VAX}.
%  NOTE: wuarchive doesn't seem to be working just now...
It is available via anonymous |ftp| from
\FTP|wuarchive.wustl.edu| (128.252.135.4) in
\path|decus/tex/|.  It can also be obtained from the \acro{DECUS} Library
(reference number VS0058) in the US, or through your \acro{DECUS} office
outside of the US. To contact the \acro{DECUS} Library, send mail or telephone:
\begin{quote}
  \acro{DECUS}\\
  LIBRARY ORDER PROCESSING\\
  334 South Street, SHR3-1/T25\\
  Shrewsbury, MA 01545-4195\\
  \acro{USA}\\[0.25\baselineskip]
  Tel: 800-DECUS55 (within the \acro{USA}, for information)\\
  Fax: (+1) 508-841-3373 (for inquiries)
\end{quote}
or send electronic mail for information to the \acro{DECUS} \TeX{} Collection
Editor, Ted Nieland (\Email|nieland@ted.hcst.com|).
\checked{RF}{1994/11/17}

Finally, of course, the \TeX{} user who has no access to any sort of
network may buy a copy of the archive on \CDROM{} (see
\Qref[question]{\TeX{} \CDROM{}s}{Q-CD}).

\Question{Contributing a file to the archives}

Use anonymous |ftp| to any \acro{CTAN} archive
(see \Qref[question]{sources of software}{Q-archives}) and retrieve
the file \CTANref{CTAN-uploads} in the root directory. It contains
instructions for uploading files and notifying the appropriate people
for that site.

If you cannot use |ftp|, mail your contribution to
\Email|ctan@urz.Uni-Heidelberg.de| and it will be passed along.  You will make
everyone's life easier if you choose a descriptive and unique name for
your submission, so it's probably a good idea to check that your style
file's name is not already in use by means of the
%CAREFUL ... wraps here, given a chance!
`\texttt{site index}'
command (\Qref{finding files in the archives}{siteindex}).

\Question{Finding new fonts}

A comprehensive list of \MF{} fonts is posted to
\Newsgroup|comp.fonts| and to \Newsgroup|comp.text.tex|, roughly
every six weeks, by Lee Quin (\Email|lee@sq.sq.com|); it is available
as \CTANref{mf-list}

The list contains details both of commercial fonts and of fonts
available via anonymous |ftp|. Most of the fonts are available via
anonymous |ftp| from the \acro{CTAN} archives
(see \Qref[question]{sources of software}{Q-archives}).
% An article on
% fonts by Dominik Wujastyk, available from \acro{CTAN} in
% \CTANref{wujastyk-txh}, contains information on \MF{} fonts as well.
% (Dominik's article is corrupt on the archive; let's forget it for
% now)

\Question[Q-CD]{\TeX{} \CDROM{}s}

If you don't have access to the Internet, you can get the \acro{CTAN}
collections on a \CDROM{}. Even those who do will
find it very convenient to have large quantities of \TeX{}-related
files to hand.

Prime Time Freeware produced \emph{\TeX{}cetera} 1.1 in July 1994,
which is a snapshot of \acro{CTAN} taken in June 1994. Regular updates are planned.
The material is all compressed in \acro{ZIP} format to fit it all on
one \acro{CD}, and to avoid the limitations of the \acro{ISO}~9660
file system directory. You can buy the \acro{CD} from:
\begin{quote}
  Prime Time Freeware\\
  370 Altair Way, Suite 150\\
  Sunnyvale \acro{CA} 94086  \\
  \acro{USA}\\
  Tel: (+1) 408 433 9662\\
  Fax: (+1) 408 433 0727\\
  Email: \Email|ptf@cfcl.com|
\end{quote}
or from many \CDROM{} resellers, or the
\htmlignore
\acro{TUG} office (\Qref{}{Q-TUG}).
\endhtmlignore
\begin{htmlversion}
  \Qref{\acro{TUG}}{Q-TUG} office.
\end{htmlversion}
Price will be around \$60. Please note that \acro{PTF} is not a big
commercial firm, and is a good friend of the \TeX{} community.

Walnut Creek \acro{CDROM} also provide a two-disc \CDROM{} set, holding
1000Mb of \TeX{}-related information.  Information about the \CDROM{} is
available at \URL|http://www.cdrom.com/titles/tex.html|, which also
has a link to an ordering page.  Walnut Creek's address, etc., are:
\begin{quote}
  Walnut Creek \acro{CDROM}\\
  4041 Pike Lane, Ste \acro{D}-www\\
  Concord, \acro{CA} 94520\\
  \acro{USA}\\
  Tel: (+1) 510 674-0783 or
  \htmlignore
  \\ \hphantom{Tel: }%
  \endhtmlignore
  800 786-9907 (within the \acro{USA} and Canada)\\
  Fax: (+1) 510 674-0821\\
  Email: \Email|info@cdrom.com| (for questions) and
  \htmlignore
  \\ \hphantom{Email: }%
  \endhtmlignore
  \Email|orders@cdrom.com| (for orders)
\end{quote}

If you want a ready-to-run \TeX{} system on \CDROM{}, one is available for \MSDOS{}
only (so far). The Dutch \TeX{} Users Group (\acro{NTG}) publish the whole
4All\TeX{} workbench on a 2-\CDROM{} set packed with all the \MSDOS{} \TeX{}
software, 
macros and fonts you can want. It is available from \acro{NTG} direct (see
\Qref[question]{user groups}{Q-othergroups}), from \acro{TUG} for 
\$40 and from \acro{UK}~\acro{TUG} for \pounds30 (a manual is included). It is a useful
resource for anyone to browse, not just for \MSDOS{} users.

%%%%%%%%%%%%%%%%%%%%%%%%%%%%%%%%%%%%%%%%%%%%%%%%%%%%%%%%%%%%%%%%%

\section{\TeX{} Systems}

\Question[TeX-systems]{\AllTeX{} for different machines}
We list here the free or shareware packages; see
\Qref[question]{vendors}{Q-commercial} for commercial packages.
\begin{description}
\item[Unix] Instructions for retrieving the Unix \TeX{} distribution
  via anonymous |ftp| are available in the document
  \CTANref{unixtexftp}

  A useful set of binaries for various common Unix systems is
  to be found as part of the te\TeX{} distribution
  (\CTANref{tetex-bin}); te\TeX{} will compile on most Unix systems,
  though it was originally developed for use under Linux (see below).

\item[AIX] \TeX{} for the \acro{IBM} RS6000 running AIX is available in
  \CTANref{aix3.2}

\item[386/ix] Executables for 386/ix are available in
  \CTANref{386ix}

\item[Linux] There are at least two fairly complete implementations of
  \TeX{} to run on Linux.  The Slackware distribution includes N\TeX{}
  (available as \CTANref{ntex}), which probably contains more
  \TeX{}-related material than you would ever want.  The more recent
  te\TeX{} (available as \CTANref{tetex}) is based on Karl Berry's
  path-searching mechanisms, and is more compact than N\TeX{} while
  still being pretty comprehensive.

\item[\acro{PC}] The em\TeX{} package for \acro{PC}s running \acro{OS/}2, \MSDOS{} or Windows
  includes \LaTeX{}, \BibTeX{}, previewers, and drivers, and is
  available in \CTANref{emtex} as a series of zip archives.  The
  package was written by Eberhard Mattes, and documentation is
  available in both German and English.  Appropriate memory managers
  for using em\TeX{} with 386 (and better) processors and under
  Windows, are included in the distribution.

  A second package, g\TeX{}, runs under \MSDOS{} or Windows (and its
  users speak well of it).  It is available from \CTANref{gtex}

  \acro{TUG} (and some of the other user groups) offer all freely-available
  \TeX{} software for the \acro{PC}.  A catalogue is available free from
\htmlignore
  \acro{TUG} (\Qref{}{Q-TUG}).
\endhtmlignore
\begin{htmlversion}
  \Qref{\acro{TUG}}{Q-TUG}.
\end{htmlversion}

\item[\acro{PC}: Win32] Mik\TeX{}, by Christian Schenk, first arrived
  on CTAN in 1996.  It has been welcomed by those that have used it
  and reported their experiences.  It will run under Windows'95 or
  Windows/NT, and is available from \CTANref{miktex}

\item[Mac] Oz\TeX{} is a shareware version of \TeX{} for the Macintosh. A
  \acro{DVI} previewer and PostScript driver are also included. It should run
  on any Macintosh Plus, SE, II, or newer model, but will not work on
  a 128K or 512K Mac. It was written by Andrew Trevorrow, and is
  available in \CTANref{oztex}, or on floppy disks from
\htmlignore
  \acro{TUG} (\Qref{with details of \acro{TUG}}{Q-TUG}).
\endhtmlignore
\begin{htmlversion}
  \Qref{\acro{TUG}}{Q-TUG}.
\end{htmlversion}
  \acro{UK}~\acro{TUG} prepays the shareware fee, so that members of
  \acro{UK}~\acro{TUG} may
  acquire the software without further payment.  Questions about
  Oz\TeX{} may be directed to \Email|oztex@midway.uchicago.edu|

  Another partly shareware program is \acro{CM}ac\TeX{} (available as
  \CTANref{cmactex}), put together by Tom Kiffe. This is much closer
  to the Unix \TeX{} setup (it uses \ProgName|dvips|, for instance).

%\item[Mac] Oz\TeX{} (\Qref{Q-oztex}) is a shareware version. Another
%  version is CMacTeX, which has \TeX{} 3.14, \MF{} 2.7, a screen
%  previewer, dvips, a PostScript printing utility for the LaserWriter,
%  and some font managing utilities. It is available from the \acro{CTAN}
%  archives (\Qref{Q-archives}).

\item[\acro{VMS}] \TeX{} for \acro{VMS} is available as \CTANref{AXPVMSTeX} (for
  Alpha-based machines) or \CTANref{VAXVMSTeX} (for \acro{VAX} machines).
  Standard tape distribution is through \acro{DECUS}
  (see \Qref[question]{sources of software}{Q-archives}).
  \checked{RF}{1994/10/11}

\item[Atari] \TeX{} is available for the Atari ST in \CTANref{atari}

  If anonymous |ftp| is not available to you, send a message
  containing the line `|help|' to
  \Email|atari@atari.archive.umich.edu|

\item[Amiga] Full implementations of \TeX{} 3.1 (Pas\TeX{}) and \MF{}
  2.7 are available in \CTANref{amiga}

  You can also order a \CDROM{} containing this and other Amiga software
  from Walnut Creek \acro{CDROM}, telephone (+1) 510-947-5997.
  \checked{RF}{1994/10/11}

\item[\acro{TOPS}-20] \TeX{} was originally written on a \acro{DEC}-10
  under \acro{WAITS},
  and so was easily ported to \acro{TOPS}-20. A distribution that runs on
  \acro{TOPS}-20 is available via anonymous |ftp| from \FTP|ftp.math.utah.edu|
  (128.110.198.34) in \path|pub/tex/pub/web|
  \checked{RF}{1994/10/17}
\end{description}

\Question[Q-editors]{\TeX{}-friendly editors and shells}

There are good \TeX{}-writing environments and editors for most
operating systems; some are described below, but this is only a
personal selection:
\begin{description}
\item[Unix] Try \acro{GNU}~\ProgName|emacs|, and the \acro{AUC}\TeX{} mode
  (\CTANref{auctex}).  This provides menu items and control sequences
  for common constructs, checks syntax, lays out markup nicely, lets
  you call \TeX{} and drivers from within the editor, and everything
  else like this that you can think of.  Complex, but very powerful.
\item[\acro{VMS}] An \ProgName|lsedit| mode for editing \TeX{} source is
  available from
  \htmlignore
  \acro{TUG} (\Qref{}{Q-TUG})
  \endhtmlignore
\begin{htmlversion}
  \Qref{\acro{TUG}}{Q-TUG}
\end{htmlversion}
  as \TeX{}niques 1, \acro{VAX} Language-Sensitive Editor, by Kent MacPherson
  (1985).
\item[\MSDOS{}] There are several choices:
  \begin{itemize}
  \item The (shareware) 4All\TeX{} workbench (\CTANref{4alltex})
    provides a very
    comprehensive environment written in 4\acro{DOS} which lets you
    access most \TeX{}-related
    software in a friendly way. You can choose your own editor;
    something such as \ProgName|QEdit| or \ProgName|Brief| is
    suitable. This whole package is available in easy-to-use form on
    \CDROM{} from \TeX{} user groups.
  \item \TeX{}shell (\CTANref{texshell}) is a simpler,
    easily-customisable environment, which can be used with the editor
    of your choice.
  \item Eddi4\TeX{} (\CTANref{e4t}; also shareware) is a
    specially-written \TeX{}
    editor which features intelligent colouring, bracket matching,
    syntax checking, online help and the ability to call \TeX{}
    programs from within the editor. It is highly customisable, and
    features a powerful macro language.
  \end{itemize}
  You can also use \acro{GNU}~\ProgName|emacs| and \acro{AUC}\TeX{}
  under \MSDOS{}.
\item[Windows] Your best public domain bet is probably to use
  MicroEmacs as an editor and control centre for \TeX{} programs. The
  g\TeX{} package (\CTANref{gtex}) comes with MicroEmacs ready to go,
  integrated with \TeX{}, previewer, \ProgName|dvips| and
  \ProgName|GhostScript|.

  \TeX{}telmExtel (\CTANref{TeXtelmExtel}) is a Shell for em\TeX{} or
  \acro{W}\TeX{} and related tools under Windows.  It includes a
  simple multiple-document editor, a built-in spelling checker,
  automatic \acro{OEM}/\acro{ANSI} character conversion,
  user-definable point-and-click Templates, support for the forward
  and inverse search mechanism of \acro{DVI} driver for Windows and
  for automatic font generation.  Besides the predefined tools, up to
  10 user-defined tools can be set up.

  On a \acro{PC} with large enough memory, a version of
  \acro{GNU}~\ProgName|emacs|,
  that will run under Windows, is available; thus you can also use
  \acro{AUC}\TeX{} under Windows.

  Y\&Y's commercial (and high-quality) Windows previewer, \ProgName|dviwindo|,
  can be used as a good \TeX{} shell, calling programs such as \TeX{},
  drivers, and editors (Y\&Y supply the public domain \acro{PE}, and
  recommend the commercial Epsilon) from customisable menus
  (see \Qref[question]{commercial vendors}{Q-commercial} for details of Y\&Y).

  Scientific Word is a \WYSIWYG{} editing program, strong on maths, which
  uses \LaTeX{} for output (see \Qref[question]{vendors}{Q-commercial}
  for contact address).
\item[\acro{OS/}2] Eddi4\TeX{} works under \acro{OS/}2; look also at
  \CTANref{epmtex} for a specific \acro{OS/}2 shell.
\item[Macintosh] The commercial Textures provides an excellent integrated
  Macintosh environment with its own editor.  More powerful still (as an
  editor) is the shareware \ProgName|Alpha| (\CTANref{alpha}) which is
  extensible enough to let you perform almost any \TeX{}-related job. It
  works well with Oz\TeX{}.
\end{description}
Atari, Amiga and \acro{N}e\acro{XT} users also have nice environments. \LaTeX{}
users who like \ProgName|make| should try \CTANref{latexmk}

There is another set of shell programs to help you manipulate
\BibTeX{} databases.

\Question[Q-commercial]{Commercial \TeX{} implementations}

There are many commercial implementations of \TeX{}. The first
appeared not long after \TeX{} itself appeared. Of the vendors,
ArborText (formerly Textset) and Personal \TeX{} are those who have
survived longest (since the mid or early 80s).

What follows is probably an incomplete list.  Naturally, no warranty or
fitness for purpose is implied by the inclusion of any vendor in this
list.  The source of the information is given to provide some clues to
its currency.

In general, a commercial implementation will come `complete', that is,
with suitable previewers and printer drivers.  They normally also have
extensive documentation (\emph{i.e}., not just the \TeX{}book!) and some
sort of support service.  In some cases this is a toll free number
(probably applicable only within the \acro{USA} and or Canada), but others
also have email, and normal telephone and fax support.
\begin{description}
\item[Unix; \TeX{}] Silicon Graphics Iris/Indigo, Solaris 2.1, \acro{IBM} RS/6000,
  \acro{DEC}/\acro{RISC}-Ultrix, HP 9000. ``Complete \TeX{} packages. Ready to use,
  fully documented and supported.''
  \begin{quote}
    ArborText Inc\\
    1000 Victors Way\\
    Suite 400\\
    Ann Arbor \acro{MI} 48108\\
    \acro{USA}\\[.25\baselineskip]
    Tel: (+1) 313-996-3566\\
    Fax: (+1) 313-996-3573
  \end{quote}
  Source: \TUGboat{} 15(1) (1994)
\item[\acro{VAX}/\acro{VMS}; Convergent \TeX{}] Complete system for \acro{VAX}/\acro{VMS} machines
  (a version for Alphas is in preparation); includes \LaTeX{},
  multinational typesetting support, \MF{} and Web.
  \begin{quote}
    Northlake Software, Inc.\\
    812 SW Washington, Ste 1100\\
    Portland, \acro{OR}  97201\\
    \acro{USA}\\[.25\baselineskip]
    Tel: (+1) 503-228-3383\\
    Fax: (+1) 503-228-5662\\
    Email: \Email|rau@nls.com|
  \end{quote}
  Source: Email from Pat Rau, November 1994
\item[\acro{PC}; True\TeX{}] Runs on Windows~3.1, Window~NT and Windows~95.
  \begin{quote}
    The Kinch Computer Co.\\
    6994 Pebble Beach Court\\
    Lake Worth \acro{FL} 33467\\
    \acro{USA}\\[.25\baselineskip]
    Tel: (+1) 561-966-8400
    Fax: (+1) 561-966-0692
    Email: \Email|kinch@holonet.net|\\
    Web: \URL|http://www.emi.net/~kinch|
  \end{quote}
  Source: Email from Richard Kinch, December 1995
\item[\acro{PC}; \TeX{}] ``Bitmap free \TeX{} for Windows.''
  \begin{quote}
    Y\&Y, Inc.\\
    45 Walden Street\\
    Concord \acro{MA} 01742\\
    \acro{USA}\\[.25\baselineskip]
    Tel: 800-742-4059 (within the \acro{USA})\\
    Tel: (+1) 508-371-3286\\
    Fax: (+1) 508-371-2004\\
    Email: \Email|sales-help@YandY.com| and
    \htmlignore
    \\ \hphantom{Email: }%
    \endhtmlignore
    \Email|tech-help@YandY.com|\\
    Web: \URL|http://www.YandY.com/|
  \end{quote}
  Source: Y\&Y announcement, February 1995
\item[pc\TeX{}] Long-established: now has a Windows implementation.
  \begin{quote}
    Personal \TeX{} Inc\\
    12 Madrona Street\\
    Mill Valley, \acro{CA} 94941\\
    \acro{USA}\\[.25\baselineskip]
    Tel: 800-808-7906 (within the \acro{USA})\\
    Fax: (+1) 415-388-8865\\
    Email: \Email|pti@crl.com|\\
    Web: \URL|http://www.crl.com/~pti/|
  \end{quote}
  Source: \TUGboat{} 16(2) (1995)
\item[\acro{PC}; V\TeX{}] Also ``Bitmap-free''.
  \begin{quote}
    MicroPress Inc\\
    68-30 Harrow Street\\
    Forest Hills, \acro{NY} 11375\\
    \acro{USA}\\[.25\baselineskip]
    Tel: (+1) 718-575-1816\\
    Fax: (+1) 718-575-8038\\
    Email: \Email|support@micropress-inc.com|
    Web: \URL|http://www.micropress-inc.com/|
  \end{quote}
  Source: MicroPress home page, April 1996
\item[\acro{PC}; micro\TeX{}] Micro\TeX{} and \TeX{}~tools.
  \begin{quote}
    Micro Programs, Inc.\\
    251 Jackson Ave.\\
    Syosset, \acro{NY} 11791\\
    \acro{USA}\\[.25\baselineskip]
    Tel: (+1) 516-921-1351\\
    Email: \Email|sales@microprograms.com|
  \end{quote}
  Source: \acro{AMS} listing, November 1994
\item[\acro{PC}; Scientific Word] Scientific Word and Scientific Workplace
  offer a mechanism for near-\WYSIWYG{} input of \LaTeX{} documents; they
  ship with True\TeX{} from Kinch (see above).  Queries within the \acro{UK}
  should be addressed to Scientific Word Ltd., others should be
  addressed directly to the publisher, \acro{TCI}.
  \begin{quote}
    Dr Christopher Mabb\\
    Scientific Word Ltd.\\
    98 Pont Adam\\
    Ruabon\\
    Wrexham\\
    Clwyd, \acro{LL}14 6\acro{EF}\\
    \acro{UK}\\[0.25\baselineskip]
    Tel: 0345 660340 (within the \acro{UK}) \\
    Tel: +44 1978 824684 \\
    Fax: 01978 823066 (within the \acro{UK}) \\
    Email: \Email|christopher@sciword.demon.co.uk|
  \end{quote}
  \begin{quote}
    \acro{TCI} Software Research Inc.\\
    1190 Foster Road\\
    Las Cruces \acro{NM} 88001--3739\\
    \acro{USA}\\[0.25\baselineskip]
    Tel: (+1) 505-522-4600\\
    Fax: (+1) 505-522-0116\\
    Email: \Email|info@tcisoft.com|\\
    Web: \URL|http://www.tcisoft.com/tcisoft.html|
  \end{quote}
  Source: Mail from Christopher Mabb, November 1995
\item[Macintosh; Textures] ``A \TeX{} system `for the rest of
  us'\thinspace''; also gives away a \MF{} implementation and some
  font manipulation tools.
  \begin{quote}
    Blue Sky Research\\
    534 SW Third Avenue\\
    Portland, \acro{OR} 97204\\
    \acro{USA}\\[.25\baselineskip]
    Tel: 800-622-8398 (within the \acro{USA})\\
    Tel: (+1) 503-222-9571\\
    Fax: (+1) 503-222-1643\\
    Email: \Email|sales@bluesky.com|\\
    Web: \URL|http://www.bluesky.com/|
  \end{quote}
  Source: \TUGboat{} 15(1) (1994)
\item[Amiga\TeX{}] A full implementation for the Commodore Amiga,
  including full, on-screen and printing support for all PostScript
  graphics and fonts, IFF raster graphics, automatic font generation,
  and all of the standard macros and utilities.
  \begin{quote}
    Radical Eye Software\\
    \acro{PO} Box 2081\\
    Stanford, \acro{CA} 94309\\
    \acro{USA}
  \end{quote}
  Source: Mail from Tom Rokicki, November 1994
\end{description}
\checked{mc}{1994/11/09}%
\checked{RF}{1994/11/24}%

%%%%%%%%%%%%%%%%%%%%%%%%%%%%%%%%%%%%%%%%%%%%%%%%%%%%%%%%%%%%%%%%%

\section{\acro{DVI} Drivers and Previewers}

\Question[Q-dvips]{\acro{DVI} to PostScript conversion programs}

The best public domain \acro{DVI} to PostScript conversion program which
runs under many operating systems is Tom Rokicki's \ProgName|dvips|.
\ProgName|dvips| is written in C and ports easily to other operating
systems; it is available as \CTANref{dvips}

\acro{VMS} versions are available through the \acro{DECUS} library
(see \Qref[question]{sources of software}{Q-archives}), 
and also from \acro{CTAN}: \CTANref{AXPVMSdvips} (for Alpha-based machines),
\CTANref{VAXVMSdvips} (for \acro{VAX} machines); support files are available
in \CTANref{VMSdvips-support}, and a set of fonts for use with
\LaTeXe{} are available in \CTANref{VMSdvips-fonts}

A precompiled version for \MSDOS{} is available from \CTANref{dvips-pc}

Karl Berry's version of \ProgName|dvips| (called \ProgName|dvipsk|)
has a configure script and path searching code similar to that in his
other programs (\emph{e.g.}, \ProgName|web2c|); it is available from
\CTANref{dvipsk}

Another good portable program is \ProgName|dvitops| by James Clark,
which is also written in C and will compile under Unix, \MSDOS{}, \acro{VMS},
and Primos; however, it does not support virtual fonts.  It is
available from
\CTANref{dvitops}

Macintosh users can use either the excellent drivers built into Oz\TeX{}
or Textures, or a port of \ProgName|dvips| in the \acro{CM}ac\TeX{} package.

\Question{\acro{DVI} drivers for \acro{HP} LaserJet}

The em\TeX{} package (see \Qref[question]{\TeX{} systems}{TeX-systems})
contains a driver for the LaserJet, \ProgName|dvihplj|.

Version 2.10 of the Beebe drivers supports the LaserJet. These drivers
will compile under Unix, \acro{VMS}, and on the Atari ST and
\acro{DEC}-20's, and are available from \CTANref{beebe}

Karl Berry's \ProgName|dviljk|, which has the same path-searching code
as his \ProgName|dvipsk| (see
\Qref[question]{DVI to PostScript programs}{Q-dvips}), is available in
\CTANref{dviljk}

\Question{\acro{DVI} previewers}

Em\TeX{} and g\TeX{} for the \acro{PC}, and Oz\TeX{} for the Macintosh, all
come with previewers that can be used on those platforms. There is a
good \acro{OS/}2 Presentation Manager previewer in em\TeX{}, and a public
domain Windows previewer (\CTANref{dviwin}).  Commercial \acro{PC} \TeX{}
packages  (see \Qref[question]{commercial vendors}{Q-commercial})
have good \MSDOS{} and Windows previewers.

Previewers available for other operating systems include:
\begin{proglist}
\item[xdvi] The most widely used previewer for the X Window System
  (and hence almost any Unix or modern \acro{VMS} workstation); available in
  \CTANref{xdvi}

  Karl Berry's version of \ProgName|xdvi|, called \ProgName|xdvik|,
  has features analogous to his \ProgName|dvipsk|
  (see \Qref[question]{DVI to PostScript programs}{Q-dvips});
  it is available in \CTANref{xdvik}
\item[dvipage] For SunView on (old enough) Sun workstations. This was
  published in volume 15 of \Newsgroup|comp.sources.unix| and is
  archived in \CTANref{dvipage}
\item[xtex] An older previewer for the X Window System; available in
  \CTANref{seetex}
\item[dviapollo] For Apollo Domain workstations; available in
  \CTANref{dviapollo}
\item[dvidis] For (old enough, \acro{VMS}) \acro{VAXstations} running
  VWS; available in \CTANref{dvidis}
\item[dvitovdu] for Tektronix 4010-compatible and other terminals
  under Unix and \acro{VMS}; available as \CTANref{dvitovdu}
\item[dvi2tty] A \acro{DVI} to \acro{ASCII} conversion program, for normal
  terminals; available as \CTANref{dvi2tty}
\item[texsgi] For SGI under Irix; both a binary and source are
  available, but be sure to get the fonts as well.  Available as
  \CTANref{texsgi}
\end{proglist}

%%%%%%%%%%%%%%%%%%%%%%%%%%%%%%%%%%%%%%%%%%%%%%%%%%%%%%%%%%%%%%%%%

\section{Support Packages for \TeX{}}

\Question{Fig, a \TeX{}-friendly drawing package}

\ProgName|(X)Fig| is a menu driven tool that allows you to
draw objects on the screen of an \acro{X} workstation.  \ProgName|transfig|
is a set of tools which translate the code \ProgName|fig| produces to
other graphics languages including PostScript and the \LaTeX{} picture
environment. They are available in \CTANref{xfig} and
\CTANref{transfig}

\ProgName|Fig| is supported by Micah Beck (\Email|beck@cs.cornell.edu|) and
\ProgName|transfig| is maintained by Brian Smith (\Email|bvsmith@lbl.gov|). 
Another tool for \ProgName|fig| conversion is \ProgName|fig2mf| which
generates \MF{} code from \ProgName|fig| input. It is available in
\CTANref{fig2mf}

\Question{\TeX{}\acro{CAD}, a drawing package for \LaTeX{}}

\TeX{}\acro{CAD} is a program for the \acro{PC} which enables the user to draw diagrams
on screen using a mouse or arrow keys, with an on-screen menu of available 
picture-elements. Its output is code for the \LaTeX{} picture environment.
Optionally, it can be set to include lines at all angles using 
the em\TeX{} driver-family
\htmlignore
|\special|s (\Qref{}{Q-specials}).
\endhtmlignore
\begin{htmlversion}
\Qref{\cs|special|s}{Q-specials}.
\end{htmlversion}
\TeX{}\acro{CAD} is part of the em\TeX{} distribution.

A Linux port of the program, \CTANref{xtexcad}, is reported also to
run on other Unix operating systems.

\Question{Spelling checkers for work with \TeX{}}

For Unix, \ProgName|ispell| is probably the program of choice. It is
available in \CTANref{ispell}; beware of any version with a number
|4.x| --- such versions represent a divergent version of the source
which lacks many useful facilities of the |3.x| series.

For \MSDOS{}, there are several programs. \ProgName|amspell| can be called from within an editor (available as
\CTANref{amspell}).  \ProgName|jspell| is an extended version of
\ProgName|ispell| (available as \CTANref{jspell}).

For the Macintosh, \ProgName|Excalibur| is the program of choice.  It
will run in native mode on both sorts of Macintosh, and is available as
\CTANref{Excalibur-sea} (there are other dictionaries in the same
directory).

For \acro{VMS}, a spell checker can be found in \CTANref{vmspell}

\Question[Q-vortex]{The \VorTeX{} package}

\VorTeX{} (available in \CTANref{vortex}) is a package of programs
written at the University of
California at Berkeley, and was described by Michael A.~Harrison in
``\emph{News from the \VorTeX{} project}'' in \TUGboat{} 10(1),
pp.~11--14, 1989.  It includes several nice previewers and some
\ProgName|emacs| modes for \TeX{} and \BibTeX{}.  The \VorTeX{}
distribution is not maintained, and now looks distinctly long in the
tooth (it was never upgraded to \TeX{} version 3).

\VorTeX{} needed a separate workstation to run \TeX{} in the
background; modern \acro{PC}s for the home can provide more processor power
(than was available to \VorTeX{}) in a single box.  This fact has been
recognised by Blue Sky Research in their `Lightning Textures' (which
runs on a Macintosh in a somewhat similar way) and by TCI Software
Research in `Scientific Word'
\htmlignore
(\Qref{}{Q-commercial}),
\endhtmlignore
\begin{htmlversion}
  (see \Qref{commercial vendors}{Q-commercial}),
\end{htmlversion}
and is also the basis of many of the other environments mentioned in
\Qref[question]{`editors and shells'}{Q-editors}.

%%%%%%%%%%%%%%%%%%%%%%%%%%%%%%%%%%%%%%%%%%%%%%%%%%%%%%%%%%%%%%%%%

\section{Literate programming}

\Question[Q-lit]{What is Literate Programming?}

Literate programming is the combination of documentation and source
together in a fashion suited for reading by human beings. 
In general, literate programs combine source
and documentation in a single file.  Literate programming tools then
parse the file to produce either readable documentation or compilable
source.  The \acro{WEB} style of literate programming was created by D.~E.~Knuth
during the development of his \TeX{} typesetting software.

Discussion of literate programming is conducted in the newsgroup
\Newsgroup|comp.programming.literate|, which is gatewayed to the
mailing list \Email|litprog@shsu.edu| (see
\Qref[question]{subscribing to mailing lists}{Q-maillists} for
details).  The literate programming FAQ is stored as
\CTANref{LitProg-FAQ}

\Question{\acro{WEB} for C, FORTRAN, and other languages}

\TeX{} is written in the programming language \acro{WEB}; \acro{WEB} is a tool to
implement the concept of ``literate programming''.  

\ProgName|CWEB|, a \acro{WEB} for \acro{C} programs, written by Silvio
Levy, is available as \CTANref{cweb}

Spidery \acro{WEB} supports many 
languages including Ada, awk, and \acro{C}. It was written by Norman Ramsey 
and, while not in the public domain, is usable free. It is available 
in \CTANref{spiderweb}

\ProgName|FWEB| is a version for Fortran, Ratfor, and \acro{C} written by John
Krommes.  % (\Email|krommes@lyman.pppl.gov|).
It is available in \CTANref{fweb}

\ProgName|SchemeWEB| is a Unix filter that translates Scheme\acro{WEB} into \LaTeX{}
source or Scheme source. It was written by John Ramsdell and is
available in \CTANref{schemeweb}

\ProgName|APLWEB| is a version of \acro{WEB} for \acro{APL} and is available in
\CTANref{aplweb}

\ProgName|FunnelWeb| is a version of \acro{WEB} that is language independent. It is
available in \CTANref{funnelweb}
% It also appeared in \Newsgroup|comp.sources.unix| volume 26 issue
% 121, posted 11 April 1993.

Other language independent versions of \acro{WEB} are \ProgName|nuweb| (which
is written in \acro{ANSI} \acro{C}), available in \CTANref{nuweb}, and
\ProgName|noweb|, available in \CTANref{noweb}

A \acro{WEB} for plain \TeX{} macro files, using \ProgName|noweb|, has
recently been made available in \CTANref{tweb}

%%%%%%%%%%%%%%%%%%%%%%%%%%%%%%%%%%%%%%%%%%%%%%%%%%%%%%%%%%%%%%%%%

\section{Format conversions}

\Question{Conversion between \AllTeX{} and others}

\begin{description}
\item[troff] \ProgName|troff-to-latex| (available as
  \CTANref{troff-to-latex}), written by Kamal Al-Yahya at Stanford
  University (California, \acro{USA}), assists in the translation of a
  \ProgName|troff| document into \LaTeX{} format.  It recognises most
  |-ms| and |-man| macros, plus most \ProgName|eqn| and some
  \ProgName|tbl| preprocessor commands.  Anything fancier needs to be
  done by hand. Two style files are provided. There is also a man page
  (which converts very well to \LaTeX{}\dots{}).  The program is
  copyrighted but free. An enhanced version of this program,
  \ProgName|tr2latex|, is available in \CTANref{tr2latex}

  The \acro{DECUS} \TeX{} distribution (see
  \Qref[question]{sources of software}{Q-archives})
  also contains a program which converts \ProgName|troff| to \TeX{}.

%\item[Scribe] Mark James (\Email|jamesm@dialogic.com|) has a copy of
%  \ProgName|scribe2latex| he has been unable to test but which he will
%  let anyone interested have. The program was written by Van Jacobson
%  of Lawrence Berkeley Laboratory.%
%  \checked{RF}{1994/11/18}

\item[WordPerfect] \ProgName|wp2latex| (available as
  \CTANref{wp2latex}) is a \acro{PC} program
  written in Turbo Pascal by R.~C.~Houtepen at the Eindhoven
  University in the Netherlands. It converts \ProgName|WordPerfect|
  5.0 documents to \LaTeX{}. Pascal source is included.  Users find it
  ``helpful'' and ``decent'' in spite of some limitations. It gets
  high marks for handling font changes, but cannot make
  indices, tables of contents, margins or graphics, and can't
  handle features new in \ProgName|WordPerfect| version~5.1, in particular
  the equation formatter. The program is copyrighted but free.

  Glenn Geers of the University of Sydney, Australia
  (\Email|glenn@qed.physics.su.oz.au|) is translating
  \ProgName|wp2latex| into C and adding some \ProgName|WordPerfect|
  5.1 features, in particular its equation handling. His work is in
  the \File|glenn| subdirectory of \CTANref{wp2latex}

\item[\acro{PC}-Write] |pcwritex.arc|, from \CTANref{pcwritex}, is a
  print driver for \acro{PC}-Write that ``prints'' a \acro{PC}-Write
  V2.71 document to a \TeX{}-compatible disk file.  It was written by Peter
  Flynn at University College, Cork, Republic of Ireland.

\item[runoff] Peter Vanroose's (\Email|vanroose@esat.kuleuven.ac.be|)
  conversion program is written in \acro{VMS} Pascal.
  The sources and a \acro{VAX} executable are available from
  \CTANref{rnototex}

\item[refer/tib] There are a few programs for converting bibliographic
  data between \BibTeX{} and \ProgName|refer|/\ProgName|tib| formats.
  They are in \CTANref{refer-tools}

  In spite of the directory name, it also contains a shell script to
  convert \BibTeX{} to \ProgName|refer| as well. The collection
  is not maintained.

\item[\acro{RTF}] A program for converting Microsoft's Rich Text Format to
  \TeX{} is available in \CTANref{rtf2tex}, which was written and is
  maintained by Robert Lupton (\Email|rhl@astro.princeton.edu|).
  There is also a convertor to \LaTeX{} by Erwin Wechtl, in
  \CTANref{rtf2latex}

  Translation \emph{to} \acro{RTF} may be done (for a somewhat
  constrained set of \LaTeX{} documents) by \TeX{}2\acro{RTF}, which
  can produce ordinary \acro{RTF}, Windows Help \acro{RTF} (as well as
  \acro{HTML}, \Qref{conversion to HTML}{Q-LaTeX2HTML}).
  \TeX{}2\acro{RTF} is supported on various Unix platforms and under
  Windows~3.1; it is available from \CTANref{tex2rtf}

\item[Microsoft Word] A rudimentary program for converting MS-Word to
  \LaTeX{} is wd2latex, for \MSDOS{} (\CTANref{wd2latex}); a better
  idea, however, is to convert the document to RTF format and use one
  of the RTF converters mentioned above.
\end{description}

An \acro{FAQ} that deals specifically with conversions between
\TeX{}-based formats and word processor formats is regularly posted to
\Newsgroup|comp.text.tex|, is available via
\URL|http://www.kfa-juelich.de/isr/1/texconv.html| and is archived as
\CTANref{texcnven}

A group at Ohio State University (\acro{USA}) is working on
a common document format based on \acro{SGML}, with the ambition that any
format could be
translated to or from this one.  \ProgName|FrameMaker| provides
``import filters'' to aid translation from alien formats
(presumably including \TeX{}) to \ProgName|Framemaker|'s own.

\Question{Conversion from \AllTeX{} to plain \acro{ASCII}}

The aim here is to emulate the Unix \ProgName|nroff|, which formats
text as best it can for the screen, from the same
input as the Unix typesetting program \ProgName|troff|.

Ralph Droms (\Email|droms@bucknell.edu|) has a style file and a
program that provide the \LaTeX{} equivalent of \ProgName|nroff|,
though it doesn't do a good job with tables and mathematics. The
software is available in \CTANref{txtdist}; the original
\ProgName|dvi2tty| often does an acceptable job and is available in
\CTANref{dvi2tty}

Another possibility is to use \File|screen.sty| (available as
\CTANref{screensty}). Use a \ProgName|dvi2tty| program of some kind;
you might try \CTANref{crudetype} as well.  Another possibility is to
use the \LaTeX{}-to-\acro{ASCII} conversion program, \ProgName|l2a|
(\CTANref{l2a}), although this is really more of a de-\TeX{}ing
program.

The canonical de-\TeX{}ing program is \ProgName|detex|
(\CTANref{detex}), which removes all comments and control sequences
from its input before writing it to its output.  Its original purpose
was to prepare input for a dumb spelling checker.

\Question[Q-SGML2TeX]{Conversion from \acro{SGML} or \acro{HTML} to \protect\TeX{}}

\acro{SGML} is a very important system for document storage and interchange,
but it has no formatting features; its companion \acro{ISO} standard
\acro{DSSSL}
(\URL|http://www.jclark.com/dsssl/|) is designed for writing
transformations and formatting,
but this has not yet been widely implemented. Some \acro{SGML} authoring
systems (e.g., SoftQuad \ProgName|Author/Editor|) have formatting
abilities, and
there are high-end specialist \acro{SGML} typesetting systems (e.g., Miles33's
\ProgName|Genera|).  However, the majority of \acro{SGML} users probably transform
the source to an existing typesetting system when they want to print. 
\TeX{} is a good candidate for this. There are three approaches to writing a
translator:
\begin{enumerate}
\item Write a free-standing translator in the traditional way, with
  tools like \ProgName|yacc| and \ProgName|lex|; this is hard, in
  practice, because of the complexity of \acro{SGML}.
\item Use a specialist language designed for \acro{SGML} transformations; the
  best known are probably \ProgName|Omnimark| and \ProgName|Balise|.
  They are expensive, but powerful, incorporating \acro{SGML} query and
  transformation abilities as well as simple translation.
\item Build a translator on top of an existing \acro{SGML} parser.  By far
  the best-known (and free!) parser is James Clark's
  \ProgName|nsgmls|, and this produces a much simpler output format,
  called \acro{ESIS}, which can be parsed quite straightforwardly (one also
  has the benefit of an \acro{SGML} parse against the \acro{DTD}). Two
  good public domain packages use this method:
  \begin{itemize}
    \item David Megginson's \ProgName|sgmlspm|, written in Perl 5.
      Available from
      \URL|http://www.uottawa.ca/~dmeggins/SGMLSpm/sgmlspm.html|
    \item Joachim Schrod and Christine Detig's \ProgName|stil|, written in
      Common Lisp. Available from
      \URL|ftp://ftp.th-darmstadt.de/pub/text/sgml/stil|
  \end{itemize}
  Both of these allow the user to write `handlers' for every \acro{SGML}
  element, with plenty of access to attributes, entities, and
  information about the context within the document tree.

  If these packages don't meet your needs for an average \acro{SGML}
  typesetting job, you need the big commercial stuff.
\end{enumerate}

Since \acro{HTML} is simply an example of \acro{SGML}, we do not need a specific
system for \acro{HTML}.  However, Nathan Torkington
(\Email|Nathan.Torkington@vuw.ac.nz|) developed
\ProgName|html2latex| from the \acro{HTML} parser in \acro{NCSA}'s
Xmosaic package.
The program takes an \acro{HTML} file and generates a \LaTeX{} file from it.
The conversion code is subject to \acro{NCSA} restrictions, but the whole
source is available as \CTANref{html2latex}

Jonathan Fine (\Email|J.Fine@pmms.cam.ac.uk|) hopes to release, during
1996, his macro package that directly interprets and typesets an
\acro{SGML} source file.

Michel Goossens and Janne Saarela published a very useful summary of
\acro{SGML}, and of public domain tools for writing and manipulating it, in
\TUGboat{} 16(2).

\Question[Q-LaTeX2HTML]{\AllTeX{} conversion to \acro{HTML}}

\TeX{} is a typesetting language, not a markup system.
With properly-used \LaTeX{}, you may  be luckier, 
but don't expect a free lunch. Remember that a) if you want a really
good Web document, you had better redesign it from scratch, and b) \acro{HTML} 
(even \acro{HTML3}) has pretty poor `typesetting' facilities, and anything
beyond the trivial will probably need to end up a graphic.

\LaTeX{}2\acro{HTML} (\CTANref{latex2html}) is a package by Nikos Drakos
(mostly of \ProgName|perl| scripts) that breaks up a \LaTeX{} document
into one or more components, and links them together so that they can
be read over the World-Wide Web as an hypertext document.
It defines a mapping between \LaTeX{} intra-document
references and hyperlinks, and extends the
mechanisms to permit reference to other (possibly remote) documents
and other Internet resources.  It translates \LaTeX{} accented and
other characters (as best it can) to things that World-Wide Web
browsers can display, and translates mathematics
(and other things that browsers can't deal with) to
images that can be loaded in-line into the hypertext document.

\LaTeX{}2\acro{HTML} needs \ProgName|Perl|, the \acro{PBM} utilities,
\ProgName|dvips|, \ProgName|GhostScript|, and other sundries; it
assumes it is running on a Unix system.
Michel Goossens and Janne Saarela published a detailed discussion of
\LaTeX{}2\acro{HTML}, and how to tailor it, in \TUGboat{} 16(2).

There are two alternative strategies:
\begin{enumerate}
\item Free-standing \LaTeX{} to \acro{HTML} translations. Hard, but
  not impossible.  Julian Smart's \ProgName|latex2rtf| (available from
  \CTANref{latex2rtf}) does a plausible job on a subset of \LaTeX{};
\item Writing an \acro{HTML}-output backend in \LaTeX{} itself.  See
  Sebastian Rahtz' paper in \TUGboat{} 16(3) for a discussion of how
  to go about this for the general case of \acro{SGML}.
\end{enumerate}

\Question[Q-hyper]{Making hypertext documents from \TeX{}}

If you want on-line hypertext with a \AllTeX{} source, probably on the
World Wide Web, consider four technologies (which overlap):
\begin{enumerate}
\item Try direct \LaTeX{} conversion to \acro{HTML}; see
  \Qref[question]{\AllTeX{} conversion to \acro{HTML}}{Q-LaTeX2HTML};
\item Rewrite your document using Texinfo
  (see \Qref[question]{Texinfo macro package}{Q-texinfo}), and
  convert that to \acro{HTML};
\item Look at Adobe Acrobat, an electronic delivery system guaranteed
  to preserve your typesetting perfectly.
  See \Qref[question]{Making Acrobat documents from \LaTeX{}}{Q-acrobat};
\item Invest in the hyper\TeX{} conventions (standardised \cs|special|
  commands); there are supporting macro packages for plain \TeX{} and
  \LaTeX{}).
\end{enumerate}

The Hyper\TeX{} project aims to extend the functionality of all the
\LaTeX{} cross-referencing commands (including the table of contents)
to produce \cs|special| commands which are parsed by \acro{DVI} processors
conforming to the Hyper\TeX{} guidelines;
it provides general hypertext links, including those
to external documents.

The Hyper\TeX{} specification says that conformant viewers/translators
must recognize the following set of \cs|special| commands:
\begin{description}
\item[href:] |html:<a href = "href_string">|
\item[name:] |html:<a name = "name_string">|
\item[end:] |html:</a>|
\item[image:] |html:<img src = "href_string">|
\item[base\_name:] |html:<base href = "href_string">|
\end{description}

The \emph{href}, \emph{name} and \emph{end} commands are used to do
the basic hypertext operations of establishing links between sections
of documents. 

Further details are available on \URL|http://xxx.lanl.gov/hypertex/|; there
are two commonly-used implementations of the specification, a
modified  \ProgName|xdvi| and a modified \ProgName|dvips|. Output from the
latter may be used in a modified \ProgName|GhostScript| or Acrobat Distiller.

\Question[Q-acrobat]{Making Acrobat documents from \LaTeX{}}

In the simplest case, use your \acro{DVI} to PostScript driver, and run the
result through Adobe's Acrobat \ProgName|Distiller|; even simpler, if
you use a Mac or Windows \TeX{} system, is to install Acrobat Exchange,
and use \acro{PDF}writer like a printer from your application. The latter is
a dead end, though fine for simple documents, since you can't insert
extra hyperlinks in the \acro{PDF} output. For that, you need the
Distiller route, which supports a special PostScript operator called
\texttt{pdfmark}, for passing through information to the \acro{PDF}. 

To translate all the \LaTeX{} cross-referencing into Acrobat
links, you need a \LaTeX{} package to suitably redefine
the internal commands.  There are two of these for \LaTeXe{}, both
based on the Hyper\TeX{} specification 
(see \Qref[question]{Making hypertext documents from \TeX{}}{Q-hyper}):
Sebastian Rahtz's \Package|hyperref| (available from
\CTANref{hyperref}), and Michael Mehlich's
\Package|hyper| (available from \CTANref{hyper}). You use
\ProgName|dvihps| (a modified \ProgName|dvips|)
to translate the \acro{DVI} into PostScript acceptable to
Distiller.  Alternatively, if you know you only want Acrobat,
\Package|hyperref| also has a `native \acro{PDF}' mode, which works with
plain \ProgName|dvips| (or most other translators) and gives access to
all the functionality of \texttt{pdfmark}.

Sadly, there are no free implementations of Distiller, nor any signs
of them.  \ProgName|GhostScript| (versions 3.51 onwards) can display
and print \acro{PDF} files, however, if you are on a platform with no Acrobat
Reader. You may see a \acro{DVI} to \acro{PDF} translator soon, but do not hold
your breath.

%%%%%%%%%%%%%%%%%%%%%%%%%%%%%%%%%%%%%%%%%%%%%%%%%%%%%%%%%%%%%%%%%

\section{\MF{}}

\Question[Q-mf]{Getting \MF{} to do what you want}

\MF{} allows you to create your own fonts, and most \TeX{} users
will never need to use it. \MF{}, unlike \TeX{}, requires some
customisation: each output device for which you will be generating
fonts needs a mode associated with it. Modes are defined using the
|mode_def| convention described on page~94 of \emph{The \MF{}book}
(see \Qref[question]{\TeX{}-related books}{Q-books}). You will need
a file, which conventionally called \File|local.mf|, containing all the
|mode_def|s you will be using. If \File|local.mf| doesn't already
exist, Karl Berry's collection of modes,
available as \CTANref{modes-file}, is a good starting point 
(it can be used as a `\File|local.mf|' without modification in a `big
enough' implementation of \MF{}). Lists of
settings for various output devices are also published periodically in
\textsl{TUGboat} (see \Qref[question]{\acro{TUG}}{Q-TUG}). Now create
a |plain| base
file using \ProgName|inimf|, |plain.mf|, and |local.mf|:
\begin{htmlversion}
\begin{verbatim}
   % inimf
   This is METAFONT...
   **plain # you type plain
   (output)
   *input local # you type this
   (output)
   *dump # you type this
   Beginning to dump on file plain...
   (output)
\end{verbatim}
\end{htmlversion}
\htmlignore
\par\vspace{\topsep}
%\begin{tabular}{@{}l@{}l@{}}
%|% inimf|\\
%This is METAFONT\dots{}\\{}
%**|plain|& you type `|plain|'\\{}
%(output)\\{}
%*|input local|& you type this\\{}
%(output)\\{}
%*|dump|& you type this\\{}
%Beginning to dump on file plain\dots{}\\{}
%(output)\\{}
%\end{tabular}\par
\begin{list}{}{}\item\relax
  |% inimf|\\
  This is METAFONT\dots{}\\{}
  \fullline{**\texttt{plain}\hss you type `\texttt{plain}'}
  (\emph{output})\\{}
  \fullline{*\texttt{input local}\hss you type this}
  (\emph{output})\\{}
  \fullline{*\texttt{dump}\hss you type this}
  Beginning to dump on file plain\dots{}\\{}
  (\emph{output})
\end{list}\par
\endhtmlignore
This will create a base file named \File|plain.base| (or something
similar; for example, it will be
\htmlignore
\acro{\File|PLAIN.BAS|}
\endhtmlignore
\begin{htmlversion}
  \File|PLAIN.BAS|
\end{htmlversion}
on \MSDOS{} systems) which should be moved to the directory containing
the base files on your system (note that some systems have two or more
such directories, one for each `size' of \MF{} used).

Now you need to make sure \MF{} loads this new base when it starts up. If
\MF{} loads the |plain| base by default on your system, then you're
ready to go. Under Unix (using the default \ProgName|web2c|
distribution\begin{footnoteenv}
The \ProgName|command|\_\ProgName|name| is symbolically linked to
\ProgName|virmf|, and \ProgName|virmf| loads \File|command_name.base|
\end{footnoteenv})
this does indeed happen, but we could for instance define a command
\ProgName|mf| which executes |virmf &plain| loading the |plain| base
file.

The usual way to create a font with |plain| \MF{} is to start 
it with the line
\htmlignore
\bgroup
\par\vskip\topsep
\hangindent3em\hangafter1\raggedright\parindent10pt\relax
|\mode=<mode name>;| |mag=<magnification>;| |input <font file name>|%
\par\vskip\topsep
\egroup
\noindent in
\endhtmlignore
\begin{htmlversion}
\begin{verbatim}
\mode=<mode name>; mag=<magnification>; input <font file name>
\end{verbatim}
in
\end{htmlversion}
response to the `**' prompt or on the \MF{} command line. (If
|<mode name>| is unknown or omitted, the mode defaults to `proof' and
\MF{} will produce an output file called \File|<font file
name>.2602gf|)  The |<magnification>| is a floating point number or
`magstep' (magsteps are defined in \emph{The \MF{}book} and \emph{The
\TeX{}book}).  If |mag=<magnification>| is omitted, then the default
is 1 (magstep 0).  For example, to generate cmr10 at 12pt for an epson
printer you would type
\begin{verbatim}
  mf \mode=epson; mag=magstep 1; input cmr10
\end{verbatim}
Note that under Unix the |\| and |;| characters must usually be quoted or
escaped, so this would typically look something like
\begin{verbatim}
  mf '\mode=epson; mag=magstep 1; input cmr10'
\end{verbatim}

If you don't have \ProgName|inimf| or need a special mode that isn't
in the base, you can put its commands in a file (\emph{e.g.},
\File|ln03.mf|) and invoke it on the fly with the |\smode| command.
For example, to create \File|cmr10.300gf| for an \acro{LN}03 printer, using
the file
\begin{verbatim}
   % This is ln03.mf as of 2/27/90
   % mode_def courtesy of John Sauter
   proofing:=0;
   fontmaking:=1;
   tracingtitles:=0;
   pixels_per_inch:=300;
   blacker:=0.65;
   fillin:=-0.1;
   o_correction:=.5;
\end{verbatim}
(note the absence of the |mode_def| and |enddef| commands), you would
type
\begin{verbatim}
   mf \smode="ln03"; input cmr10
\end{verbatim}
This technique isn't one you should regularly use, but it may
prove useful if you acquire a new printer and want to experiment with
parameters, or for some other reason are regularly editing the
parameters you're using.  Once you've settled on an appropriate set of
parameters, you should use them to rebuild the base file that you use.

A summary of the above written by Geoffrey Tobin, and tips about
common pitfalls in using \MF{}, is available as \CTANref{mf-beginners}

\Question{Which font files should be kept}

\MF{} produces from its run three files, a metrics (\acro{TFM}) file, a
generic font (\acro{GF}) file, and a log file; all of these files have the
same base name as does the input (\emph{e.g.}, if the input file was
\File|cmr10.mf|, the outputs will be \File|cmr10.tfm|,
\File|cmr10.nnngf|\begin{footnoteenv}
                    Note that the file name may be transmuted by such
                    operating systems as \MSDOS{}, which don't permit
                    long file names
                  \end{footnoteenv}
and \File|cmr10.log|).

For \TeX{} to use the font, you need a \acro{TFM} file, so you need
to keep that.  However, you are likely to generate the same font
at more than one magnification, and each time you do so you'll
(incidentally) generate another \acro{TFM} file; these files are
all the same, so you only need to keep one of them.

To preview or to produce printed output, the \acro{DVI} processor will need a
font raster file; this is what the \acro{GF} file provides.  However, while
there used (once upon a time) to be \acro{DVI} processors that could use
\acro{GF} files, modern processors use packed raster (\acro{PK}) files.
Therefore, you need to generate a \acro{PK} file from the \acro{GF} file; the
program \ProgName|gftopk| does this for you, and once you've done that you
may throw the \acro{GF} file away.

The log file should never need to be used, unless there was some sort
of problem in the \MF{} run, and need not be ordinarily kept.

\Question{Getting bitmaps from the archives}

Most people these days start using \TeX{} with a 300 dots-per-inch (dpi)
laser printer, and Computer Modern bitmap fonts for this resolution
are supplied with most \TeX{} packages.   There are also two such sets
available on \acro{CTAN}: \CTANref{pk300} (for write-black printer engines)
and \CTANref{pk300w} (for write-white engines).
However, some users want to
send their work to high quality typesetting machines (typically with a
resolution of 1270~dpi or greater); it is also becoming more common to
use a 600~dpi laser printer. Why don't the archives or suppliers
provide bitmap fonts at these sizes? There are two reasons:
\begin{enumerate}
\item When a bitmap font is created with \MF{}, it needs to know the
  characteristics of the device; who knows what 600 or 1270~dpi device
  you have?  (Of course, this objection applies equally well to
  300~dpi printers.)
\item Bitmap fonts get \emph{big} at high resolutions. Who knows what
  fonts at what sizes you need?
\end{enumerate}
It would be possible to provide some set of 1270~dpi bitmap fonts in
the archives, but it would take a lot of space, and might not be right
for you. 

So what to do?  You can build the fonts you need yourself with \MF{}:
this isn't at all hard, and some drivers help you (\ProgName|dvips|, and the
em\TeX{} drivers) construct the \MF{} commands. You might need to look
at Karl Berry's collection of \MF{} modes (\CTANref{modes-file}). 
Alternatively, if it is a PostScript device you have, consider using
the fonts in Type~1 font format. You can buy all the Computer Modern
fonts in
outline form from Blue Sky Research, Kinch or Y\&Y 
\htmlignore
(\Qref{}{Q-commercial} for addresses),
\endhtmlignore
\begin{htmlversion}
  (\Qref{commercial vendors}{Q-commercial} for addresses),
\end{htmlversion}
or you can use Basil
Malyshev's public domain versions in \CTANref{ps-type1} (the Paradissa
collection is complete, but has largely been replaced by the better
\begin{htmlversion}
  BaKoMa collection).%
\end{htmlversion}
\htmlignore
\acro{B}a\acro{K}o\acro{M}a collection).%
\endhtmlignore
%\Q{Isn't it a bit over the top to refer to whole directory?}

%%%%%%%%%%%%%%%%%%%%%%%%%%%%%%%%%%%%%%%%%%%%%%%%%%%%%%%%%%%%%%%%%

\section{PostScript and \TeX{}}

\Question[Q-usepsfont]{Using PostScript fonts with \protect\TeX{}}

In order to use PostScript fonts, \TeX{} needs
\emph{metric} (called \acro{TFM}) files. Several sets of metrics are
available from the archives; for mechanisms for generating new ones,
see \Qref[question]{metrics for PostScript fonts}{Q-metrics}.  You
also need the fonts themselves; PostScript printers come with a set of
fonts built in, but to extend your repertoire you almost invariably
need to buy from one of the many commercial font vendors (see, for
example, \Qref[question]{choice of fonts}{Q-psfchoice}).

If you use \LaTeXe{}, the best way to get PostScript fonts into your
document is to use the \acro{PSNFSS} package maintained by Sebastian Rahtz
and Alan Jeffrey (available in \CTANref{psnfss}); it's supported by
the \LaTeX{}3 project team, so bug reports can and should be submitted. 
\acro{PSNFSS} gives you a set of packages for changing the default
roman, sans-serif and typewriter fonts; \emph{e.g}., \File|times.sty| will set
up Times Roman, Helvetica and Courier in place of Computer Modern,
while \File|avant.sty| just changes the sans-serif family to
AvantGarde. To go with these packages, you will need the font metric
files (watch out for encoding problems! see
\Qref[question]{metrics for PostScript fonts}{Q-metrics})
and font description (\texttt{.fd}) files for each font family you
want to use. These can be obtained from \CTANref{psfonts}, arranged by
vendor (\emph{e.g}., Adobe, Monotype, \emph{etc}.). For convenience,
metrics for the common `35' PostScript fonts found in most printers
are provided with \acro{PSNFSS}, packaged as \CTANref{lw35nfss-zip}

For older versions of \LaTeX{} there are various schemes, of which the
simplest to use is probably the \acro{PS}\LaTeX{} macros distributed with
\ProgName|dvips|.

For |plain| \TeX{}, you load whatever fonts you like; if the encoding of
the fonts is not the same as Computer Modern it will be up to you to
redefine various macros and accents, or you can use the font
re-encoding mechanisms available in many drivers and in
\ProgName|ps2pk| and \ProgName|afm2tfm|.

Victor Eijkhout's sophisticated Lollipop package (\CTANref{lollipop})
supports declaration of font families and styles in a similar way to
\LaTeX{}'s \acro{NFSS}, and so is easy to use with PostScript fonts.

Some common problems encountered are discussed elsewhere
(see \Qref[question]{problems with \acro{PS} fonts}{psfonts-problems}).

\Question[ps-preview]{Previewing files using PostScript fonts}

Most \TeX{} previewers only display bitmap \acro{PK} fonts. If you want to
preview documents using PostScript fonts, you have three choices:
\begin{enumerate}
\item Convert the \acro{DVI} file to PostScript and use a
  PostScript previewer. Some modern Unix X implementations have this
  built in (as does \acro{N}e\acro{XT}-step); (X11) Unix, Windows, \acro{OS/}2, and \MSDOS{}
  users can use the free \ProgName|GhostScript| (\CTANref{ghostscript}), a
  complete level 2 implementation.
\item Under Windows on a \acro{PC}, or on a Macintosh, let Adobe Type Manager
  display the fonts. Textures (Macintosh) works like this, and under
  Windows you can use Y\&Y's \ProgName|dviwindo| for bitmap-free
  previewing.
\htmlignore
  (See \Qref[question]{}{Q-commercial} for details of these
  suppliers.)
\endhtmlignore
\begin{htmlversion}
    (See \Qref{commercial suppliers}{Q-commercial} for details.)
\end{htmlversion}
\item If you have the PostScript fonts in Type~1 format,
  use \ProgName|ps2pk| (\CTANref{ps2pk}) or \ProgName|gsftopk|
  (designed for use with the \ProgName|GhostScript| fonts;
  \CTANref{gsftopk}) to make \acro{PK} bitmap fonts which your previewer
  will understand. This can produce excellent results, also suitable
  for printing with non-PostScript devices. Check the legalities of
  this if you have purchased the fonts. The very commonest PostScript
  fonts such as Times and Courier come in Type~1 format on disk with
  Adobe Type Manager (often bundled with Windows, and part of \acro{OS/}2).
\end{enumerate}

\Question[Q-metrics]{\protect\TeX{} font metric files for PostScript fonts}

Font vendors such as Adobe supply metric files for each font, in \acro{AFM}
(Adobe Font Metric) form; these can be converted to \acro{TFM}
(\TeX{} Font Metric) form. The \acro{CTAN} archives have prebuilt metrics
which will be more than enough for many people (\CTANref{psfonts};
beware~--- this directory is at the root of a huge tree), but you may
need to do the conversion yourself if you have special needs or
acquire a new font. One important question is the \emph{encoding} of
(Latin character) fonts; while we all more or less agree about the
position of about 96 characters in fonts (the basic \acro{ASCII} set), the
rest of the (typically) 256 vary. The most obvious problems are with
floating accents and special characters such as the `pounds sterling'
sign. There are three ways of dealing with this: either you change the
\TeX{} macros which reference the characters (not much fun, and
error-prone); or you change the encoding of the font (easier than you
might think); or you use
\htmlignore
virtual fonts (\Qref{}{virtualfonts})
\endhtmlignore
\begin{htmlversion}
\Qref{virtual fonts}{virtualfonts}
\end{htmlversion}
to \emph{pretend} to
\TeX{} that the encoding is the same as it is used to. If you use
\LaTeXe{}, it allows for changing the encoding in \TeX{}; read the
\emph{\LaTeX{} Companion}
(see \Qref[question]{\TeX{}-related books}{Q-books}) for more details. 
In practice, if you do much non-English (but Latin script)
typesetting, you are strongly recommended to use the \texttt{fontenc}
package with option `\texttt{T1}' to select \acro{T}1
\htmlignore
  (`Cork': \Qref{}{dc-fonts})
\endhtmlignore
\begin{htmlversion}
  (\Qref{`Cork'}{dc-fonts})
\end{htmlversion}
encoding. 

Alan Jeffrey's \ProgName|fontinst| package (\CTANref{fontinst}) is an
\acro{AFM} to \acro{TFM} converter written in \TeX{}; it is used to
generate the
files used by \LaTeXe{}'s \acro{PSNFSS} package 
to support use of PostScript fonts. It is a sophisticated package, not
for the faint-hearted, but is powerful enough to cope with most needs.
Much of its power relies on the use of
\htmlignore
virtual fonts (\Qref{virtual fonts}{virtualfonts}).
\endhtmlignore
\begin{htmlversion}
\Qref{virtual fonts}{virtualfonts}.
\end{htmlversion}

For slightly simpler problems, Rokicki's \ProgName|afm2tfm|,
distributed with \ProgName|dvips| (\CTANref{dvips}), is fast and
efficient; note that the metrics and styles that come with
\ProgName|dvips| are \emph{not} currently \LaTeXe{} compatible, but
Karl Berry plans to distribute metrics directly compatible with \acro{PSNFSS}
in his \ProgName|dvipsk| package.

For the Macintosh, there is a program called \ProgName|EdMetrics| which does
the job (and more).  It comes with the Textures distribution, but is
in fact free software, available as \CTANref{edmetrics}

\MSDOS{} users can buy
(see \Qref[question]{commercial implementations}{Q-commercial})
Y\&Y's Font Manipulation Tools package which includes a powerful
\ProgName|afmtotfm| program among many other goodies.

\Question[psfonts-problems]{Problems using PostScript fonts}

For the typical \LaTeX{} user trying to use the
\htmlignore
\acro{PSNFSS} (\Qref{}{Q-usepsfont})
\endhtmlignore
\begin{htmlversion}
\Qref{\acro{PSNFSS}}{Q-usepsfont}
\end{htmlversion}
package, three questions often arise.
First, you have to declare to the \acro{DVI} driver that you are using
PostScript fonts; in the case of \ProgName|dvips|, this means adding
lines to the |psfonts.map| file. Otherwise, \ProgName|dvips| will try
to find \acro{PK} 
files. If the font isn't built into the printer, you have to acquire
it (in many cases this means buying it from a commercial supplier!).
You then have to instruct the driver to download it with each job (the
mechanism depends on your driver). So it's no
good just installing the \emph{metrics} for Optima and expecting it to
work. You have to pay hard cash for the font itself, which will come
(for Unix and \MSDOS{} users) in |pfb| (Printer Font Binary) form.

Second, you cannot expect your previewer to suddenly start displaying
PostScript fonts; most of them only know about \acro{PK} bitmap fonts
such as Computer Modern. \ProgName|ps2pk| (\CTANref{ps2pk}) can create
these from the |pfb| file you have bought; this would also let you use
the fonts with non-PostScript device drivers such as the em\TeX{}
ones.  You are responsible for making sure you are not breaking the
licence restrictions on font you bought.

Third, the stretch and shrink between words is a function of the
font metric; it is not specified in \acro{AFM} files, so different converters
choose different values. The PostScript metrics that come with \acro{PSNFSS} 
used to produce quite tight setting, but they were revised in mid 1995
to produce a compromise between American and European practice. Really
sophisticated users may not find even the new the values to their taste, and
want to override them. Even the casual user may find  more
hyphenation or overfull boxes than \acro{CMR} produces; but \acro{CMR}
is extremely generous. 

\Question[Q-psfchoice]{Choice of scalable outline fonts}

If you are interested in text alone, you can use any of over 20,000
fonts(!) in Adobe Type~1 format (called `PostScript fonts' in the
\TeX{} world and `\acro{ATM} fonts' in the DTP world), or any of several
hundred fonts in TrueType format.  That is, provided of course, that
your previewer and printer driver support scalable outline fonts.

\TeX{} itself \emph{only} cares about metrics, not the actual
character programs.  You just need to create a \TeX{} metric file
\acro{TFM} using some tool such as \ProgName|afm2tfm|, \ProgName|afmtotfm|
(from Y\&Y, see \Qref[question]{commercial implementations}{Q-commercial})
or \ProgName|fontinst|.  For the previewer or printer driver you need the
actual outline font files themselves (|pfa| for Display PostScript, |pfb|
for \acro{ATM} on \acro{IBM} \acro{PC}, Mac outline font files on Macintosh).

If you also need mathematics, then you are severely limited by the
demands that \TeX{} makes of maths fonts (for details, see the paper by B.K.P. 
Horn in \TUGboat{} 14(3)).
For maths, then, there are relatively few choices:
\begin{booklist}
\item[Computer Modern] (75 fonts --- optical scaling) Donald E. Knuth\\
    Note that \acro{CM} \emph{is} available in scalable outline form.
    There are commercial as well as public domain versions, and
    there are both Adobe Type~1 and TrueType versions.
    Some of these are `commercial grade,' with full hand-tuned hinting,
    some render very poorly, while others are merely incompatible with 
    Adobe Type Manager (\acro{ATM}).
\item[Lucida Bright \emph{with} Lucida New Math] (25 fonts) Chuck Bigelow and
  Kris Holmes\\
    Lucida is a family of related fonts including seriffed, sans serif,
    sans serif fixed width, calligraphic, blackletter, fax,
    Kris Holmes' connected handwriting font, \emph{etc}; they're
    not as `spindly' as Computer Modern, with a large x-height, and
    include a larger set of maths symbols, operators, relations and
    delimiters than \acro{CM} (over 800 instead of 384: among others, it also
    includes the \acro{AMS} |msam| and |msbm| symbol sets).
    The planned `Lucida Bright Expert' (14 fonts)
    adds seriffed fixed width, another handwriting font,
    smallcaps, bold maths, upright `maths italic', \emph{etc}., to the set
    The distribution includes support for use with |plain| \TeX{} and
    \LaTeXo{}.  Support under \LaTeXe{} is provided in
\htmlignore
    \acro{PSNFSS} (\Qref{}{Q-usepsfont})
\endhtmlignore
\begin{htmlversion}
    \Qref{\acro{PSNFSS}}{Q-usepsfont}
\end{htmlversion}
    thanks to Sebastian Rahtz.
\item[MathTime 1.1] (3 fonts) \TeX{}plorators (Michael Spivak)\\
    The set contains maths italic, symbol, and extension
    fonts, designed to work well with Times-Roman.  These are
    typically used with Times, Helvetica and Courier (which are
    resident on many printers, and which are supplied with
    some \acro{PC} versions).  In addition you may want to complement this
    basic set with Adobe's Times Smallcap, and perhaps the set of
    Adobe `Math Pi' fonts, which include blackboard bold, blackletter,
    and script faces.  The distribution includes support for use with
    |plain| \TeX{} and \LaTeXo{} (including code to link in Adobe Math
    Pi~2 and Math Pi~6).
    Support under \LaTeXe{} is provided in
\htmlignore
    \acro{PSNFSS} (\Qref{}{Q-usepsfont})
\endhtmlignore
\begin{htmlversion}
    \Qref{\acro{PSNFSS}}{Q-usepsfont}
\end{htmlversion}
    thanks to Sebastian Rahtz.
\item[Adobe Lucida, LucidaSans \emph{and} LucidaMath] (12 fonts)\\
    Lucida and LucidaMath are generally considered to be a bit heavy.
    The three maths fonts contain only the glyphs in the \acro{CM}
    maths italic, symbol, and extension fonts.
    Support for using LucidaMath with \TeX{} is not very good;
    you will need to do some work reencoding fonts \emph{etc}.
    (In some sense this set is the ancestor of the LucidaBright
    plus LucidaNewMath font set.)
\item[Concrete, \emph{the} \acro{AMS} maths fonts \emph{etc}.]Donald
  E. Knuth and the \acro{AMS}.\\
    These are sometimes mentioned as alternatives to \acro{CM}, but they are
    really adjuncts, in that you need to use at least the basic \acro{CM}
    maths fonts with them.
\item[Proprietary fonts] Various sources.\\
    Since having a high quality font set in scalable outline form that
    works with \TeX{} can give a publisher a real competitive advantage, there
    are some publishers that have paid (a lot) to have such font sets made
    for them.  Unfortunately, these sets are not available on the open
    market, despite the likelihood that they're more complete than
    those that are.
\item[Mathptm] (4 fonts) Alan Jeffrey.\\
   This set contains maths italic, symbol, extension, and roman 
   virtual fonts, built from Adobe Times, Symbol, Zapf Chancery, and
   the Computer Modern fonts.  The Mathptm fonts are free, and
   the resulting PostScript files can be freely exchanged.
   Contains most of the \acro{CM} math symbols. 
   Support under \LaTeXe{} in
\htmlignore
    \acro{PSNFSS} (\Qref{}{Q-usepsfont})
\endhtmlignore
\begin{htmlversion}
    \Qref{\acro{PSNFSS}}{Q-usepsfont}
\end{htmlversion}
    thanks to Alan Jeffrey and Sebastian Rahtz.  

    (A similar development by Thomas Esser, using the Adobe Palatino
    fonts, is available from \CTANref{mathppl})
\end{booklist}

All of the first three font sets are available in formats suitable for
\acro{IBM} \acro{PC}/Windows, Macintosh and Unix/\acro{N}e\acro{XT} from Y\&Y and from Blue Sky
Research (see \Qref[question]{commercial suppliers}{Q-commercial} for
details).  The MathTime fonts are also available from:
\begin{quote}
  \TeX{}plorators\\
  1572 West Gray \#377\\
  Houston \acro{TX} 77019\\
  \acro{USA}
\end{quote}
The very limited selection of maths font sets is a direct result of
the fact that a maths font has to be explicitly designed for use with
\TeX{} and as a result it is likely to lose some of its appeal in
other markets. Furthermore, the \TeX{} market for commercial fonts is
minute (in comparison, for example, to Microsoft TrueType font pack
\#1, which sold something like 10 million copies in a few weeks after
release of Windows 3.1!).

Text fonts in Type~1 format are available from many vendors including
Adobe, Monotype, Bitstream.  Avoid cheap rip-offs: not only are you
rewarding unethical behaviour, destroying the cottage industry of
innovative type design, but you are \emph{also} very likely to get junk.
The fonts may not render well (or at all under \acro{ATM}), may not have the
`standard' complement of 228 glyphs, or may not include metric files
(needed to make \acro{TFM} files). Also, avoid TrueType fonts from all but
the major vendors.  TrueType fonts are an order of magnitude harder to
`hint' properly than Type~1 fonts and hence TrueType fonts from places
other than Microsoft and Apple may be suspect.  In any case you may
find other problems with TrueType fonts such as service bureaux not
accepting jobs calling for them.

\begin{comment}
Incidentally, some people have been disappointed in the quality of scalable
fonts. Of course, quality is in the eye of the beholder and people disagree
sometimes on what is `better' --- but in many cases there is no question!
There are several factors that come into play.  One is the quality of
the font itself (you get what you pay for), and the other is the quality of
the rasteriser.  

Rendering under \acro{ATM} of commercial grade fonts is outstanding.
Rendering in
Display PostScript is not quite as good (which may change if \acro{DPS} systems
adopt the \acro{ATM} rasteriser).  PostScript rasterisers in printers
are not as
good as \acro{ATM} either (except for a few printers that use the \acro{ATM}
rasteriser)~--- but then at 300~dpi you are dealing with a much more forgiving
environment than at 96~dpi (Windows) or 72~dpi (Mac).  Clone \acro{PS}
interpreters
tend to not be as good as true Adobe \acro{RIP}s.  In some cases the
difference is
not huge (\acro{HP}), in other cases it is (public domain).  Utilities for
converting from \acro{PS} to \acro{PK} format typically do not produce
results comparable
to \acro{ATM} (of course, at high enough resolution --- where
grid-fitting is less
important --- it hardly matters).  
\end{comment}

\Question[postscript-pics]{Including a PostScript figure in \LaTeX{}}

\htmlignore
\LaTeXe{} (\Qref{}{latex2e})
\endhtmlignore
\begin{htmlversion}
\Qref{\LaTeXe{}}{latex2e}
\end{htmlversion}
has a standard package
for graphics inclusion, rotation, colour, and other driver-related
features. The package is documented in the second edition of the
Lamport's \LaTeX{} book
(see \Qref[question]{\TeX{}-related books}{Q-books}), and is available
in \CTANref{graphics}

If you don't use \LaTeXe{}, perhaps the best method is to use
the |psfig| macros written by Trevor Darrell, available in
\CTANref{psfig}

You will also need a \acro{DVI} to PostScript conversion program that
supports the \cs|special|s. The drivers mentioned in
\Qref[question]{DVI to PostScript programs}{Q-dvips} do, and come
with a version of |psfig| ready to use with them. The |psfig| macros
work best with Encapsulated PostScript Files (\acro{EPS}). In particular,
|psfig| will need the file to have a BoundingBox (see Appendix H of
the
\emph{PostScript Language Reference Manual}). If you don't have an \acro{EPS}
file, life can be difficult.

One point to note about including PostScript figures is that they
are not part of the \acro{DVI} file, but are only included when you
use a \acro{DVI}
to PostScript conversion program. As a result, most \acro{DVI} previewers
will simply show the blank space \TeX{} has reserved for your figure,
not the figure itself. 

There are two rather good documents on \acro{CTAN} addressing of figure
production with rather different emphasis.  Anil K. Goel's,
\CTANref{figsinlatex} covers the different ways in which you might
generate figures, and one the old (\LaTeXo{}) ways of including them
into documents.  Keith Reckdahl's, \CTANref{epslatex}, covers the
standard \LaTeXe{} facilities, as well as some of the supporting
packages, notably \Package|subfigure| (\CTANref{subfigure}) and
\Package|psfrag| (\CTANref{psfrag}).

%%%%%%%%%%%%%%%%%%%%%%%%%%%%%%%%%%%%%%%%%%%%%%%%%%%%%%%%%%%%%%%%%

\section{Special sorts of typesetting}

\Question{Drawing with \TeX{}}

There are many packages to do pictures in \AllTeX{} itself (rather than
importing graphics created externally), ranging from simple use of
\LaTeX{} |picture| environment, through enhancements like \ProgName|epic|, to
sophisticated (but slow) drawing with \PiCTeX{}. Depending on your type
of drawing, and setup, four systems should be at the top of your list
to look at:
\begin{enumerate}
\item \CTANref{pstricks}; this gives you access to all the power of
  PostScript from \TeX{} itself, by sophisticated use of
  \cs|special|s. You need a decent \acro{DVI} to PostScript driver
  (like \ProgName|dvips|), but the results are worth it.  The
  well-documented package gives you not only low-level drawing
  commands (and full colour) like lines, circles, shapes at arbitrary
  coordinates, but also high-level macros for framing text, drawing
  trees and matrices, 3\acro{D} effects, and more.
\item \MP{}; you liked \MF{}, but never got to grips with font files?
\htmlignore
  Try \MP{} (\Qref{}{Q-MP})~---
\endhtmlignore
\begin{htmlversion}
  Try \Qref[]{\MP{}}{Q-MP}~---
\end{htmlversion}
  all the power of \MF{}, but it generates PostScript figures.  Knuth
  uses it for all his work\dots{}
\item \ProgName|Mfpic|; you liked \MF{}, but can't understand the
  language?  The package (\CTANref{mfpic}) makes up \MF{} code for you
  within using familiar-looking \TeX{} macros.  Not \emph{quite} the
  full power of \MF{}, but a friendlier interface.
\item You liked \PiCTeX{} but don't have enough memory or time?  Look
  at Eitan Gurari's \CTANref{dratex}, which is as powerful as most
  other \TeX{} drawing packages, but is an entirely new
  implementation, which is not as hard on memory, is much more
  readable (and is fully documented).
\end{enumerate}

\Question[Q-linespace]{Double-spaced documents in \LaTeX{}}

Are you producing a thesis, and trying to obey regulations that were
drafted in the typewriter era?  Or are you producing copy for a
journal that insists on double spacing for the submitted articles?

\LaTeX{} is a typesetting system, so the appropriate design
conventions are for ``real books''.  If your requirement is from
thesis regulations, find whoever is responsible for the regulations,
and try to get the wording changed to cater for typeset theses
(\emph{e.g.}, to say ``if using a typesetting system, aim to make your
thesis look like a well-designed book'').  (If your requirement is
from a journal, you're probably even less likely to be able to get the
rules changed, of course.)

If you fail to convince your officials, or want some inter-line 
space for copy-editing:
\begin{itemize}
\item Try changing \cs|baselinestretch|:
  \htmlignore
  |\renewcommand|\linebreak[4]|{\baselinestretch}{1.2}|
  may be enough to give
  \endhtmlignore
\begin{htmlversion}
  |\renewcommand{\baselinestretch}{1.2}| may be enough to give
\end{htmlversion}
  officials the impression you've kept to their regulations. Don't try
  changing |\baselineskip|: its value is reset at any size-changing
  command.
\item Alternatively, use a line-spacing package.  Options available
  are:
  \begin{itemize}
  \item for simple double spacing, \CTANref{doublespace}, and
  \item for greater flexibility, \CTANref{setspace}, which has been
    upgraded for \LaTeXe{}.
  \end{itemize}
\end{itemize}

\Question{Formatting a thesis in \LaTeX{}}

Thesis styles are usually very specific to your University, so it's
usually not profitable to ask around for a package outside your own
University.  Since many Universities (in their eccentric way) still
require double-spacing, you may care to refer to
\htmlignore
\Qref[question]{}{Q-linespace}.
\endhtmlignore
\begin{htmlversion}
the question on \Qref[question]{double-spacing}{Q-linespace}.
\end{htmlversion}
If you want to write
your own, a good place to start is the University of California style
(available as \CTANref{ucthesis}), but
it's not worth going to a lot of trouble. (If officials won't allow
standard typographic conventions, you won't be able to produce an
\htmlignore
\ae{}sthetically pleasing document anyway!)
\endhtmlignore
\begin{htmlversion}
  aesthetically pleasing document anyway!)
\end{htmlversion}

\Question[text-flow]{Flowing text around figures in \LaTeX{}}

There are several \LaTeX{} packages that purport to do this, but they
all have their limitations because the \TeX{} machine isn't really
designed to solve this sort of problem.  Piet van Oostrum has
conducted a survey of the available packages; he recommends:

\begin{description}
\item[\texttt{picins}] \File|picins.sty| is part of a large package
  (\CTANref{picins}) that allows inclusion of pictures (e.g., with
  shadow boxes, various \MSDOS{} formats, etc.).  The command is:

\begin{htmlversion}
    |\parpic(width,height)(x-off,y-off)[Options][Position]{Picture}|\\
\end{htmlversion}
  \htmlignore
  \def\breakhere{\penalty0\hskip0pt\relax}
  |\parpic(|\emph{width}|,|\emph{height}|)(|\emph{x-off}|,|\emph{y-off}|)|%
  |[|\emph{Options}|][|\emph{Position}|]|\\
  |       {|\emph{Picture}|}|\\
  \endhtmlignore
  \emph{Paragraph text}

  All parameters except the \emph{Picture} are optional.  The picture
  can be positioned left or right, boxed with a rectangle, oval,
  shadowbox, dashed box, and a caption can be given which will be
  included in the list of figures.

%    This is the only package that I tried that correctly works inside an
%    enumerate/itemize item. It does not, however work with
%    enumerate/itemize besides the picture (i.e. started after the \parpic
%    command) but neither does any of the other packages as far as I could
%    discern.

  Unfortunately (for those of us whose understanding of German is not
  good), the documentation is in German.  Piet van Oostrum has written
  an English summary \CTANref{picins-summary}
\item[\texttt{floatflt}] \CTANref{floatflt} is an improved version
  (for \LaTeXe{}) of \File|floatfig.sty|, and its syntax is:

\begin{htmlversion}
\begin{verbatim}
\begin{floatingfigure}[options]{width of figure}
   figure contents
\end{floatingfigure}
\end{verbatim}
\end{htmlversion}
  \htmlignore
  |\begin{floatingfigure}[|\emph{options}|]{|\emph{width of figure}|}|\\
  |   |\emph{figure contents}\\
  |\end{floatingfigure}|
  \endhtmlignore

  There is a (more or less similar) |floatingtable| environment.

  The tables or figures can be set left or right, or alternating on
  even/odd pages in a double-sided document.

  The package works with the |multicol| package, but doesn't work well
  in the neighbourhood of list environments (unless you change your
  \LaTeX{} document).
\item[\texttt{wrapfig}] \CTANref{wrapfig} has syntax:

\begin{htmlversion}
\begin{verbatim}
\begin{wrapfigure}[height of figure in lines]{l|r}[overhang]{width}
  figure, caption, etc.
\end{wrapfigure}
\end{verbatim}
\end{htmlversion}
  \htmlignore
  |\begin{wrapfigure}[|\emph{height of figure in lines}|]|%
    |{l|,|r|,\emph{etc}|}|\\
  |                  [|\emph{overhang}|]{|\emph{width}|}|\\
  |   |\emph{figure, caption, etc.}\\
  |\end{wrapfigure}|
  \endhtmlignore

  The syntax of the |wraptable| environment is similar.

  Height can be omitted, in which case it will be calculated by the
  package; the package will use the greater of the specified and the
  actual width.  The |{l|\emph{,}|r|\emph{,etc}.|}| parameter can also be
  specified as |i|\emph{(nside)} or |o|\emph{(utside)} for two-sided
  documents, and uppercase can be used to indicate that the picture
  should float.  The overhang allows the figure to be moved into the
  margin.  The figure or table will entered into the list of figures
  or tables if you use the |\caption| command.

  The environments do not work within list environments that end
  before the figure or table has finished, but can be used in a parbox
  or minipage, and in twocolumn format.
\end{description}

%\htmlignore
%\subQ
%\endhtmlignore
\Question{Alternative head- and footlines in \LaTeX{}}

The standard \LaTeX{} document classes define a small set of `page
styles' which (in effect) specify head- and footlines for your
document.  The set defined is very restricted, but \LaTeX{} is capable
of much more; people occasionally set about employing \LaTeX{}
facilities to do the job, but that's quite unnecessary~--- Piet van
Oostrum has already done the work.

The package is found in directory \CTANref{fancyheadings} and provides
simple mechanisms for defining pretty much every head- or footline
variation you could want; the directory also contains some (rather
good) documentation and one or two smaller packages.  Fancyheadings
also deals with the tedious behaviour of the standard styles with
initial pages (\Qref{}{Q-ps@empty}), by enabling you to define
different page styles for initial and for body pages.

\Question{Including a file in verbatim in \LaTeX{}}

A good way is to use Rainer Sch\"opf's \File|verbatim.sty|,
which provides the command |\verbatiminput| that takes a file name
as argument. This package is available as part of \CTANref{2etools}

Another way is to use the |alltt| environment, which requires
\File|alltt.sty| (which is now part of \LaTeX{}).

The \File|moreverb| package (\CTANref{moreverb}) extends the
facilities of \File|verbatim| package), providing a |listing|
environment and a |\listinginput| command, which line-number the text
of the file.

\Question{Generating an index in \AllTeX{}}

Making an index is not trivial; what to index, and how to index it, is
difficult to decide, and uniform implementation is difficult to
achieve.  You will need to mark all items to be indexed in your text
(typically with |\index| commands).

It is not practical to sort
a large index within \TeX{}, so a post-processing program is used to sort
the output of one \TeX{} run, to be included into the document at the
next run.

The following programs are available:
\begin{description}
\item[makeindex] for \LaTeX{} under Unix (but runs under other \acro{OS}s
  without changes).  Available in \CTANref{makeindex}; a version for
  the Macintosh is available as \CTANref{macmakeindex}, and ones for
  \MSDOS{} are part of the em\TeX{} and g\TeX{} distributions (the
  em\TeX{} version also runs under \acro{OS/}2).

  The Makeindex documentation is a good source of information on how
  to create your own index. Makeindex can be used with some \TeX{}
  macro packages other than \LaTeX{}, such as \Eplain{}.
\item[idxtex] for \LaTeX{} under \acro{VMS}.  Available (together with a
  glossary-maker called |glotex|) in \CTANref{glo+idxtex}
%\item[texix] for \TeX{} on \acro{CMS} and Macintosh machines.
\item[texindex] A witty little shell/\ProgName|sed|-script-based
  utility for \LaTeX{} under Unix.  Available from \CTANref{texindex}

  There are other programs called \ProgName|texindex|, notably
  one that comes with the
\htmlignore
  Texinfo distribution (\Qref{}{Q-texinfo}).
\endhtmlignore
\begin{htmlversion}
  \Qref{Texinfo}{Q-texinfo} distribution.
\end{htmlversion}
\end{description}

\Question{Using \BibTeX{} with \texttt{plain} \TeX{}}

The file \CTANref{btxmactex} contains macros and documentation
for using \BibTeX{} with |plain| \TeX{}, either directly or with
\htmlignore
\Eplain{} (\Qref{}{Q-eplain}).
\endhtmlignore
\begin{htmlversion}
\Qref{\Eplain{}}{Q-eplain}.
\end{htmlversion}
See \Qref[question]{the use of \BibTeX{}}{BibTeXing} for more
information about \BibTeX{} itself.

\Question{Typesetting music in \TeX{}}

A powerful package which allows the typesetting of
polyphonic and other multiple-stave music is Music\TeX{}, written by
Daniel Taupin 
(\Email|taupin@rsovax.lps.u-psud.fr|). It is available 
in \CTANref{musictex}

In the recent past, Daniel (as well as with various other people,
notably Ross Mitchell and Andreas Egler) have been working on a development of
Music\TeX{}, known as MusiX\TeX{}.  MusiX\TeX{} is a three-pass
system (with a processor program that computes values for the element
spacing in the music), and achieves finer control than is possible in
the unmodified \TeX{}-based mechanism that Music\TeX{} uses.  Daniel
Taupin and Andreas Egler are pursuing distinct versions of
MusiX\TeX{}; they are available, respectively, from
\CTANref{musixtex-taupin} and \CTANref{musixtex-egler}

Digital music fans can typeset notation for their efforts by using
\ProgName|midi2tex|, which translates MIDI data files into Music\TeX{}
source code.  It is available from \CTANref{midi2tex}

A rather simpler notation than Music\TeX{} is supported by \ProgName|abc2mtex|;
this is a package designed to notate tunes stored in an \acro{ASCII} format
(|abc| notation). It was designed primarily for folk and traditional
tunes of Western European origin (such as Irish, English and Scottish)
which can be written on one stave in standard classical notation.
However, it should be extendable to many other types of music.  It is
available from \CTANref{abc2mtex}

There is a mailing list for discussion of typesetting music in \TeX{}.
To subscribe, send mail to \Email|mutex-request@stolaf.edu| containing
the word `|subscribe|' in the body.

\Question{Drawing Feynman diagrams in \LaTeX{}}

Michael Levine's macro package for drawing Feynman diagrams in \LaTeX{}
is available in \CTANref{feynman}

Another possibility is Thorsten Ohl's \CTANref{feynmf}, that works in
combination with \MF{} (or \MP{}).  The \ProgName|feynmf| or
\ProgName|feynmp| package reads a description of the diagram written
in \TeX{}, and writes out code.  \MF{} (or \MP{}) can then produce a
font (or PostScript file) for use in a subsequent \LaTeX{} run.  For
new users, who have access to \MP{}, the PostScript version is
probably the better route, for document portability and other reasons.

%%%%%%%%%%%%%%%%%%%%%%%%%%%%%%%%%%%%%%%%%%%%%%%%%%%%%%%%%%%%%%%%%

\section{How do I do \textsl{X} in \TeX{} or \LaTeX{}}

\Question{Proof environment}

It is not possible to make a |proof| environment which automatically
includes an `end-of-proof' symbol. Some proofs end in displayed maths;
others do not. If the input file contains |...\] \end{proof}| then
\LaTeX{} finishes off the displayed maths and gets ready for a new
line before it reads any instructions connected with ending the proof.
But traditionally the end-of-proof sign goes in the display, not on a
new line. So you just have to put it in by hand in every proof.

\Question{Symbols for the number sets}

It is a good idea to have commands such as \cs|R| for the real numbers and
other standard number sets. Traditionally these were typeset in bold.
Because mathematicians usually do not have access to bold chalk, they
invented the special symbols that are now often used for \cs|R|, \cs|C|,
\emph{etc}.  These symbols are known as ``blackboard bold''. Before
insisting on using them, consider whether going back to the old system
of ordinary bold might not be acceptable (it is certainly simpler).

A set of blackboard bold capitals is available in the \acro{AMS} fonts
``msam'' (\emph{e.g.}, ``msam10'' for 10pt) and ``msbm''. The fonts
have a large number of mathematical symbols to supplement the ones in
the standard \TeX{} distribution. The fonts are available in
\CTANref{amsfonts-symbols}

Two files which load the fonts and define the symbols are provided,
and both work with either \TeX{} or \LaTeX{}.  Questions or
suggestions regarding these fonts should be directed to
\Email|tech-support@math.ams.org|.

Another complete set of blackboard bold fonts, the bbold family, is
available in \MF{} (in \CTANref{bbold}).  This set has the interesting
property of offering blackboard bold forms of lower-case letters,
something rather rarely seen on actual blackboards.

The ``lazy person's'' blackboard bold macros:
\begin{verbatim}
  \newcommand{\R}{{\sf R\hspace*{-0.9ex}%
    \rule{0.15ex}{1.5ex}\hspace*{0.9ex}}}
  \newcommand{\N}{{\sf N\hspace*{-1.0ex}%
    \rule{0.15ex}{1.3ex}\hspace*{1.0ex}}}
  \newcommand{\Q}{{\sf Q\hspace*{-1.1ex}%
    \rule{0.15ex}{1.5ex}\hspace*{1.1ex}}}
  \newcommand{\C}{{\sf C\hspace*{-0.9ex}%
    \rule{0.15ex}{1.3ex}\hspace*{0.9ex}}}
\end{verbatim}
work well at normal size if the surrounding text is |cmr10|. However, 
they are not part of a proper maths font, and so do not work in sub- and 
superscripts. Moreover, the size and position of the vertical bar is 
affected by the font of the surrounding text.
\checked{RAB}{1994/11/12}%

\Question{Roman theorems}

If you want to take advantage of the powerful |\newtheorem| command
without the constraint that the contents of the theorem is in a sloped
font (for example, to use it to create remarks, examples, proofs,
\dots{}) then you can use the style file \File|theorem.sty| (part of
\CTANref{2etools}). Alternatively, the following sets up an
environment |remark| whose content is in roman.
\begin{verbatim}
  \newtheorem{preremark}{Remark}
  \newenvironment{remark}%
    {\begin{preremark}\rm}{\end{preremark}}
\end{verbatim}
This will not work if you are using
\htmlignore
\acro{NFSS} (\Qref{}{NFSS}) outside of \LaTeXe{} (\Qref{}{latex2e}),
\endhtmlignore
\begin{htmlversion}
\Qref{\acro{NFSS}}{NFSS} outside of \Qref{\LaTeXe{}}{latex2e},
\end{htmlversion}
because the command \cs|rm| behaves differently there.


\Question{Labels on lists}

If you want your top-level |enumerate|s to be labelled `I/', `II/',
\dots{}, then give these commands:
\begin{verbatim}
   \renewcommand{\theenumi}{\Roman{enumi}}
   \renewcommand{\labelenumi}{\theenumi/}
\end{verbatim}
The possible styles of numbering are given in Section~6.3 of Lamport's
book (see \Qref[question]{\TeX{}-related books}{Q-books}).  If you are lazy and
just change \cs|labelenumi| in one go then your cross-references will be
wrong.

For lower level |enumerate|s, replace |enumi| by |enumii|, |enumiii|
or |enumiv|, according to the level.  If your label is much larger
than the default, you should also change \cs|leftmargini|,
\cs|leftmarginii|, \emph{etc}.

If you're running \LaTeXe{}, the package \File|enumerate.sty| (part of
\CTANref{2etools}) offers similar facilities.  With
\File|enumerate.sty|, the example above would
be achieved simply by starting the enumeration
\cs|begin{enumerate}[I/]|.

\Question[secnumdep]{Unnumbered sections in the Table of Contents}

The easiest way to get headings of funny `sections' such as
prefaces in  the table of contents is to use the
counter |secnumdepth| described in Appendix~C of the \LaTeX{} 
manual. For example:
\begin{verbatim}
    \setcounter{secnumdepth}{-1}
    \chapter{Preface}
\end{verbatim}
Of course, you have to set |secnumdepth| back to its usual value
(which is~2 in the standard styles) before you do
any `section' which you want to be numbered.

Similar settings are made automatically in the \LaTeX{} book class by
the |\frontmatter| and |\backmatter| commands.

%Robin, could omit the explanation, Ro
This is why it works. 
\cs|chapter| without the star does
\begin{enumerate}
\item put something in the |.toc| file;
\htmlignore
\item if $\mbox{\texttt{secnumdepth}} \geq 0$,
\endhtmlignore
\begin{htmlversion}
\item if the |secnumdepth| counter is greater than or equal to zero,
\end{htmlversion}
      increase the counter for the chapter and write it out.
\item write the chapter title.
\end{enumerate}
Other sectioning commands are similar, but with other values 
used in the test.

\Question{Footnotes in tables}
The standard \LaTeX{} \cs|footnote| command doesn't work in tables; the table
traps the footnotes and they can't escape to the bottom of the page.

If your table is floating, your best bet is (unfortunately) to put the
table in a |minipage| environment and to put the notes
underneath the table, or to use Donald Arseneau's package
\CTANref{threeparttable}

Otherwise, if your table is not floating (it's just a
`|tabular|' in the middle of some text), there are several
things you can do to fix the problem.
\begin{enumerate}
\item Use \cs|footnotemark| to position the little marker
  appropriately, and then put in \cs|footnotetext| commands to fill in
  the text once you've closed the tabular environment.  This is
  described in Lamport's book, but it gets messy if there's more than
  one footnote.
\item Stick the table in a |minipage| anyway.  This provides
  all the ugliness of footnotes in a minipage with no extra effort.
\item Use \CTANref{threeparttable} anyway;
  the package is intended for floating tables, and the result might
  look odd if the table is not floating, but it will be reasonable.
\item Use \File|tabularx| or \File|longtable| from the \LaTeX{} tools
  distribution (\CTANref{2etools}); they're noticeably less
  efficient than the standard |tabular| environment, but they
  do allow footnotes.
\item Grab hold of \File|footnote.sty| from \acro{CTAN}, lurking in
  \CTANref{mdwtools}

  Then put your tabular environment inside a |savenotes|
  environment.  Alternatively, say \cs|makesavenoteenv{tabular}| in
  the preamble of your document, and tables will all handle footnotes
  correctly.
\item Use \File|mdwtab.sty| from the same directory
  (\CTANref{mdwtools}).

  This will handle footnotes properly, and has other facilities to
  increase the beauty of your tables.  It may also cause other
  table-related packages (not the standard `tools' ones, though) to
  become very unhappy and stop working.
\end{enumerate}

\Question{Style of section headings}

Suppose that the editor of your favourite journal has specified that section
headings must be centred, in small capitals, and subsection headings ragged 
right in italic, but that you don't want to get involved in the sort of
programming described in \emph{The \LaTeX{} Companion}
\htmlignore
(\Qref{}{Q-books}; the programming itself is discussed in
\Qref[question]{}{atsigns}).
\endhtmlignore
\begin{htmlversion}
  (see \Qref{\TeX{}-related books}{Q-books}; the
  \Qref{programming}{atsigns} itself is discussed under `@').
\end{htmlversion}
The following hack will 
probably satisfy your editor. Define yourself new commands
\begin{verbatim}
  \newcommand{\ssection}[1]{%
     \section[#1]{\centering\sc #1}}
  \newcommand{\ssubsection}[1]{%
     \subsection[#1]{\raggedright\it #1}}
\end{verbatim}
and then use |\ssection| and |\ssubsection| in place of 
|\section| and |\subsection|. This isn't perfect: section numbers 
remain in bold, and starred forms need a separate redefinition. Also, 
this will not work if you are using 
\htmlignore
\acro{NFSS} (\Qref{}{NFSS}) outside of \LaTeXe{} (\Qref{}{latex2e}),
\endhtmlignore
\begin{htmlversion}
\Qref{\acro{NFSS}}{NFSS} outside of \Qref{\LaTeXe{}}{latex2e},
\end{htmlversion}
because the font-changing commands behave differently there.

\Question{Indent after section headings}

\LaTeX{} implements a style that doesn't indent the first paragraph
after a section heading.  There are coherent reasons for this, but not
everyone likes it.
The package \File|indentfirst.sty| (part of \CTANref{2etools})
suppresses the mechanism, so that the first paragraph is
indented.

\Question{Footnotes in \LaTeX{} section headings}

The \cs|footnote| command is fragile, so that simply placing the
command in \cs|section|'s arguments isn't satisfactory.  Using
\cs|protect|\cs|footnote| isn't a good idea either: the arguments of a
section command are used in the table of contents and (more
dangerously) potentially also in page headings.  Unfortunately,
there's no mechanism to suppress the footnote in the heading while
allowing it in the table of contents, though having footnotes in the
table of contents is probably unsatisfactory anyway.

To suppress the footnote in headings and table of contents:
\begin{itemize}
\item Take advantage of the fact that the mandatory argument doesn't
  `move' if the optional argument is present:
  |\section[title]{title|\cs|footnote||{title footnote}}|
\item Use the (small) package \CTANref{stblftnt}, which makes
  \cs|footnote| automagically disappear in a moving argument.
\end{itemize}

\Question[premargin]{Changing the margins in \LaTeX{}}

Don't do it. Learn some \LaTeX{}, produce some 
documents, and then ask again. 
     
You can never change the  \emph{margins} of a document by software, 
because they depend on the actual size of the paper.  What you can change 
are the distances from the apparent top and left edges of the paper, 
and the width and height of the text. Changing the last two requires 
more skill than you might expect. The height should bear a certain 
relationship to \cs|baselineskip|. And the width should not be more 
than 75~characters. Lamport's warning in his section on 
`Customizing the Style' really must be taken seriously. One-inch 
margins on A4 paper are fine for 10- or 12-pitch typewriters, but not 
for 10pt type (or even 11pt or 12pt) because so many 
characters per line will irritate the reader.  However\dots{}
\checked{RAB}{1994/11/12}%

% \Question{Insisting on changing the margins in \LaTeX{}}
%
% This answer first helps you change the margins throughout a
% document, then tells you how to change the margins in a portion
% of the document.

Perhaps the easiest way to get more out of a page in \LaTeX{} is to
get \CTANref{fullpage}, which sets the margins of the page identical
to those of |plain| \TeX{}, \emph{i.e.}, 1-inch margins at all four
sides of the
paper. It also contains an adjustment for A4 paper.

Somewhat more flexible is \CTANref{vmargin}, which has a canned set of
paper sizes (a superset of that provided in \LaTeXe{}), provision for
custom paper, margin adjustments and provision for two-sided printing.

For details of \LaTeX{}'s page
parameters, see section~C.5.3 of the
\LaTeX{} manual (pp.~181--182).
The origin in \acro{DVI} coordinates is one inch
from the top of the paper and one inch from the left side; positive
horizontal measurements extend
right across the page, and positive vertical measurements extend down
the page. Thus, for margins closer to the left and top edges of the
page than 1 inch, the corresponding parameters, \emph{i.e}.,
\cs|evensidemargin|, \cs|oddsidemargin|, \cs|topmargin|, can be set to
negative values.

You cannot simply change the margins of part of a document within the
document by
modifying the parameters shown in Lamport's figure~C.3. They
should only be changed in the preamble of the document, \emph{i.e}., before
the \cs|begin{document}| statement. To adjust the margins within a
document we define an environment:
\begin{verbatim}
\newenvironment{changemargin}[2]{%
 \begin{list}{}{%
  \setlength{\topsep}{0pt}%
  \setlength{\leftmargin}{#1}%
  \setlength{\rightmargin}{#2}%
  \setlength{\listparindent}{\parindent}%
  \setlength{\itemindent}{\parindent}%
  \setlength{\parsep}{\parskip}%
 }%
\item[]}{\end{list}}
\end{verbatim}
This environment takes two arguments, and will indent the left
and right margins by their values, respectively. Negative values
will cause the margins to be narrowed, so
|\begin{changemargin}{-1cm}{-1cm}| narrows the left and right margins
by 1cm.
% (Note that the value given for |\parsep| is the default for
% \LaTeX{}; what should be there is ``the value in the surrounding
% text''.)

\Question{Finding the width of a letter, word, or phrase}

Put the word in a box, and measure the width of the box. For example,
\begin{verbatim}
  \newdimen\stringwidth
  \setbox0=\hbox{hi}
  \stringwidth=\wd0
\end{verbatim}
Note that if the quantity in the \cs|hbox| is a phrase, the actual
measurement only approximates the width that the phrase will occupy in
running text, since the inter-word glue can be adjusted in paragraph
mode.

The same sort of thing is expressed in \LaTeX{} by:
\begin{verbatim}
  \newlength{\gnat}
  \settowidth{\gnat}{\textbf{small}}
\end{verbatim}
This sets the value of the length command |\gnat| to the width of ``small''
in bold-face text.

\Question[Q-comment]{Excluding blocks of text from the \acro{DVI} file}

Rainer Sch\"opf's \File|verbatim.sty| provides a comment environment which
excludes everything between |\begin{comment}| and |\end{comment}|. 
This package is available as part of \CTANref{2etools}

A more general environment for doing the job is Victor Eijkhout's
\File|comment.sty|, which lets you define environments for inclusion
or exclusion in a document, thus offering a primitive configuration
structure.  It is available from the \acro{CTAN} sites in \CTANref{comment}

%\htmlignore
%% we don't use \Question here, since that has a different
%% interpretation of the optional argument;  AAAAUUUGGGGHHH
%\subsection[Defining a new log-like function in \LaTeX{}]{Defining a new
%  log-like function in \LaTeX{}\footnotemark}
%\footnotetext{It should be noted that ``log-like'' was reportedly a
%  \emph{joke} on Lamport's part; it is of course clear what was meant}
%\endhtmlignore
%\begin{htmlversion}
  \Question{Defining a new log-like function in \LaTeX{}}
%\end{htmlversion}

Use the \cs|mathop| command, as in:
\begin{verbatim}
  \newcommand{\diag}{\mathop{\rm diag}}
\end{verbatim}

Subscripts and superscripts on \cs|diag| will be placed exactly as they
are on \cs|lim|.  If you want your subscripts and superscripts always
placed to the right, do:
\begin{verbatim}
\newcommand{\diag}{\mathop{\rm diag}\nolimits}
\end{verbatim}

This works in \LaTeXo{} and in \LaTeXe{}, but not under \acro{NFSS} alone
(see \Qref[question]{problems with \cs|rm|, etc.}{rmnonsense}).
However, the canonical method for doing this in \LaTeXe{} is to use
the the \cs|DeclareMathOperator| command of |amsopn.sty| (which is
part of the \AMSLaTeX{} package: \CTANref{amslatex}).

(It should be noted that ``log-like'' was reportedly a \emph{joke} on
Lamport's part; it is of course clear what was meant.)

\Question{Typesetting all those \TeX{}-related logos}

Knuth was making a particular point about the capabilities of \TeX{}
when he defined the logo.  Unfortunately, many
believe, he thereby opened floodgates to
give the world logos such as \AMSTeX{}, \PiCTeX{}, \BibTeX{}, and so on.
Lamport invented \LaTeX{}, and marketing input led to
the current logo \LaTeXe{}.

The common people don't have to follow this stuff wherever it goes,
but, for those who insist, a large collection of logos is defined in
\CTANref{texnames}; the \MF{} logo can be set in fonts that \LaTeXe{}
knows about (so that it scales with the surrounding text) using the
package \CTANref{mflogo}

For those who don't wish to acquire the `proper' logos, the canonical
thing to do is to say |AMS-\TeX{}| (\acro{AMS}-\TeX{}) for \AMSTeX{},
|Pic\TeX{}| (Pic\TeX{}) for \PiCTeX{}, |Bib\TeX{}| (Bib\TeX{}) for
\BibTeX{}, and so on.

%%%%%%%%%%%%%%%%%%%%%%%%%%%%%%%%%%%%%%%%%%%%%%%%%%%%%%%%%%%%%%%%%

\section{Things are Going Wrong\dots{}}

\Question{Weird hyphenation of words}

You may have a version mismatch problem. \TeX{}'s hyphenation system changed
between version~2.9 and~3.0.  If you are using (|plain|) \TeX{}
version~3.0 or later, make sure your \File|plain.tex| file has a
version number which is at least~3.0.  If you are using \LaTeXo{} you
should consider upgrading to \LaTeXe{}; if for some reason you can't, the
last version of \LaTeXo{}, released on 25 March 1992, is available
(for the time being at least) from \CTANref{latex209-base} and ought
to solve this problem.

If you're using \LaTeXe{}, the problem probably arises from your
|hyphen.cfg| file, which has to be created if you're using a
multi-lingual version.

For the curious, here's what happened:
before \TeX{} 3.0 the hyphenation algorithm would not
break a word if the part before the break was not at least two
characters long, and the part after the break at least three 
characters long. Starting with version 3.0 the parameters
\cs|lefthyphenmin| and \cs|righthyphenmin| control the length of these
fragments. These are set to 2 and 3, respectively, in the new
|plain| and |lplain| formats.  They can be set to any value, of course,
but if \cs|lefthyphenmin|+\cs|righthyphenmin| is greater than 62, all
hyphenation is suppressed.

A further source of oddity can derive from the 1995 release of the
\htmlignore
\acro{DC} fonts (\Qref{}{dc-fonts}),
\endhtmlignore
\begin{htmlversion}
  \Qref{DC fonts}{dc-fonts},
\end{htmlversion}
which introduced an alternative hyphen character.  The \LaTeXe{}
configuration files in the font release specified use of the
alternative hyphen, and this could produce odd effects with words
containing an explicit hyphen.  The font configuration files in the
December 1995 release of \LaTeXe{} do \emph{not} use the alternative
hyphen character, thus removing this source of problems.

\Question{(Merely) peculiar hyphenation}

You may have found that \TeX{}'s famed automatic word-division does
not produce the break-points recommended by your dictionary. This may be
because \TeX{} is set up for American English, whose rules for word
division (as specified, for example, in Webster's Dictionary) are
completely different from the British ones (as specified, for example,
in the Oxford Dictionaries). This problem is being addressed by the \acro{UK}
\TeX{} User community (see \BV{}, issue~4.4) but an entirely
satisfactory solution will take time.  An interim hyphenation file is
available in \CTANref{ukhyph}

\Question[hyphenated-accents]{Accented words aren't hyphenated}

\TeX{}'s algorithm for hyphenation gives up when it encounters an
\cs|accent| command; there are good reasons for this, but it means
that quality typesetting in non-English languages can be difficult.

For \TeX{} itself, avoiding this effect means using Cork-encoded fonts
(\Qref{the DC fonts}{dc-fonts}) which contain accented letters as
single glyphs.  In the future, perhaps,
\htmlignore
Omega (\Qref{}{Q-omega})
\endhtmlignore
\begin{htmlversion}
  \Qref{Omega}{Q-omega}
\end{htmlversion}
will provide a rather different solution.

\Question{Enlarging \TeX{}}

People sometimes get messages saying `memory capacity exceeded'.
Most of the time this error can be fixed
\emph{without} enlarging \TeX{}. The most common causes are unmatched braces,
extra-long lines, and poorly-written macros. Extra-long lines are
often introduced when files are transferred incorrectly between
operating systems, and line-endings are not preserved properly (the
tell-tale sign of an extra-long line error is the complaint
that the `|buf_size|' has overflowed).

If you really need to extend your \TeX{}'s capacity, the proper method
depends on your installation. In the purest form, you
change the parameters in module 11 of the \acro{WEB} source.
% (``The following parameters can be changed\dots{}'').
In less pure forms, you might need to modify a
change file, or perhaps change some environment variables; em\TeX{}
allows you to adjust the memory allocation criteria on the command
line.  Consult the documentation that came with your implementation.

\Question{Moving tables and figures in \LaTeX{}}

Tables and figures have a tendency to surprise, by \emph{floating}
away from where they were specified to appear.  This is in fact
perfectly ordinary document design; any professional typesetting
package will float figures and tables to where they'll fit without
violating the certain typographic rules.  Even if you use the
placement specifier~|h| for `here', the figure or table will not be
printed `here' if doing so would break the rules; the rules themselves
are pretty simple, and are given on page~198, section~C.9 of the
\LaTeX{} manual.  In the worst case, \LaTeX{}'s rules can cause the
floating items to pile up to the extent that you get an error message
saying ``Too many unprocessed floats''; this means that the limited
set of registers in which \LaTeX{} stores floating items is full.
What follows is a simple checklist of things to do to solve these
problems (the checklist talks throughout about figures, but applies
equally well to tables).
\begin{itemize}
\item Are the placement parameters on your figures right?  The
  default (|tbp|) is reasonable; you should never simply say `|h|',
  for example, since that says ``if it can't go here, it can't go
  anywhere'', and as a result all subsequent floats pile up behind it.
\item Can you perhaps prevent your figures from floating by adjusting
  \LaTeX{}'s placement parameters?  Again, the defaults are
  reasonable, but can be overridden in case of problems.  The
  parameters are described on pages~199--200, section~C.9 of the
  \LaTeX{} manual.
\item Are there places in your document where you could `naturally'
  put a \cs|clearpage| command?  If so, do: the backlog of floats is
  cleared after a \cs|clearpage|.  (Note that the \cs|chapter|
  command implicitly executes \cs|clearpage|, so you can't float past
  the end of a chapter.)
\item Have a look at the \LaTeXe{} |afterpage| package (part of
  \CTANref{2etools}).  Its documentation gives as an example the idea
  of putting \cs|clearpage| \emph{after} the current page (where it
  will clear the backlog, but not cause an ugly gap in your text), but
  also admits that the package is somewhat fragile (though it's improving).
\item As a last resort, try the package \CTANref{morefloats}; this
  `simply' increases the number of floating inserts that \LaTeX{} can
  handle at one time (from~18 to~36), but that may suit your needs.
\item If you actually \emph{wanted} all your figures to float to the
  end (\emph{e.g}., for submitting a draft copy of a paper), don't
  rely on \LaTeX{}'s mechanism: get the package \CTANref{endfloat} to do
  the job for you.
\end{itemize}

%If you did not read the manual carefully, you may be surprised that
%your figures and tables in a \LaTeX{} document do not come out where
%you intended.  As you will find in any professional-quality
%typesetting package, figures and tables \emph{float} to a place where
%there is room for them without violating certain typographic rules.
%Even if you use the placement specifier~|h| for `here', the figure or
%table will not be printed `here' if there is not room for it or if
%printing it there would break the rules.

%On the whole, the rules are pretty sensible. It is a fact of life that
%putting each table or figure exactly `here' in the text is not
%consistent with fixed pre-specified page-size. It's good to get in the
%habit of writing `|...in Table~\ref{mytable}...|' rather than assuming
%that your table will come exactly where you put it relative to your
%words of wisdom.

%The rules, and the parameters on which they depend, are described in 
%Section~C.9.1 of the \LaTeX{} manual\checked{RAB}{1994/10/22}
%(see \Qref[question]{\TeX{}-related books}{Q-books}). Unless
%you have a document with very small pages or an unusually high density
%of floats, it is not worth fighting with the default parameter values.

%\Question{Disappearing tables and figures}

%If you have several tables one after the other (for example at the end
%of a document, as many journal editors request) you may find that
%\LaTeX{} runs out of memory without printing a single one of these
%tables. The problem is that \LaTeX{} is saving up the tables in its
%memory while it searches for some text to put on the page first. You
%can cure this by making sure that every table has~`|p|' in its
%placement specifier, and by issuing a \cs|clearpage| command every so
%often, at a `natural break'.

%If, on the other hand, you actually \emph{need} the figures and tables
%all in one place at the end, the \File|endfloat| package
%(\CTANref{endfloat}) offers useful facilities, including the means to
%leave indications for the editor where the figure really \emph{ought}
%to go.

\Question[Q-ps@empty]{\cs|pagestyle{empty}| on first page in \LaTeX{}}

If you use \cs|pagestyle{empty}|, but the first page is numbered anyway,
you are probably using the \cs|maketitle| command too. This is not a bug
but a feature! The standard \LaTeX{} styles are written so that
initial pages (pages containing a \cs|maketitle|, \cs|part|, or
\cs|chapter|) have a different page style from the rest of the document; 
Hence, the commands internally issue \cs|thispagestyle{plain}|. This is
usually not acceptable behaviour if the surrounding page style is
`empty'.

Possible workarounds include: 
\begin{itemize}
\item Put \cs|thispagestyle{empty}| immediately after the \cs|maketitle|
  command, with no blank line between them.
\item Use \File|fancyheadings.sty|, which allows you to customise the
  style for initial pages independently of that for body pages.  It is
  available in \CTANref{fancyheadings}
\item Use \File|nopageno.sty|, which suppresses this behaviour.  It is
  available in \CTANref{nopageno}
\end{itemize}

\Question[rmnonsense]{Odd behaviour of \cs|rm|, \cs|bf|, \emph{etc}.}

If commands such as \cs|rm| and \cs|bf| have suddenly stopped working in 
\LaTeX{} in the way that you expect, it is likely that your system
administrator has installed a version of \LaTeX{}~2.09 with
\htmlignore
\acro{NFSS} (\Qref{}{NFSS}).
\endhtmlignore
\begin{htmlversion}
\Qref{\acro{NFSS}}{NFSS}.
\end{htmlversion}
Complain loudly; ask 
your system administrator to replace this version with
\htmlignore
\LaTeXe{} (\Qref{}{latex2e}),
\endhtmlignore
\begin{htmlversion}
\Qref{\LaTeXe{}}{latex2e},
\end{htmlversion}
in which commands such as \cs|rm| and \cs|bf| work just as before if you
are using one of the standard classes---|article|, |report| and |book|
(among others).  In the meantime, use the option
\File|oldlfont.sty|, which should have been installed at the same time
as \acro{NFSS}.

\Question{Old \LaTeX{} font references such as \cs|tenrm|}

\LaTeX{}~2.09 defined a large set of commands for access to the fonts
that it had built in to itself.  For example, various flavours of
|cmr| could be found as \cs|fivrm|, \cs|sixrm|, \cs|sevrm|,
\cs|egtrm|, \cs|ninrm|, \cs|tenrm|, \cs|elvrm|, \cs|twlrm|,
\cs|frtnrm|, \cs|svtnrm|, \cs|twtyrm| and \cs|twfvrm|.
These commands were never documented, but certain packages
nevertheless used them to achieve effects they needed.

Since the commands weren't public, they weren't included in \LaTeXe{};
to use the unconverted \LaTeX{}~2.09 packages under \LaTeXe{}, you need
also to include the package \File|rawfonts.sty| (which is part of the
\LaTeXe{} distribution).

\Question{Missing symbols}

If some symbols, such as \cs|Box| and \cs|lhd|, no longer appear to 
exist, then your system administrator has probably upgraded your
version of \LaTeX{} to either
\htmlignore
\acro{NFSS} (\Qref{on the scheme}{NFSS}) or \LaTeXe{}
(\Qref{describing LaTeX2e}{latex2e}).
\endhtmlignore
\begin{htmlversion}
\Qref{\acro{NFSS}}{NFSS} or \Qref{\LaTeXe{}}{latex2e}.
\end{htmlversion}
In the former case, use
\htmlignore
\File|oldlfont.sty|, \Qref[as in the question]{}{rmnonsense}.
\endhtmlignore
\begin{htmlversion}
\File|oldlfont.sty| (see \Qref{problems with \cs|rm|, etc.}{rmnonsense}).
\end{htmlversion}
In the latter, use the package \File|latexsym|, which is part of the
\LaTeXe{} distribution, or the package \File|amsfonts|, if
it is available.

\Question{\LaTeX{} gets cross-references wrong}

Sometimes, however many times you run \LaTeX{}, the cross-references
are just wrong. Remember that
the |\label| command must come \emph{after} the |\caption| command, or
be part of it. For example,
\htmlignore
\par\noindent
\begin{tabular}{@{}lll}
|\begin{figure}|     &    & |\begin{figure}|\\
|\caption{A Figure}| & or & |\caption{A Figure%| \\
|\label{fig}|        &    & |      \label{fig}}| \\
|\end{figure}|       &    & |\end{figure}| \\
\end{tabular}
\endhtmlignore
\begin{htmlversion}
\begin{verbatim}
\begin{figure}          \begin{figure}
\caption{A Figure}  or  \caption{A Figure%
\label{fig}                   \label{fig}}
\end{figure}            \end{figure}
\end{verbatim}
\end{htmlversion}

\Question[atsigns]{\cs|@| and \texttt{@} in macro names}

A common source of problems in a \LaTeX{} document is the diagnostic
about the appearance of the command \cs|@|, or about other commands
containing the character |@|.  The most common complaint is ``You
can't use `\cs|spacefactor|' in vertical mode'', but others occur.

Such problems are usually caused by including a \LaTeXe{} class or
package file into a \LaTeX{} document by some means other than
\cs|documentclass| or \cs|usepackage|.  \LaTeX{} defines internal
commands whose names contain the character |@|; this enables it to
avoid clashes between its internal names and names that we would
normally use in our documents.  In order that these commands may work
at all, \cs|documentclass| and \cs|usepackage| play around with the
meaning of |@|.  Solve this problem by using the correct command to
include the file.

But, you will say, ``\emph{The \LaTeX{} Companion} tells me to use
commands containing \texttt{@}!''

Indeed; for example, there's a lengthy section about
\cs|@startsection| and how to use it to control the appearance of
section titles.  Page~15 of \emph{The Companion} explains this; and
suggests that you make such changes in the document preamble, between
\cs|makeatletter| and \cs|makeatother|.  So the definition of
\cs|subsection| on page~26 could be:

\begin{verbatim}
  \makeatletter
  \renewcommand{\subsection}{\@startsection
    {subsection}%                    % name
    ...
    {\normalfont\normalsize\itshape}}% style
  \makeatother
\end{verbatim}

\Question{Where are the \texttt{msx} and \texttt{msy} fonts?}

The |msx| and |msy| fonts were designed by the American Mathematical
Society in the very early days of \TeX{}, for use in typesetting
papers for mathematical journals.  They were designed using the `old'
\MF{}, which wasn't portable and is no longer available; for a long
time they were only available in 300dpi versions which only
imperfectly matched modern printers.  The \acro{AMS} has now redesigned the
fonts, using the current version of \MF{}, and the new versions are
called the |msa| and |msb| families; they are available from
\CTANref{amsfonts-symbols}

Nevertheless, |msx| and |msy| continue to turn up to plague us.  There
are, of course, still sites that haven't got around to upgrading; but,
even if everyone upgraded, there would still be the problem of old
documents that specify them.

If you have a |.tex| source that requests |msx| and |msy|, the best
technique is to edit it so that it requests |msa| and |msb| (you only
need to change the single letter in the font names).

If you have a \acro{DVI} file that requests the fonts, there is a package
of
\htmlignore
virtual fonts (\Qref{}{virtualfonts})
\endhtmlignore
\begin{htmlversion}
\Qref{virtual fonts}{virtualfonts}
\end{htmlversion}
to map the old to the new series; it's available in \CTANref{msx2msa}

\Question[am-fonts]{Where are the \texttt{am} fonts?}

One \emph{still} occasionally comes across a request for the |am|
series of fonts.  The initials stood for `Almost [Computer] Modern',
and they were the predecessors of the Computer Modern fonts that we
all know and love (or hate)\begin{footnoteenv}
The fonts acquired their label `Almost' following the realisation
that their first implementation in \MF{}79 still wasn't quite right;
Knuth's original intention had been that they were the final answer
\end{footnoteenv}.
There's not a lot one can do with these
fonts; they are (as their name implies) almost (but not quite) the
same as the |cm| series; if you're faced with a document that requests
them, all you can reasonably do is to edit the document.  The
appearance of \acro{DVI} files that request them is sufficiently rare that
no-one has undertaken the mammoth task of creating a translation of
them by means of virtual fonts; however, most drivers let you have a
configuration file in which you can specify font substitutions. If you
specify that every |am| font should be replaced by its corresponding
|cm| font, the output should be almost correct.

\Question{`String too long' in \BibTeX{}}

The \BibTeX{}  diagnostic ``Warning--you've exceeded 1000, the
|global-string-size|, for entry |foo|'' is not one that you can hope
to avoid by altering the \BibTeX{} style in a simple way~--- \BibTeX{}
itself needs recompiling to increase its limit on string sizes (which is
often not practical, and is never desirable).  You must therefore 
address the problem by changing your bibliography database.

The problem usually arises from a very large abstract or annotation
included in the database.  The only way forward is to take the entry
out of the database, so that you don't encounter \BibTeX{}'s limit,
but you may need to retain the entry because it will be included in
the typeset document.  In such cases, put the body of the entry in a separate
file:
\begin{verbatim}
  @article{long.boring,
    author =    "Fred Verbose",
    ...
    abstract =  "{\input{abstracts/long.tex}}"
  }
\end{verbatim}
In this way, you arrange that all \BibTeX{} has to deal with is the
file name, though it will tell \TeX{} (when appropriate) to include
all the long text.

%%%%%%%%%%%%%%%%%%%%%%%%%%%%%%%%%%%%%%%%%%%%%%%%%%%%%%%%%%%%%%%%%

\section{Why does it \emph{do} that?}

\Question{Why does it ignore paragraph parameters?}

When \TeX{} is laying out text, it doesn't work from word to word, or
from line to line; the smallest complete unit it formats is the
paragraph.  The paragraph is laid down in a buffer, as it appears, and
isn't touched further until the end-paragraph marker is processed.
It's at this point that the paragraph parameters have effect; and it's
because of this sequence that one often makes mistakes that lead to
the paragraph parameters not doing what one would have hoped (or
expected).

Consider the following sequence of \LaTeX{}:
\begin{verbatim}
  {\raggedright % declaration for ragged text
  Here's text to be ranged left in our output,
  but it's the only such paragraph, so we now
  end the group.}

  Here's more that needn't be ragged...
\end{verbatim}
\TeX{} will open a group, and set the ragged-setting parameters within
that group; it will then save a couple of sentences of text and
close the group (thus restoring the previous value of the
ragged-setting parameters).  Then it encounters a blank line, which it
knows to treat as a \cs|par| token, so it typesets the two sentences;
but because the enclosing group has now been closed, the parameter
settings have been lost, and the paragraph will be typeset normally.

The solution is simple: close the paragraph inside the group, so that
the setting parameters remain in place.  An appropriate way of doing
that is to replace the last three lines above with:
\begin{verbatim}
  end the group.\par}
  Here's more that needn't be ragged...
\end{verbatim}
In this way, the paragraph is completed while the setting parameters
are still in force within the enclosing group.

Another alternative is to define an environment that does the
appropriate job for you.  For the above example, \LaTeX{} already
defines an appropriate one:
\begin{verbatim}
\begin{flushleft}
  Here's text to be ranged left...
\end{flushleft}
\end{verbatim}

\Question[Q-protect]{What's the reason for `protection'?}

Sometimes \LaTeX{} saves data it will reread later. These data are
often the argument of some command; they are the so-called moving
arguments.  (`Moving' because data are moved around.)  Places to look for
are all arguments that may go into table of contents, list of figures,
\emph{etc}.; namely, data that are written to an auxiliary file and
read in later.  Other places are those data that might appear in head-
or footlines.  Section headers and figure captions are the most
prominent examples; there's a complete list in Lamport's book
(see \Qref[question]{\TeX{}-related books}{Q-books}).

%You don't want to care about this stuff?  Simply |\protect| all
%\LaTeX{} commands within these moving arguments.

What's going on really, behind the scenes? The commands in the moving
arguments are already expanded to their internal structure during the
process of saving. Sometimes this expansion results in invalid \TeX{}
code when processed again. ``\cs|protect|\cs|cmd|'' tells \LaTeX{} to save
\cs|cmd| as \cs|cmd|, without expansion.

What is a `fragile command'?  It's a command that expands into illegal
\TeX{} code during the save process.

What is a `robust command'?  It's a command that expands into legal
\TeX{} code during the save process.

No-one (of course) likes this situation; the \LaTeX{}3 team have
removed the need for protection of some things in the production of
\LaTeXe{}, but the techniques available to them within current
\LaTeX{} mean that this is an expensive exercise.  It remains a
long-term aim of the team to remove all need for these things.

\Question{Why doesn't \cs|verb| work within\dots{}?}

The \LaTeX{} verbatim commands work by changing category codes.  Knuth
says of this sort of thing ``Some care is needed to get the timing
right\dots{}'', since once the category code has been assigned to a
character, it doesn't change.  So \cs|verb| has to assume that it is
getting the first look at its parameter text; if it isn't, \TeX{} has
already assigned category codes so that \cs|verb| doesn't have a
chance.  For example:
\begin{verbatim}
    \verb+\error+
\end{verbatim}
will work (typesetting `\cs|error|'), but
\begin{verbatim}
    \newcommand{\unbrace}[1]{#1}
    \unbrace{\verb+\error+}
\end{verbatim}
will not (it will attempt to execute \cs|error|).

This is why the \LaTeX{} book insists that verbatim
commands must not appear in the argument of any other command; they
aren't just fragile, they're quite unusable in any command parameter,
regardless of
\htmlignore
\cs|protect|ion (\Qref{}{Q-protect}).
\endhtmlignore
\begin{htmlversion}
\Qref{\cs|protect|ion}{Q-protect}.
\end{htmlversion}

\Question{Case-changing oddities}

\TeX{} provides two primitive commands \cs|uppercase| and
\cs|lowercase| to change the case of text; they're not much used, but
are capable creating confusion.

The two commands do not expand the text that is their parameter~---
the result of \cs|uppercase{abc}| is `|ABC|', but |\uppercase{\abc}|
is always `|\abc|', whatever the meaning of \cs|abc|.  The commands
are simply interpreting a table of equivalences between upper- and
lowercase characters.
They have (for example) no mathematical sense, and
\begin{verbatim}
  \uppercase{About $y=f(x)$}
\end{verbatim}
will produce
\begin{verbatim}
  ABOUT $Y=F(X)$
\end{verbatim}
which is probably not what is wanted.

In addition, \cs|uppercase| and \cs|lowercase| do not deal very well
with non-American characters, for example |\uppercase{\ae}| is the
same as \cs|ae|.

\LaTeX{} provides commands \cs|MakeUppercase| and \cs|MakeLowercase|
which fixes the latter problem.  These commands are used in the
standard classes to produce upper case running heads for chapters
and sections.

Unfortunately \cs|MakeUppercase| and \cs|MakeLowercase| do not solve
the other problems with \cs|uppercase|, so for example a section
title containing \cs|begin{tabular}| \dots{} \cs|end{tabular}| will
produce a running head containing \cs|begin{TABULAR}|.  The simplest
solution to this problem is using a user-defined command, for
example: 
\begin{verbatim}
   \newcommand{\mytable}{\begin{tabular}...
      \end{tabular}}
   \section{A section title \protect\mytable{}
       with a table}
\end{verbatim}
Note that \cs|mytable| has to be protected, otherwise it will be
expanded and made upper case.

\Question{Why are \texttt{\#} signs doubled in macros?}

The way to think of this is that |##| gets replaced by |#| in just the
same way that |#1| gets replaced by `whatever is the first argument'.

So if you define a macro and use it as:
\begin{verbatim}
  \def\a#1{...#1...#1...#1...}  \a{b}
\end{verbatim}
the macro expansion produces `\dots{}b\dots{}b\dots{}b\dots{}',
which people find normal.  However, if we now fill in the `\dots{}':
\begin{verbatim}
  \def\a#1{---#1---\def\x #1{xxx#1}}
\end{verbatim}
\cs|a{b}| will expand to `-{}-{}-b-{}-{}-|\def\x b{xxxb}|'.  This
defines \cs|x| to be a macro \emph{delimited} by |b|, and taking no
arguments, which people may find strange, even though it is just a
specialisation of the example above.  If you want \cs|a| to
define \cs|x| to be a macro with one argument, you need to write:
\begin{verbatim}
  \def\a#1{---#1---\def\x ##1{xxx##1}}
\end{verbatim}
and \cs|a{b}| will expand to 
`-{}-{}-b-{}-{}-|\def\x #1{xxx#1}|', because |#1| gets replaced by `b'
and |##| gets replaced by |#|.

To nest a definition inside a definition inside a definition then
you need |####1|, as at each stage |##| is replaced by
|#|.  At the next level you need 8~|#|s each time, and so on.

%%%%%%%%%%%%%%%%%%%%%%%%%%%%%%%%%%%%%%%%%%%%%%%%%%%%%%%%%%%%%%%%%

\section{Recent Developments}

\Question[NFSS]{The New Font Selection Scheme (\acro{NFSS})}

\acro{NFSS} was an extension to \LaTeX{} written by Frank Mittelbach and
Rainer Sch\"opf. It is described in \TUGboat{} 10(2).  In
traditional typesetting, fonts are described by four parameters: the
\emph{family} (\emph{e.g}., computer modern), the \emph{series}
(\emph{i.e}., the weight and width of the font, such as light or bold),
the \emph{shape} (\emph{e.g}., italic), and the \emph{size}. \acro{NFSS} is a
mechanism allowing the user to change any of these independently. \acro{NFSS}
makes it relatively easy to use nonstandard fonts such as the
PostScript ones with \LaTeX{}, and easy to change maths fonts.  It
also allows dynamic loading of fonts at runtime (\emph{i.e}., not when
the format file is created).

%\acro{NFSS} is no longer supported for \LaTeX{} version 2.09, but is standard
%with \LaTeXe{} (see question 41).
%
%There is one caveat that applies to \LaTeX{} documents written for
%the OLD scheme: some of them use special styles for special fonts
%which will not work under the \acro{NFSS}.

With the demise of \LaTeX{}~2.09 as supported software, the label
`\acro{NFSS}' has become somewhat misleading, as there's no `old' scheme with
which to contrast it~--- \LaTeX{} has incorporated the \acro{NFSS}.

\htmlignore
\begingroup\boldmath
\endhtmlignore
\Question[latex2e]{\LaTeXe{} (the new standard \LaTeX{})}
\htmlignore
\endgroup\par
\endhtmlignore

\LaTeXe{} is a new version of the \LaTeX{} package, prepared and
supported by the \LaTeX{}3 project team. It moved out of its test
phase in June 1994, and is now the standard \LaTeX{}; \LaTeX{}~2.09 is
no longer supported.  New versions are released at (approximately)
6-monthly intervals; this does \emph{not} mean the functionality is
unstable, merely that the implementation is steadily being refined.

\LaTeXe{} is upwardly compatible with \LaTeX{}~2.09, but has new
features.  In the latest (December 1995) release, these include:
\begin{itemize}
\htmlignore
\item \acro{NFSS} (\Qref{}{NFSS})
\endhtmlignore
\begin{htmlversion}
\item \Qref{\acro{NFSS}}{NFSS}
\end{htmlversion}
  is part of the distribution.
\item \SliTeX{} is now merely a different document class, so that
  there is no longer a need for a separate format.
\item Better control of floating environments, such as figures.
\item There is a documented interface for package and class writers
  (though not yet for designers).
\item The box commands have been enhanced, with \emph{e.g.}, options to
  specify the height of a minipage.
\item Several standard commands are no longer
\htmlignore
  fragile (\Qref{}{Q-protect});
\endhtmlignore
\begin{htmlversion}
  \Qref{fragile}{Q-protect}
\end{htmlversion}
  they can therefore be included in the argument of commands such as
  \cs|caption| without being protected.
\item |\newcommand| can define commands with one optional argument;
  such commands are automatically robust.
\item There is now a standard package for colour and graphics
  inclusion.
\end{itemize}

% Since \LaTeXe{} is supported, you can report bugs or problems with it by
% typing `|latex latexbug|' and sending the report thus generated to 
% \Email|latex-bugs@uni-mainz.de|.

\Question[LaTeX3]{The \LaTeX{}3 project}

The \LaTeX{}3 project team is a small group of volunteers whose aim is
to produce a major new document processing system based on the
principles pioneered by Leslie Lamport in the current \LaTeX{}.  It
will remain freely available and it will be fully documented at
all levels.

The \LaTeX{}3 team has already delivered its first product,
\htmlignore
\LaTeXe{} (\Qref{}{latex2e}),
\endhtmlignore
\begin{htmlversion}
\Qref{\LaTeXe{}}{latex2e},
\end{htmlversion}
a macro package based
on Lamport's original code, but modified to be
more readily supportable than was Lamport's.

\Question[Q-omega]{The Omega project}

Omega
\htmlignore
($\Omega$)
\endhtmlignore
is a program built on top of \TeX{} which works internally with 16-bit
characters (Unicode); this allows it to work with most scripts in the
world without any complications of coding schemes.  Omega also has a
powerful concept of input and output filters to allow the user to work
with existing transliteration schemes, \emph{etc}.  Omega is an
ongoing project by John Plaice (\Email|plaice@ift.ulaval.ca|) and
Yannis Haralambous (\Emaildot|Yannis.Haralambous@univ-lille1.fr|).  An
email discussion list is available: subscribe by sending a message
`|subscribe omega <your name>|' to \Email|listserv@ens.fr|

\Question[Q-NTS]{The \NTS{} project}

The \NTS{} project first saw the light of day at the Hamburg meeting
of \acro{DANTE} during 1992, as a response to an aspiration to
produce something even better than \TeX{}.  The project is not simply
enhancing \TeX{}, for two reasons: first, that \TeX{} itself has been
frozen by Knuth
(see \Qref[question]{the future of \TeX{}}{tex-future}), and second,
even if they \emph{were} allowed to develop the program, some members
of the \NTS{} team feel that \TeX{} in its present form is simply
unsuited to further development.  While all those involved in the
project are involved with, and committed to, \TeX{}, they recognise
that the end product may very well have little in common with \TeX{}
other than its philosophy.

% However, complete compatibility is a very important criterion in
% the minds of the team, and any decision to violate this will only be
% taken after very careful introspection and after open discussions
% with existing \TeX{} users world-wide.

% The project is proceeding in several phases.
%The first is to re-write
%\TeX{} in a language of the project's choice, and using more modern
%programming techniques than were available to Knuth.  Once this
%version is complete (in the sense that it satisfies the stringent
%requirements that people have of Knuth's implementation), it will be
%made available to the world at large, and will become the basis of
%experiments in the design of advanced typesetting facilities; this new
%version will be known as \eTeX{}.
Initially, and despite the reservations expressed at the inaugural meeting,
the group is concentrating on extending \TeX{} \emph{per se}: members are
implementing extensions and enhancements to \TeX{} through the standard
medium of a change-file.  These extensions and enhancements, together
with \TeX{} proper, will form a system called \eTeX{}, which will be 100\%
compatible with \TeX{}; furthermore, it will be possible during format
creation to construct a format that \emph{is} \TeX{}: no extensions or
enhancements will be present.
% If, on the other hand, it is desired to
% create an extended format, then all existing user documents will still
% be processable using that format to produce a |dvi| file that is
% \emph{identical} to the |dvi| file which would be produced by \TeX{}; only
% if 
% a further decision is taken, to use \eTeX{} in \emph{enhanced} mode,
% will the semantics of existing \TeX{} documents be compromised.  Thus
% at the user's discretion \eTeX{} can be used (a)~as pure \TeX{}, with no
% differences whatsoever, or (b)~in extended mode, wherein exist new
% features and new facilities but which will still process existing \TeX{}
% documents in such a way as to produce results identical to \TeX{}, or (c)~in
% enhanced mode, in which case some fundamental changes are made to the
% semantics of existing \TeX{} documents such that complete compatibility
% can no longer be assured (at the time of writing, only \TeXXeT{} requires
% this enhanced-mode behaviour, but \MLTeX{}, which the group have been
% given permission to incorporate, may also require such modified semantics).

The final aim of the project will be to produce an entirely new
typesetting system, building on the experience gained in the earlier
phases.  This system is intended to provide a stable basis
for typesetting in the future, in the way that \TeX{} has since it was
first offered to the world.

%The second phase of the project will be to produce an entirely new
%typesetting system, building on the experience gained in the first
%phase.  This system, to be known as \NTS{}, is to provide a stable basis
%for typesetting in the future, in the way that \TeX{} has since it was
%first offered to the world.

%%%%%%%%%%%%%%%%%%%%%%%%%%%%%%%%%%%%%%%%%%%%%%%%%%%%%%%%%%%%%%%%%

%
% This is the last section, and is to remain the last section...
\section{Perhaps There \emph{isn't} an Answer}

% ... and it contains only one question:
\htmlignore
\par
\endhtmlignore
\Question{What to do if you find a bug}
\htmlignore
\par
\endhtmlignore
%
% This here isn't a reference to a question...
\label{lastquestion}

For a start, make entirely sure you \emph{have} found a bug.
Double-check with books about \TeX{}, \LaTeX{}, or whatever you're using;
compare what you're seeing against the other answers above; ask every
possible person you know who has any \TeX{}-related expertise.
The reasons for all this caution are various.

If you've found a bug in \TeX{} itself, you're a rare animal indeed.
Don Knuth is so sure of the quality of his code that he offers real
money prizes to finders of bugs; the cheques he writes are
such rare items that they are seldom cashed. If \emph{you}
think you have found a genuine fault in \TeX{} itself (or \MF{}, or the
\acro{CM} fonts, or the \TeX{}book), don't immediately write to Knuth,
however. He only looks at bugs once or twice a year, and even then
only after they are agreed as bugs by a small vetting team. In the
first instance, contact Barbara Beeton at the \acro{AMS}
(\Email|bnb@math.ams.org|), or contact \acro{TUG}.

If you've found a bug in \LaTeXe{}, look in the bugs database to see
if it's already been reported.  If not you should submit details of
the bug to the \LaTeX{}3 team.  To do this, you should
process the file \File|latexbug.tex| with \LaTeX{} (the file is part
of the \LaTeXe{} distribution.  The process will give you instructions
about what to do
with your bug report (it can, for example, be sent to the team by
email).  Please be sparing of the team's time; they're
doing work for the good of the whole \LaTeX{} community, and any time
they spend tracking down non-bugs is time not available to write or
debug new code.  Details of the whole process, and an interface to the
database, are available via
\URL|http://www.tex.ac.uk/ctan/latex/bugs.html|

If you've found a bug in \LaTeX{}2.09, or some other such unsupported
software, there's not a lot you can do about it.  You may find help or
\emph{de facto} support on a newsgroup such as
\Newsgroup|comp.tex.tex| or on a mailing list such as
\Email|texhax@tex.ac.uk|, but posting non-bugs to any of these forums
can lay you open to ridicule!  Otherwise you need to go out and find
yourself a willing
\htmlignore
\TeX{}-consultant\begin{footnoteenv}
\acro{TUG} maintains a register of \TeX{} consultants;
\acro{UK}~\acro{TUG} is developing one 
\end{footnoteenv}.
\endhtmlignore
\begin{htmlversion}
\TeX{}-consultant.
\end{htmlversion}
