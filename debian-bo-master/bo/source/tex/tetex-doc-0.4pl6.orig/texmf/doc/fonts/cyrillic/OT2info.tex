\documentclass[a4paper]{article}
\usepackage[OT2,OT1]{fontenc}

\def\cyr{\fontencoding{OT2}\fontfamily{wncyr}\selectfont}

\begin{document}

\title{The OT2 encoding}
\author{Mattias Ellert\\{\small t91mel@student.tdb.uu.se}}
\date{March 8 1996}
\maketitle

The OT2 encoding is used for writing Cyrillic text in \LaTeX.
The table on the next page shows how to use the encoding.
The first column shows the Cyrillic letters, the second column shows
how to write the letters in the OT2 encoding and the third column
tells you in which languages the letters are used (Rus. = Russian,
Ukr. = Ukrainian, Blr. = Belorussian, Bul. = Bulgarian. Mak. =
Makedonian and of course Serb = Serb).

Cyrillic letters with one-letter Roman equivalents can be entered `as
is': {\cyr a, b, v, g}, etc. This is also true for some letter
sequences that are treated as ligatures by the \LaTeX\ Cyrillic
fonts ({\cyr dj, zh, lj, nj, kh, ts, ch, sh, shch, yu} and {\cyr
ya}). Note that if you want to write the letter combination {\cyr
t{}s} you have to write \texttt{t\{\}s} in order not to produce a
{\cyr ts}.

For the upper case versions of these letters the case is chosen by
the first letter, i.e.\ \texttt{Ts} and \texttt{TS} are equivalent.
This is also true for the letters that are written with \LaTeX\
commands, \texttt{$\backslash$Dzh} and \texttt{$\backslash$DZH} both
produce {\cyr\Dzh}. (In the commands however, you can not mix upper and
lower case letters in the remainder of the command, e.g.\
\texttt{$\backslash$DZh} does not work.) For letters whose commands
contain the full name of the letter, and not just a transliteration,
the all-upper-case version is not defined.

Four different accents are declared in the OT2 encoding,
\texttt{$\backslash$'\{\}} is used to produce an acute accent ({\cyr\'{}}),
\texttt{$\backslash$"\{\}} is used for a dierisis ({\cyr\"{}}), and
two different commands, \texttt{$\backslash$u\{\}} and
\texttt{$\backslash$U\{\}} produce the Roman and Cyrillic style breves.
We recommend using the latter in Cyrillic text. {\cyr\"{E}} and {\cyr\U{i}}
(with the Cyrillic breve) have composites declared in the \LaTeX\
Cyrillic fonts. There is also a dotless {\cyr\i\ (\dotlessi)} in
the fonts, which can be written with the command
\texttt{$\backslash$dotlessi}. There is usually no need for this however,
since the encoding automatically replaces any {\cyr\i}'s with
{\cyr\dotlessi}'s before putting an accent on them.

In educational text it is customary to put an accent on the stressed
vowel in multi-syllable words. There is however a problem with putting
accents on {\cyr yu} and {\cyr ya} since these are treated as
ligatures. For this reason the OT2 encoding defines the commands
\texttt{$\backslash$yu, $\backslash$ya, $\backslash$Yu} and
\texttt{$\backslash$Ya} which can be used in combinations like
\texttt{$\backslash$'$\{\backslash$yu\}} to produce {\cyr\'{\yu},
\'{\ya}, \'{\Yu}} and {\cyr\'{\Ya}}.

\newpage
\thispagestyle{empty}

\begin{tabular}{|cc|cc|l|}
	\hline
	{\cyr A} & {\cyr a} & \texttt{A} & \texttt{a} &
		Rus., Ukr., Blr., Bul., Mak., Serb \\
	\hline
	{\cyr B} & {\cyr b} & \texttt{B} & \texttt{b} &
		Rus., Ukr., Blr., Bul., Mak., Serb \\
	\hline
	{\cyr V} & {\cyr v} & \texttt{V} & \texttt{v} &
		Rus., Ukr., Blr., Bul., Mak., Serb \\
	\hline
	{\cyr G} & {\cyr g} & \texttt{G} & \texttt{g} &
		Rus., Ukr., Blr., Bul., Mak., Serb \\
	\hline
	{\cyr \'{G}} & {\cyr \'{g}} & \texttt{$\backslash$'\{G\}} &
		\texttt{$\backslash$'\{g\}} &
		Mak. \\
	\hline
	{\cyr D} & {\cyr d} & \texttt{D} & \texttt{d} &
		Rus., Ukr., Blr., Bul., Mak., Serb \\
	\hline
	{\cyr Dj} & {\cyr dj} & \texttt{Dj} & \texttt{dj} &
		Serb \\
	\hline
	{\cyr E} & {\cyr e} & \texttt{E} & \texttt{e} &
		Rus., Ukr., Blr., Bul., Mak., Serb \\
	\hline
	{\cyr \"{E}} & {\cyr \"{e}} & \texttt{$\backslash$"\{E\}} &
		\texttt{$\backslash$"\{e\}} &
		Rus., Blr. \\
	\hline
	{\cyr \Ee} & {\cyr \ee} & \texttt{$\backslash$Ee} &
		\texttt{$\backslash$ee} &
		Ukr. \\
	\hline
	{\cyr Zh} & {\cyr zh} & \texttt{Zh} & \texttt{zh} &
		Rus., Ukr., Blr., Bul., Mak., Serb \\
	\hline
	{\cyr Z} & {\cyr z} & \texttt{Z} & \texttt{z} &
		Rus., Ukr., Blr., Bul., Mak., Serb \\
	\hline
	{\cyr \Dz} & {\cyr \dz} & \texttt{$\backslash$Dz} &
		\texttt{$\backslash$dz} &
		Mak. \\
	\hline
	{\cyr I} & {\cyr i} & \texttt{I} & \texttt{i} &
		Rus., Ukr., Bul., Mak., Serb \\
	\hline
	{\cyr \I} & {\cyr \i} & \texttt{$\backslash$I} &
		\texttt{$\backslash$i} &
		Ukr., Blr., Russian until 1917 \\
	\hline
	{\cyr \"{\I}} & {\cyr \"{\i}} &
		\texttt{$\backslash$"\{$\backslash$I\}} &
		\texttt{$\backslash$"\{$\backslash$i\}} &
		Ukr. \\
	\hline
	{\cyr \U{I}} & {\cyr \U{i}} & \texttt{$\backslash$U\{I\}} &
		\texttt{$\backslash$U\{i\}} &
		Rus., Ukr., Blr., Bul. \\
	\hline
	{\cyr J} & {\cyr j} & \texttt{J} & \texttt{j} &
		Mak., Serb \\
	\hline
	{\cyr K} & {\cyr k} & \texttt{K} & \texttt{k} &
		Rus., Ukr., Blr., Bul., Mak., Serb \\
	\hline
	{\cyr \'{K}} & {\cyr \'{k}} & \texttt{$\backslash$'\{K\}} &
		\texttt{$\backslash$'\{k\}} &
		Mak. \\
	\hline
	{\cyr L} & {\cyr l} & \texttt{L} & \texttt{l} &
		Rus., Ukr., Blr., Bul., Mak., Serb \\
	\hline
	{\cyr Lj} & {\cyr lj} & \texttt{Lj} & \texttt{lj} &
		Mak., Serb \\
	\hline
	{\cyr M} & {\cyr m} & \texttt{M} & \texttt{m} &
		Rus., Ukr., Blr., Bul., Mak., Serb \\
	\hline
	{\cyr N} & {\cyr n} & \texttt{N} & \texttt{n} &
		Rus., Ukr., Blr., Bul., Mak., Serb \\
	\hline
	{\cyr Nj} & {\cyr nj} & \texttt{Nj} & \texttt{nj} &
		Mak., Serb \\
	\hline
	{\cyr O} & {\cyr o} & \texttt{O} & \texttt{o} &
		Rus., Ukr., Blr., Bul., Mak., Serb \\
	\hline
	{\cyr P} & {\cyr p} & \texttt{P} & \texttt{p} &
		Rus., Ukr., Blr., Bul., Mak., Serb \\
	\hline
	{\cyr R} & {\cyr r} & \texttt{R} & \texttt{r} &
		Rus., Ukr., Blr., Bul., Mak., Serb \\
	\hline
	{\cyr S} & {\cyr s} & \texttt{S} & \texttt{s} &
		Rus., Ukr., Blr., Bul., Mak., Serb \\
	\hline
	{\cyr T} & {\cyr t} & \texttt{T} & \texttt{t} &
		Rus., Ukr., Blr., Bul., Mak., Serb \\
	\hline
	{\cyr \'{C}} & {\cyr \'{c}} & \texttt{$\backslash$'\{C\}} &
		\texttt{$\backslash$'\{c\}} &
		Serb \\
	\hline
	{\cyr U} & {\cyr u} & \texttt{U} & \texttt{u} &
		Rus., Ukr., Blr., Bul., Mak., Serb \\
	\hline
	{\cyr \U{U}} & {\cyr \U{u}} & \texttt{$\backslash$U\{U\}} &
		\texttt{$\backslash$U\{u\}} &
		Blr. \\
	\hline
	{\cyr F} & {\cyr f} & \texttt{F} & \texttt{f} &
		Rus., Ukr., Blr., Bul., Mak., Serb \\
	\hline
	{\cyr Kh} & {\cyr kh} & \texttt{Kh} & \texttt{kh} &
		Rus., Ukr., Blr., Bul., Mak., Serb \\
	\hline
	{\cyr Ts} & {\cyr ts} & \texttt{Ts} & \texttt{ts} &
		Rus., Ukr., Blr., Bul., Mak., Serb \\
	\hline
	{\cyr Ch} & {\cyr ch} & \texttt{Ch} & \texttt{ch} &
		Rus., Ukr., Blr., Bul., Mak., Serb \\
	\hline
	{\cyr \Dzh} & {\cyr \dzh} & \texttt{$\backslash$Dzh} &
		\texttt{$\backslash$dzh} &
		Mak., Serb \\
	\hline
	{\cyr Sh} & {\cyr sh} & \texttt{Sh} & \texttt{sh} &
		Rus., Ukr., Blr., Bul., Mak., Serb \\
	\hline
	{\cyr Shch} & {\cyr shch} & \texttt{Shch} & \texttt{shch} &
		Rus., Ukr., Bul. \\
	\hline
	{\cyr \Hard} & {\cyr \hard} & \texttt{$\backslash$Hard} &
		\texttt{$\backslash$hard} &
		Rus., Bul. \\
	\hline
	{\cyr Y} & {\cyr y} & \texttt{Y} & \texttt{y} &
		Rus., Blr. \\
	\hline
	{\cyr \Soft} & {\cyr \soft} & \texttt{$\backslash$Soft} &
		\texttt{$\backslash$soft} &
		Rus., Ukr., Blr., Bul. \\
	\hline
	{\cyr \E} & {\cyr \e} & \texttt{$\backslash$E} &
		\texttt{$\backslash$e} &
		Rus., Blr. \\
	\hline
	{\cyr Yu} & {\cyr yu} & \texttt{Yu} & \texttt{yu} &
		Rus., Ukr., Blr., Bul. \\
	\hline
	{\cyr Ya} & {\cyr ya} & \texttt{Ya} & \texttt{ya} &
		Rus., Ukr., Blr., Bul. \\
	\hline
	{\cyr \Izhitza} & {\cyr \izhitza} & \texttt{$\backslash$Izhitza} &
		\texttt{$\backslash$izhitza} & Russian until 1917 \\
	\hline
	{\cyr \Yatz} & {\cyr \yatz} & \texttt{$\backslash$Yatz} &
		\texttt{$\backslash$yatz} & Russian until 1917 \\
	\hline
	{\cyr \Fita} & {\cyr \fita} & \texttt{$\backslash$Fita} &
		\texttt{$\backslash$fita} &  Russian until 1917 \\
	\hline
	{\cyr N0} & & \texttt{N0} & & \\
	\hline
	{\cyr <} & {\cyr >} & \texttt{<} & \texttt{>} & \\
	\hline
\end{tabular}


\end{document}


