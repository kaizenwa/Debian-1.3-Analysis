%
% The Hyperlatex manual, version 1.4 PL 1
%
% Last $Modified: Thu Dec 14 14:06:03 1995 by otfried $
%
% If using Latex2e, use 'documentclass' 
\makeatletter
\@ifundefined{documentclass}{
  \documentstyle[hyperlatex]{article}

  %% \xspace from the xspace package---included
  %%  for the benefit of Latex2.09 users
  
  \def\xspace{\futurelet\@let@token\@xspace}
  \def\@xspace{%
    \ifx\@let@token\bgroup\else
    \ifx\@let@token\egroup\else
    \ifx\@let@token\/\else
    \ifx\@let@token\ \else
    \ifx\@let@token~\else
    \ifx\@let@token.\else
    \ifx\@let@token,\else
    \ifx\@let@token:\else
    \ifx\@let@token;\else
    \ifx\@let@token?\else
    \ifx\@let@token'\else
    \ifx\@let@token)\else
    \ifx\@let@token-\else
    \space
    \fi\fi\fi\fi\fi\fi\fi\fi\fi\fi\fi\fi\fi}

  \newcommand{\LongTableAvailable}{0}
  }{
  \documentclass[a4paper]{article}
  \usepackage{a4}
  \usepackage{hyperlatex}
  \usepackage{xspace}
  \usepackage{longtable}
  \newcommand{\LongTableAvailable}{1}
  }
\makeatother

%% Html declarations: Output directory and filenames, node title

\htmltitle{Hyperlatex Manual}

\htmldirectory{html}
\htmladdress{otfried@postech.vision.ac.kr}

\htmlattributes{BODY}{BGCOLOR="#ffffe6"}
\htmlattributes{TABLE}{BORDER}

%% Redefine toppanel to include the Index

\newcommand{\toppanel}[6]{%
  \IfLink{#1#2#3}{%
    \IfLink{#1}{\xlink{\htmlimage{\thehtmlicons/previous.xbm}}{#1}}{%
      \htmlimage{\thehtmlicons/previous.xbm}}
    \IfLink{#2}{\xlink{\htmlimage{\thehtmlicons/up.xbm}}{#2}}{%
      \htmlimage{\thehtmlicons/up.xbm}}
    \IfLink{#3}{\xlink{\htmlimage{\thehtmlicons/next.xbm}}{#3}}{%
      \htmlimage{\thehtmlicons/next.xbm}}
    \link{\textbf{Index}}{sec:index}\\
    \IfLink{#1}{\textbf{Go backward to }\xlink{#4}{#1}\\}{}%
    \IfLink{#2}{\textbf{Go up to }\xlink{#5}{#2}\\}{}%
    \IfLink{#3}{\textbf{Go forward to }\xlink{#6}{#3}}{}
    \htmlrule{}}{}}

%% two useful shortcuts: \+, \=
\newcommand{\+}{\verb+}
\renewcommand{\=}{\back{}}

%% General macros
\newcommand{\Html}{\textsc{Html}\xspace }
\newcommand{\latex}{\LaTeX\xspace }
\newcommand{\latexinfo}{\texttt{latexinfo}\xspace }
\newcommand{\texinfo}{\texttt{texinfo}\xspace }
\newcommand{\dvi}{\textsc{Dvi}\xspace }

\makeindex

\title{The Hyperlatex Markup Language}
\author{Otfried Schwarzkopf\\
  {\footnotesize Dept.\ of Computer Science, Postech,}\\[-0.7ex]
  {\footnotesize San 31, Hyoja-Dong, Pohang 790-784, South Korea}}
\date{}

\begin{document}

\maketitle
\section{Introduction}

\topnode{The Hyperlatex Markup Language}

\emph{Hyperlatex} is a package that allows you to prepare documents in
\Html, and, at the same time, to produce a neatly printed document
from your input. Unlike some other systems that you may have seen,
Hyperlatex is \emph{not} a general \latex-to-\Html converter.  In my
eyes, conversion is not a solution to \Html authoring.  A well written
\Html document must differ from a printed copy in a number of rather
subtle ways---you'll see many examples in this manual.  I doubt that
these differences can be recognized mechanically, and I believe that
converted \latex can never be as readable as a document written in
\Html. 

\htmlmenu{6}

\begin{ifhtml}
  \section{Introduction}
\end{ifhtml}

The basic idea of Hyperlatex is to make it possible to write a
document that will look like a flawless \latex document when printed
and like a handwritten \Html document when viewed with an \Html
browser. In this it completely follows the philosophy of \latexinfo
(and \texinfo).  Like \latexinfo, it defines it's own input
format---the \emph{Hyperlatex markup language}---and provides two
converters to turn a document written in Hyperlatex markup into a \dvi
file or a set of \Html documents.

\label{philosophy}
Obviously, this approach has the disadvantage that you have to learn a
``new'' language to generate \Html files. However, the mental effort
for this is quite limited. The Hyperlatex markup language is simply a
well-defined subset of \latex that has been extended with commands to
create hyperlinks, to control the conversion to \Html, and to add
concepts of \Html such as horizontal rules and embedded images.
Furthermore, you can use Hyperlatex perfectly well without knowing
anything about \Html markup.

The fact that Hyperlatex defines only a restricted subset of \latex
does not mean that you have to restrict yourself in what you can do in
the printed copy. Hyperlatex provides many commands that allow you to
include arbitrary \latex commands (including commands from any package
that you'd like to use) which will be processed to create your printed
output, but which will be ignored in the \Html document.  However, you
do have to specify that \emph{explicitely}. Whenever Hyperlatex
encounters a \latex command outside its restricted subset, it will
complain bitterly.

The rationale behind this is that when you are writing your document,
you should keep both the printed document and the \Html output in
mind.  Whenever you want to use a \latex command with no defined \Html
equivalent, you are thus forced to specify this equivalent.  If, for
instance, you have marked a logical separation between paragraphs with
\latex's \verb+\bigskip+ command (a command not in Hyperlatex's
restricted set, since there is no \Html equivalent), then Hyperlatex
will complain, since very probably you would also want to mark this
separation in the \Html output. So you would have to write
\begin{verbatim}
   \texonly{\bigskip}
   \htmlrule
\end{verbatim}
to imply that the separation will be a \verb+\bigskip+ in the printed
version and a horizontal rule in the \Html-version.  Even better, you
could define a command \verb+\separate+ in the preamble and give it a
different meaning in \dvi and \Html output. If you find that for your
documents \verb+\bigskip+ should always be ignored in the \Html
version, then you can state so in the preamble as follows.
\begin{verbatim}
   \W\newcommand{\bigskip}{}
\end{verbatim}

This philosophy implies that in general an existing \latex-file will
not make it through Hyperlatex. In many cases, however, it will
suffice to go through the file once, adding the necessary markup that
specifies how Hyperlatex should treat the unknown commands.

\section{Using Hyperlatex}
\label{sec:using-hyperlatex}

Using Hyperlatex is easy. You create a file \textit{document.tex},
say, containing your document with Hyperlatex markup (which you will
learn from this manual).

If you use the command
\begin{example}
  latex document
\end{example}
then your file will be processed by \latex, resulting in a
\dvi-file, which you can print as usual.

On the other hand, you can run the command
\begin{example}
  hyperlatex document
\end{example}
and your document will be converted to \Html format, presumably to a
set of files called \textit{document.html}, \textit{document\_1.html},
\ldots{}.You can then use any \Html-viewer or \textsc{www}-browser,
such as \code{Mosaic} or \code{netscape}, to view the document.
(The entry point for your document will be the file
\textit{document.html}.)

This document describes how to use the Hyperlatex package and explains
the Hyperlatex markup language. It does not teach you {\em how} to
write for the web. There are \xlink{style
  guides}{http://www.w3.org/hypertext/WWW/Provider/Style/Overview.html}
available, which you might want to consult. Writing an on-line
document is not the same as writing a paper. I hope that Hyperlatex
will help you to do both properly.

This manual assumes that you are familiar with \latex, and that you
have at least some familiarity with hypertext documents---that is,
that you know how to use a \textsc{www}-browser and understand what a
\emph{hyperlink} is.

If you want, you can have a look at the source of this manual, which
illustrates most points discussed here. You can also look at the
documents on my \xlink{home
  page}{http://hobak.postech.ac.kr/otfried/}, all of which are created
using Hyperlatex.  If you have used Hyperlatex to make some document
available on the world wide web, I would love to hear about it. If
there's enough feedback, I would like to set up a list with demo
documents.

I am maintaining a Hyperlatex mailing list that is used exclusively to
announce new releases of Hyperlatex.  If you are interested to hear
about new versions, send email to
\xlink{\textit{otfried@vision.postech.ac.kr}}
{mailto:otfried@vision.postech.ac.kr}.

A final footnote: The converter to \Html implemented in Hyperlatex
is written in \textsc{Gnu} Emacs Lisp. If you want, you can invoke it
directly from Emacs (see the beginning of \file{hyperlatex-1.4.el} for
instructions). But even if you don't use Emacs, even if you don't like
Emacs, or even if you subscribe to \code{alt.religion.emacs.haters},
you can happily use Hyperlatex.  Hyperlatex can be invoked from the
shell as ``hyperlatex,'' and you will never know that this script
calls Emacs to produce the \Html document.

The Hyperlatex code is based on the Emacs Lisp macros of the
\code{latexinfo} package.

Hyperlatex is \link{copyrighted.}{sec:copyright}

\section{About the Html output}
\label{sec:about-html}

\label{nodes}
\cindex{node} Hyperlatex will automatically partition your input file
into separate \Html files, using the sectioning commands in the input.
It attaches buttons and menus to every \Html file, so that the reader
can walk through your document and can easily find the information
that she is looking for.  (Note that \Html documentation usually calls
a single \Html file a ``document''. In this manual we take the
\latex point of view, and call ``document'' what is enclosed in a
\code{document} environment. We will use the term \emph{node} for the
individual \Html files.)  You may want to experiment a bit with
\texonly{the \Html version of} this manual. You'll find that every
\+\section+ and \+\subsection+ command starts a new node. The \Html
node of a section that contains subsections contains a menu whose
entries lead you to the subsections. Furthermore, every \Html node has
three buttons: \emph{Next}, \emph{Previous}, and \emph{Up}.

The \emph{Next} button leads you to the next section \emph{at the same
  level}. That means that if you are looking at the node for the
section ``Getting started,'' the \emph{Next} button takes you to
``Conditional Compilation,'' \emph{not} to ``Preparing an input file''
(the first subsection of ``Getting started''). If you are looking at the
last subsection of a section, there will be no \emph{Next} button, and
you have to go \emph{Up} again, before you can step further.  This
makes it easy to browse quickly through one level of detail, while
only delving into the lower levels when you become interested.
(It is possible to change this default behavior.)

\cindex[topnode]{\code{\back{}topnode}}
\label{topnode}
If you look at \texonly{the \Html output for} this manual, you'll find
that there is one special node that acts as the entry point to the
manual, and as the parent for all its sections. This node is called
the \emph{top node}.  Everything between the \+\topnode+ command and
the first \+\section+ (or \+\chapter+ command, if you are not using
the \texttt{article} class) goes into the top node. Hyperlatex ignores
all the text between \+\+\+begin{document}+ and \+\topnode+---use this
area to create the title information for your printed document. The
\+\topnode+ command takes a single argument, which is displayed as the
heading of your top node. You can leave it empty (for instance if you
start the top node with an image).
  
\label{htmltitle}
\label{preamble}
An \Html file needs a \emph{title}. This title {\em must be set\/} in
the preamble\footnote{\label{footnote-preamble}The \emph{preamble} of
  a \latex file is the part between the \code{\back{}documentclass}
  command and the \code{\back{}begin\{document\}} command.  \latex
  does not allow text in the preamble, you can only put definitions
  and declarations there.} of your document using the
\code{\back{}htmltitle} command. You should use something not too
long, but meaningful. (The \Html title is often displayed by browsers
in the window header, and is used in history lists or bookmark files.)
The title you specify is used directly for the top node of your
document. The other nodes get a title composed of this and the section
heading.

\label{htmladdress}
\cindex[htmladdress]{\code{\back{}htmladdress}} It is common practice
to put a short notice at the end of every \Html node, with a reference
to the author and possibly the date of creation. You can do this by
using the \code{\back{}htmladdress} command in the preamble, like
this:
\begin{verbatim}
   \htmladdress{otfried@postech.vision.ac.kr}
\end{verbatim}

\section{Trying it out}
\label{sec:trying-it-out}

For those who don't read manuals, here are a few hints to allow you
to use Hyperlatex quickly. 

Hyperlatex implements a certain subset of \latex, and adds a number of
other commands that allow you to write better \Html. If you already
have a document written in \latex, the effort to convert it to
Hyperlatex should be quite limited. You will have to add the
\+\htmltitle+ command in the preamble, and the \+\topnode+ command in
the beginning of the document. Both commands must start at the
beginning of the line.

In short, the beginning of your document ought to look something like
this:\footnote{If you are still using \latex2.09, replace the first
  two lines by
  \begin{example}
    \back{}documentstyle[hyperlatex]\{article\}
  \end{example}}
\begin{example}
  \=documentclass\{article\}
  \=usepackage\{hyperlatex\}
  
  \=htmltitle\{\textit{Title of HTML nodes}\}
  \=htmladdress\{\textit{Your Email address, for instance}\}
  
      \textit{more LaTeX declarations, if you want}
  
  \=title\{\textit{Title for LaTeX document}\}
  \=author\{\textit{Author for LaTeX document}\}
  
  \=begin\{document\}
  
  \=maketitle
  \=section\{Introduction\}
  
  \=topnode\{Welcome to this HTML Document\}
  
  This is the beginning of the section titled ``Introduction'' in
  the printed manual, and at the same time the beginning of the top
  node of the HTML document\ldots
\end{example}
Note the use of the \textit{hyperlatex} package. It contains the
definitions of the Hyperlatex commands that are not part of \latex.

Hyperlatex ignores everything before the line starting with
\code{\back{}topnode} for the \Html output. The \+\topnode+ command
does not produce any output in the printed copy.

If your document contains footnotes, then you will also need to add
\link{\code{\=htmlfootnotes}}{sec:footnotes} before the final
\+\end{document}+.

Those two (or three) commands are all that is absolutely needed by
Hyperlatex, and adding them should suffice for a simple \latex
document. You might try it on the \file{sample2e.tex} file that comes
with \LaTeXe, to get a feeling for the \Html formatting of the
different \latex concepts.

Sooner or later Hyperlatex will fail on a \latex-document. As
explained in the introduction, Hyperlatex is not meant as a general
\latex-to-\Html converter. It has been designed to understand a certain
subset of \latex, and will treat all other \latex commands with an
error message. This does not mean that you should not use any of these
instructions for getting exactly the printed document that you want.
By all means, do. But you will have to hide those commands from
Hyperlatex using the \link{escape mechanisms}{sec:escaping}.

And you should learn about the commands that allow you to generate
much more natural \Html than any plain \latex-to-\Html converter
could.  For instance, \+\ref+ and \+pageref+ are not understood by the
Hyperlatex converter, because they have no meaning in the \Html output
(as figures, sections, and pages are not numbered---there's no need
for that in \Html). Their function is taken over by the
\link{\code{\=link}}{link} command.

The following sections explain in detail what you can and cannot do in
Hyperlatex.

Many aspects of the generated output can be \link{customized}[, see
Section~\Ref]{sec:customizing}.

\section{Parsing by \latex \LaTeX{} and Hyperlatex}
\label{sec:parsing}

You are writing an input file that has to be read by \latex as well
as the Hyperlatex converter. The parsing done by \latex is complex,
and has many of us surprised in certain situations. It was hopeless to
try to imitate this complex behavior using a modest collection of
Emacs Lisp macros. Nevertheless, Hyperlatex should behave well on your
\latex files. If your source is comprehensible to \latex (with the
\file{hyperlatex.sty} package), then Hyperlatex should not have
\emph{syntactical} problems with it. There is, however, one difference
in parsing arguments: In \latex, you can write
\begin{example}
  \back{}emph example,
\end{example}
and what you will get is `\emph{e}xample'. Hyperlatex will complain
about this. To get the same effect, you will have to write
\begin{example}
  \back{}emph\{e\}xample.
\end{example}

Furthermore, Hyperlatex does not tokenize the input. It does not read
the file character by character, but jumps over your text to the next
interesting character. This has some consequences, mainly on the
processing of user-defined macros.

The parsing done by Hyperlatex is even more restricted in the
\link{preamble}{footnote-preamble}. In fact, Hyperlatex only looks in
the preamble for a small set of commands. The commands are only found
if they start at the beginning of a line, with only white space in
front of them (the command definitions may be prepended with \+\W+).
The only commands that are recognized in the preamble are
\begin{menu}
\item \link{\code{\back{}htmldirectory}}{htmldirectory}
\item \link{\code{\back{}htmlname}}{htmlname}
\item \link{\code{\back{}htmltitle}}{htmltitle}
\item \link{\code{\back{}htmldepth}}{htmldepth}
\item \link{\code{\back{}htmlautomenu}}{htmlautomenu}
\item \link{\code{\back{}htmladdress}}{htmladdress}
\item \link{\code{\back{}newcommand}}{newcommand}
\item \link{\code{\back{}newenvironment}}{newenvironment}
\item \link{\code{\back{}renewcommand}}{newcommand}
\item \link{\code{\back{}renewenvironment}}{newenvironment}
\item \link{\code{\=htmlicons}}{htmlicons}
\item \link{\code{\=NotSpecial}}{not-special}
\item \link{\code{\=htmllevel}}{htmllevel}
\item \link{\code{\=htmlpanel}}{htmlpanel}
\item \link{\code{\=htmlattributes}}{htmlattributes}
\end{menu}
  
\section[Getting started]{A \latex subset --- Getting started}
\label{sec:getting-started}

Starting with this section, we take a stroll through the
\link{\latex-book}[~\Cite]{latex-book}, explaining all features that
Hyperlatex understands, additional features of Hyperlatex, and some
missing features. For the \latex output the general rule is that
\emph{no \latex command has been changed}. If a familiar \latex
command is listed in this manual, it is understood both by \latex
and the Hyperlatex converter, and its \latex meaning is the familiar
one. If it is not listed here, you can still use it by
\link{escaping}{sec:escaping} into \TeX-only mode, but it will then
have effect in the printed output only.

\subsection{Preparing an input file}
\label{sec:special-characters}
\cindex[back]{\+\back+}
\cindex[%]{\+\%+}
\cindex[~]{\+\~+}
\cindex[^]{\+\^+}
There are ten characters that \latex and Hyperlatex treat specially:
\begin{verbatim}
      \  {  }  ~  ^  _  #  $  %  &
\end{verbatim}
To typeset one of these, use
\begin{verbatim}
      \back   \{   \}  \~{}  \^{}  \_  \#  \$  \%  \&
\end{verbatim}
Sometimes it is useful to turn off the special meaning of some of
these ten characters. For instance, when writing documentation about
programs in~C, it might be useful to be able to write
\code{some\_variable} instead of always having to type
\code{some\=\_variable}. This can be achieved with the
\link{\code{\=NotSpecial}}{not-special} command.

In principle, all other characters simply typeset themselves. This has
to be taken with a grain of salt, though. \latex still obeys
ligatures, which turns \kbd{ffi} into `ffi', and some characters, like
\kbd{>}, do not resemble themselves in some fonts \texonly{(\kbd{>}
  looks like > in roman font)}. The only characters for which this is
critical are \kbd{<}, \kbd{>}, and \kbd{|}. Better use them in a
typewriter-font.  Note that \texttt{?{}`} and \texttt{!{}`} are
ligatures in any font and are displayed and printed as \texttt{?`} and
\texttt{!`}.

\cindex[par]{\+\par+}
Like \latex, the Hyperlatex converter understands that an empty line
indicates a new paragraph. You can achieve the same effect using the
command \+\par+.

\subsection{Dashes and Quotation marks}
\label{dashes}
Hyperlatex translates a sequence of two dashes \+--+ into a single
dash, and a sequence of three dashes \+---+ into two dashes \+--+. The
quotation mark sequences \+''+ and \+``+ are translated into simple
quotation marks \kbd{"}.


\subsection{Simple text generating commands}
\cindex[latex]{\code{\back{}LaTeX}}
The following simple \latex macros are implemented in Hyperlatex:
\begin{menu}
\item \+\LaTeX+ produces \latex.
\item \+\TeX+ produces \TeX{}.
\item \+\LaTeXe+ produces {\LaTeXe}.
\item \+\ldots+ produces three dots \ldots{}
\item \+\today+ produces \today---although this might depend on when
  you use it\ldots
\end{menu}

\subsection{Preventing line breaks}
\cindex[~]{\+~+}

The \verb+~+ is a special character in Hyperlatex, and is replaced by
the \Html-tag for \xlink{``non-breakable
  space''}{http://www.w3.org/hypertext/WWW/MarkUp/Entities.html}.

As we saw before, you can typeset the \kbd{\~{}} character by typing
\+\~{}+. This is also the way to go if you need the \kbd{\~{}} in an
argument to an \Html command that is processed by Hyperlatex, such as
in the \var{URL}-argument of \link{\code{\=xlink}}{xlink}.

You can also use the \+\mbox+ command. It is implemented by replacing
all sequences of whitespace in the argument by a single
\+~+. Obviously, this restricts what you can use in the
argument. (Better don't use any math mode material in the argument.)

\subsection{Emphasizing Text}
\cindex[em]{\verb+\em+}
\cindex[emph]{\verb+\emph+}
You can emphasize text using \+\emph+ or the old-style command
\+\em+. It is also possible to use the construction \+\begin{em}+
  \ldots \+\end{em}+.

\subsection{Footnotes}
\label{sec:footnotes}
\cindex[footnote]{\+\footnote+}
\cindex[htmlfootnotes]{\+\htmlfootnotes+}
If you use footnotes in your document, you will have to tell
Hyperlatex where to place them in the \Html output. This is done with
the \+\htmlfootnotes+ command, which has to come \emph{after} all
\+\footnote+ commands in the document. The usual place is just before
the final \+\end{document}+.

\subsection{Formulas}
\label{sec:math}
\cindex[math]{\verb+\math+}

There was no \emph{math mode} in \Html prior to \Html3. \Html3
implements a \texttt{<math>} tag, but the specification is still
somewhat preliminary, and the Hyperlatex implementation for it is
therefore experimental. Whether Hyperlatex generates the
\texttt{<math>} tag depends on your setting of the
\link{\code{\=htmllevel}}{htmllevel}.

Hyperlatex understands the \+$+ sign delimiting math mode as well as
\+\(+ and \+\)+. Subscripts and superscripts produced using \+_+ and
\+^+ are understood. If the \+\htmllevel+ is \verb|html2+| or
\+html3+, then \Html tags for subscripts and superscripts are created.
Otherwise, Hyperlatex creates a textual approximation of the form
\textit{a\^{}2 = x\^{}\{2n\}}.

Hyperlatex now implements many common math mode commands, so simple
formulas in your text should be converted to some textual
representation. If you are not satisfied with that representation, you
can use the new \verb+\math+ command:
\begin{example}
  \verb+\math[+\var{{\Html}-version}]\{\var{\LaTeX-version}\}
\end{example}
In \latex, this command typesets the \var{\LaTeX-version}, which is
read in math mode (with all special characters enabled, if you
have disabled some using \link{\code{\=NotSpecial}}{not-special}).
Hyperlatex typesets the optional argument if it is present, or
otherwise the \latex-version.

If, for instance, you want to typeset the \math{i}th element
(\verb+the \math{i}th element+) of an array as \math{a_i} in \latex,
but as \code{a[i]} in \Html, you can use
\begin{verbatim}
   \math[\code{a[i]}]{a_{i}}
\end{verbatim}

It takes a bit of care to find the best representation for your
formula. This is an example of where any mechanical \latex-to-\Html
converter must fail---I hope that Hyperlatex's \+\math+ command will
help you produce a good-looking and functional representation.

You could create a bitmap for a complicated expression, but you should
be aware that bitmaps eat transmission time, and they only look good
when the resolution of the browser is nearly the same as the
resolution at which the bitmap has been created, which is not a
realistic assumption. In many situations, there are easier solutions:
If $x_{i}$ is the $i$th element of an array, then I would rather write
it as \verb+x[i]+ in \Html.  If it's a variable in a program, I'd
probably write \verb+xi+. In another context, I might want to write
\textit{x\_i}. To write Pythagoras's theorem, I might simply use
\verb/a^2 + b^2 = c^2/, or maybe \texttt{a*a + b*b = c*c}. To express
``For any $\varepsilon > 0$ there is a $\delta > 0$ such that for $|x
- x_0| < \delta$ we have $|f(x) - f(x_0)| < \varepsilon$'' in \Html, I
would write ``For any \textit{eps} \texttt{>} \textit{0} there is a
\textit{delta} \texttt{>} \textit{0} such that for
\texttt{|}\textit{x}\texttt{-}\textit{x0}\texttt{|} \texttt{<}
\textit{delta} we have
\texttt{|}\textit{f(x)}\texttt{-}\textit{f(x0)}\texttt{|} \texttt{<}
\textit{eps}.''

More on math \link*{later}[ in section~\Ref]{sec:more-math}.

\subsection{Ignorable input}
\cindex[%]{\verb+%+}
The percent character \kbd{\%} introduces a comment in Hyperlatex.
Everything after a \kbd{\%} to the end of the line is ignored, as well
as any white space on the beginning of the next line.

\subsection{Document class and title page}
This material appears before the \link{\code{\back{}topnode}}{topnode}
command and is therefore ignored by the Hyperlatex converter. You can
use everything you want there.

The \+abstract+ environment is defined in Hyperlatex, so you can use
that after the \+\topnode+ command.

\subsection{Sectioning}
\label{sec:sectioning}
\cindex[section]{\verb+\section+}
\cindex[subsection]{\verb+\subsection+}
\cindex[subsubsection]{\verb+\subsection+}
\cindex[paragraph]{\verb+\paragraph+}
\cindex[subparagraph]{\verb+\subparagraph+}
\cindex[chapter]{\verb+\chapter+}
The sectioning commands \verb+\section+, \verb+\subsection+,
\verb+\subsubsection+, \verb+\paragraph+, and \verb+\subparagraph+ are
recognized by Hyperlatex and used to partition the document into
\link{nodes}{nodes}. You can also use the starred version and the
optional argument for the sectioning commands.  The star will be
ignored, as Hyperlatex does not number sections, and the optional
argument will be used for node titles and in menus.

\cindex[htmlheading]{\verb+\htmlheading+}
\label{htmlheading}
You will probably soon want to start an \Html node without a heading,
or maybe with a bitmap before the main heading. This can be done by
leaving the argument to the sectioning command empty. (You can still
use the optional argument to set the title of the \Html node.)
Do not use \+\htmlimage+ inside the argument of the sectioning
command. The right way to start a document with an image is the
following:
\begin{verbatim}
\T\section{An example of a node starting with an image}
\W\section[Node with Image]{}
\W\begin{center}\htmlimage{theimage.gif}\end{center}
\W\htmlheading[1]{An example of a node starting with an image}
\end{verbatim}
The \+\htmlheading+ command creates a heading in the \Html output just
as \+\section+ does, but without starting a new node.  The optional
argument has to be a number from~1 to~6, and specifies the level of
the heading (in \+article+ style, level~1 corresponds to \+\section+,
level~2 to \+\subsection+, and so on).

\cindex[protect]{\+\protect+}
\cindex[noindent]{\+\noindent+}
You can use the commands \verb+\protect+ and \+\noindent+. They will be
ignored in the \Html-version.

\subsection{Displayed material}
\label{sec:displays}
\cindex[quote]{\verb+quote+ environment}
\cindex[quotation]{\verb+quotation+ environment}
\cindex[verse]{\verb+verse+ environment}
\cindex[center]{\verb+center+ environment}
\cindex[itemize]{\verb+itemize+ environment}
\cindex[menu]{\verb+menu+ environment}
\cindex[enumerate]{\verb+enumerate+ environment}
\cindex[description]{\verb+description+ environment}
The \verb+quote+, \verb+quotation+, and \verb+verse+ environment are
all implemented by the Hyperlatex converter---but they are all
identical!

The \+center+ environment is also identical to the \+quote+
environment, unless you have set the \+\htmllevel+ to \+netscape+, in
which case an \Html tag is created that is only understood by the
\code{netscape} browser.

To make lists, you can use the \verb+itemize+, \verb+enumerate+, and
\verb+description+ environments. You \emph{cannot} specify an optional
argument to \verb+\item+ in \verb+itemize+ or \verb+enumerate+, and
you \emph{must} specify one for \verb+description+.

All these environments can be nested.

The \verb+\\+ command is recognized, with and without \verb+*+. You
can use the optional argument to \+\\+, but it will be ignored.

There is also a \verb+menu+ environment, which looks like an
\verb+itemize+ environment, but is somewhat denser since the space
between items has been reduced. It is only meant for single-line
items.

Hyperlatex understands \+\[+ and \+\]+ and the environments
\+displaymath+, \+equation+, and \+equation*+ for displaying formulas.
Note that \+$$+ is not understood.

\section[Conditional Compilation]{Conditional Compilation: Escaping
  into one mode} 
\label{sec:escaping}

In many situations you want to achieve slightly (or maybe even
drastically) different behavior of the \latex code and the
\Html-output.  Hyperlatex offers several different ways of letting
your document depend on the mode.


\subsection{\latex versus Html mode}
\label{sec:versus-mode}
\cindex[texonly]{\verb+\texonly+}
\cindex[texorhtml]{\verb+\texorhtml+}
\cindex[htmlonly]{\verb+\htmlonly+}
\label{texonly}
\label{texorhtml}
\label{htmlonly}
The easiest way to put a command or text in your document that is only
included in one of the two output modes it by using a \verb+\texonly+
or \verb+\htmlonly+ command. They ignore their argument, if in the
wrong mode, and otherwise simply expand it:
\begin{verbatim}
   We are now in \texonly{\LaTeX}\htmlonly{HTML}-mode.
\end{verbatim}
In cases such as this you can simplify the notation by using the
\+\texorhtml+ command, which has two arguments:
\begin{verbatim}
   We are now in \texorhtml{\LaTeX}{HTML}-mode.
\end{verbatim}

\label{W}
\label{T}
\cindex[T]{\verb+\T+}
\cindex[W]{\verb+\W+}
Another possibility is by prefixing a line with \verb+\T+ or
\verb+\W+. \verb+\T+ acts like a comment in \Html-mode, and as a noop
in \latex-mode, and for \verb+\W+ it is the other way round:
\begin{verbatim}
   We are now in
   \T \LaTeX-mode.
   \W HTML-mode.
\end{verbatim}


\cindex[iftex]{\code{iftex}}
\cindex[ifhtml]{\code{ifhtml}}
\label{iftex}
\label{ifhtml}
The last way of achieving this effect is useful when there are large
chunks of text that you want to skip in one mode---a \Html-document
might skip a section with a detailed mathematical analysis, a
\latex-document will not contain a node with lots of hyperlinks to
other documents.  This can be done using the \code{iftex} and
\code{ifhtml} environments:
\begin{verbatim}
   We are now in
   \begin{iftex}
     \LaTeX-mode.
   \end{iftex}
   \begin{ifhtml}
     HTML-mode.
   \end{ifhtml}
\end{verbatim}

\label{tex}
\cindex[tex]{\code{tex}} Instead of the \+iftex+ environment, you can
also use the \+tex+ environment. It is different from \+iftex+ only if
you have used \link{\code{\=NotSpecial}}{not-special} in the preamble.

\subsection{Ignoring more input}
\label{sec:comment}
\cindex[comment]{\+comment+ environment}
The contents of the \+comment+ environment is ignored.

\subsection{Flags --- more on conditional compilation}
\label{sec:flags}
\cindex[ifset]{\code{ifset} environment}
\cindex[ifclear]{\code{ifclear} environment}

You can also have sections of your document that are included
depending on the setting of a flag:
\begin{example}
  \verb+\begin{ifset}{+\var{flag}\}
    Flag \var{flag} is set!
  \verb+\end{ifset}+

  \verb+\begin{ifclear}{+\var{flag}\}
    Flag \var{flag} is not set!
  \verb+\end{ifset}+
\end{example}
A flag is simply the name of a \TeX{} command. A flag is considered
set if the command is defined and its expansion is neither empty nor
the single character ``0'' (zero).

You could for instance select in the preamble which parts of a
document you want included (in this example, parts~A and~D are
included in the processed document):
\begin{example}
   \=newcommand\{\=IncludePartA\}\{1\}
   \=newcommand\{\=IncludePartB\}\{0\}
   \=newcommand\{\=IncludePartC\}\{0\}
   \=newcommand\{\=IncludePartD\}\{1\}
     \ldots
   \=begin\{ifset\}\{IncludePartA\}
     \textit{Text of part A}
   \=end\{ifset\}
     \ldots
   \=begin\{ifset\}\{IncludePartB\}
     \textit{Text of part B}
   \=end\{ifset\}
     \ldots
   \=begin\{ifset\}\{IncludePartC\}
     \textit{Text of part C}
   \=end\{ifset\}
     \ldots
   \=begin\{ifset\}\{IncludePartD\}
     \textit{Text of part D}
   \=end\{ifset\}
     \ldots
\end{example}
Note that it is permitted to redefine a flag (using \+\renewcommand+)
in the document. That is particularly useful if you use these
environments in a macro.

\section{Carrying on}
\label{sec:carrying-on}

In this section we continue to Chapter~3 of the \latex-book, dealing
with more advanced topics.


\subsection{Changing the type style}
\label{sec:type-style}
\cindex[underline]{\+\underline+}
\cindex[textit]{\+textit+}
\cindex[textbf]{\+textbf+}
\cindex[textsc]{\+textsc+}
\cindex[texttt]{\+texttt+}
\cindex[it]{\verb+\it+}
\cindex[bf]{\verb+\bf+}
\cindex[tt]{\verb+\tt+}
\label{font-changes}
\label{underline}
Hyperlatex understands the following physical font specifications of
\LaTeXe{} (yes, they are even defined if you still use \LaTeX2.09):
\begin{menu}
\item \+\textbf+ for \textbf{bold}
\item \+\textit+ for \textit{italic}
\item \+\textsc+ for \textsc{small caps}
\item \+\texttt+ for \texttt{typewriter}
\item \+\underline+ for \underline{underline}
\end{menu}
In \LaTeXe{} font changes are
cumulative---\+\textbf{\textit{BoldItalic}}+ typesets the text in a
bold italic font. Different \Html browsers will display different
things. 

The following old-style commands are also supported:
\begin{menu}
\item \verb+\bf+ for {\bf bold}
\item \verb+\it+ for {\it italic}
\item \verb+\tt+ for {\tt typewriter}
\end{menu}
So you can write
\begin{example}
  \{\=it italic text\}
\end{example}
but also
\begin{example}
  \=textit\{italic text\}
\end{example}
You can use \verb+\/+ to separate slanted and non-slanted fonts (it
will be ignored in the \Html-version).

Hyperlatex complains about any other \latex commands for font changes,
in accordance with its \link{general philosophy}{philosophy}. If you
do believe that, say, \+\sf+ should simply be ignored, you can easily
ask for that in the preamble by defining:
\begin{example}
  \=W\=newcommand\{\=sf\}\{\}
\end{example}

Both \latex and \Html encourage you to express yourself in terms
of \emph{logical concepts} instead of visual concepts. (Otherwise, you
wouldn't be using Hyperlatex but some \textsc{Wysiwyg} editor to
create \Html.) In fact, \Html defines tags for \emph{logical}
markup, whose rendering is completely left to the user agent (\Html
client). 

The Hyperlatex package defines a standard representation for these
logical tags in \latex---you can easily redefine them if you don't
like the standard setting.

The logical font specifications are:
\begin{menu}
\item \+\cit+ for \cit{citations}.
\item \+\code+ for \code{code}.
\item \+\dfn+ for \dfn{defining a term}.
\item \+\em+ and \+\emph+ for \emph{emphasized text}.
\item \+\file+ for \file{file.names}.
\item \+\kbd+ for \kbd{keyboard input}.
\item \verb+\samp+ for \samp{sample input}.
\item \verb+\strong+ for \strong{strong emphasis}.
\item \verb+\var+ for \var{variables}.
\end{menu}

\subsection{Changing type size}
\label{sec:type-size}
\cindex[normalsize]{\+\normalsize+}
\cindex[small]{\+\small+}
\cindex[footnotesize]{\+\footnotesize+}
\cindex[scriptsize]{\+\scriptsize+}
\cindex[tiny]{\+\tiny+}
\cindex[large]{\+\large+}
\cindex[Large]{\+\Large+}
\cindex[LARGE]{\+\LARGE+}
\cindex[huge]{\+\huge+}
\cindex[Huge]{\+\Huge+}
Hyperlatex understands the \latex declarations to change the type
size. \Html tags for font size changes are generated only if the
\link{\code{\=htmllevel}}{htmllevel} is set to either \+netscape+ or
\+html3+. The \Html font changes are relative to the \Html node's
\emph{basefont size}. (\+\normalfont+ being the basefont size,
\+\large+ begin the basefont size plus one etc.) To set the basefont
size, you can use
\begin{example}
  \=html\{basefont size=\var{x}\}
\end{example}
where \var{x} is a number between~1 and~7.

\subsection{Symbols from other languages}
\cindex{accents}
\cindex{\verb+\'+}
\cindex{\verb+\`+}
\cindex{\verb+\~+}
\cindex{\verb+\^+}
\cindex[c]{\verb+\c+}
 \label{accents}
Hyperlatex recognizes the accent commands
\begin{verbatim}
      \'    \`   \^   \~
\end{verbatim}
However, not all possible accents are available in \Html. Hyperlatex
will make a \Html-entity for the accents in \textsc{iso} Latin~1, but
will reject all other accent sequences. The command \verb+\c+ can be
used to put a cedilla on a letter `c' (either case), but on no other
letter.  The following is legal
\begin{verbatim}
     Der K{\"o}nig sa\ss{} am wei{\ss}en Strand von Cura\c{c}ao und
     nippte an einer Pi\~{n}a Colada \ldots
\end{verbatim}
and produces
\begin{quote}
  Der K{\"o}nig sa\ss{} am wei{\ss}en Strand von Cura\c{c}ao und
  nippte an einer Pi\~{n}a Colada \ldots
\end{quote}
\label{hungarian}
Not legal are \verb+Ji{\v r}\'{\i}+, or \verb+Erd\H{o}s+.
To get a `\'{\i}', you have to type \verb+\'{\i}+, not \verb+\'\i+.

Hyperlatex understands the following symbols:
\begin{center}
  \T\leavevmode
  \begin{tabular}{|cl|cl|cl|} \hline
    \oe & \code{\=oe} & \aa & \code{\=aa} & ?` & \code{?{}`} \\
    \OE & \code{\=OE} & \AA & \code{\=AA} & !` & \code{!{}`} \\
    \ae & \code{\=ae} & \o  & \code{\=o}  & \ss & \code{\=ss} \\
    \AE & \code{\=AE} & \O  & \code{\=O}  & & \\
    \S  & \code{\=S}  & \copyright & \code{\=copyright} & &\\
    \P  & \code{\=P}  & \pounds    & \code{\=pounds} & & \T\\ \hline
  \end{tabular}
\end{center}

\+\quad+ and \+\qquad+ produce some empty space.

\subsection{Mathematical formulas}
\label{sec:more-math}

The math mode support in Hyperlatex is still more or less
experimental. Many math mode macros are implemented, but currently
simply generate a textual representation (so \+\alpha+ generates
\textit{alpha}). Only when \+\htmllevel+ is set to \+html3+, then tags
and entities for \Html3 are created (for instance
``\texorhtml{\code{\&alpha;}}{\htmlsym{alpha}}'' for \+\alpha+). This
feature is completely experimental. 

You can use \+$+, \+\(+, \+\)+, \+\[+, \+\]+, and the environments
\+displaymath+ and \+equation+ to type mathematics.

Subscripts, superscripts, fractions (with \+\frac+) and roots (with
\+\sqrt+) are understood. So are all greek letters, and the commands
\begin{verbatim}
   \pm \times \div \ast \cdot \cdots \ldots \setminus 
   \leq \le \geq \ge \equiv \approx \neq \mid \parallel
   \infty \partial \forall \exists
   \uparrow \downarrow \leftarrow \rightarrow \leftrightarrow
   \Uparrow \Downarrow

   \sum \prod \int 
\end{verbatim}

All log-like functions are implemented. They simply typeset
their own name.

Support for other math-mode features will be added when stable \Html3
browsers become widely available.

\subsection{Defining commands and environments}
\cindex[newcommand]{\verb+\newcommand+}
\cindex[newenvironment]{\verb+\newenvironment+}
\cindex[renewcommand]{\verb+\renewcommand+}
\cindex[renewenvironment]{\verb+\renewenvironment+}
\label{newcommand}
\label{newenvironment}

Hyperlatex understands command definitions with \+\newcommand+ and
\+\newenvironment+. \+\renewcommand+ and \+\renewenvironment+ are
understood as well (Hyperlatex makes no attempt to test whether a
command is actually already defined or not.)

Note that it is not possible to redefine a Hyperlatex command that is
\emph{hard-coded} into the Hyperlatex converter. So you could redefine
the command \+\cite+ or the \+verse+ environment, but you cannot
redefine \+\T+.
(Currently, the only way to determine whether a command is hard-coded
is to try or to look at the Hyperlatex source file.)

When a command definition appears in the preamble, it must start at
the beginning of a line, with only whitespace before it (the macro
definitions may be prepended with \+\W+ as well).

Some examples:
\begin{verbatim}
   \newcommand{\Html}{\textsc{Html}}

   \T\newcommand{\bad}{$\surd$}
   \W\newcommand{\bad}{\htmlimage{badexample_bitmap.xbm}}

   \newenvironment{badexample}{\begin{description}
                   \item[\bad]}{\end{description}}

   \W \newenvironment{smallexample}{\begin{example}}{\end{example}}
   \T \newenvironment{smallexample}{\begingroup\small
               \begin{example}}{\end{example}\endgroup}
\end{verbatim}
The \verb+\bad+ command and the \verb+smallexample+ environments are
good examples for \link{conditional compilation}{sec:escaping}. The
\verb+smallexample+ environment is equal to
\link{\code{example}}{example} in \Html, but is typeset in a smaller
font in the \latex document.

Command definitions made by Hyperlatex are global, their scope is not
restricted to the enclosing environment.

Note that Hyperlatex does not tokenize its input the way \TeX{} does.
To evaluate your macros, Hyperlatex simply inserts the expansion
string, replaces occurrences of \+#1+ to \+#9+ by the arguments,
strips one \kbd{\#} from strings of at least two \kbd{\#}'s, and then
reevaluates the whole. Since the white space after a command is
already removed when the command is parsed, the following code will
\emph{not} work in Hyperlatex:
\begin{verbatim}
   \newcommand{\smallit}{\small\it}
     ...
   And now some \smallit text in smaller italics.
\end{verbatim}
Hyperlatex will complain with the error message ``Command
\+\ittext+ undefined.'' To avoid this, you should always leave a
final space in the expansion string, if it ends with a macro
invocation.
So the right way is to write:
\begin{verbatim}
   \newcommand{\smallit}{\small\it }
     ...
   And now some \smallit text in smaller italics.
\end{verbatim}
And everything will work fine.

Problems also occur when you try to use \kbd{\%}, \+\T+, or \+\W+ in
the expansion string. Don't do that.

\subsection{Theorems and such}
There is no \verb+\newtheorem+ command. But you can define an
environment which does approximately the same:
\begin{verbatim}
   %% LaTeX definition
   \newtheorem{guess}{Conjecture}

   %% HTML definition
   \begin{ifhtml}
   \newenvironment{guess}{\begin{quotation}\textbf{Conjecture.}
      \begin{it}}{\end{it}\end{quotation}}
   \end{ifhtml}
\end{verbatim}

\subsection{Figures and other floating bodies}
\cindex[figure]{\code{figure} environment}
\cindex[table]{\code{table} environment}
\cindex[caption]{\verb+\caption+}

You can use \code{figure} and \code{table} environments and the
\verb+\caption+ command. They will not float, but will simply appear
at the given position in the text. No special space is left around
them, so put a \code{center} environment in a figure. The \code{table}
environment is mainly used with the \link{\code{tabular}
  environment}{tabular}\texonly{ below}.  You can use the \+\caption+
command to place a caption. The starred versions \+table*+ and
\+figure*+ are supported as well.

\subsection{Lining it up in columns}
\label{sec:tabular}
\label{tabular}
\cindex[tabular]{\+tabular+ environment}
\cindex[hline]{\verb+\hline+}
\cindex{\verb+\\+}
\cindex{\verb+\\*+}
\cindex{\&}
\cindex[multicolumn]{\+\multicolumn+}
\cindex[htmlcaption]{\+\htmlcaption+}
The \code{tabular} environment is available in Hyperlatex.

If you use \+\htmllevel{html2}+, then Hyperlatex has to display the
table using preformatted text. In that case, Hyperlatex removes all
the \+&+ markers and the \+\\+ or \+\\*+ commands. The result is not
formatted any more, and simply included in the \Html-document as a
``preformatted'' display. This means that if you format your source
file properly, you will get a well-formatted table in the
\Html-document---but it is fully your own responsibility.
You can also use the \verb+\hline+ command to include a horizontal
rule.

If you use any \+\htmllevel+ higher than \+html2+, then Hyperlatex can
use tags for making tables. The argument to \+tabular+ may only
contain the specifiers \kbd{|}, \kbd{c}, \kbd{l}, and \kbd{r}. The
\kbd{|} specifier is silently ignored. You can force borders around
your table (and every single cell) by using
\+\htmlattributes*{TABLE}{BORDER}+ immediatly before your \+tabular+
environment.  You can use the \+\multicolumn+ command.  \+\hline+ is
understood and ignored.

The \+\htmlcaption+ has to be used right after the
\+\+\+begin{tabular}+. It sets the caption for the \Html table. (In
\Html, the caption is part of the \+tabular+ environment. However, you
can as well use \+\caption+ outside the environment.)

\cindex[cindex]{\+\htmltab+}
\label{htmltab}
If you have made the \+&+ character \link{non-special}{not-special},
you can use the macro \+\htmltab+ as a replacement.

Here is an example:
\T \begingroup\small
\begin{verbatim}
    \begin{table}[htp]
    \T\caption{Keyboard shortcuts for \textit{Ipe}}
    \begin{center}
    \begin{tabular}{|l|lll|}
    \htmlcaption{Keyboard shortcuts for \textit{Ipe}}
    \hline
                & Left Mouse      & Middle Mouse  & Right Mouse      \\
    \hline
    Plain       & (start drawing) & move          & select           \\
    Shift       & scale           & pan           & select more      \\
    Ctrl        & stretch         & rotate        & select type      \\
    Shift+Ctrl  &                 &               & select more type \T\\
    \hline
    \end{tabular}
    \end{center}
    \end{table}
\end{verbatim}
\T \endgroup
The example is typeset as follows:
\begin{table}[htp]
\T\caption{Keyboard shortcuts for \textit{Ipe}}
\begin{center}
\begin{tabular}{|l|lll|}
\htmlcaption{Keyboard shortcuts for \textit{Ipe}}
\hline
            & Left Mouse      & Middle Mouse  & Right Mouse      \\
\hline
Plain       & (start drawing) & move          & select           \\
Shift       & scale           & pan           & select more      \\
Ctrl        & stretch         & rotate        & select type      \\
Shift+Ctrl  &                 &               & select more type \T\\
\hline
\end{tabular}
\end{center}
\end{table}

Some more complicated examples:
\begin{table}[h]
  \begin{center}
    \T\leavevmode
    \begin{tabular}{|l|l|r|}
      \hline\hline
      \emph{type} & \multicolumn{2}{c|}{\emph{style}} \\ \hline
      smart & red & short \\
      rather silly & puce & tall \T\\ \hline\hline
    \end{tabular}
  \end{center}
\end{table}

To create certain effects you need to employ the
\link{\code{\=htmlattributes}}{htmlattributes} command:
\begin{table}[h]
  \begin{center}
    \T\leavevmode
    \htmlattributes*{TABLE}{BORDER}
    \htmlattributes*{TD}{ROWSPAN="2"}
    \begin{tabular}{||l|lr||}\hline
      gnats & gram & \$13.65 \\ \T\cline{2-3}
            \texonly{&} each & \multicolumn{1}{r||}{.01} \\ \hline
      gnu \htmlattributes*{TD}{ROWSPAN="2"} & stuffed
                   & 92.50 \\ \T\cline{1-1}\cline{3-3}
      emu   &      \texonly{&} \multicolumn{1}{r||}{33.33} \\ \hline
      armadillo & frozen & 8.99 \T\\ \hline
    \end{tabular}
  \end{center}
\end{table}


\subsection{Simulating typed text}
\cindex[verbatim]{\code{verbatim} environment}
\cindex[verb]{\verb+\verb+}
\label{verbatim}
The \code{verbatim} environment and the \verb+\verb+ command are
implemented. The starred varieties are currently not implemented.
(The implementation of the \code{verbatim} environment is not the
standard \latex implementation, but the one from the \+verbatim+
package by Rainer Sch\"opf). The command \verb-\+-\var{verb}\verb-+-
can be used as a shortcut for \verb-\verb+-\var{verb}\verb-+-.

\cindex[example]{\code{example} environment}
\label{example}
Furthermore, there is another, new environment \code{example}.
\code{example} is also useful for including program listings or code
examples. Like \code{verbatim}, it is typeset in a typewriter font
with a fixed character pitch, and obeys spaces and line breaks. But
here ends the similarity, since \code{example} obeys the special
characters \+\+, \+{+, \+}+, and \+%+. You can 
still use font changes within an \code{example} environment, and you
can also place \link{hyperlinks}{sec:cross-references} there.  Here is
an example:
\begin{verbatim}
   To clear a flag, use
   \begin{example}
     {\back}clear\{\var{flag}\}
   \end{example}
\end{verbatim}

\cindex[exampleindent]{\verb+\exampleindent+}
Note also that an \code{example} environment is indented
automatically, while a \code{verbatim} environment is not.
In the \latex document, you can set the amount of indentation by
setting \code{\=exampleindent}:
\begin{example}
  \+\setlength{\exampleindent}{4mm}+
\end{example}

(The \+example+ environment is very similar to the \+alltt+
environment of the \+alltt+ package. The differences are that example
is automatically indented and obeys the \+%+ character.)

\section{Moving information around}
\label{sec:moving-information}

In this section we deal with questions related to cross referencing
between parts of your document, and between your document and the
outside world. This is where Hyperlatex gives you the power to write
natural \Html documents, unlike those produced by any \latex
converter.  A converter can turn a reference into a hyperlink, but it
will have to keep the text more or less the same. If we wrote ``More
details can be found in the classical analysis by Harakiri [8]'', then
a converter may turn ``[8]'' into a hyperlink to the bibliography in
the \Html document. In handwritten \Html, however, we would probably
leave out the ``[8]'' alltogether, and make the \emph{name}
``Harakiri'' a hyperlink.

The same holds for references to sections and pages. The Ipe manual
says ``This parameter can be set in the configuration panel
(Section~11.1)''. A converted document would have the ``11.1'' as a
hyperlink. Much nicer \Html is to write ``This parameter can be set in
the configuration panel'', with ``configuration panel'' a hyperlink to
the section that describes it.  If the printed copy reads ``We will
study this more closely on page~42,'' then a converter must turn
the~``42'' into a symbol that is a hyperlink to the text that appears
on page~42. What we would really like to write is ``We will later
study this more closely,'' with ``later'' a hyperlink---after all, it
makes no sense to even allude to page numbers in an \Html document.

The Ipe manual also says ``Such a file is at the same time a legal
Encapsulated Postscript file and a legal \latex file---see
Section~13.'' In the \Html copy the ``Such a file'' is a hyperlink to
Section~13, and there's no need for the ``---see Section~13'' anymore.

\subsection{Cross-references}
\label{sec:cross-references}
\label{label}
\label{link}
\cindex[label]{\verb+\label+}
\cindex[link]{\verb+\link+}
\cindex[Ref]{\verb+\Ref+}
\cindex[Pageref]{\verb+\Pageref+}

You can use the \verb+\label{+\var{label}\} command to attach a
\var{label} to a position in your document. This label can be used to
create a hyperlink to this position from any other point in the
document.
This is done using the \verb+\link+ command:
\begin{example}
  \verb+\link{+\var{anchor}\}\{\var{label}\}
\end{example}
This command typesets anchor, expanding any commands in there, and
makes it an active hyperlink to the position marked with \var{label}:
\begin{verbatim}
   This parameter can be set in the
   \link{configuration panel}{sect:con-panel} to influence ...
\end{verbatim}

The \verb+\link+ command does not do anything exciting in the printed
document. It simply typesets the text \var{anchor}. If you also want a
reference in the \latex output, you will have to add a reference
using \verb+\ref+ or \verb+\pageref+. This reference has to be escaped
from the Hyperlatex converter. Sometimes you will want to place the
reference directly behind the \var{anchor} text. In that case you can
use the optional argument to \verb+\link+:
\begin{verbatim}
   This parameter can be set in the
   \link{configuration
     panel}[~(Section~\ref{sect:con-panel})]{sect:con-panel} to
   influence ... 
\end{verbatim}
The optional argument is ignored in the \Html-output.

The starred version \verb+\link*+ suppressed the anchor in the printed
version, so that we can write
\begin{verbatim}
   We will see \link*{later}[in Section~\ref{sl}]{sl}
   how this is done.
\end{verbatim}
It is very common to use \verb+\ref{+\textit{label}\verb+}+ or
\verb+\pageref{+\textit{label}\verb+}+ inside the optional
argument, where \textit{label} is the label set by the link command.
In that case the reference can be abbreviated as \verb+\Ref+ or
\verb+\Pageref+ (with capitals). These definitions are already active
when the optional arguments are expanded, so we can write the example
above as
\begin{verbatim}
   We will see \link*{later}[in Section~\Ref]{sl}
   how this is done.
\end{verbatim}
Often this format is not useful, because you want to put it
differently in the printed manual. Still, as long as the reference
comes after the \verb+\link+ command, you can use \verb+\Ref+ and
\verb+\Pageref+.
\begin{verbatim}
   \link{Such a file}{ipe-file} is at
   the same time ... a legal \LaTeX{}
   file\texonly{---see Section~\Ref}.
\end{verbatim}

\cindex[ref]{\verb+\ref+, problems with}
Note that when you use \latex's \verb+\ref+ command, the label does
not mark a \emph{position} in the document, but a certain
\emph{object}, like a section, equation etc. It sometimes requires
some care to make sure that both the hyperlink and the printed
reference point to the right place, and sometimes you will have to
place the label twice. The \Html-label tends to be placed \emph{before} the
interesting object---a figure, say---, while the \latex-label tends
to be put \emph{after} the object (when the \verb+\caption+ command
has set the counter for the label).

A special case occurs for section headings. Always place labels
\emph{after} the heading. In that way, the \latex reference will be
correct, and the Hyperlatex converter makes sure that the link will
actually lead to a point directly before the heading---so you can see
the heading when you follow the link.


\subsection{Links to external information}
\label{sec:external-hyperlinks}
\label{xlink}
\cindex[xlink]{\verb+\xlink+}

You can place a hyperlink to a given \var{URL} (\xlink{Universal
  Resource Locator}
{http://www.w3.org/hypertext/WWW/Addressing/Addressing.html}) using
the \verb+\xlink+ command. Like the \verb+\link+ command, it takes an
optional argument, which is typeset in the printed output only:
\begin{example}
  \verb+\xlink{+\var{anchor}\}\{\var{URL}\}
  \verb+\xlink{+\var{anchor}\}[\var{printed reference}]\{\var{URL}\}
\end{example}
In the \Html-document, \var{anchor} will be an active hyperlink to the
object \var{URL}. In the printed document, \var{anchor} will simply be
typeset, followed by the optional argument, if present. A starred
version \+\xlink*+ has the same function as for \+\link+.

If you need to use a \+~+ in the \var{URL} of an \+\xlink+ command, you have
to escape it as \+\~{}+ (the \var{URL} argument is an evaluated argument, so
that you can define macros for common \var{URL}'s).

\subsection{Links into your document}
\label{sec:into-hyperlinks}
\cindex[xname]{\verb+\xname+}
\cindex[xlabel]{\verb+\xlabel+}
\label{xname}
\label{xlabel}
The Hyperlatex converter automatically partitions your document into
\Html-nodes and generates \Html-tags for your \+\label+'s.
These automatically created names are simply numbers, and are not
useful for external references into your document---after all, the
exact numbers are going to change whenever you add or delete a
section or label, or when you change the \link{\code{\=htmldepth}}{htmldepth}.

If you want to allow links from the outside world into your new
document, you will have to do two things: First, you should give that
\Html node a mnemonic name that is not going to change when the
document is revised. Furthermore, you may want to place a mnemonic
label inside the node.

The \+\xname{+\var{name}\+}+ command is used to give the mnemonic name
\var{name} to the \emph{next} node created by Hyperlatex. This means
that you ought to place it \emph{in front of} a sectioning command.
The \+\xname+ command has no function for the \LaTeX-document. No
warning is created if no new node is started in between two \+\xname+
commands.

If you need an \Html label within a node to be referenced from
the outside, you can use the \+\xlabel{+\var{label}\+}+ command.
\var{label} has to be a legal \Html label. 

The argument of \+\xname+ and \+\xlabel+ is not expanded, so you
should not escape any special characters (such as~\+_+). On the other
hand, if you reference them using \+\xlink+, you will have to escape
special characters.

Here is an example: The section \xlink{``Changes between
  Hyperlatex~1.0 and Hyperlatex~1.1''}{hyperlatex\_changes.html} in
this document starts as follows.
\begin{verbatim}
   \xname{hyperlatex_changes}
   \section{Changes from Hyperlatex~1.0 to Hyperlatex~1.1}
   \label{sec:changes}
\end{verbatim}
It can be referenced inside this document with
\+\link{Changes}{sec:changes}+, and both inside and outside this
document with \+\xlink{Changes}{hyperlatex\_changes.html}+.

The entry about \+\xname+ and \+\xlabel+ in that section has been
marked using \+\xlabel{external_labels}+. You can therefore directly
\xlink{refer to that
  position}{hyperlatex\_changes.html\#external\_labels} from anywhere
using
\begin{verbatim}
   \xlink{xlabel is new}{hyperlatex\_changes.html\#external\_labels}
\end{verbatim}

\subsection{Bibliography and citation}
\label{sec:bibliography}
\cindex[thebibliography]{\code{thebibliography} environment}
\cindex[bibitem]{\verb+\bibitem+}
\cindex[Cite]{\verb+\Cite+}

Hyperlatex understands the \code{thebibliography} environment. Like
\latex, it creates a section titled ``References''.  The
\verb+\bibitem+ command sets a label with the given \var{cite key} at
the position of the reference. This means that you can use the
\verb+\link+ command to define a hyperlink to a bibliography entry.

The command \verb+\Cite+ is defined analogously to \verb+\Ref+ and
\verb+\Pageref+ by \verb+\link+.  If you define a bibliography like
this
\begin{verbatim}
   \begin{thebibliography}{99}
      \bibitem{latex-book}
      Leslie Lamport, \cit{\LaTeX: A Document Preparation System,}
      Addison-Wesley, 1986.
   \end{thebibliography}
\end{verbatim}
then you can add a reference to the \latex-book as follows:
\begin{verbatim}
   ... we take a stroll through the
   \link{\LaTeX-book}[~\Cite]{latex-book}, explaining ...
\end{verbatim}

\cindex[htmlcite]{\+\htmlcite+}
Furthermore, the command \+\htmlcite+ generates the citation (in our
case, \+\htmlcite{latex-book}+ would generate
``\htmlcite{latex-book}''). Finally, \+\cite+ is implemented as
\+\link{\htmlcite{#1}}{#1}+, so you can use it as usual in \latex, and
it will automatically become an active hyperlink, as in
``\cite{latex-book}''. 

\cindex[bibliography]{\verb+\bibliography+}
\cindex[bibliographystyle]{\verb+\bibliographystyle+}
Hyperlatex also understands the \verb+\bibliographystyle+ command
(which is ignored) and the \verb+\bibliography+ command. It reads the
\textit{.bbl} file, inserts its contents at the given position and
proceeds as  usual. Using this feature, you can include bibliographies
created with Bib\TeX{} in your \Html-document!
It would be possible to design a \textsc{www}-server that takes queries
into a Bib\TeX{} database, runs Bib\TeX{} and Hyperlatex
to format the output, and sends back an \Html-document.

\cindex[htmlbibitem]{\+\htmlbibitem+}
The formatting of the bibliography can be customized by redefining the
environment \+thebibliography+ and the macro \+\htmlbibitem+. The
default definitions are
\begin{verbatim}
   \newenvironment{thebibliography}[1]%
      {\chapter{References}\begin{description}}{\end{description}}
   \newcommand{\htmlbibitem}[2]{\label{#2}\item[{[#1]}]}
\end{verbatim}

If you use Bib\TeX{} to generate your bibliographies, then you will
probably want to incorporate hyperlinks into your \file{.bib}
files. No problem, you can simply use \+\xlink+. But what if you also
want to use the same \file{.bib} file with other (vanilla) \latex
files, that do not define the \+\xlink+ command? What if you want to
share your \file{.bib} files with colleagues around the world who do
not know about Hyperlatex?

Here is a trick that solves this problem without defining a new
Bib\TeX{} style or something similar: You can put an \var{URL} into the
\emph{note} field of a Bib\TeX{} entry as follows:
\begin{verbatim}
   note = "\def\HTML{\XURL}{ftp://nowhere.com/paper.ps}"
\end{verbatim}
This is perfectly understandable for plain \latex, which will simply
ignore the funny prefix \+\def\HTML{\XURL}+ and typeset the \var{URL}.

In your Hyperlatex source, however, you can put these definitions in
the preamble:
\begin{verbatim}
   \W\newcommand{\def}{}
   \W\newcommand{\HTML}[1]{#1}
   \W\newcommand{\XURL}[1]{\xlink{#1}{#1}}
\end{verbatim}
This will turn the \emph{note} field into an active hyperlink to the
document in question.

(An alternative approach would be to redefine some \latex command in
Hyperlatex, such as \+\relax+.)


\subsection{Splitting your input}
\label{sec:splitting}
\label{input}
\cindex[input]{\verb+\input+}
The \verb+\input+ command is implemented in Hyperlatex. The subfile is
inserted into the main document, and typesetting proceeds as usual.
You have to include the argument to \verb+\input+ in braces.

\subsection{Making an index or glossary}
\label{sec:index-glossary}
\cindex[index]{\verb+\index+}
\cindex[cindex]{\verb+\cindex+}
\cindex[htmlprintindex]{\verb+\htmlprintindex+}

The Hyperlatex converter understands the commands \verb+\index+ and
\verb+\cindex+, which are synonymous. It collects the entries
specified with these commands, and you can include a sorted index
using \verb+\htmlprintindex+. This index takes the form of a menu with
hyperlinks to the positions where the original \verb+\index+ commands
where located.
You can specify a different sort key for an index intry using the
optional argument of \verb+\cindex+:
\begin{verbatim}
   \cindex[index]{\verb+\index+}
\end{verbatim}
This entry will be sorted like ``\code{index}'', but typeset in the
index as ``\verb/\verb+\index+/''.

The \textit{hyperlatex.sty} style defines \verb+\cindex+ as follows:
\begin{menu}
\item
  \verb+\cindex{+\var{entry}\+}+ is expanded to
  \verb+\index{+\var{entry}\+}+, and
\item
  \verb+\cindex[+\var{sortkey}]\{\var{entry}\} ist expanded to
  \verb+\index{+\var{sortkey}\verb+@+\var{entry}\}.
\end{menu}
This realizes the same behavior as in the Hyperlatex converter if you
use the index processor \code{makeindex}. If not, you will have to
consult your \cit{Local Guide} and redefine \verb+\cindex+
appropriately.

The index in this manual was created using \verb+\cindex+ commands in
the source file, the index processor \code{makeindex} and the following
code:
\begin{verbatim}
   \W \section*{Index}
   \W \htmlprintindex
   \T %%
%% LaTeX style to handle hyperlatex files, version 1.4
%%
%% suitable for both Latex2.09 and Latex2e
%%
%%   $Modified: Tue Dec  5 18:47:06 1995 by otfried $
%%
%% This code has GNU copyleft, 1994,1995 Otfried Schwarzkopf
%%
\newif\if@hyla@ldLatex
%%
\@ifundefined{NeedsTeXFormat}{
  % LaTeX 2.09 
  \typeout{Package: 'hyperlatex' v1.4  Otfried Schwarzkopf}
  \@hyla@ldLatextrue
  \input{verbatim.sty}
  }{
  % LaTeX2e
  \NeedsTeXFormat{LaTeX2e}
  
  \ProvidesPackage{hyperlatex}
  [1995/10/01 v1.4 LaTeX2e package for Hyperlatex mode]
  \typeout{Package: 'hyperlatex' v1.4  Otfried Schwarzkopf}
  \@hyla@ldLatexfalse
  \RequirePackage{verbatim}
  }

\chardef\other=12

%%
%% Comments,  
%%
{\obeylines\gdef\hyla@W#1^^M{\endgroup\ignorespaces}}
\def\W{\begingroup\obeylines\catcode`\{=\other\catcode`\}=\other\hyla@W}

\newcommand{\htmlonly}[1]{}
\newcommand{\texorhtml}[2]{#1}
\newenvironment{iftex}{}{}
\newcommand{\texonly}[1]{#1}
\newcommand{\T}{}

%%
%% Treatment of special characters
%%

\def\hyla@nonspecials{}
\def\NotSpecial{\def\hyla@nonspecials}
\def\hyla@turnon{\let\do=\@makeother\hyla@nonspecials}

\if@hyla@ldLatex
\newcommand{\topnode}[1]{\hyla@turnon}
\else
\newcommand{\topnode}[1]{}
\AtBeginDocument{\hyla@turnon}
\fi

\let\htmltab=&

%%
%% \begin{tex} ... \end{tex}    escapes into raw Tex temporarily.
%% you can write {\tex .....} as well, if already escaped from Html
%%
\newenvironment{tex}{\catcode `\$=3 \catcode `\&=4 \catcode `\#=6
\catcode `\^=7 \catcode `\_=8 \catcode `\%=14}{}

%%
%% \back \LaTeXe
%% 

\newcommand{\back}{{\tt\char`\\}}

\if@hyla@ldLatex
\newcommand{\LaTeXe}{\LaTeX2$\epsilon$}
\fi

%%
%% \math
%%

\def\math{\@ifnextchar [{\@math}{\@math[]}}
\def\@math[#1]{\begingroup\tex\@@math}
\def\@@math#1{$#1$\endgroup}

%%
%% Commands that don't do anything interesting in Latex
%%

\def\hyla@pass{\begingroup\let\protect\@unexpandable@noexpand\@sanitize}

\newcommand{\htmldirectory}[1]{\gdef\gif@directory{#1}}
\newcommand{\htmlname}[1]{}
\newcommand{\htmldepth}[1]{}
\newcommand{\htmltitle}[1]{}
\newcommand{\htmladdress}[1]{}
\newcommand{\html}{\hyla@pass\@html}
\def\@html#1{\endgroup}

\newcommand{\htmlsym}[1]{}
\def\htmlrule{\@ifnextchar [{\@htmlrule}{\@htmlrule[]}}
\def\@htmlrule{\hyla@pass\@@htmlrule}
\def\@@htmlrule[#1]{\endgroup}

\newcommand{\htmllevel}[1]{}
\newcommand{\htmlmenu}[1]{}
\newcommand{\htmlautomenu}[1]{}
\newcommand{\htmlprintindex}{}
\newcommand{\htmlfootnotes}{}

\def\htmlimage{\@ifnextchar [{\@htmlimage}{\@htmlimage[]}}
\def\@htmlimage{\hyla@pass\@@htmlimage}
\def\@@htmlimage[#1]#2{\endgroup}

\newcommand{\xlabel}{\hyla@pass\hyla@ignore}
\newcommand{\xname}{\hyla@pass\hyla@ignore}
\def\hyla@ignore#1{\endgroup}

\newcommand{\htmlicons}[1]{}
\newcommand{\htmlpanel}[1]{}
\newcommand{\htmlheading}{\@ifnextchar[{\@htmlheading}{\@htmlheading[]}}
\def\@htmlheading[#1]#2{}

\newcommand{\htmlcaption}[1]{}

\def\htmlattributes{\@ifstar{\@htmlattributes}{\@htmlattributes}}
\def\@htmlattributes{\hyla@pass\@@htmlattributes}
\def\@@htmlattributes#1#2{\endgroup}

%%
%% GIF environment,
%%  will generate bitmaps when `\makegifs' is defined
%%
\newif\if@makegifs
\@ifundefined{makegifs}{\@makegifsfalse}{\@makegifstrue}

\def\gif{\@ifnextchar[{\@gif}{\@gif[b]}}
\def\@gif[#1]{\@ifnextchar[{\@@gif}{\@@gif[100]}}
\def\@@gif[#1]{\gdef\gif@resolution{#1}\@ifnextchar[{\@@@gif}{\@@@gif[300]}}
\def\@@@gif[#1]#2{\gdef\gif@dpi{#1}\gdef\gif@name{#2}\@@@@gif}

\if@makegifs
\typeout{**********************************************}
\typeout{* Making GIF bitmaps from Hyperlatex source! *}
\typeout{**********************************************}
\newwrite\@makegifcmds
\immediate\openout\@makegifcmds=\jobname.makegif
\newbox\@gifbox
\newcount\@gifcount\@gifcount=10000
\def\@@@@gif{\setbox\@gifbox=\vbox\bgroup\tex}
\def\endgif{\egroup
  {\global\advance\@gifcount by 1\count0=\@gifcount
    \immediate\write\@makegifcmds{dvips\space -f\space -p\space
      \the\@gifcount\space -n\space 1\space -D\space \gif@dpi\space
      \jobname.dvi\space>\space\gif@name.ps^^J%
      ps2gif\space -res\space \gif@resolution\space \gif@name.ps^^J%
      mv\space \gif@name.gif\space \gif@directory^^J}
    \shipout\copy\@gifbox}\unvbox\@gifbox}
\else
\def\@@@@gif{\tex}
\def\endgif{}
\fi
%%
%% Font style definitions
%%

\if@hyla@ldLatex
\newcommand{\emph}[1]{{\it #1}}
\newcommand{\textsl}[1]{{\sl #1}}
\newcommand{\textit}[1]{{\it #1}}
\newcommand{\textbf}[1]{{\bf #1}}
\newcommand{\texttt}[1]{{\tt #1}}
\newcommand{\textsc}[1]{{\sc #1}}
\fi

\let\cit=\textit
\let\code=\texttt
\let\kbd=\texttt
\let\samp=\texttt
\let\strong=\textbf
\let\var=\textsl
\let\dfn=\textit
\let\file=\textit

%%
%% \begin{example} ... \end{example} obeys spaces and lines
%%
%% the indent can be controlled by \exampleindent
%%
\newdimen\exampleindent
\setlength{\exampleindent}{7mm}
\def\hyla@example{\do\$\do\&\do\#\do\^\do\_\do\~}
{\obeyspaces
  \gdef\turnon@spaces{\let =\ \obeyspaces\catcode``=\active\@noligs}}
{\obeylines%
  \gdef\turnon@lines{\obeylines\def^^M{\par\def^^M{\leavevmode\par}}}}

\def\example{\list{}{\leftmargin\exampleindent
    \itemindent\z@ \rightmargin\z@ \parsep \z@ plus\p@}\item[]\tt
  \turnon@spaces\turnon@lines\let\do\@makeother\hyla@example}
\let\endexample=\endlist

\newenvironment{menu}{\list{$\bullet$}{\itemsep0pt\parsep0pt}}{\endlist}
\newenvironment{ifhtml}{\comment}{\endcomment}

\def\hyla@empty{}
\def\hyla@zero{0}
\newif\if@hyla

\def\hyla@setclear#1{\@ifundefined{#1}{\@hylafalse}{%
    \expandafter\ifx\csname#1\endcsname\hyla@empty\@hylafalse
    \else\expandafter\ifx\csname#1\endcsname\hyla@zero\@hylafalse
    \else\@hylatrue\fi\fi}}

\def\ifclear#1{\hyla@setclear{#1}
  \if@hyla
  %% arg empty -> skip
  \let\endifset\endcomment\let\hyla@comment\comment
  \else
  %% arg set -> ignore begin and end
  \let\endifset\relax\let\hyla@comment\relax
  \fi\hyla@comment}

\def\ifset#1{\hyla@setclear{#1}
  \if@hyla
  % arg empty -> ignore begin and end
  \let\endifclear\relax\let\hyla@comment\relax
  \else
  % arg set  -> skip environment
  \let\endifclear\endcomment\let\hyla@comment\comment
  \fi\hyla@comment}

%%
%% Define \link and \xlink macros
%%
\newcommand{\hyla@label}{}

\newcommand{\Ref}{\ref{\hyla@label}}
\newcommand{\Pageref}{\pageref{\hyla@label}}
\newcommand{\Cite}{\cite{\hyla@label}}

\newcommand{\htmlcite}[1]{\cite{#1}}

\def\link{\@ifstar{\@star@link}{\@@link}}
\def\@@link#1{#1\@@@link}
\def\@star@link#1{\@@@link}
\def\@@@link{\@ifnextchar [{\@link}% ] balance
  {\@link[]}}
\def\@link[#1]#2{\gdef\hyla@label{#2}#1}

\def\xlink{\@ifstar{\@star@xlink}{\@@xlink}}
\def\@@xlink#1{\@@@xlink{#1}}
\def\@star@xlink#1{\@@@xlink{}}
\def\@@@xlink#1{\@ifnextchar [{\@xlink{#1}}{\@xlink{#1}[]}}
\def\@xlink#1[#2]#3{\formatxlink{#1#2}{#3}}
\newcommand{\formatxlink}[2]{#1}

%
% index from latex.tex, and changed to include optional argument
%
\def\cindex{\@bsphack\begingroup
  \let\protect\@unexpandable@noexpand
  \@sanitize
  \@ifnextchar [{\hyla@argwrindex}% balance ]
  {\@wrindex}}
\def\hyla@argwrindex[#1]#2{\@wrindex{#1@#2}}

%% end of hyperlatex.sty

\end{verbatim}

\section{Designing it yourself}
\label{sec:design}

In this section we discuss the commands used to make things that only
occur in \Html-documents, not in printed papers. Practically all
commands discussed here start with \verb+\html+, indicating that the
command has no effect whatsoever in \latex.

\subsection{Making menus}
\label{sec:menus}

\label{htmlmenu}
\cindex[htmlmenu]{\verb+\htmlmenu+}

The \verb+\htmlmenu+ command generates a menu for the subsections
of the current section.
It takes a single argument, the depth of the desired menu. If you use
\verb+\htmlmenu{2}+ in a subsection, say, you will get a menu of
all subsubsections and paragraphs of this subsection.

If you use this command in a section, no \link{automatic
  menu}{htmlautomenu} for this section is created.

A typical application of this command is to put a ``master menu'' (the
analogon of a table of contents) in the \link{top node}{topnode},
containing all sections of all levels of the document. This can be
achieved by putting \verb+\htmlmenu{6}+ in the text for the top node.

\htmlrule{}
\T\bigskip
Some people like to close off a section after some subsections of that
section, somewhat like this:
\begin{verbatim}
   \section{S1}
   text at the beginning of section S1
     \subsection{SS1}
     \subsection{SS2}
   closing off S1 text

   \section{S2}
\end{verbatim}
This is a bit of a problem for Hyperlatex, as it requires the text for
any given node to be consecutive in the file. A workaround is the
following:
\begin{verbatim}
   \section{S1}
   text at the beginning of section S1
   \htmlmenu{1}
   \texonly{\def\savedtext}{closing off S1 text}
     \subsection{SS1}
     \subsection{SS2}
   \texonly{\bigskip\savedtext}

   \section{S2}
\end{verbatim}

\subsection{Rulers and images}
\label{sec:bitmap}

\label{htmlrule}
\cindex[htmlrule]{\verb+\htmlrule+}
\cindex[htmlimage]{\verb+\htmlimage+}
The command \verb+\htmlrule+ creates a horizontal rule spanning the
full screen width at the current position in the \Html-document.
It has an optional argument that you can use to add additional
attributes to the \Html tag. The optional argument is not evaluated
further, so you should not escape any special characters.

Additional tags are currently only understood by the some browsers, so
use the optional argument at your own risk.
Here is an example.
\begin{verbatim}
   \htmlrule[width=70% align=center]
\end{verbatim}
\htmlonly{This will result in the following rule.}
\htmlrule[width=70% align=center]

\label{htmlimage}
The command \verb+\htmlimage{+\var{URL}\} makes an inline bitmap % }
with the given \var{URL}. It takes an optional argument that can be
used to specify additional attributes understood by some \Html
browsers. One of the letters ``t'', ``c'', ``b'', ``l'', or ``r'' can
be specified as a shortcut for the alignments ``top'', ``center'',
``bottom'', ``left'', or ``right''. So \verb+\htmlimage[c]{image.xbm}+
includes the image in \textit{image.xbm}, vertically centered at the
current text position. A more complicated example is (using attributes
that are \emph{not} in \Html2):
\begin{example}
  \=htmlimage[align=left width=50 height=75 hspace=3]\{image.gif\}
\end{example}
The optional argument is not evaluated further, so you should not
escape any special characters.
The \var{URL} argument, on the other hand, is an evaluated argument, so that
you can define macros for common \var{URL}'s (such as your home page). That
means that if you need to use a special character (\+~+~is quite
common), you have to escape it (as~\+\~{}+ for the~\+~+).

This is what I use for figures in the \xlink{Ipe
  Manual}{http://hobak.postech.ac.kr/otfried/html/Ipe/Ipe.html} that
appear in both the printed document and the \Html-document:
\begin{verbatim}
   \begin{figure}
     \caption{The Ipe window}
     \begin{center}
       \texorhtml{\Ipe{window.ipe}}{\htmlimage{window.gif}}
     \end{center}
   \end{figure}
\end{verbatim}
(\verb+\Ipe+ is the command to include ``Ipe'' figures.)

\subsection{Adding raw Html}
\label{sec:raw-html}
\cindex[html]{\verb+\Html+}
\label{html}
\cindex[htmlsym]{\verb+\htmlsym+}
\cindex[rawhtml]{\verb+rawhtml+ environment}

Hyperlatex provides two commands to access the \Html-tag level.

\verb+\html{+\var{tag}\+}+ creates the \Html tag
\samp{\code{<}\var{tag}\code{>}}, and \verb+\htmlsym{+\var{entity}\+}+
creates the \Html entity description
\samp{\code{\&}\var{entity}\code{;}}.

\T \newcommand{\onequarter}{$1/4$}
\W \newcommand{\onequarter}{\htmlsym{##188}}
The \verb+\htmlsym+ command is useful if you need symbols from the
\textsc{iso} Latin~1 alphabet which are not predefined in Hyperlatex.
You could, for instance, define a macro for the fraction \onequarter{}
as follows:
\begin{verbatim}
   \T \newcommand{\onequarter}{$1/4$}
   \W \newcommand{\onequarter}{\htmlsym{##188}}
\end{verbatim}

Finally, the \+rawhtml+ environment allows you to write plain \Html,
if so is your desire. Everything between \+\begin{rawhtml}+ and
  \+\end{rawhtml}+ will simply be included literally in the \Html
output.

\subsection{Turning \TeX{} into bitmaps}
\label{sec:gif}
\cindex[gif]{\+gif+ environment}

Sometimes the only sensible way to represent some \latex concept in an
\Html-document is by turning it into a bitmap. Hyperlatex has an
environment \+gif+ that does exactly this: In the
\Html-version, it is turned into a reference to an inline
bitmap (just like \+\htmlimage+). In the \latex-version, the \+gif+
environment is equivalent to a \+tex+ environment. Note that running
the Hyperlatex converter doesn't create the bitmaps yet, you have to
do that in an extra step as described below.

The \+gif+ environment has three optional and one required arguments:
\begin{example}
  \=begin\{gif\}[\var{tags}][\var{resolution}][\var{font\_resolution}]%
\{\var{name}\}
    \var{\TeX{} material \ldots}
  \=end\{gif\}
\end{example}
For the \LaTeX-document, this is equivalent to
\begin{example}
  \=begin\{tex\}
    \var{\TeX{} material \ldots}
  \=end\{tex\}
\end{example}
For the \Html-version, it is equivalent to
\begin{example}
  \=htmlimage[\var{tags}]\{\var{name}.gif\}
\end{example}
The other two parameters, \var{resolution} and \var{font\_resolution},
are used when creating the \+gif+-file. They default to \math{100} and
\math{300} dots per inch.

Here is an example:
\begin{verbatim}
   \W\begin{quote}
   \begin{gif}{eqn1}
     \[
     \sum_{i=1}^{n} x_{i} = \int_{0}^{1} f
     \]
   \end{gif}
   \W\end{quote}
\end{verbatim}
produces the following output:
\W\begin{quote}
  \begin{gif}{eqn1}
    \[
    \sum_{i=1}^{n} x_{i} = \int_{0}^{1} f
    \]
  \end{gif}
\W\end{quote}

We could as well include a picture environment. The code
\texonly{\begin{footnotesize}}
\begin{verbatim}
  \begin{center}
    \begin{gif}[b][80]{boxes}
      \setlength{\unitlength}{0.1mm}
      \begin{picture}(700,500)
        \put(40,-30){\line(3,2){520}}
        \put(-50,0){\line(1,0){650}}
        \put(150,5){\makebox(0,0)[b]{$\alpha$}}
        \put(200,80){\circle*{10}}
        \put(210,80){\makebox(0,0)[lt]{$v_{1}(r)$}}
        \put(410,220){\circle*{10}}
        \put(420,220){\makebox(0,0)[lt]{$v_{2}(r)$}}
        \put(300,155){\makebox(0,0)[rb]{$a$}}
        \put(200,80){\line(-2,3){100}}
        \put(100,230){\circle*{10}}
        \put(100,230){\line(3,2){210}}
        \put(90,230){\makebox(0,0)[r]{$v_{4}(r)$}}
        \put(410,220){\line(-2,3){100}}
        \put(310,370){\circle*{10}}
        \put(355,290){\makebox(0,0)[rt]{$b$}}
        \put(310,390){\makebox(0,0)[b]{$v_{3}(r)$}}
        \put(430,360){\makebox(0,0)[l]{$\frac{b}{a} = \sigma$}}
        \put(530,75){\makebox(0,0)[l]{$r \in {\cal R}(\alpha, \sigma)$}}
      \end{picture}
    \end{gif}
  \end{center}
\end{verbatim}
\texonly{\end{footnotesize}}
creates the following image.
\begin{center}
  \begin{gif}[b][80]{boxes}
    \setlength{\unitlength}{0.1mm}
    \begin{picture}(700,500)
      \put(40,-30){\line(3,2){520}}
      \put(-50,0){\line(1,0){650}}
      \put(150,5){\makebox(0,0)[b]{$\alpha$}}
      \put(200,80){\circle*{10}}
      \put(210,80){\makebox(0,0)[lt]{$v_{1}(r)$}}
      \put(410,220){\circle*{10}}
      \put(420,220){\makebox(0,0)[lt]{$v_{2}(r)$}}
      \put(300,155){\makebox(0,0)[rb]{$a$}}
      \put(200,80){\line(-2,3){100}}
      \put(100,230){\circle*{10}}
      \put(100,230){\line(3,2){210}}
      \put(90,230){\makebox(0,0)[r]{$v_{4}(r)$}}
      \put(410,220){\line(-2,3){100}}
      \put(310,370){\circle*{10}}
      \put(355,290){\makebox(0,0)[rt]{$b$}}
      \put(310,390){\makebox(0,0)[b]{$v_{3}(r)$}}
      \put(430,360){\makebox(0,0)[l]{$\frac{b}{a} = \sigma$}}
      \put(530,75){\makebox(0,0)[l]{$r \in {\cal R}(\alpha, \sigma)$}}
    \end{picture}
  \end{gif}
\end{center}

It remains to describe how you actually generate those bitmaps from
your Hyperlatex source. This is done by running \latex on the input
file, setting a special flag that makes the resulting \dvi-file
contain an extra page for every \+gif+ environment.  Furthermore, this
\latex-run produces another file with extension \textit{.makegif},
which contains commands to run \+dvips+ and \+ps2gif+ to to extract
the interesting pages into Postscript files which are then converted
to \+gif+ format. Obviously you need to have \+dvips+ and \+ps2gif+
installed if you want to use this feature.  (A shellscript \+ps2gif+
is supplied with Hyperlatex. This shellscript uses \+ghostscript+ to
convert the Postscript files to \+ppm+ format, and then runs
\+ppmtogif+ to convert these into \+gif+-files.)

Assuming that everything has been installed properly, using this is
actually quite easy: To generate the \+gif+ bitmaps defined in your
Hyperlatex source file \file{source.tex}, you simply use
\begin{example}
  hyperlatex -gif source.tex
\end{example}
Note that since this runs latex on \file{source.tex}, the
\dvi-file \file{source.dvi} will no longer be what you want!

\section{Controlling Hyperlatex}
\label{sec:customizing}

Hyperlatex can be adapted to your wishes in a number of ways.

\subsection{Splitting into nodes and menus}
\label{htmldirectory}
\label{htmlname}
\cindex[htmldirectory]{\code{\back{}htmldirectory}}
\cindex[htmlname]{\code{\back{}htmlname}} \cindex[xname]{\+\xname+}
Normally, the \Html output for your document \file{document.tex} are
created in files \file{document\_?.html} in the same directory. You can
change both the name of these files as well as the directory using the
two commands \+\htmlname+ and \+\htmldirectory+ in the
\link{preamble}{preamble} of your source file:
\begin{example}
  \back{}htmldirectory\{\var{directory}\}
  \back{}htmlname\{\var{basename}\}
\end{example}
The actual files created by Hyperlatex are called
\file{directory/basename.html}, \file{directory/basename\_1.html},
\file{directory/basename\_2.html}, and so on. The filename can be
changed for individual nodes using the \link{\code{\=xname}}{xname}
command. 

\label{htmldepth}
\cindex[htmldepth]{\code{\back{}htmldepth}} Hyperlatex automatically
partitions the document into several \link{nodes}{nodes}. This is done
based on the \latex sectioning. The section commands
\code{\back{}chapter}, \code{\back{}section},
\code{\back{}subsection}, \code{\back{}subsubsection},
\code{\back{}paragraph}, and \code{\back{}subparagraph} are assigned
levels~1 to~6.  (If you use the \textit{article} document style,
\code{\back{}section} to \code{\back{}subparagraph} are given
levels~1 to~5, as there are no chapters).

The \code{\back{}htmldepth} command in the \link{preamble}{preamble}
determines at what depth separate nodes are created. The default
setting is~4, which means that (for \textit{article} style)
sections, subsections, and subsubsections are given their own nodes,
while paragraphs and subparagraphs are put into the node of their
parent subsection. You can change this by putting
\begin{example}
  \back{}htmldepth\{\var{depth}\}
\end{example}
in the \link{preamble}{preamble}. A value of~1 means that
the full document will be stored in a single file.

\label{htmlautomenu}
\cindex[htmlautomenu]{\code{\back{}htmlautomenu}}
The individual nodes of a \Html document are linked together using
\emph{hyperlinks}. Hyperlatex automatically places buttons on every
node that link it to the previous and next node of the same depth, if
they exist, and a button to go to the parent node.

Furthermore, Hyperlatex automatically adds a menu to every node,
containing pointers to all subsections of this section. (Here,
``section'' is used as the generic term for chapters, sections,
subsections, \ldots.) This may not always be what you want. You might
want to add nicer menus, with a short description of the subsections.
In that case you can turn off the automatic menus by putting
\begin{example}
  \back{}htmlautomenu\{0\}
\end{example}
in the preamble. On the other hand, you might also want to have more
detailed menus, containing not only pointers to the direct
subsections, but also to all subsubsections and so on. This can be
achieved by putting 
\begin{example}
  \back{}htmlautomenu\{\var{depth}\}
\end{example}
in the preamble, where \var{depth} is the desired depth of recursion.
The default behavior corresponds to a \var{depth} of 1.

Remember that all commands in the preamble must start at the beginning
of a line if you want Hyperlatex to see them.

\subsection{Customizing the navigation panels}
\label{sec:navigation}
\label{htmlpanel}
\cindex[htmlpanel]{\+\htmlpanel+}
\cindex[toppanel]{\+\toppanel+}
\cindex[bottompanel]{\+\bottompanel+}
\cindex[bottommatter]{\+\bottommatter+}
Normally, Hyperlatex adds a ``navigation panel'' at the beginning of
every \Html node. This panel has links to the next and previous
node on the same level, as well as to the parent node. The panel for
the top node has a link to the first chapter or section.

The easiest way to customize the navigation panel is to turn it off
for selected nodes. This is done using the commands \+\htmlpanel{0}+
and \+\htmlpanel{1}+. All nodes started while \+\htmlpanel+ is set
to~\math{0} are created without a navigation panel.

Furthermore, the navigation panels (and in fact the complete outline
of the created \Html files) can be customized to your own taste
by redefining some Hyperlatex macros. In fact, when it formats an
\Html node, Hyperlatex inserts the macro \+\toppanel+ at the
beginning, and the two macros \+\bottommatter+ and \+bottompanel+ at
the end. When \+\htmlpanel{0}+ has been set, then only \+\bottommatter+
is inserted.

The macros \+\toppanel+ and \+\bottompanel+ take six arguments. These
are (in this order) the \var{URL} of the previous node, the parent node, the
next node; and their titles. If a node has no previous or next node,
then the argument is empty. You can test for an empty string using the
\+\IfLink+ command. Its first argument is a string, it expands the
second argument if the string is non-empty, and the third argument
otherwise. 

The default definitions for the three macros are as follows.
\begin{verbatim}
\newcommand{\toppanel}[6]{%
  \IfLink{#1#2#3}{%
    \IfLink{#1}{\xlink{\htmlimage[ALT=""]{\thehtmlicons/previous.xbm}}{#1}}{%
      \htmlimage[ALT=""]{\thehtmlicons/previous.xbm}}
    \IfLink{#2}{\xlink{\htmlimage[ALT=""]{\thehtmlicons/up.xbm}}{#2}}{%
      \htmlimage[ALT=""]{\thehtmlicons/up.xbm}}
    \IfLink{#3}{\xlink{\htmlimage[ALT=""]{\thehtmlicons/next.xbm}}{#3}}{%
      \htmlimage[ALT=""]{\thehtmlicons/next.xbm}}\\
    \IfLink{#1}{\textbf{Go backward to }\xlink{#4}{#1}\\}{}%
    \IfLink{#2}{\textbf{Go up to }\xlink{#5}{#2}\\}{}%
    \IfLink{#3}{\textbf{Go forward to }\xlink{#6}{#3}}{}
    \htmlrule{}}{}}
\newcommand{\bottommatter}{\htmlrule\thehtmladdress\\}
\newcommand{\bottompanel}[6]{%
  \IfLink{#1#2#3}{%
    \IfLink{#1}
      {\xlink{\htmlimage[ALT="Prev"]{\thehtmlicons/previous.xbm}}{#1}}
      {\htmlimage[ALT=""]{\thehtmlicons/previous.xbm}}
    \IfLink{#2}
      {\xlink{\htmlimage[ALT="Up"]{\thehtmlicons/up.xbm}}{#2}}
      {\htmlimage[ALT=""]{\thehtmlicons/up.xbm}}
    \IfLink{#3}
      {\xlink{\htmlimage[ALT="Next"]{\thehtmlicons/next.xbm}}{#3}}
      {\htmlimage[ALT=""]{\thehtmlicons/next.xbm}}}{}}
\end{verbatim}
You can simply redefine them using \+\newcommand+. This manual, for
instance, redefines the top panel to include a pointer to the
\link{index}{sec:index}.

\cindex[thehtmladdress]{\+\thehtmladdress+}
\cindex[htmladdress]{\+\htmladdress+}
\cindex[thehtmlicons]{\+\thehtmlicons+}
\cindex[htmlicons]{\+\htmlicons+}
\label{htmlicons}
The command \+\thehtmladdress+ returns the string that you had set
using \+\htmladdress+, enclosed in \Html address tags. The command
\+\thehtmlicons+ returns the \var{URL} of the directory on your server where
the icons reside (assuming the person who installed Hyperlatex at your
site did that properly). If necessary, you can set this \var{URL} yourself,
using the command \+\htmlicons+ in the preamble.

\subsection{Setting the Html level}
\label{sec:html-level}
\label{htmllevel}

The growing number of \Html-dialects is a serious problem. While it
would be possible to restrict Hyperlatex to generate only standard
\Html2, many users requested for instance the fontsize changing
commands implemented by \+netscape+.

To solve this mess, Hyperlatex now has a declaration
\verb+\htmllevel+, that you can use in the preamble to set the type of
\Html that will be generated.  Legal values for the argument are
\texttt{html2}, \texttt{html2+}, \texttt{netscape}, or \texttt{html3}.

The setting of the \+\htmllevel+ has influence on the following
concepts:
\begin{table}[h]
  \begin{center}
    \T\leavevmode
    \begin{tabular}{lcccc}
                         & \Html2 & \Html2+ & \code{netscape} & \Html3 \\
    changing type-size   &        &         &      X          &    X   \\
    sub- \& superscripts &        &    X    &                 &    X   \\
    tables               &        &    X    &      X          &    X   \\ 
    ``real'' math mode   &        &         &                 &    X   \\
    centering            &        &         &      X          &    X   
    \end{tabular}
  \end{center}
\end{table}

Note that the optional argument of \link{\code{\=htmlrule}}{htmlrule}
and \link{\code{\=htmlimage}}{htmlimage} allows you to add arbitrary
attributes that may violate the standard that you have chosen with
\+\htmllevel+. The same may be true if you use
\link{\code{\=htmlattributes}}{htmlattributes}. 

\subsection{Changing the formatting of footnotes}
The appearance of footnotes in the \Html output can be customized by
redefining several macros:

The macro \code{\=htmlfootnotemark\{\var{n}\}} typesets the mark that
is placed in the text as a hyperlink to the footnote text. The default
definition is
\begin{verbatim}
  \newcommand{\htmlfootnotemark}[1]{\link{(#1)}{footnote-#1}}
\end{verbatim}

The environment \+thefootnotes+ generates the \Html node with the
footnote text. Every footnote is formatted with the macro
\code{\=htmlfootnoteitem\{\var{n}\}\{\var{text}\}}. The default
definitions are
\begin{verbatim}
   \newenvironment{thefootnotes}%
      {\chapter{Footnotes}
       \begin{description}}%
      {\end{description}}
   \newcommand{\htmlfootnoteitem}[2]%
      {\label{footnote-#1}\item[(#1)]#2}
\end{verbatim}

\subsection{Setting Html attributes}
\label{htmlattributes}
\cindex[htmlattributes]{\+\htmlattributes+}

If you are familiar with \Html, then you will sometimes want to be
able to add certain \Html attributes to the \Html tags generated by
Hyperlatex. This is possible using the command \+\htmlattributes+. Its
first argument is the name of an \Html tag (in capitals!), the second
argument can be used to specify attributes for that tag. The
declaration can be used in the preamble as well as in the document. A
new declaration for the same tag cancels any previous declaration,
unless you use the starred version of the command: It has effect only on
the next occurrence of the named tag, after which Hyperlatex reverts
to the previous state.

Note that not all tags generated by Hyperlatex can be influenced using
this declaration. You can only try and see (and complain to me if you
need access to a tag that Hyperlatex doesn't grant you).

Some common applications:

\+netscape+ allows you to specify the background color of an \Html
node using an attribute that you can set as follows. (If you do this
in the preamble, all nodes of your document will be colored this way.)
\begin{verbatim}
   \htmlattributes{BODY}{BGCOLOR="#ffffe6"}
\end{verbatim}

The following declaration makes the tables in your document have
borders. 
\begin{verbatim}
   \htmlattributes{TABLE}{BORDER}
\end{verbatim}

A more compact representation of the list environments can be enforced
using (this is for the \+itemize+ environment):
\begin{verbatim}
   \htmlattributes{UL}{COMPACT}
\end{verbatim}


\subsection{Making characters non-special}
\label{not-special}
\cindex[notspecial]{\+\NotSpecial+}
\cindex[tex]{\code{tex}}

Sometimes it is useful to turn off the special meaning of some of the
ten special characters of \latex. For instance, when writing
documentation about programs in~C, it might be useful to be able to
write \code{some\_variable} instead of always having to type
\code{some\=\_variable}, especially if you never use any formula and
hence do not need the subscript function. This can be achieved with
the \link{\code{\=NotSpecial}}{not-special} command.
The characters that you can make non-special are
\begin{verbatim}
      ~  ^  _  #  $  &
\end{verbatim}
For instance, to make characters \kbd{\$} and \kbd{\^{}} non-special,
you need to use the command
\begin{verbatim}
      \NotSpecial{\do\$\do\^}
\end{verbatim}
Yes, this syntax is weird, but it makes the implementation much easier.

Note that whereever you put this declaration in the preamble, it will
only be turned on by \+\+\+begin{document}+. This means that you can
still use the regular \latex special characters in the
preamble.\footnote{For technical reasons the special input mode is
  turned on by \code{\back{}topnode} if you are using \LaTeX2.09.}

Even within the \link{\code{iftex}}{iftex} environment the characters
you specified will remain non-special. Sometimes you will want to
return them their full power. This can be done in a \code{tex}
environment. It is equivalent to \code{iftex}, but also turns on all
ten special \latex characters.
  
\subsection{Extending Hyperlatex}
\label{sec:extending}

Hyperlatex scans the preamble of your document for \+\usepackage+
declarations. For every such declaration it tries to find an extension
that defines the \latex macros from that package. These extensions are
Emacs lisp files residing in either a system-wide Hyperlatex extension
directory (set up by the person who installed Hyperlatex), or in your
personal directory \file{.hyperlatex} in your home directory.

Currently, Hyperlatex is distributed with only one extension file, for
the package \+longtable+ (and even that is minimal, just to show how
to write an extension file). Support for other packages will be added
when users need it and cannot create the extension files themselves.

To write a general Hyperlatex extension file, you would have to
program in Emacs lisp. In many cases, however, it is sufficient to
define a number of commands in terms of the basic Hyperlatex commands
(just like \+\newcommand+ or \+\newenvironment+ do). This can be done
quite easily, and you can create you own extension files on the model
of the file \file{hlx-longtable.el} provided with hyperlatex.
(Note that Hyperlatex looks for support for the package \+package+ in
a file \file{hlx-package.el}. That file should bind the function
\+hyperlatex-package-hook+.)

If you write a support file for one of the standard \latex packages,
please share it with us.


\subsubsection{Supported packages}
\label{sec:support}

Packages for which support is available:

\paragraph{Xspace}
\cindex[xspace]{\xspace}
Support for the \+xspace+ package is already built into
Hyperlatex. The macro \+\xspace+ works as it does in \latex. But
remember that you should leave a space behind it if you use it in an
expansion string:
\begin{verbatim}
   \newcommand{\Html}{\textsc{Html}\xspace }
   \newcommand{\latex}{\LaTeX\xspace }
\end{verbatim}


\paragraph{Longtable}
\cindex[longtable]{\+longtable+ environment}

The \+longtable+ environment allows for tables that are split over
multiple pages. In \Html, obviously splitting is unnecessary, so
Hyperlatex treats a \+longtable+ environment identical to a \+tabular+
environment. You can use \+\label+ and \+\link+ inside a \+longtable+
environment to create crossreferences between entries.

The macros \+\endhead+ and \+\endfirsthead+ are ignored by
Hyperlatex. All other macros defined by the package have to be
escaped.

\begin{ifset}{LongTableAvailable}
  Here is an example:
  \T\setlongtables
  \W\begin{center}
    \begin{longtable}[c]{|cl|}
      \hline
      \multicolumn{2}{|c|}{Language Codes (ISO 639:1988)} \\
      code & language \\ \hline
      \endfirsthead
      \T\hline
      \T\multicolumn{2}{|l|}{\small continued from prev.\ page}\\ \hline
      \T code & language \\ \hline
      \T\endhead
      \T\hline\multicolumn{2}{|r|}{\small continued on next page}\\ \hline
      \T\endfoot
      \T\hline
      \T\endlastfoot
      \texttt{aa} & Afar \\
      \texttt{am} & Amharic \\
      \texttt{ay} & Aymara \\
      \texttt{ba} & Bashkir \\
      \texttt{bh} & Bihari \\
      \texttt{bo} & Tibetan \\
      \texttt{ca} & Catalan \\
      \texttt{cy} & Welch
    \end{longtable}
  \W\end{center}
\end{ifset}

\xname{hyperlatex_upgrade}
\section{Upgrading from Hyperlatex~1.3}
\label{sec:upgrading}

If you have used Hyperlatex~1.3 before, then you may be surprised by
this new version of Hyperlatex. A number of things have changed in an
incompatible way. In this section we'll go through them to make the
transition easier. (See \link{below}{easy-transition} for an easy way
to use your old input files with Hyperlatex~1.4.)

You may wonder why those incompatible changes were made. The reason is
that I wrote the first version of Hyperlatex purely for personal use
(to write the Ipe manual), and didn't spent much care on some design
decisions that were not important for my application.  In particular,
there were a few ideosyncrasies that stem from Hyperlatex's
origin in the Emacs \latexinfo package. As there seem to be more and
more Hyperlatex users all over the world, I decided that it was time
to do things properly. I realize that this is a burden to everyone who
is already using Hyperlatex~1.3, but think of the new users who will
find Hyperlatex~1.4 so much more familiar and consistent.

\begin{enumerate}
\item In Hyperlatex~1.4 all \link{ten special
    characters}{sec:special-characters} of \latex are recognized, and
  have their usual function. However, Hyperlatex now offers the
  command \link{\code{\=NotSpecial}}{not-special} that allows you to
  turn off a special character, if you use it very often.

  The treatment of special characters was really a historic relict
  from the \latexinfo macros that I used to write Hyperlatex.
  \latexinfo has only three special characters, namely \verb+\+,
  \verb+{+, and \verb+}+.  (\latexinfo is mainly used for software
  documentation, where one often has to use these characters without
  there special meaning, and since there is no math mode in info
  files, most of them are useless anyway.)

\item A line that should be ignored in the \dvi output has to be
  prefixed with \+\W+ (instead of \+\H+).

  The old command \+\H+ redefined the \latex command for the Hungarian
  accent. This was really an oversight, as this manual even
  \link{shows an example}{hungarian} using that accent!
  
\item The old Hyperlatex commands \verb-\+-, \+\=+, \+\S+, \+\C+,
  \+\minus+, \+\sim+ \ldots{} are no longer recognized by
  Hyperlatex~1.4.

  It feels wrong to deviate from \latex without any reason. You can
  easily define these commands yourself, if you use them (see below).
    
\item The \+\htmlmathitalics+ command has disappeared (it's now the
  default, unless we use the \emph{real} math mode of \Html3).
  
\item Within the \code{example} environment, only the four
  characters \+%+, \+\+, \+{+, and \+}+ are special.

  In Hyperlatex~1.3, the \+~+ was special as well, to be more
  consistent. The new behavior seems more consistent with having ten
  special characters.
  
\item The \+\set+ and \+\clear+ commands have been removed, and their
  function has been \link{taken over}{sec:flags} by
  \+\newcommand+\texonly{, see Section~\Ref}.

\item So far we have only been talking about things that may be a
  burden when migrating to Hyperlatex~1.4.  Here are some new features
  that may compensate you for your troubles:
  \begin{menu}
  \item The \link{starred versions}{link} of \+\link*+ and \+\xlink*+.
  \item The command \link{\code{\=texorhtml}}{texorhtml}.
  \item It was difficult to start an \Html node without a heading, or
    with a bitmap before the heading. This is now
    \link{possible}{sec:sectioning} in a clean way.
  \item The new \link{math mode support}{sec:math}.
  \item \link{Footnotes}{sec:footnotes} are implemented.
  \item Support for \Html \link{tables}{sec:tabular}.
  \item You can select the \link{\Html level}{sec:html-level} that you
    want to generate.
  \item Lots of possibilities for customization.
  \end{menu}
\end{enumerate}

\label{easy-transition}
Most of your files that you used to process with Hyperlatex~1.3 will
probably not work with Hyperlatex~1.4 immediately. To make the
transition easier, you can include the following declarations in the
preamble of your document, just after the \+\documentclass+ and all
the \+\usepackage+ declarations. These declarations will make
Hyperlatex~1.4 behave very much like Hyperlatex~1.3---only five
special characters, the control sequences \+\C+, \+\H+, and \+\S+,
\+\set+ and \+\clear+ are defined, and so are the small commands that
have disappeared in Hyperlatex~1.4.  If you need only some features of
Hyperlatex~1.3, pick them and copy them into your preamble.
\begin{quotation}\T\small
\begin{verbatim}

%% In Hyperlatex 1.3, ^ _ & $ # were not special
\NotSpecial{\do\^\do\_\do\&\do\$\do\#}

%% commands that have disappeared
\newcommand{\scap}{\textsc}
\newcommand{\italic}{\textit}
\newcommand{\bold}{\textbf}
\newcommand{\typew}{\texttt}
\newcommand{\dmn}[1]{#1}
\newcommand{\minus}{$-$}
\newcommand{\htmlmathitalics}{}

%% redefinition of Latex \sim, \+, \=
\W\newcommand{\sim}{\~{}}
\let\TexSim=\sim
\T\newcommand{\sim}{\ifmmode\TexSim\else\~{}\fi}
\newcommand{\+}{\verb+}
\renewcommand{\=}{\back{}}

%% \C for comments
\W\newcommand{\C}{%}
\T\newcommand{\C}{\W}

%% \S to separate cells in tabular environment
\newcommand{\S}{\htmltab}

%% \H for Html mode
\T\let\H=\W
\W\newcommand{\H}{}

%% \set and \clear
\W\newcommand{\set}[1]{\renewcommand{\#1}{1}}
\W\newcommand{\clear}[1]{\renewcommand{\#1}{0}}
\T\newcommand{\set}[1]{\expandafter\def\csname#1\endcsname{1}}
\T\newcommand{\clear}[1]{\expandafter\def\csname#1\endcsname{0}}
\end{verbatim}
\end{quotation}
There is only one problem that you have to fix by hand: Hyperlatex~1.4
will not recognize a \+\set+ or \+\clear+ command in the preamble. You
have to either move those commands into the main document
(\emph{after} the \+\topnode+ declaration) or else you have to convert
them into the new \+\newcommand+ format.

You should also check whether your file contains a \+\C+ at the end of
a line (with only white space behind it). Replace it with \kbd{\%}
(the definition of \+\C+ doesn't work properly in that situation,
because Hyperlatex doesn't tokenize its input).

\xname{hyperlatex_changes}
\section{Changes since Hyperlatex~1.0}
\label{sec:changes}

\paragraph{Changes from~1.0 to~1.1}
\begin{menu}
\item
  The only change that introduces a real incompatibility concerns
  the percent sign \kbd{\%}. It has its usual \LaTeX-meaning of
  introducing a comment in Hyperlatex~1.1, but was not special in
  Hyperlatex~1.0.
\item
  Fixed a bug that made Hyperlatex swallow certain \textsc{iso}
  characters embedded in the text.
\item
  Fixed \Html tags generated for labels such that they can be
  parsed by \code{lynx}.
\item
  The commands \link{\code{\=+\var{verb}+}}{verbatim} and
  \code{\==} are now shortcuts for
  \verb-\verb+-\var{verb}\verb-+- and \+\back+.
\xlabel{external_labels}
\item
  It is now possible to place labels that can be accessed from the
  outside of the document using \link{\code{\=xname}}{xname} and
  \link{\code{\=xlabel}}{xlabel}.
\item
  The navigation panels can now be suppressed using
  \link{\code{\=htmlpanel}}{sec:navigation}.
\item
  If you are using \LaTeXe, the Hyperlatex input
    mode is now turned on at \+\begin{document}+. For
  \LaTeX2.09 it is still turned on by \+\topnode+.
\item
  The environment \link{\code{gif}}{sec:gif} can now be used to turn
  \dvi information into a bitmap that is included in the
  \Html-document.
\end{menu}

\paragraph{Changes from~1.1 to~1.2}
Hyperlatex~1.2 has a few new options that allow you to better use the
extended \Html tags of the \code{netscape} browser.
\begin{menu}
\item \link{\code{\=htmlrule}}{htmlrule} now has an optional argument.
\item The optional argument for the \link{\code{\=htmlimage}}{htmlimage}
  command and the \link{\code{gif} environment}{sec:gif} has been
  extended.
\item The \link{\code{center} environment}{sec:displays} now uses the
  \emph{center} \Html tag understood by some browsers.
\item The \link{font changing commands}{font-changes} have been
  changed to adhere to \LaTeXe. The \link{font size}{sec:type-size} can be
  changed now as well, using the usual \latex commands.
\end{menu}

\paragraph{Changes from~1.2  to~1.3}
Hyperlatex~1.3 fixes a few bugs.

\paragraph{Changes from~1.3  to~1.4}
Hyperlatex~1.4 introduces some incompatible changes, in particular the
ten special characters. There is support for a number of
\Html3 features.
\begin{menu}
\item All ten special \latex characters are now also special in
  Hyperlatex. However, the \+\NotSpecial+ command can be used to make
  characters non-special. 
\item Some non-standard-\latex commands (such as \+\H+, \verb-\+-,
  \+\=+, \+\S+, \+\C+, \+\minus+) are no longer recognized by
  Hyperlatex to be more like standard Latex.
\item The \+\htmlmathitalics+ command has disappeared (it's now the
  default, unless we use \texttt{<math>} tags.)
\item Within the \code{example} environment, only the four
  characters \+%+, \+\+, \+{+, and \+}+ are special now.
\item Added the starred versions of \+\link*+ and \+\xlink*+.
\item Added \+\texorhtml+.
\item The \+\set+ and \+\clear+ commands have been removed, and their
  function has been taken over by \+\newcommand+.
\item Added \+\htmlheading+, and the possibility of leaving section
  headings empty in \Html.
\item Added math mode support.
\item Added tables using the \texttt{<table>} tag.
\item \ldots and many other things. Please read
  \link*{this}[Section~\Ref]{sec:upgrading} if you have used
  Hyperlatex~1.3 before.
\end{menu}

\section{Acknowledgments}
\label{sec:acknowledgments}

Thanks to everybody who reported bugs or who suggested useful new
features. This includes Arne Helme, Bob Kanefsky, Greg Franks, Jim
Donnelly, Jon Brinkmann, Nick Galbreath, Piet van Oostrum,
Robert M. Gray, Peter D. Mosses, Chris George, Barbara Beeton,
Ajay Shah, and Erick Branderhorst.

\xname{hyperlatex_copyright}
\section{Copyright}
\label{sec:copyright}

Hyperlatex is ``free,'' this means that everyone is free to use it and
free to redistribute it on certain conditions. Hyperlatex is not in
the public domain; it is copyrighted and there are restrictions on its
distribution as follows:
  
Copyright \copyright{} 1994,1995 Otfried Schwarzkopf
  
This program is free software; you can redistribute it and/or modify
it under the terms of the \textsc{Gnu} General Public License as published by
the Free Software Foundation; either version 2 of the License, or (at
your option) any later version.
     
This program is distributed in the hope that it will be useful, but
\emph{without any warranty}; without even the implied warranty of
\emph{merchantability} or \emph{fitness for a particular purpose}.
See the \xlink{\textsc{Gnu} General Public
  License}{file://localhost/usr/doc/copyright/GPL} for
more details.
\begin{iftex}
  A copy of the \textsc{Gnu} General Public License is available on
  every Debian \textsc{Gnu}/Linux
  system.\footnote{file://localhost/usr/doc/copyright/GPL} You can
  also obtain it by writing to the Free Software Foundation, Inc., 675
  Mass Ave, Cambridge, MA 02139, USA.
\end{iftex}


\begin{thebibliography}{99}
\bibitem{latex-book}
  Leslie Lamport, \cit{\LaTeX: A Document Preparation System,}
  Second Edition, Addison-Wesley, 1994.
\end{thebibliography}

\htmlpanel{0}
\W \section*{Index}
\label{sec:index}
\htmlprintindex
\T %%
%% LaTeX style to handle hyperlatex files, version 1.4
%%
%% suitable for both Latex2.09 and Latex2e
%%
%%   $Modified: Tue Dec  5 18:47:06 1995 by otfried $
%%
%% This code has GNU copyleft, 1994,1995 Otfried Schwarzkopf
%%
\newif\if@hyla@ldLatex
%%
\@ifundefined{NeedsTeXFormat}{
  % LaTeX 2.09 
  \typeout{Package: 'hyperlatex' v1.4  Otfried Schwarzkopf}
  \@hyla@ldLatextrue
  \input{verbatim.sty}
  }{
  % LaTeX2e
  \NeedsTeXFormat{LaTeX2e}
  
  \ProvidesPackage{hyperlatex}
  [1995/10/01 v1.4 LaTeX2e package for Hyperlatex mode]
  \typeout{Package: 'hyperlatex' v1.4  Otfried Schwarzkopf}
  \@hyla@ldLatexfalse
  \RequirePackage{verbatim}
  }

\chardef\other=12

%%
%% Comments,  
%%
{\obeylines\gdef\hyla@W#1^^M{\endgroup\ignorespaces}}
\def\W{\begingroup\obeylines\catcode`\{=\other\catcode`\}=\other\hyla@W}

\newcommand{\htmlonly}[1]{}
\newcommand{\texorhtml}[2]{#1}
\newenvironment{iftex}{}{}
\newcommand{\texonly}[1]{#1}
\newcommand{\T}{}

%%
%% Treatment of special characters
%%

\def\hyla@nonspecials{}
\def\NotSpecial{\def\hyla@nonspecials}
\def\hyla@turnon{\let\do=\@makeother\hyla@nonspecials}

\if@hyla@ldLatex
\newcommand{\topnode}[1]{\hyla@turnon}
\else
\newcommand{\topnode}[1]{}
\AtBeginDocument{\hyla@turnon}
\fi

\let\htmltab=&

%%
%% \begin{tex} ... \end{tex}    escapes into raw Tex temporarily.
%% you can write {\tex .....} as well, if already escaped from Html
%%
\newenvironment{tex}{\catcode `\$=3 \catcode `\&=4 \catcode `\#=6
\catcode `\^=7 \catcode `\_=8 \catcode `\%=14}{}

%%
%% \back \LaTeXe
%% 

\newcommand{\back}{{\tt\char`\\}}

\if@hyla@ldLatex
\newcommand{\LaTeXe}{\LaTeX2$\epsilon$}
\fi

%%
%% \math
%%

\def\math{\@ifnextchar [{\@math}{\@math[]}}
\def\@math[#1]{\begingroup\tex\@@math}
\def\@@math#1{$#1$\endgroup}

%%
%% Commands that don't do anything interesting in Latex
%%

\def\hyla@pass{\begingroup\let\protect\@unexpandable@noexpand\@sanitize}

\newcommand{\htmldirectory}[1]{\gdef\gif@directory{#1}}
\newcommand{\htmlname}[1]{}
\newcommand{\htmldepth}[1]{}
\newcommand{\htmltitle}[1]{}
\newcommand{\htmladdress}[1]{}
\newcommand{\html}{\hyla@pass\@html}
\def\@html#1{\endgroup}

\newcommand{\htmlsym}[1]{}
\def\htmlrule{\@ifnextchar [{\@htmlrule}{\@htmlrule[]}}
\def\@htmlrule{\hyla@pass\@@htmlrule}
\def\@@htmlrule[#1]{\endgroup}

\newcommand{\htmllevel}[1]{}
\newcommand{\htmlmenu}[1]{}
\newcommand{\htmlautomenu}[1]{}
\newcommand{\htmlprintindex}{}
\newcommand{\htmlfootnotes}{}

\def\htmlimage{\@ifnextchar [{\@htmlimage}{\@htmlimage[]}}
\def\@htmlimage{\hyla@pass\@@htmlimage}
\def\@@htmlimage[#1]#2{\endgroup}

\newcommand{\xlabel}{\hyla@pass\hyla@ignore}
\newcommand{\xname}{\hyla@pass\hyla@ignore}
\def\hyla@ignore#1{\endgroup}

\newcommand{\htmlicons}[1]{}
\newcommand{\htmlpanel}[1]{}
\newcommand{\htmlheading}{\@ifnextchar[{\@htmlheading}{\@htmlheading[]}}
\def\@htmlheading[#1]#2{}

\newcommand{\htmlcaption}[1]{}

\def\htmlattributes{\@ifstar{\@htmlattributes}{\@htmlattributes}}
\def\@htmlattributes{\hyla@pass\@@htmlattributes}
\def\@@htmlattributes#1#2{\endgroup}

%%
%% GIF environment,
%%  will generate bitmaps when `\makegifs' is defined
%%
\newif\if@makegifs
\@ifundefined{makegifs}{\@makegifsfalse}{\@makegifstrue}

\def\gif{\@ifnextchar[{\@gif}{\@gif[b]}}
\def\@gif[#1]{\@ifnextchar[{\@@gif}{\@@gif[100]}}
\def\@@gif[#1]{\gdef\gif@resolution{#1}\@ifnextchar[{\@@@gif}{\@@@gif[300]}}
\def\@@@gif[#1]#2{\gdef\gif@dpi{#1}\gdef\gif@name{#2}\@@@@gif}

\if@makegifs
\typeout{**********************************************}
\typeout{* Making GIF bitmaps from Hyperlatex source! *}
\typeout{**********************************************}
\newwrite\@makegifcmds
\immediate\openout\@makegifcmds=\jobname.makegif
\newbox\@gifbox
\newcount\@gifcount\@gifcount=10000
\def\@@@@gif{\setbox\@gifbox=\vbox\bgroup\tex}
\def\endgif{\egroup
  {\global\advance\@gifcount by 1\count0=\@gifcount
    \immediate\write\@makegifcmds{dvips\space -f\space -p\space
      \the\@gifcount\space -n\space 1\space -D\space \gif@dpi\space
      \jobname.dvi\space>\space\gif@name.ps^^J%
      ps2gif\space -res\space \gif@resolution\space \gif@name.ps^^J%
      mv\space \gif@name.gif\space \gif@directory^^J}
    \shipout\copy\@gifbox}\unvbox\@gifbox}
\else
\def\@@@@gif{\tex}
\def\endgif{}
\fi
%%
%% Font style definitions
%%

\if@hyla@ldLatex
\newcommand{\emph}[1]{{\it #1}}
\newcommand{\textsl}[1]{{\sl #1}}
\newcommand{\textit}[1]{{\it #1}}
\newcommand{\textbf}[1]{{\bf #1}}
\newcommand{\texttt}[1]{{\tt #1}}
\newcommand{\textsc}[1]{{\sc #1}}
\fi

\let\cit=\textit
\let\code=\texttt
\let\kbd=\texttt
\let\samp=\texttt
\let\strong=\textbf
\let\var=\textsl
\let\dfn=\textit
\let\file=\textit

%%
%% \begin{example} ... \end{example} obeys spaces and lines
%%
%% the indent can be controlled by \exampleindent
%%
\newdimen\exampleindent
\setlength{\exampleindent}{7mm}
\def\hyla@example{\do\$\do\&\do\#\do\^\do\_\do\~}
{\obeyspaces
  \gdef\turnon@spaces{\let =\ \obeyspaces\catcode``=\active\@noligs}}
{\obeylines%
  \gdef\turnon@lines{\obeylines\def^^M{\par\def^^M{\leavevmode\par}}}}

\def\example{\list{}{\leftmargin\exampleindent
    \itemindent\z@ \rightmargin\z@ \parsep \z@ plus\p@}\item[]\tt
  \turnon@spaces\turnon@lines\let\do\@makeother\hyla@example}
\let\endexample=\endlist

\newenvironment{menu}{\list{$\bullet$}{\itemsep0pt\parsep0pt}}{\endlist}
\newenvironment{ifhtml}{\comment}{\endcomment}

\def\hyla@empty{}
\def\hyla@zero{0}
\newif\if@hyla

\def\hyla@setclear#1{\@ifundefined{#1}{\@hylafalse}{%
    \expandafter\ifx\csname#1\endcsname\hyla@empty\@hylafalse
    \else\expandafter\ifx\csname#1\endcsname\hyla@zero\@hylafalse
    \else\@hylatrue\fi\fi}}

\def\ifclear#1{\hyla@setclear{#1}
  \if@hyla
  %% arg empty -> skip
  \let\endifset\endcomment\let\hyla@comment\comment
  \else
  %% arg set -> ignore begin and end
  \let\endifset\relax\let\hyla@comment\relax
  \fi\hyla@comment}

\def\ifset#1{\hyla@setclear{#1}
  \if@hyla
  % arg empty -> ignore begin and end
  \let\endifclear\relax\let\hyla@comment\relax
  \else
  % arg set  -> skip environment
  \let\endifclear\endcomment\let\hyla@comment\comment
  \fi\hyla@comment}

%%
%% Define \link and \xlink macros
%%
\newcommand{\hyla@label}{}

\newcommand{\Ref}{\ref{\hyla@label}}
\newcommand{\Pageref}{\pageref{\hyla@label}}
\newcommand{\Cite}{\cite{\hyla@label}}

\newcommand{\htmlcite}[1]{\cite{#1}}

\def\link{\@ifstar{\@star@link}{\@@link}}
\def\@@link#1{#1\@@@link}
\def\@star@link#1{\@@@link}
\def\@@@link{\@ifnextchar [{\@link}% ] balance
  {\@link[]}}
\def\@link[#1]#2{\gdef\hyla@label{#2}#1}

\def\xlink{\@ifstar{\@star@xlink}{\@@xlink}}
\def\@@xlink#1{\@@@xlink{#1}}
\def\@star@xlink#1{\@@@xlink{}}
\def\@@@xlink#1{\@ifnextchar [{\@xlink{#1}}{\@xlink{#1}[]}}
\def\@xlink#1[#2]#3{\formatxlink{#1#2}{#3}}
\newcommand{\formatxlink}[2]{#1}

%
% index from latex.tex, and changed to include optional argument
%
\def\cindex{\@bsphack\begingroup
  \let\protect\@unexpandable@noexpand
  \@sanitize
  \@ifnextchar [{\hyla@argwrindex}% balance ]
  {\@wrindex}}
\def\hyla@argwrindex[#1]#2{\@wrindex{#1@#2}}

%% end of hyperlatex.sty


\htmlfootnotes

\end{document}

