%% BEGIN gradient.tex/gradient.sty
%%
%% Defines the gradient fillstyle for PSTricks.
%%
\def\fileversion{0.93a}
\def\filedate{93/03/12}
%%
%% COPYRIGHT 1993, by Timothy Van Zandt, tvz@Princeton.EDU
%% See pstricks.tex or pstricks.doc for copying restrictions.
%%
%% See the PSTricks User's Guide for description.
%%
%% This uses the header file gradient.ps.
%%
%% Based on some EPS files by leeweyr!bill@nuchat.sccsi.com (W. R. Lee).
%
% This defines a new fill style, "gradient", for use with PSTricks,
% which has gradiated color. The following parameters are used:
%
%    gradbegin=color    : Beginning color.
%    gradend=color      : Final color.
%    gradlines=int      : Number of lines to use. The higher the number,
%                           the slower the rendering.
%    gradmidpoint=num   : Gradient color goes from gradbegin to gradend,
%                           and then back to beginning. Midpoint (point
%                           where "gradend" color appears, is gradmidpoint
%                           from the top.  (0 <= Gmidpoint <= 1).
%    gradangle=angle    : Rotate image by angle.

\message{ v\fileversion, \filedate}

\csname GradientLoaded\endcsname
\let\GradientLoaded\endinput

\ifx\PSTricksLoaded\endinput\else
  \def\next{\input pstricks.tex }\expandafter\next
\fi

\edef\TheAtCode{\the\catcode`\@}
\catcode`\@=11

\pstheader{gradient.pro}

\newrgbcolor{gradbegin}{0 .1 .95}
\def\psset@gradbegin#1{\pst@getcolor{#1}\psgradbegin}
\psset@gradbegin{gradbegin}

\newrgbcolor{gradend}{0 1 1}
\def\psset@gradend#1{\pst@getcolor{#1}\psgradend}
\psset@gradend{gradend}

\def\psset@gradlines#1{\pst@getint{#1}\psgradlines}
\psset@gradlines{500}

\def\psset@gradmidpoint#1{\pst@checknum{#1}\psgradmidpoint}
\psset@gradmidpoint{.9}

\def\psset@gradangle#1{\pst@getangle{#1}\psk@gradangle}
\psset@gradangle{0}

\def\psfs@gradient{%
  gsave \pst@usecolor\psgradbegin currentrgbcolor grestore
  gsave \pst@usecolor\psgradend currentrgbcolor grestore
  \psgradlines
  \psgradmidpoint
  \psk@gradangle
  tx@GradientDict begin GradientFill end}

\catcode`\@=\TheAtCode\relax
\endinput
%% END gradient.tex
