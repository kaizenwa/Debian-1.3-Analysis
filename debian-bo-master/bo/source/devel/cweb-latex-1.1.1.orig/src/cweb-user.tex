% $DocId: cweb-user.tex,v 2.10 1995/11/20 22:34:16 schrod Exp $
%------------------------------------------------------------

%
% user manual/requirement definition for cweb class
%
% [LaTeX]
% (history at end)


% This file is either a subdocument of the style doc or a document on
% its own. In the former case it's a chapter, in the latter it's a
% ``normal'' LaTeX progltx document.
%     If it's a subdocument, this file will be included after
% \begin{document}. We can detect this: \document redefines
% \documentclass to be \@twoclasseserror. Then we also have to define
% how this document is ended: Either by \endinput or by an additional
% revision log.
% Of course, this test works only if LaTeX 2e is used for processing.

\expandafter\ifx \csname @twoclasseserror\endcsname \documentclass

        \let\endSubDocument=\endinput

        \chap What's this class for?.

\else

        \let\endSubDocument=\relax

        \documentclass{progltx}

        \usepackage{cweb-doc}           % additional document-specific markup
        \usepackage{a4-9}               % Tschichold's A4 layout


        \begin{document}

        \title{The \texttt{cweb} Class}
        \author{Joachim Schrod%
            \thanks{Email: \texttt{jschrod@acm.org}}%
            }

        \RCS $DocDate: 1995/11/20 22:34:16 $
        \date{%
            \RCSDocDate\\[3pt] % LaTeX Error: Paragraph terminated too early
            (Revision \RCSStyleRevision{} of \texttt{cweb.cls})%
            }

        \maketitle

        \sect

\fi

\cweb{} is a Literate Programming tool that enables you to combine
documentation and actual C or \C++ code in one document. The \cls{cweb}
class allows you to use \LaTeX{} for the documentation of your \cweb{}
programs. Now a \cweb{} document is like a usual \LaTeX{}
document---as you have probably written dozens before.  With all the
features of \LaTeX{}, and with all the quirks of \LaTeX{}.

Originally, \cweb{} was only usable with Plain \TeX{}, as \cweave{}
outputs \TeX{} code that looks as if it were Plain \TeX{}. Well, it
isn't---it's just not the normal \LaTeX{} markup. No change in
\cweave{} is necessary to use \LaTeX{}.

For the sake of those \cweb{} programmers who have used the original
markup interface (i.e., the Plain \TeX{} macros) before, I'll
sometimes contrast the approach presented here with the original one.
Then I'll call the original interface ``Plain \cweb{}''. The new
interface is called ``\LaTeX{} \cweb{}''.

If you never used \cweb{} before, you won't find enough information in
this manual. Please read the user manual of \cweb{} first, and then
come back again. I assume a working knowledge, and will only present
the new stuff introduced by this class.


\sect This bundle does \emph{not} support inclusion of \cweb{} code in
other (general) \LaTeX{} documents. This is not the task of the
\cls{cweb} class -- its task is the usage of \LaTeX{} for creation of
\cweb{} documents.

Note also that you will not be able to \LaTeX{} your \cweb{} documents
(i.e., the program files) directly. You must still process them first
by \cweave{}, then you can use \LaTeX{} to process the resulting
\TeX{} file.


\sect You have to use \cweb{} version~3 and you need at least \LaTeX{}
version \mbox{$\langle$1994/12/01$\rangle$} (or any later one), this
class does not work with older versions.


\sect A \cweb{} document has two concurrent structures, a
\textsl{document structure} and a \textsl{program structure}.

The basic building block of the document structure, the bricks that
build up our document, are called \textsl{chunks}. We use chunks to
present one point of our work to our reader, the presentation has both
informal (explanation) and formal parts (code). This is the essence of
Literate Programming, that we tell each point twice, once on an
abstract and once on a technical level.

A chunk consists of three parts: (1)~the \textsl{documentation part},
(2)~the \textsl{definition part}, and (3)~the \textsl{program part}.
Each of these parts can be empty. The documentation part is mostly
text with \LaTeX{} tags. In this text material from \textsl{restricted
program mode} can appear. The definition part consists of a series of
either \textsl{macro} or \textsl{format definitions}. The program part
is one piece of a refinement, identified by a name (see below).

A chunk starts with `\verb*|@ |', and gets assigned a unique
identification, a number; we'll use that number to refer to that
chunk. In particular, the number is of interest when we want to refer
to the code part of the chunk, otherwise it's often regarded as
superfluous.

\sect And we have higher-level structure elements as well. They combine a
sequence of chunks and a sequence of structure elements on a lower
level, both sequences may be empty. As usual in \LaTeX{}, we call
these elements \textsl{section divisions} or short \textsl{sections}.
We do not give them names
like `part,' `chapter,' `subsection,' etc., to denote their level;
most often we'll use level-numbers (sometimes called \textit{rank})
instead.

A section division starts with `|@*|' and is followed by a character, the rank
indicator. The highest rank is~`|@**|', you can think of it as an
equivalence to \LaTeX{}'s |\part|. (Often we don't use that rank in
smaller programs, as we don't use |\part| in smaller articles.) A
section on the next rank is started with~`\verb*|@* |' (synonym to
`|@*0|'), roughly equivalent to chapters.  Then we'll use numbers to
mark the next ranks, from |@*1| to~|@*9| (sections to
sub$^{9}$sections). That's outrageous---I'll hope you'll never use
11~hierarchy levels in your documents, they would be unreadable.

A section has a title, that title comes after the section start tag
and ends with a full stop. If you need periods in your title, enclose
them in braces. (That behavior can be changed if there's enough
interest. I can also make it possible to use braces to delimit the
section title, as in |\section|, etc. Send mail with your opinion.)

As a matter of convenience, a section implicitly starts a chunk;
i.e., one does not need to use `\verb*|@ |' after the section title.
Of course, it doesn't hurt to use it either.

\medskip

The explanation might have been a bit too abstract, so let me make one
thing plain clear: You should \emph{never ever} use |\part|,
|\chapter|, |\section|, or any other \LaTeX{} sectioning command in
your \cweb{} documents. Use the sectioning commands of \cweb{}, i.e.,
`|@*|\textit{rank}'.



\sect The description above does not mean that the output is fixed to
the style you'll know from the \cls{book} or \cls{report} document
classes. Quite in contrary, the default layout is oriented along the
\cls{article} class.

There is a very easy possibility to change that default layout: With
the option |baseclass| (see chunk~\ref{chunk:baseclass}) you can
specify the document class \cls{cweb} inherits from. If that document
class has chapters (like \cls{report} or \cls{book}), you'll get the
layout as you're used from these classes.


\sect In Plain \cweb{}, the term \textit{section} was used for
`chunk.' We abstain from that usage, as sections are used in \LaTeX{}
(and in almost all other documentation systems, for that matter) to
designate higher-level structure elements. We prefer to follow the
traditional usage of terms here, even if that means that we'll have to
confront Plain \cweb{} users with a slight shift in semantics.

And you'll find the term \textit{starred section} in Plain \cweb{} and
might think these are our sections. Well, almost. There's a subtle,
but important difference between these starred sections and our
terminology: A starred section is just a chunk with a special
attribute (a title). That title will appear in a table of contents,
but that does not mean that the starred section is an element to
denote a structural \emph{hierarchy}.

You won't find a section number like `1.5.2' in Plain \cweb{}, do you?
--- In \LaTeX{} \cweb{}, you've got the choice. You have the new
support of the hierarchical structure, or you may stay with Plain
\cweb{}'s (flat) structure.


\sect We still have one open point in our explanation of the
document's structure. As mentioned at the start, we've got two
concurrent structures, let's have a look at the program structure
finally.

A \cweb{} program consists of a tree of \textsl{refinements}. A
refinement is a sequence of program parts with the same name, ordered
in appearance in the document. The root of the tree is the refinement
with the special name~|@c|.

The complete program text is defined by the depth-first traversal
(DFS, i.e.\ infix-order) of the tree, the tool \ctangle{} extracts the
program text from a \cweb{} document.




\chap Markup for your document structure.

Like every \LaTeX{} document, a \cweb{} document starts with a
document class specification; then comes the preamble, terminated by
the document start. This way we brought some structure in the \cweb{}
limbo, but this shouldn't be a problem for you. You must tag the end
of the document, as you do in other \LaTeX{} files. But here you must
take a bit more care:
%
\begin{quote}
  \itshape

  Assert that the\/ |\end{document}| is in the documentation part of
  a\/ chunk, neither in the definition nor in the program part.

\end{quote}
%
If it would be in the program part, \cweave{} would readily process it
and \LaTeX{} would never see this tag. Nevertheless, if you make this
mistake, you will detect it early: \ctangle{} will copy this tag to
the C file as well, and your program will be erroneous\,\dots

And there's another difference you'll have to take care of: Usually
you can put arbitrary text after the end of your document, \LaTeX{}
will not see it. Please note that both \ctangle{} and \cweave{} will
still see it---don't put a program part or even complete chunks there.
For technical reasons, there must not be any |\fi| token after the
document end, too.


\sect The resulting structure of your \cweb{} document is exemplified
in figure~\ref{fig:structure}. Note that |\end{document}| is placed
in a new, empty, chunk. This isn't necessary, you could write more
documentation there---but it's considered good style to use a sole
chunk for document finish.

%% my little joke
\begin{figure}

\setbox0=\vbox{\parindent=0pt \hsize=.85\textwidth
\begin{minipage}{\hsize}
\begin{verbatim}
\documentclass{cweb}

\begin{document}

\title{My Program, Doomed for the ACM Software Systems Award}
\author{Joe L. User}

\maketitle

\tableofcontents        % if you want


@* A PSPACE solution for the Traveling Salesman.

< insert your program here >



@
\end{document}
\end{verbatim}
\end{minipage}
}

\begin{center}
\fboxsep=1em
\fbox{\box0}
\end{center}

\caption{Exemplified \cweb{} document structure}
\label{fig:structure}
\end{figure}


\sect If you don't call \cweave{} with the |-x| option, an identifier
index and a list of all refinements is created at the end of the
document. You can specify an introductionary text for the index with
the tag |\cwebIndexIntro|, the introduction is the argument of this
tag.




\chap Configuration.

You may configure the \cls{cweb} class in several ways, with class
options, redefinitions in the preamble, supplying additional
packages, and by subclassing it.

At start of the processing, the file |cweb.cfg| is read if it exists.
That file may contain configuration code that shall apply to all your
\cweb{} documents. You can set the paper size there, etc. Please don't
set the default structure there (see
chunk~\ref{chunk:structure-option}), others would get different results
when they process your document.


\sect Options are specified with the new ``keyword$=$value''-scheme,
first introduced with the \pkg{graphicx} package. I.e., the options have a
set of possible values, you may choose one. For instance, the option
|structure| has the possible values |flat| and |hierarchic|. You tell
that you want a flat document structure by
%
\begin{quote}
  |\documentclass[structure=flat]{cweb}|
\end{quote}
%
As usual, options are separated by commas. From the set of possible
option values, two ones are special: The \textsl{predefined value} and
the \textsl{default value}. The predefined value is the one that's
used when you don't specify the option at all. The default value is
used of you don't specify the value in the option; e.g.,
%
\begin{quote}
  |\documentclass[suppress]{cweb}|
\end{quote}
%
selects the default value of the option |suppress|. (Not that there is
one currently. :--)

There is always a predefined value, a default value may not be there.
If a default value exists, it is always different from the predefined
one. The default value gives you an easy specification of an option
value that is expected to be common. If that value would be the
predefined one, you won't need to specify the option in the first
hand.

The following options are available and explained below:
%
\begin{options}
\item[structure] Select the structure model used by the \cls{cweb}
  class.
\item[suppress] Suppress output of different document parts.
\item[baseclass] Select the base document class.
\item[language] Select a language for inserted texts.
\end{options}


\sect \label{chunk:structure-option}

We have two possibilities to output chunks and sections, either
as a flat or as a hierarchic structure. This configuration is selected
by the option |structure|, either through the value |flat| or
|hierarchic|.
%
\begin{options}

\item[flat] The flat structure is the ``classic'' approach,
  the way Plain \cweb{} renders its chunks. Then each chunk starts
  with its number. The chunk numbers are output in boldface, followed
  by a dot and a quad.

  Sections show chunk numbers as well (remember, they implicitly
  start a chunk). ``Important'' sections are added to the table of
  contents.

\item[hierarchic] The hierarchic structure is the
  ``modernistic'' (aka \LaTeX{}) approach, then we mark the start of a
  chunk by a start sign. (By default, that's a paragraph sign, `\P').
  Chunk numbers are rendered in the margin of the program part. If
  there's no program part, no chunk number will be rendered---after
  all, they are only used for identification of program parts.

  Section numbers are hierarchic, like in other \LaTeX{} document
  classes.

\end{options}
%
The predefined value of |structure| is |hierarchic|. There doesn't
exist a default value.


\sect The option |suppress| allows to select suppression of different
document parts. You have different parts you can suppress. The value
of the |suppress| option is a comma-separated list of identifiers,
enclosed in braces. If you have only one thing you want to suppress,
you don't need to surround that identifier by braces. Possible
identifiers are:

\begin{options}

\item[changehints] suppress output of hints that a changefile was
  used: The change flag (by default a star at the start of a chunk)
  and the list of changed chunks at the end of the document.

\item[unchanged] suppress output of chunks that are not changed by
  a changefile. This value implies |changehints|---by definition all
  printed chunks are changed, it doesn't make any sense to hint to
  that fact in addition.

\item[index] suppress the identifier index.

\item[reflist] suppress the refinement list.

\item[format] suppress output of `|@f|' directives.

\end{options}

The predefined value is the empty list, i.e., nothing is suppressed.
There is no default value since I don't know which parts are
suppressed most often. Send email with your demands, I'll add a
default if I see that there is one major application of this option.


\sect \label{chunk:baseclass} The option |baseclass| allows to select
the base document class used by the \cls{cweb} class.  In particular,
if the base class has |\chapter| defined the layout of major section
divisions (`|@**|' and `\verb*|@* |') will be changed to be in a
chapter layout.

The predefined value is |article|, the default value is |report|.


\sect The option |language| allows to adapt inserted texts to
different languages. E.g., with the option `|language=german|' German
texts are inserted for cross references, headings, etc. That option
will neither switch on language-specific hyphenation, nor will it
trigger any other processing -- that's the task of Babel or any other
\LaTeX{} language style.

There is neither a predefined nor a default value. (Without the
|language| option, inserted texts are in English.)

That option actually triggers the usage of a package
\pkg{cwbl-\textit{language}}. To see if a language is supported, you
have to look for respective files in the \LaTeX{} \cweb{} macro
directory.%
    \footnote{That's |texmf/tex/latex/cweb|, on a TDS compliant
    installation. Please beware that pseudo operating systems like DOS
    cripple file names and shorten them to eight characters; if you
    see a file |cwbl-ger.sty| you'll know that the option value
    |german| is supported.}
Of course, the easiest way to check the availability is to try it
out -- you'll get a comprehensible error message if there is no
respective \cweb{} language package. In November 1995, |german|,
|french|, and |italian| were supported, more might have been added
later.

\smallskip

For compatibility with Babel, some other options (e.g., |german| or
|french|) are also supported as nicknames for
|language=|\textit{option}. If you use packages that make also use of
these options (e.g., \pkg{babel} or \pkg{varioref}), you'll probably
prefer them over |language| since they are picked up by those packages
as well.





\chap Restrictions.

The following restrictions will not be withdrawn as far as I can
see---except if somebody will send me a patch with changes. I tried to
sort the restrictions in order of severity.


\sect \emph{(Restrictions due to the design of \cweave{}.)}

Please be aware that the vertical bar (`\verb"|"') is used by \cweb{}
to delimit small C code pieces in the documentation parts, and is
therefore processed (and mangled) by \cweave{}. \emph{You cannot use
it for \TeX{} any more.}

In particular, you cannot specify rules for the |tabular| or the
|array| environment. Since you most certainly want to do so: You have
two choices left:
%
\begin{enumerate}

\item Make sure you have the \pkg{array} package (by Frank Mittelbach and
  David Carlisle, from the Tools bundle) installed. Then you may use
  the package \pkg{cwebarray}, it defines `|I|' (that's an uppercase~`i')
  as a specifier for rules. I.e., instead of
  \verb"\begin{tabular}{l|l}" you have to write
  |\begin{tabular}{lIl}|. Not the most elegant solution, but it
  works\,\dots

\item
  Use `|^^7c|' instead of `\verb"|"'. I.e., instead of
  \verb"\begin{tabular}{l|l}" you may write |\begin{tabular}{l^^7cl}|.

\end{enumerate}
%
These two choices are compatible, you may use both in one document.
Needless to say, I consider the first alternative the better one.


\sect Neither a refinement name nor an index entry made by~|@^| may
consist of a \emph{single} dot-accented term. I.e., you must not
write `|@<\.O@>|', `|@^\.o@>|', or even `|@^\.{foolish}@>|'. Of
course you may write `|@^\.o, accent@>|' or `|@< Handle accent \.o @>|'.


\sect \emph{(Restriction due to the implementation of \LaTeX{}.)}

One cannot use restricted C mode in moving arguments. Most notably,
this is disturbing in the section titles and in |\caption|
tags. Sorry, folks.

(Basically, that's because the definition of |\addcontentsline|, etc.,
in the \LaTeX{} kernel is brain damaged; it wants to expand its
argument. And I don't want to maintain redefinitions of such basic
macros.)


\sect \emph{(Restrictions due to \TeX{}.)}

\C++ comments (i.e., from |//| to the end of the line) are typeset as C
comments. This is especially bad if they are used for a whole block of
comment lines, as it is quite common. Please put such comment blocks
in the documentation part. (An |\everyline| implementation would be
needed to lift that restriction, and that's impossible in general.)




\chap Details.

The following tags are reserved and must neither be used nor redefined:
 |\ATL|,
 |\B|,
 |\M|,
 |\N|,
 |\PB|,
 and~|\Y|.

|\9| is already explained in the \cweb{} user manual: It's a special
control sequence used for the index entries tagged with~`|@:|'. Its
default definition is setup in such a way that you can cheat
\cweave{} concerning the sort order of this entry: If you enter
`|@:sort}{print@>|' you will get an index entry ``print'' next to the
place where the index entry ``sort'' would be. But you're allowed to
change this default definition.%
    \footnote{Remember: A package is a very good place to place such
    redefinitions; your document should be concerned with contents, not
    with appearance.}

The names of all other control sequences defined by this
class---besides the common \LaTeX{} control sequences---start with
|cweb|, |Cweb|, or |cwbb|. Please don't define new control sequences starting
with this prefix. (The control sequences starting with |Cweb| may be
redefined in a package to change the appearance of the \cweb{}
document or the behavior in a controlled manner, check the
description of the internal interface for this possibility.)


\sect Plain \cweb{} users
should note that the macros from |cwebmac.tex| are not available any
more. E.g., you cannot use |\.| to typeset typewriter material; use
either |\texttt| or~|\verb|, as it fits the situation. On the other hand,
now you're able to use |\.| for the dot accent, you can define |\3|
for the `sharp~s', |\C++| for the \C++ logo,%
  \footnote{I will not define any such logos in this class. A
  package like \pkg{logos} with lots of name definitions is the
  appropriate place for this.}
etc.

Another detail for ex-plainies: The table of contents is produced by
the |\tableofcontents| tag (during the second \LaTeX{} run), not
automatically. But this is the standard \LaTeX{} behavior.




\chap Known Problems.
\label{sec:problems}

There is a bunch of known bugs, problems, and ommisions.

\bigbreak

\noindent \textit{Bugs:}
 %
\begin{itemize}

\item The presentation of |@l| redefinitions is not proper. But it
wasn't in Plain \cweb{}, either.

\end{itemize}

\noindent \textit{Problems:}
 %
\begin{itemize}

\item You can't use |\index| and |\makeindex| as-is to create an
  index. Add your entries instead to \cweave{}'s index using `|@^|',
  `|@.|', or `|@:|'. That problem is caused by a clash in the usage of
  the filename extension |idx| by \cweave{} and \LaTeX{}. The
  \pkg{index} package by \textsc{David Jones} will help you to define
  additional index categories (that use an other extension) if you
  don't want to mix identifiers and your other index entries.

\end{itemize}

\noindent \textit{Ommisions} (``wouldn't it be nice, if''s):
 %
\begin{itemize}

\item The current chunk number and most section titles (below rank 1
  or~2, depending on the base class) are not available as a mark for
  inclusion in a running head.

\item Better integration of \pkg{rcs} package. That package is used to
  support including information from the revision management system
  RCS into the document. E.g., the revision log at the end of that
  document was produced by that package.

\end{itemize}


\sect I would like to thank those who helped me to improve this bundle.
% In particular, XXX provided XXXsubstantial parts of the code.

\textsc{Michael M\"uller} and \textsc{Zden\v{e}k Wagner} did thorough
checks that helped me to improve the alpha test version.

\textsc{Christian Kumpf} triggered the addition of language support
and pointed out where configuration could be made easier by supplying
more macros in the protected interface. \textsc{Andreas Scherer}
provided packages for French and Italian \cweb{} documents.

\textsc{Laurent Desnogues},
\textsc{Felix G\"artner},
and
\textsc{John S. Robinson}
provided error and problem reports.
\textsc{Bronne Louis} pointed out the embarrassing error that section
numbers were not reset when a higher-level section began.



\endSubDocument


%%%%%%%%%%%%%%%%%%%%%%%%%%%%%%%%%%%%%%%%%%%%%%%%%%%%%%%%%%%%%%%%%%%%%%

\newpage

\begin{rcslog}
$DocLog: cweb-user.tex,v $
\Revision 2.10 1995/11/20 22:34:16 schrod
Add language support.\\
Triggered by Christian Kumpf \path|<smurf@igd.fhg.de>| and Andreas
Scherer \path|<scherer@physik.rwth-aachen.de>|.

Use my `official' (ACM) email address.

\Revision 2.9 1995/11/20 10:39:34 schrod
    Explicate that one must not use |\section| etc., that one cannot
\LaTeX{} the \cweb{} program directly but must run \cweave{} first,
and that the \cls{cweb} class does not provide inclusion of \cweb{}
programs in other \LaTeX{} documents.\\
Clarification triggered by Laurent Desnogues
\path|<laurent.desnogues@aiguemarine.unice.fr>|.

\Revision 2.8 1995/09/15 10:22:58 schrod
The \cls{cweb} class needs at least \LaTeX{} version
\mbox{$\langle$1994/12/01$\rangle$}.\\
Problem reported by John S. Robinson \path|jsrobin@umiacs.umd.edu|.

\Revision 2.7 1995/09/12 22:58:38 schrod
|cwbb| is also a reserved namespace.

Added acknowledgements.

\Revision 2.6 1995/08/29 17:34:14 schrod
    \LaTeX{} \cweb{}, version 1.0.

\Revision 2.5 1995/08/29 02:07:28 schrod
Discard dependencies on 10\,pt fonts.

Support suppression of format directives.

\Revision 2.4 1995/08/27 19:31:11 schrod
Suppression of index and reference list is supported.

Discard documentation on |\cwebSecNoEject|, that's part of the
protected interface.

\Revision 2.3 1995/08/27 17:24:45 schrod
Make usage of baseclass with chapters work.

\Revision 2.2 1995/08/27 13:25:59 schrod
Add possibility to suppress change hints. Suppression of unchanged
chunks suppress change hints as well, they are meaningless as all
printed chunks are changed by definition.

Update problems section, they were partly resolved by the current
changes. Added hint to \pkg{index} package that resolves the problem
of extensions clashes with \MakeIndex{}.

\Revision 2.1 1995/08/25 19:11:17 schrod
Add keyword-value option style, with new \pkg{keyvald} package.

Hierarchic strucutures are supported now, in addition to the flat
structure of the beta-test version. One can choose with an option. For
that step, the terminology was cleaned up, too: Chunks are not named
sections any more. (That change involved reimplementation of almost
all the structure and toc stuff.)

The chunk number supplied by \cweave{} is used now, not some computed
number. Change flags are printed, too.

One can suppress output of unchanged sections.

One can select the baseclass with an option. That may be used to use
\cls{report} or \cls{book} to get chapter-style layout. Of course,
using an arbitrary baseclass is dangerous, it must conform to the
conventions of \LaTeX{} standard classes.

\Revision 1.10 1995/08/08 00:14:28 schrod
Updated to \LaTeXe{}, the |cweb| style is now a document class. Used
my standard templates for that, no changes in functionality.

\Revision 1.9  1993/08/10  14:15:40  schrod
New page on main section only if group level $<$ |\cwebSecNoEject|.
Default for the latter is 3.

Document that logos will not be defined in this style file.

Copy of plain macros for |\CwebNumber| does not work. Repaired the
most important one (subscript must be accessed via |\sb|).
Incompatibility to NFSS will be addressed later.\\
(Problems reported by Zden\v{e}k Wagner \path|<wagner@csearn.bitnet>|.)

\Revision 1.8  1993/08/10  11:21:03  schrod
Reference to section number does not render a period after the
number any more.

\Revision 1.7  1993/08/09  18:05:54  schrod
Mentioned that `\verb"|"' cannot be used for LaTeX purposes, in
particular, not for ruled tables. Described workarounds, one of them
is the new style option |cwebarray|.\\
(Problem reported by Felix G\"artner
\path|<theedge@rbg.informatik.th-darmstadt.de>|.)

\Revision 1.6  1993/06/17  16:10:06  schrod
`|cweb|' was still an option in the example.

\Revision 1.5  1993/06/17  15:09:30  schrod
\cweb{} 3.0 was released officially on June 16, 1993. Mentioned in
the documentation that this version is needed for the |cweb| style.

\Revision 1.4  1993/06/15  15:19:11  schrod
It's a style now, was an option formerly.

\Revision 1.3  1993/06/14  15:40:42  schrod
Added section about restrictions (no restricted C mode material
in moving arguments, no single dot-accented term in an index entry).

\Revision 1.2  1993/05/12  20:23:32  schrod
Adapted to recent changes of \cweave{} (of April 93): Main sections
have a group level, represented in the table of contents.

Boxed the example, it was not legible formerly.

\Revision 1.1  1993/04/09  15:00:37  schrod
Initial revision

\end{rcslog}



\end{document}

% LocalWords:  baseclass DFS PSPACE graphicx changehints changefile reflist idx
% LocalWords:  cwebarray cwebmac plainies ommisions if''s rcs cwbb uller Zden
