\documentstyle[twoside,11pt,myformat]{report}

% NOTE: this file controls which chapters/sections of the library
% manual are actually printed.  It is easy to customize your manual
% by commenting out sections that you're not interested in.

\title{Python Library Reference}

\author{
	Guido van Rossum \\
	Corporation for National Research Initiatives (CNRI) \\
	1895 Preston White Drive, Reston, Va 20191, USA \\
	E-mail: {\tt guido@CNRI.Reston.Va.US}, {\tt guido@python.org}
}

\date{October 25, 1996 \\ Release 1.4} % XXX update before release!


\makeindex			% tell \index to actually write the .idx file


\begin{document}

\pagenumbering{roman}

\maketitle

Copyright \copyright{} 1991-1995 by Stichting Mathematisch Centrum,
Amsterdam, The Netherlands.

\begin{center}
All Rights Reserved
\end{center}

Permission to use, copy, modify, and distribute this software and its
documentation for any purpose and without fee is hereby granted,
provided that the above copyright notice appear in all copies and that
both that copyright notice and this permission notice appear in
supporting documentation, and that the names of Stichting Mathematisch
Centrum or CWI or Corporation for National Research Initiatives or
CNRI not be used in advertising or publicity pertaining to
distribution of the software without specific, written prior
permission.

While CWI is the initial source for this software, a modified version
is made available by the Corporation for National Research Initiatives
(CNRI) at the Internet address ftp://ftp.python.org.

STICHTING MATHEMATISCH CENTRUM AND CNRI DISCLAIM ALL WARRANTIES WITH
REGARD TO THIS SOFTWARE, INCLUDING ALL IMPLIED WARRANTIES OF
MERCHANTABILITY AND FITNESS, IN NO EVENT SHALL STICHTING MATHEMATISCH
CENTRUM OR CNRI BE LIABLE FOR ANY SPECIAL, INDIRECT OR CONSEQUENTIAL
DAMAGES OR ANY DAMAGES WHATSOEVER RESULTING FROM LOSS OF USE, DATA OR
PROFITS, WHETHER IN AN ACTION OF CONTRACT, NEGLIGENCE OR OTHER
TORTIOUS ACTION, ARISING OUT OF OR IN CONNECTION WITH THE USE OR
PERFORMANCE OF THIS SOFTWARE.


\begin{abstract}

\noindent
Python is an extensible, interpreted, object-oriented programming
language.  It supports a wide range of applications, from simple text
processing scripts to interactive WWW browsers.

While the {\em Python Reference Manual} describes the exact syntax and
semantics of the language, it does not describe the standard library
that is distributed with the language, and which greatly enhances its
immediate usability.  This library contains built-in modules (written
in C) that provide access to system functionality such as file I/O
that would otherwise be inaccessible to Python programmers, as well as
modules written in Python that provide standardized solutions for many
problems that occur in everyday programming.  Some of these modules
are explicitly designed to encourage and enhance the portability of
Python programs.

This library reference manual documents Python's standard library, as
well as many optional library modules (which may or may not be
available, depending on whether the underlying platform supports them
and on the configuration choices made at compile time).  It also
documents the standard types of the language and its built-in
functions and exceptions, many of which are not or incompletely
documented in the Reference Manual.

This manual assumes basic knowledge about the Python language.  For an
informal introduction to Python, see the {\em Python Tutorial}; the
Python Reference Manual remains the highest authority on syntactic and
semantic questions.  Finally, the manual entitled {\em Extending and
Embedding the Python Interpreter} describes how to add new extensions
to Python and how to embed it in other applications.

\end{abstract}

\pagebreak

{
\parskip = 0mm
\tableofcontents
}

\pagebreak

\pagenumbering{arabic}

				% Chapter title:

\chapter{Introduction}

The ``Python library'' contains several different kinds of components.

It contains data types that would normally be considered part of the
``core'' of a language, such as numbers and lists.  For these types,
the Python language core defines the form of literals and places some
constraints on their semantics, but does not fully define the
semantics.  (On the other hand, the language core does define
syntactic properties like the spelling and priorities of operators.)

The library also contains built-in functions and exceptions ---
objects that can be used by all Python code without the need of an
\code{import} statement.  Some of these are defined by the core
language, but many are not essential for the core semantics and are
only described here.

The bulk of the library, however, consists of a collection of modules.
There are many ways to dissect this collection.  Some modules are
written in C and built in to the Python interpreter; others are
written in Python and imported in source form.  Some modules provide
interfaces that are highly specific to Python, like printing a stack
trace; some provide interfaces that are specific to particular
operating systems, like socket I/O; others provide interfaces that are
specific to a particular application domain, like the World-Wide Web.
Some modules are avaiable in all versions and ports of Python; others
are only available when the underlying system supports or requires
them; yet others are available only when a particular configuration
option was chosen at the time when Python was compiled and installed.

This manual is organized ``from the inside out'': it first describes
the built-in data types, then the built-in functions and exceptions,
and finally the modules, grouped in chapters of related modules.  The
ordering of the chapters as well as the ordering of the modules within
each chapter is roughly from most relevant to least important.

This means that if you start reading this manual from the start, and
skip to the next chapter when you get bored, you will get a reasonable
overview of the available modules and application areas that are
supported by the Python library.  Of course, you don't \emph{have} to
read it like a novel --- you can also browse the table of contents (in
front of the manual), or look for a specific function, module or term
in the index (in the back).  And finally, if you enjoy learning about
random subjects, you choose a random page number (see module
\code{rand}) and read a section or two.

Let the show begin!
		% Introduction

\chapter{Built-in Types, Exceptions and Functions}

\nodename{Built-in Objects}

Names for built-in exceptions and functions are found in a separate
symbol table.  This table is searched last when the interpreter looks
up the meaning of a name, so local and global
user-defined names can override built-in names.  Built-in types are
described together here for easy reference.%
\footnote{Most descriptions sorely lack explanations of the exceptions
	that may be raised --- this will be fixed in a future version of
	this manual.}
\indexii{built-in}{types}
\indexii{built-in}{exceptions}
\indexii{built-in}{functions}
\index{symbol table}
\bifuncindex{type}

The tables in this chapter document the priorities of operators by
listing them in order of ascending priority (within a table) and
grouping operators that have the same priority in the same box.
Binary operators of the same priority group from left to right.
(Unary operators group from right to left, but there you have no real
choice.)  See Chapter 5 of the Python Reference Manual for the
complete picture on operator priorities.
			% Built-in Types, Exceptions and Functions
\section{Built-in Types}

The following sections describe the standard types that are built into
the interpreter.  These are the numeric types, sequence types, and
several others, including types themselves.  There is no explicit
Boolean type; use integers instead.
\indexii{built-in}{types}
\indexii{Boolean}{type}

Some operations are supported by several object types; in particular,
all objects can be compared, tested for truth value, and converted to
a string (with the \code{`{\rm \ldots}`} notation).  The latter conversion is
implicitly used when an object is written by the \code{print} statement.
\stindex{print}

\subsection{Truth Value Testing}

Any object can be tested for truth value, for use in an \code{if} or
\code{while} condition or as operand of the Boolean operations below.
The following values are considered false:
\stindex{if}
\stindex{while}
\indexii{truth}{value}
\indexii{Boolean}{operations}
\index{false}

\begin{itemize}
\renewcommand{\indexsubitem}{(Built-in object)}

\item	\code{None}
	\ttindex{None}

\item	zero of any numeric type, e.g., \code{0}, \code{0L}, \code{0.0}.

\item	any empty sequence, e.g., \code{''}, \code{()}, \code{[]}.

\item	any empty mapping, e.g., \code{\{\}}.

\item	instances of user-defined classes, if the class defines a
	\code{__nonzero__()} or \code{__len__()} method, when that
	method returns zero.

\end{itemize}

All other values are considered true --- so objects of many types are
always true.
\index{true}

Operations and built-in functions that have a Boolean result always
return \code{0} for false and \code{1} for true, unless otherwise
stated.  (Important exception: the Boolean operations \samp{or} and
\samp{and} always return one of their operands.)

\subsection{Boolean Operations}

These are the Boolean operations, ordered by ascending priority:
\indexii{Boolean}{operations}

\begin{tableiii}{|c|l|c|}{code}{Operation}{Result}{Notes}
  \lineiii{\var{x} or \var{y}}{if \var{x} is false, then \var{y}, else \var{x}}{(1)}
  \hline
  \lineiii{\var{x} and \var{y}}{if \var{x} is false, then \var{x}, else \var{y}}{(1)}
  \hline
  \lineiii{not \var{x}}{if \var{x} is false, then \code{1}, else \code{0}}{(2)}
\end{tableiii}
\opindex{and}
\opindex{or}
\opindex{not}

\noindent
Notes:

\begin{description}

\item[(1)]
These only evaluate their second argument if needed for their outcome.

\item[(2)]
\samp{not} has a lower priority than non-Boolean operators, so e.g.
\code{not a == b} is interpreted as \code{not(a == b)}, and
\code{a == not b} is a syntax error.

\end{description}

\subsection{Comparisons}

Comparison operations are supported by all objects.  They all have the
same priority (which is higher than that of the Boolean operations).
Comparisons can be chained arbitrarily, e.g. \code{x < y <= z} is
equivalent to \code{x < y and y <= z}, except that \code{y} is
evaluated only once (but in both cases \code{z} is not evaluated at
all when \code{x < y} is found to be false).
\indexii{chaining}{comparisons}

This table summarizes the comparison operations:

\begin{tableiii}{|c|l|c|}{code}{Operation}{Meaning}{Notes}
  \lineiii{<}{strictly less than}{}
  \lineiii{<=}{less than or equal}{}
  \lineiii{>}{strictly greater than}{}
  \lineiii{>=}{greater than or equal}{}
  \lineiii{==}{equal}{}
  \lineiii{<>}{not equal}{(1)}
  \lineiii{!=}{not equal}{(1)}
  \lineiii{is}{object identity}{}
  \lineiii{is not}{negated object identity}{}
\end{tableiii}
\indexii{operator}{comparison}
\opindex{==} % XXX *All* others have funny characters < ! >
\opindex{is}
\opindex{is not}

\noindent
Notes:

\begin{description}

\item[(1)]
\code{<>} and \code{!=} are alternate spellings for the same operator.
(I couldn't choose between \ABC{} and \C{}! :-)
\indexii{\ABC{}}{language}
\indexii{\C{}}{language}

\end{description}

Objects of different types, except different numeric types, never
compare equal; such objects are ordered consistently but arbitrarily
(so that sorting a heterogeneous array yields a consistent result).
Furthermore, some types (e.g., windows) support only a degenerate
notion of comparison where any two objects of that type are unequal.
Again, such objects are ordered arbitrarily but consistently.
\indexii{types}{numeric}
\indexii{objects}{comparing}

(Implementation note: objects of different types except numbers are
ordered by their type names; objects of the same types that don't
support proper comparison are ordered by their address.)

Two more operations with the same syntactic priority, \code{in} and
\code{not in}, are supported only by sequence types (below).
\opindex{in}
\opindex{not in}

\subsection{Numeric Types}

There are three numeric types: \dfn{plain integers}, \dfn{long integers}, and
\dfn{floating point numbers}.  Plain integers (also just called \dfn{integers})
are implemented using \code{long} in \C{}, which gives them at least 32
bits of precision.  Long integers have unlimited precision.  Floating
point numbers are implemented using \code{double} in \C{}.  All bets on
their precision are off unless you happen to know the machine you are
working with.
\indexii{numeric}{types}
\indexii{integer}{types}
\indexii{integer}{type}
\indexiii{long}{integer}{type}
\indexii{floating point}{type}
\indexii{\C{}}{language}

Numbers are created by numeric literals or as the result of built-in
functions and operators.  Unadorned integer literals (including hex
and octal numbers) yield plain integers.  Integer literals with an \samp{L}
or \samp{l} suffix yield long integers
(\samp{L} is preferred because \code{1l} looks too much like eleven!).
Numeric literals containing a decimal point or an exponent sign yield
floating point numbers.
\indexii{numeric}{literals}
\indexii{integer}{literals}
\indexiii{long}{integer}{literals}
\indexii{floating point}{literals}
\indexii{hexadecimal}{literals}
\indexii{octal}{literals}

Python fully supports mixed arithmetic: when a binary arithmetic
operator has operands of different numeric types, the operand with the
``smaller'' type is converted to that of the other, where plain
integer is smaller than long integer is smaller than floating point.
Comparisons between numbers of mixed type use the same rule.%
\footnote{As a consequence, the list \code{[1, 2]} is considered equal
	to \code{[1.0, 2.0]}, and similar for tuples.}
The functions \code{int()}, \code{long()} and \code{float()} can be used
to coerce numbers to a specific type.
\index{arithmetic}
\bifuncindex{int}
\bifuncindex{long}
\bifuncindex{float}

All numeric types support the following operations, sorted by
ascending priority (operations in the same box have the same
priority; all numeric operations have a higher priority than
comparison operations):

\begin{tableiii}{|c|l|c|}{code}{Operation}{Result}{Notes}
  \lineiii{\var{x} + \var{y}}{sum of \var{x} and \var{y}}{}
  \lineiii{\var{x} - \var{y}}{difference of \var{x} and \var{y}}{}
  \hline
  \lineiii{\var{x} * \var{y}}{product of \var{x} and \var{y}}{}
  \lineiii{\var{x} / \var{y}}{quotient of \var{x} and \var{y}}{(1)}
  \lineiii{\var{x} \%{} \var{y}}{remainder of \code{\var{x} / \var{y}}}{}
  \hline
  \lineiii{-\var{x}}{\var{x} negated}{}
  \lineiii{+\var{x}}{\var{x} unchanged}{}
  \hline
  \lineiii{abs(\var{x})}{absolute value of \var{x}}{}
  \lineiii{int(\var{x})}{\var{x} converted to integer}{(2)}
  \lineiii{long(\var{x})}{\var{x} converted to long integer}{(2)}
  \lineiii{float(\var{x})}{\var{x} converted to floating point}{}
  \lineiii{divmod(\var{x}, \var{y})}{the pair \code{(\var{x} / \var{y}, \var{x} \%{} \var{y})}}{(3)}
  \lineiii{pow(\var{x}, \var{y})}{\var{x} to the power \var{y}}{}
\end{tableiii}
\indexiii{operations on}{numeric}{types}

\noindent
Notes:
\begin{description}

\item[(1)]
For (plain or long) integer division, the result is an integer.
The result is always rounded towards minus infinity: 1/2 is 0, 
(-1)/2 is -1, 1/(-2) is -1, and (-1)/(-2) is 0.
\indexii{integer}{division}
\indexiii{long}{integer}{division}

\item[(2)]
Conversion from floating point to (long or plain) integer may round or
truncate as in \C{}; see functions \code{floor()} and \code{ceil()} in
module \code{math} for well-defined conversions.
\bifuncindex{floor}
\bifuncindex{ceil}
\indexii{numeric}{conversions}
\stmodindex{math}
\indexii{\C{}}{language}

\item[(3)]
See the section on built-in functions for an exact definition.

\end{description}
% XXXJH exceptions: overflow (when? what operations?) zerodivision

\subsubsection{Bit-string Operations on Integer Types}
\nodename{Bit-string Operations}

Plain and long integer types support additional operations that make
sense only for bit-strings.  Negative numbers are treated as their 2's
complement value (for long integers, this assumes a sufficiently large
number of bits that no overflow occurs during the operation).

The priorities of the binary bit-wise operations are all lower than
the numeric operations and higher than the comparisons; the unary
operation \samp{~} has the same priority as the other unary numeric
operations (\samp{+} and \samp{-}).

This table lists the bit-string operations sorted in ascending
priority (operations in the same box have the same priority):

\begin{tableiii}{|c|l|c|}{code}{Operation}{Result}{Notes}
  \lineiii{\var{x} | \var{y}}{bitwise \dfn{or} of \var{x} and \var{y}}{}
  \hline
  \lineiii{\var{x} \^{} \var{y}}{bitwise \dfn{exclusive or} of \var{x} and \var{y}}{}
  \hline
  \lineiii{\var{x} \&{} \var{y}}{bitwise \dfn{and} of \var{x} and \var{y}}{}
  \hline
  \lineiii{\var{x} << \var{n}}{\var{x} shifted left by \var{n} bits}{(1), (2)}
  \lineiii{\var{x} >> \var{n}}{\var{x} shifted right by \var{n} bits}{(1), (3)}
  \hline
  \hline
  \lineiii{\~\var{x}}{the bits of \var{x} inverted}{}
\end{tableiii}
\indexiii{operations on}{integer}{types}
\indexii{bit-string}{operations}
\indexii{shifting}{operations}
\indexii{masking}{operations}

\noindent
Notes:
\begin{description}
\item[(1)] Negative shift counts are illegal.
\item[(2)] A left shift by \var{n} bits is equivalent to
multiplication by \code{pow(2, \var{n})} without overflow check.
\item[(3)] A right shift by \var{n} bits is equivalent to
division by \code{pow(2, \var{n})} without overflow check.
\end{description}

\subsection{Sequence Types}

There are three sequence types: strings, lists and tuples.

Strings literals are written in single or double quotes:
\code{'xyzzy'}, \code{"frobozz"}.  See Chapter 2 of the Python
Reference Manual for more about string literals.  Lists are
constructed with square brackets, separating items with commas:
\code{[a, b, c]}.  Tuples are constructed by the comma operator (not
within square brackets), with or without enclosing parentheses, but an
empty tuple must have the enclosing parentheses, e.g.,
\code{a, b, c} or \code{()}.  A single item tuple must have a trailing
comma, e.g., \code{(d,)}.
\indexii{sequence}{types}
\indexii{string}{type}
\indexii{tuple}{type}
\indexii{list}{type}

Sequence types support the following operations.  The \samp{in} and
\samp{not\,in} operations have the same priorities as the comparison
operations.  The \samp{+} and \samp{*} operations have the same
priority as the corresponding numeric operations.\footnote{They must
have since the parser can't tell the type of the operands.}

This table lists the sequence operations sorted in ascending priority
(operations in the same box have the same priority).  In the table,
\var{s} and \var{t} are sequences of the same type; \var{n}, \var{i}
and \var{j} are integers:

\begin{tableiii}{|c|l|c|}{code}{Operation}{Result}{Notes}
  \lineiii{\var{x} in \var{s}}{\code{1} if an item of \var{s} is equal to \var{x}, else \code{0}}{}
  \lineiii{\var{x} not in \var{s}}{\code{0} if an item of \var{s} is
equal to \var{x}, else \code{1}}{}
  \hline
  \lineiii{\var{s} + \var{t}}{the concatenation of \var{s} and \var{t}}{}
  \hline
  \lineiii{\var{s} * \var{n}{\rm ,} \var{n} * \var{s}}{\var{n} copies of \var{s} concatenated}{}
  \hline
  \lineiii{\var{s}[\var{i}]}{\var{i}'th item of \var{s}, origin 0}{(1)}
  \lineiii{\var{s}[\var{i}:\var{j}]}{slice of \var{s} from \var{i} to \var{j}}{(1), (2)}
  \hline
  \lineiii{len(\var{s})}{length of \var{s}}{}
  \lineiii{min(\var{s})}{smallest item of \var{s}}{}
  \lineiii{max(\var{s})}{largest item of \var{s}}{}
\end{tableiii}
\indexiii{operations on}{sequence}{types}
\bifuncindex{len}
\bifuncindex{min}
\bifuncindex{max}
\indexii{concatenation}{operation}
\indexii{repetition}{operation}
\indexii{subscript}{operation}
\indexii{slice}{operation}
\opindex{in}
\opindex{not in}

\noindent
Notes:

\begin{description}
  
\item[(1)] If \var{i} or \var{j} is negative, the index is relative to
  the end of the string, i.e., \code{len(\var{s}) + \var{i}} or
  \code{len(\var{s}) + \var{j}} is substituted.  But note that \code{-0} is
  still \code{0}.
  
\item[(2)] The slice of \var{s} from \var{i} to \var{j} is defined as
  the sequence of items with index \var{k} such that \code{\var{i} <=
  \var{k} < \var{j}}.  If \var{i} or \var{j} is greater than
  \code{len(\var{s})}, use \code{len(\var{s})}.  If \var{i} is omitted,
  use \code{0}.  If \var{j} is omitted, use \code{len(\var{s})}.  If
  \var{i} is greater than or equal to \var{j}, the slice is empty.

\end{description}

\subsubsection{More String Operations}

String objects have one unique built-in operation: the \code{\%}
operator (modulo) with a string left argument interprets this string
as a C sprintf format string to be applied to the right argument, and
returns the string resulting from this formatting operation.

The right argument should be a tuple with one item for each argument
required by the format string; if the string requires a single
argument, the right argument may also be a single non-tuple object.%
\footnote{A tuple object in this case should be a singleton.}
The following format characters are understood:
\%, c, s, i, d, u, o, x, X, e, E, f, g, G.
Width and precision may be a * to specify that an integer argument
specifies the actual width or precision.  The flag characters -, +,
blank, \# and 0 are understood.  The size specifiers h, l or L may be
present but are ignored.  The \code{\%s} conversion takes any Python
object and converts it to a string using \code{str()} before
formatting it.  The ANSI features \code{\%p} and \code{\%n}
are not supported.  Since Python strings have an explicit length,
\code{\%s} conversions don't assume that \code{'\e0'} is the end of
the string.

For safety reasons, floating point precisions are clipped to 50;
\code{\%f} conversions for numbers whose absolute value is over 1e25
are replaced by \code{\%g} conversions.%
\footnote{These numbers are fairly arbitrary.  They are intended to
avoid printing endless strings of meaningless digits without hampering
correct use and without having to know the exact precision of floating
point values on a particular machine.}
All other errors raise exceptions.

If the right argument is a dictionary (or any kind of mapping), then
the formats in the string must have a parenthesized key into that
dictionary inserted immediately after the \code{\%} character, and
each format formats the corresponding entry from the mapping.  E.g.
\begin{verbatim}
    >>> count = 2
    >>> language = 'Python'
    >>> print '%(language)s has %(count)03d quote types.' % vars()
    Python has 002 quote types.
    >>> 
\end{verbatim}
In this case no * specifiers may occur in a format (since they
require a sequential parameter list).

Additional string operations are defined in standard module
\code{string} and in built-in module \code{regex}.
\index{string}
\index{regex}

\subsubsection{Mutable Sequence Types}

List objects support additional operations that allow in-place
modification of the object.
These operations would be supported by other mutable sequence types
(when added to the language) as well.
Strings and tuples are immutable sequence types and such objects cannot
be modified once created.
The following operations are defined on mutable sequence types (where
\var{x} is an arbitrary object):
\indexiii{mutable}{sequence}{types}
\indexii{list}{type}

\begin{tableiii}{|c|l|c|}{code}{Operation}{Result}{Notes}
  \lineiii{\var{s}[\var{i}] = \var{x}}
	{item \var{i} of \var{s} is replaced by \var{x}}{}
  \lineiii{\var{s}[\var{i}:\var{j}] = \var{t}}
  	{slice of \var{s} from \var{i} to \var{j} is replaced by \var{t}}{}
  \lineiii{del \var{s}[\var{i}:\var{j}]}
	{same as \code{\var{s}[\var{i}:\var{j}] = []}}{}
  \lineiii{\var{s}.append(\var{x})}
	{same as \code{\var{s}[len(\var{s}):len(\var{s})] = [\var{x}]}}{}
  \lineiii{\var{s}.count(\var{x})}
	{return number of \var{i}'s for which \code{\var{s}[\var{i}] == \var{x}}}{}
  \lineiii{\var{s}.index(\var{x})}
	{return smallest \var{i} such that \code{\var{s}[\var{i}] == \var{x}}}{(1)}
  \lineiii{\var{s}.insert(\var{i}, \var{x})}
	{same as \code{\var{s}[\var{i}:\var{i}] = [\var{x}]}
	  if \code{\var{i} >= 0}}{}
  \lineiii{\var{s}.remove(\var{x})}
	{same as \code{del \var{s}[\var{s}.index(\var{x})]}}{(1)}
  \lineiii{\var{s}.reverse()}
	{reverses the items of \var{s} in place}{}
  \lineiii{\var{s}.sort()}
	{permutes the items of \var{s} to satisfy
        \code{\var{s}[\var{i}] <= \var{s}[\var{j}]},
        for \code{\var{i} < \var{j}}}{(2)}
\end{tableiii}
\indexiv{operations on}{mutable}{sequence}{types}
\indexiii{operations on}{sequence}{types}
\indexiii{operations on}{list}{type}
\indexii{subscript}{assignment}
\indexii{slice}{assignment}
\stindex{del}
\renewcommand{\indexsubitem}{(list method)}
\ttindex{append}
\ttindex{count}
\ttindex{index}
\ttindex{insert}
\ttindex{remove}
\ttindex{reverse}
\ttindex{sort}

\noindent
Notes:
\begin{description}
\item[(1)] Raises an exception when \var{x} is not found in \var{s}.
  
\item[(2)] The \code{sort()} method takes an optional argument
  specifying a comparison function of two arguments (list items) which
  should return \code{-1}, \code{0} or \code{1} depending on whether the
  first argument is considered smaller than, equal to, or larger than the
  second argument.  Note that this slows the sorting process down
  considerably; e.g. to sort a list in reverse order it is much faster
  to use calls to \code{sort()} and \code{reverse()} than to use
  \code{sort()} with a comparison function that reverses the ordering of
  the elements.
\end{description}

\subsection{Mapping Types}

A \dfn{mapping} object maps values of one type (the key type) to
arbitrary objects.  Mappings are mutable objects.  There is currently
only one standard mapping type, the \dfn{dictionary}.  A dictionary's keys are
almost arbitrary values.  The only types of values not acceptable as
keys are values containing lists or dictionaries or other mutable
types that are compared by value rather than by object identity.
Numeric types used for keys obey the normal rules for numeric
comparison: if two numbers compare equal (e.g. 1 and 1.0) then they
can be used interchangeably to index the same dictionary entry.

\indexii{mapping}{types}
\indexii{dictionary}{type}

Dictionaries are created by placing a comma-separated list of
\code{\var{key}:\,\var{value}} pairs within braces, for example:
\code{\{'jack':\,4098, 'sjoerd':\,4127\}} or
\code{\{4098:\,'jack', 4127:\,'sjoerd'\}}.

The following operations are defined on mappings (where \var{a} is a
mapping, \var{k} is a key and \var{x} is an arbitrary object):

\begin{tableiii}{|c|l|c|}{code}{Operation}{Result}{Notes}
  \lineiii{len(\var{a})}{the number of items in \var{a}}{}
  \lineiii{\var{a}[\var{k}]}{the item of \var{a} with key \var{k}}{(1)}
  \lineiii{\var{a}[\var{k}] = \var{x}}{set \code{\var{a}[\var{k}]} to \var{x}}{}
  \lineiii{del \var{a}[\var{k}]}{remove \code{\var{a}[\var{k}]} from \var{a}}{(1)}
  \lineiii{\var{a}.items()}{a copy of \var{a}'s list of (key, item) pairs}{(2)}
  \lineiii{\var{a}.keys()}{a copy of \var{a}'s list of keys}{(2)}
  \lineiii{\var{a}.values()}{a copy of \var{a}'s list of values}{(2)}
  \lineiii{\var{a}.has_key(\var{k})}{\code{1} if \var{a} has a key \var{k}, else \code{0}}{}
\end{tableiii}
\indexiii{operations on}{mapping}{types}
\indexiii{operations on}{dictionary}{type}
\stindex{del}
\bifuncindex{len}
\renewcommand{\indexsubitem}{(dictionary method)}
\ttindex{keys}
\ttindex{has_key}

\noindent
Notes:
\begin{description}
\item[(1)] Raises an exception if \var{k} is not in the map.

\item[(2)] Keys and values are listed in random order.
\end{description}

\subsection{Other Built-in Types}

The interpreter supports several other kinds of objects.
Most of these support only one or two operations.

\subsubsection{Modules}

The only special operation on a module is attribute access:
\code{\var{m}.\var{name}}, where \var{m} is a module and \var{name} accesses
a name defined in \var{m}'s symbol table.  Module attributes can be
assigned to.  (Note that the \code{import} statement is not, strictly
spoken, an operation on a module object; \code{import \var{foo}} does not
require a module object named \var{foo} to exist, rather it requires
an (external) \emph{definition} for a module named \var{foo}
somewhere.)

A special member of every module is \code{__dict__}.
This is the dictionary containing the module's symbol table.
Modifying this dictionary will actually change the module's symbol
table, but direct assignment to the \code{__dict__} attribute is not
possible (i.e., you can write \code{\var{m}.__dict__['a'] = 1}, which
defines \code{\var{m}.a} to be \code{1}, but you can't write \code{\var{m}.__dict__ = \{\}}.

Modules are written like this: \code{<module 'sys'>}.

\subsubsection{Classes and Class Instances}
\nodename{Classes and Instances}

(See Chapters 3 and 7 of the Python Reference Manual for these.)

\subsubsection{Functions}

Function objects are created by function definitions.  The only
operation on a function object is to call it:
\code{\var{func}(\var{argument-list})}.

There are really two flavors of function objects: built-in functions
and user-defined functions.  Both support the same operation (to call
the function), but the implementation is different, hence the
different object types.

The implementation adds two special read-only attributes:
\code{\var{f}.func_code} is a function's \dfn{code object} (see below) and
\code{\var{f}.func_globals} is the dictionary used as the function's
global name space (this is the same as \code{\var{m}.__dict__} where
\var{m} is the module in which the function \var{f} was defined).

\subsubsection{Methods}
\obindex{method}

Methods are functions that are called using the attribute notation.
There are two flavors: built-in methods (such as \code{append()} on
lists) and class instance methods.  Built-in methods are described
with the types that support them.

The implementation adds two special read-only attributes to class
instance methods: \code{\var{m}.im_self} is the object whose method this
is, and \code{\var{m}.im_func} is the function implementing the method.
Calling \code{\var{m}(\var{arg-1}, \var{arg-2}, {\rm \ldots},
\var{arg-n})} is completely equivalent to calling
\code{\var{m}.im_func(\var{m}.im_self, \var{arg-1}, \var{arg-2}, {\rm
\ldots}, \var{arg-n})}.

(See the Python Reference Manual for more info.)

\subsubsection{Code Objects}
\obindex{code}

Code objects are used by the implementation to represent
``pseudo-compiled'' executable Python code such as a function body.
They differ from function objects because they don't contain a
reference to their global execution environment.  Code objects are
returned by the built-in \code{compile()} function and can be
extracted from function objects through their \code{func_code}
attribute.
\bifuncindex{compile}
\ttindex{func_code}

A code object can be executed or evaluated by passing it (instead of a
source string) to the \code{exec} statement or the built-in
\code{eval()} function.
\stindex{exec}
\bifuncindex{eval}

(See the Python Reference Manual for more info.)

\subsubsection{Type Objects}

Type objects represent the various object types.  An object's type is
accessed by the built-in function \code{type()}.  There are no special
operations on types.  The standard module \code{types} defines names
for all standard built-in types.
\bifuncindex{type}
\stmodindex{types}

Types are written like this: \code{<type 'int'>}.

\subsubsection{The Null Object}

This object is returned by functions that don't explicitly return a
value.  It supports no special operations.  There is exactly one null
object, named \code{None} (a built-in name).

It is written as \code{None}.

\subsubsection{File Objects}

File objects are implemented using \C{}'s \code{stdio} package and can be
created with the built-in function \code{open()} described under
Built-in Functions below.  They are also returned by some other
built-in functions and methods, e.g.\ \code{posix.popen()} and
\code{posix.fdopen()} and the \code{makefile()} method of socket
objects.
\bifuncindex{open}

When a file operation fails for an I/O-related reason, the exception
\code{IOError} is raised.  This includes situations where the
operation is not defined for some reason, like \code{seek()} on a tty
device or writing a file opened for reading.

Files have the following methods:


\renewcommand{\indexsubitem}{(file method)}

\begin{funcdesc}{close}{}
  Close the file.  A closed file cannot be read or written anymore.
\end{funcdesc}

\begin{funcdesc}{flush}{}
  Flush the internal buffer, like \code{stdio}'s \code{fflush()}.
\end{funcdesc}

\begin{funcdesc}{isatty}{}
  Return \code{1} if the file is connected to a tty(-like) device, else
  \code{0}.
\end{funcdesc}

\begin{funcdesc}{read}{\optional{size}}
  Read at most \var{size} bytes from the file (less if the read hits
  \EOF{} or no more data is immediately available on a pipe, tty or
  similar device).  If the \var{size} argument is negative or omitted,
  read all data until \EOF{} is reached.  The bytes are returned as a string
  object.  An empty string is returned when \EOF{} is encountered
  immediately.  (For certain files, like ttys, it makes sense to
  continue reading after an \EOF{} is hit.)
\end{funcdesc}

\begin{funcdesc}{readline}{\optional{size}}
  Read one entire line from the file.  A trailing newline character is
  kept in the string%
\footnote{The advantage of leaving the newline on is that an empty string 
	can be returned to mean \EOF{} without being ambiguous.  Another 
	advantage is that (in cases where it might matter, e.g. if you 
	want to make an exact copy of a file while scanning its lines) 
	you can tell whether the last line of a file ended in a newline
	or not (yes this happens!).}
  (but may be absent when a file ends with an
  incomplete line).  If the \var{size} argument is present and
  non-negative, it is a maximum byte count (including the trailing
  newline) and an incomplete line may be returned.
  An empty string is returned when \EOF{} is hit
  immediately.  Note: unlike \code{stdio}'s \code{fgets()}, the returned
  string contains null characters (\code{'\e 0'}) if they occurred in the
  input.
\end{funcdesc}

\begin{funcdesc}{readlines}{}
  Read until \EOF{} using \code{readline()} and return a list containing
  the lines thus read.
\end{funcdesc}

\begin{funcdesc}{seek}{offset\, whence}
  Set the file's current position, like \code{stdio}'s \code{fseek()}.
  The \var{whence} argument is optional and defaults to \code{0}
  (absolute file positioning); other values are \code{1} (seek
  relative to the current position) and \code{2} (seek relative to the
  file's end).  There is no return value.
\end{funcdesc}

\begin{funcdesc}{tell}{}
  Return the file's current position, like \code{stdio}'s \code{ftell()}.
\end{funcdesc}

\begin{funcdesc}{truncate}{\optional{size}}
Truncate the file's size.  If the optional size argument present, the
file is truncated to (at most) that size.  The size defaults to the
current position.  Availability of this function depends on the
operating system version (e.g., not all \UNIX{} versions support this
operation).
\end{funcdesc}

\begin{funcdesc}{write}{str}
Write a string to the file.  There is no return value.  Note: due to
buffering, the string may not actually show up in the file until
the \code{flush()} or \code{close()} method is called.
\end{funcdesc}

\begin{funcdesc}{writelines}{list}
Write a list of strings to the file.  There is no return value.
(The name is intended to match \code{readlines}; \code{writelines}
does not add line separators.)
\end{funcdesc}

\subsubsection{Internal Objects}

(See the Python Reference Manual for these.)

\subsection{Special Attributes}

The implementation adds a few special read-only attributes to several
object types, where they are relevant:

\begin{itemize}

\item
\code{\var{x}.__dict__} is a dictionary of some sort used to store an
object's (writable) attributes;

\item
\code{\var{x}.__methods__} lists the methods of many built-in object types,
e.g., \code{[].__methods__} yields
\code{['append', 'count', 'index', 'insert', 'remove', 'reverse', 'sort']};

\item
\code{\var{x}.__members__} lists data attributes;

\item
\code{\var{x}.__class__} is the class to which a class instance belongs;

\item
\code{\var{x}.__bases__} is the tuple of base classes of a class object.

\end{itemize}

\section{Built-in Exceptions}

Exceptions are string objects.  Two distinct string objects with the
same value are different exceptions.  This is done to force programmers
to use exception names rather than their string value when specifying
exception handlers.  The string value of all built-in exceptions is
their name, but this is not a requirement for user-defined exceptions
or exceptions defined by library modules.

The following exceptions can be generated by the interpreter or
built-in functions.  Except where mentioned, they have an `associated
value' indicating the detailed cause of the error.  This may be a
string or a tuple containing several items of information (e.g., an
error code and a string explaining the code).

User code can raise built-in exceptions.  This can be used to test an
exception handler or to report an error condition `just like' the
situation in which the interpreter raises the same exception; but
beware that there is nothing to prevent user code from raising an
inappropriate error.

\renewcommand{\indexsubitem}{(built-in exception)}

\begin{excdesc}{AttributeError}
% xref to attribute reference?
  Raised when an attribute reference or assignment fails.  (When an
  object does not support attribute references or attribute assignments
  at all, \code{TypeError} is raised.)
\end{excdesc}

\begin{excdesc}{EOFError}
% XXXJH xrefs here
  Raised when one of the built-in functions (\code{input()} or
  \code{raw_input()}) hits an end-of-file condition (\EOF{}) without
  reading any data.
% XXXJH xrefs here
  (N.B.: the \code{read()} and \code{readline()} methods of file
  objects return an empty string when they hit \EOF{}.)  No associated value.
\end{excdesc}

\begin{excdesc}{IOError}
% XXXJH xrefs here
  Raised when an I/O operation (such as a \code{print} statement, the
  built-in \code{open()} function or a method of a file object) fails
  for an I/O-related reason, e.g., `file not found', `disk full'.
\end{excdesc}

\begin{excdesc}{ImportError}
% XXXJH xref to import statement?
  Raised when an \code{import} statement fails to find the module
  definition or when a \code{from {\rm \ldots} import} fails to find a
  name that is to be imported.
\end{excdesc}

\begin{excdesc}{IndexError}
% XXXJH xref to sequences
  Raised when a sequence subscript is out of range.  (Slice indices are
  silently truncated to fall in the allowed range; if an index is not a
  plain integer, \code{TypeError} is raised.)
\end{excdesc}

\begin{excdesc}{KeyError}
% XXXJH xref to mapping objects?
  Raised when a mapping (dictionary) key is not found in the set of
  existing keys.
\end{excdesc}

\begin{excdesc}{KeyboardInterrupt}
  Raised when the user hits the interrupt key (normally
  \kbd{Control-C} or
\key{DEL}).  During execution, a check for interrupts is made regularly.
% XXXJH xrefs here
  Interrupts typed when a built-in function \code{input()} or
  \code{raw_input()}) is waiting for input also raise this exception.  No
  associated value.
\end{excdesc}

\begin{excdesc}{MemoryError}
  Raised when an operation runs out of memory but the situation may
  still be rescued (by deleting some objects).  The associated value is
  a string indicating what kind of (internal) operation ran out of memory.
  Note that because of the underlying memory management architecture
  (\C{}'s \code{malloc()} function), the interpreter may not always be able
  to completely recover from this situation; it nevertheless raises an
  exception so that a stack traceback can be printed, in case a run-away
  program was the cause.
\end{excdesc}

\begin{excdesc}{NameError}
  Raised when a local or global name is not found.  This applies only
  to unqualified names.  The associated value is the name that could
  not be found.
\end{excdesc}

\begin{excdesc}{OverflowError}
% XXXJH reference to long's and/or int's?
  Raised when the result of an arithmetic operation is too large to be
  represented.  This cannot occur for long integers (which would rather
  raise \code{MemoryError} than give up).  Because of the lack of
  standardization of floating point exception handling in \C{}, most
  floating point operations also aren't checked.  For plain integers,
  all operations that can overflow are checked except left shift, where
  typical applications prefer to drop bits than raise an exception.
\end{excdesc}

\begin{excdesc}{RuntimeError}
  Raised when an error is detected that doesn't fall in any of the
  other categories.  The associated value is a string indicating what
  precisely went wrong.  (This exception is a relic from a previous
  version of the interpreter; it is not used any more except by some
  extension modules that haven't been converted to define their own
  exceptions yet.)
\end{excdesc}

\begin{excdesc}{SyntaxError}
% XXXJH xref to these functions?
  Raised when the parser encounters a syntax error.  This may occur in
  an \code{import} statement, in an \code{exec} statement, in a call
  to the built-in function \code{eval()} or \code{input()}, or
  when reading the initial script or standard input (also
  interactively).
\end{excdesc}

\begin{excdesc}{SystemError}
  Raised when the interpreter finds an internal error, but the
  situation does not look so serious to cause it to abandon all hope.
  The associated value is a string indicating what went wrong (in
  low-level terms).
  
  You should report this to the author or maintainer of your Python
  interpreter.  Be sure to report the version string of the Python
  interpreter (\code{sys.version}; it is also printed at the start of an
  interactive Python session), the exact error message (the exception's
  associated value) and if possible the source of the program that
  triggered the error.
\end{excdesc}

\begin{excdesc}{SystemExit}
% XXXJH xref to module sys?
  This exception is raised by the \code{sys.exit()} function.  When it
  is not handled, the Python interpreter exits; no stack traceback is
  printed.  If the associated value is a plain integer, it specifies the
  system exit status (passed to \C{}'s \code{exit()} function); if it is
  \code{None}, the exit status is zero; if it has another type (such as
  a string), the object's value is printed and the exit status is one.
  
  A call to \code{sys.exit} is translated into an exception so that
  clean-up handlers (\code{finally} clauses of \code{try} statements)
  can be executed, and so that a debugger can execute a script without
  running the risk of losing control.  The \code{posix._exit()} function
  can be used if it is absolutely positively necessary to exit
  immediately (e.g., after a \code{fork()} in the child process).
\end{excdesc}

\begin{excdesc}{TypeError}
  Raised when a built-in operation or function is applied to an object
  of inappropriate type.  The associated value is a string giving
  details about the type mismatch.
\end{excdesc}

\begin{excdesc}{ValueError}
  Raised when a built-in operation or function receives an argument
  that has the right type but an inappropriate value, and the
  situation is not described by a more precise exception such as
  \code{IndexError}.
\end{excdesc}

\begin{excdesc}{ZeroDivisionError}
  Raised when the second argument of a division or modulo operation is
  zero.  The associated value is a string indicating the type of the
  operands and the operation.
\end{excdesc}

\section{Built-in Functions}

The Python interpreter has a number of functions built into it that
are always available.  They are listed here in alphabetical order.


\renewcommand{\indexsubitem}{(built-in function)}
\begin{funcdesc}{abs}{x}
  Return the absolute value of a number.  The argument may be a plain
  or long integer or a floating point number.
\end{funcdesc}

\begin{funcdesc}{apply}{function\, args\optional{, keywords}}
The \var{function} argument must be a callable object (a user-defined or
built-in function or method, or a class object) and the \var{args}
argument must be a tuple.  The \var{function} is called with
\var{args} as argument list; the number of arguments is the the length
of the tuple.  (This is different from just calling
\code{\var{func}(\var{args})}, since in that case there is always
exactly one argument.)
If the optional \var{keywords} argument is present, it must be a
dictionary whose keys are strings.  It specifies keyword arguments to
be added to the end of the the argument list.
\end{funcdesc}

\begin{funcdesc}{chr}{i}
  Return a string of one character whose \ASCII{} code is the integer
  \var{i}, e.g., \code{chr(97)} returns the string \code{'a'}.  This is the
  inverse of \code{ord()}.  The argument must be in the range [0..255],
  inclusive.
\end{funcdesc}

\begin{funcdesc}{cmp}{x\, y}
  Compare the two objects \var{x} and \var{y} and return an integer
  according to the outcome.  The return value is negative if \code{\var{x}
  < \var{y}}, zero if \code{\var{x} == \var{y}} and strictly positive if
  \code{\var{x} > \var{y}}.
\end{funcdesc}

\begin{funcdesc}{coerce}{x\, y}
  Return a tuple consisting of the two numeric arguments converted to
  a common type, using the same rules as used by arithmetic
  operations.
\end{funcdesc}

\begin{funcdesc}{compile}{string\, filename\, kind}
  Compile the \var{string} into a code object.  Code objects can be
  executed by an \code{exec} statement or evaluated by a call to
  \code{eval()}.  The \var{filename} argument should
  give the file from which the code was read; pass e.g. \code{'<string>'}
  if it wasn't read from a file.  The \var{kind} argument specifies
  what kind of code must be compiled; it can be \code{'exec'} if
  \var{string} consists of a sequence of statements, \code{'eval'}
  if it consists of a single expression, or \code{'single'} if
  it consists of a single interactive statement (in the latter case,
  expression statements that evaluate to something else than
  \code{None} will printed).
\end{funcdesc}

\begin{funcdesc}{delattr}{object\, name}
  This is a relative of \code{setattr}.  The arguments are an
  object and a string.  The string must be the name
  of one of the object's attributes.  The function deletes
  the named attribute, provided the object allows it.  For example,
  \code{delattr(\var{x}, '\var{foobar}')} is equivalent to
  \code{del \var{x}.\var{foobar}}.
\end{funcdesc}

\begin{funcdesc}{dir}{}
  Without arguments, return the list of names in the current local
  symbol table.  With a module, class or class instance object as
  argument (or anything else that has a \code{__dict__} attribute),
  returns the list of names in that object's attribute dictionary.
  The resulting list is sorted.  For example:

\bcode\begin{verbatim}
>>> import sys
>>> dir()
['sys']
>>> dir(sys)
['argv', 'exit', 'modules', 'path', 'stderr', 'stdin', 'stdout']
>>> 
\end{verbatim}\ecode
\end{funcdesc}

\begin{funcdesc}{divmod}{a\, b}
  Take two numbers as arguments and return a pair of integers
  consisting of their integer quotient and remainder.  With mixed
  operand types, the rules for binary arithmetic operators apply.  For
  plain and long integers, the result is the same as
  \code{(\var{a} / \var{b}, \var{a} \%{} \var{b})}.
  For floating point numbers the result is the same as
  \code{(math.floor(\var{a} / \var{b}), \var{a} \%{} \var{b})}.
\end{funcdesc}

\begin{funcdesc}{eval}{expression\optional{\, globals\optional{\, locals}}}
  The arguments are a string and two optional dictionaries.  The
  \var{expression} argument is parsed and evaluated as a Python
  expression (technically speaking, a condition list) using the
  \var{globals} and \var{locals} dictionaries as global and local name
  space.  If the \var{locals} dictionary is omitted it defaults to
  the \var{globals} dictionary.  If both dictionaries are omitted, the
  expression is executed in the environment where \code{eval} is
  called.  The return value is the result of the evaluated expression.
  Syntax errors are reported as exceptions.  Example:

\bcode\begin{verbatim}
>>> x = 1
>>> print eval('x+1')
2
>>> 
\end{verbatim}\ecode

  This function can also be used to execute arbitrary code objects
  (e.g.\ created by \code{compile()}).  In this case pass a code
  object instead of a string.  The code object must have been compiled
  passing \code{'eval'} to the \var{kind} argument.

  Hints: dynamic execution of statements is supported by the
  \code{exec} statement.  Execution of statements from a file is
  supported by the \code{execfile()} function.  The \code{globals()}
  and \code{locals()} functions returns the current global and local
  dictionary, respectively, which may be useful
  to pass around for use by \code{eval()} or \code{execfile()}.

\end{funcdesc}

\begin{funcdesc}{execfile}{file\optional{\, globals\optional{\, locals}}}
  This function is similar to the
  \code{exec} statement, but parses a file instead of a string.  It is
  different from the \code{import} statement in that it does not use
  the module administration --- it reads the file unconditionally and
  does not create a new module.\footnote{It is used relatively rarely
  so does not warrant being made into a statement.}

  The arguments are a file name and two optional dictionaries.  The
  file is parsed and evaluated as a sequence of Python statements
  (similarly to a module) using the \var{globals} and \var{locals}
  dictionaries as global and local name space.  If the \var{locals}
  dictionary is omitted it defaults to the \var{globals} dictionary.
  If both dictionaries are omitted, the expression is executed in the
  environment where \code{execfile()} is called.  The return value is
  \code{None}.
\end{funcdesc}

\begin{funcdesc}{filter}{function\, list}
Construct a list from those elements of \var{list} for which
\var{function} returns true.  If \var{list} is a string or a tuple,
the result also has that type; otherwise it is always a list.  If
\var{function} is \code{None}, the identity function is assumed,
i.e.\ all elements of \var{list} that are false (zero or empty) are
removed.
\end{funcdesc}

\begin{funcdesc}{float}{x}
  Convert a number to floating point.  The argument may be a plain or
  long integer or a floating point number.
\end{funcdesc}

\begin{funcdesc}{getattr}{object\, name}
  The arguments are an object and a string.  The string must be the
  name
  of one of the object's attributes.  The result is the value of that
  attribute.  For example, \code{getattr(\var{x}, '\var{foobar}')} is equivalent to
  \code{\var{x}.\var{foobar}}.
\end{funcdesc}

\begin{funcdesc}{globals}{}
Return a dictionary representing the current global symbol table.
This is always the dictionary of the current module (inside a
function or method, this is the module where it is defined, not the
module from which it is called).
\end{funcdesc}

\begin{funcdesc}{hasattr}{object\, name}
  The arguments are an object and a string.  The result is 1 if the
  string is the name of one of the object's attributes, 0 if not.
  (This is implemented by calling \code{getattr(object, name)} and
  seeing whether it raises an exception or not.)
\end{funcdesc}

\begin{funcdesc}{hash}{object}
  Return the hash value of the object (if it has one).  Hash values
  are 32-bit integers.  They are used to quickly compare dictionary
  keys during a dictionary lookup.  Numeric values that compare equal
  have the same hash value (even if they are of different types, e.g.
  1 and 1.0).
\end{funcdesc}

\begin{funcdesc}{hex}{x}
  Convert an integer number (of any size) to a hexadecimal string.
  The result is a valid Python expression.
\end{funcdesc}

\begin{funcdesc}{id}{object}
  Return the `identity' of an object.  This is an integer which is
  guaranteed to be unique and constant for this object during its
  lifetime.  (Two objects whose lifetimes are disjunct may have the
  same id() value.)  (Implementation note: this is the address of the
  object.)
\end{funcdesc}

\begin{funcdesc}{input}{\optional{prompt}}
  Almost equivalent to \code{eval(raw_input(\var{prompt}))}.  Like
  \code{raw_input()}, the \var{prompt} argument is optional.  The difference
  is that a long input expression may be broken over multiple lines using
  the backslash convention.
\end{funcdesc}

\begin{funcdesc}{int}{x}
  Convert a number to a plain integer.  The argument may be a plain or
  long integer or a floating point number.  Conversion of floating
  point numbers to integers is defined by the C semantics; normally
  the conversion truncates towards zero.\footnote{This is ugly --- the
  language definition should require truncation towards zero.}
\end{funcdesc}

\begin{funcdesc}{len}{s}
  Return the length (the number of items) of an object.  The argument
  may be a sequence (string, tuple or list) or a mapping (dictionary).
\end{funcdesc}

\begin{funcdesc}{locals}{}
Return a dictionary representing the current local symbol table.
Inside a function, modifying this dictionary does not always have the
desired effect.
\end{funcdesc}

\begin{funcdesc}{long}{x}
  Convert a number to a long integer.  The argument may be a plain or
  long integer or a floating point number.
\end{funcdesc}

\begin{funcdesc}{map}{function\, list\, ...}
Apply \var{function} to every item of \var{list} and return a list
of the results.  If additional \var{list} arguments are passed, 
\var{function} must take that many arguments and is applied to
the items of all lists in parallel; if a list is shorter than another
it is assumed to be extended with \code{None} items.  If
\var{function} is \code{None}, the identity function is assumed; if
there are multiple list arguments, \code{map} returns a list
consisting of tuples containing the corresponding items from all lists
(i.e. a kind of transpose operation).  The \var{list} arguments may be
any kind of sequence; the result is always a list.
\end{funcdesc}

\begin{funcdesc}{max}{s}
  Return the largest item of a non-empty sequence (string, tuple or
  list).
\end{funcdesc}

\begin{funcdesc}{min}{s}
  Return the smallest item of a non-empty sequence (string, tuple or
  list).
\end{funcdesc}

\begin{funcdesc}{oct}{x}
  Convert an integer number (of any size) to an octal string.  The
  result is a valid Python expression.
\end{funcdesc}

\begin{funcdesc}{open}{filename\optional{\, mode\optional{\, bufsize}}}
  Return a new file object (described earlier under Built-in Types).
  The first two arguments are the same as for \code{stdio}'s
  \code{fopen()}: \var{filename} is the file name to be opened,
  \var{mode} indicates how the file is to be opened: \code{'r'} for
  reading, \code{'w'} for writing (truncating an existing file), and
  \code{'a'} opens it for appending (which on {\em some} \UNIX{}
  systems means that {\em all} writes append to the end of the file,
  regardless of the current seek position).
  Modes \code{'r+'}, \code{'w+'} and
  \code{'a+'} open the file for updating, provided the underlying
  \code{stdio} library understands this.  On systems that differentiate
  between binary and text files, \code{'b'} appended to the mode opens
  the file in binary mode.  If the file cannot be opened, \code{IOError}
  is raised.
If \var{mode} is omitted, it defaults to \code{'r'}.
The optional \var{bufsize} argument specifies the file's desired
buffer size: 0 means unbuffered, 1 means line buffered, any other
positive value means use a buffer of (approximately) that size.  A
negative \var{bufsize} means to use the system default, which is
usually line buffered for for tty devices and fully buffered for other
files.%
\footnote{Specifying a buffer size currently has no effect on systems
that don't have \code{setvbuf()}.  The interface to specify the buffer
size is not done using a method that calls \code{setvbuf()}, because
that may dump core when called after any I/O has been performed, and
there's no reliable way to determine whether this is the case.}
\end{funcdesc}

\begin{funcdesc}{ord}{c}
  Return the \ASCII{} value of a string of one character.  E.g.,
  \code{ord('a')} returns the integer \code{97}.  This is the inverse of
  \code{chr()}.
\end{funcdesc}

\begin{funcdesc}{pow}{x\, y\optional{\, z}}
  Return \var{x} to the power \var{y}; if \var{z} is present, return
  \var{x} to the power \var{y}, modulo \var{z} (computed more
  efficiently than \code{pow(\var{x}, \var{y}) \% \var{z}}).
  The arguments must have
  numeric types.  With mixed operand types, the rules for binary
  arithmetic operators apply.  The effective operand type is also the
  type of the result; if the result is not expressible in this type, the
  function raises an exception; e.g., \code{pow(2, -1)} or \code{pow(2,
  35000)} is not allowed.
\end{funcdesc}

\begin{funcdesc}{range}{\optional{start\,} end\optional{\, step}}
  This is a versatile function to create lists containing arithmetic
  progressions.  It is most often used in \code{for} loops.  The
  arguments must be plain integers.  If the \var{step} argument is
  omitted, it defaults to \code{1}.  If the \var{start} argument is
  omitted, it defaults to \code{0}.  The full form returns a list of
  plain integers \code{[\var{start}, \var{start} + \var{step},
  \var{start} + 2 * \var{step}, \ldots]}.  If \var{step} is positive,
  the last element is the largest \code{\var{start} + \var{i} *
  \var{step}} less than \var{end}; if \var{step} is negative, the last
  element is the largest \code{\var{start} + \var{i} * \var{step}}
  greater than \var{end}.  \var{step} must not be zero (or else an
  exception is raised).  Example:

\bcode\begin{verbatim}
>>> range(10)
[0, 1, 2, 3, 4, 5, 6, 7, 8, 9]
>>> range(1, 11)
[1, 2, 3, 4, 5, 6, 7, 8, 9, 10]
>>> range(0, 30, 5)
[0, 5, 10, 15, 20, 25]
>>> range(0, 10, 3)
[0, 3, 6, 9]
>>> range(0, -10, -1)
[0, -1, -2, -3, -4, -5, -6, -7, -8, -9]
>>> range(0)
[]
>>> range(1, 0)
[]
>>> 
\end{verbatim}\ecode
\end{funcdesc}

\begin{funcdesc}{raw_input}{\optional{prompt}}
  If the \var{prompt} argument is present, it is written to standard output
  without a trailing newline.  The function then reads a line from input,
  converts it to a string (stripping a trailing newline), and returns that.
  When \EOF{} is read, \code{EOFError} is raised. Example:

\bcode\begin{verbatim}
>>> s = raw_input('--> ')
--> Monty Python's Flying Circus
>>> s
"Monty Python's Flying Circus"
>>> 
\end{verbatim}\ecode
\end{funcdesc}

\begin{funcdesc}{reduce}{function\, list\optional{\, initializer}}
Apply the binary \var{function} to the items of \var{list} so as to
reduce the list to a single value.  E.g.,
\code{reduce(lambda x, y: x*y, \var{list}, 1)} returns the product of
the elements of \var{list}.  The optional \var{initializer} can be
thought of as being prepended to \var{list} so as to allow reduction
of an empty \var{list}.  The \var{list} arguments may be any kind of
sequence.
\end{funcdesc}

\begin{funcdesc}{reload}{module}
Re-parse and re-initialize an already imported \var{module}.  The
argument must be a module object, so it must have been successfully
imported before.  This is useful if you have edited the module source
file using an external editor and want to try out the new version
without leaving the Python interpreter.  The return value is the
module object (i.e.\ the same as the \var{module} argument).

There are a number of caveats:

If a module is syntactically correct but its initialization fails, the
first \code{import} statement for it does not bind its name locally,
but does store a (partially initialized) module object in
\code{sys.modules}.  To reload the module you must first
\code{import} it again (this will bind the name to the partially
initialized module object) before you can \code{reload()} it.

When a module is reloaded, its dictionary (containing the module's
global variables) is retained.  Redefinitions of names will override
the old definitions, so this is generally not a problem.  If the new
version of a module does not define a name that was defined by the old
version, the old definition remains.  This feature can be used to the
module's advantage if it maintains a global table or cache of objects
--- with a \code{try} statement it can test for the table's presence
and skip its initialization if desired.

It is legal though generally not very useful to reload built-in or
dynamically loaded modules, except for \code{sys}, \code{__main__} and
\code{__builtin__}.  In certain cases, however, extension modules are
not designed to be initialized more than once, and may fail in
arbitrary ways when reloaded.

If a module imports objects from another module using \code{from}
{\ldots} \code{import} {\ldots}, calling \code{reload()} for the other
module does not redefine the objects imported from it --- one way
around this is to re-execute the \code{from} statement, another is to
use \code{import} and qualified names (\var{module}.\var{name})
instead.

If a module instantiates instances of a class, reloading the module
that defines the class does not affect the method definitions of the
instances --- they continue to use the old class definition.  The same
is true for derived classes.
\end{funcdesc}

\begin{funcdesc}{repr}{object}
Return a string containing a printable representation of an object.
This is the same value yielded by conversions (reverse quotes).
It is sometimes useful to be able to access this operation as an
ordinary function.  For many types, this function makes an attempt
to return a string that would yield an object with the same value
when passed to \code{eval()}.
\end{funcdesc}

\begin{funcdesc}{round}{x\, n}
  Return the floating point value \var{x} rounded to \var{n} digits
  after the decimal point.  If \var{n} is omitted, it defaults to zero.
  The result is a floating point number.  Values are rounded to the
  closest multiple of 10 to the power minus \var{n}; if two multiples
  are equally close, rounding is done away from 0 (so e.g.
  \code{round(0.5)} is \code{1.0} and \code{round(-0.5)} is \code{-1.0}).
\end{funcdesc}

\begin{funcdesc}{setattr}{object\, name\, value}
  This is the counterpart of \code{getattr}.  The arguments are an
  object, a string and an arbitrary value.  The string must be the name
  of one of the object's attributes.  The function assigns the value to
  the attribute, provided the object allows it.  For example,
  \code{setattr(\var{x}, '\var{foobar}', 123)} is equivalent to
  \code{\var{x}.\var{foobar} = 123}.
\end{funcdesc}

\begin{funcdesc}{str}{object}
Return a string containing a nicely printable representation of an
object.  For strings, this returns the string itself.  The difference
with \code{repr(\var{object})} is that \code{str(\var{object})} does not
always attempt to return a string that is acceptable to \code{eval()};
its goal is to return a printable string.
\end{funcdesc}

\begin{funcdesc}{tuple}{sequence}
Return a tuple whose items are the same and in the same order as
\var{sequence}'s items.  If \var{sequence} is alread a tuple, it
is returned unchanged.  For instance, \code{tuple('abc')} returns
returns \code{('a', 'b', 'c')} and \code{tuple([1, 2, 3])} returns
\code{(1, 2, 3)}.
\end{funcdesc}

\begin{funcdesc}{type}{object}
Return the type of an \var{object}.  The return value is a type
object.  The standard module \code{types} defines names for all
built-in types.
\stmodindex{types}
\obindex{type}
For instance:

\bcode\begin{verbatim}
>>> import types
>>> if type(x) == types.StringType: print "It's a string"
\end{verbatim}\ecode
\end{funcdesc}

\begin{funcdesc}{vars}{\optional{object}}
Without arguments, return a dictionary corresponding to the current
local symbol table.  With a module, class or class instance object as
argument (or anything else that has a \code{__dict__} attribute),
returns a dictionary corresponding to the object's symbol table.
The returned dictionary should not be modified: the effects on the
corresponding symbol table are undefined.%
\footnote{In the current implementation, local variable bindings
cannot normally be affected this way, but variables retrieved from
other scopes (e.g. modules) can be.  This may change.}
\end{funcdesc}

\begin{funcdesc}{xrange}{\optional{start\,} end\optional{\, step}}
This function is very similar to \code{range()}, but returns an
``xrange object'' instead of a list.  This is an opaque sequence type
which yields the same values as the corresponding list, without
actually storing them all simultaneously.  The advantage of
\code{xrange()} over \code{range()} is minimal (since \code{xrange()}
still has to create the values when asked for them) except when a very
large range is used on a memory-starved machine (e.g. MS-DOS) or when all
of the range's elements are never used (e.g. when the loop is usually
terminated with \code{break}).
\end{funcdesc}


\chapter{Python Services}

The modules described in this chapter provide a wide range of services
related to the Python interpreter and its interaction with its
environment.  Here's an overview:

\begin{description}

\item[sys]
--- Access system specific parameters and functions.

\item[types]
--- Names for all built-in types.

\item[traceback]
--- Print or retrieve a stack traceback.

\item[pickle]
--- Convert Python objects to streams of bytes and back.

\item[shelve]
--- Python object persistency.

\item[copy]
--- Shallow and deep copy operations.

\item[marshal]
--- Convert Python objects to streams of bytes and back (with
different constraints).

\item[imp]
--- Access the implementation of the \code{import} statement.

\item[parser]
--- Retrieve and submit parse trees from and to the runtime support
environment.

\item[__builtin__]
--- The set of built-in functions.

\item[__main__]
--- The environment where the top-level script is run.

\end{description}
		% Python Services
\section{Built-in Module \sectcode{sys}}

\bimodindex{sys}
This module provides access to some variables used or maintained by the
interpreter and to functions that interact strongly with the interpreter.
It is always available.

\renewcommand{\indexsubitem}{(in module sys)}

\begin{datadesc}{argv}
  The list of command line arguments passed to a Python script.
  \code{sys.argv[0]} is the script name (it is operating system
  dependent whether this is a full pathname or not).
  If the command was executed using the \samp{-c} command line option
  to the interpreter, \code{sys.argv[0]} is set to the string
  \code{"-c"}.
  If no script name was passed to the Python interpreter,
  \code{sys.argv} has zero length.
\end{datadesc}

\begin{datadesc}{builtin_module_names}
  A list of strings giving the names of all modules that are compiled
  into this Python interpreter.  (This information is not available in
  any other way --- \code{sys.modules.keys()} only lists the imported
  modules.)
\end{datadesc}

\begin{datadesc}{exc_type}
\dataline{exc_value}
\dataline{exc_traceback}
  These three variables are not always defined; they are set when an
  exception handler (an \code{except} clause of a \code{try} statement) is
  invoked.  Their meaning is: \code{exc_type} gets the exception type of
  the exception being handled; \code{exc_value} gets the exception
  parameter (its \dfn{associated value} or the second argument to
  \code{raise}); \code{exc_traceback} gets a traceback object (see the
  Reference Manual) which
  encapsulates the call stack at the point where the exception
  originally occurred.
\obindex{traceback}
\end{datadesc}

\begin{funcdesc}{exit}{n}
  Exit from Python with numeric exit status \var{n}.  This is
  implemented by raising the \code{SystemExit} exception, so cleanup
  actions specified by \code{finally} clauses of \code{try} statements
  are honored, and it is possible to catch the exit attempt at an outer
  level.
\end{funcdesc}

\begin{datadesc}{exitfunc}
  This value is not actually defined by the module, but can be set by
  the user (or by a program) to specify a clean-up action at program
  exit.  When set, it should be a parameterless function.  This function
  will be called when the interpreter exits in any way (except when a
  fatal error occurs: in that case the interpreter's internal state
  cannot be trusted).
\end{datadesc}

\begin{datadesc}{last_type}
\dataline{last_value}
\dataline{last_traceback}
  These three variables are not always defined; they are set when an
  exception is not handled and the interpreter prints an error message
  and a stack traceback.  Their intended use is to allow an interactive
  user to import a debugger module and engage in post-mortem debugging
  without having to re-execute the command that caused the error (which
  may be hard to reproduce).  The meaning of the variables is the same
  as that of \code{exc_type}, \code{exc_value} and \code{exc_tracaback},
  respectively.
\end{datadesc}

\begin{datadesc}{modules}
  Gives the list of modules that have already been loaded.
  This can be manipulated to force reloading of modules and other tricks.
\end{datadesc}

\begin{datadesc}{path}
  A list of strings that specifies the search path for modules.
  Initialized from the environment variable \code{PYTHONPATH}, or an
  installation-dependent default.
\end{datadesc}

\begin{datadesc}{platform}
This string contains a platform identifier.  This can be used to
append platform-specific components to \code{sys.path}, for instance.
\end{datadesc}

\begin{datadesc}{ps1}
\dataline{ps2}
  Strings specifying the primary and secondary prompt of the
  interpreter.  These are only defined if the interpreter is in
  interactive mode.  Their initial values in this case are
  \code{'>>> '} and \code{'... '}.
\end{datadesc}

\begin{funcdesc}{setcheckinterval}{interval}
Set the interpreter's ``check interval''.  This integer value
determines how often the interpreter checks for periodic things such
as thread switches and signal handlers.  The default is 10, meaning
the check is performed every 10 Python virtual instructions.  Setting
it to a larger value may increase performance for programs using
threads.  Setting it to a value $\leq 0$ checks every virtual instruction,
maximizing responsiveness as well as overhead.
\end{funcdesc}

\begin{funcdesc}{settrace}{tracefunc}
  Set the system's trace function, which allows you to implement a
  Python source code debugger in Python.  See section ``How It Works''
  in the chapter on the Python Debugger.
\end{funcdesc}
\index{trace function}
\index{debugger}

\begin{funcdesc}{setprofile}{profilefunc}
  Set the system's profile function, which allows you to implement a
  Python source code profiler in Python.  See the chapter on the
  Python Profiler.  The system's profile function
  is called similarly to the system's trace function (see
  \code{sys.settrace}), but it isn't called for each executed line of
  code (only on call and return and when an exception occurs).  Also,
  its return value is not used, so it can just return \code{None}.
\end{funcdesc}
\index{profile function}
\index{profiler}

\begin{datadesc}{stdin}
\dataline{stdout}
\dataline{stderr}
  File objects corresponding to the interpreter's standard input,
  output and error streams.  \code{sys.stdin} is used for all
  interpreter input except for scripts but including calls to
  \code{input()} and \code{raw_input()}.  \code{sys.stdout} is used
  for the output of \code{print} and expression statements and for the
  prompts of \code{input()} and \code{raw_input()}.  The interpreter's
  own prompts and (almost all of) its error messages go to
  \code{sys.stderr}.  \code{sys.stdout} and \code{sys.stderr} needn't
  be built-in file objects: any object is acceptable as long as it has
  a \code{write} method that takes a string argument.  (Changing these
  objects doesn't affect the standard I/O streams of processes
  executed by \code{popen()}, \code{system()} or the \code{exec*()}
  family of functions in the \code{os} module.)
\stmodindex{os}
\end{datadesc}

\begin{datadesc}{tracebacklimit}
When this variable is set to an integer value, it determines the
maximum number of levels of traceback information printed when an
unhandled exception occurs.  The default is 1000.  When set to 0 or
less, all traceback information is suppressed and only the exception
type and value are printed.
\end{datadesc}

\section{Standard Module \sectcode{types}}
\stmodindex{types}

\renewcommand{\indexsubitem}{(in module types)}

This module defines names for all object types that are used by the
standard Python interpreter (but not for the types defined by various
extension modules).  It is safe to use ``\code{from types import *}'' ---
the module does not export any other names besides the ones listed
here.  New names exported by future versions of this module will
all end in \code{Type}.

Typical use is for functions that do different things depending on
their argument types, like the following:

\begin{verbatim}
from types import *
def delete(list, item):
    if type(item) is IntType:
       del list[item]
    else:
       list.remove(item)
\end{verbatim}

The module defines the following names:

\begin{datadesc}{NoneType}
The type of \code{None}.
\end{datadesc}

\begin{datadesc}{TypeType}
The type of type objects (such as returned by \code{type()}).
\end{datadesc}

\begin{datadesc}{IntType}
The type of integers (e.g. \code{1}).
\end{datadesc}

\begin{datadesc}{LongType}
The type of long integers (e.g. \code{1L}).
\end{datadesc}

\begin{datadesc}{FloatType}
The type of floating point numbers (e.g. \code{1.0}).
\end{datadesc}

\begin{datadesc}{StringType}
The type of character strings (e.g. \code{'Spam'}).
\end{datadesc}

\begin{datadesc}{TupleType}
The type of tuples (e.g. \code{(1, 2, 3, 'Spam')}).
\end{datadesc}

\begin{datadesc}{ListType}
The type of lists (e.g. \code{[0, 1, 2, 3]}).
\end{datadesc}

\begin{datadesc}{DictType}
The type of dictionaries (e.g. \code{\{'Bacon': 1, 'Ham': 0\}}).
\end{datadesc}

\begin{datadesc}{DictionaryType}
An alternative name for \code{DictType}.
\end{datadesc}

\begin{datadesc}{FunctionType}
The type of user-defined functions and lambdas.
\end{datadesc}

\begin{datadesc}{LambdaType}
	An alternative name for \code{FunctionType}.
\end{datadesc}

\begin{datadesc}{CodeType}
The type for code objects such as returned by \code{compile()}.
\end{datadesc}

\begin{datadesc}{ClassType}
The type of user-defined classes.
\end{datadesc}

\begin{datadesc}{InstanceType}
The type of instances of user-defined classes.
\end{datadesc}

\begin{datadesc}{MethodType}
The type of methods of user-defined class instances.
\end{datadesc}

\begin{datadesc}{UnboundMethodType}
An alternative name for \code{MethodType}.
\end{datadesc}

\begin{datadesc}{BuiltinFunctionType}
The type of built-in functions like \code{len} or \code{sys.exit}.
\end{datadesc}

\begin{datadesc}{BuiltinMethodType}
An alternative name for \code{BuiltinFunction}.
\end{datadesc}

\begin{datadesc}{ModuleType}
The type of modules.
\end{datadesc}

\begin{datadesc}{FileType}
The type of open file objects such as \code{sys.stdout}.
\end{datadesc}

\begin{datadesc}{XRangeType}
The type of range objects returned by \code{xrange()}.
\end{datadesc}

\begin{datadesc}{TracebackType}
The type of traceback objects such as found in \code{sys.exc_traceback}.
\end{datadesc}

\begin{datadesc}{FrameType}
The type of frame objects such as found in \code{tb.tb_frame} if
\code{tb} is a traceback object.
\end{datadesc}
		% types is already taken :-(
\section{Standard Module \sectcode{traceback}}
\stmodindex{traceback}

\renewcommand{\indexsubitem}{(in module traceback)}

This module provides a standard interface to format and print stack
traces of Python programs.  It exactly mimics the behavior of the
Python interpreter when it prints a stack trace.  This is useful when
you want to print stack traces under program control, e.g. in a
``wrapper'' around the interpreter.

The module uses traceback objects --- this is the object type
that is stored in the variables \code{sys.exc_traceback} and
\code{sys.last_traceback}.

The module defines the following functions:

\begin{funcdesc}{print_tb}{traceback\optional{\, limit}}
Print up to \var{limit} stack trace entries from \var{traceback}.  If
\var{limit} is omitted or \code{None}, all entries are printed.
\end{funcdesc}

\begin{funcdesc}{extract_tb}{traceback\optional{\, limit}}
Return a list of up to \var{limit} ``pre-processed'' stack trace
entries extracted from \var{traceback}.  It is useful for alternate
formatting of stack traces.  If \var{limit} is omitted or \code{None},
all entries are extracted.  A ``pre-processed'' stack trace entry is a
quadruple (\var{filename}, \var{line number}, \var{function name},
\var{line text}) representing the information that is usually printed
for a stack trace.  The \var{line text} is a string with leading and
trailing whitespace stripped; if the source is not available it is
\code{None}.
\end{funcdesc}

\begin{funcdesc}{print_exception}{type\, value\, traceback\optional{\, limit}}
Print exception information and up to \var{limit} stack trace entries
from \var{traceback}.  This differs from \code{print_tb} in the
following ways: (1) if \var{traceback} is not \code{None}, it prints a
header ``\code{Traceback (innermost last):}''; (2) it prints the
exception \var{type} and \var{value} after the stack trace; (3) if
\var{type} is \code{SyntaxError} and \var{value} has the appropriate
format, it prints the line where the syntax error occurred with a
caret indication the approximate position of the error.
\end{funcdesc}

\begin{funcdesc}{print_exc}{\optional{limit}}
This is a shorthand for \code{print_exception(sys.exc_type,}
\code{sys.exc_value,} \code{sys.exc_traceback,} \code{limit)}.
\end{funcdesc}

\begin{funcdesc}{print_last}{\optional{limit}}
This is a shorthand for \code{print_exception(sys.last_type,}
\code{sys.last_value,} \code{sys.last_traceback,} \code{limit)}.
\end{funcdesc}

\section{Standard Module \sectcode{pickle}}
\stmodindex{pickle}
\index{persistency}
\indexii{persistent}{objects}
\indexii{serializing}{objects}
\indexii{marshalling}{objects}
\indexii{flattening}{objects}
\indexii{pickling}{objects}

\renewcommand{\indexsubitem}{(in module pickle)}

The \code{pickle} module implements a basic but powerful algorithm for
``pickling'' (a.k.a.\ serializing, marshalling or flattening) nearly
arbitrary Python objects.  This is the act of converting objects to a
stream of bytes (and back: ``unpickling'').
This is a more primitive notion than
persistency --- although \code{pickle} reads and writes file objects,
it does not handle the issue of naming persistent objects, nor the
(even more complicated) area of concurrent access to persistent
objects.  The \code{pickle} module can transform a complex object into
a byte stream and it can transform the byte stream into an object with
the same internal structure.  The most obvious thing to do with these
byte streams is to write them onto a file, but it is also conceivable
to send them across a network or store them in a database.  The module
\code{shelve} provides a simple interface to pickle and unpickle
objects on ``dbm''-style database files.
\stmodindex{shelve}

Unlike the built-in module \code{marshal}, \code{pickle} handles the
following correctly:
\stmodindex{marshal}

\begin{itemize}

\item recursive objects (objects containing references to themselves)

\item object sharing (references to the same object in different places)

\item user-defined classes and their instances

\end{itemize}

The data format used by \code{pickle} is Python-specific.  This has
the advantage that there are no restrictions imposed by external
standards such as CORBA (which probably can't represent pointer
sharing or recursive objects); however it means that non-Python
programs may not be able to reconstruct pickled Python objects.

The \code{pickle} data format uses a printable \ASCII{} representation.
This is slightly more voluminous than a binary representation.
However, small integers actually take {\em less} space when
represented as minimal-size decimal strings than when represented as
32-bit binary numbers, and strings are only much longer if they
contain many control characters or 8-bit characters.  The big
advantage of using printable \ASCII{} (and of some other characteristics
of \code{pickle}'s representation) is that for debugging or recovery
purposes it is possible for a human to read the pickled file with a
standard text editor.  (I could have gone a step further and used a
notation like S-expressions, but the parser
(currently written in Python) would have been
considerably more complicated and slower, and the files would probably
have become much larger.)

The \code{pickle} module doesn't handle code objects, which the
\code{marshal} module does.  I suppose \code{pickle} could, and maybe
it should, but there's probably no great need for it right now (as
long as \code{marshal} continues to be used for reading and writing
code objects), and at least this avoids the possibility of smuggling
Trojan horses into a program.
\stmodindex{marshal}

For the benefit of persistency modules written using \code{pickle}, it
supports the notion of a reference to an object outside the pickled
data stream.  Such objects are referenced by a name, which is an
arbitrary string of printable \ASCII{} characters.  The resolution of
such names is not defined by the \code{pickle} module --- the
persistent object module will have to implement a method
\code{persistent_load}.  To write references to persistent objects,
the persistent module must define a method \code{persistent_id} which
returns either \code{None} or the persistent ID of the object.

There are some restrictions on the pickling of class instances.

First of all, the class must be defined at the top level in a module.

\renewcommand{\indexsubitem}{(pickle protocol)}

Next, it must normally be possible to create class instances by
calling the class without arguments.  Usually, this is best
accomplished by providing default values for all arguments to its
\code{__init__} method (if it has one).  If this is undesirable, the
class can define a method \code{__getinitargs__()}, which should
return a {\em tuple} containing the arguments to be passed to the
class constructor (\code{__init__()}).
\ttindex{__getinitargs__}
\ttindex{__init__}

Classes can further influence how their instances are pickled --- if the class
defines the method \code{__getstate__()}, it is called and the return
state is pickled as the contents for the instance, and if the class
defines the method \code{__setstate__()}, it is called with the
unpickled state.  (Note that these methods can also be used to
implement copying class instances.)  If there is no
\code{__getstate__()} method, the instance's \code{__dict__} is
pickled.  If there is no \code{__setstate__()} method, the pickled
object must be a dictionary and its items are assigned to the new
instance's dictionary.  (If a class defines both \code{__getstate__()}
and \code{__setstate__()}, the state object needn't be a dictionary
--- these methods can do what they want.)  This protocol is also used
by the shallow and deep copying operations defined in the \code{copy}
module.
\ttindex{__getstate__}
\ttindex{__setstate__}
\ttindex{__dict__}

Note that when class instances are pickled, their class's code and
data are not pickled along with them.  Only the instance data are
pickled.  This is done on purpose, so you can fix bugs in a class or
add methods and still load objects that were created with an earlier
version of the class.  If you plan to have long-lived objects that
will see many versions of a class, it may be worthwhile to put a version
number in the objects so that suitable conversions can be made by the
class's \code{__setstate__()} method.

When a class itself is pickled, only its name is pickled --- the class
definition is not pickled, but re-imported by the unpickling process.
Therefore, the restriction that the class must be defined at the top
level in a module applies to pickled classes as well.

\renewcommand{\indexsubitem}{(in module pickle)}

The interface can be summarized as follows.

To pickle an object \code{x} onto a file \code{f}, open for writing:

\begin{verbatim}
p = pickle.Pickler(f)
p.dump(x)
\end{verbatim}

A shorthand for this is:

\begin{verbatim}
pickle.dump(x, f)
\end{verbatim}

To unpickle an object \code{x} from a file \code{f}, open for reading:

\begin{verbatim}
u = pickle.Unpickler(f)
x = u.load()
\end{verbatim}

A shorthand is:

\begin{verbatim}
x = pickle.load(f)
\end{verbatim}

The \code{Pickler} class only calls the method \code{f.write} with a
string argument.  The \code{Unpickler} calls the methods \code{f.read}
(with an integer argument) and \code{f.readline} (without argument),
both returning a string.  It is explicitly allowed to pass non-file
objects here, as long as they have the right methods.
\ttindex{Unpickler}
\ttindex{Pickler}

The following types can be pickled:
\begin{itemize}

\item \code{None}

\item integers, long integers, floating point numbers

\item strings

\item tuples, lists and dictionaries containing only picklable objects

\item classes that are defined at the top level in a module

\item instances of such classes whose \code{__dict__} or
\code{__setstate__()} is picklable

\end{itemize}

Attempts to pickle unpicklable objects will raise the
\code{PicklingError} exception; when this happens, an unspecified
number of bytes may have been written to the file.

It is possible to make multiple calls to the \code{dump()} method of
the same \code{Pickler} instance.  These must then be matched to the
same number of calls to the \code{load()} instance of the
corresponding \code{Unpickler} instance.  If the same object is
pickled by multiple \code{dump()} calls, the \code{load()} will all
yield references to the same object.  {\em Warning}: this is intended
for pickling multiple objects without intervening modifications to the
objects or their parts.  If you modify an object and then pickle it
again using the same \code{Pickler} instance, the object is not
pickled again --- a reference to it is pickled and the
\code{Unpickler} will return the old value, not the modified one.
(There are two problems here: (a) detecting changes, and (b)
marshalling a minimal set of changes.  I have no answers.  Garbage
Collection may also become a problem here.)

Apart from the \code{Pickler} and \code{Unpickler} classes, the
module defines the following functions, and an exception:

\begin{funcdesc}{dump}{object\, file}
Write a pickled representation of \var{obect} to the open file object
\var{file}.  This is equivalent to \code{Pickler(file).dump(object)}.
\end{funcdesc}

\begin{funcdesc}{load}{file}
Read a pickled object from the open file object \var{file}.  This is
equivalent to \code{Unpickler(file).load()}.
\end{funcdesc}

\begin{funcdesc}{dumps}{object}
Return the pickled representation of the object as a string, instead
of writing it to a file.
\end{funcdesc}

\begin{funcdesc}{loads}{string}
Read a pickled object from a string instead of a file.  Characters in
the string past the pickled object's representation are ignored.
\end{funcdesc}

\begin{excdesc}{PicklingError}
This exception is raised when an unpicklable object is passed to
\code{Pickler.dump()}.
\end{excdesc}

\section{Standard Module \sectcode{shelve}}
\stmodindex{shelve}
\stmodindex{pickle}
\bimodindex{dbm}
\bimodindex{gdbm}

A ``shelf'' is a persistent, dictionary-like object.  The difference
with ``dbm'' databases is that the values (not the keys!) in a shelf
can be essentially arbitrary Python objects --- anything that the
\code{pickle} module can handle.  This includes most class instances,
recursive data types, and objects containing lots of shared
sub-objects.  The keys are ordinary strings.

To summarize the interface (\code{key} is a string, \code{data} is an
arbitrary object):

\begin{verbatim}
import shelve

d = shelve.open(filename) # open, with (g)dbm filename -- no suffix

d[key] = data   # store data at key (overwrites old data if
                # using an existing key)
data = d[key]   # retrieve data at key (raise KeyError if no
                # such key)
del d[key]      # delete data stored at key (raises KeyError
                # if no such key)
flag = d.has_key(key)   # true if the key exists
list = d.keys() # a list of all existing keys (slow!)

d.close()       # close it
\end{verbatim}

Restrictions:

\begin{itemize}

\item
The choice of which database package will be used (e.g. dbm or gdbm)
depends on which interface is available.  Therefore it isn't safe to
open the database directly using dbm.  The database is also
(unfortunately) subject to the limitations of dbm, if it is used ---
this means that (the pickled representation of) the objects stored in
the database should be fairly small, and in rare cases key collisions
may cause the database to refuse updates.

\item
Dependent on the implementation, closing a persistent dictionary may
or may not be necessary to flush changes to disk.

\item
The \code{shelve} module does not support {\em concurrent} read/write
access to shelved objects.  (Multiple simultaneous read accesses are
safe.)  When a program has a shelf open for writing, no other program
should have it open for reading or writing.  \UNIX{} file locking can
be used to solve this, but this differs across \UNIX{} versions and
requires knowledge about the database implementation used.

\end{itemize}

\section{Standard Module \sectcode{copy}}
\stmodindex{copy}
\renewcommand{\indexsubitem}{(copy function)}
\ttindex{copy}
\ttindex{deepcopy}

This module provides generic (shallow and deep) copying operations.

Interface summary:

\begin{verbatim}
import copy

x = copy.copy(y)        # make a shallow copy of y
x = copy.deepcopy(y)    # make a deep copy of y
\end{verbatim}

For module specific errors, \code{copy.error} is raised.

The difference between shallow and deep copying is only relevant for
compound objects (objects that contain other objects, like lists or
class instances):

\begin{itemize}

\item
A {\em shallow copy} constructs a new compound object and then (to the
extent possible) inserts {\em references} into it to the objects found
in the original.

\item
A {\em deep copy} constructs a new compound object and then,
recursively, inserts {\em copies} into it of the objects found in the
original.

\end{itemize}

Two problems often exist with deep copy operations that don't exist
with shallow copy operations:

\begin{itemize}

\item
Recursive objects (compound objects that, directly or indirectly,
contain a reference to themselves) may cause a recursive loop.

\item
Because deep copy copies {\em everything} it may copy too much, e.g.\
administrative data structures that should be shared even between
copies.

\end{itemize}

Python's \code{deepcopy()} operation avoids these problems by:

\begin{itemize}

\item
keeping a table of objects already copied during the current
copying pass; and

\item
letting user-defined classes override the copying operation or the
set of components copied.

\end{itemize}

This version does not copy types like module, class, function, method,
nor stack trace, stack frame, nor file, socket, window, nor array, nor
any similar types.

Classes can use the same interfaces to control copying that they use
to control pickling: they can define methods called
\code{__getinitargs__()}, \code{__getstate__()} and
\code{__setstate__()}.  See the description of module \code{pickle}
for information on these methods.
\stmodindex{pickle}
\renewcommand{\indexsubitem}{(copy protocol)}
\ttindex{__getinitargs__}
\ttindex{__getstate__}
\ttindex{__setstate__}

\section{Built-in Module \sectcode{marshal}}

\bimodindex{marshal}
This module contains functions that can read and write Python
values in a binary format.  The format is specific to Python, but
independent of machine architecture issues (e.g., you can write a
Python value to a file on a PC, transport the file to a Sun, and read
it back there).  Details of the format are undocumented on purpose;
it may change between Python versions (although it rarely does).%
\footnote{The name of this module stems from a bit of terminology used
by the designers of Modula-3 (amongst others), who use the term
``marshalling'' for shipping of data around in a self-contained form.
Strictly speaking, ``to marshal'' means to convert some data from
internal to external form (in an RPC buffer for instance) and
``unmarshalling'' for the reverse process.}

This is not a general ``persistency'' module.  For general persistency
and transfer of Python objects through RPC calls, see the modules
\code{pickle} and \code{shelve}.  The \code{marshal} module exists
mainly to support reading and writing the ``pseudo-compiled'' code for
Python modules of \samp{.pyc} files.
\stmodindex{pickle}
\stmodindex{shelve}
\obindex{code}

Not all Python object types are supported; in general, only objects
whose value is independent from a particular invocation of Python can
be written and read by this module.  The following types are supported:
\code{None}, integers, long integers, floating point numbers,
strings, tuples, lists, dictionaries, and code objects, where it
should be understood that tuples, lists and dictionaries are only
supported as long as the values contained therein are themselves
supported; and recursive lists and dictionaries should not be written
(they will cause infinite loops).

{\bf Caveat:} On machines where C's \code{long int} type has more than
32 bits (such as the DEC Alpha), it
is possible to create plain Python integers that are longer than 32
bits.  Since the current \code{marshal} module uses 32 bits to
transfer plain Python integers, such values are silently truncated.
This particularly affects the use of very long integer literals in
Python modules --- these will be accepted by the parser on such
machines, but will be silently be truncated when the module is read
from the \code{.pyc} instead.%
\footnote{A solution would be to refuse such literals in the parser,
since they are inherently non-portable.  Another solution would be to
let the \code{marshal} module raise an exception when an integer value
would be truncated.  At least one of these solutions will be
implemented in a future version.}

There are functions that read/write files as well as functions
operating on strings.

The module defines these functions:

\renewcommand{\indexsubitem}{(in module marshal)}

\begin{funcdesc}{dump}{value\, file}
  Write the value on the open file.  The value must be a supported
  type.  The file must be an open file object such as
  \code{sys.stdout} or returned by \code{open()} or
  \code{posix.popen()}.
  
  If the value has (or contains an object that has) an unsupported type,
  a \code{ValueError} exception is raised -- but garbage data will also
  be written to the file.  The object will not be properly read back by
  \code{load()}.
\end{funcdesc}

\begin{funcdesc}{load}{file}
  Read one value from the open file and return it.  If no valid value
  is read, raise \code{EOFError}, \code{ValueError} or
  \code{TypeError}.  The file must be an open file object.

  Warning: If an object containing an unsupported type was marshalled
  with \code{dump()}, \code{load()} will substitute \code{None} for the
  unmarshallable type.
\end{funcdesc}

\begin{funcdesc}{dumps}{value}
  Return the string that would be written to a file by
  \code{dump(value, file)}.  The value must be a supported type.
  Raise a \code{ValueError} exception if value has (or contains an
  object that has) an unsupported type.
\end{funcdesc}

\begin{funcdesc}{loads}{string}
  Convert the string to a value.  If no valid value is found, raise
  \code{EOFError}, \code{ValueError} or \code{TypeError}.  Extra
  characters in the string are ignored.
\end{funcdesc}

\section{Built-in Module \sectcode{imp}}
\bimodindex{imp}
\index{import}

This module provides an interface to the mechanisms used to implement
the \code{import} statement.  It defines the following constants and
functions:

\renewcommand{\indexsubitem}{(in module imp)}

\begin{funcdesc}{get_magic}{}
Return the magic string value used to recognize byte-compiled code
files (``\code{.pyc} files'').
\end{funcdesc}

\begin{funcdesc}{get_suffixes}{}
Return a list of triples, each describing a particular type of file.
Each triple has the form \code{(\var{suffix}, \var{mode},
\var{type})}, where \var{suffix} is a string to be appended to the
module name to form the filename to search for, \var{mode} is the mode
string to pass to the built-in \code{open} function to open the file
(this can be \code{'r'} for text files or \code{'rb'} for binary
files), and \var{type} is the file type, which has one of the values
\code{PY_SOURCE}, \code{PY_COMPILED} or \code{C_EXTENSION}, defined
below.  (System-dependent values may also be returned.)
\end{funcdesc}

\begin{funcdesc}{find_module}{name\, \optional{path}}
Try to find the module \var{name} on the search path \var{path}.  The
default \var{path} is \code{sys.path}.  The return value is a triple
\code{(\var{file}, \var{pathname}, \var{description})} where
\var{file} is an open file object positioned at the beginning,
\var{pathname} is the pathname of the
file found, and \var{description} is a triple as contained in the list
returned by \code{get_suffixes} describing the kind of file found.
\end{funcdesc}

\begin{funcdesc}{init_builtin}{name}
Initialize the built-in module called \var{name} and return its module
object.  If the module was already initialized, it will be initialized
{\em again}.  A few modules cannot be initialized twice --- attempting
to initialize these again will raise an \code{ImportError} exception.
If there is no
built-in module called \var{name}, \code{None} is returned.
\end{funcdesc}

\begin{funcdesc}{init_frozen}{name}
Initialize the frozen module called \var{name} and return its module
object.  If the module was already initialized, it will be initialized
{\em again}.  If there is no frozen module called \var{name},
\code{None} is returned.  (Frozen modules are modules written in
Python whose compiled byte-code object is incorporated into a
custom-built Python interpreter by Python's \code{freeze} utility.
See \code{Tools/freeze} for now.)
\end{funcdesc}

\begin{funcdesc}{is_builtin}{name}
Return \code{1} if there is a built-in module called \var{name} which can be
initialized again.  Return \code{-1} if there is a built-in module
called \var{name} which cannot be initialized again (see
\code{init_builtin}).  Return \code{0} if there is no built-in module
called \var{name}.
\end{funcdesc}

\begin{funcdesc}{is_frozen}{name}
Return \code{1} if there is a frozen module (see \code{init_frozen})
called \var{name}, \code{0} if there is no such module.
\end{funcdesc}

\begin{funcdesc}{load_compiled}{name\, pathname\, file}
Load and initialize a module implemented as a byte-compiled code file
and return its module object.  If the module was already initialized,
it will be initialized {\em again}.  The \var{name} argument is used
to create or access a module object.  The \var{pathname} argument
points to the byte-compiled code file.  The \var{file}
argument is the byte-compiled code file, open for reading in binary
mode, from the beginning.
It must currently be a real file object, not a
user-defined class emulating a file.
\end{funcdesc}

\begin{funcdesc}{load_dynamic}{name\, pathname\, \optional{file}}
Load and initialize a module implemented as a dynamically loadable
shared library and return its module object.  If the module was
already initialized, it will be initialized {\em again}.  Some modules
don't like that and may raise an exception.  The \var{pathname}
argument must point to the shared library.  The \var{name} argument is
used to construct the name of the initialization function: an external
C function called \code{init\var{name}()} in the shared library is
called.  The optional \var{file} argment is ignored.  (Note: using
shared libraries is highly system dependent, and not all systems
support it.)
\end{funcdesc}

\begin{funcdesc}{load_source}{name\, pathname\, file}
Load and initialize a module implemented as a Python source file and
return its module object.  If the module was already initialized, it
will be initialized {\em again}.  The \var{name} argument is used to
create or access a module object.  The \var{pathname} argument points
to the source file.  The \var{file} argument is the source
file, open for reading as text, from the beginning.
It must currently be a real file
object, not a user-defined class emulating a file.  Note that if a
properly matching byte-compiled file (with suffix \code{.pyc}) exists,
it will be used instead of parsing the given source file.
\end{funcdesc}

\begin{funcdesc}{new_module}{name}
Return a new empty module object called \var{name}.  This object is
{\em not} inserted in \code{sys.modules}.
\end{funcdesc}

The following constants with integer values, defined in the module,
are used to indicate the search result of \code{imp.find_module}.

\begin{datadesc}{SEARCH_ERROR}
The module was not found.
\end{datadesc}

\begin{datadesc}{PY_SOURCE}
The module was found as a source file.
\end{datadesc}

\begin{datadesc}{PY_COMPILED}
The module was found as a compiled code object file.
\end{datadesc}

\begin{datadesc}{C_EXTENSION}
The module was found as dynamically loadable shared library.
\end{datadesc}

\subsection{Examples}
The following function emulates the default import statement:

\begin{verbatim}
import imp
import sys

def __import__(name, globals=None, locals=None, fromlist=None):
    # Fast path: see if the module has already been imported.
    if sys.modules.has_key(name):
        return sys.modules[name]

    # If any of the following calls raises an exception,
    # there's a problem we can't handle -- let the caller handle it.

    # See if it's a built-in module.
    m = imp.init_builtin(name)
    if m:
        return m

    # See if it's a frozen module.
    m = imp.init_frozen(name)
    if m:
        return m

    # Search the default path (i.e. sys.path).
    fp, pathname, (suffix, mode, type) = imp.find_module(name)

    # See what we got.
    try:
        if type == imp.C_EXTENSION:
            return imp.load_dynamic(name, pathname)
        if type == imp.PY_SOURCE:
            return imp.load_source(name, pathname, fp)
        if type == imp.PY_COMPILED:
            return imp.load_compiled(name, pathname, fp)

        # Shouldn't get here at all.
        raise ImportError, '%s: unknown module type (%d)' % (name, type)
    finally:
        # Since we may exit via an exception, close fp explicitly.
        fp.close()
\end{verbatim}

% libparser.tex
%
% Introductory documentation for the new parser built-in module.
%
% Copyright 1995 Virginia Polytechnic Institute and State University
% and Fred L. Drake, Jr.  This copyright notice must be distributed on
% all copies, but this document otherwise may be distributed as part
% of the Python distribution.  No fee may be charged for this document
% in any representation, either on paper or electronically.  This
% restriction does not affect other elements in a distributed package
% in any way.
%

\section{Built-in Module \sectcode{parser}}
\bimodindex{parser}

The \code{parser} module provides an interface to Python's internal
parser and byte-code compiler.  The primary purpose for this interface
is to allow Python code to edit the parse tree of a Python expression
and create executable code from this.  This is better than trying
to parse and modify an arbitrary Python code fragment as a string
because parsing is performed in a manner identical to the code
forming the application.  It is also faster.

There are a few things to note about this module which are important
to making use of the data structures created.  This is not a tutorial
on editing the parse trees for Python code, but some examples of using
the \code{parser} module are presented.

Most importantly, a good understanding of the Python grammar processed
by the internal parser is required.  For full information on the
language syntax, refer to the Language Reference.  The parser itself
is created from a grammar specification defined in the file
\file{Grammar/Grammar} in the standard Python distribution.  The parse
trees stored in the ``AST objects'' created by this module are the
actual output from the internal parser when created by the
\code{expr()} or \code{suite()} functions, described below.  The AST
objects created by \code{sequence2ast()} faithfully simulate those
structures.  Be aware that the values of the sequences which are
considered ``correct'' will vary from one version of Python to another
as the formal grammar for the language is revised.  However,
transporting code from one Python version to another as source text
will always allow correct parse trees to be created in the target
version, with the only restriction being that migrating to an older
version of the interpreter will not support more recent language
constructs.  The parse trees are not typically compatible from one
version to another, whereas source code has always been
forward-compatible.

Each element of the sequences returned by \code{ast2list} or
\code{ast2tuple()} has a simple form.  Sequences representing
non-terminal elements in the grammar always have a length greater than
one.  The first element is an integer which identifies a production in
the grammar.  These integers are given symbolic names in the C header
file \file{Include/graminit.h} and the Python module
\file{Lib/symbol.py}.  Each additional element of the sequence represents
a component of the production as recognized in the input string: these
are always sequences which have the same form as the parent.  An
important aspect of this structure which should be noted is that
keywords used to identify the parent node type, such as the keyword
\code{if} in an \code{if_stmt}, are included in the node tree without
any special treatment.  For example, the \code{if} keyword is
represented by the tuple \code{(1, 'if')}, where \code{1} is the
numeric value associated with all \code{NAME} tokens, including
variable and function names defined by the user.  In an alternate form
returned when line number information is requested, the same token
might be represented as \code{(1, 'if', 12)}, where the \code{12}
represents the line number at which the terminal symbol was found.

Terminal elements are represented in much the same way, but without
any child elements and the addition of the source text which was
identified.  The example of the \code{if} keyword above is
representative.  The various types of terminal symbols are defined in
the C header file \file{Include/token.h} and the Python module
\file{Lib/token.py}.

The AST objects are not required to support the functionality of this
module, but are provided for three purposes: to allow an application
to amortize the cost of processing complex parse trees, to provide a
parse tree representation which conserves memory space when compared
to the Python list or tuple representation, and to ease the creation
of additional modules in C which manipulate parse trees.  A simple
``wrapper'' class may be created in Python to hide the use of AST
objects; the \code{AST} library module provides a variety of such
classes.

The \code{parser} module defines functions for a few distinct
purposes.  The most important purposes are to create AST objects and
to convert AST objects to other representations such as parse trees
and compiled code objects, but there are also functions which serve to
query the type of parse tree represented by an AST object.

\renewcommand{\indexsubitem}{(in module parser)}


\subsection{Creating AST Objects}

AST objects may be created from source code or from a parse tree.
When creating an AST object from source, different functions are used
to create the \code{'eval'} and \code{'exec'} forms.

\begin{funcdesc}{expr}{string}
The \code{expr()} function parses the parameter \code{\var{string}}
as if it were an input to \code{compile(\var{string}, 'eval')}.  If
the parse succeeds, an AST object is created to hold the internal
parse tree representation, otherwise an appropriate exception is
thrown.
\end{funcdesc}

\begin{funcdesc}{suite}{string}
The \code{suite()} function parses the parameter \code{\var{string}}
as if it were an input to \code{compile(\var{string}, 'exec')}.  If
the parse succeeds, an AST object is created to hold the internal
parse tree representation, otherwise an appropriate exception is
thrown.
\end{funcdesc}

\begin{funcdesc}{sequence2ast}{sequence}
This function accepts a parse tree represented as a sequence and
builds an internal representation if possible.  If it can validate
that the tree conforms to the Python grammar and all nodes are valid
node types in the host version of Python, an AST object is created
from the internal representation and returned to the called.  If there
is a problem creating the internal representation, or if the tree
cannot be validated, a \code{ParserError} exception is thrown.  An AST
object created this way should not be assumed to compile correctly;
normal exceptions thrown by compilation may still be initiated when
the AST object is passed to \code{compileast()}.  This may indicate
problems not related to syntax (such as a \code{MemoryError}
exception), but may also be due to constructs such as the result of
parsing \code{del f(0)}, which escapes the Python parser but is
checked by the bytecode compiler.

Sequences representing terminal tokens may be represented as either
two-element lists of the form \code{(1, 'name')} or as three-element
lists of the form \code{(1, 'name', 56)}.  If the third element is
present, it is assumed to be a valid line number.  The line number
may be specified for any subset of the terminal symbols in the input
tree.
\end{funcdesc}

\begin{funcdesc}{tuple2ast}{sequence}
This is the same function as \code{sequence2ast()}.  This entry point
is maintained for backward compatibility.
\end{funcdesc}


\subsection{Converting AST Objects}

AST objects, regardless of the input used to create them, may be
converted to parse trees represented as list- or tuple- trees, or may
be compiled into executable code objects.  Parse trees may be
extracted with or without line numbering information.

\begin{funcdesc}{ast2list}{ast\optional{\, line_info\code{ = 0}}}
This function accepts an AST object from the caller in
\code{\var{ast}} and returns a Python list representing the
equivelent parse tree.  The resulting list representation can be used
for inspection or the creation of a new parse tree in list form.  This
function does not fail so long as memory is available to build the
list representation.  If the parse tree will only be used for
inspection, \code{ast2tuple()} should be used instead to reduce memory
consumption and fragmentation.  When the list representation is
required, this function is significantly faster than retrieving a
tuple representation and converting that to nested lists.

If \code{\var{line_info}} is true, line number information will be
included for all terminal tokens as a third element of the list
representing the token.  This information is omitted if the flag is
false or omitted.
\end{funcdesc}

\begin{funcdesc}{ast2tuple}{ast\optional{\, line_info\code{ = 0}}}
This function accepts an AST object from the caller in
\code{\var{ast}} and returns a Python tuple representing the
equivelent parse tree.  Other than returning a tuple instead of a
list, this function is identical to \code{ast2list()}.

If \code{\var{line_info}} is true, line number information will be
included for all terminal tokens as a third element of the list
representing the token.  This information is omitted if the flag is
false or omitted.
\end{funcdesc}

\begin{funcdesc}{compileast}{ast\optional{\, filename\code{ = '<ast>'}}}
The Python byte compiler can be invoked on an AST object to produce
code objects which can be used as part of an \code{exec} statement or
a call to the built-in \code{eval()} function.  This function provides
the interface to the compiler, passing the internal parse tree from
\code{\var{ast}} to the parser, using the source file name specified
by the \code{\var{filename}} parameter.  The default value supplied
for \code{\var{filename}} indicates that the source was an AST object.

Compiling an AST object may result in exceptions related to
compilation; an example would be a \code{SyntaxError} caused by the
parse tree for \code{del f(0)}: this statement is considered legal
within the formal grammar for Python but is not a legal language
construct.  The \code{SyntaxError} raised for this condition is
actually generated by the Python byte-compiler normally, which is why
it can be raised at this point by the \code{parser} module.  Most
causes of compilation failure can be diagnosed programmatically by
inspection of the parse tree.
\end{funcdesc}


\subsection{Queries on AST Objects}

Two functions are provided which allow an application to determine if
an AST was create as an expression or a suite.  Neither of these
functions can be used to determine if an AST was created from source
code via \code{expr()} or \code{suite()} or from a parse tree via
\code{sequence2ast()}.

\begin{funcdesc}{isexpr}{ast}
When \code{\var{ast}} represents an \code{'eval'} form, this function
returns a true value (\code{1}), otherwise it returns false
(\code{0}).  This is useful, since code objects normally cannot be
queried for this information using existing built-in functions.  Note
that the code objects created by \code{compileast()} cannot be queried
like this either, and are identical to those created by the built-in
\code{compile()} function.
\end{funcdesc}


\begin{funcdesc}{issuite}{ast}
This function mirrors \code{isexpr()} in that it reports whether an
AST object represents an \code{'exec'} form, commonly known as a
``suite.''  It is not safe to assume that this function is equivelent
to \code{not isexpr(\var{ast})}, as additional syntactic fragments may
be supported in the future.
\end{funcdesc}


\subsection{Exceptions and Error Handling}

The parser module defines a single exception, but may also pass other
built-in exceptions from other portions of the Python runtime
environment.  See each function for information about the exceptions
it can raise.

\begin{excdesc}{ParserError}
Exception raised when a failure occurs within the parser module.  This
is generally produced for validation failures rather than the built in
\code{SyntaxError} thrown during normal parsing.
The exception argument is either a string describing the reason of the
failure or a tuple containing a sequence causing the failure from a parse
tree passed to \code{sequence2ast()} and an explanatory string.  Calls to
\code{sequence2ast()} need to be able to handle either type of exception,
while calls to other functions in the module will only need to be
aware of the simple string values.
\end{excdesc}

Note that the functions \code{compileast()}, \code{expr()}, and
\code{suite()} may throw exceptions which are normally thrown by the
parsing and compilation process.  These include the built in
exceptions \code{MemoryError}, \code{OverflowError},
\code{SyntaxError}, and \code{SystemError}.  In these cases, these
exceptions carry all the meaning normally associated with them.  Refer
to the descriptions of each function for detailed information.


\subsection{AST Objects}

AST objects returned by \code{expr()}, \code{suite()}, and
\code{sequence2ast()} have no methods of their own.
Some of the functions defined which accept an AST object as their
first argument may change to object methods in the future.  The type
of these objects is available as \code{ASTType} in the module.

Ordered and equality comparisons are supported between AST objects.


\subsection{Examples}

The parser modules allows operations to be performed on the parse tree
of Python source code before the bytecode is generated, and provides
for inspection of the parse tree for information gathering purposes.
Two examples are presented.  The simple example demonstrates emulation
of the \code{compile()} built-in function and the complex example
shows the use of a parse tree for information discovery.

\subsubsection{Emulation of \sectcode{compile()}}

While many useful operations may take place between parsing and
bytecode generation, the simplest operation is to do nothing.  For
this purpose, using the \code{parser} module to produce an
intermediate data structure is equivelent to the code

\begin{verbatim}
>>> code = compile('a + 5', 'eval')
>>> a = 5
>>> eval(code)
10
\end{verbatim}

The equivelent operation using the \code{parser} module is somewhat
longer, and allows the intermediate internal parse tree to be retained
as an AST object:

\begin{verbatim}
>>> import parser
>>> ast = parser.expr('a + 5')
>>> code = parser.compileast(ast)
>>> a = 5
>>> eval(code)
10
\end{verbatim}

An application which needs both AST and code objects can package this
code into readily available functions:

\begin{verbatim}
import parser

def load_suite(source_string):
    ast = parser.suite(source_string)
    code = parser.compileast(ast)
    return ast, code

def load_expression(source_string):
    ast = parser.expr(source_string)
    code = parser.compileast(ast)
    return ast, code
\end{verbatim}

\subsubsection{Information Discovery}

Some applications benefit from direct access to the parse tree.  The
remainder of this section demonstrates how the parse tree provides
access to module documentation defined in docstrings without requiring
that the code being examined be loaded into a running interpreter via
\code{import}.  This can be very useful for performing analyses of
untrusted code.

Generally, the example will demonstrate how the parse tree may be
traversed to distill interesting information.  Two functions and a set
of classes are developed which provide programmatic access to high
level function and class definitions provided by a module.  The
classes extract information from the parse tree and provide access to
the information at a useful semantic level, one function provides a
simple low-level pattern matching capability, and the other function
defines a high-level interface to the classes by handling file
operations on behalf of the caller.  All source files mentioned here
which are not part of the Python installation are located in the
\file{Demo/parser/} directory of the distribution.

The dynamic nature of Python allows the programmer a great deal of
flexibility, but most modules need only a limited measure of this when
defining classes, functions, and methods.  In this example, the only
definitions that will be considered are those which are defined in the
top level of their context, e.g., a function defined by a \code{def}
statement at column zero of a module, but not a function defined
within a branch of an \code{if} ... \code{else} construct, though
there are some good reasons for doing so in some situations.  Nesting
of definitions will be handled by the code developed in the example.

To construct the upper-level extraction methods, we need to know what
the parse tree structure looks like and how much of it we actually
need to be concerned about.  Python uses a moderately deep parse tree
so there are a large number of intermediate nodes.  It is important to
read and understand the formal grammar used by Python.  This is
specified in the file \file{Grammar/Grammar} in the distribution.
Consider the simplest case of interest when searching for docstrings:
a module consisting of a docstring and nothing else.  (See file
\file{docstring.py}.)

\begin{verbatim}
"""Some documentation.
"""
\end{verbatim}

Using the interpreter to take a look at the parse tree, we find a
bewildering mass of numbers and parentheses, with the documentation
buried deep in nested tuples.

\begin{verbatim}
>>> import parser
>>> import pprint
>>> ast = parser.suite(open('docstring.py').read())
>>> tup = parser.ast2tuple(ast)
>>> pprint.pprint(tup)
(257,
 (264,
  (265,
   (266,
    (267,
     (307,
      (287,
       (288,
        (289,
         (290,
          (292,
           (293,
            (294,
             (295,
              (296,
               (297,
                (298,
                 (299,
                  (300, (3, '"""Some documentation.\012"""'))))))))))))))))),
   (4, ''))),
 (4, ''),
 (0, ''))
\end{verbatim}

The numbers at the first element of each node in the tree are the node
types; they map directly to terminal and non-terminal symbols in the
grammar.  Unfortunately, they are represented as integers in the
internal representation, and the Python structures generated do not
change that.  However, the \code{symbol} and \code{token} modules
provide symbolic names for the node types and dictionaries which map
from the integers to the symbolic names for the node types.

In the output presented above, the outermost tuple contains four
elements: the integer \code{257} and three additional tuples.  Node
type \code{257} has the symbolic name \code{file_input}.  Each of
these inner tuples contains an integer as the first element; these
integers, \code{264}, \code{4}, and \code{0}, represent the node types
\code{stmt}, \code{NEWLINE}, and \code{ENDMARKER}, respectively.
Note that these values may change depending on the version of Python
you are using; consult \file{symbol.py} and \file{token.py} for
details of the mapping.  It should be fairly clear that the outermost
node is related primarily to the input source rather than the contents
of the file, and may be disregarded for the moment.  The \code{stmt}
node is much more interesting.  In particular, all docstrings are
found in subtrees which are formed exactly as this node is formed,
with the only difference being the string itself.  The association
between the docstring in a similar tree and the defined entity (class,
function, or module) which it describes is given by the position of
the docstring subtree within the tree defining the described
structure.

By replacing the actual docstring with something to signify a variable
component of the tree, we allow a simple pattern matching approach to
check any given subtree for equivelence to the general pattern for
docstrings.  Since the example demonstrates information extraction, we
can safely require that the tree be in tuple form rather than list
form, allowing a simple variable representation to be
\code{['variable_name']}.  A simple recursive function can implement
the pattern matching, returning a boolean and a dictionary of variable
name to value mappings.  (See file \file{example.py}.)

\begin{verbatim}
from types import ListType, TupleType

def match(pattern, data, vars=None):
    if vars is None:
        vars = {}
    if type(pattern) is ListType:
        vars[pattern[0]] = data
        return 1, vars
    if type(pattern) is not TupleType:
        return (pattern == data), vars
    if len(data) != len(pattern):
        return 0, vars
    for pattern, data in map(None, pattern, data):
        same, vars = match(pattern, data, vars)
        if not same:
            break
    return same, vars
\end{verbatim}

Using this simple representation for syntactic variables and the symbolic
node types, the pattern for the candidate docstring subtrees becomes
fairly readable.  (See file \file{example.py}.)

\begin{verbatim}
import symbol
import token

DOCSTRING_STMT_PATTERN = (
    symbol.stmt,
    (symbol.simple_stmt,
     (symbol.small_stmt,
      (symbol.expr_stmt,
       (symbol.testlist,
        (symbol.test,
         (symbol.and_test,
          (symbol.not_test,
           (symbol.comparison,
            (symbol.expr,
             (symbol.xor_expr,
              (symbol.and_expr,
               (symbol.shift_expr,
                (symbol.arith_expr,
                 (symbol.term,
                  (symbol.factor,
                   (symbol.power,
                    (symbol.atom,
                     (token.STRING, ['docstring'])
                     )))))))))))))))),
     (token.NEWLINE, '')
     ))
\end{verbatim}

Using the \code{match()} function with this pattern, extracting the
module docstring from the parse tree created previously is easy:

\begin{verbatim}
>>> found, vars = match(DOCSTRING_STMT_PATTERN, tup[1])
>>> found
1
>>> vars
{'docstring': '"""Some documentation.\012"""'}
\end{verbatim}

Once specific data can be extracted from a location where it is
expected, the question of where information can be expected
needs to be answered.  When dealing with docstrings, the answer is
fairly simple: the docstring is the first \code{stmt} node in a code
block (\code{file_input} or \code{suite} node types).  A module
consists of a single \code{file_input} node, and class and function
definitions each contain exactly one \code{suite} node.  Classes and
functions are readily identified as subtrees of code block nodes which
start with \code{(stmt, (compound_stmt, (classdef, ...} or
\code{(stmt, (compound_stmt, (funcdef, ...}.  Note that these subtrees
cannot be matched by \code{match()} since it does not support multiple
sibling nodes to match without regard to number.  A more elaborate
matching function could be used to overcome this limitation, but this
is sufficient for the example.

Given the ability to determine whether a statement might be a
docstring and extract the actual string from the statement, some work
needs to be performed to walk the parse tree for an entire module and
extract information about the names defined in each context of the
module and associate any docstrings with the names.  The code to
perform this work is not complicated, but bears some explanation.

The public interface to the classes is straightforward and should
probably be somewhat more flexible.  Each ``major'' block of the
module is described by an object providing several methods for inquiry
and a constructor which accepts at least the subtree of the complete
parse tree which it represents.  The \code{ModuleInfo} constructor
accepts an optional \code{\var{name}} parameter since it cannot
otherwise determine the name of the module.

The public classes include \code{ClassInfo}, \code{FunctionInfo},
and \code{ModuleInfo}.  All objects provide the
methods \code{get_name()}, \code{get_docstring()},
\code{get_class_names()}, and \code{get_class_info()}.  The
\code{ClassInfo} objects support \code{get_method_names()} and
\code{get_method_info()} while the other classes provide
\code{get_function_names()} and \code{get_function_info()}.

Within each of the forms of code block that the public classes
represent, most of the required information is in the same form and is
accessed in the same way, with classes having the distinction that
functions defined at the top level are referred to as ``methods.''
Since the difference in nomenclature reflects a real semantic
distinction from functions defined outside of a class, the
implementation needs to maintain the distinction.
Hence, most of the functionality of the public classes can be
implemented in a common base class, \code{SuiteInfoBase}, with the
accessors for function and method information provided elsewhere.
Note that there is only one class which represents function and method
information; this paralels the use of the \code{def} statement to
define both types of elements.

Most of the accessor functions are declared in \code{SuiteInfoBase}
and do not need to be overriden by subclasses.  More importantly, the
extraction of most information from a parse tree is handled through a
method called by the \code{SuiteInfoBase} constructor.  The example
code for most of the classes is clear when read alongside the formal
grammar, but the method which recursively creates new information
objects requires further examination.  Here is the relevant part of
the \code{SuiteInfoBase} definition from \file{example.py}:

\begin{verbatim}
class SuiteInfoBase:
    _docstring = ''
    _name = ''

    def __init__(self, tree = None):
        self._class_info = {}
        self._function_info = {}
        if tree:
            self._extract_info(tree)

    def _extract_info(self, tree):
        # extract docstring
        if len(tree) == 2:
            found, vars = match(DOCSTRING_STMT_PATTERN[1], tree[1])
        else:
            found, vars = match(DOCSTRING_STMT_PATTERN, tree[3])
        if found:
            self._docstring = eval(vars['docstring'])
        # discover inner definitions
        for node in tree[1:]:
            found, vars = match(COMPOUND_STMT_PATTERN, node)
            if found:
                cstmt = vars['compound']
                if cstmt[0] == symbol.funcdef:
                    name = cstmt[2][1]
                    self._function_info[name] = FunctionInfo(cstmt)
                elif cstmt[0] == symbol.classdef:
                    name = cstmt[2][1]
                    self._class_info[name] = ClassInfo(cstmt)
\end{verbatim}

After initializing some internal state, the constructor calls the
\code{_extract_info()} method.  This method performs the bulk of the
information extraction which takes place in the entire example.  The
extraction has two distinct phases: the location of the docstring for
the parse tree passed in, and the discovery of additional definitions
within the code block represented by the parse tree.

The initial \code{if} test determines whether the nested suite is of
the ``short form'' or the ``long form.''  The short form is used when
the code block is on the same line as the definition of the code
block, as in

\begin{verbatim}
def square(x): "Square an argument."; return x ** 2
\end{verbatim}

while the long form uses an indented block and allows nested
definitions:

\begin{verbatim}
def make_power(exp):
    "Make a function that raises an argument to the exponent `exp'."
    def raiser(x, y=exp):
        return x ** y
    return raiser
\end{verbatim}

When the short form is used, the code block may contain a docstring as
the first, and possibly only, \code{small_stmt} element.  The
extraction of such a docstring is slightly different and requires only
a portion of the complete pattern used in the more common case.  As
implemented, the docstring will only be found if there is only
one \code{small_stmt} node in the \code{simple_stmt} node.  Since most
functions and methods which use the short form do not provide a
docstring, this may be considered sufficient.  The extraction of the
docstring proceeds using the \code{match()} function as described
above, and the value of the docstring is stored as an attribute of the
\code{SuiteInfoBase} object.

After docstring extraction, a simple definition discovery
algorithm operates on the \code{stmt} nodes of the \code{suite} node.  The
special case of the short form is not tested; since there are no
\code{stmt} nodes in the short form, the algorithm will silently skip
the single \code{simple_stmt} node and correctly not discover any
nested definitions.

Each statement in the code block is categorized as
a class definition, function or method definition, or
something else.  For the definition statements, the name of the
element defined is extracted and a representation object
appropriate to the definition is created with the defining subtree
passed as an argument to the constructor.  The repesentation objects
are stored in instance variables and may be retrieved by name using
the appropriate accessor methods.

The public classes provide any accessors required which are more
specific than those provided by the \code{SuiteInfoBase} class, but
the real extraction algorithm remains common to all forms of code
blocks.  A high-level function can be used to extract the complete set
of information from a source file.  (See file \file{example.py}.)

\begin{verbatim}
def get_docs(fileName):
    source = open(fileName).read()
    import os
    basename = os.path.basename(os.path.splitext(fileName)[0])
    import parser
    ast = parser.suite(source)
    tup = parser.ast2tuple(ast)
    return ModuleInfo(tup, basename)
\end{verbatim}

This provides an easy-to-use interface to the documentation of a
module.  If information is required which is not extracted by the code
of this example, the code may be extended at clearly defined points to
provide additional capabilities.


%%
%%  end of file

\section{Built-in Module \sectcode{__builtin__}}
\bimodindex{__builtin__}

This module provides direct access to all `built-in' identifiers of
Python; e.g. \code{__builtin__.open} is the full name for the built-in
function \code{open}.  See the section on Built-in Functions in the
previous chapter.
		% really __builtin__
\section{Built-in Module \sectcode{__main__}}

\bimodindex{__main__}
This module represents the (otherwise anonymous) scope in which the
interpreter's main program executes --- commands read either from
standard input or from a script file.
			% really __main__

\chapter{String Services}

The modules described in this chapter provide a wide range of string
manipulation operations.  Here's an overview:

\begin{description}

\item[string]
--- Common string operations.

\item[regex]
--- Regular expression search and match operations.

\item[regsub]
--- Substitution and splitting operations that use regular expressions.

\item[struct]
--- Interpret strings as packed binary data.

\end{description}
		% String Services
\section{Standard Module \sectcode{string}}

\stmodindex{string}

This module defines some constants useful for checking character
classes and some useful string functions.  See the modules
\code{regex} and \code{regsub} for string functions based on regular
expressions.

The constants defined in this module are are:

\renewcommand{\indexsubitem}{(data in module string)}
\begin{datadesc}{digits}
  The string \code{'0123456789'}.
\end{datadesc}

\begin{datadesc}{hexdigits}
  The string \code{'0123456789abcdefABCDEF'}.
\end{datadesc}

\begin{datadesc}{letters}
  The concatenation of the strings \code{lowercase} and
  \code{uppercase} described below.
\end{datadesc}

\begin{datadesc}{lowercase}
  A string containing all the characters that are considered lowercase
  letters.  On most systems this is the string
  \code{'abcdefghijklmnopqrstuvwxyz'}.  Do not change its definition ---
  the effect on the routines \code{upper} and \code{swapcase} is
  undefined.
\end{datadesc}

\begin{datadesc}{octdigits}
  The string \code{'01234567'}.
\end{datadesc}

\begin{datadesc}{uppercase}
  A string containing all the characters that are considered uppercase
  letters.  On most systems this is the string
  \code{'ABCDEFGHIJKLMNOPQRSTUVWXYZ'}.  Do not change its definition ---
  the effect on the routines \code{lower} and \code{swapcase} is
  undefined.
\end{datadesc}

\begin{datadesc}{whitespace}
  A string containing all characters that are considered whitespace.
  On most systems this includes the characters space, tab, linefeed,
  return, formfeed, and vertical tab.  Do not change its definition ---
  the effect on the routines \code{strip} and \code{split} is
  undefined.
\end{datadesc}

The functions defined in this module are:

\renewcommand{\indexsubitem}{(in module string)}

\begin{funcdesc}{atof}{s}
Convert a string to a floating point number.  The string must have
the standard syntax for a floating point literal in Python, optionally
preceded by a sign (\samp{+} or \samp{-}).
\end{funcdesc}

\begin{funcdesc}{atoi}{s\optional{\, base}}
Convert string \var{s} to an integer in the given \var{base}.  The
string must consist of one or more digits, optionally preceded by a
sign (\samp{+} or \samp{-}).  The \var{base} defaults to 10.  If it is
0, a default base is chosen depending on the leading characters of the
string (after stripping the sign): \samp{0x} or \samp{0X} means 16,
\samp{0} means 8, anything else means 10.  If \var{base} is 16, a
leading \samp{0x} or \samp{0X} is always accepted.  (Note: for a more
flexible interpretation of numeric literals, use the built-in function
\code{eval()}.)
\bifuncindex{eval}
\end{funcdesc}

\begin{funcdesc}{atol}{s\optional{\, base}}
Convert string \var{s} to a long integer in the given \var{base}.  The
string must consist of one or more digits, optionally preceded by a
sign (\samp{+} or \samp{-}).  The \var{base} argument has the same
meaning as for \code{atoi()}.  A trailing \samp{l} or \samp{L} is not
allowed, except if the base is 0.
\end{funcdesc}

\begin{funcdesc}{capitalize}{word}
Capitalize the first character of the argument.
\end{funcdesc}

\begin{funcdesc}{capwords}{s}
Split the argument into words using \code{split}, capitalize each word
using \code{capitalize}, and join the capitalized words using
\code{join}.  Note that this replaces runs of whitespace characters by
a single space.  (See also \code{regsub.capwords()} for a version
that doesn't change the delimiters, and lets you specify a word
separator.)
\end{funcdesc}

\begin{funcdesc}{expandtabs}{s\, tabsize}
Expand tabs in a string, i.e.\ replace them by one or more spaces,
depending on the current column and the given tab size.  The column
number is reset to zero after each newline occurring in the string.
This doesn't understand other non-printing characters or escape
sequences.
\end{funcdesc}

\begin{funcdesc}{find}{s\, sub\optional{\, start}}
Return the lowest index in \var{s} not smaller than \var{start} where the
substring \var{sub} is found.  Return \code{-1} when \var{sub}
does not occur as a substring of \var{s} with index at least \var{start}.
If \var{start} is omitted, it defaults to \code{0}.  If \var{start} is
negative, \code{len(\var{s})} is added.
\end{funcdesc}

\begin{funcdesc}{rfind}{s\, sub\optional{\, start}}
Like \code{find} but find the highest index.
\end{funcdesc}

\begin{funcdesc}{index}{s\, sub\optional{\, start}}
Like \code{find} but raise \code{ValueError} when the substring is
not found.
\end{funcdesc}

\begin{funcdesc}{rindex}{s\, sub\optional{\, start}}
Like \code{rfind} but raise \code{ValueError} when the substring is
not found.
\end{funcdesc}

\begin{funcdesc}{count}{s\, sub\optional{\, start}}
Return the number of (non-overlapping) occurrences of substring
\var{sub} in string \var{s} with index at least \var{start}.
If \var{start} is omitted, it defaults to \code{0}.  If \var{start} is
negative, \code{len(\var{s})} is added.
\end{funcdesc}

\begin{funcdesc}{lower}{s}
Convert letters to lower case.
\end{funcdesc}

\begin{funcdesc}{maketrans}{from, to}
Return a translation table suitable for passing to \code{string.translate}
or \code{regex.compile}, that will map each character in \var{from} 
into the character at the same position in \var{to}; \var{from} and
\var{to} must have the same length. 
\end{funcdesc}

\begin{funcdesc}{split}{s\optional{\, sep\optional{\, maxsplit}}}
Return a list of the words of the string \var{s}.  If the optional
second argument \var{sep} is absent or \code{None}, the words are
separated by arbitrary strings of whitespace characters (space, tab,
newline, return, formfeed).  If the second argument \var{sep} is
present and not \code{None}, it specifies a string to be used as the
word separator.  The returned list will then have one more items than
the number of non-overlapping occurrences of the separator in the
string.  The optional third argument \var{maxsplit} defaults to 0.  If
it is nonzero, at most \var{maxsplit} number of splits occur, and the
remainder of the string is returned as the final element of the list
(thus, the list will have at most \code{\var{maxsplit}+1} elements).
(See also \code{regsub.split()} for a version that allows specifying a
regular expression as the separator.)
\end{funcdesc}

\begin{funcdesc}{splitfields}{s\optional{\, sep\optional{\, maxsplit}}}
This function behaves identical to \code{split}.  (In the past,
\code{split} was only used with one argument, while \code{splitfields}
was only used with two arguments.)
\end{funcdesc}

\begin{funcdesc}{join}{words\optional{\, sep}}
Concatenate a list or tuple of words with intervening occurrences of
\var{sep}.  The default value for \var{sep} is a single space character.
It is always true that
\code{string.join(string.split(\var{s}, \var{sep}), \var{sep})}
equals \var{s}.
\end{funcdesc}

\begin{funcdesc}{joinfields}{words\optional{\, sep}}
This function behaves identical to \code{join}.  (In the past,
\code{join} was only used with one argument, while \code{joinfields}
was only used with two arguments.)
\end{funcdesc}

\begin{funcdesc}{lstrip}{s}
Remove leading whitespace from the string \var{s}.
\end{funcdesc}

\begin{funcdesc}{rstrip}{s}
Remove trailing whitespace from the string \var{s}.
\end{funcdesc}

\begin{funcdesc}{strip}{s}
Remove leading and trailing whitespace from the string \var{s}.
\end{funcdesc}

\begin{funcdesc}{swapcase}{s}
Convert lower case letters to upper case and vice versa.
\end{funcdesc}

\begin{funcdesc}{translate}{s, table\optional{, deletechars}}
Delete all characters from \var{s} that are in \var{deletechars} (if present), and 
then translate the characters using \var{table}, which must be
a 256-character string giving the translation for each character
value, indexed by its ordinal.  
\end{funcdesc}

\begin{funcdesc}{upper}{s}
Convert letters to upper case.
\end{funcdesc}

\begin{funcdesc}{ljust}{s\, width}
\funcline{rjust}{s\, width}
\funcline{center}{s\, width}
These functions respectively left-justify, right-justify and center a
string in a field of given width.
They return a string that is at least
\var{width}
characters wide, created by padding the string
\var{s}
with spaces until the given width on the right, left or both sides.
The string is never truncated.
\end{funcdesc}

\begin{funcdesc}{zfill}{s\, width}
Pad a numeric string on the left with zero digits until the given
width is reached.  Strings starting with a sign are handled correctly.
\end{funcdesc}

This module is implemented in Python.  Much of its functionality has
been reimplemented in the built-in module \code{strop}.  However, you
should \emph{never} import the latter module directly.  When
\code{string} discovers that \code{strop} exists, it transparently
replaces parts of itself with the implementation from \code{strop}.
After initialization, there is \emph{no} overhead in using
\code{string} instead of \code{strop}.
\bimodindex{strop}

\section{Built-in Module \sectcode{regex}}

\bimodindex{regex}
This module provides regular expression matching operations similar to
those found in Emacs.  It is always available.

By default the patterns are Emacs-style regular expressions
(with one exception).  There is
a way to change the syntax to match that of several well-known
\UNIX{} utilities.  The exception is that Emacs' \samp{\e s}
pattern is not supported, since the original implementation references
the Emacs syntax tables.

This module is 8-bit clean: both patterns and strings may contain null
bytes and characters whose high bit is set.

\strong{Please note:} There is a little-known fact about Python string
literals which means that you don't usually have to worry about
doubling backslashes, even though they are used to escape special
characters in string literals as well as in regular expressions.  This
is because Python doesn't remove backslashes from string literals if
they are followed by an unrecognized escape character.
\emph{However}, if you want to include a literal \dfn{backslash} in a
regular expression represented as a string literal, you have to
\emph{quadruple} it.  E.g.\  to extract \LaTeX\ \samp{\e section\{{\rm
\ldots}\}} headers from a document, you can use this pattern:
\code{'\e \e \e \e section\{\e (.*\e )\}'}.  \emph{Another exception:}
the escape sequece \samp{\e b} is significant in string literals
(where it means the ASCII bell character) as well as in Emacs regular
expressions (where it stands for a word boundary), so in order to
search for a word boundary, you should use the pattern \code{'\e \e b'}.
Similarly, a backslash followed by a digit 0-7 should be doubled to
avoid interpretation as an octal escape.

\subsection{Regular Expressions}

A regular expression (or RE) specifies a set of strings that matches
it; the functions in this module let you check if a particular string
matches a given regular expression (or if a given regular expression
matches a particular string, which comes down to the same thing).

Regular expressions can be concatenated to form new regular
expressions; if \emph{A} and \emph{B} are both regular expressions,
then \emph{AB} is also an regular expression.  If a string \emph{p}
matches A and another string \emph{q} matches B, the string \emph{pq}
will match AB.  Thus, complex expressions can easily be constructed
from simpler ones like the primitives described here.  For details of
the theory and implementation of regular expressions, consult almost
any textbook about compiler construction.

% XXX The reference could be made more specific, say to 
% "Compilers: Principles, Techniques and Tools", by Alfred V. Aho, 
% Ravi Sethi, and Jeffrey D. Ullman, or some FA text.   

A brief explanation of the format of regular expressions follows.

Regular expressions can contain both special and ordinary characters.
Ordinary characters, like '\code{A}', '\code{a}', or '\code{0}', are
the simplest regular expressions; they simply match themselves.  You
can concatenate ordinary characters, so '\code{last}' matches the
characters 'last'.  (In the rest of this section, we'll write RE's in
\code{this special font}, usually without quotes, and strings to be
matched 'in single quotes'.)

Special characters either stand for classes of ordinary characters, or
affect how the regular expressions around them are interpreted.

The special characters are:
\begin{itemize}
\item[\code{.}]{(Dot.)  Matches any character except a newline.}
\item[\code{\^}]{(Caret.)  Matches the start of the string.}
\item[\code{\$}]{Matches the end of the string.  
\code{foo} matches both 'foo' and 'foobar', while the regular
expression '\code{foo\$}' matches only 'foo'.}
\item[\code{*}] Causes the resulting RE to
match 0 or more repetitions of the preceding RE.  \code{ab*} will
match 'a', 'ab', or 'a' followed by any number of 'b's.
\item[\code{+}] Causes the
resulting RE to match 1 or more repetitions of the preceding RE.
\code{ab+} will match 'a' followed by any non-zero number of 'b's; it
will not match just 'a'.
\item[\code{?}] Causes the resulting RE to
match 0 or 1 repetitions of the preceding RE.  \code{ab?} will
match either 'a' or 'ab'.

\item[\code{\e}] Either escapes special characters (permitting you to match
characters like '*?+\&\$'), or signals a special sequence; special
sequences are discussed below.  Remember that Python also uses the
backslash as an escape sequence in string literals; if the escape
sequence isn't recognized by Python's parser, the backslash and
subsequent character are included in the resulting string.  However,
if Python would recognize the resulting sequence, the backslash should
be repeated twice.  

\item[\code{[]}] Used to indicate a set of characters.  Characters can
be listed individually, or a range is indicated by giving two
characters and separating them by a '-'.  Special characters are
not active inside sets.  For example, \code{[akm\$]}
will match any of the characters 'a', 'k', 'm', or '\$'; \code{[a-z]} will
match any lowercase letter.  

If you want to include a \code{]} inside a
set, it must be the first character of the set; to include a \code{-},
place it as the first or last character. 

Characters \emph{not} within a range can be matched by including a
\code{\^} as the first character of the set; \code{\^} elsewhere will
simply match the '\code{\^}' character.  
\end{itemize}

The special sequences consist of '\code{\e}' and a character
from the list below.  If the ordinary character is not on the list,
then the resulting RE will match the second character.  For example,
\code{\e\$} matches the character '\$'.  Ones where the backslash
should be doubled are indicated.

\begin{itemize}
\item[\code{\e|}]\code{A\e|B}, where A and B can be arbitrary REs,
creates a regular expression that will match either A or B.  This can
be used inside groups (see below) as well.
%
\item[\code{\e( \e)}]{Indicates the start and end of a group; the
contents of a group can be matched later in the string with the
\code{\e \[1-9]} special sequence, described next.}
%
{\fulllineitems\item[\code{\e \e 1, ... \e \e 7, \e 8, \e 9}]
{Matches the contents of the group of the same
number.  For example, \code{\e (.+\e ) \e \e 1} matches 'the the' or
'55 55', but not 'the end' (note the space after the group).  This
special sequence can only be used to match one of the first 9 groups;
groups with higher numbers can be matched using the \code{\e v}
sequence.  (\code{\e 8} and \code{\e 9} don't need a double backslash
because they are not octal digits.)}}
%
\item[\code{\e \e b}]{Matches the empty string, but only at the
beginning or end of a word.  A word is defined as a sequence of
alphanumeric characters, so the end of a word is indicated by
whitespace or a non-alphanumeric character.}
%
\item[\code{\e B}]{Matches the empty string, but when it is \emph{not} at the
beginning or end of a word.} 
%
\item[\code{\e v}]{Must be followed by a two digit decimal number, and
matches the contents of the group of the same number.  The group number must be between 1 and 99, inclusive.}
%
\item[\code{\e w}]Matches any alphanumeric character; this is
equivalent to the set \code{[a-zA-Z0-9]}.
%
\item[\code{\e W}]{Matches any non-alphanumeric character; this is
equivalent to the set \code{[\^a-zA-Z0-9]}.} 
\item[\code{\e <}]{Matches the empty string, but only at the beginning of a
word.  A word is defined as a sequence of alphanumeric characters, so
the end of a word is indicated by whitespace or a non-alphanumeric 
character.}
\item[\code{\e >}]{Matches the empty string, but only at the end of a
word.}

\item[\code{\e \e \e \e}]{Matches a literal backslash.}

% In Emacs, the following two are start of buffer/end of buffer.  In
% Python they seem to be synonyms for ^$.
\item[\code{\e `}]{Like \code{\^}, this only matches at the start of the
string.}
\item[\code{\e \e '}] Like \code{\$}, this only matches at the end of the
string.
% end of buffer
\end{itemize}

\subsection{Module Contents}

The module defines these functions, and an exception:

\renewcommand{\indexsubitem}{(in module regex)}

\begin{funcdesc}{match}{pattern\, string}
  Return how many characters at the beginning of \var{string} match
  the regular expression \var{pattern}.  Return \code{-1} if the
  string does not match the pattern (this is different from a
  zero-length match!).
\end{funcdesc}

\begin{funcdesc}{search}{pattern\, string}
  Return the first position in \var{string} that matches the regular
  expression \var{pattern}.  Return \code{-1} if no position in the string
  matches the pattern (this is different from a zero-length match
  anywhere!).
\end{funcdesc}

\begin{funcdesc}{compile}{pattern\optional{\, translate}}
  Compile a regular expression pattern into a regular expression
  object, which can be used for matching using its \code{match} and
  \code{search} methods, described below.  The optional argument
  \var{translate}, if present, must be a 256-character string
  indicating how characters (both of the pattern and of the strings to
  be matched) are translated before comparing them; the \code{i}-th
  element of the string gives the translation for the character with
  \ASCII{} code \code{i}.  This can be used to implement
  case-insensitive matching; see the \code{casefold} data item below.

  The sequence

\bcode\begin{verbatim}
prog = regex.compile(pat)
result = prog.match(str)
\end{verbatim}\ecode

is equivalent to

\bcode\begin{verbatim}
result = regex.match(pat, str)
\end{verbatim}\ecode

but the version using \code{compile()} is more efficient when multiple
regular expressions are used concurrently in a single program.  (The
compiled version of the last pattern passed to \code{regex.match()} or
\code{regex.search()} is cached, so programs that use only a single
regular expression at a time needn't worry about compiling regular
expressions.)
\end{funcdesc}

\begin{funcdesc}{set_syntax}{flags}
  Set the syntax to be used by future calls to \code{compile},
  \code{match} and \code{search}.  (Already compiled expression objects
  are not affected.)  The argument is an integer which is the OR of
  several flag bits.  The return value is the previous value of
  the syntax flags.  Names for the flags are defined in the standard
  module \code{regex_syntax}; read the file \file{regex_syntax.py} for
  more information.
\end{funcdesc}

\begin{funcdesc}{symcomp}{pattern\optional{\, translate}}
This is like \code{compile}, but supports symbolic group names: if a
parenthesis-enclosed group begins with a group name in angular
brackets, e.g. \code{'\e(<id>[a-z][a-z0-9]*\e)'}, the group can
be referenced by its name in arguments to the \code{group} method of
the resulting compiled regular expression object, like this:
\code{p.group('id')}.  Group names may contain alphanumeric characters
and \code{'_'} only.
\end{funcdesc}

\begin{excdesc}{error}
  Exception raised when a string passed to one of the functions here
  is not a valid regular expression (e.g., unmatched parentheses) or
  when some other error occurs during compilation or matching.  (It is
  never an error if a string contains no match for a pattern.)
\end{excdesc}

\begin{datadesc}{casefold}
A string suitable to pass as \var{translate} argument to
\code{compile} to map all upper case characters to their lowercase
equivalents.
\end{datadesc}

\noindent
Compiled regular expression objects support these methods:

\renewcommand{\indexsubitem}{(regex method)}
\begin{funcdesc}{match}{string\optional{\, pos}}
  Return how many characters at the beginning of \var{string} match
  the compiled regular expression.  Return \code{-1} if the string
  does not match the pattern (this is different from a zero-length
  match!).
  
  The optional second parameter \var{pos} gives an index in the string
  where the search is to start; it defaults to \code{0}.  This is not
  completely equivalent to slicing the string; the \code{'\^'} pattern
  character matches at the real begin of the string and at positions
  just after a newline, not necessarily at the index where the search
  is to start.
\end{funcdesc}

\begin{funcdesc}{search}{string\optional{\, pos}}
  Return the first position in \var{string} that matches the regular
  expression \code{pattern}.  Return \code{-1} if no position in the
  string matches the pattern (this is different from a zero-length
  match anywhere!).
  
  The optional second parameter has the same meaning as for the
  \code{match} method.
\end{funcdesc}

\begin{funcdesc}{group}{index\, index\, ...}
This method is only valid when the last call to the \code{match}
or \code{search} method found a match.  It returns one or more
groups of the match.  If there is a single \var{index} argument,
the result is a single string; if there are multiple arguments, the
result is a tuple with one item per argument.  If the \var{index} is
zero, the corresponding return value is the entire matching string; if
it is in the inclusive range [1..99], it is the string matching the
the corresponding parenthesized group (using the default syntax,
groups are parenthesized using \code{\\(} and \code{\\)}).  If no
such group exists, the corresponding result is \code{None}.

If the regular expression was compiled by \code{symcomp} instead of
\code{compile}, the \var{index} arguments may also be strings
identifying groups by their group name.
\end{funcdesc}

\noindent
Compiled regular expressions support these data attributes:

\renewcommand{\indexsubitem}{(regex attribute)}

\begin{datadesc}{regs}
When the last call to the \code{match} or \code{search} method found a
match, this is a tuple of pairs of indices corresponding to the
beginning and end of all parenthesized groups in the pattern.  Indices
are relative to the string argument passed to \code{match} or
\code{search}.  The 0-th tuple gives the beginning and end or the
whole pattern.  When the last match or search failed, this is
\code{None}.
\end{datadesc}

\begin{datadesc}{last}
When the last call to the \code{match} or \code{search} method found a
match, this is the string argument passed to that method.  When the
last match or search failed, this is \code{None}.
\end{datadesc}

\begin{datadesc}{translate}
This is the value of the \var{translate} argument to
\code{regex.compile} that created this regular expression object.  If
the \var{translate} argument was omitted in the \code{regex.compile}
call, this is \code{None}.
\end{datadesc}

\begin{datadesc}{givenpat}
The regular expression pattern as passed to \code{compile} or
\code{symcomp}.
\end{datadesc}

\begin{datadesc}{realpat}
The regular expression after stripping the group names for regular
expressions compiled with \code{symcomp}.  Same as \code{givenpat}
otherwise.
\end{datadesc}

\begin{datadesc}{groupindex}
A dictionary giving the mapping from symbolic group names to numerical
group indices for regular expressions compiled with \code{symcomp}.
\code{None} otherwise.
\end{datadesc}

\section{Standard Module \sectcode{regsub}}

\stmodindex{regsub}
This module defines a number of functions useful for working with
regular expressions (see built-in module \code{regex}).

Warning: these functions are not thread-safe.

\renewcommand{\indexsubitem}{(in module regsub)}

\begin{funcdesc}{sub}{pat\, repl\, str}
Replace the first occurrence of pattern \var{pat} in string
\var{str} by replacement \var{repl}.  If the pattern isn't found,
the string is returned unchanged.  The pattern may be a string or an
already compiled pattern.  The replacement may contain references
\samp{\e \var{digit}} to subpatterns and escaped backslashes.
\end{funcdesc}

\begin{funcdesc}{gsub}{pat\, repl\, str}
Replace all (non-overlapping) occurrences of pattern \var{pat} in
string \var{str} by replacement \var{repl}.  The same rules as for
\code{sub()} apply.  Empty matches for the pattern are replaced only
when not adjacent to a previous match, so e.g.
\code{gsub('', '-', 'abc')} returns \code{'-a-b-c-'}.
\end{funcdesc}

\begin{funcdesc}{split}{str\, pat\optional{\, maxsplit}}
Split the string \var{str} in fields separated by delimiters matching
the pattern \var{pat}, and return a list containing the fields.  Only
non-empty matches for the pattern are considered, so e.g.
\code{split('a:b', ':*')} returns \code{['a', 'b']} and
\code{split('abc', '')} returns \code{['abc']}.  The \var{maxsplit}
defaults to 0. If it is nonzero, only \var{maxsplit} number of splits
occur, and the remainder of the string is returned as the final
element of the list.
\end{funcdesc}

\begin{funcdesc}{splitx}{str\, pat\optional{\, maxsplit}}
Split the string \var{str} in fields separated by delimiters matching
the pattern \var{pat}, and return a list containing the fields as well
as the separators.  For example, \code{splitx('a:::b', ':*')} returns
\code{['a', ':::', 'b']}.  Otherwise, this function behaves the same
as \code{split}.
\end{funcdesc}

\begin{funcdesc}{capwords}{s\optional{\, pat}}
Capitalize words separated by optional pattern \var{pat}.  The default
pattern uses any characters except letters, digits and underscores as
word delimiters.  Capitalization is done by changing the first
character of each word to upper case.
\end{funcdesc}

\section{Built-in Module \sectcode{struct}}
\bimodindex{struct}
\indexii{C}{structures}

This module performs conversions between Python values and C
structs represented as Python strings.  It uses \dfn{format strings}
(explained below) as compact descriptions of the lay-out of the C
structs and the intended conversion to/from Python values.

See also built-in module \code{array}.
\bimodindex{array}

The module defines the following exception and functions:

\renewcommand{\indexsubitem}{(in module struct)}
\begin{excdesc}{error}
  Exception raised on various occasions; argument is a string
  describing what is wrong.
\end{excdesc}

\begin{funcdesc}{pack}{fmt\, v1\, v2\, {\rm \ldots}}
  Return a string containing the values
  \code{\var{v1}, \var{v2}, {\rm \ldots}} packed according to the given
  format.  The arguments must match the values required by the format
  exactly.
\end{funcdesc}

\begin{funcdesc}{unpack}{fmt\, string}
  Unpack the string (presumably packed by \code{pack(\var{fmt}, {\rm \ldots})})
  according to the given format.  The result is a tuple even if it
  contains exactly one item.  The string must contain exactly the
  amount of data required by the format (i.e.  \code{len(\var{string})} must
  equal \code{calcsize(\var{fmt})}).
\end{funcdesc}

\begin{funcdesc}{calcsize}{fmt}
  Return the size of the struct (and hence of the string)
  corresponding to the given format.
\end{funcdesc}

Format characters have the following meaning; the conversion between C
and Python values should be obvious given their types:

\begin{tableiii}{|c|l|l|}{samp}{Format}{C}{Python}
  \lineiii{x}{pad byte}{no value}
  \lineiii{c}{char}{string of length 1}
  \lineiii{b}{signed char}{integer}
  \lineiii{h}{short}{integer}
  \lineiii{i}{int}{integer}
  \lineiii{l}{long}{integer}
  \lineiii{f}{float}{float}
  \lineiii{d}{double}{float}
\end{tableiii}

A format character may be preceded by an integral repeat count; e.g.\
the format string \code{'4h'} means exactly the same as \code{'hhhh'}.

C numbers are represented in the machine's native format and byte
order, and properly aligned by skipping pad bytes if necessary
(according to the rules used by the C compiler).

Examples (all on a big-endian machine):

\bcode\begin{verbatim}
pack('hhl', 1, 2, 3) == '\000\001\000\002\000\000\000\003'
unpack('hhl', '\000\001\000\002\000\000\000\003') == (1, 2, 3)
calcsize('hhl') == 8
\end{verbatim}\ecode

Hint: to align the end of a structure to the alignment requirement of
a particular type, end the format with the code for that type with a
repeat count of zero, e.g.\ the format \code{'llh0l'} specifies two
pad bytes at the end, assuming longs are aligned on 4-byte boundaries.

(More format characters are planned, e.g.\ \code{'s'} for character
arrays, upper case for unsigned variants, and a way to specify the
byte order, which is useful for [de]constructing network packets and
reading/writing portable binary file formats like TIFF and AIFF.)


\chapter{Miscellaneous Services}

The modules described in this chapter provide miscellaneous services
that are available in all Python versions.  Here's an overview:

\begin{description}

\item[math]
--- Mathematical functions (\code{sin()} etc.).

\item[rand]
--- Integer random number generator.

\item[whrandom]
--- Floating point random number generator.

\item[array]
--- Efficient arrays of uniformly typed numeric values.

\end{description}
			% Miscellaneous Services
\section{Built-in Module \sectcode{math}}

\bimodindex{math}
\renewcommand{\indexsubitem}{(in module math)}
This module is always available.
It provides access to the mathematical functions defined by the C
standard.
They are:
\iftexi
\begin{funcdesc}{acos}{x}
\funcline{asin}{x}
\funcline{atan}{x}
\funcline{atan2}{x, y}
\funcline{ceil}{x}
\funcline{cos}{x}
\funcline{cosh}{x}
\funcline{exp}{x}
\funcline{fabs}{x}
\funcline{floor}{x}
\funcline{fmod}{x, y}
\funcline{frexp}{x}
\funcline{hypot}{x, y}
\funcline{ldexp}{x, y}
\funcline{log}{x}
\funcline{log10}{x}
\funcline{modf}{x}
\funcline{pow}{x, y}
\funcline{sin}{x}
\funcline{sinh}{x}
\funcline{sqrt}{x}
\funcline{tan}{x}
\funcline{tanh}{x}
\end{funcdesc}
\else
\code{acos(\varvars{x})},
\code{asin(\varvars{x})},
\code{atan(\varvars{x})},
\code{atan2(\varvars{x\, y})},
\code{ceil(\varvars{x})},
\code{cos(\varvars{x})},
\code{cosh(\varvars{x})},
\code{exp(\varvars{x})},
\code{fabs(\varvars{x})},
\code{floor(\varvars{x})},
\code{fmod(\varvars{x\, y})},
\code{frexp(\varvars{x})},
\code{hypot(\varvars{x\, y})},
\code{ldexp(\varvars{x\, y})},
\code{log(\varvars{x})},
\code{log10(\varvars{x})},
\code{modf(\varvars{x})},
\code{pow(\varvars{x\, y})},
\code{sin(\varvars{x})},
\code{sinh(\varvars{x})},
\code{sqrt(\varvars{x})},
\code{tan(\varvars{x})},
\code{tanh(\varvars{x})}.
\fi

Note that \code{frexp} and \code{modf} have a different call/return
pattern than their C equivalents: they take a single argument and
return a pair of values, rather than returning their second return
value through an `output parameter' (there is no such thing in Python).

The module also defines two mathematical constants:
\iftexi
\begin{datadesc}{pi}
\dataline{e}
\end{datadesc}
\else
\code{pi} and \code{e}.
\fi

\section{Standard Module \sectcode{rand}}

\stmodindex{rand} This module implements a pseudo-random number
generator with an interface similar to \code{rand()} in C\@.  It defines
the following functions:

\renewcommand{\indexsubitem}{(in module rand)}
\begin{funcdesc}{rand}{}
Returns an integer random number in the range [0 ... 32768).
\end{funcdesc}

\begin{funcdesc}{choice}{s}
Returns a random element from the sequence (string, tuple or list)
\var{s}.
\end{funcdesc}

\begin{funcdesc}{srand}{seed}
Initializes the random number generator with the given integral seed.
When the module is first imported, the random number is initialized with
the current time.
\end{funcdesc}

\section{Standard Module \sectcode{whrandom}}

\stmodindex{whrandom}
This module implements a Wichmann-Hill pseudo-random number generator.
It defines the following functions:

\renewcommand{\indexsubitem}{(in module whrandom)}
\begin{funcdesc}{random}{}
Returns the next random floating point number in the range [0.0 ... 1.0).
\end{funcdesc}

\begin{funcdesc}{seed}{x\, y\, z}
Initializes the random number generator from the integers
\var{x},
\var{y}
and
\var{z}.
When the module is first imported, the random number is initialized
using values derived from the current time.
\end{funcdesc}

\section{Built-in Module \sectcode{array}}
\bimodindex{array}
\index{arrays}

This module defines a new object type which can efficiently represent
an array of basic values: characters, integers, floating point
numbers.  Arrays are sequence types and behave very much like lists,
except that the type of objects stored in them is constrained.  The
type is specified at object creation time by using a \dfn{type code},
which is a single character.  The following type codes are defined:

\begin{tableiii}{|c|c|c|}{code}{Typecode}{Type}{Minimal size in bytes}
\lineiii{'c'}{character}{1}
\lineiii{'b'}{signed integer}{1}
\lineiii{'h'}{signed integer}{2}
\lineiii{'i'}{signed integer}{2}
\lineiii{'l'}{signed integer}{4}
\lineiii{'f'}{floating point}{4}
\lineiii{'d'}{floating point}{8}
\end{tableiii}

The actual representation of values is determined by the machine
architecture (strictly speaking, by the C implementation).  The actual
size can be accessed through the \var{itemsize} attribute.

See also built-in module \code{struct}.
\bimodindex{struct}

The module defines the following function:

\renewcommand{\indexsubitem}{(in module array)}

\begin{funcdesc}{array}{typecode\optional{\, initializer}}
Return a new array whose items are restricted by \var{typecode}, and
initialized from the optional \var{initializer} value, which must be a
list or a string.  The list or string is passed to the new array's
\code{fromlist()} or \code{fromstring()} method (see below) to add
initial items to the array.
\end{funcdesc}

Array objects support the following data items and methods:

\begin{datadesc}{typecode}
The typecode character used to create the array.
\end{datadesc}

\begin{datadesc}{itemsize}
The length in bytes of one array item in the internal representation.
\end{datadesc}

\begin{funcdesc}{append}{x}
Append a new item with value \var{x} to the end of the array.
\end{funcdesc}

\begin{funcdesc}{byteswap}{x}
``Byteswap'' all items of the array.  This is only supported for
integer values.  It is useful when reading data from a file written
on a machine with a different byte order.
\end{funcdesc}

\begin{funcdesc}{fromfile}{f\, n}
Read \var{n} items (as machine values) from the file object \var{f}
and append them to the end of the array.  If less than \var{n} items
are available, \code{EOFError} is raised, but the items that were
available are still inserted into the array.  \var{f} must be a real
built-in file object; something else with a \code{read()} method won't
do.
\end{funcdesc}

\begin{funcdesc}{fromlist}{list}
Append items from the list.  This is equivalent to
\code{for x in \var{list}:\ a.append(x)}
except that if there is a type error, the array is unchanged.
\end{funcdesc}

\begin{funcdesc}{fromstring}{s}
Appends items from the string, interpreting the string as an
array of machine values (i.e. as if it had been read from a
file using the \code{fromfile()} method).
\end{funcdesc}

\begin{funcdesc}{insert}{i\, x}
Insert a new item with value \var{x} in the array before position
\var{i}.
\end{funcdesc}

\begin{funcdesc}{tofile}{f}
Write all items (as machine values) to the file object \var{f}.
\end{funcdesc}

\begin{funcdesc}{tolist}{}
Convert the array to an ordinary list with the same items.
\end{funcdesc}

\begin{funcdesc}{tostring}{}
Convert the array to an array of machine values and return the
string representation (the same sequence of bytes that would
be written to a file by the \code{tofile()} method.)
\end{funcdesc}

When an array object is printed or converted to a string, it is
represented as \code{array(\var{typecode}, \var{initializer})}.  The
\var{initializer} is omitted if the array is empty, otherwise it is a
string if the \var{typecode} is \code{'c'}, otherwise it is a list of
numbers.  The string is guaranteed to be able to be converted back to
an array with the same type and value using reverse quotes
(\code{``}).  Examples:

\bcode\begin{verbatim}
array('l')
array('c', 'hello world')
array('l', [1, 2, 3, 4, 5])
array('d', [1.0, 2.0, 3.14])
\end{verbatim}\ecode


\chapter{Generic Operating System Services}

The modules described in this chapter provide interfaces to operating
system features that are available on (almost) all operating systems,
such as files and a clock.  The interfaces are generally modelled
after the \UNIX{} or C interfaces but they are available on most other
systems as well.  Here's an overview:

\begin{description}

\item[os]
--- Miscellaneous OS interfaces.

\item[time]
--- Time access and conversions.

\item[getopt]
--- Parser for command line options.

\item[tempfile]
--- Generate temporary file names.

\end{description}
		% Generic Operating System Services
\section{Standard Module \sectcode{os}}

\stmodindex{os}
This module provides a more portable way of using operating system
(OS) dependent functionality than importing an OS dependent built-in
module like \code{posix}.

When the optional built-in module \code{posix} is available, this
module exports the same functions and data as \code{posix}; otherwise,
it searches for an OS dependent built-in module like \code{mac} and
exports the same functions and data as found there.  The design of all
Python's built-in OS dependent modules is such that as long as the same
functionality is available, it uses the same interface; e.g., the
function \code{os.stat(\var{file})} returns stat info about a \var{file} in a
format compatible with the POSIX interface.

Extensions peculiar to a particular OS are also available through the
\code{os} module, but using them is of course a threat to portability!

Note that after the first time \code{os} is imported, there is \emph{no}
performance penalty in using functions from \code{os} instead of
directly from the OS dependent built-in module, so there should be
\emph{no} reason not to use \code{os}!

In addition to whatever the correct OS dependent module exports, the
following variables and functions are always exported by \code{os}:

\renewcommand{\indexsubitem}{(in module os)}

\begin{datadesc}{name}
The name of the OS dependent module imported.  The following names
have currently been registered: \code{'posix'}, \code{'nt'},
\code{'dos'}, \code{'mac'}.
\end{datadesc}

\begin{datadesc}{path}
The corresponding OS dependent standard module for pathname
operations, e.g., \code{posixpath} or \code{macpath}.  Thus, (given
the proper imports), \code{os.path.split(\var{file})} is equivalent to but
more portable than \code{posixpath.split(\var{file})}.
\end{datadesc}

\begin{datadesc}{curdir}
The constant string used by the OS to refer to the current directory,
e.g. \code{'.'} for POSIX or \code{':'} for the Mac.
\end{datadesc}

\begin{datadesc}{pardir}
The constant string used by the OS to refer to the parent directory,
e.g. \code{'..'} for POSIX or \code{'::'} for the Mac.
\end{datadesc}

\begin{datadesc}{sep}
The character used by the OS to separate pathname components, e.g.\
\code{'/'} for POSIX or \code{':'} for the Mac.  Note that knowing this
is not sufficient to be able to parse or concatenate pathnames---better
use \code{os.path.split()} and \code{os.path.join()}---but it is
occasionally useful.
\end{datadesc}

\begin{datadesc}{pathsep}
The character conventionally used by the OS to separate search patch
components (as in \code{\$PATH}), e.g.\ \code{':'} for POSIX or
\code{';'} for MS-DOS.
\end{datadesc}

\begin{datadesc}{defpath}
The default search path used by \code{os.exec*p*()} if the environment
doesn't have a \code{'PATH'} key.
\end{datadesc}

\begin{funcdesc}{execl}{path\, arg0\, arg1\, ...}
This is equivalent to
\code{os.execv(\var{path}, (\var{arg0}, \var{arg1}, ...))}.
\end{funcdesc}

\begin{funcdesc}{execle}{path\, arg0\, arg1\, ...\, env}
This is equivalent to
\code{os.execve(\var{path}, (\var{arg0}, \var{arg1}, ...), \var{env})}.
\end{funcdesc}

\begin{funcdesc}{execlp}{path\, arg0\, arg1\, ...}
This is equivalent to
\code{os.execvp(\var{path}, (\var{arg0}, \var{arg1}, ...))}.
\end{funcdesc}

\begin{funcdesc}{execvp}{path\, args}
This is like \code{os.execv(\var{path}, \var{args})} but duplicates
the shell's actions in searching for an executable file in a list of
directories.  The directory list is obtained from
\code{os.environ['PATH']}.
\end{funcdesc}

\begin{funcdesc}{execvpe}{path\, args\, env}
This is a cross between \code{os.execve()} and \code{os.execvp()}.
The directory list is obtained from \code{\var{env}['PATH']}.
\end{funcdesc}

(The functions \code{os.execv()} and \code{execve()} are not
documented here, since they are implemented by the OS dependent
module.  If the OS dependent module doesn't define either of these,
the functions that rely on it will raise an exception.  They are
documented in the section on module \code{posix}, together with all
other functions that \code{os} imports from the OS dependent module.)

\section{Built-in Module \sectcode{time}}

\bimodindex{time}
This module provides various time-related functions.
It is always available.

An explanation of some terminology and conventions is in order.

\begin{itemize}

\item
The ``epoch'' is the point where the time starts.  On January 1st of that
year, at 0 hours, the ``time since the epoch'' is zero.  For UNIX, the
epoch is 1970.  To find out what the epoch is, look at \code{gmtime(0)}.

\item
UTC is Coordinated Universal Time (formerly known as Greenwich Mean
Time).  The acronym UTC is not a mistake but a compromise between
English and French.

\item
DST is Daylight Saving Time, an adjustment of the timezone by
(usually) one hour during part of the year.  DST rules are magic
(determined by local law) and can change from year to year.  The C
library has a table containing the local rules (often it is read from
a system file for flexibility) and is the only source of True Wisdom
in this respect.

\item
The precision of the various real-time functions may be less than
suggested by the units in which their value or argument is expressed.
E.g.\ on most UNIX systems, the clock ``ticks'' only 50 or 100 times a
second, and on the Mac, times are only accurate to whole seconds.

\item
The time tuple as returned by \code{gmtime()} and \code{localtime()},
or as accpted by \code{mktime()} is a tuple of 9
integers: year (e.g.\ 1993), month (1--12), day (1--31), hour
(0--23), minute (0--59), second (0--59), weekday (0--6, monday is 0),
Julian day (1--366) and daylight savings flag (-1, 0  or 1).
Note that unlike the C structure, the month value is a range of 1-12, not
0-11.  A year value of $<$ 100 will typically be silently converted to
1900 $+$ year value.  A -1 argument as daylight savings flag, passed to
\code{mktime()} will usually result in the correct daylight savings
state to be filled in.


\end{itemize}

The module defines the following functions and data items:

\renewcommand{\indexsubitem}{(in module time)}

\begin{datadesc}{altzone}
The offset of the local DST timezone, in seconds west of the 0th
meridian, if one is defined.  Negative if the local DST timezone is
east of the 0th meridian (as in Western Europe, including the UK).
Only use this if \code{daylight} is nonzero.
\end{datadesc}

\begin{funcdesc}{asctime}{tuple}
Convert a tuple representing a time as returned by \code{gmtime()} or
\code{localtime()} to a 24-character string of the following form:
\code{'Sun Jun 20 23:21:05 1993'}.  Note: unlike the C function of
the same name, there is no trailing newline.
\end{funcdesc}

\begin{funcdesc}{clock}{}
Return the current CPU time as a floating point number expressed in
seconds.  The precision, and in fact the very definiton of the meaning
of ``CPU time'', depends on that of the C function of the same name.
\end{funcdesc}

\begin{funcdesc}{ctime}{secs}
Convert a time expressed in seconds since the epoch to a string
representing local time.  \code{ctime(t)} is equivalent to
\code{asctime(localtime(t))}.
\end{funcdesc}

\begin{datadesc}{daylight}
Nonzero if a DST timezone is defined.
\end{datadesc}

\begin{funcdesc}{gmtime}{secs}
Convert a time expressed in seconds since the epoch to a time tuple
in UTC in which the dst flag is always zero.  Fractions of a second are
ignored.
\end{funcdesc}

\begin{funcdesc}{localtime}{secs}
Like \code{gmtime} but converts to local time.  The dst flag is set
to 1 when DST applies to the given time.
\end{funcdesc}

\begin{funcdesc}{mktime}{tuple}
This is the inverse function of \code{localtime}.  Its argument is the
full 9-tuple (since the dst flag is needed --- pass -1 as the dst flag if
it is unknown) which expresses the time
in \em{local} time, not UTC.  It returns a floating
point number, for compatibility with \code{time.time()}.  If the input
value can't be represented as a valid time, OverflowError is raised.
\end{funcdesc}

\begin{funcdesc}{sleep}{secs}
Suspend execution for the given number of seconds.  The argument may
be a floating point number to indicate a more precise sleep time.
\end{funcdesc}

\begin{funcdesc}{strftime}{format, tuple}
Convert a tuple representing a time as returned by \code{gmtime()} or
\code{localtime()} to a string as specified by the format argument.

      The following directives, shown without the optional field width and
      precision specification, are replaced by the indicated characters:

\begin{tabular}{lp{25em}}
           \%a  &      Locale's abbreviated weekday name. \\
           \%A  &      Locale's full weekday name. \\
           \%b  &      Locale's abbreviated month name. \\
           \%B  &      Locale's full month name. \\
           \%c  &      Locale's appropriate date and time representation. \\
           \%d  &      Day of the month as a decimal number [01,31]. \\
           \%E  &      Locale's combined Emperor/Era name and year. \\
           \%H  &      Hour (24-hour clock) as a decimal number [00,23]. \\
           \%I  &      Hour (12-hour clock) as a decimal number [01,12]. \\
           \%j  &      Day of the year as a decimal number [001,366]. \\
           \%m  &      Month as a decimal number [01,12]. \\
           \%M  &      Minute as a decimal number [00,59]. \\
           \%n  &      New-line character. \\
           \%N  &      Locale's Emperor/Era name. \\
           \%o  &      Locale's Emperor/Era year. \\
           \%p  &      Locale's equivalent of either AM or PM. \\
           \%S  &      Second as a decimal number [00,61]. \\
           \%t  &      Tab character. \\
           \%U  &      Week number of the year (Sunday as the first day of the
                     week) as a decimal number [00,53].  All days in a new
                     year preceding the first Sunday are considered to be in
                     week 0. \\
           \%w  &      Weekday as a decimal number [0(Sunday),6]. \\
           \%W  &      Week number of the year (Monday as the first day of the
                     week) as a decimal number [00,53].  All days in a new
                     year preceding the first Sunday are considered to be in
                     week 0. \\
           \%x  &      Locale's appropriate date representation. \\
           \%X  &      Locale's appropriate time representation. \\
           \%y  &      Year without century as a decimal number [00,99]. \\
           \%Y  &      Year with century as a decimal number. \\
           \%Z  &      Time zone name (or by no characters if no time zone
                     exists). \\
           \%\%  &     \% \\
\end{tabular}

      An optional field width and precision specification can immediately
      follow the initial \% of a directive in the following order: \\

\begin{tabular}{lp{25em}}
      [-|0]w  &       the decimal digit string w specifies a minimum field
                     width in which the result of the conversion is right-
                     or left-justified.  It is right-justified (with space
                     padding) by default.  If the optional flag `-' is
                     specified, it is left-justified with space padding on
                     the right.  If the optional flag `0' is specified, it
                     is right-justified and padded with zeros on the left. \\
      .p      &       the decimal digit string p specifies the minimum number
                     of digits to appear for the d, H, I, j, m, M, o, S, U,
                     w, W, y and Y directives, and the maximum number of
                     characters to be used from the a, A, b, B, c, D, E, F,
                     h, n, N, p, r, t, T, x, X, z, Z, and % directives.  In
                     the first case, if a directive supplies fewer digits
                     than specified by the precision, it will be expanded
                     with leading zeros.  In the second case, if a directive
                     supplies more characters than specified by the
                     precision, excess characters will truncated on the
                     right.
\end{tabular}

      If no field width or precision is specified for a d, H, I, m, M, S, U,
      W, y, or j directive, a default of .2 is used for all but j for which
      .3 is used.

\end{funcdesc}

\begin{funcdesc}{time}{}
Return the time as a floating point number expressed in seconds since
the epoch, in UTC.  Note that even though the time is always returned
as a floating point number, not all systems provide time with a better
precision than 1 second.
\end{funcdesc}

\begin{datadesc}{timezone}
The offset of the local (non-DST) timezone, in seconds west of the 0th
meridian (i.e. negative in most of Western Europe, positive in the US,
zero in the UK).
\end{datadesc}

\begin{datadesc}{tzname}
A tuple of two strings: the first is the name of the local non-DST
timezone, the second is the name of the local DST timezone.  If no DST
timezone is defined, the second string should not be used.
\end{datadesc}


\section{Standard Module \sectcode{getopt}}

\stmodindex{getopt}
This module helps scripts to parse the command line arguments in
\code{sys.argv}.
It supports the same conventions as the \UNIX{}
\code{getopt()}
function (including the special meanings of arguments of the form
\samp{-} and \samp{--}).  Long options similar to those supported by
GNU software may be used as well via an optional third argument.
It defines the function
\code{getopt.getopt(args, options [, long_options])}
and the exception
\code{getopt.error}.

The first argument to
\code{getopt()}
is the argument list passed to the script with its first element
chopped off (i.e.,
\code{sys.argv[1:]}).
The second argument is the string of option letters that the
script wants to recognize, with options that require an argument
followed by a colon (i.e., the same format that \UNIX{}
\code{getopt()}
uses).
The third option, if specified, is a list of strings with the names of
the long options which should be supported.  The leading \code{'--'}
characters should not be included in the option name.  Options which
require an argument should be followed by an equal sign (\code{'='}).
The return value consists of two elements: the first is a list of
option-and-value pairs; the second is the list of program arguments
left after the option list was stripped (this is a trailing slice of the
first argument).
Each option-and-value pair returned has the option as its first element,
prefixed with a hyphen (e.g.,
\code{'-x'}),
and the option argument as its second element, or an empty string if the
option has no argument.
The options occur in the list in the same order in which they were
found, thus allowing multiple occurrences.  Long and short options may
be mixed.

An example using only \UNIX{} style options:

\bcode\begin{verbatim}
>>> import getopt, string
>>> args = string.split('-a -b -cfoo -d bar a1 a2')
>>> args
['-a', '-b', '-cfoo', '-d', 'bar', 'a1', 'a2']
>>> optlist, args = getopt.getopt(args, 'abc:d:')
>>> optlist
[('-a', ''), ('-b', ''), ('-c', 'foo'), ('-d', 'bar')]
>>> args
['a1', 'a2']
>>> 
\end{verbatim}\ecode

Using long option names is equally easy:

\bcode\begin{verbatim}
>>> s = '--condition=foo --testing --output-file abc.def -x a1 a2'
>>> args = string.split(s)
>>> args
['--condition=foo', '--testing', '--output-file', 'abc.def', '-x', 'a1', 'a2']
>>> optlist, args = getopt.getopt(args, 'x', [
...     'condition=', 'output-file=', 'testing'])
>>> optlist
[('--condition', 'foo'), ('--testing', ''), ('--output-file', 'abc.def'), ('-x', '')]
>>> args
['a1', 'a2']
>>> 
\end{verbatim}\ecode

The exception
\code{getopt.error = 'getopt.error'}
is raised when an unrecognized option is found in the argument list or
when an option requiring an argument is given none.
The argument to the exception is a string indicating the cause of the
error.  For long options, an argument given to an option which does
not require one will also cause this exception to be raised.

\section{Standard Module \sectcode{tempfile}}
\stmodindex{tempfile}
\indexii{temporary}{file name}
\indexii{temporary}{file}

\renewcommand{\indexsubitem}{(in module tempfile)}

This module generates temporary file names.  It is not \UNIX{} specific,
but it may require some help on non-\UNIX{} systems.

Note: the modules does not create temporary files, nor does it
automatically remove them when the current process exits or dies.

The module defines a single user-callable function:

\begin{funcdesc}{mktemp}{}
Return a unique temporary filename.  This is an absolute pathname of a
file that does not exist at the time the call is made.  No two calls
will return the same filename.
\end{funcdesc}

The module uses two global variables that tell it how to construct a
temporary name.  The caller may assign values to them; by default they
are initialized at the first call to \code{mktemp()}.

\begin{datadesc}{tempdir}
When set to a value other than \code{None}, this variable defines the
directory in which filenames returned by \code{mktemp()} reside.  The
default is taken from the environment variable \code{TMPDIR}; if this
is not set, either \code{/usr/tmp} is used (on \UNIX{}), or the current
working directory (all other systems).  No check is made to see
whether its value is valid.
\end{datadesc}
\ttindex{TMPDIR}

\begin{datadesc}{template}
When set to a value other than \code{None}, this variable defines the
prefix of the final component of the filenames returned by
\code{mktemp()}.  A string of decimal digits is added to generate
unique filenames.  The default is either ``\code{@\var{pid}.}'' where
\var{pid} is the current process ID (on \UNIX{}), or ``\code{tmp}'' (all
other systems).
\end{datadesc}

Warning: if a \UNIX{} process uses \code{mktemp()}, then calls
\code{fork()} and both parent and child continue to use
\code{mktemp()}, the processes will generate conflicting temporary
names.  To resolve this, the child process should assign \code{None}
to \code{template}, to force recomputing the default on the next call
to \code{mktemp()}.

\section{Standard Module \sectcode{errno}}
\stmodindex{errno}

\renewcommand{\indexsubitem}{(in module errno)}

This module makes available standard errno system symbols.
The value of each symbol is the corresponding integer value.
The names and descriptions are borrowed from linux/include/errno.h,
which should be pretty all-inclusive.  Of the following list, symbols
that are not used on the current platform are not defined by the
module.

Symbols available can include:
\begin{datadesc}{EPERM} Operation not permitted \end{datadesc}
\begin{datadesc}{ENOENT} No such file or directory \end{datadesc}
\begin{datadesc}{ESRCH} No such process \end{datadesc}
\begin{datadesc}{EINTR} Interrupted system call \end{datadesc}
\begin{datadesc}{EIO} I/O error \end{datadesc}
\begin{datadesc}{ENXIO} No such device or address \end{datadesc}
\begin{datadesc}{E2BIG} Arg list too long \end{datadesc}
\begin{datadesc}{ENOEXEC} Exec format error \end{datadesc}
\begin{datadesc}{EBADF} Bad file number \end{datadesc}
\begin{datadesc}{ECHILD} No child processes \end{datadesc}
\begin{datadesc}{EAGAIN} Try again \end{datadesc}
\begin{datadesc}{ENOMEM} Out of memory \end{datadesc}
\begin{datadesc}{EACCES} Permission denied \end{datadesc}
\begin{datadesc}{EFAULT} Bad address \end{datadesc}
\begin{datadesc}{ENOTBLK} Block device required \end{datadesc}
\begin{datadesc}{EBUSY} Device or resource busy \end{datadesc}
\begin{datadesc}{EEXIST} File exists \end{datadesc}
\begin{datadesc}{EXDEV} Cross-device link \end{datadesc}
\begin{datadesc}{ENODEV} No such device \end{datadesc}
\begin{datadesc}{ENOTDIR} Not a directory \end{datadesc}
\begin{datadesc}{EISDIR} Is a directory \end{datadesc}
\begin{datadesc}{EINVAL} Invalid argument \end{datadesc}
\begin{datadesc}{ENFILE} File table overflow \end{datadesc}
\begin{datadesc}{EMFILE} Too many open files \end{datadesc}
\begin{datadesc}{ENOTTY} Not a typewriter \end{datadesc}
\begin{datadesc}{ETXTBSY} Text file busy \end{datadesc}
\begin{datadesc}{EFBIG} File too large \end{datadesc}
\begin{datadesc}{ENOSPC} No space left on device \end{datadesc}
\begin{datadesc}{ESPIPE} Illegal seek \end{datadesc}
\begin{datadesc}{EROFS} Read-only file system \end{datadesc}
\begin{datadesc}{EMLINK} Too many links \end{datadesc}
\begin{datadesc}{EPIPE} Broken pipe \end{datadesc}
\begin{datadesc}{EDOM} Math argument out of domain of func \end{datadesc}
\begin{datadesc}{ERANGE} Math result not representable \end{datadesc}
\begin{datadesc}{EDEADLK} Resource deadlock would occur \end{datadesc}
\begin{datadesc}{ENAMETOOLONG} File name too long \end{datadesc}
\begin{datadesc}{ENOLCK} No record locks available \end{datadesc}
\begin{datadesc}{ENOSYS} Function not implemented \end{datadesc}
\begin{datadesc}{ENOTEMPTY} Directory not empty \end{datadesc}
\begin{datadesc}{ELOOP} Too many symbolic links encountered \end{datadesc}
\begin{datadesc}{EWOULDBLOCK} Operation would block \end{datadesc}
\begin{datadesc}{ENOMSG} No message of desired type \end{datadesc}
\begin{datadesc}{EIDRM} Identifier removed \end{datadesc}
\begin{datadesc}{ECHRNG} Channel number out of range \end{datadesc}
\begin{datadesc}{EL2NSYNC} Level 2 not synchronized \end{datadesc}
\begin{datadesc}{EL3HLT} Level 3 halted \end{datadesc}
\begin{datadesc}{EL3RST} Level 3 reset \end{datadesc}
\begin{datadesc}{ELNRNG} Link number out of range \end{datadesc}
\begin{datadesc}{EUNATCH} Protocol driver not attached \end{datadesc}
\begin{datadesc}{ENOCSI} No CSI structure available \end{datadesc}
\begin{datadesc}{EL2HLT} Level 2 halted \end{datadesc}
\begin{datadesc}{EBADE} Invalid exchange \end{datadesc}
\begin{datadesc}{EBADR} Invalid request descriptor \end{datadesc}
\begin{datadesc}{EXFULL} Exchange full \end{datadesc}
\begin{datadesc}{ENOANO} No anode \end{datadesc}
\begin{datadesc}{EBADRQC} Invalid request code \end{datadesc}
\begin{datadesc}{EBADSLT} Invalid slot \end{datadesc}
\begin{datadesc}{EDEADLOCK} File locking deadlock error \end{datadesc}
\begin{datadesc}{EBFONT} Bad font file format \end{datadesc}
\begin{datadesc}{ENOSTR} Device not a stream \end{datadesc}
\begin{datadesc}{ENODATA} No data available \end{datadesc}
\begin{datadesc}{ETIME} Timer expired \end{datadesc}
\begin{datadesc}{ENOSR} Out of streams resources \end{datadesc}
\begin{datadesc}{ENONET} Machine is not on the network \end{datadesc}
\begin{datadesc}{ENOPKG} Package not installed \end{datadesc}
\begin{datadesc}{EREMOTE} Object is remote \end{datadesc}
\begin{datadesc}{ENOLINK} Link has been severed \end{datadesc}
\begin{datadesc}{EADV} Advertise error \end{datadesc}
\begin{datadesc}{ESRMNT} Srmount error \end{datadesc}
\begin{datadesc}{ECOMM} Communication error on send \end{datadesc}
\begin{datadesc}{EPROTO} Protocol error \end{datadesc}
\begin{datadesc}{EMULTIHOP} Multihop attempted \end{datadesc}
\begin{datadesc}{EDOTDOT} RFS specific error \end{datadesc}
\begin{datadesc}{EBADMSG} Not a data message \end{datadesc}
\begin{datadesc}{EOVERFLOW} Value too large for defined data type \end{datadesc}
\begin{datadesc}{ENOTUNIQ} Name not unique on network \end{datadesc}
\begin{datadesc}{EBADFD} File descriptor in bad state \end{datadesc}
\begin{datadesc}{EREMCHG} Remote address changed \end{datadesc}
\begin{datadesc}{ELIBACC} Can not access a needed shared library \end{datadesc}
\begin{datadesc}{ELIBBAD} Accessing a corrupted shared library \end{datadesc}
\begin{datadesc}{ELIBSCN} .lib section in a.out corrupted \end{datadesc}
\begin{datadesc}{ELIBMAX} Attempting to link in too many shared libraries \end{datadesc}
\begin{datadesc}{ELIBEXEC} Cannot exec a shared library directly \end{datadesc}
\begin{datadesc}{EILSEQ} Illegal byte sequence \end{datadesc}
\begin{datadesc}{ERESTART} Interrupted system call should be restarted \end{datadesc}
\begin{datadesc}{ESTRPIPE} Streams pipe error \end{datadesc}
\begin{datadesc}{EUSERS} Too many users \end{datadesc}
\begin{datadesc}{ENOTSOCK} Socket operation on non-socket \end{datadesc}
\begin{datadesc}{EDESTADDRREQ} Destination address required \end{datadesc}
\begin{datadesc}{EMSGSIZE} Message too long \end{datadesc}
\begin{datadesc}{EPROTOTYPE} Protocol wrong type for socket \end{datadesc}
\begin{datadesc}{ENOPROTOOPT} Protocol not available \end{datadesc}
\begin{datadesc}{EPROTONOSUPPORT} Protocol not supported \end{datadesc}
\begin{datadesc}{ESOCKTNOSUPPORT} Socket type not supported \end{datadesc}
\begin{datadesc}{EOPNOTSUPP} Operation not supported on transport endpoint \end{datadesc}
\begin{datadesc}{EPFNOSUPPORT} Protocol family not supported \end{datadesc}
\begin{datadesc}{EAFNOSUPPORT} Address family not supported by protocol \end{datadesc}
\begin{datadesc}{EADDRINUSE} Address already in use \end{datadesc}
\begin{datadesc}{EADDRNOTAVAIL} Cannot assign requested address \end{datadesc}
\begin{datadesc}{ENETDOWN} Network is down \end{datadesc}
\begin{datadesc}{ENETUNREACH} Network is unreachable \end{datadesc}
\begin{datadesc}{ENETRESET} Network dropped connection because of reset \end{datadesc}
\begin{datadesc}{ECONNABORTED} Software caused connection abort \end{datadesc}
\begin{datadesc}{ECONNRESET} Connection reset by peer \end{datadesc}
\begin{datadesc}{ENOBUFS} No buffer space available \end{datadesc}
\begin{datadesc}{EISCONN} Transport endpoint is already connected \end{datadesc}
\begin{datadesc}{ENOTCONN} Transport endpoint is not connected \end{datadesc}
\begin{datadesc}{ESHUTDOWN} Cannot send after transport endpoint shutdown \end{datadesc}
\begin{datadesc}{ETOOMANYREFS} Too many references: cannot splice \end{datadesc}
\begin{datadesc}{ETIMEDOUT} Connection timed out \end{datadesc}
\begin{datadesc}{ECONNREFUSED} Connection refused \end{datadesc}
\begin{datadesc}{EHOSTDOWN} Host is down \end{datadesc}
\begin{datadesc}{EHOSTUNREACH} No route to host \end{datadesc}
\begin{datadesc}{EALREADY} Operation already in progress \end{datadesc}
\begin{datadesc}{EINPROGRESS} Operation now in progress \end{datadesc}
\begin{datadesc}{ESTALE} Stale NFS file handle \end{datadesc}
\begin{datadesc}{EUCLEAN} Structure needs cleaning \end{datadesc}
\begin{datadesc}{ENOTNAM} Not a XENIX named type file \end{datadesc}
\begin{datadesc}{ENAVAIL} No XENIX semaphores available \end{datadesc}
\begin{datadesc}{EISNAM} Is a named type file \end{datadesc}
\begin{datadesc}{EREMOTEIO} Remote I/O error \end{datadesc}
\begin{datadesc}{EDQUOT} Quota exceeded \end{datadesc}



\chapter{Optional Operating System Services}

The modules described in this chapter provide interfaces to operating
system features that are available on selected operating systems only.
The interfaces are generally modelled after the \UNIX{} or C
interfaces but they are available on some other systems as well
(e.g. Windows or NT).  Here's an overview:

\begin{description}

\item[signal]
--- Set handlers for asynchronous events.

\item[socket]
--- Low-level networking interface.

\item[select]
--- Wait for I/O completion on multiple streams.

\item[thread]
--- Create multiple threads of control within one namespace.

\end{description}
		% Optional Operating System Services
\section{Built-in Module \sectcode{signal}}

\bimodindex{signal}
This module provides mechanisms to use signal handlers in Python.
Some general rules for working with signals handlers:

\begin{itemize}

\item
A handler for a particular signal, once set, remains installed until
it is explicitly reset (i.e. Python emulates the BSD style interface
regardless of the underlying implementation), with the exception of
the handler for \code{SIGCHLD}, which follows the underlying
implementation.

\item
There is no way to ``block'' signals temporarily from critical
sections (since this is not supported by all \UNIX{} flavors).

\item
Although Python signal handlers are called asynchronously as far as
the Python user is concerned, they can only occur between the
``atomic'' instructions of the Python interpreter.  This means that
signals arriving during long calculations implemented purely in C
(e.g.\ regular expression matches on large bodies of text) may be
delayed for an arbitrary amount of time.

\item
When a signal arrives during an I/O operation, it is possible that the
I/O operation raises an exception after the signal handler returns.
This is dependent on the underlying \UNIX{} system's semantics regarding
interrupted system calls.

\item
Because the C signal handler always returns, it makes little sense to
catch synchronous errors like \code{SIGFPE} or \code{SIGSEGV}.

\item
Python installs a small number of signal handlers by default:
\code{SIGPIPE} is ignored (so write errors on pipes and sockets can be
reported as ordinary Python exceptions), \code{SIGINT} is translated
into a \code{KeyboardInterrupt} exception, and \code{SIGTERM} is
caught so that necessary cleanup (especially \code{sys.exitfunc}) can
be performed before actually terminating.  All of these can be
overridden.

\item
Some care must be taken if both signals and threads are used in the
same program.  The fundamental thing to remember in using signals and
threads simultaneously is:\ always perform \code{signal()} operations
in the main thread of execution.  Any thread can perform an
\code{alarm()}, \code{getsignal()}, or \code{pause()}; only the main
thread can set a new signal handler, and the main thread will be the
only one to receive signals (this is enforced by the Python signal
module, even if the underlying thread implementation supports sending
signals to individual threads).  This means that signals can't be used
as a means of interthread communication.  Use locks instead.

\end{itemize}

The variables defined in the signal module are:

\renewcommand{\indexsubitem}{(in module signal)}
\begin{datadesc}{SIG_DFL}
  This is one of two standard signal handling options; it will simply
  perform the default function for the signal.  For example, on most
  systems the default action for SIGQUIT is to dump core and exit,
  while the default action for SIGCLD is to simply ignore it.
\end{datadesc}

\begin{datadesc}{SIG_IGN}
  This is another standard signal handler, which will simply ignore
  the given signal.
\end{datadesc}

\begin{datadesc}{SIG*}
  All the signal numbers are defined symbolically.  For example, the
  hangup signal is defined as \code{signal.SIGHUP}; the variable names
  are identical to the names used in C programs, as found in
  \file{signal.h}.
  The \UNIX{} man page for \file{signal} lists the existing signals (on
  some systems this is \file{signal(2)}, on others the list is in
  \file{signal(7)}).
  Note that not all systems define the same set of signal names; only
  those names defined by the system are defined by this module.
\end{datadesc}

\begin{datadesc}{NSIG}
  One more than the number of the highest signal number.
\end{datadesc}

The signal module defines the following functions:

\begin{funcdesc}{alarm}{time}
  If \var{time} is non-zero, this function requests that a
  \code{SIGALRM} signal be sent to the process in \var{time} seconds.
  Any previously scheduled alarm is canceled (i.e.\ only one alarm can
  be scheduled at any time).  The returned value is then the number of
  seconds before any previously set alarm was to have been delivered.
  If \var{time} is zero, no alarm id scheduled, and any scheduled
  alarm is canceled.  The return value is the number of seconds
  remaining before a previously scheduled alarm.  If the return value
  is zero, no alarm is currently scheduled.  (See the \UNIX{} man page
  \code{alarm(2)}.)
\end{funcdesc}

\begin{funcdesc}{getsignal}{signalnum}
  Return the current signal handler for the signal \var{signalnum}.
  The returned value may be a callable Python object, or one of the
  special values \code{signal.SIG_IGN}, \code{signal.SIG_DFL} or
  \code{None}.  Here, \code{signal.SIG_IGN} means that the signal was
  previously ignored, \code{signal.SIG_DFL} means that the default way
  of handling the signal was previously in use, and \code{None} means
  that the previous signal handler was not installed from Python.
\end{funcdesc}

\begin{funcdesc}{pause}{}
  Cause the process to sleep until a signal is received; the
  appropriate handler will then be called.  Returns nothing.  (See the
  \UNIX{} man page \code{signal(2)}.)
\end{funcdesc}

\begin{funcdesc}{signal}{signalnum\, handler}
  Set the handler for signal \var{signalnum} to the function
  \var{handler}.  \var{handler} can be any callable Python object, or
  one of the special values \code{signal.SIG_IGN} or
  \code{signal.SIG_DFL}.  The previous signal handler will be returned
  (see the description of \code{getsignal()} above).  (See the \UNIX{}
  man page \code{signal(2)}.)

  When threads are enabled, this function can only be called from the
  main thread; attempting to call it from other threads will cause a
  \code{ValueError} exception to be raised.

  The \var{handler} is called with two arguments: the signal number
  and the current stack frame (\code{None} or a frame object; see the
  reference manual for a description of frame objects).
\obindex{frame}
\end{funcdesc}

\section{Built-in Module \sectcode{socket}}

\bimodindex{socket}
This module provides access to the BSD {\em socket} interface.
It is available on \UNIX{} systems that support this interface.

For an introduction to socket programming (in C), see the following
papers: \emph{An Introductory 4.3BSD Interprocess Communication
Tutorial}, by Stuart Sechrest and \emph{An Advanced 4.3BSD Interprocess
Communication Tutorial}, by Samuel J.  Leffler et al, both in the
\UNIX{} Programmer's Manual, Supplementary Documents 1 (sections PS1:7
and PS1:8).  The \UNIX{} manual pages for the various socket-related
system calls are also a valuable source of information on the details of
socket semantics.

The Python interface is a straightforward transliteration of the
\UNIX{} system call and library interface for sockets to Python's
object-oriented style: the \code{socket()} function returns a
\dfn{socket object} whose methods implement the various socket system
calls.  Parameter types are somewhat higer-level than in the C
interface: as with \code{read()} and \code{write()} operations on Python
files, buffer allocation on receive operations is automatic, and
buffer length is implicit on send operations.

Socket addresses are represented as a single string for the
\code{AF_UNIX} address family and as a pair
\code{(\var{host}, \var{port})} for the \code{AF_INET} address family,
where \var{host} is a string representing
either a hostname in Internet domain notation like
\code{'daring.cwi.nl'} or an IP address like \code{'100.50.200.5'},
and \var{port} is an integral port number.  Other address families are
currently not supported.  The address format required by a particular
socket object is automatically selected based on the address family
specified when the socket object was created.

All errors raise exceptions.  The normal exceptions for invalid
argument types and out-of-memory conditions can be raised; errors
related to socket or address semantics raise the error \code{socket.error}.

Non-blocking mode is supported through the \code{setblocking()}
method.

The module \code{socket} exports the following constants and functions:

\renewcommand{\indexsubitem}{(in module socket)}
\begin{excdesc}{error}
This exception is raised for socket- or address-related errors.
The accompanying value is either a string telling what went wrong or a
pair \code{(\var{errno}, \var{string})}
representing an error returned by a system
call, similar to the value accompanying \code{posix.error}.
\end{excdesc}

\begin{datadesc}{AF_UNIX}
\dataline{AF_INET}
These constants represent the address (and protocol) families,
used for the first argument to \code{socket()}.  If the \code{AF_UNIX}
constant is not defined then this protocol is unsupported.
\end{datadesc}

\begin{datadesc}{SOCK_STREAM}
\dataline{SOCK_DGRAM}
\dataline{SOCK_RAW}
\dataline{SOCK_RDM}
\dataline{SOCK_SEQPACKET}
These constants represent the socket types,
used for the second argument to \code{socket()}.
(Only \code{SOCK_STREAM} and
\code{SOCK_DGRAM} appear to be generally useful.)
\end{datadesc}

\begin{datadesc}{SO_*}
\dataline{SOMAXCONN}
\dataline{MSG_*}
\dataline{SOL_*}
\dataline{IPPROTO_*}
\dataline{IPPORT_*}
\dataline{INADDR_*}
\dataline{IP_*}
Many constants of these forms, documented in the \UNIX{} documentation on
sockets and/or the IP protocol, are also defined in the socket module.
They are generally used in arguments to the \code{setsockopt} and
\code{getsockopt} methods of socket objects.  In most cases, only
those symbols that are defined in the \UNIX{} header files are defined;
for a few symbols, default values are provided.
\end{datadesc}

\begin{funcdesc}{gethostbyname}{hostname}
Translate a host name to IP address format.  The IP address is
returned as a string, e.g.,  \code{'100.50.200.5'}.  If the host name
is an IP address itself it is returned unchanged.
\end{funcdesc}

\begin{funcdesc}{gethostname}{}
Return a string containing the hostname of the machine where 
the Python interpreter is currently executing.  If you want to know the
current machine's IP address, use
\code{socket.gethostbyname(socket.gethostname())}.
\end{funcdesc}

\begin{funcdesc}{gethostbyaddr}{ip_address}
Return a triple \code{(hostname, aliaslist, ipaddrlist)} where
\code{hostname} is the primary host name responding to the given
\var{ip_address}, \code{aliaslist} is a (possibly empty) list of
alternative host names for the same address, and \code{ipaddrlist} is
a list of IP addresses for the same interface on the same
host (most likely containing only a single address).
\end{funcdesc}

\begin{funcdesc}{getservbyname}{servicename\, protocolname}
Translate an Internet service name and protocol name to a port number
for that service.  The protocol name should be \code{'tcp'} or
\code{'udp'}.
\end{funcdesc}

\begin{funcdesc}{socket}{family\, type\optional{\, proto}}
Create a new socket using the given address family, socket type and
protocol number.  The address family should be \code{AF_INET} or
\code{AF_UNIX}.  The socket type should be \code{SOCK_STREAM},
\code{SOCK_DGRAM} or perhaps one of the other \samp{SOCK_} constants.
The protocol number is usually zero and may be omitted in that case.
\end{funcdesc}

\begin{funcdesc}{fromfd}{fd\, family\, type\optional{\, proto}}
Build a socket object from an existing file descriptor (an integer as
returned by a file object's \code{fileno} method).  Address family,
socket type and protocol number are as for the \code{socket} function
above.  The file descriptor should refer to a socket, but this is not
checked --- subsequent operations on the object may fail if the file
descriptor is invalid.  This function is rarely needed, but can be
used to get or set socket options on a socket passed to a program as
standard input or output (e.g.\ a server started by the \UNIX{} inet
daemon).
\end{funcdesc}

\subsection{Socket Objects}

\noindent
Socket objects have the following methods.  Except for
\code{makefile()} these correspond to \UNIX{} system calls applicable to
sockets.

\renewcommand{\indexsubitem}{(socket method)}
\begin{funcdesc}{accept}{}
Accept a connection.
The socket must be bound to an address and listening for connections.
The return value is a pair \code{(\var{conn}, \var{address})}
where \var{conn} is a \emph{new} socket object usable to send and
receive data on the connection, and \var{address} is the address bound
to the socket on the other end of the connection.
\end{funcdesc}

\begin{funcdesc}{bind}{address}
Bind the socket to \var{address}.  The socket must not already be bound.
(The format of \var{address} depends on the address family --- see above.)
\end{funcdesc}

\begin{funcdesc}{close}{}
Close the socket.  All future operations on the socket object will fail.
The remote end will receive no more data (after queued data is flushed).
Sockets are automatically closed when they are garbage-collected.
\end{funcdesc}

\begin{funcdesc}{connect}{address}
Connect to a remote socket at \var{address}.
(The format of \var{address} depends on the address family --- see above.)
\end{funcdesc}

\begin{funcdesc}{fileno}{}
Return the socket's file descriptor (a small integer).  This is useful
with \code{select}.
\end{funcdesc}

\begin{funcdesc}{getpeername}{}
Return the remote address to which the socket is connected.  This is
useful to find out the port number of a remote IP socket, for instance.
(The format of the address returned depends on the address family ---
see above.)  On some systems this function is not supported.
\end{funcdesc}

\begin{funcdesc}{getsockname}{}
Return the socket's own address.  This is useful to find out the port
number of an IP socket, for instance.
(The format of the address returned depends on the address family ---
see above.)
\end{funcdesc}

\begin{funcdesc}{getsockopt}{level\, optname\optional{\, buflen}}
Return the value of the given socket option (see the \UNIX{} man page
{\it getsockopt}(2)).  The needed symbolic constants (\code{SO_*} etc.)
are defined in this module.  If \var{buflen}
is absent, an integer option is assumed and its integer value
is returned by the function.  If \var{buflen} is present, it specifies
the maximum length of the buffer used to receive the option in, and
this buffer is returned as a string.  It is up to the caller to decode
the contents of the buffer (see the optional built-in module
\code{struct} for a way to decode C structures encoded as strings).
\end{funcdesc}

\begin{funcdesc}{listen}{backlog}
Listen for connections made to the socket.  The \var{backlog} argument
specifies the maximum number of queued connections and should be at
least 1; the maximum value is system-dependent (usually 5).
\end{funcdesc}

\begin{funcdesc}{makefile}{\optional{mode\optional{\, bufsize}}}
Return a \dfn{file object} associated with the socket.  (File objects
were described earlier under Built-in Types.)  The file object
references a \code{dup()}ped version of the socket file descriptor, so
the file object and socket object may be closed or garbage-collected
independently.  The optional \var{mode} and \var{bufsize} arguments
are interpreted the same way as by the built-in
\code{open()} function.
\end{funcdesc}

\begin{funcdesc}{recv}{bufsize\optional{\, flags}}
Receive data from the socket.  The return value is a string representing
the data received.  The maximum amount of data to be received
at once is specified by \var{bufsize}.  See the \UNIX{} manual page
for the meaning of the optional argument \var{flags}; it defaults to
zero.
\end{funcdesc}

\begin{funcdesc}{recvfrom}{bufsize\optional{\, flags}}
Receive data from the socket.  The return value is a pair
\code{(\var{string}, \var{address})} where \var{string} is a string
representing the data received and \var{address} is the address of the
socket sending the data.  The optional \var{flags} argument has the
same meaning as for \code{recv()} above.
(The format of \var{address} depends on the address family --- see above.)
\end{funcdesc}

\begin{funcdesc}{send}{string\optional{\, flags}}
Send data to the socket.  The socket must be connected to a remote
socket.  The optional \var{flags} argument has the same meaning as for
\code{recv()} above.  Return the number of bytes sent.
\end{funcdesc}

\begin{funcdesc}{sendto}{string\optional{\, flags}\, address}
Send data to the socket.  The socket should not be connected to a
remote socket, since the destination socket is specified by
\code{address}.  The optional \var{flags} argument has the same
meaning as for \code{recv()} above.  Return the number of bytes sent.
(The format of \var{address} depends on the address family --- see above.)
\end{funcdesc}

\begin{funcdesc}{setblocking}{flag}
Set blocking or non-blocking mode of the socket: if \var{flag} is 0,
the socket is set to non-blocking, else to blocking mode.  Initially
all sockets are in blocking mode.  In non-blocking mode, if a
\code{recv} call doesn't find any data, or if a \code{send} call can't
immediately dispose of the data, a \code{socket.error} exception is
raised; in blocking mode, the calls block until they can proceed.
\end{funcdesc}

\begin{funcdesc}{setsockopt}{level\, optname\, value}
Set the value of the given socket option (see the \UNIX{} man page
{\it setsockopt}(2)).  The needed symbolic constants are defined in
the \code{socket} module (\code{SO_*} etc.).  The value can be an
integer or a string representing a buffer.  In the latter case it is
up to the caller to ensure that the string contains the proper bits
(see the optional built-in module
\code{struct} for a way to encode C structures as strings).
\end{funcdesc}

\begin{funcdesc}{shutdown}{how}
Shut down one or both halves of the connection.  If \var{how} is \code{0},
further receives are disallowed.  If \var{how} is \code{1}, further sends are
disallowed.  If \var{how} is \code{2}, further sends and receives are
disallowed.
\end{funcdesc}

Note that there are no methods \code{read()} or \code{write()}; use
\code{recv()} and \code{send()} without \var{flags} argument instead.

\subsection{Example}
\nodename{Socket Example}

Here are two minimal example programs using the TCP/IP protocol:\ a
server that echoes all data that it receives back (servicing only one
client), and a client using it.  Note that a server must perform the
sequence \code{socket}, \code{bind}, \code{listen}, \code{accept}
(possibly repeating the \code{accept} to service more than one client),
while a client only needs the sequence \code{socket}, \code{connect}.
Also note that the server does not \code{send}/\code{receive} on the
socket it is listening on but on the new socket returned by
\code{accept}.

\bcode\begin{verbatim}
# Echo server program
from socket import *
HOST = ''                 # Symbolic name meaning the local host
PORT = 50007              # Arbitrary non-privileged server
s = socket(AF_INET, SOCK_STREAM)
s.bind(HOST, PORT)
s.listen(1)
conn, addr = s.accept()
print 'Connected by', addr
while 1:
    data = conn.recv(1024)
    if not data: break
    conn.send(data)
conn.close()
\end{verbatim}\ecode

\bcode\begin{verbatim}
# Echo client program
from socket import *
HOST = 'daring.cwi.nl'    # The remote host
PORT = 50007              # The same port as used by the server
s = socket(AF_INET, SOCK_STREAM)
s.connect(HOST, PORT)
s.send('Hello, world')
data = s.recv(1024)
s.close()
print 'Received', `data`
\end{verbatim}\ecode

\section{Built-in Module \sectcode{select}}
\bimodindex{select}

This module provides access to the function \code{select} available in
most \UNIX{} versions.  It defines the following:

\renewcommand{\indexsubitem}{(in module select)}
\begin{excdesc}{error}
The exception raised when an error occurs.  The accompanying value is
a pair containing the numeric error code from \code{errno} and the
corresponding string, as would be printed by the C function
\code{perror()}.
\end{excdesc}

\begin{funcdesc}{select}{iwtd\, owtd\, ewtd\optional{\, timeout}}
This is a straightforward interface to the \UNIX{} \code{select()}
system call.  The first three arguments are lists of `waitable
objects': either integers representing \UNIX{} file descriptors or
objects with a parameterless method named \code{fileno()} returning
such an integer.  The three lists of waitable objects are for input,
output and `exceptional conditions', respectively.  Empty lists are
allowed.  The optional \var{timeout} argument specifies a time-out as a
floating point number in seconds.  When the \var{timeout} argument
is omitted the function blocks until at least one file descriptor is
ready.  A time-out value of zero specifies a poll and never blocks.

The return value is a triple of lists of objects that are ready:
subsets of the first three arguments.  When the time-out is reached
without a file descriptor becoming ready, three empty lists are
returned.

Amongst the acceptable object types in the lists are Python file
objects (e.g. \code{sys.stdin}, or objects returned by \code{open()}
or \code{posix.popen()}), socket objects returned by
\code{socket.socket()}, and the module \code{stdwin} which happens to
define a function \code{fileno()} for just this purpose.  You may
also define a \dfn{wrapper} class yourself, as long as it has an
appropriate \code{fileno()} method (that really returns a \UNIX{} file
descriptor, not just a random integer).
\end{funcdesc}
\ttindex{socket}
\ttindex{stdwin}

\section{Built-in Module \sectcode{thread}}
\bimodindex{thread}

This module provides low-level primitives for working with multiple
threads (a.k.a.\ \dfn{light-weight processes} or \dfn{tasks}) --- multiple
threads of control sharing their global data space.  For
synchronization, simple locks (a.k.a.\ \dfn{mutexes} or \dfn{binary
semaphores}) are provided.

The module is optional and supported on SGI IRIX 4.x and 5.x and Sun
Solaris 2.x systems, as well as on systems that have a PTHREAD
implementation (e.g.\ KSR).

It defines the following constant and functions:

\renewcommand{\indexsubitem}{(in module thread)}
\begin{excdesc}{error}
Raised on thread-specific errors.
\end{excdesc}

\begin{funcdesc}{start_new_thread}{func\, arg}
Start a new thread.  The thread executes the function \var{func}
with the argument list \var{arg} (which must be a tuple).  When the
function returns, the thread silently exits.  When the function
terminates with an unhandled exception, a stack trace is printed and
then the thread exits (but other threads continue to run).
\end{funcdesc}

\begin{funcdesc}{exit}{}
This is a shorthand for \code{thread.exit_thread()}.
\end{funcdesc}

\begin{funcdesc}{exit_thread}{}
Raise the \code{SystemExit} exception.  When not caught, this will
cause the thread to exit silently.
\end{funcdesc}

%\begin{funcdesc}{exit_prog}{status}
%Exit all threads and report the value of the integer argument
%\var{status} as the exit status of the entire program.
%\strong{Caveat:} code in pending \code{finally} clauses, in this thread
%or in other threads, is not executed.
%\end{funcdesc}

\begin{funcdesc}{allocate_lock}{}
Return a new lock object.  Methods of locks are described below.  The
lock is initially unlocked.
\end{funcdesc}

\begin{funcdesc}{get_ident}{}
Return the `thread identifier' of the current thread.  This is a
nonzero integer.  Its value has no direct meaning; it is intended as a
magic cookie to be used e.g. to index a dictionary of thread-specific
data.  Thread identifiers may be recycled when a thread exits and
another thread is created.
\end{funcdesc}

Lock objects have the following methods:

\renewcommand{\indexsubitem}{(lock method)}
\begin{funcdesc}{acquire}{\optional{waitflag}}
Without the optional argument, this method acquires the lock
unconditionally, if necessary waiting until it is released by another
thread (only one thread at a time can acquire a lock --- that's their
reason for existence), and returns \code{None}.  If the integer
\var{waitflag} argument is present, the action depends on its value:\
if it is zero, the lock is only acquired if it can be acquired
immediately without waiting, while if it is nonzero, the lock is
acquired unconditionally as before.  If an argument is present, the
return value is 1 if the lock is acquired successfully, 0 if not.
\end{funcdesc}

\begin{funcdesc}{release}{}
Releases the lock.  The lock must have been acquired earlier, but not
necessarily by the same thread.
\end{funcdesc}

\begin{funcdesc}{locked}{}
Return the status of the lock:\ 1 if it has been acquired by some
thread, 0 if not.
\end{funcdesc}

{\bf Caveats:}

\begin{itemize}
\item
Threads interact strangely with interrupts: the
\code{KeyboardInterrupt} exception will be received by an arbitrary
thread.  (When the \code{signal} module is available, interrupts
always go to the main thread.)

\item
Calling \code{sys.exit()} or raising the \code{SystemExit} is
equivalent to calling \code{thread.exit_thread()}.

\item
Not all built-in functions that may block waiting for I/O allow other
threads to run.  (The most popular ones (\code{sleep}, \code{read},
\code{select}) work as expected.)

\end{itemize}


\chapter{UNIX Specific Services}

The modules described in this chapter provide interfaces to features
that are unique to the \UNIX{} operating system, or in some cases to
some or many variants of it.  Here's an overview:

\begin{description}

\item[posix]
--- The most common Posix system calls (normally used via module \code{os}).

\item[posixpath]
--- Common Posix pathname manipulations (normally used via \code{os.path}).

\item[pwd]
--- The password database (\code{getpwnam()} and friends).

\item[grp]
--- The group database (\code{getgrnam()} and friends).

\item[crypt]
--- The (\code{crypt()} function used to check Unix passwords).

\item[dbm]
--- The standard ``database'' interface, based on \code{ndbm}.

\item[gdbm]
--- GNU's reinterpretation of dbm.

\item[termios]
--- Posix style tty control.

\item[fcntl]
--- The \code{fcntl()} and \code{ioctl()} system calls.

\item[posixfile]
--- A file-like object with support for locking.

\end{description}
			% UNIX Specific Services
\section{Built-in Module \sectcode{posix}}
\bimodindex{posix}

This module provides access to operating system functionality that is
standardized by the C Standard and the POSIX standard (a thinly disguised
\UNIX{} interface).

\strong{Do not import this module directly.}  Instead, import the
module \code{os}, which provides a \emph{portable} version of this
interface.  On \UNIX{}, the \code{os} module provides a superset of
the \code{posix} interface.  On non-\UNIX{} operating systems the
\code{posix} module is not available, but a subset is always available
through the \code{os} interface.  Once \code{os} is imported, there is
\emph{no} performance penalty in using it instead of
\code{posix}.
\stmodindex{os}

The descriptions below are very terse; refer to the
corresponding \UNIX{} manual entry for more information.  Arguments
called \var{path} refer to a pathname given as a string.

Errors are reported as exceptions; the usual exceptions are given
for type errors, while errors reported by the system calls raise
\code{posix.error}, described below.

Module \code{posix} defines the following data items:

\renewcommand{\indexsubitem}{(data in module posix)}
\begin{datadesc}{environ}
A dictionary representing the string environment at the time
the interpreter was started.
For example,
\code{posix.environ['HOME']}
is the pathname of your home directory, equivalent to
\code{getenv("HOME")}
in C.
Modifying this dictionary does not affect the string environment
passed on by \code{execv()}, \code{popen()} or \code{system()}; if you
need to change the environment, pass \code{environ} to \code{execve()}
or add variable assignments and export statements to the command
string for \code{system()} or \code{popen()}.%
\footnote{The problem with automatically passing on \code{environ} is
that there is no portable way of changing the environment.}
\end{datadesc}

\renewcommand{\indexsubitem}{(exception in module posix)}
\begin{excdesc}{error}
This exception is raised when a POSIX function returns a
POSIX-related error (e.g., not for illegal argument types).  Its
string value is \code{'posix.error'}.  The accompanying value is a
pair containing the numeric error code from \code{errno} and the
corresponding string, as would be printed by the C function
\code{perror()}.
\end{excdesc}

It defines the following functions and constants:

\renewcommand{\indexsubitem}{(in module posix)}
\begin{funcdesc}{chdir}{path}
Change the current working directory to \var{path}.
\end{funcdesc}

\begin{funcdesc}{chmod}{path\, mode}
Change the mode of \var{path} to the numeric \var{mode}.
\end{funcdesc}

\begin{funcdesc}{chown}{path\, uid, gid}
Change the owner and group id of \var{path} to the numeric \var{uid}
and \var{gid}.
(Not on MS-DOS.)
\end{funcdesc}

\begin{funcdesc}{close}{fd}
Close file descriptor \var{fd}.

Note: this function is intended for low-level I/O and must be applied
to a file descriptor as returned by \code{posix.open()} or
\code{posix.pipe()}.  To close a ``file object'' returned by the
built-in function \code{open} or by \code{posix.popen} or
\code{posix.fdopen}, use its \code{close()} method.
\end{funcdesc}

\begin{funcdesc}{dup}{fd}
Return a duplicate of file descriptor \var{fd}.
\end{funcdesc}

\begin{funcdesc}{dup2}{fd\, fd2}
Duplicate file descriptor \var{fd} to \var{fd2}, closing the latter
first if necessary.  Return \code{None}.
\end{funcdesc}

\begin{funcdesc}{execv}{path\, args}
Execute the executable \var{path} with argument list \var{args},
replacing the current process (i.e., the Python interpreter).
The argument list may be a tuple or list of strings.
(Not on MS-DOS.)
\end{funcdesc}

\begin{funcdesc}{execve}{path\, args\, env}
Execute the executable \var{path} with argument list \var{args},
and environment \var{env},
replacing the current process (i.e., the Python interpreter).
The argument list may be a tuple or list of strings.
The environment must be a dictionary mapping strings to strings.
(Not on MS-DOS.)
\end{funcdesc}

\begin{funcdesc}{_exit}{n}
Exit to the system with status \var{n}, without calling cleanup
handlers, flushing stdio buffers, etc.
(Not on MS-DOS.)

Note: the standard way to exit is \code{sys.exit(\var{n})}.
\code{posix._exit()} should normally only be used in the child process
after a \code{fork()}.
\end{funcdesc}

\begin{funcdesc}{fdopen}{fd\optional{\, mode\optional{\, bufsize}}}
Return an open file object connected to the file descriptor \var{fd}.
The \var{mode} and \var{bufsize} arguments have the same meaning as
the corresponding arguments to the built-in \code{open()} function.
\end{funcdesc}

\begin{funcdesc}{fork}{}
Fork a child process.  Return 0 in the child, the child's process id
in the parent.
(Not on MS-DOS.)
\end{funcdesc}

\begin{funcdesc}{fstat}{fd}
Return status for file descriptor \var{fd}, like \code{stat()}.
\end{funcdesc}

\begin{funcdesc}{getcwd}{}
Return a string representing the current working directory.
\end{funcdesc}

\begin{funcdesc}{getegid}{}
Return the current process's effective group id.
(Not on MS-DOS.)
\end{funcdesc}

\begin{funcdesc}{geteuid}{}
Return the current process's effective user id.
(Not on MS-DOS.)
\end{funcdesc}

\begin{funcdesc}{getgid}{}
Return the current process's group id.
(Not on MS-DOS.)
\end{funcdesc}

\begin{funcdesc}{getpgrp}{}
Return the current process group id.
(Not on MS-DOS.)
\end{funcdesc}

\begin{funcdesc}{getpid}{}
Return the current process id.
(Not on MS-DOS.)
\end{funcdesc}

\begin{funcdesc}{getppid}{}
Return the parent's process id.
(Not on MS-DOS.)
\end{funcdesc}

\begin{funcdesc}{getuid}{}
Return the current process's user id.
(Not on MS-DOS.)
\end{funcdesc}

\begin{funcdesc}{kill}{pid\, sig}
Kill the process \var{pid} with signal \var{sig}.
(Not on MS-DOS.)
\end{funcdesc}

\begin{funcdesc}{link}{src\, dst}
Create a hard link pointing to \var{src} named \var{dst}.
(Not on MS-DOS.)
\end{funcdesc}

\begin{funcdesc}{listdir}{path}
Return a list containing the names of the entries in the directory.
The list is in arbitrary order.  It does not include the special
entries \code{'.'} and \code{'..'} even if they are present in the
directory.
\end{funcdesc}

\begin{funcdesc}{lseek}{fd\, pos\, how}
Set the current position of file descriptor \var{fd} to position
\var{pos}, modified by \var{how}: 0 to set the position relative to
the beginning of the file; 1 to set it relative to the current
position; 2 to set it relative to the end of the file.
\end{funcdesc}

\begin{funcdesc}{lstat}{path}
Like \code{stat()}, but do not follow symbolic links.  (On systems
without symbolic links, this is identical to \code{posix.stat}.)
\end{funcdesc}

\begin{funcdesc}{mkfifo}{path\optional{\, mode}}
Create a FIFO (a POSIX named pipe) named \var{path} with numeric mode
\var{mode}.  The default \var{mode} is 0666 (octal).  The current
umask value is first masked out from the mode.
(Not on MS-DOS.)

FIFOs are pipes that can be accessed like regular files.  FIFOs exist
until they are deleted (for example with \code{os.unlink}).
Generally, FIFOs are used as rendez-vous between ``client'' and
``server'' type processes: the server opens the FIFO for reading, and
the client opens it for writing.  Note that \code{mkfifo()} doesn't
open the FIFO -- it just creates the rendez-vous point.
\end{funcdesc}

\begin{funcdesc}{mkdir}{path\optional{\, mode}}
Create a directory named \var{path} with numeric mode \var{mode}.
The default \var{mode} is 0777 (octal).  On some systems, \var{mode}
is ignored.  Where it is used, the current umask value is first
masked out.
\end{funcdesc}

\begin{funcdesc}{nice}{increment}
Add \var{incr} to the process' ``niceness''.  Return the new niceness.
(Not on MS-DOS.)
\end{funcdesc}

\begin{funcdesc}{open}{file\, flags\, mode}
Open the file \var{file} and set various flags according to
\var{flags} and possibly its mode according to \var{mode}.
Return the file descriptor for the newly opened file.

Note: this function is intended for low-level I/O.  For normal usage,
use the built-in function \code{open}, which returns a ``file object''
with \code{read()} and  \code{write()} methods (and many more).
\end{funcdesc}

\begin{funcdesc}{pipe}{}
Create a pipe.  Return a pair of file descriptors \code{(r, w)}
usable for reading and writing, respectively.
(Not on MS-DOS.)
\end{funcdesc}

\begin{funcdesc}{plock}{op}
Lock program segments into memory.  The value of \var{op}
(defined in \code{<sys/lock.h>}) determines which segments are locked.
(Not on MS-DOS.)
\end{funcdesc}

\begin{funcdesc}{popen}{command\optional{\, mode\optional{\, bufsize}}}
Open a pipe to or from \var{command}.  The return value is an open
file object connected to the pipe, which can be read or written
depending on whether \var{mode} is \code{'r'} (default) or \code{'w'}.
The \var{bufsize} argument has the same meaning as the corresponding
argument to the built-in \code{open()} function.
(Not on MS-DOS.)
\end{funcdesc}

\begin{funcdesc}{read}{fd\, n}
Read at most \var{n} bytes from file descriptor \var{fd}.
Return a string containing the bytes read.

Note: this function is intended for low-level I/O and must be applied
to a file descriptor as returned by \code{posix.open()} or
\code{posix.pipe()}.  To read a ``file object'' returned by the
built-in function \code{open} or by \code{posix.popen} or
\code{posix.fdopen}, or \code{sys.stdin}, use its
\code{read()} or \code{readline()} methods.
\end{funcdesc}

\begin{funcdesc}{readlink}{path}
Return a string representing the path to which the symbolic link
points.  (On systems without symbolic links, this always raises
\code{posix.error}.)
\end{funcdesc}

\begin{funcdesc}{remove}{path}
Remove the file \var{path}.  See \code{rmdir} below to remove a directory.
\end{funcdesc}

\begin{funcdesc}{rename}{src\, dst}
Rename the file or directory \var{src} to \var{dst}.
\end{funcdesc}

\begin{funcdesc}{rmdir}{path}
Remove the directory \var{path}.
\end{funcdesc}

\begin{funcdesc}{setgid}{gid}
Set the current process's group id.
(Not on MS-DOS.)
\end{funcdesc}

\begin{funcdesc}{setpgrp}{}
Calls the system call \code{setpgrp()} or \code{setpgrp(0, 0)}
depending on which version is implemented (if any).  See the {\UNIX}
manual for the semantics.
(Not on MS-DOS.)
\end{funcdesc}

\begin{funcdesc}{setpgid}{pid\, pgrp}
Calls the system call \code{setpgid()}.  See the {\UNIX} manual for
the semantics.
(Not on MS-DOS.)
\end{funcdesc}

\begin{funcdesc}{setsid}{}
Calls the system call \code{setsid()}.  See the {\UNIX} manual for the
semantics.
(Not on MS-DOS.)
\end{funcdesc}

\begin{funcdesc}{setuid}{uid}
Set the current process's user id.
(Not on MS-DOS.)
\end{funcdesc}

\begin{funcdesc}{stat}{path}
Perform a {\em stat} system call on the given path.  The return value
is a tuple of at least 10 integers giving the most important (and
portable) members of the {\em stat} structure, in the order
\code{st_mode},
\code{st_ino},
\code{st_dev},
\code{st_nlink},
\code{st_uid},
\code{st_gid},
\code{st_size},
\code{st_atime},
\code{st_mtime},
\code{st_ctime}.
More items may be added at the end by some implementations.
(On MS-DOS, some items are filled with dummy values.)

Note: The standard module \code{stat} defines functions and constants
that are useful for extracting information from a stat structure.
\end{funcdesc}

\begin{funcdesc}{symlink}{src\, dst}
Create a symbolic link pointing to \var{src} named \var{dst}.  (On
systems without symbolic links, this always raises
\code{posix.error}.)
\end{funcdesc}

\begin{funcdesc}{system}{command}
Execute the command (a string) in a subshell.  This is implemented by
calling the Standard C function \code{system()}, and has the same
limitations.  Changes to \code{posix.environ}, \code{sys.stdin} etc.\ are
not reflected in the environment of the executed command.  The return
value is the exit status of the process as returned by Standard C
\code{system()}.
\end{funcdesc}

\begin{funcdesc}{tcgetpgrp}{fd}
Return the process group associated with the terminal given by
\var{fd} (an open file descriptor as returned by \code{posix.open()}).
(Not on MS-DOS.)
\end{funcdesc}

\begin{funcdesc}{tcsetpgrp}{fd\, pg}
Set the process group associated with the terminal given by
\var{fd} (an open file descriptor as returned by \code{posix.open()})
to \var{pg}.
(Not on MS-DOS.)
\end{funcdesc}

\begin{funcdesc}{times}{}
Return a 5-tuple of floating point numbers indicating accumulated (CPU
or other)
times, in seconds.  The items are: user time, system time, children's
user time, children's system time, and elapsed real time since a fixed
point in the past, in that order.  See the \UNIX{}
manual page {\it times}(2).  (Not on MS-DOS.)
\end{funcdesc}

\begin{funcdesc}{umask}{mask}
Set the current numeric umask and returns the previous umask.
(Not on MS-DOS.)
\end{funcdesc}

\begin{funcdesc}{uname}{}
Return a 5-tuple containing information identifying the current
operating system.  The tuple contains 5 strings:
\code{(\var{sysname}, \var{nodename}, \var{release}, \var{version}, \var{machine})}.
Some systems truncate the nodename to 8
characters or to the leading component; a better way to get the
hostname is \code{socket.gethostname()}.  (Not on MS-DOS, nor on older
\UNIX{} systems.)
\end{funcdesc}

\begin{funcdesc}{unlink}{path}
Remove the file \var{path}.  This is the same function as \code{remove};
the \code{unlink} name is its traditional \UNIX{} name.
\end{funcdesc}

\begin{funcdesc}{utime}{path\, \(atime\, mtime\)}
Set the access and modified time of the file to the given values.
(The second argument is a tuple of two items.)
\end{funcdesc}

\begin{funcdesc}{wait}{}
Wait for completion of a child process, and return a tuple containing
its pid and exit status indication (encoded as by \UNIX{}).
(Not on MS-DOS.)
\end{funcdesc}

\begin{funcdesc}{waitpid}{pid\, options}
Wait for completion of a child process given by proces id, and return
a tuple containing its pid and exit status indication (encoded as by
\UNIX{}).  The semantics of the call are affected by the value of
the integer options, which should be 0 for normal operation.  (If the
system does not support \code{waitpid()}, this always raises
\code{posix.error}.  Not on MS-DOS.)
\end{funcdesc}

\begin{funcdesc}{write}{fd\, str}
Write the string \var{str} to file descriptor \var{fd}.
Return the number of bytes actually written.

Note: this function is intended for low-level I/O and must be applied
to a file descriptor as returned by \code{posix.open()} or
\code{posix.pipe()}.  To write a ``file object'' returned by the
built-in function \code{open} or by \code{posix.popen} or
\code{posix.fdopen}, or \code{sys.stdout} or \code{sys.stderr}, use
its \code{write()} method.
\end{funcdesc}

\begin{datadesc}{WNOHANG}
The option for \code{waitpid()} to avoid hanging if no child process
status is available immediately.
\end{datadesc}

\section{Standard Module \sectcode{posixpath}}
\stmodindex{posixpath}

This module implements some useful functions on POSIX pathnames.

\strong{Do not import this module directly.}  Instead, import the
module \code{os} and use \code{os.path}.
\stmodindex{os}

\renewcommand{\indexsubitem}{(in module posixpath)}

\begin{funcdesc}{basename}{p}
Return the base name of pathname
\var{p}.
This is the second half of the pair returned by
\code{posixpath.split(\var{p})}.
\end{funcdesc}

\begin{funcdesc}{commonprefix}{list}
Return the longest string that is a prefix of all strings in
\var{list}.
If
\var{list}
is empty, return the empty string (\code{''}).
\end{funcdesc}

\begin{funcdesc}{exists}{p}
Return true if
\var{p}
refers to an existing path.
\end{funcdesc}

\begin{funcdesc}{expanduser}{p}
Return the argument with an initial component of \samp{\~} or
\samp{\~\var{user}} replaced by that \var{user}'s home directory.  An
initial \samp{\~{}} is replaced by the environment variable \code{\${}HOME};
an initial \samp{\~\var{user}} is looked up in the password directory through
the built-in module \code{pwd}.  If the expansion fails, or if the
path does not begin with a tilde, the path is returned unchanged.
\end{funcdesc}

\begin{funcdesc}{expandvars}{p}
Return the argument with environment variables expanded.  Substrings
of the form \samp{\$\var{name}} or \samp{\$\{\var{name}\}} are
replaced by the value of environment variable \var{name}.  Malformed
variable names and references to non-existing variables are left
unchanged.
\end{funcdesc}

\begin{funcdesc}{isabs}{p}
Return true if \var{p} is an absolute pathname (begins with a slash).
\end{funcdesc}

\begin{funcdesc}{isfile}{p}
Return true if \var{p} is an existing regular file.  This follows
symbolic links, so both \code{islink()} and \code{isfile()} can be true for the same
path.
\end{funcdesc}

\begin{funcdesc}{isdir}{p}
Return true if \var{p} is an existing directory.  This follows
symbolic links, so both \code{islink()} and \code{isdir()} can be true for the same
path.
\end{funcdesc}

\begin{funcdesc}{islink}{p}
Return true if
\var{p}
refers to a directory entry that is a symbolic link.
Always false if symbolic links are not supported.
\end{funcdesc}

\begin{funcdesc}{ismount}{p}
Return true if pathname \var{p} is a \dfn{mount point}: a point in a
file system where a different file system has been mounted.  The
function checks whether \var{p}'s parent, \file{\var{p}/..}, is on a
different device than \var{p}, or whether \file{\var{p}/..} and
\var{p} point to the same i-node on the same device --- this should
detect mount points for all \UNIX{} and POSIX variants.
\end{funcdesc}

\begin{funcdesc}{join}{p\, q}
Join the paths
\var{p}
and
\var{q} intelligently:
If
\var{q}
is an absolute path, the return value is
\var{q}.
Otherwise, the concatenation of
\var{p}
and
\var{q}
is returned, with a slash (\code{'/'}) inserted unless
\var{p}
is empty or ends in a slash.
\end{funcdesc}

\begin{funcdesc}{normcase}{p}
Normalize the case of a pathname.  This returns the path unchanged;
however, a similar function in \code{macpath} converts upper case to
lower case.
\end{funcdesc}

\begin{funcdesc}{samefile}{p\, q}
Return true if both pathname arguments refer to the same file or directory
(as indicated by device number and i-node number).
Raise an exception if a stat call on either pathname fails.
\end{funcdesc}

\begin{funcdesc}{split}{p}
Split the pathname \var{p} in a pair \code{(\var{head}, \var{tail})},
where \var{tail} is the last pathname component and \var{head} is
everything leading up to that.  The \var{tail} part will never contain
a slash; if \var{p} ends in a slash, \var{tail} will be empty.  If
there is no slash in \var{p}, \var{head} will be empty.  If \var{p} is
empty, both \var{head} and \var{tail} are empty.  Trailing slashes are
stripped from \var{head} unless it is the root (one or more slashes
only).  In nearly all cases, \code{join(\var{head}, \var{tail})}
equals \var{p} (the only exception being when there were multiple
slashes separating \var{head} from \var{tail}).
\end{funcdesc}

\begin{funcdesc}{splitext}{p}
Split the pathname \var{p} in a pair \code{(\var{root}, \var{ext})}
such that \code{\var{root} + \var{ext} == \var{p}},
and \var{ext} is empty or begins with a period and contains
at most one period.
\end{funcdesc}

\begin{funcdesc}{walk}{p\, visit\, arg}
Calls the function \var{visit} with arguments
\code{(\var{arg}, \var{dirname}, \var{names})} for each directory in the
directory tree rooted at \var{p} (including \var{p} itself, if it is a
directory).  The argument \var{dirname} specifies the visited directory,
the argument \var{names} lists the files in the directory (gotten from
\code{posix.listdir(\var{dirname})}, so including \samp{.} and
\samp{..}).  The \var{visit} function may modify \var{names} to
influence the set of directories visited below \var{dirname}, e.g., to
avoid visiting certain parts of the tree.  (The object referred to by
\var{names} must be modified in place, using \code{del} or slice
assignment.)
\end{funcdesc}
		% == posixpath
\section{Built-in Module \sectcode{pwd}}

\bimodindex{pwd}
This module provides access to the \UNIX{} password database.
It is available on all \UNIX{} versions.

Password database entries are reported as 7-tuples containing the
following items from the password database (see \file{<pwd.h>}), in order:
\code{pw_name},
\code{pw_passwd},
\code{pw_uid},
\code{pw_gid},
\code{pw_gecos},
\code{pw_dir},
\code{pw_shell}.
The uid and gid items are integers, all others are strings.
An exception is raised if the entry asked for cannot be found.

It defines the following items:

\renewcommand{\indexsubitem}{(in module pwd)}
\begin{funcdesc}{getpwuid}{uid}
Return the password database entry for the given numeric user ID.
\end{funcdesc}

\begin{funcdesc}{getpwnam}{name}
Return the password database entry for the given user name.
\end{funcdesc}

\begin{funcdesc}{getpwall}{}
Return a list of all available password database entries, in arbitrary order.
\end{funcdesc}

\section{Built-in Module \sectcode{grp}}

\bimodindex{grp}
This module provides access to the \UNIX{} group database.
It is available on all \UNIX{} versions.

Group database entries are reported as 4-tuples containing the
following items from the group database (see \file{<grp.h>}), in order:
\code{gr_name},
\code{gr_passwd},
\code{gr_gid},
\code{gr_mem}.
The gid is an integer, name and password are strings, and the member
list is a list of strings.
(Note that most users are not explicitly listed as members of the
group they are in according to the password database.)
An exception is raised if the entry asked for cannot be found.

It defines the following items:

\renewcommand{\indexsubitem}{(in module grp)}
\begin{funcdesc}{getgrgid}{gid}
Return the group database entry for the given numeric group ID.
\end{funcdesc}

\begin{funcdesc}{getgrnam}{name}
Return the group database entry for the given group name.
\end{funcdesc}

\begin{funcdesc}{getgrall}{}
Return a list of all available group entries, in arbitrary order.
\end{funcdesc}

\section{Built-in module {\tt crypt}}
\bimodindex{crypt}

This module implements an interface to the crypt({\bf 3}) routine,
which is a one-way hash function based upon a modified DES algorithm;
see the Unix man page for further details.  Possible uses include
allowing Python scripts to accept typed passwords from the user, or
attempting to crack Unix passwords with a dictionary.
\index{crypt(3)}

\begin{funcdesc}{crypt}{word\, salt} 
\var{word} will usually be a user's password.  \var{salt} is a
2-character string which will be used to select one of 4096 variations
of DES.  The characters in \var{salt} must be either \code{.},
\code{/}, or an alphanumeric character.  Returns the hashed password
as a string, which will be composed of characters from the same
alphabet as the salt.
\end{funcdesc}

The module and documentation were written by Steve Majewski.
\index{Majewski, Steve}

\section{Built-in Module \sectcode{dbm}}
\bimodindex{dbm}

The \code{dbm} module provides an interface to the {\UNIX}
\code{(n)dbm} library.  Dbm objects behave like mappings
(dictionaries), except that keys and values are always strings.
Printing a dbm object doesn't print the keys and values, and the
\code{items()} and \code{values()} methods are not supported.

See also the \code{gdbm} module, which provides a similar interface
using the GNU GDBM library.
\bimodindex{gdbm}

The module defines the following constant and functions:

\renewcommand{\indexsubitem}{(in module dbm)}
\begin{excdesc}{error}
Raised on dbm-specific errors, such as I/O errors. \code{KeyError} is
raised for general mapping errors like specifying an incorrect key.
\end{excdesc}

\begin{funcdesc}{open}{filename\, \optional{flag\, \optional{mode}}}
Open a dbm database and return a dbm object.  The \var{filename}
argument is the name of the database file (without the \file{.dir} or
\file{.pag} extensions).

The optional \var{flag} argument can be
\code{'r'} (to open an existing database for reading only --- default),
\code{'w'} (to open an existing database for reading and writing),
\code{'c'} (which creates the database if it doesn't exist), or
\code{'n'} (which always creates a new empty database).

The optional \var{mode} argument is the \UNIX{} mode of the file, used
only when the database has to be created.  It defaults to octal
\code{0666}.
\end{funcdesc}

\section{Built-in Module \sectcode{gdbm}}
\bimodindex{gdbm}

This module is nearly identical to the \code{dbm} module, but uses
GDBM instead.  Its interface is identical, and not repeated here.

Warning: the file formats created by gdbm and dbm are incompatible.
\bimodindex{dbm}

\section{Built-in Module \sectcode{termios}}
\bimodindex{termios}
\indexii{Posix}{I/O control}
\indexii{tty}{I/O control}

\renewcommand{\indexsubitem}{(in module termios)}

This module provides an interface to the Posix calls for tty I/O
control.  For a complete description of these calls, see the Posix or
\UNIX{} manual pages.  It is only available for those \UNIX{} versions
that support Posix \code{termios} style tty I/O control (and then
only if configured at installation time).

All functions in this module take a file descriptor \var{fd} as their
first argument.  This must be an integer file descriptor, such as
returned by \code{sys.stdin.fileno()}.

This module should be used in conjunction with the \code{TERMIOS}
module, which defines the relevant symbolic constants (see the next
section).

The module defines the following functions:

\begin{funcdesc}{tcgetattr}{fd}
Return a list containing the tty attributes for file descriptor
\var{fd}, as follows: \code{[\var{iflag}, \var{oflag}, \var{cflag},
\var{lflag}, \var{ispeed}, \var{ospeed}, \var{cc}]} where \var{cc} is
a list of the tty special characters (each a string of length 1,
except the items with indices \code{VMIN} and \code{VTIME}, which are
integers when these fields are defined).  The interpretation of the
flags and the speeds as well as the indexing in the \var{cc} array
must be done using the symbolic constants defined in the
\code{TERMIOS} module.
\end{funcdesc}

\begin{funcdesc}{tcsetattr}{fd\, when\, attributes}
Set the tty attributes for file descriptor \var{fd} from the
\var{attributes}, which is a list like the one returned by
\code{tcgetattr()}.  The \var{when} argument determines when the
attributes are changed: \code{TERMIOS.TCSANOW} to change immediately,
\code{TERMIOS.TCSADRAIN} to change after transmitting all queued
output, or \code{TERMIOS.TCSAFLUSH} to change after transmitting all
queued output and discarding all queued input.
\end{funcdesc}

\begin{funcdesc}{tcsendbreak}{fd\, duration}
Send a break on file descriptor \var{fd}.  A zero \var{duration} sends
a break for 0.25--0.5 seconds; a nonzero \var{duration} has a system
dependent meaning.
\end{funcdesc}

\begin{funcdesc}{tcdrain}{fd}
Wait until all output written to file descriptor \var{fd} has been
transmitted.
\end{funcdesc}

\begin{funcdesc}{tcflush}{fd\, queue}
Discard queued data on file descriptor \var{fd}.  The \var{queue}
selector specifies which queue: \code{TERMIOS.TCIFLUSH} for the input
queue, \code{TERMIOS.TCOFLUSH} for the output queue, or
\code{TERMIOS.TCIOFLUSH} for both queues.
\end{funcdesc}

\begin{funcdesc}{tcflow}{fd\, action}
Suspend or resume input or output on file descriptor \var{fd}.  The
\var{action} argument can be \code{TERMIOS.TCOOFF} to suspend output,
\code{TERMIOS.TCOON} to restart output, \code{TERMIOS.TCIOFF} to
suspend input, or \code{TERMIOS.TCION} to restart input.
\end{funcdesc}

\subsection{Example}
\nodename{termios Example}

Here's a function that prompts for a password with echoing turned off.
Note the technique using a separate \code{termios.tcgetattr()} call
and a \code{try {\ldots} finally} statement to ensure that the old tty
attributes are restored exactly no matter what happens:

\begin{verbatim}
def getpass(prompt = "Password: "):
    import termios, TERMIOS, sys
    fd = sys.stdin.fileno()
    old = termios.tcgetattr(fd)
    new = termios.tcgetattr(fd)
    new[3] = new[3] & ~TERMIOS.ECHO          # lflags
    try:
        termios.tcsetattr(fd, TERMIOS.TCSADRAIN, new)
        passwd = raw_input(prompt)
    finally:
        termios.tcsetattr(fd, TERMIOS.TCSADRAIN, old)
    return passwd
\end{verbatim}


\section{Standard Module \sectcode{TERMIOS}}
\stmodindex{TERMIOS}
\indexii{Posix}{I/O control}
\indexii{tty}{I/O control}

\renewcommand{\indexsubitem}{(in module TERMIOS)}

This module defines the symbolic constants required to use the
\code{termios} module (see the previous section).  See the Posix or
\UNIX{} manual pages (or the source) for a list of those constants.

Note: this module resides in a system-dependent subdirectory of the
Python library directory.  You may have to generate it for your
particular system using the script \file{Tools/scripts/h2py.py}.

% Manual text by Jaap Vermeulen
\section{Built-in Module \sectcode{fcntl}}
\bimodindex{fcntl}
\indexii{\UNIX{}}{file control}
\indexii{\UNIX{}}{I/O control}

This module performs file control and I/O control on file descriptors.
It is an interface to the \dfn{fcntl()} and \dfn{ioctl()} \UNIX{} routines.
File descriptors can be obtained with the \dfn{fileno()} method of a
file or socket object.

The module defines the following functions:

\renewcommand{\indexsubitem}{(in module struct)}

\begin{funcdesc}{fcntl}{fd\, op\optional{\, arg}}
  Perform the requested operation on file descriptor \code{\var{fd}}.
  The operation is defined by \code{\var{op}} and is operating system
  dependent.  Typically these codes can be retrieved from the library
  module \code{FCNTL}. The argument \code{\var{arg}} is optional, and
  defaults to the integer value \code{0}.  When
  it is present, it can either be an integer value, or a string.  With
  the argument missing or an integer value, the return value of this
  function is the integer return value of the real \code{fcntl()}
  call.  When the argument is a string it represents a binary
  structure, e.g.\ created by \code{struct.pack()}. The binary data is
  copied to a buffer whose address is passed to the real \code{fcntl()}
  call.  The return value after a successful call is the contents of
  the buffer, converted to a string object.  In case the
  \code{fcntl()} fails, an \code{IOError} will be raised.
\end{funcdesc}

\begin{funcdesc}{ioctl}{fd\, op\, arg}
  This function is identical to the \code{fcntl()} function, except
  that the operations are typically defined in the library module
  \code{IOCTL}.
\end{funcdesc}

\begin{funcdesc}{flock}{fd\, op}
Perform the lock operation \var{op} on file descriptor \var{fd}.
See the Unix manual for details.  (On some systems, this function is
emulated using \code{fcntl}.)
\end{funcdesc}

\begin{funcdesc}{lockf}{fd\, code\, \optional{len\, \optional{start\, \optional{whence}}}}
This is a wrapper around the \code{F_SETLK} and \code{F_SETLKW}
\code{fcntl()} calls.  See the Unix manual for details.
\end{funcdesc}

If the library modules \code{FCNTL} or \code{IOCTL} are missing, you
can find the opcodes in the C include files \code{sys/fcntl} and
\code{sys/ioctl}. You can create the modules yourself with the h2py
script, found in the \code{Tools/scripts} directory.
\stmodindex{FCNTL}
\stmodindex{IOCTL}

Examples (all on a SVR4 compliant system):

\bcode\begin{verbatim}
import struct, FCNTL

file = open(...)
rv = fcntl(file.fileno(), FCNTL.O_NDELAY, 1)

lockdata = struct.pack('hhllhh', FCNTL.F_WRLCK, 0, 0, 0, 0, 0)
rv = fcntl(file.fileno(), FCNTL.F_SETLKW, lockdata)
\end{verbatim}\ecode

Note that in the first example the return value variable \code{rv} will
hold an integer value; in the second example it will hold a string
value.  The structure lay-out for the \var{lockadata} variable is
system dependent -- therefore using the \code{flock()} call may be
better.

% Manual text and implementation by Jaap Vermeulen
\section{Standard Module \sectcode{posixfile}}
\bimodindex{posixfile}
\indexii{posix}{file object}

\emph{Note:} This module will become obsolete in a future release.
The locking operation that it provides is done better and more
portably by the \code{fcntl.lockf()} call.

This module implements some additional functionality over the built-in
file objects.  In particular, it implements file locking, control over
the file flags, and an easy interface to duplicate the file object.
The module defines a new file object, the posixfile object.  It
has all the standard file object methods and adds the methods
described below.  This module only works for certain flavors of
\UNIX{}, since it uses \code{fcntl()} for file locking.

To instantiate a posixfile object, use the \code{open()} function in
the posixfile module.  The resulting object looks and feels roughly
the same as a standard file object.

The posixfile module defines the following constants:

\renewcommand{\indexsubitem}{(in module posixfile)}
\begin{datadesc}{SEEK_SET}
offset is calculated from the start of the file
\end{datadesc}

\begin{datadesc}{SEEK_CUR}
offset is calculated from the current position in the file
\end{datadesc}

\begin{datadesc}{SEEK_END}
offset is calculated from the end of the file
\end{datadesc}

The posixfile module defines the following functions:

\renewcommand{\indexsubitem}{(in module posixfile)}

\begin{funcdesc}{open}{filename\optional{\, mode\optional{\, bufsize}}}
 Create a new posixfile object with the given filename and mode.  The
 \var{filename}, \var{mode} and \var{bufsize} arguments are
 interpreted the same way as by the built-in \code{open()} function.
\end{funcdesc}

\begin{funcdesc}{fileopen}{fileobject}
 Create a new posixfile object with the given standard file object.
 The resulting object has the same filename and mode as the original
 file object.
\end{funcdesc}

The posixfile object defines the following additional methods:

\renewcommand{\indexsubitem}{(posixfile method)}
\begin{funcdesc}{lock}{fmt\, \optional{len\optional{\, start
\optional{\, whence}}}}
 Lock the specified section of the file that the file object is
 referring to.  The format is explained
 below in a table.  The \var{len} argument specifies the length of the
 section that should be locked. The default is \code{0}. \var{start}
 specifies the starting offset of the section, where the default is
 \code{0}.  The \var{whence} argument specifies where the offset is
 relative to. It accepts one of the constants \code{SEEK_SET},
 \code{SEEK_CUR} or \code{SEEK_END}.  The default is \code{SEEK_SET}.
 For more information about the arguments refer to the fcntl
 manual page on your system.
\end{funcdesc}

\begin{funcdesc}{flags}{\optional{flags}}
 Set the specified flags for the file that the file object is referring
 to.  The new flags are ORed with the old flags, unless specified
 otherwise.  The format is explained below in a table.  Without
 the \var{flags} argument
 a string indicating the current flags is returned (this is
 the same as the '?' modifier).  For more information about the flags
 refer to the fcntl manual page on your system.
\end{funcdesc}

\begin{funcdesc}{dup}{}
 Duplicate the file object and the underlying file pointer and file
 descriptor.  The resulting object behaves as if it were newly
 opened.
\end{funcdesc}

\begin{funcdesc}{dup2}{fd}
 Duplicate the file object and the underlying file pointer and file
 descriptor.  The new object will have the given file descriptor.
 Otherwise the resulting object behaves as if it were newly opened.
\end{funcdesc}

\begin{funcdesc}{file}{}
 Return the standard file object that the posixfile object is based
 on.  This is sometimes necessary for functions that insist on a
 standard file object.
\end{funcdesc}

All methods return \code{IOError} when the request fails.

Format characters for the \code{lock()} method have the following meaning:

\begin{tableiii}{|c|l|c|}{samp}{Format}{Meaning}{}
  \lineiii{u}{unlock the specified region}{}
  \lineiii{r}{request a read lock for the specified section}{}
  \lineiii{w}{request a write lock for the specified section}{}
\end{tableiii}

In addition the following modifiers can be added to the format:

\begin{tableiii}{|c|l|c|}{samp}{Modifier}{Meaning}{Notes}
  \lineiii{|}{wait until the lock has been granted}{}
  \lineiii{?}{return the first lock conflicting with the requested lock, or
              \code{None} if there is no conflict.}{(1)} 
\end{tableiii}

Note:

(1) The lock returned is in the format \code{(mode, len, start,
whence, pid)} where mode is a character representing the type of lock
('r' or 'w').  This modifier prevents a request from being granted; it
is for query purposes only.

Format character for the \code{flags()} method have the following meaning:

\begin{tableiii}{|c|l|c|}{samp}{Format}{Meaning}{}
  \lineiii{a}{append only flag}{}
  \lineiii{c}{close on exec flag}{}
  \lineiii{n}{no delay flag (also called non-blocking flag)}{}
  \lineiii{s}{synchronization flag}{}
\end{tableiii}

In addition the following modifiers can be added to the format:

\begin{tableiii}{|c|l|c|}{samp}{Modifier}{Meaning}{Notes}
  \lineiii{!}{turn the specified flags 'off', instead of the default 'on'}{(1)}
  \lineiii{=}{replace the flags, instead of the default 'OR' operation}{(1)}
  \lineiii{?}{return a string in which the characters represent the flags that
  are set.}{(2)}
\end{tableiii}

Note:

(1) The \code{!} and \code{=} modifiers are mutually exclusive.

(2) This string represents the flags after they may have been altered
by the same call.

Examples:

\bcode\begin{verbatim}
from posixfile import *

file = open('/tmp/test', 'w')
file.lock('w|')
...
file.lock('u')
file.close()
\end{verbatim}\ecode

\section{Built-in Module \sectcode{syslog}}
\bimodindex{syslog}

This module provides an interface to the Unix \code{syslog} library
routines.  Refer to the \UNIX{} manual pages for a detailed description
of the \code{syslog} facility.

The module defines the following functions:

\begin{funcdesc}{syslog}{\optional{priority\,} message}
Send the string \var{message} to the system logger.
A trailing newline is added if necessary.
Each message is tagged with a priority composed of a \var{facility} and
a \var{level}.
The optional \var{priority} argument, which defaults to
\code{(LOG_USER | LOG_INFO)}, determines the message priority.
\end{funcdesc}

\begin{funcdesc}{openlog}{ident\, \optional{logopt\, \optional{facility}}}
Logging options other than the defaults can be set by explicitly opening
the log file with \code{openlog()} prior to calling \code{syslog()}.
The defaults are (usually) \var{ident} = \samp{syslog}, \var{logopt} = 0,
\var{facility} = \code{LOG_USER}.
The \var{ident} argument is a string which is prepended to every message.
The optional \var{logopt} argument is a bit field - see below for possible
values to combine.
The optional \var{facility} argument sets the default facility for messages
which do not have a facility explicitly encoded.
\end{funcdesc}

\begin{funcdesc}{closelog}{}
Close the log file.
\end{funcdesc}

\begin{funcdesc}{setlogmask}{maskpri}
This function set the priority mask to \var{maskpri} and returns the
previous mask value.
Calls to \code{syslog} with a priority level not set in \var{maskpri}
are ignored.
The default is to log all priorities.
The function \code{LOG_MASK(\var{pri})} calculates the mask for the
individual priority \var{pri}.
The function \code{LOG_UPTO(\var{pri})} calculates the mask for all priorities
up to and including \var{pri}.
\end{funcdesc}

The module defines the following constants:

\begin{description}

\item[Priority levels (high to low):]

\code{LOG_EMERG}, \code{LOG_ALERT}, \code{LOG_CRIT}, \code{LOG_ERR},
\code{LOG_WARNING}, \code{LOG_NOTICE}, \code{LOG_INFO}, \code{LOG_DEBUG}.

\item[Facilities:]

\code{LOG_KERN}, \code{LOG_USER}, \code{LOG_MAIL}, \code{LOG_DAEMON},
\code{LOG_AUTH}, \code{LOG_LPR}, \code{LOG_NEWS}, \code{LOG_UUCP},
\code{LOG_CRON} and \code{LOG_LOCAL0} to \code{LOG_LOCAL7}.

\item[Log options:]

\code{LOG_PID}, \code{LOG_CONS}, \code{LOG_NDELAY}, \code{LOG_NOWAIT}
and \code{LOG_PERROR} if defined in \file{syslog.h}.

\end{description}


\chapter{The Python Debugger}
\stmodindex{pdb}
\index{debugging}

\renewcommand{\indexsubitem}{(in module pdb)}

The module \code{pdb} defines an interactive source code debugger for
Python programs.  It supports setting breakpoints and single stepping
at the source line level, inspection of stack frames, source code
listing, and evaluation of arbitrary Python code in the context of any
stack frame.  It also supports post-mortem debugging and can be called
under program control.

The debugger is extensible --- it is actually defined as a class
\code{Pdb}.  This is currently undocumented but easily understood by
reading the source.  The extension interface uses the (also
undocumented) modules \code{bdb} and \code{cmd}.
\ttindex{Pdb}
\ttindex{bdb}
\ttindex{cmd}

A primitive windowing version of the debugger also exists --- this is
module \code{wdb}, which requires STDWIN (see the chapter on STDWIN
specific modules).
\index{stdwin}
\ttindex{wdb}

The debugger's prompt is ``\code{(Pdb) }''.
Typical usage to run a program under control of the debugger is:

\begin{verbatim}
>>> import pdb
>>> import mymodule
>>> pdb.run('mymodule.test()')
> <string>(0)?()
(Pdb) continue
> <string>(1)?()
(Pdb) continue
NameError: 'spam'
> <string>(1)?()
(Pdb) 
\end{verbatim}

Typical usage to inspect a crashed program is:

\begin{verbatim}
>>> import pdb
>>> import mymodule
>>> mymodule.test()
Traceback (innermost last):
  File "<stdin>", line 1, in ?
  File "./mymodule.py", line 4, in test
    test2()
  File "./mymodule.py", line 3, in test2
    print spam
NameError: spam
>>> pdb.pm()
> ./mymodule.py(3)test2()
-> print spam
(Pdb) 
\end{verbatim}

The module defines the following functions; each enters the debugger
in a slightly different way:

\begin{funcdesc}{run}{statement\optional{\, globals\optional{\, locals}}}
Execute the \var{statement} (given as a string) under debugger
control.  The debugger prompt appears before any code is executed; you
can set breakpoints and type \code{continue}, or you can step through
the statement using \code{step} or \code{next} (all these commands are
explained below).  The optional \var{globals} and \var{locals}
arguments specify the environment in which the code is executed; by
default the dictionary of the module \code{__main__} is used.  (See
the explanation of the \code{exec} statement or the \code{eval()}
built-in function.)
\end{funcdesc}

\begin{funcdesc}{runeval}{expression\optional{\, globals\optional{\, locals}}}
Evaluate the \var{expression} (given as a a string) under debugger
control.  When \code{runeval()} returns, it returns the value of the
expression.  Otherwise this function is similar to
\code{run()}.
\end{funcdesc}

\begin{funcdesc}{runcall}{function\optional{\, argument\, ...}}
Call the \var{function} (a function or method object, not a string)
with the given arguments.  When \code{runcall()} returns, it returns
whatever the function call returned.  The debugger prompt appears as
soon as the function is entered.
\end{funcdesc}

\begin{funcdesc}{set_trace}{}
Enter the debugger at the calling stack frame.  This is useful to
hard-code a breakpoint at a given point in a program, even if the code
is not otherwise being debugged (e.g. when an assertion fails).
\end{funcdesc}

\begin{funcdesc}{post_mortem}{traceback}
Enter post-mortem debugging of the given \var{traceback} object.
\end{funcdesc}

\begin{funcdesc}{pm}{}
Enter post-mortem debugging of the traceback found in
\code{sys.last_traceback}.
\end{funcdesc}

\section{Debugger Commands}

The debugger recognizes the following commands.  Most commands can be
abbreviated to one or two letters; e.g. ``\code{h(elp)}'' means that
either ``\code{h}'' or ``\code{help}'' can be used to enter the help
command (but not ``\code{he}'' or ``\code{hel}'', nor ``\code{H}'' or
``\code{Help} or ``\code{HELP}'').  Arguments to commands must be
separated by whitespace (spaces or tabs).  Optional arguments are
enclosed in square brackets (``\code{[]}'') in the command syntax; the
square brackets must not be typed.  Alternatives in the command syntax
are separated by a vertical bar (``\code{|}'').

Entering a blank line repeats the last command entered.  Exception: if
the last command was a ``\code{list}'' command, the next 11 lines are
listed.

Commands that the debugger doesn't recognize are assumed to be Python
statements and are executed in the context of the program being
debugged.  Python statements can also be prefixed with an exclamation
point (``\code{!}'').  This is a powerful way to inspect the program
being debugged; it is even possible to change a variable or call a
function.  When an
exception occurs in such a statement, the exception name is printed
but the debugger's state is not changed.

\begin{description}

\item[h(elp) [\var{command}]]

Without argument, print the list of available commands.
With a \var{command} as argument, print help about that command.
``\code{help pdb}'' displays the full documentation file; if the
environment variable \code{PAGER} is defined, the file is piped
through that command instead.  Since the \var{command} argument must be
an identifier, ``\code{help exec}'' must be entered to get help on the
``\code{!}'' command.

\item[w(here)]

Print a stack trace, with the most recent frame at the bottom.
An arrow indicates the current frame, which determines the
context of most commands.

\item[d(own)]

Move the current frame one level down in the stack trace
(to an older frame).

\item[u(p)]

Move the current frame one level up in the stack trace
(to a newer frame).

\item[b(reak) [\var{lineno}\code{|}\var{function}]]

With a \var{lineno} argument, set a break there in the current
file.  With a \var{function} argument, set a break at the entry of
that function.  Without argument, list all breaks.

\item[cl(ear) [\var{lineno}]]

With a \var{lineno} argument, clear that break in the current file.
Without argument, clear all breaks (but first ask confirmation).

\item[s(tep)]

Execute the current line, stop at the first possible occasion
(either in a function that is called or on the next line in the
current function).

\item[n(ext)]

Continue execution until the next line in the current function
is reached or it returns.  (The difference between \code{next} and
\code{step} is that \code{step} stops inside a called function, while
\code{next} executes called functions at (nearly) full speed, only
stopping at the next line in the current function.)

\item[r(eturn)]

Continue execution until the current function returns.

\item[c(ont(inue))]

Continue execution, only stop when a breakpoint is encountered.

\item[l(ist) [\var{first} [, \var{last}]]]

List source code for the current file.  Without arguments, list 11
lines around the current line or continue the previous listing.  With
one argument, list 11 lines around at that line.  With two arguments,
list the given range; if the second argument is less than the first,
it is interpreted as a count.

\item[a(rgs)]

Print the argument list of the current function.

\item[p \var{expression}]

Evaluate the \var{expression} in the current context and print its
value.  (Note: \code{print} can also be used, but is not a debugger
command --- this executes the Python \code{print} statement.)

\item[[!] \var{statement}]

Execute the (one-line) \var{statement} in the context of
the current stack frame.
The exclamation point can be omitted unless the first word
of the statement resembles a debugger command.
To set a global variable, you can prefix the assignment
command with a ``\code{global}'' command on the same line, e.g.:
\begin{verbatim}
(Pdb) global list_options; list_options = ['-l']
(Pdb)
\end{verbatim}

\item[q(uit)]

Quit from the debugger.
The program being executed is aborted.

\end{description}

\section{How It Works}

Some changes were made to the interpreter:

\begin{itemize}
\item sys.settrace(func) sets the global trace function
\item there can also a local trace function (see later)
\end{itemize}

Trace functions have three arguments: (\var{frame}, \var{event}, \var{arg})

\begin{description}

\item[\var{frame}] is the current stack frame

\item[\var{event}] is a string: \code{'call'}, \code{'line'}, \code{'return'}
or \code{'exception'}

\item[\var{arg}] is dependent on the event type

\end{description}

A trace function should return a new trace function or None.
Class methods are accepted (and most useful!) as trace methods.

The events have the following meaning:

\begin{description}

\item[\code{'call'}]
A function is called (or some other code block entered).  The global
trace function is called; arg is the argument list to the function;
the return value specifies the local trace function.

\item[\code{'line'}]
The interpreter is about to execute a new line of code (sometimes
multiple line events on one line exist).  The local trace function is
called; arg in None; the return value specifies the new local trace
function.

\item[\code{'return'}]
A function (or other code block) is about to return.  The local trace
function is called; arg is the value that will be returned.  The trace
function's return value is ignored.

\item[\code{'exception'}]
An exception has occurred.  The local trace function is called; arg is
a triple (exception, value, traceback); the return value specifies the
new local trace function

\end{description}

Note that as an exception is propagated down the chain of callers, an
\code{'exception'} event is generated at each level.

Stack frame objects have the following read-only attributes:

\begin{description}
\item[f_code]      the code object being executed
\item[f_lineno]    the current line number (\code{-1} for \code{'call'} events)
\item[f_back]      the stack frame of the caller, or None
\item[f_locals]    dictionary containing local name bindings
\item[f_globals]   dictionary containing global name bindings
\end{description}

Code objects have the following read-only attributes:

\begin{description}
\item[co_code]     the code string
\item[co_names]    the list of names used by the code
\item[co_consts]   the list of (literal) constants used by the code
\item[co_filename] the filename from which the code was compiled
\end{description}
			% The Python Debugger

\chapter{The Python Profiler}
\stmodindex{profile}
\stmodindex{pstats}

Copyright \copyright\ 1994, by InfoSeek Corporation, all rights reserved.

Written by James Roskind%
\footnote{
Updated and converted to \LaTeX\ by Guido van Rossum.  The references to
the old profiler are left in the text, although it no longer exists.
}

Permission to use, copy, modify, and distribute this Python software
and its associated documentation for any purpose (subject to the
restriction in the following sentence) without fee is hereby granted,
provided that the above copyright notice appears in all copies, and
that both that copyright notice and this permission notice appear in
supporting documentation, and that the name of InfoSeek not be used in
advertising or publicity pertaining to distribution of the software
without specific, written prior permission.  This permission is
explicitly restricted to the copying and modification of the software
to remain in Python, compiled Python, or other languages (such as C)
wherein the modified or derived code is exclusively imported into a
Python module.

INFOSEEK CORPORATION DISCLAIMS ALL WARRANTIES WITH REGARD TO THIS
SOFTWARE, INCLUDING ALL IMPLIED WARRANTIES OF MERCHANTABILITY AND
FITNESS. IN NO EVENT SHALL INFOSEEK CORPORATION BE LIABLE FOR ANY
SPECIAL, INDIRECT OR CONSEQUENTIAL DAMAGES OR ANY DAMAGES WHATSOEVER
RESULTING FROM LOSS OF USE, DATA OR PROFITS, WHETHER IN AN ACTION OF
CONTRACT, NEGLIGENCE OR OTHER TORTIOUS ACTION, ARISING OUT OF OR IN
CONNECTION WITH THE USE OR PERFORMANCE OF THIS SOFTWARE.


The profiler was written after only programming in Python for 3 weeks.
As a result, it is probably clumsy code, but I don't know for sure yet
'cause I'm a beginner :-).  I did work hard to make the code run fast,
so that profiling would be a reasonable thing to do.  I tried not to
repeat code fragments, but I'm sure I did some stuff in really awkward
ways at times.  Please send suggestions for improvements to:
\code{jar@netscape.com}.  I won't promise \emph{any} support.  ...but
I'd appreciate the feedback.


\section{Introduction to the profiler}
\nodename{Profiler Introduction}

A \dfn{profiler} is a program that describes the run time performance
of a program, providing a variety of statistics.  This documentation
describes the profiler functionality provided in the modules
\code{profile} and \code{pstats.}  This profiler provides
\dfn{deterministic profiling} of any Python programs.  It also
provides a series of report generation tools to allow users to rapidly
examine the results of a profile operation.


\section{How Is This Profiler Different From The Old Profiler?}
\nodename{Profiler Changes}

The big changes from old profiling module are that you get more
information, and you pay less CPU time.  It's not a trade-off, it's a
trade-up.

To be specific:

\begin{description}

\item[Bugs removed:]
Local stack frame is no longer molested, execution time is now charged
to correct functions.

\item[Accuracy increased:]
Profiler execution time is no longer charged to user's code,
calibration for platform is supported, file reads are not done \emph{by}
profiler \emph{during} profiling (and charged to user's code!).

\item[Speed increased:]
Overhead CPU cost was reduced by more than a factor of two (perhaps a
factor of five), lightweight profiler module is all that must be
loaded, and the report generating module (\code{pstats}) is not needed
during profiling.

\item[Recursive functions support:]
Cumulative times in recursive functions are correctly calculated;
recursive entries are counted.

\item[Large growth in report generating UI:]
Distinct profiles runs can be added together forming a comprehensive
report; functions that import statistics take arbitrary lists of
files; sorting criteria is now based on keywords (instead of 4 integer
options); reports shows what functions were profiled as well as what
profile file was referenced; output format has been improved.

\end{description}


\section{Instant Users Manual}

This section is provided for users that ``don't want to read the
manual.'' It provides a very brief overview, and allows a user to
rapidly perform profiling on an existing application.

To profile an application with a main entry point of \samp{foo()}, you
would add the following to your module:

\begin{verbatim}
    import profile
    profile.run("foo()")
\end{verbatim}

The above action would cause \samp{foo()} to be run, and a series of
informative lines (the profile) to be printed.  The above approach is
most useful when working with the interpreter.  If you would like to
save the results of a profile into a file for later examination, you
can supply a file name as the second argument to the \code{run()}
function:

\begin{verbatim}
    import profile
    profile.run("foo()", 'fooprof')
\end{verbatim}

When you wish to review the profile, you should use the methods in the
\code{pstats} module.  Typically you would load the statistics data as
follows:

\begin{verbatim}
    import pstats
    p = pstats.Stats('fooprof')
\end{verbatim}

The class \code{Stats} (the above code just created an instance of
this class) has a variety of methods for manipulating and printing the
data that was just read into \samp{p}.  When you ran
\code{profile.run()} above, what was printed was the result of three
method calls:

\begin{verbatim}
    p.strip_dirs().sort_stats(-1).print_stats()
\end{verbatim}

The first method removed the extraneous path from all the module
names. The second method sorted all the entries according to the
standard module/line/name string that is printed (this is to comply
with the semantics of the old profiler).  The third method printed out
all the statistics.  You might try the following sort calls:

\begin{verbatim}
    p.sort_stats('name')
    p.print_stats()
\end{verbatim}

The first call will actually sort the list by function name, and the
second call will print out the statistics.  The following are some
interesting calls to experiment with:

\begin{verbatim}
    p.sort_stats('cumulative').print_stats(10)
\end{verbatim}

This sorts the profile by cumulative time in a function, and then only
prints the ten most significant lines.  If you want to understand what
algorithms are taking time, the above line is what you would use.

If you were looking to see what functions were looping a lot, and
taking a lot of time, you would do:

\begin{verbatim}
    p.sort_stats('time').print_stats(10)
\end{verbatim}

to sort according to time spent within each function, and then print
the statistics for the top ten functions.

You might also try:

\begin{verbatim}
    p.sort_stats('file').print_stats('__init__')
\end{verbatim}

This will sort all the statistics by file name, and then print out
statistics for only the class init methods ('cause they are spelled
with \code{__init__} in them).  As one final example, you could try:

\begin{verbatim}
    p.sort_stats('time', 'cum').print_stats(.5, 'init')
\end{verbatim}

This line sorts statistics with a primary key of time, and a secondary
key of cumulative time, and then prints out some of the statistics.
To be specific, the list is first culled down to 50\% (re: \samp{.5})
of its original size, then only lines containing \code{init} are
maintained, and that sub-sub-list is printed.

If you wondered what functions called the above functions, you could
now (\samp{p} is still sorted according to the last criteria) do:

\begin{verbatim}
    p.print_callers(.5, 'init')
\end{verbatim}

and you would get a list of callers for each of the listed functions. 

If you want more functionality, you're going to have to read the
manual, or guess what the following functions do:

\begin{verbatim}
    p.print_callees()
    p.add('fooprof')
\end{verbatim}


\section{What Is Deterministic Profiling?}
\nodename{Deterministic Profiling}

\dfn{Deterministic profiling} is meant to reflect the fact that all
\dfn{function call}, \dfn{function return}, and \dfn{exception} events
are monitored, and precise timings are made for the intervals between
these events (during which time the user's code is executing).  In
contrast, \dfn{statistical profiling} (which is not done by this
module) randomly samples the effective instruction pointer, and
deduces where time is being spent.  The latter technique traditionally
involves less overhead (as the code does not need to be instrumented),
but provides only relative indications of where time is being spent.

In Python, since there is an interpreter active during execution, the
presence of instrumented code is not required to do deterministic
profiling.  Python automatically provides a \dfn{hook} (optional
callback) for each event.  In addition, the interpreted nature of
Python tends to add so much overhead to execution, that deterministic
profiling tends to only add small processing overhead in typical
applications.  The result is that deterministic profiling is not that
expensive, yet provides extensive run time statistics about the
execution of a Python program.

Call count statistics can be used to identify bugs in code (surprising
counts), and to identify possible inline-expansion points (high call
counts).  Internal time statistics can be used to identify ``hot
loops'' that should be carefully optimized.  Cumulative time
statistics should be used to identify high level errors in the
selection of algorithms.  Note that the unusual handling of cumulative
times in this profiler allows statistics for recursive implementations
of algorithms to be directly compared to iterative implementations.


\section{Reference Manual}

\renewcommand{\indexsubitem}{(profiler function)}

The primary entry point for the profiler is the global function
\code{profile.run()}.  It is typically used to create any profile
information.  The reports are formatted and printed using methods of
the class \code{pstats.Stats}.  The following is a description of all
of these standard entry points and functions.  For a more in-depth
view of some of the code, consider reading the later section on
Profiler Extensions, which includes discussion of how to derive
``better'' profilers from the classes presented, or reading the source
code for these modules.

\begin{funcdesc}{profile.run}{string\optional{\, filename\optional{\, ...}}}

This function takes a single argument that has can be passed to the
\code{exec} statement, and an optional file name.  In all cases this
routine attempts to \code{exec} its first argument, and gather profiling
statistics from the execution. If no file name is present, then this
function automatically prints a simple profiling report, sorted by the
standard name string (file/line/function-name) that is presented in
each line.  The following is a typical output from such a call:

\small{
\begin{verbatim}
      main()
      2706 function calls (2004 primitive calls) in 4.504 CPU seconds

Ordered by: standard name

ncalls  tottime  percall  cumtime  percall filename:lineno(function)
     2    0.006    0.003    0.953    0.477 pobject.py:75(save_objects)
  43/3    0.533    0.012    0.749    0.250 pobject.py:99(evaluate)
 ...
\end{verbatim}
}

The first line indicates that this profile was generated by the call:\\
\code{profile.run('main()')}, and hence the exec'ed string is
\code{'main()'}.  The second line indicates that 2706 calls were
monitored.  Of those calls, 2004 were \dfn{primitive}.  We define
\dfn{primitive} to mean that the call was not induced via recursion.
The next line: \code{Ordered by:\ standard name}, indicates that
the text string in the far right column was used to sort the output.
The column headings include:

\begin{description}

\item[ncalls ]
for the number of calls, 

\item[tottime ]
for the total time spent in the given function (and excluding time
made in calls to sub-functions),

\item[percall ]
is the quotient of \code{tottime} divided by \code{ncalls}

\item[cumtime ]
is the total time spent in this and all subfunctions (i.e., from
invocation till exit). This figure is accurate \emph{even} for recursive
functions.

\item[percall ]
is the quotient of \code{cumtime} divided by primitive calls

\item[filename:lineno(function) ]
provides the respective data of each function

\end{description}

When there are two numbers in the first column (e.g.: \samp{43/3}),
then the latter is the number of primitive calls, and the former is
the actual number of calls.  Note that when the function does not
recurse, these two values are the same, and only the single figure is
printed.

\end{funcdesc}

\begin{funcdesc}{pstats.Stats}{filename\optional{\, ...}}
This class constructor creates an instance of a ``statistics object''
from a \var{filename} (or set of filenames).  \code{Stats} objects are
manipulated by methods, in order to print useful reports.

The file selected by the above constructor must have been created by
the corresponding version of \code{profile}.  To be specific, there is
\emph{NO} file compatibility guaranteed with future versions of this
profiler, and there is no compatibility with files produced by other
profilers (e.g., the old system profiler).

If several files are provided, all the statistics for identical
functions will be coalesced, so that an overall view of several
processes can be considered in a single report.  If additional files
need to be combined with data in an existing \code{Stats} object, the
\code{add()} method can be used.
\end{funcdesc}


\subsection{The \sectcode{Stats} Class}

\renewcommand{\indexsubitem}{(Stats method)}

\begin{funcdesc}{strip_dirs}{}
This method for the \code{Stats} class removes all leading path information
from file names.  It is very useful in reducing the size of the
printout to fit within (close to) 80 columns.  This method modifies
the object, and the stripped information is lost.  After performing a
strip operation, the object is considered to have its entries in a
``random'' order, as it was just after object initialization and
loading.  If \code{strip_dirs()} causes two function names to be
indistinguishable (i.e., they are on the same line of the same
filename, and have the same function name), then the statistics for
these two entries are accumulated into a single entry.
\end{funcdesc}


\begin{funcdesc}{add}{filename\optional{\, ...}}
This method of the \code{Stats} class accumulates additional profiling
information into the current profiling object.  Its arguments should
refer to filenames created by the corresponding version of
\code{profile.run()}.  Statistics for identically named (re: file,
line, name) functions are automatically accumulated into single
function statistics.
\end{funcdesc}

\begin{funcdesc}{sort_stats}{key\optional{\, ...}}
This method modifies the \code{Stats} object by sorting it according to the
supplied criteria.  The argument is typically a string identifying the
basis of a sort (example: \code{"time"} or \code{"name"}).

When more than one key is provided, then additional keys are used as
secondary criteria when the there is equality in all keys selected
before them.  For example, sort_stats('name', 'file') will sort all
the entries according to their function name, and resolve all ties
(identical function names) by sorting by file name.

Abbreviations can be used for any key names, as long as the
abbreviation is unambiguous.  The following are the keys currently
defined: 

\begin{tableii}{|l|l|}{code}{Valid Arg}{Meaning}
\lineii{"calls"}{call count}
\lineii{"cumulative"}{cumulative time}
\lineii{"file"}{file name}
\lineii{"module"}{file name}
\lineii{"pcalls"}{primitive call count}
\lineii{"line"}{line number}
\lineii{"name"}{function name}
\lineii{"nfl"}{name/file/line}
\lineii{"stdname"}{standard name}
\lineii{"time"}{internal time}
\end{tableii}

Note that all sorts on statistics are in descending order (placing
most time consuming items first), where as name, file, and line number
searches are in ascending order (i.e., alphabetical). The subtle
distinction between \code{"nfl"} and \code{"stdname"} is that the
standard name is a sort of the name as printed, which means that the
embedded line numbers get compared in an odd way.  For example, lines
3, 20, and 40 would (if the file names were the same) appear in the
string order 20, 3 and 40.  In contrast, \code{"nfl"} does a numeric
compare of the line numbers.  In fact, \code{sort_stats("nfl")} is the
same as \code{sort_stats("name", "file", "line")}.

For compatibility with the old profiler, the numeric arguments
\samp{-1}, \samp{0}, \samp{1}, and \samp{2} are permitted.  They are
interpreted as \code{"stdname"}, \code{"calls"}, \code{"time"}, and
\code{"cumulative"} respectively.  If this old style format (numeric)
is used, only one sort key (the numeric key) will be used, and
additional arguments will be silently ignored.
\end{funcdesc}


\begin{funcdesc}{reverse_order}{}
This method for the \code{Stats} class reverses the ordering of the basic
list within the object.  This method is provided primarily for
compatibility with the old profiler.  Its utility is questionable
now that ascending vs descending order is properly selected based on
the sort key of choice.
\end{funcdesc}

\begin{funcdesc}{print_stats}{restriction\optional{\, ...}}
This method for the \code{Stats} class prints out a report as described
in the \code{profile.run()} definition.

The order of the printing is based on the last \code{sort_stats()}
operation done on the object (subject to caveats in \code{add()} and
\code{strip_dirs())}.

The arguments provided (if any) can be used to limit the list down to
the significant entries.  Initially, the list is taken to be the
complete set of profiled functions.  Each restriction is either an
integer (to select a count of lines), or a decimal fraction between
0.0 and 1.0 inclusive (to select a percentage of lines), or a regular
expression (to pattern match the standard name that is printed).  If
several restrictions are provided, then they are applied sequentially.
For example:

\begin{verbatim}
    print_stats(.1, "foo:")
\end{verbatim}

would first limit the printing to first 10\% of list, and then only
print functions that were part of filename \samp{.*foo:}.  In
contrast, the command:

\begin{verbatim}
    print_stats("foo:", .1)
\end{verbatim}

would limit the list to all functions having file names \samp{.*foo:},
and then proceed to only print the first 10\% of them.
\end{funcdesc}


\begin{funcdesc}{print_callers}{restrictions\optional{\, ...}}
This method for the \code{Stats} class prints a list of all functions
that called each function in the profiled database.  The ordering is
identical to that provided by \code{print_stats()}, and the definition
of the restricting argument is also identical.  For convenience, a
number is shown in parentheses after each caller to show how many
times this specific call was made.  A second non-parenthesized number
is the cumulative time spent in the function at the right.
\end{funcdesc}

\begin{funcdesc}{print_callees}{restrictions\optional{\, ...}}
This method for the \code{Stats} class prints a list of all function
that were called by the indicated function.  Aside from this reversal
of direction of calls (re: called vs was called by), the arguments and
ordering are identical to the \code{print_callers()} method.
\end{funcdesc}

\begin{funcdesc}{ignore}{}
This method of the \code{Stats} class is used to dispose of the value
returned by earlier methods.  All standard methods in this class
return the instance that is being processed, so that the commands can
be strung together.  For example:

\begin{verbatim}
pstats.Stats('foofile').strip_dirs().sort_stats('cum') \
                       .print_stats().ignore()
\end{verbatim}

would perform all the indicated functions, but it would not return
the final reference to the \code{Stats} instance.%
\footnote{
This was once necessary, when Python would print any unused expression
result that was not \code{None}.  The method is still defined for
backward compatibility.
}
\end{funcdesc}


\section{Limitations}

There are two fundamental limitations on this profiler.  The first is
that it relies on the Python interpreter to dispatch \dfn{call},
\dfn{return}, and \dfn{exception} events.  Compiled C code does not
get interpreted, and hence is ``invisible'' to the profiler.  All time
spent in C code (including builtin functions) will be charged to the
Python function that invoked the C code.  If the C code calls out
to some native Python code, then those calls will be profiled
properly.

The second limitation has to do with accuracy of timing information.
There is a fundamental problem with deterministic profilers involving
accuracy.  The most obvious restriction is that the underlying ``clock''
is only ticking at a rate (typically) of about .001 seconds.  Hence no
measurements will be more accurate that that underlying clock.  If
enough measurements are taken, then the ``error'' will tend to average
out. Unfortunately, removing this first error induces a second source
of error...

The second problem is that it ``takes a while'' from when an event is
dispatched until the profiler's call to get the time actually
\emph{gets} the state of the clock.  Similarly, there is a certain lag
when exiting the profiler event handler from the time that the clock's
value was obtained (and then squirreled away), until the user's code
is once again executing.  As a result, functions that are called many
times, or call many functions, will typically accumulate this error.
The error that accumulates in this fashion is typically less than the
accuracy of the clock (i.e., less than one clock tick), but it
\emph{can} accumulate and become very significant.  This profiler
provides a means of calibrating itself for a given platform so that
this error can be probabilistically (i.e., on the average) removed.
After the profiler is calibrated, it will be more accurate (in a least
square sense), but it will sometimes produce negative numbers (when
call counts are exceptionally low, and the gods of probability work
against you :-). )  Do \emph{NOT} be alarmed by negative numbers in
the profile.  They should \emph{only} appear if you have calibrated
your profiler, and the results are actually better than without
calibration.


\section{Calibration}

The profiler class has a hard coded constant that is added to each
event handling time to compensate for the overhead of calling the time
function, and socking away the results.  The following procedure can
be used to obtain this constant for a given platform (see discussion
in section Limitations above).

\begin{verbatim}
    import profile
    pr = profile.Profile()
    pr.calibrate(100)
    pr.calibrate(100)
    pr.calibrate(100)
\end{verbatim}

The argument to calibrate() is the number of times to try to do the
sample calls to get the CPU times.  If your computer is \emph{very}
fast, you might have to do:

\begin{verbatim}
    pr.calibrate(1000)
\end{verbatim}

or even:

\begin{verbatim}
    pr.calibrate(10000)
\end{verbatim}

The object of this exercise is to get a fairly consistent result.
When you have a consistent answer, you are ready to use that number in
the source code.  For a Sun Sparcstation 1000 running Solaris 2.3, the
magical number is about .00053.  If you have a choice, you are better
off with a smaller constant, and your results will ``less often'' show
up as negative in profile statistics.

The following shows how the trace_dispatch() method in the Profile
class should be modified to install the calibration constant on a Sun
Sparcstation 1000:

\begin{verbatim}
    def trace_dispatch(self, frame, event, arg):
        t = self.timer()
        t = t[0] + t[1] - self.t - .00053 # Calibration constant

        if self.dispatch[event](frame,t):
            t = self.timer()
            self.t = t[0] + t[1]
        else:
            r = self.timer()
            self.t = r[0] + r[1] - t # put back unrecorded delta
        return
\end{verbatim}

Note that if there is no calibration constant, then the line
containing the callibration constant should simply say:

\begin{verbatim}
        t = t[0] + t[1] - self.t  # no calibration constant
\end{verbatim}

You can also achieve the same results using a derived class (and the
profiler will actually run equally fast!!), but the above method is
the simplest to use.  I could have made the profiler ``self
calibrating'', but it would have made the initialization of the
profiler class slower, and would have required some \emph{very} fancy
coding, or else the use of a variable where the constant \samp{.00053}
was placed in the code shown.  This is a \strong{VERY} critical
performance section, and there is no reason to use a variable lookup
at this point, when a constant can be used.


\section{Extensions --- Deriving Better Profilers}
\nodename{Profiler Extensions}

The \code{Profile} class of module \code{profile} was written so that
derived classes could be developed to extend the profiler.  Rather
than describing all the details of such an effort, I'll just present
the following two examples of derived classes that can be used to do
profiling.  If the reader is an avid Python programmer, then it should
be possible to use these as a model and create similar (and perchance
better) profile classes.

If all you want to do is change how the timer is called, or which
timer function is used, then the basic class has an option for that in
the constructor for the class.  Consider passing the name of a
function to call into the constructor:

\begin{verbatim}
    pr = profile.Profile(your_time_func)
\end{verbatim}

The resulting profiler will call \code{your_time_func()} instead of
\code{os.times()}.  The function should return either a single number
or a list of numbers (like what \code{os.times()} returns).  If the
function returns a single time number, or the list of returned numbers
has length 2, then you will get an especially fast version of the
dispatch routine.

Be warned that you \emph{should} calibrate the profiler class for the
timer function that you choose.  For most machines, a timer that
returns a lone integer value will provide the best results in terms of
low overhead during profiling.  (os.times is \emph{pretty} bad, 'cause
it returns a tuple of floating point values, so all arithmetic is
floating point in the profiler!).  If you want to substitute a
better timer in the cleanest fashion, you should derive a class, and
simply put in the replacement dispatch method that better handles your
timer call, along with the appropriate calibration constant :-).


\subsection{OldProfile Class}

The following derived profiler simulates the old style profiler,
providing errant results on recursive functions. The reason for the
usefulness of this profiler is that it runs faster (i.e., less
overhead) than the old profiler.  It still creates all the caller
stats, and is quite useful when there is \emph{no} recursion in the
user's code.  It is also a lot more accurate than the old profiler, as
it does not charge all its overhead time to the user's code.

\begin{verbatim}
class OldProfile(Profile):

    def trace_dispatch_exception(self, frame, t):
        rt, rtt, rct, rfn, rframe, rcur = self.cur
        if rcur and not rframe is frame:
            return self.trace_dispatch_return(rframe, t)
        return 0

    def trace_dispatch_call(self, frame, t):
        fn = `frame.f_code`
        
        self.cur = (t, 0, 0, fn, frame, self.cur)
        if self.timings.has_key(fn):
            tt, ct, callers = self.timings[fn]
            self.timings[fn] = tt, ct, callers
        else:
            self.timings[fn] = 0, 0, {}
        return 1

    def trace_dispatch_return(self, frame, t):
        rt, rtt, rct, rfn, frame, rcur = self.cur
        rtt = rtt + t
        sft = rtt + rct

        pt, ptt, pct, pfn, pframe, pcur = rcur
        self.cur = pt, ptt+rt, pct+sft, pfn, pframe, pcur

        tt, ct, callers = self.timings[rfn]
        if callers.has_key(pfn):
            callers[pfn] = callers[pfn] + 1
        else:
            callers[pfn] = 1
        self.timings[rfn] = tt+rtt, ct + sft, callers

        return 1


    def snapshot_stats(self):
        self.stats = {}
        for func in self.timings.keys():
            tt, ct, callers = self.timings[func]
            nor_func = self.func_normalize(func)
            nor_callers = {}
            nc = 0
            for func_caller in callers.keys():
                nor_callers[self.func_normalize(func_caller)]=\
                      callers[func_caller]
                nc = nc + callers[func_caller]
            self.stats[nor_func] = nc, nc, tt, ct, nor_callers
\end{verbatim}
        

\subsection{HotProfile Class}

This profiler is the fastest derived profile example.  It does not
calculate caller-callee relationships, and does not calculate
cumulative time under a function.  It only calculates time spent in a
function, so it runs very quickly (re: very low overhead).  In truth,
the basic profiler is so fast, that is probably not worth the savings
to give up the data, but this class still provides a nice example.

\begin{verbatim}
class HotProfile(Profile):

    def trace_dispatch_exception(self, frame, t):
        rt, rtt, rfn, rframe, rcur = self.cur
        if rcur and not rframe is frame:
            return self.trace_dispatch_return(rframe, t)
        return 0

    def trace_dispatch_call(self, frame, t):
        self.cur = (t, 0, frame, self.cur)
        return 1

    def trace_dispatch_return(self, frame, t):
        rt, rtt, frame, rcur = self.cur

        rfn = `frame.f_code`

        pt, ptt, pframe, pcur = rcur
        self.cur = pt, ptt+rt, pframe, pcur

        if self.timings.has_key(rfn):
            nc, tt = self.timings[rfn]
            self.timings[rfn] = nc + 1, rt + rtt + tt
        else:
            self.timings[rfn] =      1, rt + rtt

        return 1


    def snapshot_stats(self):
        self.stats = {}
        for func in self.timings.keys():
            nc, tt = self.timings[func]
            nor_func = self.func_normalize(func)
            self.stats[nor_func] = nc, nc, tt, 0, {}
\end{verbatim}
		% The Python Profiler

\chapter{Internet and WWW Services}
\nodename{Internet and WWW}
\index{WWW}
\index{Internet}
\index{World-Wide Web}

The modules described in this chapter provide various services to
World-Wide Web (WWW) clients and/or services, and a few modules
related to news and email.  They are all implemented in Python.  Some
of these modules require the presence of the system-dependent module
\code{sockets}, which is currently only fully supported on Unix and
Windows NT.  Here is an overview:

\begin{description}

\item[cgi]
--- Common Gateway Interface, used to interpret forms in server-side
scripts.

\item[urllib]
--- Open an arbitrary object given by URL (requires sockets).

\item[httplib]
--- HTTP protocol client (requires sockets).

\item[ftplib]
--- FTP protocol client (requires sockets).

\item[gopherlib]
--- Gopher protocol client (requires sockets).

\item[nntplib]
--- NNTP protocol client (requires sockets).

\item[urlparse]
--- Parse a URL string into a tuple (addressing scheme identifier, network
location, path, parameters, query string, fragment identifier).

\item[sgmllib]
--- Only as much of an SGML parser as needed to parse HTML.

\item[htmllib]
--- A (slow) parser for HTML documents.

\item[formatter]
--- Generic output formatter and device interface.

\item[rfc822]
--- Parse RFC-822 style mail headers.

\item[mimetools]
--- Tools for parsing MIME style message bodies.

\end{description}
			% Internet and WWW Services
\section{Standard Module \sectcode{cgi}}
\stmodindex{cgi}
\indexii{WWW}{server}
\indexii{CGI}{protocol}
\indexii{HTTP}{protocol}
\indexii{MIME}{headers}
\index{URL}

\renewcommand{\indexsubitem}{(in module cgi)}

Support module for CGI (Common Gateway Interface) scripts.

This module defines a number of utilities for use by CGI scripts
written in Python.

\subsection{Introduction}
\nodename{Introduction to the CGI module}

A CGI script is invoked by an HTTP server, usually to process user
input submitted through an HTML \code{<FORM>} or \code{<ISINPUT>} element.

Most often, CGI scripts live in the server's special \code{cgi-bin}
directory.  The HTTP server places all sorts of information about the
request (such as the client's hostname, the requested URL, the query
string, and lots of other goodies) in the script's shell environment,
executes the script, and sends the script's output back to the client.

The script's input is connected to the client too, and sometimes the
form data is read this way; at other times the form data is passed via
the ``query string'' part of the URL.  This module (\code{cgi.py}) is intended
to take care of the different cases and provide a simpler interface to
the Python script.  It also provides a number of utilities that help
in debugging scripts, and the latest addition is support for file
uploads from a form (if your browser supports it -- Grail 0.3 and
Netscape 2.0 do).

The output of a CGI script should consist of two sections, separated
by a blank line.  The first section contains a number of headers,
telling the client what kind of data is following.  Python code to
generate a minimal header section looks like this:

\begin{verbatim}
	print "Content-type: text/html"	# HTML is following
	print				# blank line, end of headers
\end{verbatim}

The second section is usually HTML, which allows the client software
to display nicely formatted text with header, in-line images, etc.
Here's Python code that prints a simple piece of HTML:

\begin{verbatim}
	print "<TITLE>CGI script output</TITLE>"
	print "<H1>This is my first CGI script</H1>"
	print "Hello, world!"
\end{verbatim}

(It may not be fully legal HTML according to the letter of the
standard, but any browser will understand it.)

\subsection{Using the cgi module}
\nodename{Using the cgi module}

Begin by writing \code{import cgi}.  Don't use \code{from cgi import *} -- the
module defines all sorts of names for its own use or for backward 
compatibility that you don't want in your namespace.

It's best to use the \code{FieldStorage} class.  The other classes define in this 
module are provided mostly for backward compatibility.  Instantiate it 
exactly once, without arguments.  This reads the form contents from 
standard input or the environment (depending on the value of various 
environment variables set according to the CGI standard).  Since it may 
consume standard input, it should be instantiated only once.

The \code{FieldStorage} instance can be accessed as if it were a Python 
dictionary.  For instance, the following code (which assumes that the 
\code{Content-type} header and blank line have already been printed) checks that 
the fields \code{name} and \code{addr} are both set to a non-empty string:

\begin{verbatim}
	form = cgi.FieldStorage()
	form_ok = 0
	if form.has_key("name") and form.has_key("addr"):
		if form["name"].value != "" and form["addr"].value != "":
			form_ok = 1
	if not form_ok:
		print "<H1>Error</H1>"
		print "Please fill in the name and addr fields."
		return
	...further form processing here...
\end{verbatim}

Here the fields, accessed through \code{form[key]}, are themselves instances
of \code{FieldStorage} (or \code{MiniFieldStorage}, depending on the form encoding).

If the submitted form data contains more than one field with the same
name, the object retrieved by \code{form[key]} is not a \code{(Mini)FieldStorage}
instance but a list of such instances.  If you expect this possibility
(i.e., when your HTML form comtains multiple fields with the same
name), use the \code{type()} function to determine whether you have a single
instance or a list of instances.  For example, here's code that
concatenates any number of username fields, separated by commas:

\begin{verbatim}
	username = form["username"]
	if type(username) is type([]):
		# Multiple username fields specified
		usernames = ""
		for item in username:
			if usernames:
				# Next item -- insert comma
				usernames = usernames + "," + item.value
			else:
				# First item -- don't insert comma
				usernames = item.value
	else:
		# Single username field specified
		usernames = username.value
\end{verbatim}

If a field represents an uploaded file, the value attribute reads the 
entire file in memory as a string.  This may not be what you want.  You can 
test for an uploaded file by testing either the filename attribute or the 
file attribute.  You can then read the data at leasure from the file 
attribute:

\begin{verbatim}
	fileitem = form["userfile"]
	if fileitem.file:
		# It's an uploaded file; count lines
		linecount = 0
		while 1:
			line = fileitem.file.readline()
			if not line: break
			linecount = linecount + 1
\end{verbatim}

The file upload draft standard entertains the possibility of uploading
multiple files from one field (using a recursive \code{multipart/*}
encoding).  When this occurs, the item will be a dictionary-like
FieldStorage item.  This can be determined by testing its type
attribute, which should have the value \code{multipart/form-data} (or
perhaps another string beginning with \code{multipart/}  It this case, it
can be iterated over recursively just like the top-level form object.

When a form is submitted in the ``old'' format (as the query string or as a 
single data part of type \code{application/x-www-form-urlencoded}), the items 
will actually be instances of the class \code{MiniFieldStorage}.  In this case,
the list, file and filename attributes are always \code{None}.


\subsection{Old classes}

These classes, present in earlier versions of the \code{cgi} module, are still 
supported for backward compatibility.  New applications should use the
FieldStorage class.

\code{SvFormContentDict}: single value form content as dictionary; assumes each 
field name occurs in the form only once.

\code{FormContentDict}: multiple value form content as dictionary (the form
items are lists of values).  Useful if your form contains multiple
fields with the same name.

Other classes (\code{FormContent}, \code{InterpFormContentDict}) are present for
backwards compatibility with really old applications only.  If you still 
use these and would be inconvenienced when they disappeared from a next 
version of this module, drop me a note.


\subsection{Functions}

These are useful if you want more control, or if you want to employ
some of the algorithms implemented in this module in other
circumstances.

\begin{funcdesc}{parse}{fp}: Parse a query in the environment or from a file (default \code{sys.stdin}).
\end{funcdesc}

\begin{funcdesc}{parse_qs}{qs}: parse a query string given as a string argument (data of type 
\code{application/x-www-form-urlencoded}).
\end{funcdesc}

\begin{funcdesc}{parse_multipart}{fp\, pdict}: parse input of type \code{multipart/form-data} (for 
file uploads).  Arguments are \code{fp} for the input file and 
    \code{pdict} for the dictionary containing other parameters of \code{content-type} header

    Returns a dictionary just like \code{parse_qs()}: keys are the field names, each 
    value is a list of values for that field.  This is easy to use but not 
    much good if you are expecting megabytes to be uploaded -- in that case, 
    use the \code{FieldStorage} class instead which is much more flexible.  Note 
    that \code{content-type} is the raw, unparsed contents of the \code{content-type} 
    header.

    Note that this does not parse nested multipart parts -- use \code{FieldStorage} for 
    that.
\end{funcdesc}

\begin{funcdesc}{parse_header}{string}: parse a header like \code{Content-type} into a main
content-type and a dictionary of parameters.
\end{funcdesc}

\begin{funcdesc}{test}{}: robust test CGI script, usable as main program.
    Writes minimal HTTP headers and formats all information provided to
    the script in HTML form.
\end{funcdesc}

\begin{funcdesc}{print_environ}{}: format the shell environment in HTML.
\end{funcdesc}

\begin{funcdesc}{print_form}{form}: format a form in HTML.
\end{funcdesc}

\begin{funcdesc}{print_directory}{}: format the current directory in HTML.
\end{funcdesc}

\begin{funcdesc}{print_environ_usage}{}: print a list of useful (used by CGI) environment variables in
HTML.
\end{funcdesc}

\begin{funcdesc}{escape}{}: convert the characters ``\code{\&}'', ``\code{<}'' and ``\code{>}'' to HTML-safe
sequences.  Use this if you need to display text that might contain
such characters in HTML.  To translate URLs for inclusion in the HREF
attribute of an \code{<A>} tag, use \code{urllib.quote()}.
\end{funcdesc}


\subsection{Caring about security}

There's one important rule: if you invoke an external program (e.g.
via the \code{os.system()} or \code{os.popen()} functions), make very sure you don't
pass arbitrary strings received from the client to the shell.  This is
a well-known security hole whereby clever hackers anywhere on the web
can exploit a gullible CGI script to invoke arbitrary shell commands.
Even parts of the URL or field names cannot be trusted, since the
request doesn't have to come from your form!

To be on the safe side, if you must pass a string gotten from a form
to a shell command, you should make sure the string contains only
alphanumeric characters, dashes, underscores, and periods.


\subsection{Installing your CGI script on a Unix system}

Read the documentation for your HTTP server and check with your local
system administrator to find the directory where CGI scripts should be
installed; usually this is in a directory \code{cgi-bin} in the server tree.

Make sure that your script is readable and executable by ``others''; the
Unix file mode should be 755 (use \code{chmod 755 filename}).  Make sure
that the first line of the script contains \code{\#!} starting in column 1
followed by the pathname of the Python interpreter, for instance:

\begin{verbatim}
	#!/usr/local/bin/python
\end{verbatim}

Make sure the Python interpreter exists and is executable by ``others''.

Make sure that any files your script needs to read or write are
readable or writable, respectively, by ``others'' -- their mode should
be 644 for readable and 666 for writable.  This is because, for
security reasons, the HTTP server executes your script as user
``nobody'', without any special privileges.  It can only read (write,
execute) files that everybody can read (write, execute).  The current
directory at execution time is also different (it is usually the
server's cgi-bin directory) and the set of environment variables is
also different from what you get at login.  in particular, don't count
on the shell's search path for executables (\code{\$PATH}) or the Python
module search path (\code{\$PYTHONPATH}) to be set to anything interesting.

If you need to load modules from a directory which is not on Python's
default module search path, you can change the path in your script,
before importing other modules, e.g.:

\begin{verbatim}
	import sys
	sys.path.insert(0, "/usr/home/joe/lib/python")
	sys.path.insert(0, "/usr/local/lib/python")
\end{verbatim}

(This way, the directory inserted last will be searched first!)

Instructions for non-Unix systems will vary; check your HTTP server's
documentation (it will usually have a section on CGI scripts).


\subsection{Testing your CGI script}

Unfortunately, a CGI script will generally not run when you try it
from the command line, and a script that works perfectly from the
command line may fail mysteriously when run from the server.  There's
one reason why you should still test your script from the command
line: if it contains a syntax error, the python interpreter won't
execute it at all, and the HTTP server will most likely send a cryptic
error to the client.

Assuming your script has no syntax errors, yet it does not work, you
have no choice but to read the next section:


\subsection{Debugging CGI scripts}

First of all, check for trivial installation errors -- reading the
section above on installing your CGI script carefully can save you a
lot of time.  If you wonder whether you have understood the
installation procedure correctly, try installing a copy of this module
file (\code{cgi.py}) as a CGI script.  When invoked as a script, the file
will dump its environment and the contents of the form in HTML form.
Give it the right mode etc, and send it a request.  If it's installed
in the standard \code{cgi-bin} directory, it should be possible to send it a
request by entering a URL into your browser of the form:

\begin{verbatim}
	http://yourhostname/cgi-bin/cgi.py?name=Joe+Blow&addr=At+Home
\end{verbatim}

If this gives an error of type 404, the server cannot find the script
-- perhaps you need to install it in a different directory.  If it
gives another error (e.g.  500), there's an installation problem that
you should fix before trying to go any further.  If you get a nicely
formatted listing of the environment and form content (in this
example, the fields should be listed as ``addr'' with value ``At Home''
and ``name'' with value ``Joe Blow''), the \code{cgi.py} script has been
installed correctly.  If you follow the same procedure for your own
script, you should now be able to debug it.

The next step could be to call the \code{cgi} module's test() function from
your script: replace its main code with the single statement

\begin{verbatim}
	cgi.test()
\end{verbatim}
	
This should produce the same results as those gotten from installing
the \code{cgi.py} file itself.

When an ordinary Python script raises an unhandled exception
(e.g. because of a typo in a module name, a file that can't be opened,
etc.), the Python interpreter prints a nice traceback and exits.
While the Python interpreter will still do this when your CGI script
raises an exception, most likely the traceback will end up in one of
the HTTP server's log file, or be discarded altogether.

Fortunately, once you have managed to get your script to execute
*some* code, it is easy to catch exceptions and cause a traceback to
be printed.  The \code{test()} function below in this module is an example.
Here are the rules:

\begin{enumerate}
	\item Import the traceback module (before entering the
	   try-except!)
	
	\item Make sure you finish printing the headers and the blank
	   line early
	
	\item Assign \code{sys.stderr} to \code{sys.stdout}
	
	\item Wrap all remaining code in a try-except statement
	
	\item In the except clause, call \code{traceback.print_exc()}
\end{enumerate}

For example:

\begin{verbatim}
	import sys
	import traceback
	print "Content-type: text/html"
	print
	sys.stderr = sys.stdout
	try:
		...your code here...
	except:
		print "\n\n<PRE>"
		traceback.print_exc()
\end{verbatim}

Notes: The assignment to \code{sys.stderr} is needed because the traceback
prints to \code{sys.stderr}.  The \code{print "$\backslash$n$\backslash$n<PRE>"} statement is necessary to
disable the word wrapping in HTML.

If you suspect that there may be a problem in importing the traceback
module, you can use an even more robust approach (which only uses
built-in modules):

\begin{verbatim}
	import sys
	sys.stderr = sys.stdout
	print "Content-type: text/plain"
	print
	...your code here...
\end{verbatim}

This relies on the Python interpreter to print the traceback.  The
content type of the output is set to plain text, which disables all
HTML processing.  If your script works, the raw HTML will be displayed
by your client.  If it raises an exception, most likely after the
first two lines have been printed, a traceback will be displayed.
Because no HTML interpretation is going on, the traceback will
readable.


\subsection{Common problems and solutions}

\begin{itemize}
\item Most HTTP servers buffer the output from CGI scripts until the
script is completed.  This means that it is not possible to display a
progress report on the client's display while the script is running.

\item Check the installation instructions above.

\item Check the HTTP server's log files.  (\code{tail -f logfile} in a separate
window may be useful!)

\item Always check a script for syntax errors first, by doing something
like \code{python script.py}.

\item When using any of the debugging techniques, don't forget to add
\code{import sys} to the top of the script.

\item When invoking external programs, make sure they can be found.
Usually, this means using absolute path names -- \code{\$PATH} is usually not
set to a very useful value in a CGI script.

\item When reading or writing external files, make sure they can be read
or written by every user on the system.

\item Don't try to give a CGI script a set-uid mode.  This doesn't work on
most systems, and is a security liability as well.
\end{itemize}


\section{Standard Module \sectcode{urllib}}
\stmodindex{urllib}
\index{WWW}
\index{World-Wide Web}
\index{URL}

\renewcommand{\indexsubitem}{(in module urllib)}

This module provides a high-level interface for fetching data across
the World-Wide Web.  In particular, the \code{urlopen} function is
similar to the built-in function \code{open}, but accepts URLs
(Universal Resource Locators) instead of filenames.  Some restrictions
apply --- it can only open URLs for reading, and no seek operations
are available.

it defines the following public functions:

\begin{funcdesc}{urlopen}{url}
Open a network object denoted by a URL for reading.  If the URL does
not have a scheme identifier, or if it has \samp{file:} as its scheme
identifier, this opens a local file; otherwise it opens a socket to a
server somewhere on the network.  If the connection cannot be made, or
if the server returns an error code, the \code{IOError} exception is
raised.  If all went well, a file-like object is returned.  This
supports the following methods: \code{read()}, \code{readline()},
\code{readlines()}, \code{fileno()}, \code{close()} and \code{info()}.
Except for the last one, these methods have the same interface as for
file objects --- see the section on File Objects earlier in this
manual.  (It's not a built-in file object, however, so it can't be
used at those few places where a true built-in file object is
required.)

The \code{info()} method returns an instance of the class
\code{rfc822.Message} containing the headers received from the server,
if the protocol uses such headers (currently the only supported
protocol that uses this is HTTP).  See the description of the
\code{rfc822} module.
\end{funcdesc}

\begin{funcdesc}{urlretrieve}{url}
Copy a network object denoted by a URL to a local file, if necessary.
If the URL points to a local file, or a valid cached copy of the
object exists, the object is not copied.  Return a tuple (\var{filename},
\var{headers}) where \var{filename} is the local file name under which
the object can be found, and \var{headers} is either \code{None} (for
a local object) or whatever the \code{info()} method of the object
returned by \code{urlopen()} returned (for a remote object, possibly
cached).  Exceptions are the same as for \code{urlopen()}.
\end{funcdesc}

\begin{funcdesc}{urlcleanup}{}
Clear the cache that may have been built up by previous calls to
\code{urlretrieve()}.
\end{funcdesc}

\begin{funcdesc}{quote}{string\optional{\, addsafe}}
Replace special characters in \var{string} using the \code{\%xx} escape.
Letters, digits, and the characters ``\code{_,.-}'' are never quoted.
The optional \var{addsafe} parameter specifies additional characters
that should not be quoted --- its default value is \code{'/'}.

Example: \code{quote('/\~conolly/')} yields \code{'/\%7econnolly/'}.
\end{funcdesc}

\begin{funcdesc}{unquote}{string}
Replace \samp{\%xx} escapes by their single-character equivalent.

Example: \code{unquote('/\%7Econnolly/')} yields \code{'/\~connolly/'}.
\end{funcdesc}

Restrictions:

\begin{itemize}

\item
Currently, only the following protocols are supported: HTTP, (versions
0.9 and 1.0), Gopher (but not Gopher-+), FTP, and local files.
\index{HTTP}
\index{Gopher}
\index{FTP}

\item
The caching feature of \code{urlretrieve()} has been disabled until I
find the time to hack proper processing of Expiration time headers.

\item
There should be a function to query whether a particular URL is in
the cache.

\item
For backward compatibility, if a URL appears to point to a local file
but the file can't be opened, the URL is re-interpreted using the FTP
protocol.  This can sometimes cause confusing error messages.

\item
The \code{urlopen()} and \code{urlretrieve()} functions can cause
arbitrarily long delays while waiting for a network connection to be
set up.  This means that it is difficult to build an interactive
web client using these functions without using threads.

\item
The data returned by \code{urlopen()} or \code{urlretrieve()} is the
raw data returned by the server.  This may be binary data (e.g. an
image), plain text or (for example) HTML.  The HTTP protocol provides
type information in the reply header, which can be inspected by
looking at the \code{Content-type} header.  For the Gopher protocol,
type information is encoded in the URL; there is currently no easy way
to extract it.  If the returned data is HTML, you can use the module
\code{htmllib} to parse it.
\index{HTML}
\index{HTTP}
\index{Gopher}
\stmodindex{htmllib}

\item
Although the \code{urllib} module contains (undocumented) routines to
parse and unparse URL strings, the recommended interface for URL
manipulation is in module \code{urlparse}.
\stmodindex{urlparse}

\end{itemize}

\section{Standard Module \sectcode{httplib}}
\stmodindex{httplib}
\index{HTTP}

\renewcommand{\indexsubitem}{(in module httplib)}

This module defines a class which implements the client side of the
HTTP protocol.  It is normally not used directly --- the module
\code{urllib} uses it to handle URLs that use HTTP.
\stmodindex{urllib}

The module defines one class, \code{HTTP}.  An \code{HTTP} instance
represents one transaction with an HTTP server.  It should be
instantiated passing it a host and optional port number.  If no port
number is passed, the port is extracted from the host string if it has
the form \code{host:port}, else the default HTTP port (80) is used.
If no host is passed, no connection is made, and the \code{connect}
method should be used to connect to a server.  For example, the
following calls all create instances that connect to the server at the
same host and port:

\begin{verbatim}
>>> h1 = httplib.HTTP('www.cwi.nl')
>>> h2 = httplib.HTTP('www.cwi.nl:80')
>>> h3 = httplib.HTTP('www.cwi.nl', 80)
\end{verbatim}

Once an \code{HTTP} instance has been connected to an HTTP server, it
should be used as follows:

\begin{enumerate}

\item[1.] Make exactly one call to the \code{putrequest()} method.

\item[2.] Make zero or more calls to the \code{putheader()} method.

\item[3.] Call the \code{endheaders()} method (this can be omitted if
step 4 makes no calls).

\item[4.] Optional calls to the \code{send()} method.

\item[5.] Call the \code{getreply()} method.

\item[6.] Call the \code{getfile()} method and read the data off the
file object that it returns.

\end{enumerate}

\subsection{HTTP Objects}

\code{HTTP} instances have the following methods:

\renewcommand{\indexsubitem}{(HTTP method)}

\begin{funcdesc}{set_debuglevel}{level}
Set the debugging level (the amount of debugging output printed).
The default debug level is \code{0}, meaning no debugging output is
printed.
\end{funcdesc}

\begin{funcdesc}{connect}{host\optional{\, port}}
Connect to the server given by \var{host} and \var{port}.  See the
intro for the default port.  This should be called directly only if
the instance was instantiated without passing a host.
\end{funcdesc}

\begin{funcdesc}{send}{data}
Send data to the server.  This should be used directly only after the
\code{endheaders()} method has been called and before
\code{getreply()} has been called.
\end{funcdesc}

\begin{funcdesc}{putrequest}{request\, selector}
This should be the first call after the connection to the server has
been made.  It sends a line to the server consisting of the
\var{request} string, the \var{selector} string, and the HTTP version
(\code{HTTP/1.0}).
\end{funcdesc}

\begin{funcdesc}{putheader}{header\, argument\optional{\, ...}}
Send an RFC-822 style header to the server.  It sends a line to the
server consisting of the header, a colon and a space, and the first
argument.  If more arguments are given, continuation lines are sent,
each consisting of a tab and an argument.
\end{funcdesc}

\begin{funcdesc}{endheaders}{}
Send a blank line to the server, signalling the end of the headers.
\end{funcdesc}

\begin{funcdesc}{getreply}{}
Complete the request by shutting down the sending end of the socket,
read the reply from the server, and return a triple (\var{replycode},
\var{message}, \var{headers}).  Here \var{replycode} is the integer
reply code from the request (e.g.\ \code{200} if the request was
handled properly); \var{message} is the message string corresponding
to the reply code; and \var{header} is an instance of the class
\code{rfc822.Message} containing the headers received from the server.
See the description of the \code{rfc822} module.
\stmodindex{rfc822}
\end{funcdesc}

\begin{funcdesc}{getfile}{}
Return a file object from which the data returned by the server can be
read, using the \code{read()}, \code{readline()} or \code{readlines()}
methods.
\end{funcdesc}

\subsection{Example}
\nodename{HTTP Example}

Here is an example session:

\begin{verbatim}
>>> import httplib
>>> h = httplib.HTTP('www.cwi.nl')
>>> h.putrequest('GET', '/index.html')
>>> h.putheader('Accept', 'text/html')
>>> h.putheader('Accept', 'text/plain')
>>> h.endheaders()
>>> errcode, errmsg, headers = h.getreply()
>>> print errcode # Should be 200
>>> f = h.getfile()
>>> data f.read() # Get the raw HTML
>>> f.close()
>>> 
\end{verbatim}

\section{Standard Module \sectcode{ftplib}}
\stmodindex{ftplib}

\renewcommand{\indexsubitem}{(in module ftplib)}

This module defines the class \code{FTP} and a few related items.  The
\code{FTP} class implements the client side of the FTP protocol.  You
can use this to write Python programs that perform a variety of
automated FTP jobs, such as mirroring other ftp servers.  It is also
used by the module \code{urllib} to handle URLs that use FTP.  For
more information on FTP (File Transfer Protocol), see Internet RFC
959.

Here's a sample session using the \code{ftplib} module:

\begin{verbatim}
>>> from ftplib import FTP
>>> ftp = FTP('ftp.cwi.nl')   # connect to host, default port
>>> ftp.login()               # user anonymous, passwd user@hostname
>>> ftp.retrlines('LIST')     # list directory contents
total 24418
drwxrwsr-x   5 ftp-usr  pdmaint     1536 Mar 20 09:48 .
dr-xr-srwt 105 ftp-usr  pdmaint     1536 Mar 21 14:32 ..
-rw-r--r--   1 ftp-usr  pdmaint     5305 Mar 20 09:48 INDEX
 .
 .
 .
>>> ftp.quit()
\end{verbatim}

The module defines the following items:

\begin{funcdesc}{FTP}{\optional{host\optional{\, user\, passwd\, acct}}}
Return a new instance of the \code{FTP} class.  When
\var{host} is given, the method call \code{connect(\var{host})} is
made.  When \var{user} is given, additionally the method call
\code{login(\var{user}, \var{passwd}, \var{acct})} is made (where
\var{passwd} and \var{acct} default to the empty string when not given).
\end{funcdesc}

\begin{datadesc}{all_errors}
The set of all exceptions (as a tuple) that methods of \code{FTP}
instances may raise as a result of problems with the FTP connection
(as opposed to programming errors made by the caller).  This set
includes the four exceptions listed below as well as
\code{socket.error} and \code{IOError}.
\end{datadesc}

\begin{excdesc}{error_reply}
Exception raised when an unexpected reply is received from the server.
\end{excdesc}

\begin{excdesc}{error_temp}
Exception raised when an error code in the range 400--499 is received.
\end{excdesc}

\begin{excdesc}{error_perm}
Exception raised when an error code in the range 500--599 is received.
\end{excdesc}

\begin{excdesc}{error_proto}
Exception raised when a reply is received from the server that does
not begin with a digit in the range 1--5.
\end{excdesc}

\subsection{FTP Objects}

FTP instances have the following methods:

\renewcommand{\indexsubitem}{(FTP object method)}

\begin{funcdesc}{set_debuglevel}{level}
Set the instance's debugging level.  This controls the amount of
debugging output printed.  The default, 0, produces no debugging
output.  A value of 1 produces a moderate amount of debugging output,
generally a single line per request.  A value of 2 or higher produces
the maximum amount of debugging output, logging each line sent and
received on the control connection.
\end{funcdesc}

\begin{funcdesc}{connect}{host\optional{\, port}}
Connect to the given host and port.  The default port number is 21, as
specified by the FTP protocol specification.  It is rarely needed to
specify a different port number.  This function should be called only
once for each instance; it should not be called at all if a host was
given when the instance was created.  All other methods can only be
used after a connection has been made.
\end{funcdesc}

\begin{funcdesc}{getwelcome}{}
Return the welcome message sent by the server in reply to the initial
connection.  (This message sometimes contains disclaimers or help
information that may be relevant to the user.)
\end{funcdesc}

\begin{funcdesc}{login}{\optional{user\optional{\, passwd\optional{\, acct}}}}
Log in as the given \var{user}.  The \var{passwd} and \var{acct}
parameters are optional and default to the empty string.  If no
\var{user} is specified, it defaults to \samp{anonymous}.  If
\var{user} is \code{anonymous}, the default \var{passwd} is
\samp{\var{realuser}@\var{host}} where \var{realuser} is the real user
name (glanced from the \samp{LOGNAME} or \samp{USER} environment
variable) and \var{host} is the hostname as returned by
\code{socket.gethostname()}.  This function should be called only
once for each instance, after a connection has been established; it
should not be called at all if a host and user were given when the
instance was created.  Most FTP commands are only allowed after the
client has logged in.
\end{funcdesc}

\begin{funcdesc}{abort}{}
Abort a file transfer that is in progress.  Using this does not always
work, but it's worth a try.
\end{funcdesc}

\begin{funcdesc}{sendcmd}{command}
Send a simple command string to the server and return the response
string.
\end{funcdesc}

\begin{funcdesc}{voidcmd}{command}
Send a simple command string to the server and handle the response.
Return nothing if a response code in the range 200--299 is received.
Raise an exception otherwise.
\end{funcdesc}

\begin{funcdesc}{retrbinary}{command\, callback\, maxblocksize}
Retrieve a file in binary transfer mode.  \var{command} should be an
appropriate \samp{RETR} command, i.e.\ \code{"RETR \var{filename}"}.
The \var{callback} function is called for each block of data received,
with a single string argument giving the data block.
The \var{maxblocksize} argument specifies the maximum block size
(which may not be the actual size of the data blocks passed to
\var{callback}).
\end{funcdesc}

\begin{funcdesc}{retrlines}{command\optional{\, callback}}
Retrieve a file or directory listing in \ASCII{} transfer mode.
var{command} should be an appropriate \samp{RETR} command (see
\code{retrbinary()} or a \samp{LIST} command (usually just the string
\code{"LIST"}).  The \var{callback} function is called for each line,
with the trailing CRLF stripped.  The default \var{callback} prints
the line to \code{sys.stdout}.
\end{funcdesc}

\begin{funcdesc}{storbinary}{command\, file\, blocksize}
Store a file in binary transfer mode.  \var{command} should be an
appropriate \samp{STOR} command, i.e.\ \code{"STOR \var{filename}"}.
\var{file} is an open file object which is read until EOF using its
\code{read()} method in blocks of size \var{blocksize} to provide the
data to be stored.
\end{funcdesc}

\begin{funcdesc}{storlines}{command\, file}
Store a file in \ASCII{} transfer mode.  \var{command} should be an
appropriate \samp{STOR} command (see \code{storbinary()}).  Lines are
read until EOF from the open file object \var{file} using its
\code{readline()} method to privide the data to be stored.
\end{funcdesc}

\begin{funcdesc}{nlst}{argument\optional{\, \ldots}}
Return a list of files as returned by the \samp{NLST} command.  The
optional var{argument} is a directory to list (default is the current
server directory).  Multiple arguments can be used to pass
non-standard options to the \samp{NLST} command.
\end{funcdesc}

\begin{funcdesc}{dir}{argument\optional{\, \ldots}}
Return a directory listing as returned by the \samp{LIST} command, as
a list of lines.  The optional var{argument} is a directory to list
(default is the current server directory).  Multiple arguments can be
used to pass non-standard options to the \samp{LIST} command.  If the
last argument is a function, it is used as a \var{callback} function
as for \code{retrlines()}.
\end{funcdesc}

\begin{funcdesc}{rename}{fromname\, toname}
Rename file \var{fromname} on the server to \var{toname}.
\end{funcdesc}

\begin{funcdesc}{cwd}{pathname}
Set the current directory on the server.
\end{funcdesc}

\begin{funcdesc}{mkd}{pathname}
Create a new directory on the server.
\end{funcdesc}

\begin{funcdesc}{pwd}{}
Return the pathname of the current directory on the server.
\end{funcdesc}

\begin{funcdesc}{quit}{}
Send a \samp{QUIT} command to the server and close the connection.
This is the ``polite'' way to close a connection, but it may raise an
exception of the server reponds with an error to the \code{QUIT}
command.
\end{funcdesc}

\begin{funcdesc}{close}{}
Close the connection unilaterally.  This should not be applied to an
already closed connection (e.g.\ after a successful call to
\code{quit()}.
\end{funcdesc}

\section{Standard Module \sectcode{gopherlib}}
\stmodindex{gopherlib}

\renewcommand{\indexsubitem}{(in module gopherlib)}

This module provides a minimal implementation of client side of the
the Gopher protocol.  It is used by the module \code{urllib} to handle
URLs that use the Gopher protocol.

The module defines the following functions:

\begin{funcdesc}{send_selector}{selector\, host\optional{\, port}}
Send a \var{selector} string to the gopher server at \var{host} and
\var{port} (default 70).  Return an open file object from which the
returned document can be read.
\end{funcdesc}

\begin{funcdesc}{send_query}{selector\, query\, host\optional{\, port}}
Send a \var{selector} string and a \var{query} string to a gopher
server at \var{host} and \var{port} (default 70).  Return an open file
object from which the returned document can be read.
\end{funcdesc}

Note that the data returned by the Gopher server can be of any type,
depending on the first character of the selector string.  If the data
is text (first character of the selector is \samp{0}), lines are
terminated by CRLF, and the data is terminated by a line consisting of
a single \samp{.}, and a leading \samp{.} should be stripped from
lines that begin with \samp{..}.  Directory listings (first charactger
of the selector is \samp{1}) are transferred using the same protocol.

\section{Standard Module \sectcode{nntplib}}
\stmodindex{nntplib}

\renewcommand{\indexsubitem}{(in module nntplib)}

This module defines the class \code{NNTP} which implements the client
side of the NNTP protocol.  It can be used to implement a news reader
or poster, or automated news processors.  For more information on NNTP
(Network News Transfer Protocol), see Internet RFC 977.

Here are two small examples of how it can be used.  To list some
statistics about a newsgroup and print the subjects of the last 10
articles:

\small{
\begin{verbatim}
>>> s = NNTP('news.cwi.nl')
>>> resp, count, first, last, name = s.group('comp.lang.python')
>>> print 'Group', name, 'has', count, 'articles, range', first, 'to', last
Group comp.lang.python has 59 articles, range 3742 to 3803
>>> resp, subs = s.xhdr('subject', first + '-' + last)
>>> for id, sub in subs[-10:]: print id, sub
... 
3792 Re: Removing elements from a list while iterating...
3793 Re: Who likes Info files?
3794 Emacs and doc strings
3795 a few questions about the Mac implementation
3796 Re: executable python scripts
3797 Re: executable python scripts
3798 Re: a few questions about the Mac implementation 
3799 Re: PROPOSAL: A Generic Python Object Interface for Python C Modules
3802 Re: executable python scripts 
3803 Re: POSIX wait and SIGCHLD
>>> s.quit()
'205 news.cwi.nl closing connection.  Goodbye.'
>>> 
\end{verbatim}
}

To post an article from a file (this assumes that the article has
valid headers):

\begin{verbatim}
>>> s = NNTP('news.cwi.nl')
>>> f = open('/tmp/article')
>>> s.post(f)
'240 Article posted successfully.'
>>> s.quit()
'205 news.cwi.nl closing connection.  Goodbye.'
>>> 
\end{verbatim}

The module itself defines the following items:

\begin{funcdesc}{NNTP}{host\optional{\, port}}
Return a new instance of the \code{NNTP} class, representing a
connection to the NNTP server running on host \var{host}, listening at
port \var{port}.  The default \var{port} is 119.
\end{funcdesc}

\begin{excdesc}{error_reply}
Exception raised when an unexpected reply is received from the server.
\end{excdesc}

\begin{excdesc}{error_temp}
Exception raised when an error code in the range 400--499 is received.
\end{excdesc}

\begin{excdesc}{error_perm}
Exception raised when an error code in the range 500--599 is received.
\end{excdesc}

\begin{excdesc}{error_proto}
Exception raised when a reply is received from the server that does
not begin with a digit in the range 1--5.
\end{excdesc}

\subsection{NNTP Objects}

NNTP instances have the following methods.  The \var{response} that is
returned as the first item in the return tuple of almost all methods
is the server's response: a string beginning with a three-digit code.
If the server's response indicates an error, the method raises one of
the above exceptions.

\renewcommand{\indexsubitem}{(NNTP object method)}

\begin{funcdesc}{getwelcome}{}
Return the welcome message sent by the server in reply to the initial
connection.  (This message sometimes contains disclaimers or help
information that may be relevant to the user.)
\end{funcdesc}

\begin{funcdesc}{set_debuglevel}{level}
Set the instance's debugging level.  This controls the amount of
debugging output printed.  The default, 0, produces no debugging
output.  A value of 1 produces a moderate amount of debugging output,
generally a single line per request or response.  A value of 2 or
higher produces the maximum amount of debugging output, logging each
line sent and received on the connection (including message text).
\end{funcdesc}

\begin{funcdesc}{newgroups}{date\, time}
Send a \samp{NEWGROUPS} command.  The \var{date} argument should be a
string of the form \code{"\var{yy}\var{mm}\var{dd}"} indicating the
date, and \var{time} should be a string of the form
\code{"\var{hh}\var{mm}\var{ss}"} indicating the time.  Return a pair
\code{(\var{response}, \var{groups})} where \var{groups} is a list of
group names that are new since the given date and time.
\end{funcdesc}

\begin{funcdesc}{newnews}{group\, date\, time}
Send a \samp{NEWNEWS} command.  Here, \var{group} is a group name or
\code{"*"}, and \var{date} and \var{time} have the same meaning as for
\code{newgroups()}.  Return a pair \code{(\var{response},
\var{articles})} where \var{articles} is a list of article ids.
\end{funcdesc}

\begin{funcdesc}{list}{}
Send a \samp{LIST} command.  Return a pair \code{(\var{response},
\var{list})} where \var{list} is a list of tuples.  Each tuple has the
form \code{(\var{group}, \var{last}, \var{first}, \var{flag})}, where
\var{group} is a group name, \var{last} and \var{first} are the last
and first article numbers (as strings), and \var{flag} is \code{'y'}
if posting is allowed, \code{'n'} if not, and \code{'m'} if the
newsgroup is moderated.  (Note the ordering: \var{last}, \var{first}.)
\end{funcdesc}

\begin{funcdesc}{group}{name}
Send a \samp{GROUP} command, where \var{name} is the group name.
Return a tuple \code{(\var{response}, \var{count}, \var{first},
\var{last}, \var{name})} where \var{count} is the (estimated) number
of articles in the group, \var{first} is the first article number in
the group, \var{last} is the last article number in the group, and
\var{name} is the group name.  The numbers are returned as strings.
\end{funcdesc}

\begin{funcdesc}{help}{}
Send a \samp{HELP} command.  Return a pair \code{(\var{response},
\var{list})} where \var{list} is a list of help strings.
\end{funcdesc}

\begin{funcdesc}{stat}{id}
Send a \samp{STAT} command, where \var{id} is the message id (enclosed
in \samp{<} and \samp{>}) or an article number (as a string).
Return a triple \code{(var{response}, \var{number}, \var{id})} where
\var{number} is the article number (as a string) and \var{id} is the
article id  (enclosed in \samp{<} and \samp{>}).
\end{funcdesc}

\begin{funcdesc}{next}{}
Send a \samp{NEXT} command.  Return as for \code{stat()}.
\end{funcdesc}

\begin{funcdesc}{last}{}
Send a \samp{LAST} command.  Return as for \code{stat()}.
\end{funcdesc}

\begin{funcdesc}{head}{id}
Send a \samp{HEAD} command, where \var{id} has the same meaning as for
\code{stat()}.  Return a pair \code{(\var{response}, \var{list})}
where \var{list} is a list of the article's headers (an uninterpreted
list of lines, without trailing newlines).
\end{funcdesc}

\begin{funcdesc}{body}{id}
Send a \samp{BODY} command, where \var{id} has the same meaning as for
\code{stat()}.  Return a pair \code{(\var{response}, \var{list})}
where \var{list} is a list of the article's body text (an
uninterpreted list of lines, without trailing newlines).
\end{funcdesc}

\begin{funcdesc}{article}{id}
Send a \samp{ARTICLE} command, where \var{id} has the same meaning as
for \code{stat()}.  Return a pair \code{(\var{response}, \var{list})}
where \var{list} is a list of the article's header and body text (an
uninterpreted list of lines, without trailing newlines).
\end{funcdesc}

\begin{funcdesc}{slave}{}
Send a \samp{SLAVE} command.  Return the server's \var{response}.
\end{funcdesc}

\begin{funcdesc}{xhdr}{header\, string}
Send an \samp{XHDR} command.  This command is not defined in the RFC
but is a common extension.  The \var{header} argument is a header
keyword, e.g. \code{"subject"}.  The \var{string} argument should have
the form \code{"\var{first}-\var{last}"} where \var{first} and
\var{last} are the first and last article numbers to search.  Return a
pair \code{(\var{response}, \var{list})}, where \var{list} is a list of
pairs \code{(\var{id}, \var{text})}, where \var{id} is an article id
(as a string) and \var{text} is the text of the requested header for
that article.
\end{funcdesc}

\begin{funcdesc}{post}{file}
Post an article using the \samp{POST} command.  The \var{file}
argument is an open file object which is read until EOF using its
\code{readline()} method.  It should be a well-formed news article,
including the required headers.  The \code{post()} method
automatically escapes lines beginning with \samp{.}.
\end{funcdesc}

\begin{funcdesc}{ihave}{id\, file}
Send an \samp{IHAVE} command.  If the response is not an error, treat
\var{file} exactly as for the \code{post()} method.
\end{funcdesc}

\begin{funcdesc}{quit}{}
Send a \samp{QUIT} command and close the connection.  Once this method
has been called, no other methods of the NNTP object should be called.
\end{funcdesc}

\section{Standard Module \sectcode{urlparse}}
\stmodindex{urlparse}
\index{WWW}
\index{World-Wide Web}
\index{URL}
\indexii{URL}{parsing}
\indexii{relative}{URL}

\renewcommand{\indexsubitem}{(in module urlparse)}

This module defines a standard interface to break URL strings up in
components (addessing scheme, network location, path etc.), to combine
the components back into a URL string, and to convert a ``relative
URL'' to an absolute URL given a ``base URL''.

The module has been designed to match the current Internet draft on
Relative Uniform Resource Locators (and discovered a bug in an earlier
draft!).

It defines the following functions:

\begin{funcdesc}{urlparse}{urlstring\optional{\,
default_scheme\optional{\, allow_fragments}}}
Parse a URL into 6 components, returning a 6-tuple: (addressing
scheme, network location, path, parameters, query, fragment
identifier).  This corresponds to the general structure of a URL:
\code{\var{scheme}://\var{netloc}/\var{path};\var{parameters}?\var{query}\#\var{fragment}}.
Each tuple item is a string, possibly empty.
The components are not broken up in smaller parts (e.g. the network
location is a single string), and \% escapes are not expanded.
The delimiters as shown above are not part of the tuple items,
except for a leading slash in the \var{path} component, which is
retained if present.

Example:

\begin{verbatim}
urlparse('http://www.cwi.nl:80/%7Eguido/Python.html')
\end{verbatim}

yields the tuple

\begin{verbatim}
('http', 'www.cwi.nl:80', '/%7Eguido/Python.html', '', '', '')
\end{verbatim}

If the \var{default_scheme} argument is specified, it gives the
default addressing scheme, to be used only if the URL string does not
specify one.  The default value for this argument is the empty string.

If the \var{allow_fragments} argument is zero, fragment identifiers
are not allowed, even if the URL's addressing scheme normally does
support them.  The default value for this argument is \code{1}.
\end{funcdesc}

\begin{funcdesc}{urlunparse}{tuple}
Construct a URL string from a tuple as returned by \code{urlparse}.
This may result in a slightly different, but equivalent URL, if the
URL that was parsed originally had redundant delimiters, e.g. a ? with
an empty query (the draft states that these are equivalent).
\end{funcdesc}

\begin{funcdesc}{urljoin}{base\, url\optional{\, allow_fragments}}
Construct a full (``absolute'') URL by combining a ``base URL''
(\var{base}) with a ``relative URL'' (\var{url}).  Informally, this
uses components of the base URL, in particular the addressing scheme,
the network location and (part of) the path, to provide missing
components in the relative URL.

Example:

\begin{verbatim}
urljoin('http://www.cwi.nl/%7Eguido/Python.html', 'FAQ.html')
\end{verbatim}

yields the string

\begin{verbatim}
'http://www.cwi.nl/%7Eguido/FAQ.html'
\end{verbatim}

The \var{allow_fragments} argument has the same meaning as for
\code{urlparse}.
\end{funcdesc}

\section{Standard Module \sectcode{sgmllib}}
\stmodindex{sgmllib}
\index{SGML}

This module defines a class \code{SGMLParser} which serves as the
basis for parsing text files formatted in SGML (Standard Generalized
Mark-up Language).  In fact, it does not provide a full SGML parser
--- it only parses SGML insofar as it is used by HTML, and the module
only exists as a base for the \code{htmllib} module.
\stmodindex{htmllib}

In particular, the parser is hardcoded to recognize the following
constructs:

\begin{itemize}

\item
Opening and closing tags of the form
``\code{<\var{tag} \var{attr}="\var{value}" ...>}'' and
``\code{</\var{tag}>}'', respectively.

\item
Numeric character references of the form ``\code{\&\#\var{name};}''.

\item
Entity references of the form ``\code{\&\var{name};}''.

\item
SGML comments of the form ``\code{<!--\var{text}-->}''.  Note that
spaces, tabs, and newlines are allowed between the trailing
``\code{>}'' and the immediately preceeding ``\code{--}''.

\end{itemize}

The \code{SGMLParser} class must be instantiated without arguments.
It has the following interface methods:

\renewcommand{\indexsubitem}{({\tt SGMLParser} method)}

\begin{funcdesc}{reset}{}
Reset the instance.  Loses all unprocessed data.  This is called
implicitly at instantiation time.
\end{funcdesc}

\begin{funcdesc}{setnomoretags}{}
Stop processing tags.  Treat all following input as literal input
(CDATA).  (This is only provided so the HTML tag \code{<PLAINTEXT>}
can be implemented.)
\end{funcdesc}

\begin{funcdesc}{setliteral}{}
Enter literal mode (CDATA mode).
\end{funcdesc}

\begin{funcdesc}{feed}{data}
Feed some text to the parser.  It is processed insofar as it consists
of complete elements; incomplete data is buffered until more data is
fed or \code{close()} is called.
\end{funcdesc}

\begin{funcdesc}{close}{}
Force processing of all buffered data as if it were followed by an
end-of-file mark.  This method may be redefined by a derived class to
define additional processing at the end of the input, but the
redefined version should always call \code{SGMLParser.close()}.
\end{funcdesc}

\begin{funcdesc}{handle_starttag}{tag\, method\, attributes}
This method is called to handle start tags for which either a
\code{start_\var{tag}()} or \code{do_\var{tag}()} method has been
defined.  The \code{tag} argument is the name of the tag converted to
lower case, and the \code{method} argument is the bound method which
should be used to support semantic interpretation of the start tag.
The \var{attributes} argument is a list of (\var{name}, \var{value})
pairs containing the attributes found inside the tag's \code{<>}
brackets.  The \var{name} has been translated to lower case and double
quotes and backslashes in the \var{value} have been interpreted.  For
instance, for the tag \code{<A HREF="http://www.cwi.nl/">}, this
method would be called as \code{unknown_starttag('a', [('href',
'http://www.cwi.nl/')])}.  The base implementation simply calls
\code{method} with \code{attributes} as the only argument.
\end{funcdesc}

\begin{funcdesc}{handle_endtag}{tag\, method}

This method is called to handle endtags for which an
\code{end_\var{tag}()} method has been defined.  The \code{tag}
argument is the name of the tag converted to lower case, and the
\code{method} argument is the bound method which should be used to
support semantic interpretation of the end tag.  If no
\code{end_\var{tag}()} method is defined for the closing element, this
handler is not called.  The base implementation simply calls
\code{method}.
\end{funcdesc}

\begin{funcdesc}{handle_data}{data}
This method is called to process arbitrary data.  It is intended to be
overridden by a derived class; the base class implementation does
nothing.
\end{funcdesc}

\begin{funcdesc}{handle_charref}{ref}
This method is called to process a character reference of the form
``\code{\&\#\var{ref};}''.  In the base implementation, \var{ref} must
be a decimal number in the
range 0-255.  It translates the character to \ASCII{} and calls the
method \code{handle_data()} with the character as argument.  If
\var{ref} is invalid or out of range, the method
\code{unknown_charref(\var{ref})} is called to handle the error.  A
subclass must override this method to provide support for named
character entities.
\end{funcdesc}

\begin{funcdesc}{handle_entityref}{ref}
This method is called to process a general entity reference of the form
``\code{\&\var{ref};}'' where \var{ref} is an general entity
reference.  It looks for \var{ref} in the instance (or class)
variable \code{entitydefs} which should be a mapping from entity names
to corresponding translations.
If a translation is found, it calls the method \code{handle_data()}
with the translation; otherwise, it calls the method
\code{unknown_entityref(\var{ref})}.  The default \code{entitydefs}
defines translations for \code{\&amp;}, \code{\&apos}, \code{\&gt;},
\code{\&lt;}, and \code{\&quot;}.
\end{funcdesc}

\begin{funcdesc}{handle_comment}{comment}
This method is called when a comment is encountered.  The
\code{comment} argument is a string containing the text between the
``\code{<!--}'' and ``\code{-->}'' delimiters, but not the delimiters
themselves.  For example, the comment ``\code{<!--text-->}'' will
cause this method to be called with the argument \code{'text'}.  The
default method does nothing.
\end{funcdesc}

\begin{funcdesc}{report_unbalanced}{tag}
This method is called when an end tag is found which does not
correspond to any open element.
\end{funcdesc}

\begin{funcdesc}{unknown_starttag}{tag\, attributes}
This method is called to process an unknown start tag.  It is intended
to be overridden by a derived class; the base class implementation
does nothing.
\end{funcdesc}

\begin{funcdesc}{unknown_endtag}{tag}
This method is called to process an unknown end tag.  It is intended
to be overridden by a derived class; the base class implementation
does nothing.
\end{funcdesc}

\begin{funcdesc}{unknown_charref}{ref}
This method is called to process unresolvable numeric character
references.  It is intended to be overridden by a derived class; the
base class implementation does nothing.
\end{funcdesc}

\begin{funcdesc}{unknown_entityref}{ref}
This method is called to process an unknown entity reference.  It is
intended to be overridden by a derived class; the base class
implementation does nothing.
\end{funcdesc}

Apart from overriding or extending the methods listed above, derived
classes may also define methods of the following form to define
processing of specific tags.  Tag names in the input stream are case
independent; the \var{tag} occurring in method names must be in lower
case:

\begin{funcdesc}{start_\var{tag}}{attributes}
This method is called to process an opening tag \var{tag}.  It has
preference over \code{do_\var{tag}()}.  The \var{attributes} argument
has the same meaning as described for \code{handle_starttag()} above.
\end{funcdesc}

\begin{funcdesc}{do_\var{tag}}{attributes}
This method is called to process an opening tag \var{tag} that does
not come with a matching closing tag.  The \var{attributes} argument
has the same meaning as described for \code{handle_starttag()} above.
\end{funcdesc}

\begin{funcdesc}{end_\var{tag}}{}
This method is called to process a closing tag \var{tag}.
\end{funcdesc}

Note that the parser maintains a stack of open elements for which no
end tag has been found yet.  Only tags processed by
\code{start_\var{tag}()} are pushed on this stack.  Definition of an
\code{end_\var{tag}()} method is optional for these tags.  For tags
processed by \code{do_\var{tag}()} or by \code{unknown_tag()}, no
\code{end_\var{tag}()} method must be defined; if defined, it will not
be used.  If both \code{start_\var{tag}()} and \code{do_\var{tag}()}
methods exist for a tag, the \code{start_\var{tag}()} method takes
precedence.

\section{Standard Module \sectcode{htmllib}}
\stmodindex{htmllib}
\index{HTML}
\index{hypertext}

\renewcommand{\indexsubitem}{(in module htmllib)}

This module defines a class which can serve as a base for parsing text
files formatted in the HyperText Mark-up Language (HTML).  The class
is not directly concerned with I/O --- it must be provided with input
in string form via a method, and makes calls to methods of a
``formatter'' object in order to produce output.  The
\code{HTMLParser} class is designed to be used as a base class for
other classes in order to add functionality, and allows most of its
methods to be extended or overridden.  In turn, this class is derived
from and extends the \code{SGMLParser} class defined in module
\code{sgmllib}.  Two implementations of formatter objects are
provided in the \code{formatter} module; refer to the documentation
for that module for information on the formatter interface.
\index{SGML}
\stmodindex{sgmllib}
\ttindex{SGMLParser}
\index{formatter}
\stmodindex{formatter}

The following is a summary of the interface defined by
\code{sgmllib.SGMLParser}:

\begin{itemize}

\item
The interface to feed data to an instance is through the \code{feed()}
method, which takes a string argument.  This can be called with as
little or as much text at a time as desired; \code{p.feed(a);
p.feed(b)} has the same effect as \code{p.feed(a+b)}.  When the data
contains complete HTML tags, these are processed immediately;
incomplete elements are saved in a buffer.  To force processing of all
unprocessed data, call the \code{close()} method.

For example, to parse the entire contents of a file, use:
\begin{verbatim}
parser.feed(open('myfile.html').read())
parser.close()
\end{verbatim}

\item
The interface to define semantics for HTML tags is very simple: derive
a class and define methods called \code{start_\var{tag}()},
\code{end_\var{tag}()}, or \code{do_\var{tag}()}.  The parser will
call these at appropriate moments: \code{start_\var{tag}} or
\code{do_\var{tag}} is called when an opening tag of the form
\code{<\var{tag} ...>} is encountered; \code{end_\var{tag}} is called
when a closing tag of the form \code{<\var{tag}>} is encountered.  If
an opening tag requires a corresponding closing tag, like \code{<H1>}
... \code{</H1>}, the class should define the \code{start_\var{tag}}
method; if a tag requires no closing tag, like \code{<P>}, the class
should define the \code{do_\var{tag}} method.

\end{itemize}

The module defines a single class:

\begin{funcdesc}{HTMLParser}{formatter}
This is the basic HTML parser class.  It supports all entity names
required by the HTML 2.0 specification (RFC 1866).  It also defines
handlers for all HTML 2.0 and many HTML 3.0 and 3.2 elements.
\end{funcdesc}

In addition to tag methods, the \code{HTMLParser} class provides some
additional methods and instance variables for use within tag methods.

\renewcommand{\indexsubitem}{({\tt HTMLParser} method)}

\begin{datadesc}{formatter}
This is the formatter instance associated with the parser.
\end{datadesc}

\begin{datadesc}{nofill}
Boolean flag which should be true when whitespace should not be
collapsed, or false when it should be.  In general, this should only
be true when character data is to be treated as ``preformatted'' text,
as within a \code{<PRE>} element.  The default value is false.  This
affects the operation of \code{handle_data()} and \code{save_end()}.
\end{datadesc}

\begin{funcdesc}{anchor_bgn}{href\, name\, type}
This method is called at the start of an anchor region.  The arguments
correspond to the attributes of the \code{<A>} tag with the same
names.  The default implementation maintains a list of hyperlinks
(defined by the \code{href} argument) within the document.  The list
of hyperlinks is available as the data attribute \code{anchorlist}.
\end{funcdesc}

\begin{funcdesc}{anchor_end}{}
This method is called at the end of an anchor region.  The default
implementation adds a textual footnote marker using an index into the
list of hyperlinks created by \code{anchor_bgn()}.
\end{funcdesc}

\begin{funcdesc}{handle_image}{source\, alt\optional{\, ismap\optional{\, align\optional{\, width\optional{\, height}}}}}
This method is called to handle images.  The default implementation
simply passes the \code{alt} value to the \code{handle_data()}
method.
\end{funcdesc}

\begin{funcdesc}{save_bgn}{}
Begins saving character data in a buffer instead of sending it to the
formatter object.  Retrieve the stored data via \code{save_end()}
Use of the \code{save_bgn()} / \code{save_end()} pair may not be
nested.
\end{funcdesc}

\begin{funcdesc}{save_end}{}
Ends buffering character data and returns all data saved since the
preceeding call to \code{save_bgn()}.  If \code{nofill} flag is false,
whitespace is collapsed to single spaces.  A call to this method
without a preceeding call to \code{save_bgn()} will raise a
\code{TypeError} exception.
\end{funcdesc}

\section{Standard Module \sectcode{formatter}}
\stmodindex{formatter}

\renewcommand{\indexsubitem}{(in module formatter)}

This module supports two interface definitions, each with mulitple
implementations.  The \emph{formatter} interface is used by the
\code{HTMLParser} class of the \code{htmllib} module, and the
\emph{writer} interface is required by the formatter interface.

Formatter objects transform an abstract flow of formatting events into
specific output events on writer objects.  Formatters manage several
stack structures to allow various properties of a writer object to be
changed and restored; writers need not be able to handle relative
changes nor any sort of ``change back'' operation.  Specific writer
properties which may be controlled via formatter objects are
horizontal alignment, font, and left margin indentations.  A mechanism
is provided which supports providing arbitrary, non-exclusive style
settings to a writer as well.  Additional interfaces facilitate
formatting events which are not reversible, such as paragraph
separation.

Writer objects encapsulate device interfaces.  Abstract devices, such
as file formats, are supported as well as physical devices.  The
provided implementations all work with abstract devices.  The
interface makes available mechanisms for setting the properties which
formatter objects manage and inserting data into the output.


\subsection{The Formatter Interface}

Interfaces to create formatters are dependent on the specific
formatter class being instantiated.  The interfaces described below
are the required interfaces which all formatters must support once
initialized.

One data element is defined at the module level:

\begin{datadesc}{AS_IS}
Value which can be used in the font specification passed to the
\code{push_font()} method described below, or as the new value to any
other \code{push_\var{property}()} method.  Pushing the \code{AS_IS}
value allows the corresponding \code{pop_\var{property}()} method to
be called without having to track whether the property was changed.
\end{datadesc}

The following attributes are defined for formatter instance objects:

\renewcommand{\indexsubitem}{(formatter object data)}

\begin{datadesc}{writer}
The writer instance with which the formatter interacts.
\end{datadesc}


\renewcommand{\indexsubitem}{(formatter object method)}

\begin{funcdesc}{end_paragraph}{blanklines}
Close any open paragraphs and insert at least \code{blanklines}
before the next paragraph.
\end{funcdesc}

\begin{funcdesc}{add_line_break}{}
Add a hard line break if one does not already exist.  This does not
break the logical paragraph.
\end{funcdesc}

\begin{funcdesc}{add_hor_rule}{*args\, **kw}
Insert a horizontal rule in the output.  A hard break is inserted if
there is data in the current paragraph, but the logical paragraph is
not broken.  The arguments and keywords are passed on to the writer's
\code{send_line_break()} method.
\end{funcdesc}

\begin{funcdesc}{add_flowing_data}{data}
Provide data which should be formatted with collapsed whitespaces.
Whitespace from preceeding and successive calls to
\code{add_flowing_data()} is considered as well when the whitespace
collapse is performed.  The data which is passed to this method is
expected to be word-wrapped by the output device.  Note that any
word-wrapping still must be performed by the writer object due to the
need to rely on device and font information.
\end{funcdesc}

\begin{funcdesc}{add_literal_data}{data}
Provide data which should be passed to the writer unchanged.
Whitespace, including newline and tab characters, are considered legal
in the value of \code{data}.  
\end{funcdesc}

\begin{funcdesc}{add_label_data}{format, counter}
Insert a label which should be placed to the left of the current left
margin.  This should be used for constructing bulleted or numbered
lists.  If the \code{format} value is a string, it is interpreted as a
format specification for \code{counter}, which should be an integer.
The result of this formatting becomes the value of the label; if
\code{format} is not a string it is used as the label value directly.
The label value is passed as the only argument to the writer's
\code{send_label_data()} method.  Interpretation of non-string label
values is dependent on the associated writer.

Format specifications are strings which, in combination with a counter
value, are used to compute label values.  Each character in the format
string is copied to the label value, with some characters recognized
to indicate a transform on the counter value.  Specifically, the
character ``\code{1}'' represents the counter value formatter as an
arabic number, the characters ``\code{A}'' and ``\code{a}'' represent
alphabetic representations of the counter value in upper and lower
case, respectively, and ``\code{I}'' and ``\code{i}'' represent the
counter value in Roman numerals, in upper and lower case.  Note that
the alphabetic and roman transforms require that the counter value be
greater than zero.
\end{funcdesc}

\begin{funcdesc}{flush_softspace}{}
Send any pending whitespace buffered from a previous call to
\code{add_flowing_data()} to the associated writer object.  This
should be called before any direct manipulation of the writer object.
\end{funcdesc}

\begin{funcdesc}{push_alignment}{align}
Push a new alignment setting onto the alignment stack.  This may be
\code{AS_IS} if no change is desired.  If the alignment value is
changed from the previous setting, the writer's \code{new_alignment()}
method is called with the \code{align} value.
\end{funcdesc}

\begin{funcdesc}{pop_alignment}{}
Restore the previous alignment.
\end{funcdesc}

\begin{funcdesc}{push_font}{(size, italic, bold, teletype)}
Change some or all font properties of the writer object.  Properties
which are not set to \code{AS_IS} are set to the values passed in
while others are maintained at their current settings.  The writer's
\code{new_font()} method is called with the fully resolved font
specification.
\end{funcdesc}

\begin{funcdesc}{pop_font}{}
Restore the previous font.
\end{funcdesc}

\begin{funcdesc}{push_margin}{margin}
Increase the number of left margin indentations by one, associating
the logical tag \code{margin} with the new indentation.  The initial
margin level is \code{0}.  Changed values of the logical tag must be
true values; false values other than \code{AS_IS} are not sufficient
to change the margin.
\end{funcdesc}

\begin{funcdesc}{pop_margin}{}
Restore the previous margin.
\end{funcdesc}

\begin{funcdesc}{push_style}{*styles}
Push any number of arbitrary style specifications.  All styles are
pushed onto the styles stack in order.  A tuple representing the
entire stack, including \code{AS_IS} values, is passed to the writer's
\code{new_styles()} method.
\end{funcdesc}

\begin{funcdesc}{pop_style}{\optional{n\code{ = 1}}}
Pop the last \code{n} style specifications passed to
\code{push_style()}.  A tuple representing the revised stack,
including \code{AS_IS} values, is passed to the writer's
\code{new_styles()} method.
\end{funcdesc}

\begin{funcdesc}{set_spacing}{spacing}
Set the spacing style for the writer.
\end{funcdesc}

\begin{funcdesc}{assert_line_data}{\optional{flag\code{ = 1}}}
Inform the formatter that data has been added to the current paragraph
out-of-band.  This should be used when the writer has been manipulated
directly.  The optional \code{flag} argument can be set to false if
the writer manipulations produced a hard line break at the end of the
output.
\end{funcdesc}


\subsection{Formatter Implementations}

Two implementations of formatter objects are provided by this module.
Most applications may use one of these classes without modification or
subclassing.

\renewcommand{\indexsubitem}{(in module formatter)}

\begin{funcdesc}{NullFormatter}{\optional{writer\code{ = None}}}
A formatter which does nothing.  If \code{writer} is omitted, a
\code{NullWriter} instance is created.  No methods of the writer are
called by \code{NullWriter} instances.  Implementations should inherit
from this class if implementing a writer interface but don't need to
inherit any implementation.
\end{funcdesc}

\begin{funcdesc}{AbstractFormatter}{writer}
The standard formatter.  This implementation has demonstrated wide
applicability to many writers, and may be used directly in most
circumstances.  It has been used to implement a full-featured
world-wide web browser.
\end{funcdesc}



\subsection{The Writer Interface}

Interfaces to create writers are dependent on the specific writer
class being instantiated.  The interfaces described below are the
required interfaces which all writers must support once initialized.
Note that while most applications can use the \code{AbstractFormatter}
class as a formatter, the writer must typically be provided by the
application.

\renewcommand{\indexsubitem}{(writer object method)}

\begin{funcdesc}{new_alignment}{align}
Set the alignment style.  The \code{align} value can be any object,
but by convention is a string or \code{None}, where \code{None}
indicates that the writer's ``preferred'' alignment should be used.
Conventional \code{align} values are \code{'left'}, \code{'center'},
\code{'right'}, and \code{'justify'}.
\end{funcdesc}

\begin{funcdesc}{new_font}{font}
Set the font style.  The value of \code{font} will be \code{None},
indicating that the device's default font should be used, or a tuple
of the form (\var{size}, \var{italic}, \var{bold}, \var{teletype}).
Size will be a string indicating the size of font that should be used;
specific strings and their interpretation must be defined by the
application.  The \var{italic}, \var{bold}, and \var{teletype} values
are boolean indicators specifying which of those font attributes
should be used.
\end{funcdesc}

\begin{funcdesc}{new_margin}{margin, level}
Set the margin level to the integer \code{level} and the logical tag
to \code{margin}.  Interpretation of the logical tag is at the
writer's discretion; the only restriction on the value of the logical
tag is that it not be a false value for non-zero values of
\code{level}.
\end{funcdesc}

\begin{funcdesc}{new_spacing}{spacing}
Set the spacing style to \code{spacing}.
\end{funcdesc}

\begin{funcdesc}{new_styles}{styles}
Set additional styles.  The \code{styles} value is a tuple of
arbitrary values; the value \code{AS_IS} should be ignored.  The
\code{styles} tuple may be interpreted either as a set or as a stack
depending on the requirements of the application and writer
implementation.
\end{funcdesc}

\begin{funcdesc}{send_line_break}{}
Break the current line.
\end{funcdesc}

\begin{funcdesc}{send_paragraph}{blankline}
Produce a paragraph separation of at least \code{blankline} blank
lines, or the equivelent.  The \code{blankline} value will be an
integer.
\end{funcdesc}

\begin{funcdesc}{send_hor_rule}{*args\, **kw}
Display a horizontal rule on the output device.  The arguments to this
method are entirely application- and writer-specific, and should be
interpreted with care.  The method implementation may assume that a
line break has already been issued via \code{send_line_break()}.
\end{funcdesc}

\begin{funcdesc}{send_flowing_data}{data}
Output character data which may be word-wrapped and re-flowed as
needed.  Within any sequence of calls to this method, the writer may
assume that spans of multiple whitespace characters have been
collapsed to single space characters.
\end{funcdesc}

\begin{funcdesc}{send_literal_data}{data}
Output character data which has already been formatted
for display.  Generally, this should be interpreted to mean that line
breaks indicated by newline characters should be preserved and no new
line breaks should be introduced.  The data may contain embedded
newline and tab characters, unlike data provided to the
\code{send_formatted_data()} interface.
\end{funcdesc}

\begin{funcdesc}{send_label_data}{data}
Set \code{data} to the left of the current left margin, if possible.
The value of \code{data} is not restricted; treatment of non-string
values is entirely application- and writer-dependent.  This method
will only be called at the beginning of a line.
\end{funcdesc}


\subsection{Writer Implementations}

Three implementations of the writer object interface are provided as
examples by this module.  Most applications will need to derive new
writer classes from the \code{NullWriter} class.

\renewcommand{\indexsubitem}{(in module formatter)}

\begin{funcdesc}{NullWriter}{}
A writer which only provides the interface definition; no actions are
taken on any methods.  This should be the base class for all writers
which do not need to inherit any implementation methods.
\end{funcdesc}

\begin{funcdesc}{AbstractWriter}{}
A writer which can be used in debugging formatters, but not much
else.  Each method simply accounces itself by printing its name and
arguments on standard output.
\end{funcdesc}

\begin{funcdesc}{DumbWriter}{\optional{file\code{ = None}\optional{\, maxcol\code{ = 72}}}}
Simple writer class which writes output on the file object passed in
as \code{file} or, if \code{file} is omitted, on standard output.  The
output is simply word-wrapped to the number of columns specified by
\code{maxcol}.  This class is suitable for reflowing a sequence of
paragraphs.
\end{funcdesc}

\section{Standard Module \sectcode{rfc822}}
\stmodindex{rfc822}

\renewcommand{\indexsubitem}{(in module rfc822)}

This module defines a class, \code{Message}, which represents a
collection of ``email headers'' as defined by the Internet standard
RFC 822.  It is used in various contexts, usually to read such headers
from a file.

A \code{Message} instance is instantiated with an open file object as
parameter.  Instantiation reads headers from the file up to a blank
line and stores them in the instance; after instantiation, the file is
positioned directly after the blank line that terminates the headers.

Input lines as read from the file may either be terminated by CR-LF or
by a single linefeed; a terminating CR-LF is replaced by a single
linefeed before the line is stored.

All header matching is done independent of upper or lower case;
e.g. \code{m['From']}, \code{m['from']} and \code{m['FROM']} all yield
the same result.

\subsection{Message Objects}

A \code{Message} instance has the following methods:

\begin{funcdesc}{rewindbody}{}
Seek to the start of the message body.  This only works if the file
object is seekable.
\end{funcdesc}

\begin{funcdesc}{getallmatchingheaders}{name}
Return a list of lines consisting of all headers matching
\var{name}, if any.  Each physical line, whether it is a continuation
line or not, is a separate list item.  Return the empty list if no
header matches \var{name}.
\end{funcdesc}

\begin{funcdesc}{getfirstmatchingheader}{name}
Return a list of lines comprising the first header matching
\var{name}, and its continuation line(s), if any.  Return \code{None}
if there is no header matching \var{name}.
\end{funcdesc}

\begin{funcdesc}{getrawheader}{name}
Return a single string consisting of the text after the colon in the
first header matching \var{name}.  This includes leading whitespace,
the trailing linefeed, and internal linefeeds and whitespace if there
any continuation line(s) were present.  Return \code{None} if there is
no header matching \var{name}.
\end{funcdesc}

\begin{funcdesc}{getheader}{name}
Like \code{getrawheader(\var{name})}, but strip leading and trailing
whitespace (but not internal whitespace).
\end{funcdesc}

\begin{funcdesc}{getaddr}{name}
Return a pair (full name, email address) parsed from the string
returned by \code{getheader(\var{name})}.  If no header matching
\var{name} exists, return \code{None, None}; otherwise both the full
name and the address are (possibly empty )strings.

Example: If \code{m}'s first \code{From} header contains the string\\
\code{'jack@cwi.nl (Jack Jansen)'}, then
\code{m.getaddr('From')} will yield the pair
\code{('Jack Jansen', 'jack@cwi.nl')}.
If the header contained
\code{'Jack Jansen <jack@cwi.nl>'} instead, it would yield the
exact same result.
\end{funcdesc}

\begin{funcdesc}{getaddrlist}{name}
This is similar to \code{getaddr(\var{list})}, but parses a header
containing a list of email addresses (e.g. a \code{To} header) and
returns a list of (full name, email address) pairs (even if there was
only one address in the header).  If there is no header matching
\var{name}, return an empty list.

XXX The current version of this function is not really correct.  It
yields bogus results if a full name contains a comma.
\end{funcdesc}

\begin{funcdesc}{getdate}{name}
Retrieve a header using \code{getheader} and parse it into a 9-tuple
compatible with \code{time.mktime()}.  If there is no header matching
\var{name}, or it is unparsable, return \code{None}.

Date parsing appears to be a black art, and not all mailers adhere to
the standard.  While it has been tested and found correct on a large
collection of email from many sources, it is still possible that this
function may occasionally yield an incorrect result.
\end{funcdesc}

\code{Message} instances also support a read-only mapping interface.
In particular: \code{m[name]} is the same as \code{m.getheader(name)};
and \code{len(m)}, \code{m.has_key(name)}, \code{m.keys()},
\code{m.values()} and \code{m.items()} act as expected (and
consistently).

Finally, \code{Message} instances have two public instance variables:

\begin{datadesc}{headers}
A list containing the entire set of header lines, in the order in
which they were read.  Each line contains a trailing newline.  The
blank line terminating the headers is not contained in the list.
\end{datadesc}

\begin{datadesc}{fp}
The file object passed at instantiation time.
\end{datadesc}

\section{Standard Module \sectcode{mimetools}}
\stmodindex{mimetools}

\renewcommand{\indexsubitem}{(in module mimetools)}

This module defines a subclass of the class \code{rfc822.Message} and
a number of utility functions that are useful for the manipulation for
MIME style multipart or encoded message.

It defines the following items:

\begin{funcdesc}{Message}{fp}
Return a new instance of the \code{mimetools.Message} class.  This is
a subclass of the \code{rfc822.Message} class, with some additional
methods (see below).
\end{funcdesc}

\begin{funcdesc}{choose_boundary}{}
Return a unique string that has a high likelihood of being usable as a
part boundary.  The string has the form
\code{"\var{hostipaddr}.\var{uid}.\var{pid}.\var{timestamp}.\var{random}"}.
\end{funcdesc}

\begin{funcdesc}{decode}{input\, output\, encoding}
Read data encoded using the allowed MIME \var{encoding} from open file
object \var{input} and write the decoded data to open file object
\var{output}.  Valid values for \var{encoding} include
\code{"base64"}, \code{"quoted-printable"} and \code{"uuencode"}.
\end{funcdesc}

\begin{funcdesc}{encode}{input\, output\, encoding}
Read data from open file object \var{input} and write it encoded using
the allowed MIME \var{encoding} to open file object \var{output}.
Valid values for \var{encoding} are the same as for \code{decode()}.
\end{funcdesc}

\begin{funcdesc}{copyliteral}{input\, output}
Read lines until EOF from open file \var{input} and write them to open
file \var{output}.
\end{funcdesc}

\begin{funcdesc}{copybinary}{input\, output}
Read blocks until EOF from open file \var{input} and write them to open
file \var{output}.  The block size is currently fixed at 8192.
\end{funcdesc}


\subsection{Additional Methods of Message objects}
\nodename{mimetools.Message Methods}

The \code{mimetools.Message} class defines the following methods in
addition to the \code{rfc822.Message} class:

\renewcommand{\indexsubitem}{(mimetool.Message method)}

\begin{funcdesc}{getplist}{}
Return the parameter list of the \code{Content-type} header.  This is
a list if strings.  For parameters of the form
\samp{\var{key}=\var{value}}, \var{key} is converted to lower case but
\var{value} is not.  For example, if the message contains the header
\samp{Content-type: text/html; spam=1; Spam=2; Spam} then
\code{getplist()} will return the Python list \code{['spam=1',
'spam=2', 'Spam']}.
\end{funcdesc}

\begin{funcdesc}{getparam}{name}
Return the \var{value} of the first parameter (as returned by
\code{getplist()} of the form \samp{\var{name}=\var{value}} for the
given \var{name}.  If \var{value} is surrounded by quotes of the form
\var{<...>} or \var{"..."}, these are removed.
\end{funcdesc}

\begin{funcdesc}{getencoding}{}
Return the encoding specified in the \samp{Content-transfer-encoding}
message header.  If no such header exists, return \code{"7bit"}.  The
encoding is converted to lower case.
\end{funcdesc}

\begin{funcdesc}{gettype}{}
Return the message type (of the form \samp{\var{type}/var{subtype}})
as specified in the \samp{Content-type} header.  If no such header
exists, return \code{"text/plain"}.  The type is converted to lower
case.
\end{funcdesc}

\begin{funcdesc}{getmaintype}{}
Return the main type as specified in the \samp{Content-type} header.
If no such header exists, return \code{"text"}.  The main type is
converted to lower case.
\end{funcdesc}

\begin{funcdesc}{getsubtype}{}
Return the subtype as specified in the \samp{Content-type} header.  If
no such header exists, return \code{"plain"}.  The subtype is
converted to lower case.
\end{funcdesc}

\section{Standard module \sectcode{binhex}}
\stmodindex{binhex}

This module encodes and decodes files in binhex4 format, a format
allowing representation of Macintosh files in ASCII. On the macintosh,
both forks of a file and the finder information are encoded (or
decoded), on other platforms only the data fork is handled.

The \code{binhex} module defines the following functions:

\renewcommand{\indexsubitem}{(in module binhex)}

\begin{funcdesc}{binhex}{input\, output}
Convert a binary file with filename \var{input} to binhex file
\var{output}. The \var{output} parameter can either be a filename or a
file-like object (any object supporting a \var{write} and \var{close}
method).
\end{funcdesc}

\begin{funcdesc}{hexbin}{input\optional{\, output}}
Decode a binhex file \var{input}. \var{Input} may be a filename or a
file-like object supporting \var{read} and \var{close} methods.
The resulting file is written to a file named \var{output}, unless the
argument is empty in which case the output filename is read from the
binhex file.
\end{funcdesc}

\subsection{notes}
There is an alternative, more powerful interface to the coder and
decoder, see the source for details.

If you code or decode textfiles on non-Macintosh platforms they will
still use the macintosh newline convention (carriage-return as end of
line).

As of this writing, \var{hexbin} appears to not work in all cases.

\section{Standard module \sectcode{uu}}
\stmodindex{uu}

This module encodes and decodes files in uuencode format, allowing
arbitrary binary data to be transferred over ascii-only connections.
Whereever a file argument is expected, the methods accept either a
pathname (\code{'-'} for stdin/stdout) or a file-like object.

Normally you would pass filenames, but there is one case where you
have to open the file yourself: if you are on a non-unix platform and
your binary file is actually a textfile that you want encoded
unix-compatible you will have to open the file yourself as a textfile,
so newline conversion is performed.

This code was contributed by Lance Ellinghouse, and modified by Jack
Jansen.

The \code{uu} module defines the following functions:

\renewcommand{\indexsubitem}{(in module uu)}

\begin{funcdesc}{encode}{in_file\, out_file\optional{\, name\, mode}}
Uuencode file \var{in_file} into file \var{out_file}.  The uuencoded
file will have the header specifying \var{name} and \var{mode} as the
defaults for the results of decoding the file. The default defaults
are taken from \var{in_file}, or \code{'-'} and \code{0666}
respectively. 
\end{funcdesc}

\begin{funcdesc}{decode}{in_file\optional{\, out_file\, mode}}
This call decodes uuencoded file \var{in_file} placing the result on
file \var{out_file}. If \var{out_file} is a pathname the \var{mode} is
also set. Defaults for \var{out_file} and \var{mode} are taken from
the uuencode header.
\end{funcdesc}

\section{Built-in Module \sectcode{binascii}}	% If implemented in C
\bimodindex{binascii}

The binascii module contains a number of methods to convert between
binary and various ascii-encoded binary representations. Normally, you
will not use these modules directly but use wrapper modules like
\var{uu} or \var{hexbin} in stead, this module solely exists because
bit-manipuation of large amounts of data is slow in python.

The \code{binascii} module defines the following functions:

\renewcommand{\indexsubitem}{(in module binascii)}

\begin{funcdesc}{a2b_uu}{string}
Convert a single line of uuencoded data back to binary and return the
binary data. Lines normally contain 45 (binary) bytes, except for the
last line. Line data may be followed by whitespace.
\end{funcdesc}

\begin{funcdesc}{b2a_uu}{data}
Convert binary data to a line of ascii characters, the return value is
the converted line, including a newline char. The length of \var{data}
should be at most 45.
\end{funcdesc}

\begin{funcdesc}{a2b_base64}{string}
Convert a block of base64 data back to binary and return the
binary data. More than one line may be passed at a time.
\end{funcdesc}

\begin{funcdesc}{b2a_base64}{data}
Convert binary data to a line of ascii characters in base64 coding.
The return value is the converted line, including a newline char.
The length of \var{data} should be at most 57 to adhere to the base64
standard.
\end{funcdesc}

\begin{funcdesc}{a2b_hqx}{string}
Convert binhex4 formatted ascii data to binary, without doing
rle-decompression. The string should contain a complete number of
binary bytes, or (in case of the last portion of the binhex4 data)
have the remaining bits zero.
\end{funcdesc}

\begin{funcdesc}{rledecode_hqx}{data}
Perform RLE-decompression on the data, as per the binhex4
standard. The algorithm uses \code{0x90} after a byte as a repeat
indicator, followed by a count. A count of \code{0} specifies a byte
value of \code{0x90}. The routine returns the decompressed data,
unless data input data ends in an orphaned repeat indicator, in which
case the \var{Incomplete} exception is raised.
\end{funcdesc}

\begin{funcdesc}{rlecode_hqx}{data}
Perform binhex4 style RLE-compression on \var{data} and return the
result.
\end{funcdesc}

\begin{funcdesc}{b2a_hqx}{data}
Perform hexbin4 binary-to-ascii translation and return the resulting
string. The argument should already be rle-coded, and have a length
divisible by 3 (except possibly the last fragment).
\end{funcdesc}

\begin{funcdesc}{crc_hqx}{data, crc}
Compute the binhex4 crc value of \var{data}, starting with an initial
\var{crc} and returning the result.
\end{funcdesc}
 
\begin{excdesc}{Error}
Exception raised on errors. These are usually programming errors.
\end{excdesc}

\begin{excdesc}{Incomplete}
Exception raised on incomplete data. These are usually not programming
errors, but handled by reading a little more data and trying again.
\end{excdesc}

\section{Standard module \sectcode{xdrlib}}
\stmodindex{xdrlib}
\index{XDR}

\renewcommand{\indexsubitem}{(in module xdrlib)}


The \code{xdrlib} module supports the External Data Representation
Standard as described in RFC 1014, written by Sun Microsystems,
Inc. June 1987.  It supports most of the data types described in the
RFC, although some, most notably \code{float} and \code{double} are
only supported on those operating systems that provide an XDR
library.

The \code{xdrlib} module defines two classes, one for packing
variables into XDR representation, and another for unpacking from XDR
representation.  There are also two exception classes.


\subsection{Packer Objects}

\code{Packer} is the class for packing data into XDR representation.
The \code{Packer} class is instantiated with no arguments.

\begin{funcdesc}{get_buffer}{}
Returns the current pack buffer as a string.
\end{funcdesc}

\begin{funcdesc}{reset}{}
Resets the pack buffer to the empty string.
\end{funcdesc}

In general, you can pack any of the most common XDR data types by
calling the appropriate \code{pack_\var{type}} method.  Each method
takes a single argument, the value to pack.  The following simple data
type packing methods are supported: \code{pack_uint}, \code{pack_int},
\code{pack_enum}, \code{pack_bool}, \code{pack_uhyper},
and \code{pack_hyper}.

The following methods pack floating point numbers, however they
require C library support.  Without the optional C built-in module,
both of these methods will raise an \code{xdrlib.ConversionError}
exception.  See the note at the end of this chapter for details.

\begin{funcdesc}{pack_float}{value}
Packs the single-precision floating point number \var{value}.
\end{funcdesc}

\begin{funcdesc}{pack_double}{value}
Packs the double-precision floating point number \var{value}.
\end{funcdesc}

The following methods support packing strings, bytes, and opaque data:

\begin{funcdesc}{pack_fstring}{n\, s}
Packs a fixed length string, \var{s}.  \var{n} is the length of the
string but it is \emph{not} packed into the data buffer.  The string
is padded with null bytes if necessary to guaranteed 4 byte alignment.
\end{funcdesc}

\begin{funcdesc}{pack_fopaque}{n\, data}
Packs a fixed length opaque data stream, similarly to
\code{pack_fstring}.
\end{funcdesc}

\begin{funcdesc}{pack_string}{s}
Packs a variable length string, \var{s}.  The length of the string is
first packed as an unsigned integer, then the string data is packed
with \code{pack_fstring}.
\end{funcdesc}

\begin{funcdesc}{pack_opaque}{data}
Packs a variable length opaque data string, similarly to
\code{pack_string}.
\end{funcdesc}

\begin{funcdesc}{pack_bytes}{bytes}
Packs a variable length byte stream, similarly to \code{pack_string}.
\end{funcdesc}

The following methods support packing arrays and lists:

\begin{funcdesc}{pack_list}{list\, pack_item}
Packs a \var{list} of homogeneous items.  This method is useful for
lists with an indeterminate size; i.e. the size is not available until
the entire list has been walked.  For each item in the list, an
unsigned integer \code{1} is packed first, followed by the data value
from the list.  \var{pack_item} is the function that is called to pack
the individual item.  At the end of the list, an unsigned integer
\code{0} is packed.
\end{funcdesc}

\begin{funcdesc}{pack_farray}{n\, array\, pack_item}
Packs a fixed length list (\var{array}) of homogeneous items.  \var{n}
is the length of the list; it is \emph{not} packed into the buffer,
but a \code{ValueError} exception is raised if \code{len(array)} is not
equal to \var{n}.  As above, \var{pack_item} is the function used to
pack each element.
\end{funcdesc}

\begin{funcdesc}{pack_array}{list\, pack_item}
Packs a variable length \var{list} of homogeneous items.  First, the
length of the list is packed as an unsigned integer, then each element
is packed as in \code{pack_farray} above.
\end{funcdesc}

\subsection{Unpacker Objects}

\code{Unpacker} is the complementary class which unpacks XDR data
values from a string buffer, and has the following methods:

\begin{funcdesc}{__init__}{data}
Instantiates an \code{Unpacker} object with the string buffer
\var{data}.
\end{funcdesc}

\begin{funcdesc}{reset}{data}
Resets the string buffer with the given \var{data}.
\end{funcdesc}

\begin{funcdesc}{get_position}{}
Returns the current unpack position in the data buffer.
\end{funcdesc}

\begin{funcdesc}{set_position}{position}
Sets the data buffer unpack position to \var{position}.  You should be
careful about using \code{get_position()} and \code{set_position()}.
\end{funcdesc}

\begin{funcdesc}{done}{}
Indicates unpack completion.  Raises an \code{xdrlib.Error} exception
if all of the data has not been unpacked.
\end{funcdesc}

In addition, every data type that can be packed with a \code{Packer},
can be unpacked with an \code{Unpacker}.  Unpacking methods are of the
form \code{unpack_\var{type}}, and take no arguments.  They return the
unpacked object.  The same caveats apply for \code{unpack_float} and
\code{unpack_double} as above.

\begin{funcdesc}{unpack_float}{}
Unpacks a single-precision floating point number.
\end{funcdesc}

\begin{funcdesc}{unpack_double}{}
Unpacks a double-precision floating point number, similarly to
\code{unpack_float}.
\end{funcdesc}

In addition, the following methods unpack strings, bytes, and opaque
data:

\begin{funcdesc}{unpack_fstring}{n}
Unpacks and returns a fixed length string.  \var{n} is the number of
characters expected.  Padding with null bytes to guaranteed 4 byte
alignment is assumed.
\end{funcdesc}

\begin{funcdesc}{unpack_fopaque}{n}
Unpacks and returns a fixed length opaque data stream, similarly to
\code{unpack_fstring}.
\end{funcdesc}

\begin{funcdesc}{unpack_string}{}
Unpacks and returns a variable length string.  The length of the
string is first unpacked as an unsigned integer, then the string data
is unpacked with \code{unpack_fstring}.
\end{funcdesc}

\begin{funcdesc}{unpack_opaque}{}
Unpacks and returns a variable length opaque data string, similarly to
\code{unpack_string}.
\end{funcdesc}

\begin{funcdesc}{unpack_bytes}{}
Unpacks and returns a variable length byte stream, similarly to
\code{unpack_string}.
\end{funcdesc}

The following methods support unpacking arrays and lists:

\begin{funcdesc}{unpack_list}{unpack_item}
Unpacks and returns a list of homogeneous items.  The list is unpacked
one element at a time
by first unpacking an unsigned integer flag.  If the flag is \code{1},
then the item is unpacked and appended to the list.  A flag of
\code{0} indicates the end of the list.  \var{unpack_item} is the
function that is called to unpack the items.
\end{funcdesc}

\begin{funcdesc}{unpack_farray}{n\, unpack_item}
Unpacks and returns (as a list) a fixed length array of homogeneous
items.  \var{n} is number of list elements to expect in the buffer.
As above, \var{unpack_item} is the function used to unpack each element.
\end{funcdesc}

\begin{funcdesc}{unpack_array}{unpack_item}
Unpacks and returns a variable length \var{list} of homogeneous items.
First, the length of the list is unpacked as an unsigned integer, then
each element is unpacked as in \code{unpack_farray} above.
\end{funcdesc}

\subsection{Exceptions}

Exceptions in this module are coded as class instances:

\begin{excdesc}{Error}
The base exception class.  \code{Error} has a single public data
member \code{msg} containing the description of the error.
\end{excdesc}

\begin{excdesc}{ConversionError}
Class derived from \code{Error}.  Contains no additional instance
variables.
\end{excdesc}

Here is an example of how you would catch one of these exceptions:

\begin{verbatim}
import xdrlib
p = xdrlib.Packer()
try:
    p.pack_double(8.01)
except xdrlib.ConversionError, instance:
    print 'packing the double failed:', instance.msg
\end{verbatim}

\subsection{Supporting Floating Point Data}

Packing and unpacking floating point data,
i.e. \code{Packer.pack_float}, \code{Packer.pack_double},
\code{Unpacker.unpack_float}, and \code{Unpacker.unpack_double}, are
only supported with the helper built-in \code{_xdr} module, which
relies on your operating system having the appropriate XDR library
routines.

If you have built the Python interpeter with the \code{_xdr} module,
or have built the \code{_xdr} module as a shared library,
\code{xdrlib} will use these to pack and unpack floating point
numbers.  Otherwise, using these routines will raise a
\code{ConversionError} exception.

See the Python installation instructions for details on building the
\code{_xdr} module.


\chapter{Restricted Execution}

In general, Python programs have complete access to the underlying
operating system throug the various functions and classes, For
example, a Python program can open any file for reading and writing by
using the \code{open()} built-in function (provided the underlying OS
gives you permission!).  This is exactly what you want for most
applications.

There exists a class of applications for which this ``openness'' is
inappropriate.  Take Grail: a web browser that accepts ``applets'',
snippets of Python code, from anywhere on the Internet for execution
on the local system.  This can be used to improve the user interface
of forms, for instance.  Since the originator of the code is unknown,
it is obvious that it cannot be trusted with the full resources of the
local machine.

\emph{Restricted execution} is the basic framework in Python that allows
for the segregation of trusted and untrusted code.  It is based on the
notion that trusted Python code (a \emph{supervisor}) can create a
``padded cell' (or environment) with limited permissions, and run the
untrusted code within this cell.  The untrusted code cannot break out
of its cell, and can only interact with sensitive system resources
through interfaces defined and managed by the trusted code.  The term
``restricted execution'' is favored over ``safe-Python''
since true safety is hard to define, and is determined by the way the
restricted environment is created.  Note that the restricted
environments can be nested, with inner cells creating subcells of
lesser, but never greater, privilege.

An interesting aspect of Python's restricted execution model is that
the interfaces presented to untrusted code usually have the same names
as those presented to trusted code.  Therefore no special interfaces
need to be learned to write code designed to run in a restricted
environment.  And because the exact nature of the padded cell is
determined by the supervisor, different restrictions can be imposed,
depending on the application.  For example, it might be deemed
``safe'' for untrusted code to read any file within a specified
directory, but never to write a file.  In this case, the supervisor
may redefine the built-in
\code{open()} function so that it raises an exception whenever the
\var{mode} parameter is \code{'w'}.  It might also perform a
\code{chroot()}-like operation on the \var{filename} parameter, such
that root is always relative to some safe ``sandbox'' area of the
filesystem.  In this case, the untrusted code would still see an
built-in \code{open()} function in its environment, with the same
calling interface.  The semantics would be identical too, with
\code{IOError}s being raised when the supervisor determined that an
unallowable parameter is being used.

The Python run-time determines whether a particular code block is
executing in restricted execution mode based on the identity of the
\code{__builtins__} object in its global variables: if this is (the
dictionary of) the standard \code{__builtin__} module, the code is
deemed to be unrestricted, else it is deemed to be restricted.

Python code executing in restricted mode faces a number of limitations
that are designed to prevent it from escaping from the padded cell.
For instance, the function object attribute \code{func_globals} and the
class and instance object attribute \code{__dict__} are unavailable.

Two modules provide the framework for setting up restricted execution
environments:

\begin{description}

\item[rexec]
--- Basic restricted execution framework.

\item[Bastion]
--- Providing restricted access to objects.

\end{description}

\section{Standard Module \sectcode{rexec}}
\stmodindex{rexec}
\renewcommand{\indexsubitem}{(in module rexec)}

This module contains the \code{RExec} class, which supports
\code{r_exec()}, \code{r_eval()}, \code{r_execfile()}, and
\code{r_import()} methods, which are restricted versions of the standard
Python functions \code{exec()}, \code{eval()}, \code{execfile()}, and
the \code{import} statement.
Code executed in this restricted environment will
only have access to modules and functions that are deemed safe; you
can subclass \code{RExec} to add or remove capabilities as desired.

\emph{Note:} The \code{RExec} class can prevent code from performing
unsafe operations like reading or writing disk files, or using TCP/IP
sockets.  However, it does not protect against code using extremely
large amounts of memory or CPU time.  

\begin{funcdesc}{RExec}{\optional{hooks\optional{\, verbose}}}
Returns an instance of the \code{RExec} class.  

\var{hooks} is an instance of the \code{RHooks} class or a subclass of it.
If it is omitted or \code{None}, the default \code{RHooks} class is
instantiated.
Whenever the RExec module searches for a module (even a built-in one)
or reads a module's code, it doesn't actually go out to the file
system itself.  Rather, it calls methods of an RHooks instance that
was passed to or created by its constructor.  (Actually, the RExec
object doesn't make these calls---they are made by a module loader
object that's part of the RExec object.  This allows another level of
flexibility, e.g. using packages.)

By providing an alternate RHooks object, we can control the
file system accesses made to import a module, without changing the
actual algorithm that controls the order in which those accesses are
made.  For instance, we could substitute an RHooks object that passes
all filesystem requests to a file server elsewhere, via some RPC
mechanism such as ILU.  Grail's applet loader uses this to support
importing applets from a URL for a directory.

If \var{verbose} is true, additional debugging output may be sent to
standard output.
\end{funcdesc}

The RExec class has the following class attributes, which are used by the
\code{__init__} method.  Changing them on an existing instance won't
have any effect; instead, create a subclass of \code{RExec} and assign
them new values in the class definition.  Instances of the new class
will then use those new values.  All these attributes are tuples of
strings.

\renewcommand{\indexsubitem}{(RExec object attribute)}
\begin{datadesc}{nok_builtin_names}
Contains the names of built-in functions which will \emph{not} be
available to programs running in the restricted environment.  The
value for \code{RExec} is \code{('open',} \code{'reload',}
\code{'__import__')}.  (This gives the exceptions, because by far the
majority of built-in functions are harmless.  A subclass that wants to
override this variable should probably start with the value from the
base class and concatenate additional forbidden functions --- when new
dangerous built-in functions are added to Python, they will also be
added to this module.)
\end{datadesc}

\begin{datadesc}{ok_builtin_modules}
Contains the names of built-in modules which can be safely imported.
The value for \code{RExec} is \code{('audioop',} \code{'array',}
\code{'binascii',} \code{'cmath',} \code{'errno',} \code{'imageop',}
\code{'marshal',} \code{'math',} \code{'md5',} \code{'operator',}
\code{'parser',} \code{'regex',} \code{'rotor',} \code{'select',}
\code{'strop',} \code{'struct',} \code{'time')}.  A similar remark
about overriding this variable applies --- use the value from the base
class as a starting point.
\end{datadesc}

\begin{datadesc}{ok_path}
Contains the directories which will be searched when an \code{import}
is performed in the restricted environment.  
The value for \code{RExec} is the same as \code{sys.path} (at the time
the module is loaded) for unrestricted code.
\end{datadesc}

\begin{datadesc}{ok_posix_names}
% Should this be called ok_os_names?
Contains the names of the functions in the \code{os} module which will be
available to programs running in the restricted environment.  The
value for \code{RExec} is \code{('error',} \code{'fstat',}
\code{'listdir',} \code{'lstat',} \code{'readlink',} \code{'stat',}
\code{'times',} \code{'uname',} \code{'getpid',} \code{'getppid',}
\code{'getcwd',} \code{'getuid',} \code{'getgid',} \code{'geteuid',}
\code{'getegid')}.
\end{datadesc}

\begin{datadesc}{ok_sys_names}
Contains the names of the functions and variables in the \code{sys}
module which will be available to programs running in the restricted
environment.  The value for \code{RExec} is \code{('ps1',}
\code{'ps2',} \code{'copyright',} \code{'version',} \code{'platform',}
\code{'exit',} \code{'maxint')}.
\end{datadesc}

RExec instances support the following methods:
\renewcommand{\indexsubitem}{(RExec object method)}

\begin{funcdesc}{r_eval}{code}
\var{code} must either be a string containing a Python expression, or
a compiled code object, which will be evaluated in the restricted
environment's \code{__main__} module.  The value of the expression or
code object will be returned.
\end{funcdesc}

\begin{funcdesc}{r_exec}{code}
\var{code} must either be a string containing one or more lines of
Python code, or a compiled code object, which will be executed in the
restricted environment's \code{__main__} module.
\end{funcdesc}

\begin{funcdesc}{r_execfile}{filename}
Execute the Python code contained in the file \var{filename} in the
restricted environment's \code{__main__} module.
\end{funcdesc}

Methods whose names begin with \code{s_} are similar to the functions
beginning with \code{r_}, but the code will be granted access to
restricted versions of the standard I/O streans \code{sys.stdin},
\code{sys.stderr}, and \code{sys.stdout}.  

\begin{funcdesc}{s_eval}{code}
\var{code} must be a string containing a Python expression, which will
be evaluated in the restricted environment.  
\end{funcdesc}

\begin{funcdesc}{s_exec}{code}
\var{code} must be a string containing one or more lines of Python code,
which will be executed in the restricted environment.  
\end{funcdesc}

\begin{funcdesc}{s_execfile}{code}
Execute the Python code contained in the file \var{filename} in the
restricted environment.
\end{funcdesc}

\code{RExec} objects must also support various methods which will be
implicitly called by code executing in the restricted environment.
Overriding these methods in a subclass is used to change the policies
enforced by a restricted environment.

\begin{funcdesc}{r_import}{modulename\optional{\, globals\, locals\, fromlist}}
Import the module \var{modulename}, raising an \code{ImportError}
exception if the module is considered unsafe.
\end{funcdesc}

\begin{funcdesc}{r_open}{filename\optional{\, mode\optional{\, bufsize}}}
Method called when \code{open()} is called in the restricted
environment.  The arguments are identical to those of \code{open()},
and a file object (or a class instance compatible with file objects)
should be returned.  \code{RExec}'s default behaviour is allow opening
any file for reading, but forbidding any attempt to write a file.  See
the example below for an implementation of a less restrictive
\code{r_open()}.
\end{funcdesc}

\begin{funcdesc}{r_reload}{module}
Reload the module object \var{module}, re-parsing and re-initializing it.  
\end{funcdesc}

\begin{funcdesc}{r_unload}{module}
Unload the module object \var{module} (i.e., remove it from the
restricted environment's \code{sys.modules} dictionary).
\end{funcdesc}

And their equivalents with access to restricted standard I/O streams:

\begin{funcdesc}{s_import}{modulename\optional{\, globals, locals, fromlist}}
Import the module \var{modulename}, raising an \code{ImportError}
exception if the module is considered unsafe.
\end{funcdesc}

\begin{funcdesc}{s_reload}{module}
Reload the module object \var{module}, re-parsing and re-initializing it.  
\end{funcdesc}

\begin{funcdesc}{s_unload}{module}
Unload the module object \var{module}.   
% XXX what are the semantics of this?  
\end{funcdesc}

\subsection{An example}

Let us say that we want a slightly more relaxed policy than the
standard RExec class.  For example, if we're willing to allow files in
\file{/tmp} to be written, we can subclass the \code{RExec} class:

\bcode\begin{verbatim}
class TmpWriterRExec(rexec.RExec):
    def r_open(self, file, mode='r', buf=-1):
        if mode in ('r', 'rb'):
            pass
        elif mode in ('w', 'wb', 'a', 'ab'):
            # check filename : must begin with /tmp/
            if file[:5]!='/tmp/': 
                raise IOError, "can't write outside /tmp"
            elif (string.find(file, '/../') >= 0 or
                 file[:3] == '../' or file[-3:] == '/..'):
                raise IOError, "'..' in filename forbidden"
        else: raise IOError, "Illegal open() mode"
        return open(file, mode, buf)
\end{verbatim}\ecode

Notice that the above code will occasionally forbid a perfectly valid
filename; for example, code in the restricted environment won't be
able to open a file called \file{/tmp/foo/../bar}.  To fix this, the
\code{r_open} method would have to simplify the filename to
\file{/tmp/bar}, which would require splitting apart the filename and
performing various operations on it.  In cases where security is at
stake, it may be preferable to write simple code which is sometimes
overly restrictive, instead of more general code that is also more
complex and may harbor a subtle security hole.

\section{Standard Module \sectcode{Bastion}}
\stmodindex{Bastion}
\renewcommand{\indexsubitem}{(in module Bastion)}

% I'm concerned that the word 'bastion' won't be understood by people
% for whom English is a second language, making the module name
% somewhat mysterious.  Thus, the brief definition... --amk

According to the dictionary, a bastion is ``a fortified area or
position'', or ``something that is considered a stronghold.''  It's a
suitable name for this module, which provides a way to forbid access
to certain attributes of an object.  It must always be used with the
\code{rexec} module, in order to allow restricted-mode programs access
to certain safe attributes of an object, while denying access to
other, unsafe attributes.

% I've punted on the issue of documenting keyword arguments for now.

\begin{funcdesc}{Bastion}{object\optional{\, filter\, name\, class}}
Protect the class instance \var{object}, returning a bastion for the
object.  Any attempt to access one of the object's attributes will
have to be approved by the \var{filter} function; if the access is
denied an AttributeError exception will be raised.

If present, \var{filter} must be a function that accepts a string
containing an attribute name, and returns true if access to that
attribute will be permitted; if \var{filter} returns false, the access
is denied.  The default filter denies access to any function beginning
with an underscore (\code{_}).  The bastion's string representation
will be \code{<Bastion for \var{name}>} if a value for
\var{name} is provided; otherwise, \code{repr(\var{object})} will be used.

\var{class}, if present, would be a subclass of \code{BastionClass};
see the code in \file{bastion.py} for the details.  Overriding the
default \code{BastionClass} will rarely be required.  

\end{funcdesc}


\chapter{Multimedia Services}

The modules described in this chapter implement various algorithms or
interfaces that are mainly useful for multimedia applications.  They
are available at the discretion of the installation.  Here's an overview:

\begin{description}

\item[audioop]
--- Manipulate raw audio data.

\item[imageop]
--- Manipulate raw image data.

\item[aifc]
--- Read and write audio files in AIFF or AIFC format.

\item[jpeg]
--- Read and write image files in compressed JPEG format.

\item[rgbimg]
--- Read and write image files in ``SGI RGB'' format (the module is
\emph{not} SGI specific though)!

\end{description}
			% Multimedia Services
\section{Built-in Module \sectcode{audioop}}
\bimodindex{audioop}

The \code{audioop} module contains some useful operations on sound fragments.
It operates on sound fragments consisting of signed integer samples
8, 16 or 32 bits wide, stored in Python strings.  This is the same
format as used by the \code{al} and \code{sunaudiodev} modules.  All
scalar items are integers, unless specified otherwise.

A few of the more complicated operations only take 16-bit samples,
otherwise the sample size (in bytes) is always a parameter of the operation.

The module defines the following variables and functions:

\renewcommand{\indexsubitem}{(in module audioop)}
\begin{excdesc}{error}
This exception is raised on all errors, such as unknown number of bytes
per sample, etc.
\end{excdesc}

\begin{funcdesc}{add}{fragment1\, fragment2\, width}
Return a fragment which is the addition of the two samples passed as
parameters.  \var{width} is the sample width in bytes, either
\code{1}, \code{2} or \code{4}.  Both fragments should have the same
length.
\end{funcdesc}

\begin{funcdesc}{adpcm2lin}{adpcmfragment\, width\, state}
Decode an Intel/DVI ADPCM coded fragment to a linear fragment.  See
the description of \code{lin2adpcm} for details on ADPCM coding.
Return a tuple \code{(\var{sample}, \var{newstate})} where the sample
has the width specified in \var{width}.
\end{funcdesc}

\begin{funcdesc}{adpcm32lin}{adpcmfragment\, width\, state}
Decode an alternative 3-bit ADPCM code.  See \code{lin2adpcm3} for
details.
\end{funcdesc}

\begin{funcdesc}{avg}{fragment\, width}
Return the average over all samples in the fragment.
\end{funcdesc}

\begin{funcdesc}{avgpp}{fragment\, width}
Return the average peak-peak value over all samples in the fragment.
No filtering is done, so the usefulness of this routine is
questionable.
\end{funcdesc}

\begin{funcdesc}{bias}{fragment\, width\, bias}
Return a fragment that is the original fragment with a bias added to
each sample.
\end{funcdesc}

\begin{funcdesc}{cross}{fragment\, width}
Return the number of zero crossings in the fragment passed as an
argument.
\end{funcdesc}

\begin{funcdesc}{findfactor}{fragment\, reference}
Return a factor \var{F} such that
\code{rms(add(fragment, mul(reference, -F)))} is minimal, i.e.,
return the factor with which you should multiply \var{reference} to
make it match as well as possible to \var{fragment}.  The fragments
should both contain 2-byte samples.

The time taken by this routine is proportional to \code{len(fragment)}. 
\end{funcdesc}

\begin{funcdesc}{findfit}{fragment\, reference}
This routine (which only accepts 2-byte sample fragments)

Try to match \var{reference} as well as possible to a portion of
\var{fragment} (which should be the longer fragment).  This is
(conceptually) done by taking slices out of \var{fragment}, using
\code{findfactor} to compute the best match, and minimizing the
result.  The fragments should both contain 2-byte samples.  Return a
tuple \code{(\var{offset}, \var{factor})} where \var{offset} is the
(integer) offset into \var{fragment} where the optimal match started
and \var{factor} is the (floating-point) factor as per
\code{findfactor}.
\end{funcdesc}

\begin{funcdesc}{findmax}{fragment\, length}
Search \var{fragment} for a slice of length \var{length} samples (not
bytes!)\ with maximum energy, i.e., return \var{i} for which
\code{rms(fragment[i*2:(i+length)*2])} is maximal.  The fragments
should both contain 2-byte samples.

The routine takes time proportional to \code{len(fragment)}.
\end{funcdesc}

\begin{funcdesc}{getsample}{fragment\, width\, index}
Return the value of sample \var{index} from the fragment.
\end{funcdesc}

\begin{funcdesc}{lin2lin}{fragment\, width\, newwidth}
Convert samples between 1-, 2- and 4-byte formats.
\end{funcdesc}

\begin{funcdesc}{lin2adpcm}{fragment\, width\, state}
Convert samples to 4 bit Intel/DVI ADPCM encoding.  ADPCM coding is an
adaptive coding scheme, whereby each 4 bit number is the difference
between one sample and the next, divided by a (varying) step.  The
Intel/DVI ADPCM algorithm has been selected for use by the IMA, so it
may well become a standard.

\code{State} is a tuple containing the state of the coder.  The coder
returns a tuple \code{(\var{adpcmfrag}, \var{newstate})}, and the
\var{newstate} should be passed to the next call of lin2adpcm.  In the
initial call \code{None} can be passed as the state.  \var{adpcmfrag}
is the ADPCM coded fragment packed 2 4-bit values per byte.
\end{funcdesc}

\begin{funcdesc}{lin2adpcm3}{fragment\, width\, state}
This is an alternative ADPCM coder that uses only 3 bits per sample.
It is not compatible with the Intel/DVI ADPCM coder and its output is
not packed (due to laziness on the side of the author).  Its use is
discouraged.
\end{funcdesc}

\begin{funcdesc}{lin2ulaw}{fragment\, width}
Convert samples in the audio fragment to U-LAW encoding and return
this as a Python string.  U-LAW is an audio encoding format whereby
you get a dynamic range of about 14 bits using only 8 bit samples.  It
is used by the Sun audio hardware, among others.
\end{funcdesc}

\begin{funcdesc}{minmax}{fragment\, width}
Return a tuple consisting of the minimum and maximum values of all
samples in the sound fragment.
\end{funcdesc}

\begin{funcdesc}{max}{fragment\, width}
Return the maximum of the {\em absolute value} of all samples in a
fragment.
\end{funcdesc}

\begin{funcdesc}{maxpp}{fragment\, width}
Return the maximum peak-peak value in the sound fragment.
\end{funcdesc}

\begin{funcdesc}{mul}{fragment\, width\, factor}
Return a fragment that has all samples in the original framgent
multiplied by the floating-point value \var{factor}.  Overflow is
silently ignored.
\end{funcdesc}

\begin{funcdesc}{reverse}{fragment\, width}
Reverse the samples in a fragment and returns the modified fragment.
\end{funcdesc}

\begin{funcdesc}{rms}{fragment\, width}
Return the root-mean-square of the fragment, i.e.
\iftexi
the square root of the quotient of the sum of all squared sample value,
divided by the sumber of samples.
\else
% in eqn: sqrt { sum S sub i sup 2  over n }
\begin{displaymath}
\catcode`_=8
\sqrt{\frac{\sum{{S_{i}}^{2}}}{n}}
\end{displaymath}
\fi
This is a measure of the power in an audio signal.
\end{funcdesc}

\begin{funcdesc}{tomono}{fragment\, width\, lfactor\, rfactor} 
Convert a stereo fragment to a mono fragment.  The left channel is
multiplied by \var{lfactor} and the right channel by \var{rfactor}
before adding the two channels to give a mono signal.
\end{funcdesc}

\begin{funcdesc}{tostereo}{fragment\, width\, lfactor\, rfactor}
Generate a stereo fragment from a mono fragment.  Each pair of samples
in the stereo fragment are computed from the mono sample, whereby left
channel samples are multiplied by \var{lfactor} and right channel
samples by \var{rfactor}.
\end{funcdesc}

\begin{funcdesc}{ulaw2lin}{fragment\, width}
Convert sound fragments in ULAW encoding to linearly encoded sound
fragments.  ULAW encoding always uses 8 bits samples, so \var{width}
refers only to the sample width of the output fragment here.
\end{funcdesc}

Note that operations such as \code{mul} or \code{max} make no
distinction between mono and stereo fragments, i.e.\ all samples are
treated equal.  If this is a problem the stereo fragment should be split
into two mono fragments first and recombined later.  Here is an example
of how to do that:
\bcode\begin{verbatim}
def mul_stereo(sample, width, lfactor, rfactor):
    lsample = audioop.tomono(sample, width, 1, 0)
    rsample = audioop.tomono(sample, width, 0, 1)
    lsample = audioop.mul(sample, width, lfactor)
    rsample = audioop.mul(sample, width, rfactor)
    lsample = audioop.tostereo(lsample, width, 1, 0)
    rsample = audioop.tostereo(rsample, width, 0, 1)
    return audioop.add(lsample, rsample, width)
\end{verbatim}\ecode

If you use the ADPCM coder to build network packets and you want your
protocol to be stateless (i.e.\ to be able to tolerate packet loss)
you should not only transmit the data but also the state.  Note that
you should send the \var{initial} state (the one you passed to
\code{lin2adpcm}) along to the decoder, not the final state (as returned by
the coder).  If you want to use \code{struct} to store the state in
binary you can code the first element (the predicted value) in 16 bits
and the second (the delta index) in 8.

The ADPCM coders have never been tried against other ADPCM coders,
only against themselves.  It could well be that I misinterpreted the
standards in which case they will not be interoperable with the
respective standards.

The \code{find...} routines might look a bit funny at first sight.
They are primarily meant to do echo cancellation.  A reasonably
fast way to do this is to pick the most energetic piece of the output
sample, locate that in the input sample and subtract the whole output
sample from the input sample:
\bcode\begin{verbatim}
def echocancel(outputdata, inputdata):
    pos = audioop.findmax(outputdata, 800)    # one tenth second
    out_test = outputdata[pos*2:]
    in_test = inputdata[pos*2:]
    ipos, factor = audioop.findfit(in_test, out_test)
    # Optional (for better cancellation):
    # factor = audioop.findfactor(in_test[ipos*2:ipos*2+len(out_test)], 
    #              out_test)
    prefill = '\0'*(pos+ipos)*2
    postfill = '\0'*(len(inputdata)-len(prefill)-len(outputdata))
    outputdata = prefill + audioop.mul(outputdata,2,-factor) + postfill
    return audioop.add(inputdata, outputdata, 2)
\end{verbatim}\ecode

\section{Built-in Module \sectcode{imageop}}
\bimodindex{imageop}

The \code{imageop} module contains some useful operations on images.
It operates on images consisting of 8 or 32 bit pixels
stored in Python strings.  This is the same format as used
by \code{gl.lrectwrite} and the \code{imgfile} module.

The module defines the following variables and functions:

\renewcommand{\indexsubitem}{(in module imageop)}

\begin{excdesc}{error}
This exception is raised on all errors, such as unknown number of bits
per pixel, etc.
\end{excdesc}


\begin{funcdesc}{crop}{image\, psize\, width\, height\, x0\, y0\, x1\, y1}
Return the selected part of \var{image}, which should by
\var{width} by \var{height} in size and consist of pixels of
\var{psize} bytes. \var{x0}, \var{y0}, \var{x1} and \var{y1} are like
the \code{lrectread} parameters, i.e.\ the boundary is included in the
new image.  The new boundaries need not be inside the picture.  Pixels
that fall outside the old image will have their value set to zero.  If
\var{x0} is bigger than \var{x1} the new image is mirrored.  The same
holds for the y coordinates.
\end{funcdesc}

\begin{funcdesc}{scale}{image\, psize\, width\, height\, newwidth\, newheight}
Return \var{image} scaled to size \var{newwidth} by \var{newheight}.
No interpolation is done, scaling is done by simple-minded pixel
duplication or removal.  Therefore, computer-generated images or
dithered images will not look nice after scaling.
\end{funcdesc}

\begin{funcdesc}{tovideo}{image\, psize\, width\, height}
Run a vertical low-pass filter over an image.  It does so by computing
each destination pixel as the average of two vertically-aligned source
pixels.  The main use of this routine is to forestall excessive
flicker if the image is displayed on a video device that uses
interlacing, hence the name.
\end{funcdesc}

\begin{funcdesc}{grey2mono}{image\, width\, height\, threshold}
Convert a 8-bit deep greyscale image to a 1-bit deep image by
tresholding all the pixels.  The resulting image is tightly packed and
is probably only useful as an argument to \code{mono2grey}.
\end{funcdesc}

\begin{funcdesc}{dither2mono}{image\, width\, height}
Convert an 8-bit greyscale image to a 1-bit monochrome image using a
(simple-minded) dithering algorithm.
\end{funcdesc}

\begin{funcdesc}{mono2grey}{image\, width\, height\, p0\, p1}
Convert a 1-bit monochrome image to an 8 bit greyscale or color image.
All pixels that are zero-valued on input get value \var{p0} on output
and all one-value input pixels get value \var{p1} on output.  To
convert a monochrome black-and-white image to greyscale pass the
values \code{0} and \code{255} respectively.
\end{funcdesc}

\begin{funcdesc}{grey2grey4}{image\, width\, height}
Convert an 8-bit greyscale image to a 4-bit greyscale image without
dithering.
\end{funcdesc}

\begin{funcdesc}{grey2grey2}{image\, width\, height}
Convert an 8-bit greyscale image to a 2-bit greyscale image without
dithering.
\end{funcdesc}

\begin{funcdesc}{dither2grey2}{image\, width\, height}
Convert an 8-bit greyscale image to a 2-bit greyscale image with
dithering.  As for \code{dither2mono}, the dithering algorithm is
currently very simple.
\end{funcdesc}

\begin{funcdesc}{grey42grey}{image\, width\, height}
Convert a 4-bit greyscale image to an 8-bit greyscale image.
\end{funcdesc}

\begin{funcdesc}{grey22grey}{image\, width\, height}
Convert a 2-bit greyscale image to an 8-bit greyscale image.
\end{funcdesc}

\section{Standard Module \sectcode{aifc}}
\stmodindex{aifc}

This module provides support for reading and writing AIFF and AIFF-C
files.  AIFF is Audio Interchange File Format, a format for storing
digital audio samples in a file.  AIFF-C is a newer version of the
format that includes the ability to compress the audio data.

Audio files have a number of parameters that describe the audio data.
The sampling rate or frame rate is the number of times per second the
sound is sampled.  The number of channels indicate if the audio is
mono, stereo, or quadro.  Each frame consists of one sample per
channel.  The sample size is the size in bytes of each sample.  Thus a
frame consists of \var{nchannels}*\var{samplesize} bytes, and a
second's worth of audio consists of
\var{nchannels}*\var{samplesize}*\var{framerate} bytes.

For example, CD quality audio has a sample size of two bytes (16
bits), uses two channels (stereo) and has a frame rate of 44,100
frames/second.  This gives a frame size of 4 bytes (2*2), and a
second's worth occupies 2*2*44100 bytes, i.e.\ 176,400 bytes.

Module \code{aifc} defines the following function:

\renewcommand{\indexsubitem}{(in module aifc)}
\begin{funcdesc}{open}{file\, mode}
Open an AIFF or AIFF-C file and return an object instance with
methods that are described below.  The argument file is either a
string naming a file or a file object.  The mode is either the string
\code{'r'} when the file must be opened for reading, or \code{'w'}
when the file must be opened for writing.  When used for writing, the
file object should be seekable, unless you know ahead of time how many
samples you are going to write in total and use
\code{writeframesraw()} and \code{setnframes()}.
\end{funcdesc}

Objects returned by \code{aifc.open()} when a file is opened for
reading have the following methods:

\renewcommand{\indexsubitem}{(aifc object method)}
\begin{funcdesc}{getnchannels}{}
Return the number of audio channels (1 for mono, 2 for stereo).
\end{funcdesc}

\begin{funcdesc}{getsampwidth}{}
Return the size in bytes of individual samples.
\end{funcdesc}

\begin{funcdesc}{getframerate}{}
Return the sampling rate (number of audio frames per second).
\end{funcdesc}

\begin{funcdesc}{getnframes}{}
Return the number of audio frames in the file.
\end{funcdesc}

\begin{funcdesc}{getcomptype}{}
Return a four-character string describing the type of compression used
in the audio file.  For AIFF files, the returned value is
\code{'NONE'}.
\end{funcdesc}

\begin{funcdesc}{getcompname}{}
Return a human-readable description of the type of compression used in
the audio file.  For AIFF files, the returned value is \code{'not
compressed'}.
\end{funcdesc}

\begin{funcdesc}{getparams}{}
Return a tuple consisting of all of the above values in the above
order.
\end{funcdesc}

\begin{funcdesc}{getmarkers}{}
Return a list of markers in the audio file.  A marker consists of a
tuple of three elements.  The first is the mark ID (an integer), the
second is the mark position in frames from the beginning of the data
(an integer), the third is the name of the mark (a string).
\end{funcdesc}

\begin{funcdesc}{getmark}{id}
Return the tuple as described in \code{getmarkers} for the mark with
the given id.
\end{funcdesc}

\begin{funcdesc}{readframes}{nframes}
Read and return the next \var{nframes} frames from the audio file.  The
returned data is a string containing for each frame the uncompressed
samples of all channels.
\end{funcdesc}

\begin{funcdesc}{rewind}{}
Rewind the read pointer.  The next \code{readframes} will start from
the beginning.
\end{funcdesc}

\begin{funcdesc}{setpos}{pos}
Seek to the specified frame number.
\end{funcdesc}

\begin{funcdesc}{tell}{}
Return the current frame number.
\end{funcdesc}

\begin{funcdesc}{close}{}
Close the AIFF file.  After calling this method, the object can no
longer be used.
\end{funcdesc}

Objects returned by \code{aifc.open()} when a file is opened for
writing have all the above methods, except for \code{readframes} and
\code{setpos}.  In addition the following methods exist.  The
\code{get} methods can only be called after the corresponding
\code{set} methods have been called.  Before the first
\code{writeframes} or \code{writeframesraw}, all parameters except for
the number of frames must be filled in.

\begin{funcdesc}{aiff}{}
Create an AIFF file.  The default is that an AIFF-C file is created,
unless the name of the file ends in '.aiff' in which case the default
is an AIFF file.
\end{funcdesc}

\begin{funcdesc}{aifc}{}
Create an AIFF-C file.  The default is that an AIFF-C file is created,
unless the name of the file ends in '.aiff' in which case the default
is an AIFF file.
\end{funcdesc}

\begin{funcdesc}{setnchannels}{nchannels}
Specify the number of channels in the audio file.
\end{funcdesc}

\begin{funcdesc}{setsampwidth}{width}
Specify the size in bytes of audio samples.
\end{funcdesc}

\begin{funcdesc}{setframerate}{rate}
Specify the sampling frequency in frames per second.
\end{funcdesc}

\begin{funcdesc}{setnframes}{nframes}
Specify the number of frames that are to be written to the audio file.
If this parameter is not set, or not set correctly, the file needs to
support seeking.
\end{funcdesc}

\begin{funcdesc}{setcomptype}{type\, name}
Specify the compression type.  If not specified, the audio data will
not be compressed.  In AIFF files, compression is not possible.  The
name parameter should be a human-readable description of the
compression type, the type parameter should be a four-character
string.  Currently the following compression types are supported:
NONE, ULAW, ALAW, G722.
\end{funcdesc}

\begin{funcdesc}{setparams}{nchannels\, sampwidth\, framerate\, comptype\, compname}
Set all the above parameters at once.  The argument is a tuple
consisting of the various parameters.  This means that it is possible
to use the result of a \code{getparams} call as argument to
\code{setparams}.
\end{funcdesc}

\begin{funcdesc}{setmark}{id\, pos\, name}
Add a mark with the given id (larger than 0), and the given name at
the given position.  This method can be called at any time before
\code{close}.
\end{funcdesc}

\begin{funcdesc}{tell}{}
Return the current write position in the output file.  Useful in
combination with \code{setmark}.
\end{funcdesc}

\begin{funcdesc}{writeframes}{data}
Write data to the output file.  This method can only be called after
the audio file parameters have been set.
\end{funcdesc}

\begin{funcdesc}{writeframesraw}{data}
Like \code{writeframes}, except that the header of the audio file is
not updated.
\end{funcdesc}

\begin{funcdesc}{close}{}
Close the AIFF file.  The header of the file is updated to reflect the
actual size of the audio data. After calling this method, the object
can no longer be used.
\end{funcdesc}

\section{Built-in Module \sectcode{jpeg}}
\bimodindex{jpeg}

The module \code{jpeg} provides access to the jpeg compressor and
decompressor written by the Independent JPEG Group. JPEG is a (draft?)\
standard for compressing pictures.  For details on jpeg or the
Independent JPEG Group software refer to the JPEG standard or the
documentation provided with the software.

The \code{jpeg} module defines these functions:

\renewcommand{\indexsubitem}{(in module jpeg)}
\begin{funcdesc}{compress}{data\, w\, h\, b}
Treat data as a pixmap of width \var{w} and height \var{h}, with \var{b} bytes per
pixel.  The data is in SGI GL order, so the first pixel is in the
lower-left corner. This means that \code{lrectread} return data can
immediately be passed to compress.  Currently only 1 byte and 4 byte
pixels are allowed, the former being treated as greyscale and the
latter as RGB color.  Compress returns a string that contains the
compressed picture, in JFIF format.
\end{funcdesc}

\begin{funcdesc}{decompress}{data}
Data is a string containing a picture in JFIF format. It returns a
tuple
\code{(\var{data}, \var{width}, \var{height}, \var{bytesperpixel})}.
Again, the data is suitable to pass to \code{lrectwrite}.
\end{funcdesc}

\begin{funcdesc}{setoption}{name\, value}
Set various options.  Subsequent compress and decompress calls
will use these options.  The following options are available:
\begin{description}
\item[\code{'forcegray' }]
Force output to be grayscale, even if input is RGB.

\item[\code{'quality' }]
Set the quality of the compressed image to a
value between \code{0} and \code{100} (default is \code{75}).  Compress only.

\item[\code{'optimize' }]
Perform Huffman table optimization.  Takes longer, but results in
smaller compressed image.  Compress only.

\item[\code{'smooth' }]
Perform inter-block smoothing on uncompressed image.  Only useful for
low-quality images.  Decompress only.
\end{description}
\end{funcdesc}

Compress and uncompress raise the error \code{jpeg.error} in case of errors.

\section{Built-in Module \sectcode{rgbimg}}
\bimodindex{rgbimg}

The rgbimg module allows python programs to access SGI imglib image
files (also known as \file{.rgb} files).  The module is far from
complete, but is provided anyway since the functionality that there is
is enough in some cases.  Currently, colormap files are not supported.

The module defines the following variables and functions:

\renewcommand{\indexsubitem}{(in module rgbimg)}
\begin{excdesc}{error}
This exception is raised on all errors, such as unsupported file type, etc.
\end{excdesc}

\begin{funcdesc}{sizeofimage}{file}
This function returns a tuple \code{(\var{x}, \var{y})} where
\var{x} and \var{y} are the size of the image in pixels.
Only 4 byte RGBA pixels, 3 byte RGB pixels, and 1 byte greyscale pixels
are currently supported.
\end{funcdesc}

\begin{funcdesc}{longimagedata}{file}
This function reads and decodes the image on the specified file, and
returns it as a Python string. The string has 4 byte RGBA pixels.
The bottom left pixel is the first in
the string. This format is suitable to pass to \code{gl.lrectwrite},
for instance.
\end{funcdesc}

\begin{funcdesc}{longstoimage}{data\, x\, y\, z\, file}
This function writes the RGBA data in \var{data} to image
file \var{file}. \var{x} and \var{y} give the size of the image.
\var{z} is 1 if the saved image should be 1 byte greyscale, 3 if the
saved image should be 3 byte RGB data, or 4 if the saved images should
be 4 byte RGBA data.  The input data always contains 4 bytes per pixel.
These are the formats returned by \code{gl.lrectread}.
\end{funcdesc}

\begin{funcdesc}{ttob}{flag}
This function sets a global flag which defines whether the scan lines
of the image are read or written from bottom to top (flag is zero,
compatible with SGI GL) or from top to bottom(flag is one,
compatible with X)\@.  The default is zero.
\end{funcdesc}

\section{Standard module \sectcode{imghdr}}
\stmodindex{imghdr}

The \code{imghdr} module determines the type of image contained in a
file or byte stream.

The \code{imghdr} module defines the following function:

\renewcommand{\indexsubitem}{(in module imghdr)}

\begin{funcdesc}{what}{filename\optional{\, h}}
Tests the image data contained in the file named by \var{filename},
and returns a string describing the image type.  If optional \var{h}
is provided, the \var{filename} is ignored and \var{h} is assumed to
contain the byte stream to test.
\end{funcdesc}

The following image types are recognized, as listed below with the
return value from \code{what}:

\begin{enumerate}
\item[``rgb''] SGI ImgLib Files

\item[``gif''] GIF 87a and 89a Files

\item[``pbm''] Portable Bitmap Files

\item[``pgm''] Portable Graymap Files

\item[``ppm''] Portable Pixmap Files

\item[``tiff''] TIFF Files

\item[``rast''] Sun Raster Files

\item[``xbm''] X Bitmap Files

\item[``jpeg''] JPEG data in JIFF format
\end{enumerate}

You can extend the list of file types \code{imghdr} can recognize by
appending to this variable:

\begin{datadesc}{tests}
A list of functions performing the individual tests.  Each function
takes two arguments: the byte-stream and an open file-like object.
When \code{what()} is called with a byte-stream, the file-like
object will be \code{None}.

The test function should return a string describing the image type if
the test succeeded, or \code{None} if it failed.
\end{datadesc}

Example:

\begin{verbatim}
>>> import imghdr
>>> imghdr.what('/tmp/bass.gif')
'gif'
\end{verbatim}


\chapter{Cryptographic Services}
\index{cryptography}

The modules described in this chapter implement various algorithms of
a cryptographic nature.  They are available at the discretion of the
installation.  Here's an overview:

\begin{description}

\item[md5]
--- RSA's MD5 message digest algorithm.

\item[mpz]
--- Interface to the GNU MP library for arbitrary precision arithmetic.

\item[rotor]
--- Enigma-like encryption and decryption.

\end{description}

Hardcore cypherpunks will probably find the cryptographic modules
written by Andrew Kuchling of further interest; the package adds
built-in modules for DES and IDEA encryption, provides a Python module
for reading and decrypting PGP files, and then some.  These modules
are not distributed with Python but available separately.  See the URL
\file{http://www.magnet.com/~amk/python/pct.html} or send email to
\file{amk@magnet.com} for more information.
\index{PGP}
\indexii{DES}{cipher}
\indexii{IDEA}{cipher}
\index{cryptography}
		% Cryptographic Services
\section{Built-in Module \sectcode{md5}}
\bimodindex{md5}

This module implements the interface to RSA's MD5 message digest
algorithm (see also Internet RFC 1321).  Its use is quite
straightforward:\ use the \code{md5.new()} to create an md5 object.
You can now feed this object with arbitrary strings using the
\code{update()} method, and at any point you can ask it for the
\dfn{digest} (a strong kind of 128-bit checksum,
a.k.a. ``fingerprint'') of the contatenation of the strings fed to it
so far using the \code{digest()} method.

For example, to obtain the digest of the string {\tt"Nobody inspects
the spammish repetition"}:

\bcode\begin{verbatim}
>>> import md5
>>> m = md5.new()
>>> m.update("Nobody inspects")
>>> m.update(" the spammish repetition")
>>> m.digest()
'\273d\234\203\335\036\245\311\331\336\311\241\215\360\377\351'
\end{verbatim}\ecode

More condensed:

\bcode\begin{verbatim}
>>> md5.new("Nobody inspects the spammish repetition").digest()
'\273d\234\203\335\036\245\311\331\336\311\241\215\360\377\351'
\end{verbatim}\ecode

\renewcommand{\indexsubitem}{(in module md5)}

\begin{funcdesc}{new}{\optional{arg}}
Return a new md5 object.  If \var{arg} is present, the method call
\code{update(\var{arg})} is made.
\end{funcdesc}

\begin{funcdesc}{md5}{\optional{arg}}
For backward compatibility reasons, this is an alternative name for the
\code{new()} function.
\end{funcdesc}

An md5 object has the following methods:

\renewcommand{\indexsubitem}{(md5 method)}
\begin{funcdesc}{update}{arg}
Update the md5 object with the string \var{arg}.  Repeated calls are
equivalent to a single call with the concatenation of all the
arguments, i.e.\ \code{m.update(a); m.update(b)} is equivalent to
\code{m.update(a+b)}.
\end{funcdesc}

\begin{funcdesc}{digest}{}
Return the digest of the strings passed to the \code{update()}
method so far.  This is an 16-byte string which may contain
non-\ASCII{} characters, including null bytes.
\end{funcdesc}

\begin{funcdesc}{copy}{}
Return a copy (``clone'') of the md5 object.  This can be used to
efficiently compute the digests of strings that share a common initial
substring.
\end{funcdesc}

\section{Built-in Module \sectcode{mpz}}
\bimodindex{mpz}

This is an optional module.  It is only available when Python is
configured to include it, which requires that the GNU MP software is
installed.

This module implements the interface to part of the GNU MP library,
which defines arbitrary precision integer and rational number
arithmetic routines.  Only the interfaces to the \emph{integer}
(\samp{mpz_{\rm \ldots}}) routines are provided. If not stated
otherwise, the description in the GNU MP documentation can be applied.

In general, \dfn{mpz}-numbers can be used just like other standard
Python numbers, e.g.\ you can use the built-in operators like \code{+},
\code{*}, etc., as well as the standard built-in functions like
\code{abs}, \code{int}, \ldots, \code{divmod}, \code{pow}.
\strong{Please note:} the {\it bitwise-xor} operation has been implemented as
a bunch of {\it and}s, {\it invert}s and {\it or}s, because the library
lacks an \code{mpz_xor} function, and I didn't need one.

You create an mpz-number by calling the function called \code{mpz} (see
below for an exact description). An mpz-number is printed like this:
\code{mpz(\var{value})}.

\renewcommand{\indexsubitem}{(in module mpz)}
\begin{funcdesc}{mpz}{value}
  Create a new mpz-number. \var{value} can be an integer, a long,
  another mpz-number, or even a string. If it is a string, it is
  interpreted as an array of radix-256 digits, least significant digit
  first, resulting in a positive number. See also the \code{binary}
  method, described below.
\end{funcdesc}

A number of {\em extra} functions are defined in this module. Non
mpz-arguments are converted to mpz-values first, and the functions
return mpz-numbers.

\begin{funcdesc}{powm}{base\, exponent\, modulus}
  Return \code{pow(\var{base}, \var{exponent}) \%{} \var{modulus}}. If
  \code{\var{exponent} == 0}, return \code{mpz(1)}. In contrast to the
  \C-library function, this version can handle negative exponents.
\end{funcdesc}

\begin{funcdesc}{gcd}{op1\, op2}
  Return the greatest common divisor of \var{op1} and \var{op2}.
\end{funcdesc}

\begin{funcdesc}{gcdext}{a\, b}
  Return a tuple \code{(\var{g}, \var{s}, \var{t})}, such that
  \code{\var{a}*\var{s} + \var{b}*\var{t} == \var{g} == gcd(\var{a}, \var{b})}.
\end{funcdesc}

\begin{funcdesc}{sqrt}{op}
  Return the square root of \var{op}. The result is rounded towards zero.
\end{funcdesc}

\begin{funcdesc}{sqrtrem}{op}
  Return a tuple \code{(\var{root}, \var{remainder})}, such that
  \code{\var{root}*\var{root} + \var{remainder} == \var{op}}.
\end{funcdesc}

\begin{funcdesc}{divm}{numerator\, denominator\, modulus}
  Returns a number \var{q}. such that
  \code{\var{q} * \var{denominator} \%{} \var{modulus} == \var{numerator}}.
  One could also implement this function in Python, using \code{gcdext}.
\end{funcdesc}

An mpz-number has one method:

\renewcommand{\indexsubitem}{(mpz method)}
\begin{funcdesc}{binary}{}
  Convert this mpz-number to a binary string, where the number has been
  stored as an array of radix-256 digits, least significant digit first.

  The mpz-number must have a value greater than or equal to zero,
  otherwise a \code{ValueError}-exception will be raised.
\end{funcdesc}

\section{Built-in Module \sectcode{rotor}}
\bimodindex{rotor}

This module implements a rotor-based encryption algorithm, contributed by
Lance Ellinghouse.  The design is derived from the Enigma device, a machine
used during World War II to encipher messages.  A rotor is simply a
permutation.  For example, if the character `A' is the origin of the rotor,
then a given rotor might map `A' to `L', `B' to `Z', `C' to `G', and so on.
To encrypt, we choose several different rotors, and set the origins of the
rotors to known positions; their initial position is the ciphering key.  To
encipher a character, we permute the original character by the first rotor,
and then apply the second rotor's permutation to the result. We continue
until we've applied all the rotors; the resulting character is our
ciphertext.  We then change the origin of the final rotor by one position,
from `A' to `B'; if the final rotor has made a complete revolution, then we
rotate the next-to-last rotor by one position, and apply the same procedure
recursively.  In other words, after enciphering one character, we advance
the rotors in the same fashion as a car's odometer. Decoding works in the
same way, except we reverse the permutations and apply them in the opposite
order.
\index{Ellinghouse, Lance}
\indexii{Enigma}{cipher}

The available functions in this module are:

\renewcommand{\indexsubitem}{(in module rotor)}
\begin{funcdesc}{newrotor}{key\optional{\, numrotors}}
Return a rotor object. \var{key} is a string containing the encryption key
for the object; it can contain arbitrary binary data. The key will be used
to randomly generate the rotor permutations and their initial positions.
\var{numrotors} is the number of rotor permutations in the returned object;
if it is omitted, a default value of 6 will be used.
\end{funcdesc}

Rotor objects have the following methods:

\renewcommand{\indexsubitem}{(rotor method)}
\begin{funcdesc}{setkey}{}
Reset the rotor to its initial state.
\end{funcdesc}

\begin{funcdesc}{encrypt}{plaintext}
Reset the rotor object to its initial state and encrypt \var{plaintext},
returning a string containing the ciphertext.  The ciphertext is always the
same length as the original plaintext.
\end{funcdesc}

\begin{funcdesc}{encryptmore}{plaintext}
Encrypt \var{plaintext} without resetting the rotor object, and return a
string containing the ciphertext.
\end{funcdesc}

\begin{funcdesc}{decrypt}{ciphertext}
Reset the rotor object to its initial state and decrypt \var{ciphertext},
returning a string containing the ciphertext.  The plaintext string will
always be the same length as the ciphertext.
\end{funcdesc}

\begin{funcdesc}{decryptmore}{ciphertext}
Decrypt \var{ciphertext} without resetting the rotor object, and return a
string containing the ciphertext.
\end{funcdesc}

An example usage:
\bcode\begin{verbatim}
>>> import rotor
>>> rt = rotor.newrotor('key', 12)
>>> rt.encrypt('bar')
'\2534\363'
>>> rt.encryptmore('bar')
'\357\375$'
>>> rt.encrypt('bar')
'\2534\363'
>>> rt.decrypt('\2534\363')
'bar'
>>> rt.decryptmore('\357\375$')
'bar'
>>> rt.decrypt('\357\375$')
'l(\315'
>>> del rt
\end{verbatim}\ecode

The module's code is not an exact simulation of the original Enigma device;
it implements the rotor encryption scheme differently from the original. The
most important difference is that in the original Enigma, there were only 5
or 6 different rotors in existence, and they were applied twice to each
character; the cipher key was the order in which they were placed in the
machine.  The Python rotor module uses the supplied key to initialize a
random number generator; the rotor permutations and their initial positions
are then randomly generated.  The original device only enciphered the
letters of the alphabet, while this module can handle any 8-bit binary data;
it also produces binary output.  This module can also operate with an
arbitrary number of rotors.

The original Enigma cipher was broken in 1944. % XXX: Is this right?
The version implemented here is probably a good deal more difficult to crack
(especially if you use many rotors), but it won't be impossible for
a truly skilful and determined attacker to break the cipher.  So if you want
to keep the NSA out of your files, this rotor cipher may well be unsafe, but
for discouraging casual snooping through your files, it will probably be
just fine, and may be somewhat safer than using the Unix \file{crypt}
command.
\index{National Security Agency}\index{crypt(1)}
% XXX How were Unix commands represented in the docs?



%\chapter{Amoeba Specific Services}

\section{Built-in Module \sectcode{amoeba}}

\bimodindex{amoeba}
This module provides some object types and operations useful for
Amoeba applications.  It is only available on systems that support
Amoeba operations.  RPC errors and other Amoeba errors are reported as
the exception \code{amoeba.error = 'amoeba.error'}.

The module \code{amoeba} defines the following items:

\renewcommand{\indexsubitem}{(in module amoeba)}
\begin{funcdesc}{name_append}{path\, cap}
Stores a capability in the Amoeba directory tree.
Arguments are the pathname (a string) and the capability (a capability
object as returned by
\code{name_lookup()}).
\end{funcdesc}

\begin{funcdesc}{name_delete}{path}
Deletes a capability from the Amoeba directory tree.
Argument is the pathname.
\end{funcdesc}

\begin{funcdesc}{name_lookup}{path}
Looks up a capability.
Argument is the pathname.
Returns a
\dfn{capability}
object, to which various interesting operations apply, described below.
\end{funcdesc}

\begin{funcdesc}{name_replace}{path\, cap}
Replaces a capability in the Amoeba directory tree.
Arguments are the pathname and the new capability.
(This differs from
\code{name_append()}
in the behavior when the pathname already exists:
\code{name_append()}
finds this an error while
\code{name_replace()}
allows it, as its name suggests.)
\end{funcdesc}

\begin{datadesc}{capv}
A table representing the capability environment at the time the
interpreter was started.
(Alas, modifying this table does not affect the capability environment
of the interpreter.)
For example,
\code{amoeba.capv['ROOT']}
is the capability of your root directory, similar to
\code{getcap("ROOT")}
in C.
\end{datadesc}

\begin{excdesc}{error}
The exception raised when an Amoeba function returns an error.
The value accompanying this exception is a pair containing the numeric
error code and the corresponding string, as returned by the C function
\code{err_why()}.
\end{excdesc}

\begin{funcdesc}{timeout}{msecs}
Sets the transaction timeout, in milliseconds.
Returns the previous timeout.
Initially, the timeout is set to 2 seconds by the Python interpreter.
\end{funcdesc}

\subsection{Capability Operations}

Capabilities are written in a convenient \ASCII{} format, also used by the
Amoeba utilities
{\it c2a}(U)
and
{\it a2c}(U).
For example:

\bcode\begin{verbatim}
>>> amoeba.name_lookup('/profile/cap')
aa:1c:95:52:6a:fa/14(ff)/8e:ba:5b:8:11:1a
>>> 
\end{verbatim}\ecode

The following methods are defined for capability objects.

\renewcommand{\indexsubitem}{(capability method)}
\begin{funcdesc}{dir_list}{}
Returns a list of the names of the entries in an Amoeba directory.
\end{funcdesc}

\begin{funcdesc}{b_read}{offset\, maxsize}
Reads (at most)
\var{maxsize}
bytes from a bullet file at offset
\var{offset.}
The data is returned as a string.
EOF is reported as an empty string.
\end{funcdesc}

\begin{funcdesc}{b_size}{}
Returns the size of a bullet file.
\end{funcdesc}

\begin{funcdesc}{dir_append}{}
\funcline{dir_delete}{}\ 
\funcline{dir_lookup}{}\ 
\funcline{dir_replace}{}
Like the corresponding
\samp{name_}*
functions, but with a path relative to the capability.
(For paths beginning with a slash the capability is ignored, since this
is the defined semantics for Amoeba.)
\end{funcdesc}

\begin{funcdesc}{std_info}{}
Returns the standard info string of the object.
\end{funcdesc}

\begin{funcdesc}{tod_gettime}{}
Returns the time (in seconds since the Epoch, in UCT, as for POSIX) from
a time server.
\end{funcdesc}

\begin{funcdesc}{tod_settime}{t}
Sets the time kept by a time server.
\end{funcdesc}
		% AMOEBA ONLY

\chapter{Macintosh Specific Services}

The modules in this chapter are available on the Apple Macintosh only.

Aside from the modules described here there are also interfaces to
various MacOS toolboxes, which are currently not extensively
described. The toolboxes for which modules exist are:
\code{AE} (Apple Events),
\code{Cm} (Component Manager),
\code{Ctl} (Control Manager),
\code{Dlg} (Dialog Manager),
\code{Evt} (Event Manager),
\code{Fm} (Font Manager),
\code{List} (List Manager),
\code{Menu} (Moenu Manager),
\code{Qd} (QuickDraw),
\code{Qt} (QuickTime),
\code{Res} (Resource Manager and Handles),
\code{Scrap} (Scrap Manager),
\code{Snd} (Sound Manager),
\code{TE} (TextEdit),
\code{Waste} (non-Apple TextEdit replacement) and
\code{Win} (Window Manager).

If applicable the module will define a number of Python objects for
the various structures declared by the toolbox, and operations will be
implemented as methods of the object. Other operations will be
implemented as functions in the module. Not all operations possible in
C will also be possible in Python (callbacks are often a problem), and
parameters will occasionally be different in Python (input and output
buffers, especially). All methods and functions have a \code{__doc__}
string describing their arguments and return values, and for
additional description you are referred to Inside Mac or similar
works.

\section{Built-in Module \sectcode{mac}}

\bimodindex{mac}
This module provides a subset of the operating system dependent
functionality provided by the optional built-in module \code{posix}.
It is best accessed through the more portable standard module
\code{os}.

The following functions are available in this module:
\code{chdir},
\code{close},
\code{dup},
\code{fdopen},
\code{getcwd},
\code{lseek},
\code{listdir},
\code{mkdir},
\code{open},
\code{read},
\code{rename},
\code{rmdir},
\code{stat},
\code{sync},
\code{unlink},
\code{write},
as well as the exception \code{error}. Note that the times returned by
\code{stat} are floating-point values, like all time values in
MacPython.

One additional function is available: \code{xstat}. This function
returns the same information as \code{stat}, but with three extra
values appended: the size of the resource fork of the file and its
4-char creator and type.

\section{Standard Module \sectcode{macpath}}

\stmodindex{macpath}
This module provides a subset of the pathname manipulation functions
available from the optional standard module \code{posixpath}.  It is
best accessed through the more portable standard module \code{os}, as
\code{os.path}.

The following functions are available in this module:
\code{normcase},
\code{normpath},
\code{isabs},
\code{join},
\code{split},
\code{isdir},
\code{isfile},
\code{walk},
\code{exists}.
For other functions available in \code{posixpath} dummy counterparts
are available.
			% MACINTOSH ONLY
\section{Built-in Module \sectcode{ctb}}
\bimodindex{ctb}
\renewcommand{\indexsubitem}{(in module ctb)}

This module provides a partial interface to the Macintosh
Communications Toolbox. Currently, only Connection Manager tools are
supported.  It may not be available in all Mac Python versions.

\begin{datadesc}{error}
The exception raised on errors.
\end{datadesc}

\begin{datadesc}{cmData}
\dataline{cmCntl}
\dataline{cmAttn}
Flags for the \var{channel} argument of the \var{Read} and \var{Write}
methods.
\end{datadesc}

\begin{datadesc}{cmFlagsEOM}
End-of-message flag for \var{Read} and \var{Write}.
\end{datadesc}

\begin{datadesc}{choose*}
Values returned by \var{Choose}.
\end{datadesc}

\begin{datadesc}{cmStatus*}
Bits in the status as returned by \var{Status}.
\end{datadesc}

\begin{funcdesc}{available}{}
Return 1 if the communication toolbox is available, zero otherwise.
\end{funcdesc}

\begin{funcdesc}{CMNew}{name\, sizes}
Create a connection object using the connection tool named
\var{name}. \var{sizes} is a 6-tuple given buffer sizes for data in,
data out, control in, control out, attention in and attention out.
Alternatively, passing \code{None} will result in default buffer sizes.
\end{funcdesc}

\subsection{connection object}
For all connection methods that take a \var{timeout} argument, a value
of \code{-1} is indefinite, meaning that the command runs to completion.

\renewcommand{\indexsubitem}{(connection object attribute)}

\begin{datadesc}{callback}
If this member is set to a value other than \code{None} it should point
to a function accepting a single argument (the connection
object). This will make all connection object methods work
asynchronously, with the callback routine being called upon
completion.

{\em Note:} for reasons beyond my understanding the callback routine
is currently never called. You are advised against using asynchronous
calls for the time being.
\end{datadesc}


\renewcommand{\indexsubitem}{(connection object method)}

\begin{funcdesc}{Open}{timeout}
Open an outgoing connection, waiting at most \var{timeout} seconds for
the connection to be established.
\end{funcdesc}

\begin{funcdesc}{Listen}{timeout}
Wait for an incoming connection. Stop waiting after \var{timeout}
seconds. This call is only meaningful to some tools.
\end{funcdesc}

\begin{funcdesc}{accept}{yesno}
Accept (when \var{yesno} is non-zero) or reject an incoming call after
\var{Listen} returned.
\end{funcdesc}

\begin{funcdesc}{Close}{timeout\, now}
Close a connection. When \var{now} is zero, the close is orderly
(i.e.\ outstanding output is flushed, etc.)\ with a timeout of
\var{timeout} seconds. When \var{now} is non-zero the close is
immediate, discarding output.
\end{funcdesc}

\begin{funcdesc}{Read}{len\, chan\, timeout}
Read \var{len} bytes, or until \var{timeout} seconds have passed, from
the channel \var{chan} (which is one of \var{cmData}, \var{cmCntl} or
\var{cmAttn}). Return a 2-tuple:\ the data read and the end-of-message
flag.
\end{funcdesc}

\begin{funcdesc}{Write}{buf\, chan\, timeout\, eom}
Write \var{buf} to channel \var{chan}, aborting after \var{timeout}
seconds. When \var{eom} has the value \var{cmFlagsEOM} an
end-of-message indicator will be written after the data (if this
concept has a meaning for this communication tool). The method returns
the number of bytes written.
\end{funcdesc}

\begin{funcdesc}{Status}{}
Return connection status as the 2-tuple \code{(\var{sizes},
\var{flags})}. \var{sizes} is a 6-tuple giving the actual buffer sizes used
(see \var{CMNew}), \var{flags} is a set of bits describing the state
of the connection.
\end{funcdesc}

\begin{funcdesc}{GetConfig}{}
Return the configuration string of the communication tool. These
configuration strings are tool-dependent, but usually easily parsed
and modified.
\end{funcdesc}

\begin{funcdesc}{SetConfig}{str}
Set the configuration string for the tool. The strings are parsed
left-to-right, with later values taking precedence. This means
individual configuration parameters can be modified by simply appending
something like \code{'baud 4800'} to the end of the string returned by
\var{GetConfig} and passing that to this method. The method returns
the number of characters actually parsed by the tool before it
encountered an error (or completed successfully).
\end{funcdesc}

\begin{funcdesc}{Choose}{}
Present the user with a dialog to choose a communication tool and
configure it. If there is an outstanding connection some choices (like
selecting a different tool) may cause the connection to be
aborted. The return value (one of the \var{choose*} constants) will
indicate this.
\end{funcdesc}

\begin{funcdesc}{Idle}{}
Give the tool a chance to use the processor. You should call this
method regularly.
\end{funcdesc}

\begin{funcdesc}{Abort}{}
Abort an outstanding asynchronous \var{Open} or \var{Listen}.
\end{funcdesc}

\begin{funcdesc}{Reset}{}
Reset a connection. Exact meaning depends on the tool.
\end{funcdesc}

\begin{funcdesc}{Break}{length}
Send a break. Whether this means anything, what it means and
interpretation of the \var{length} parameter depend on the tool in
use.
\end{funcdesc}

\section{Built-in Module \sectcode{macconsole}}
\bimodindex{macconsole}

\renewcommand{\indexsubitem}{(in module macconsole)}

This module is available on the Macintosh, provided Python has been
built using the Think C compiler. It provides an interface to the
Think console package, with which basic text windows can be created.

\begin{datadesc}{options}
An object allowing you to set various options when creating windows,
see below.
\end{datadesc}

\begin{datadesc}{C_ECHO}
\dataline{C_NOECHO}
\dataline{C_CBREAK}
\dataline{C_RAW}
Options for the \code{setmode} method. \var{C_ECHO} and \var{C_CBREAK}
enable character echo, the other two disable it, \var{C_ECHO} and
\var{C_NOECHO} enable line-oriented input (erase/kill processing,
etc).
\end{datadesc}

\begin{funcdesc}{copen}{}
Open a new console window. Return a console window object.
\end{funcdesc}

\begin{funcdesc}{fopen}{fp}
Return the console window object corresponding with the given file
object. \var{fp} should be one of \code{sys.stdin}, \code{sys.stdout} or
\code{sys.stderr}.
\end{funcdesc}

\subsection{macconsole options object}
These options are examined when a window is created:

\renewcommand{\indexsubitem}{(macconsole option)}
\begin{datadesc}{top}
\dataline{left}
The origin of the window.
\end{datadesc}

\begin{datadesc}{nrows}
\dataline{ncols}
The size of the window.
\end{datadesc}

\begin{datadesc}{txFont}
\dataline{txSize}
\dataline{txStyle}
The font, fontsize and fontstyle to be used in the window.
\end{datadesc}

\begin{datadesc}{title}
The title of the window.
\end{datadesc}

\begin{datadesc}{pause_atexit}
If set non-zero, the window will wait for user action before closing.
\end{datadesc}

\subsection{console window object}

\renewcommand{\indexsubitem}{(console window attribute)}

\begin{datadesc}{file}
The file object corresponding to this console window. If the file is
buffered, you should call \code{file.flush()} between \code{write()}
and \code{read()} calls.
\end{datadesc}

\renewcommand{\indexsubitem}{(console window method)}

\begin{funcdesc}{setmode}{mode}
Set the input mode of the console to \var{C_ECHO}, etc.
\end{funcdesc}

\begin{funcdesc}{settabs}{n}
Set the tabsize to \var{n} spaces.
\end{funcdesc}

\begin{funcdesc}{cleos}{}
Clear to end-of-screen.
\end{funcdesc}

\begin{funcdesc}{cleol}{}
Clear to end-of-line.
\end{funcdesc}

\begin{funcdesc}{inverse}{onoff}
Enable inverse-video mode:\ characters with the high bit set are
displayed in inverse video (this disables the upper half of a
non-\ASCII{} character set).
\end{funcdesc}

\begin{funcdesc}{gotoxy}{x\, y}
Set the cursor to position \code{(\var{x}, \var{y})}.
\end{funcdesc}

\begin{funcdesc}{hide}{}
Hide the window, remembering the contents.
\end{funcdesc}

\begin{funcdesc}{show}{}
Show the window again.
\end{funcdesc}

\begin{funcdesc}{echo2printer}{}
Copy everything written to the window to the printer as well.
\end{funcdesc}


\section{Built-in Module \sectcode{macdnr}}
\bimodindex{macdnr}

This module provides an interface to the Macintosh Domain Name
Resolver.  It is usually used in conjunction with the \var{mactcp}
module, to map hostnames to IP-addresses.  It may not be available in
all Mac Python versions.

The \code{macdnr} module defines the following functions:

\renewcommand{\indexsubitem}{(in module macdnr)}

\begin{funcdesc}{Open}{\optional{filename}}
Open the domain name resolver extension.  If \var{filename} is given it
should be the pathname of the extension, otherwise a default is
used.  Normally, this call is not needed since the other calls will
open the extension automatically.
\end{funcdesc}

\begin{funcdesc}{Close}{}
Close the resolver extension.  Again, not needed for normal use.
\end{funcdesc}

\begin{funcdesc}{StrToAddr}{hostname}
Look up the IP address for \var{hostname}.  This call returns a dnr
result object of the ``address'' variation.
\end{funcdesc}

\begin{funcdesc}{AddrToName}{addr}
Do a reverse lookup on the 32-bit integer IP-address
\var{addr}.  Returns a dnr result object of the ``address'' variation.
\end{funcdesc}

\begin{funcdesc}{AddrToStr}{addr}
Convert the 32-bit integer IP-address \var{addr} to a dotted-decimal
string.  Returns the string.
\end{funcdesc}

\begin{funcdesc}{HInfo}{hostname}
Query the nameservers for a \code{HInfo} record for host
\var{hostname}.  These records contain hardware and software
information about the machine in question (if they are available in
the first place).  Returns a dnr result object of the ``hinfo''
variety.
\end{funcdesc}

\begin{funcdesc}{MXInfo}{domain}
Query the nameservers for a mail exchanger for \var{domain}.  This is
the hostname of a host willing to accept SMTP mail for the given
domain.  Returns a dnr result object of the ``mx'' variety.
\end{funcdesc}

\subsection{dnr result object}

Since the DNR calls all execute asynchronously you do not get the
results back immediately.  Instead, you get a dnr result object.  You
can check this object to see whether the query is complete, and access
its attributes to obtain the information when it is.

Alternatively, you can also reference the result attributes directly,
this will result in an implicit wait for the query to complete.

The \var{rtnCode} and \var{cname} attributes are always available, the
others depend on the type of query (address, hinfo or mx).

\renewcommand{\indexsubitem}{(dnr result object method)}

% Add args, as in {arg1\, arg2 \optional{\, arg3}}
\begin{funcdesc}{wait}{}
Wait for the query to complete.
\end{funcdesc}

% Add args, as in {arg1\, arg2 \optional{\, arg3}}
\begin{funcdesc}{isdone}{}
Return 1 if the query is complete.
\end{funcdesc}

\renewcommand{\indexsubitem}{(dnr result object attribute)}

\begin{datadesc}{rtnCode}
The error code returned by the query.
\end{datadesc}

\begin{datadesc}{cname}
The canonical name of the host that was queried.
\end{datadesc}

\begin{datadesc}{ip0}
\dataline{ip1}
\dataline{ip2}
\dataline{ip3}
At most four integer IP addresses for this host.  Unused entries are
zero.  Valid only for address queries.
\end{datadesc}

\begin{datadesc}{cpuType}
\dataline{osType}
Textual strings giving the machine type an OS name.  Valid for hinfo
queries.
\end{datadesc}

\begin{datadesc}{exchange}
The name of a mail-exchanger host.  Valid for mx queries.
\end{datadesc}

\begin{datadesc}{preference}
The preference of this mx record.  Not too useful, since the Macintosh
will only return a single mx record.  Mx queries only.
\end{datadesc}

The simplest way to use the module to convert names to dotted-decimal
strings, without worrying about idle time, etc:
\begin{verbatim}
>>> def gethostname(name):
...     import macdnr
...     dnrr = macdnr.StrToAddr(name)
...     return macdnr.AddrToStr(dnrr.ip0)
\end{verbatim}

\section{Built-in Module \sectcode{macfs}}
\bimodindex{macfs}

\renewcommand{\indexsubitem}{(in module macfs)}

This module provides access to macintosh FSSpec handling, the Alias
Manager, finder aliases and the Standard File package.

Whenever a function or method expects a \var{file} argument, this
argument can be one of three things:\ (1) a full or partial Macintosh
pathname, (2) an FSSpec object or (3) a 3-tuple \code{(wdRefNum,
parID, name)} as described in Inside Mac VI\@. A description of aliases
and the standard file package can also be found there.

\begin{funcdesc}{FSSpec}{file}
Create an FSSpec object for the specified file.
\end{funcdesc}

\begin{funcdesc}{RawFSSpec}{data}
Create an FSSpec object given the raw data for the C structure for the
FSSpec as a string.  This is mainly useful if you have obtained an
FSSpec structure over a network.
\end{funcdesc}

\begin{funcdesc}{RawAlias}{data}
Create an Alias object given the raw data for the C structure for the
alias as a string.  This is mainly useful if you have obtained an
FSSpec structure over a network.
\end{funcdesc}

\begin{funcdesc}{FInfo}{}
Create a zero-filled FInfo object.
\end{funcdesc}

\begin{funcdesc}{ResolveAliasFile}{file}
Resolve an alias file. Returns a 3-tuple \code{(\var{fsspec}, \var{isfolder},
\var{aliased})} where \var{fsspec} is the resulting FSSpec object,
\var{isfolder} is true if \var{fsspec} points to a folder and
\var{aliased} is true if the file was an alias in the first place
(otherwise the FSSpec object for the file itself is returned).
\end{funcdesc}

\begin{funcdesc}{StandardGetFile}{\optional{type\, ...}}
Present the user with a standard ``open input file''
dialog. Optionally, you can pass up to four 4-char file types to limit
the files the user can choose from. The function returns an FSSpec
object and a flag indicating that the user completed the dialog
without cancelling.
\end{funcdesc}

\begin{funcdesc}{PromptGetFile}{prompt\optional{\, type\, ...}}
Similar to \var{StandardGetFile} but allows you to specify a prompt.
\end{funcdesc}

\begin{funcdesc}{StandardPutFile}{prompt\, \optional{default}}
Present the user with a standard ``open output file''
dialog. \var{prompt} is the prompt string, and the optional
\var{default} argument initializes the output file name. The function
returns an FSSpec object and a flag indicating that the user completed
the dialog without cancelling.
\end{funcdesc}

\begin{funcdesc}{GetDirectory}{\optional{prompt}}
Present the user with a non-standard ``select a directory''
dialog. \var{prompt} is the prompt string, and the optional.
Return an FSSpec object and a success-indicator.
\end{funcdesc}

\begin{funcdesc}{SetFolder}{\optional{fsspec}}
Set the folder that is initially presented to the user when one of
the file selection dialogs is presented. \var{Fsspec} should point to
a file in the folder, not the folder itself (the file need not exist,
though). If no argument is passed the folder will be set to the
current directory, i.e. what \code{os.getcwd()} returns.

Note that starting with system 7.5 the user can change Standard File
behaviour with the ``general controls'' controlpanel, thereby making
this call inoperative.
\end{funcdesc}

\begin{funcdesc}{FindFolder}{where\, which\, create}
Locates one of the ``special'' folders that MacOS knows about, such as
the trash or the Preferences folder. \var{Where} is the disk to
search, \var{which} is the 4-char string specifying which folder to
locate. Setting \var{create} causes the folder to be created if it
does not exist. Returns a \code{(vrefnum, dirid)} tuple.

The constants for \var{where} and \var{which} can be obtained from the
standard module \var{MACFS}.
\end{funcdesc}

\begin{funcdesc}{FindApplication}{creator}
Locate the application with 4-char creator code \var{creator}. The
function returns an FSSpec object pointing to the application.
\end{funcdesc}

\subsection{FSSpec objects}

\renewcommand{\indexsubitem}{(FSSpec object attribute)}
\begin{datadesc}{data}
The raw data from the FSSpec object, suitable for passing
to other applications, for instance.
\end{datadesc}

\renewcommand{\indexsubitem}{(FSSpec object method)}
\begin{funcdesc}{as_pathname}{}
Return the full pathname of the file described by the FSSpec object.
\end{funcdesc}

\begin{funcdesc}{as_tuple}{}
Return the \code{(\var{wdRefNum}, \var{parID}, \var{name})} tuple of the file described
by the FSSpec object.
\end{funcdesc}

\begin{funcdesc}{NewAlias}{\optional{file}}
Create an Alias object pointing to the file described by this
FSSpec. If the optional \var{file} parameter is present the alias
will be relative to that file, otherwise it will be absolute.
\end{funcdesc}

\begin{funcdesc}{NewAliasMinimal}{}
Create a minimal alias pointing to this file.
\end{funcdesc}

\begin{funcdesc}{GetCreatorType}{}
Return the 4-char creator and type of the file.
\end{funcdesc}

\begin{funcdesc}{SetCreatorType}{creator\, type}
Set the 4-char creator and type of the file.
\end{funcdesc}

\begin{funcdesc}{GetFInfo}{}
Return a FInfo object describing the finder info for the file.
\end{funcdesc}

\begin{funcdesc}{SetFInfo}{finfo}
Set the finder info for the file to the values specified in the
\var{finfo} object.
\end{funcdesc}

\begin{funcdesc}{GetDates}{}
Return a tuple with three floating point values representing the
creation date, modification date and backup date of the file.
\end{funcdesc}

\begin{funcdesc}{SetDates}{crdate\, moddate\, backupdate}
Set the creation, modification and backup date of the file. The values
are in the standard floating point format used for times throughout
Python.
\end{funcdesc}

\subsection{alias objects}

\renewcommand{\indexsubitem}{(alias object attribute)}
\begin{datadesc}{data}
The raw data for the Alias record, suitable for storing in a resource
or transmitting to other programs.
\end{datadesc}

\renewcommand{\indexsubitem}{(alias object method)}
\begin{funcdesc}{Resolve}{\optional{file}}
Resolve the alias. If the alias was created as a relative alias you
should pass the file relative to which it is. Return the FSSpec for
the file pointed to and a flag indicating whether the alias object
itself was modified during the search process. 
\end{funcdesc}

\begin{funcdesc}{GetInfo}{num}
An interface to the C routine \code{GetAliasInfo()}.
\end{funcdesc}

\begin{funcdesc}{Update}{file\, \optional{file2}}
Update the alias to point to the \var{file} given. If \var{file2} is
present a relative alias will be created.
\end{funcdesc}

Note that it is currently not possible to directly manipulate a resource
as an alias object. Hence, after calling \var{Update} or after
\var{Resolve} indicates that the alias has changed the Python program
is responsible for getting the \var{data} from the alias object and
modifying the resource.


\subsection{FInfo objects}

See Inside Mac for a complete description of what the various fields
mean.

\renewcommand{\indexsubitem}{(FInfo object attribute)}
\begin{datadesc}{Creator}
The 4-char creator code of the file.
\end{datadesc}

\begin{datadesc}{Type}
The 4-char type code of the file.
\end{datadesc}

\begin{datadesc}{Flags}
The finder flags for the file as 16-bit integer. The bit values in
\var{Flags} are defined in standard module \var{MACFS}.
\end{datadesc}

\begin{datadesc}{Location}
A Point giving the position of the file's icon in its folder.
\end{datadesc}

\begin{datadesc}{Fldr}
The folder the file is in (as an integer).
\end{datadesc}

\section{Built-in Module \sectcode{MacOS}}
\bimodindex{MacOS}

\renewcommand{\indexsubitem}{(in module MacOS)}

This module provides access to MacOS specific functionality in the
python interpreter, such as how the interpreter eventloop functions
and the like. Use with care.

Note the capitalisation of the module name, this is a historical
artefact.

\begin{excdesc}{Error}
This exception is raised on MacOS generated errors, either from
functions in this module or from other mac-specific modules like the
toolbox interfaces. The arguments are the integer error code (the
\var{OSErr} value) and a textual description of the error code.
Symbolic names for all known error codes are defined in the standard
module \var{macerrors}.
\end{excdesc}

\begin{funcdesc}{SetHighLevelEventHandler}{handler}
Pass a python function that will be called upon reception of a
high-level event. The previous handler is returned. The handler
function is called with the event as argument.

Note that your event handler is currently only called dependably if
your main event loop is in \var{stdwin}.
\end{funcdesc}

\begin{funcdesc}{AcceptHighLevelEvent}{}
Read a high-level event. The return value is a tuple \code{(sender,
refcon, data)}.
\end{funcdesc}

\begin{funcdesc}{SetScheduleTimes}{fgi\, fgy \optional{\, bgi\, bgy}}
Controls how often the interpreter checks the event queue and how
long it will yield the processor to other processes. \var{fgi}
specifies after how many clicks (one click is one 60th of a second)
the interpreter should check the event queue, and \var{fgy} specifies
for how many clicks the CPU should be yielded when in the
foreground. The optional \var{bgi} and \var{bgy} allow you to specify
different values to use when python runs in the background, otherwise
the background values will be set the the same as the foreground
values. The function returns nothing.

The default values, which are based on minimal empirical testing, are 12, 1, 6
and 2 respectively.
\end{funcdesc}

\begin{funcdesc}{EnableAppswitch}{onoff}
Enable or disable the python event loop, based on the value of
\var{onoff}. The old value is returned. If the event loop is disabled
no time is granted to other applications, checking for command-period
is not performed and it is impossible to switch applications. This
should only be used by programs providing their own complete event
loop.

Note that based on the compiler used to build python it is still
possible to loose events even with the python event loop disabled. If
you have a \code{sys.stdout} window its handler will often also look
in the event queue. Making sure nothing is ever printed works around
this.
\end{funcdesc}

\begin{funcdesc}{HandleEvent}{ev}
Pass the event record \code{ev} back to the python event loop, or
possibly to the handler for the \code{sys.stdout} window (based on the
compiler used to build python). This allows python programs that do
their own event handling to still have some command-period and
window-switching capability.
\end{funcdesc}

\begin{funcdesc}{GetErrorString}{errno}
Return the textual description of MacOS error code \var{errno}.
\end{funcdesc}

\begin{funcdesc}{splash}{resid}
This function will put a splash window
on-screen, with the contents of the DLOG resource specified by
\code{resid}. Calling with a zero argument will remove the splash
screen. This function is useful if you want an applet to post a splash screen
early in initialization without first having to load numerous
extension modules.
\end{funcdesc}

\begin{funcdesc}{DebugStr}{message \optional{\, object}}
Drop to the low-level debugger with message \var{message}. The
optional \var{object} argument is not used, but can easily be
inspected from the debugger.

Note that you should use this function with extreme care: if no
low-level debugger like MacsBug is installed this call will crash your
system. It is intended mainly for developers of Python extension
modules.
\end{funcdesc}

\begin{funcdesc}{openrf}{name \optional{\, mode}}
Open the resource fork of a file. Arguments are the same as for the
builtin function \code{open}. The object returned has file-like
semantics, but it is not a python file object, so there may be subtle
differences.
\end{funcdesc}


\section{Standard module \sectcode{macostools}}
\stmodindex{macostools}

This module contains some convenience routines for file-manipulation
on the Macintosh.

The \code{macostools} module defines the following functions:

\renewcommand{\indexsubitem}{(in module macostools)}

\begin{funcdesc}{copy}{src\, dst\optional{\, createpath, copytimes}}
Copy file \var{src} to \var{dst}. The files can be specified as
pathnames or \code{FSSpec} objects. If \var{createpath} is non-zero
\var{dst} must be a pathname and the folders leading to the
destination are created if necessary.  The method copies data and
resource fork and some finder information (creator, type, flags) and
optionally the creation, modification and backup times (default is to
copy them). Custom icons, comments and icon position are not copied.

If the source is an alias the original to which the alias points is
copied, not the aliasfile.
\end{funcdesc}

\begin{funcdesc}{copytree}{src\, dst}
Recursively copy a file tree from \var{src} to \var{dst}, creating
folders as needed. \var{Src} and \var{dst} should be specified as
pathnames.
\end{funcdesc}

\begin{funcdesc}{mkalias}{src\, dst}
Create a finder alias \var{dst} pointing to \var{src}. Both may be
specified as pathnames or \var{FSSpec} objects.
\end{funcdesc}

\begin{funcdesc}{touched}{dst}
Tell the finder that some bits of finder-information such as creator
or type for file \var{dst} has changed. The file can be specified by
pathname or fsspec. This call should prod the finder into redrawing the
files icon.
\end{funcdesc}

\begin{datadesc}{BUFSIZ}
The buffer size for \code{copy}, default 1 megabyte.
\end{datadesc}

Note that the process of creating finder aliases is not specified in
the Apple documentation. Hence, aliases created with \code{mkalias}
could conceivably have incompatible behaviour in some cases.

\section{Standard module \sectcode{findertools}}
\stmodindex{findertools}

This module contains routines that give Python programs access to some
functionality provided by the finder. They are implemented as wrappers
around the AppleEvent interface to the finder.

All file and folder parameters can be specified either as full
pathnames or as \code{FSSpec} objects.

The \code{findertools} module defines the following functions:

\renewcommand{\indexsubitem}{(in module macostools)}

\begin{funcdesc}{launch}{file}
Tell the finder to launch \var{file}. What launching means depends on the file:
applications are started, folders are opened and documents are opened
in the correct application.
\end{funcdesc}

\begin{funcdesc}{Print}{file}
Tell the finder to print a file (again specified by full pathname or
FSSpec). The behaviour is identical to selecting the file and using
the print command in the finder.
\end{funcdesc}

\begin{funcdesc}{copy}{file, destdir}
Tell the finder to copy a file or folder \var{file} to folder
\var{destdir}. The function returns an \code{Alias} object pointing to
the new file.
\end{funcdesc}

\begin{funcdesc}{move}{file, destdir}
Tell the finder to move a file or folder \var{file} to folder
\var{destdir}. The function returns an \code{Alias} object pointing to
the new file.
\end{funcdesc}

\begin{funcdesc}{sleep}{}
Tell the finder to put the mac to sleep, if your machine supports it.
\end{funcdesc}

\begin{funcdesc}{restart}{}
Tell the finder to perform an orderly restart of the machine.
\end{funcdesc}

\begin{funcdesc}{shutdown}{}
Tell the finder to perform an orderly shutdown of the machine.
\end{funcdesc}

\section{Built-in Module \sectcode{mactcp}}
\bimodindex{mactcp}

\renewcommand{\indexsubitem}{(in module mactcp)}

This module provides an interface to the Macintosh TCP/IP driver
MacTCP\@. There is an accompanying module \code{macdnr} which provides an
interface to the name-server (allowing you to translate hostnames to
ip-addresses), a module \code{MACTCPconst} which has symbolic names for
constants constants used by MacTCP. Since the builtin module
\code{socket} is also available on the mac it is usually easier to use
sockets in stead of the mac-specific MacTCP API.

A complete description of the MacTCP interface can be found in the
Apple MacTCP API documentation.

\begin{funcdesc}{MTU}{}
Return the Maximum Transmit Unit (the packet size) of the network
interface.
\end{funcdesc}

\begin{funcdesc}{IPAddr}{}
Return the 32-bit integer IP address of the network interface.
\end{funcdesc}

\begin{funcdesc}{NetMask}{}
Return the 32-bit integer network mask of the interface.
\end{funcdesc}

\begin{funcdesc}{TCPCreate}{size}
Create a TCP Stream object. \var{size} is the size of the receive
buffer, \code{4096} is suggested by various sources.
\end{funcdesc}

\begin{funcdesc}{UDPCreate}{size, port}
Create a UDP stream object. \var{size} is the size of the receive
buffer (and, hence, the size of the biggest datagram you can receive
on this port). \var{port} is the UDP port number you want to receive
datagrams on, a value of zero will make MacTCP select a free port.
\end{funcdesc}

\subsection{TCP Stream Objects}

\renewcommand{\indexsubitem}{(TCP stream attribute)}

\begin{datadesc}{asr}
When set to a value different than \code{None} this should point to a
function with two integer parameters:\ an event code and a detail. This
function will be called upon network-generated events such as urgent
data arrival. In addition, it is called with eventcode
\code{MACTCP.PassiveOpenDone} when a \code{PassiveOpen} completes. This
is a Python addition to the MacTCP semantics.
It is safe to do further calls from the \code{asr}.
\end{datadesc}

\renewcommand{\indexsubitem}{(TCP stream method)}

\begin{funcdesc}{PassiveOpen}{port}
Wait for an incoming connection on TCP port \var{port} (zero makes the
system pick a free port). The call returns immediately, and you should
use \var{wait} to wait for completion. You should not issue any method
calls other than
\code{wait}, \code{isdone} or \code{GetSockName} before the call
completes.
\end{funcdesc}

\begin{funcdesc}{wait}{}
Wait for \code{PassiveOpen} to complete.
\end{funcdesc}

\begin{funcdesc}{isdone}{}
Return 1 if a \code{PassiveOpen} has completed.
\end{funcdesc}

\begin{funcdesc}{GetSockName}{}
Return the TCP address of this side of a connection as a 2-tuple
\code{(host, port)}, both integers.
\end{funcdesc}

\begin{funcdesc}{ActiveOpen}{lport\, host\, rport}
Open an outgoing connection to TCP address \code{(\var{host}, \var{rport})}. Use
local port \var{lport} (zero makes the system pick a free port). This
call blocks until the connection has been established.
\end{funcdesc}

\begin{funcdesc}{Send}{buf\, push\, urgent}
Send data \var{buf} over the connection. \var{Push} and \var{urgent}
are flags as specified by the TCP standard.
\end{funcdesc}

\begin{funcdesc}{Rcv}{timeout}
Receive data. The call returns when \var{timeout} seconds have passed
or when (according to the MacTCP documentation) ``a reasonable amount
of data has been received''. The return value is a 3-tuple
\code{(\var{data}, \var{urgent}, \var{mark})}. If urgent data is outstanding \code{Rcv}
will always return that before looking at any normal data. The first
call returning urgent data will have the \var{urgent} flag set, the
last will have the \var{mark} flag set.
\end{funcdesc}

\begin{funcdesc}{Close}{}
Tell MacTCP that no more data will be transmitted on this
connection. The call returns when all data has been acknowledged by
the receiving side.
\end{funcdesc}

\begin{funcdesc}{Abort}{}
Forcibly close both sides of a connection, ignoring outstanding data.
\end{funcdesc}

\begin{funcdesc}{Status}{}
Return a TCP status object for this stream giving the current status
(see below).
\end{funcdesc}

\subsection{TCP Status Objects}
This object has no methods, only some members holding information on
the connection. A complete description of all fields in this objects
can be found in the Apple documentation. The most interesting ones are:

\renewcommand{\indexsubitem}{(TCP status attribute)}

\begin{datadesc}{localHost}
\dataline{localPort}
\dataline{remoteHost}
\dataline{remotePort}
The integer IP-addresses and port numbers of both endpoints of the
connection. 
\end{datadesc}

\begin{datadesc}{sendWindow}
The current window size.
\end{datadesc}

\begin{datadesc}{amtUnackedData}
The number of bytes sent but not yet acknowledged. \code{sendWindow -
amtUnackedData} is what you can pass to \code{Send} without blocking.
\end{datadesc}

\begin{datadesc}{amtUnreadData}
The number of bytes received but not yet read (what you can \code{Recv}
without blocking).
\end{datadesc}



\subsection{UDP Stream Objects}
Note that, unlike the name suggests, there is nothing stream-like
about UDP.

\renewcommand{\indexsubitem}{(UDP stream attribute)}

\begin{datadesc}{asr}
The asynchronous service routine to be called on events such as
datagram arrival without outstanding \code{Read} call. The \code{asr} has a
single argument, the event code.
\end{datadesc}

\begin{datadesc}{port}
A read-only member giving the port number of this UDP stream.
\end{datadesc}

\renewcommand{\indexsubitem}{(UDP stream method)}

\begin{funcdesc}{Read}{timeout}
Read a datagram, waiting at most \var{timeout} seconds ($-1$ is
infinite).  Return the data.
\end{funcdesc}

\begin{funcdesc}{Write}{host\, port\, buf}
Send \var{buf} as a datagram to IP-address \var{host}, port
\var{port}.
\end{funcdesc}

\section{Built-in Module \sectcode{macspeech}}
\bimodindex{macspeech}

\renewcommand{\indexsubitem}{(in module macspeech)}

This module provides an interface to the Macintosh Speech Manager,
allowing you to let the Macintosh utter phrases. You need a version of
the speech manager extension (version 1 and 2 have been tested) in
your \code{Extensions} folder for this to work. The module does not
provide full access to all features of the Speech Manager yet.  It may
not be available in all Mac Python versions.

\begin{funcdesc}{Available}{}
Test availability of the Speech Manager extension (and, on the
PowerPC, the Speech Manager shared library). Return 0 or 1. 
\end{funcdesc}

\begin{funcdesc}{Version}{}
Return the (integer) version number of the Speech Manager.
\end{funcdesc}

\begin{funcdesc}{SpeakString}{str}
Utter the string \var{str} using the default voice,
asynchronously. This aborts any speech that may still be active from
prior \code{SpeakString} invocations.
\end{funcdesc}

\begin{funcdesc}{Busy}{}
Return the number of speech channels busy, system-wide.
\end{funcdesc}

\begin{funcdesc}{CountVoices}{}
Return the number of different voices available.
\end{funcdesc}

\begin{funcdesc}{GetIndVoice}{num}
Return a voice object for voice number \var{num}.
\end{funcdesc}

\subsection{voice objects}
Voice objects contain the description of a voice. It is currently not
yet possible to access the parameters of a voice.

\renewcommand{\indexsubitem}{(voice object method)}

\begin{funcdesc}{GetGender}{}
Return the gender of the voice:\ 0 for male, 1 for female and $-1$ for neuter.
\end{funcdesc}

\begin{funcdesc}{NewChannel}{}
Return a new speech channel object using this voice.
\end{funcdesc}

\subsection{speech channel objects}
A speech channel object allows you to speak strings with slightly more
control than \code{SpeakString()}, and allows you to use multiple
speakers at the same time. Please note that channel pitch and rate are
interrelated in some way, so that to make your Macintosh sing you will
have to adjust both.

\renewcommand{\indexsubitem}{(speech channel object method)}
\begin{funcdesc}{SpeakText}{str}
Start uttering the given string.
\end{funcdesc}

\begin{funcdesc}{Stop}{}
Stop babbling.
\end{funcdesc}

\begin{funcdesc}{GetPitch}{}
Return the current pitch of the channel, as a floating-point number.
\end{funcdesc}

\begin{funcdesc}{SetPitch}{pitch}
Set the pitch of the channel.
\end{funcdesc}

\begin{funcdesc}{GetRate}{}
Get the speech rate (utterances per minute) of the channel as a
floating point number.
\end{funcdesc}

\begin{funcdesc}{SetRate}{rate}
Set the speech rate of the channel.
\end{funcdesc}


\section{Standard module \sectcode{EasyDialogs}}
\stmodindex{EasyDialogs}

The \code{EasyDialogs} module contains some simple dialogs for
the Macintosh, modelled after the \code{stdwin} dialogs with similar
names.

The \code{EasyDialogs} module defines the following functions:

\renewcommand{\indexsubitem}{(in module EasyDialogs)}

\begin{funcdesc}{Message}{str}
A modal dialog with the message text \var{str}, which should be at
most 255 characters long, is displayed. Control is returned when the
user clicks ``OK''.
\end{funcdesc}

\begin{funcdesc}{AskString}{prompt\optional{\, default}}
Ask the user to input a string value, in a modal dialog. \var{Prompt}
is the promt message, the optional \var{default} arg is the initial
value for the string. All strings can be at most 255 bytes
long. \var{AskString} returns the string entered or \code{None} in
case the user cancelled.
\end{funcdesc}

\begin{funcdesc}{AskYesNoCancel}{question\optional{\, default}}
Present a dialog with text \var{question} and three buttons labelled
``yes'', ``no'' and ``cancel''. Return \code{1} for yes, \code{0} for
no and \code{-1} for cancel. The default return value chosen by
hitting return is \code{0}. This can be changed with the optional
\var{default} argument.
\end{funcdesc}

\begin{funcdesc}{ProgressBar}{\optional{label\, maxval}}
Display a modeless progress dialog with a thermometer bar. \var{Label}
is the textstring displayed (default ``Working...''), \var{maxval} is
the value at which progress is complete (default 100). The returned
object has one method, \code{set(value)}, which sets the value of the
progress bar. The bar remains visible until the object returned is
discarded.

The progress bar has a ``cancel'' button, but it is currently
non-functional.
\end{funcdesc}

Note that \code{EasyDialogs} does not currently use the notification
manager. This means that displaying dialogs while the program is in
the background will lead to unexpected results and possibly
crashes. Also, all dialogs are modeless and hence expect to be at the
top of the stacking order. This is true when the dialogs are created,
but windows that pop-up later (like a console window) may also result
in crashes.


\section{Standard module \sectcode{FrameWork}}
\stmodindex{FrameWork}

The \code{FrameWork} module contains classes that together provide a
framework for an interactive Macintosh application. The programmer
builds an application by creating subclasses that override various
methods of the bases classes, thereby implementing the functionality
wanted. Overriding functionality can often be done on various
different levels, i.e. to handle clicks in a single dialog window in a
non-standard way it is not necessary to override the complete event
handling.

The \code{FrameWork} is still very much work-in-progress, and the
documentation describes only the most important functionality, and not
in the most logical manner at that. Examine the source or the examples
for more details.

The \code{FrameWork} module defines the following functions:

\renewcommand{\indexsubitem}{(in module FrameWork)}

\begin{funcdesc}{Application}{}
An object representing the complete application. See below for a
description of the methods. The default \code{__init__} routine
creates an empty window dictionary and a menu bar with an apple menu.
\end{funcdesc}

\begin{funcdesc}{MenuBar}{}
An object representing the menubar. This object is usually not created
by the user.
\end{funcdesc}

\begin{funcdesc}{Menu}{bar\, title\optional{\, after}}
An object representing a menu. Upon creation you pass the
\code{MenuBar} the menu appears in, the \var{title} string and a
position (1-based) \var{after} where the menu should appear (default:
at the end).
\end{funcdesc}

\begin{funcdesc}{MenuItem}{menu\, title\optional{\, shortcut\, callback}}
Create a menu item object. The arguments are the menu to crate the
item it, the item title string and optionally the keyboard shortcut
and a callback routine. The callback is called with the arguments
menu-id, item number within menu (1-based), current front window and
the event record.

In stead of a callable object the callback can also be a string. In
this case menu selection causes the lookup of a method in the topmost
window and the application. The method name is the callback string
with \code{'domenu_'} prepended.

Calling the \code{MenuBar} \code{fixmenudimstate} method sets the
correct dimming for all menu items based on the current front window.
\end{funcdesc}

\begin{funcdesc}{Separator}{menu}
Add a separator to the end of a menu.
\end{funcdesc}

\begin{funcdesc}{SubMenu}{menu\, label}
Create a submenu named \var{label} under menu \var{menu}. The menu
object is returned.
\end{funcdesc}

\begin{funcdesc}{Window}{parent}
Creates a (modeless) window. \var{Parent} is the application object to
which the window belongs. The window is not displayed until later.
\end{funcdesc}

\begin{funcdesc}{DialogWindow}{parent}
Creates a modeless dialog window.
\end{funcdesc}

\begin{funcdesc}{windowbounds}{width\, height}
Return a \code{(left, top, right, bottom)} tuple suitable for creation
of a window of given width and height. The window will be staggered
with respect to previous windows, and an attempt is made to keep the
whole window on-screen. The window will however always be exact the
size given, so parts may be offscreen.
\end{funcdesc}

\begin{funcdesc}{setwatchcursor}{}
Set the mouse cursor to a watch.
\end{funcdesc}

\begin{funcdesc}{setarrowcursor}{}
Set the mouse cursor to an arrow.
\end{funcdesc}

\subsection{Application objects}
Application objects have the following methods, among others:

\renewcommand{\indexsubitem}{(Application method)}

\begin{funcdesc}{makeusermenus}{}
Override this method if you need menus in your application. Append the
menus to \code{self.menubar}.
\end{funcdesc}

\begin{funcdesc}{getabouttext}{}
Override this method to return a text string describing your
application. Alternatively, override the \code{do_about} method for
more elaborate about messages.
\end{funcdesc}

\begin{funcdesc}{mainloop}{\optional{mask\, wait}}
This routine is the main event loop, call it to set your application
rolling. \var{Mask} is the mask of events you want to handle,
\var{wait} is the number of ticks you want to leave to other
concurrent application (default 0, which is probably not a good
idea). While raising \code{self} to exit the mainloop is still
supported it is not recommended, call \code{self._quit} instead.

The event loop is split into many small parts, each of which can be
overridden. The default methods take care of dispatching events to
windows and dialogs, handling drags and resizes, Apple Events, events
for non-FrameWork windows, etc.
\end{funcdesc}

\begin{funcdesc}{_quit}{}
Terminate the event \code{mainloop} at the next convenient moment.
\end{funcdesc}

\begin{funcdesc}{do_char}{c\, event}
The user typed character \var{c}. The complete details of the event
can be found in the \var{event} structure. This method can also be
provided in a \code{Window} object, which overrides the
application-wide handler if the window is frontmost.
\end{funcdesc}

\begin{funcdesc}{do_dialogevent}{event}
Called early in the event loop to handle modeless dialog events. The
default method simply dispatches the event to the relevant dialog (not
through the the \code{DialogWindow} object involved). Override if you
need special handling of dialog events (keyboard shortcuts, etc).
\end{funcdesc}

\begin{funcdesc}{idle}{event}
Called by the main event loop when no events are available. The
null-event is passed (so you can look at mouse position, etc).
\end{funcdesc}

\subsection{Window Objects}

Window objects have the following methods, among others:

\renewcommand{\indexsubitem}{(Window method)}

\begin{funcdesc}{open}{}
Override this method to open a window. Store the MacOS window-id in
\code{self.wid} and call \code{self.do_postopen} to register the
window with the parent application.
\end{funcdesc}

\begin{funcdesc}{close}{}
Override this method to do any special processing on window
close. Call \code{self.do_postclose} to cleanup the parent state.
\end{funcdesc}

\begin{funcdesc}{do_postresize}{width\, height\, macoswindowid}
Called after the window is resized. Override if more needs to be done
than calling \code{InvalRect}.
\end{funcdesc}

\begin{funcdesc}{do_contentclick}{local\, modifiers\, event}
The user clicked in the content part of a window. The arguments are
the coordinates (window-relative), the key modifiers and the raw
event.
\end{funcdesc}

\begin{funcdesc}{do_update}{macoswindowid\, event}
An update event for the window was received. Redraw the window.
\end{funcdesc}

\begin{funcdesc}{do_activate}{activate\, event}
The window was activated (\code{activate==1}) or deactivated
(\code{activate==0}). Handle things like focus highlighting, etc.
\end{funcdesc}

\subsection{ControlsWindow Object}

ControlsWindow objects have the following methods besides those of
\code{Window} objects:

\renewcommand{\indexsubitem}{(ControlsWindow method)}

\begin{funcdesc}{do_controlhit}{window\, control\, pcode\, event}
Part \code{pcode} of control \code{control} was hit by the
user. Tracking and such has already been taken care of.
\end{funcdesc}

\subsection{ScrolledWindow Object}

ScrolledWindow objects are ControlsWindow objects with the following
extra methods:

\renewcommand{\indexsubitem}{(ScrolledWindow method)}

\begin{funcdesc}{scrollbars}{\optional{wantx\, wanty}}
Create (or destroy) horizontal and vertical scrollbars. The arguments
specify which you want (default: both). The scrollbars always have
minimum \code{0} and maximum \code{32767}.
\end{funcdesc}

\begin{funcdesc}{getscrollbarvalues}{}
You must supply this method. It should return a tuple \code{x, y}
giving the current position of the scrollbars (between \code{0} and
\code{32767}). You can return \code{None} for either to indicate the
whole document is visible in that direction.
\end{funcdesc}

\begin{funcdesc}{updatescrollbars}{}
Call this method when the document has changed. It will call
\code{getscrollbarvalues} and update the scrollbars.
\end{funcdesc}

\begin{funcdesc}{scrollbar_callback}{which\, what\, value}
Supplied by you and called after user interaction. \code{Which} will
be \code{'x'} or \code{'y'}, \code{what} will be \code{'-'},
\code{'--'}, \code{'set'}, \code{'++'} or \code{'+'}. For
\code{'set'}, \code{value} will contain the new scrollbar position.
\end{funcdesc}

\begin{funcdesc}{scalebarvalues}{absmin\, absmax\, curmin\, curmax}
Auxiliary method to help you calculate values to return from
\code{getscrollbarvalues}. You pass document minimum and maximum value
and topmost (leftmost) and bottommost (rightmost) visible values and
it returns the correct number or \code{None}.
\end{funcdesc}

\begin{funcdesc}{do_activate}{onoff\, event}
Takes care of dimming/highlighting scrollbars when a window becomes
frontmost vv. If you override this method call this one at the end of
your method.
\end{funcdesc}

\begin{funcdesc}{do_postresize}{width\, height\, window}
Moves scrollbars to the correct position. Call this method initially
if you override it.
\end{funcdesc}

\begin{funcdesc}{do_controlhit}{window\, control\, pcode\, event}
Handles scrollbar interaction. If you override it call this method
first, a nonzero return value indicates the hit was in the scrollbars
and has been handled.
\end{funcdesc}

\subsection{DialogWindow Objects}

DialogWindow objects have the following methods besides those of
\code{Window} objects:

\renewcommand{\indexsubitem}{(DialogWindow method)}

\begin{funcdesc}{open}{resid}
Create the dialog window, from the DLOG resource with id
\var{resid}. The dialog object is stored in \code{self.wid}.
\end{funcdesc}

\begin{funcdesc}{do_itemhit}{item\, event}
Item number \var{item} was hit. You are responsible for redrawing
toggle buttons, etc.
\end{funcdesc}

\section{Standard module \sectcode{MiniAEFrame}}
\stmodindex{MiniAEFrame}

The module \var{MiniAEFrame} provides a framework for an application
that can function as an OSA server, i.e. receive and process
AppleEvents. It can be used in conjunction with \var{FrameWork} or
standalone.

This module is temporary, it will eventually be replaced by a module
that handles argument names better and possibly automates making your
application scriptable.

The \var{MiniAEFrame} module defines the following classes:

\renewcommand{\indexsubitem}{(in module MiniAEFrame)}

\begin{funcdesc}{AEServer}{}
A class that handles AppleEvent dispatch. Your application should
subclass this class together with either
\code{MiniAEFrame.MiniApplication} or
\code{FrameWork.Application}. Your \code{__init__} method should call
the \code{__init__} method for both classes.
\end{funcdesc}

\begin{funcdesc}{MiniApplication}{}
A class that is more or less compatible with
\code{FrameWork.Application} but with less functionality. Its
eventloop supports the apple menu, command-dot and AppleEvents, other
events are passed on to the Python interpreter and/or Sioux.
Useful if your application wants to use \code{AEServer} but does not
provide its own windows, etc.
\end{funcdesc}

\subsection{AEServer Objects}

\renewcommand{\indexsubitem}{(AEServer method)}

\begin{funcdesc}{installaehandler}{classe\, type\, callback}
Installs an AppleEvent handler. \code{Classe} and \code{type} are the
four-char OSA Class and Type designators, \code{'****'} wildcards are
allowed. When a matching AppleEvent is received the parameters are
decoded and your callback is invoked.
\end{funcdesc}

\begin{funcdesc}{callback}{_object\, **kwargs}
Your callback is called with the OSA Direct Object as first positional
parameter. The other parameters are passed as keyword arguments, with
the 4-char designator as name. Three extra keyword parameters are
passed: \code{_class} and \code{_type} are the Class and Type
designators and \code{_attributes} is a dictionary with the AppleEvent
attributes.

The return value of your method is packed with
\code{aetools.packevent} and sent as reply.
\end{funcdesc}

Note that there are some serious problems with the current
design. AppleEvents which have non-identifier 4-char designators for
arguments are not implementable, and it is not possible to return an
error to the originator. This will be addressed in a future release.


\chapter{Standard Windowing Interface}

The modules in this chapter are available only on those systems where
the STDWIN library is available.  STDWIN runs on \UNIX{} under X11 and
on the Macintosh.  See CWI report CS-R8817.

\strong{Warning:} Using STDWIN is not recommended for new
applications.  It has never been ported to Microsoft Windows or
Windows NT, and for X11 or the Macintosh it lacks important
functionality --- in particular, it has no tools for the construction
of dialogs.  For most platforms, alternative, native solutions exist
(though none are currently documented in this manual): Tkinter for
\UNIX{} under X11, native Xt with Motif or Athena widgets for \UNIX{}
under X11, Win32 for Windows and Windows NT, and a collection of
native toolkit interfaces for the Macintosh.

\section{Built-in Module \sectcode{stdwin}}
\bimodindex{stdwin}

This module defines several new object types and functions that
provide access to the functionality of STDWIN.

On Unix running X11, it can only be used if the \code{DISPLAY}
environment variable is set or an explicit \samp{-display
\var{displayname}} argument is passed to the Python interpreter.

Functions have names that usually resemble their C STDWIN counterparts
with the initial `w' dropped.
Points are represented by pairs of integers; rectangles
by pairs of points.
For a complete description of STDWIN please refer to the documentation
of STDWIN for C programmers (aforementioned CWI report).

\subsection{Functions Defined in Module \sectcode{stdwin}}
\nodename{STDWIN Functions}

The following functions are defined in the \code{stdwin} module:

\renewcommand{\indexsubitem}{(in module stdwin)}
\begin{funcdesc}{open}{title}
Open a new window whose initial title is given by the string argument.
Return a window object; window object methods are described below.%
\footnote{The Python version of STDWIN does not support draw procedures; all
	drawing requests are reported as draw events.}
\end{funcdesc}

\begin{funcdesc}{getevent}{}
Wait for and return the next event.
An event is returned as a triple: the first element is the event
type, a small integer; the second element is the window object to which
the event applies, or
\code{None}
if it applies to no window in particular;
the third element is type-dependent.
Names for event types and command codes are defined in the standard
module
\code{stdwinevent}.
\end{funcdesc}

\begin{funcdesc}{pollevent}{}
Return the next event, if one is immediately available.
If no event is available, return \code{()}.
\end{funcdesc}

\begin{funcdesc}{getactive}{}
Return the window that is currently active, or \code{None} if no
window is currently active.  (This can be emulated by monitoring
WE_ACTIVATE and WE_DEACTIVATE events.)
\end{funcdesc}

\begin{funcdesc}{listfontnames}{pattern}
Return the list of font names in the system that match the pattern (a
string).  The pattern should normally be \code{'*'}; returns all
available fonts.  If the underlying window system is X11, other
patterns follow the standard X11 font selection syntax (as used e.g.
in resource definitions), i.e. the wildcard character \code{'*'}
matches any sequence of characters (including none) and \code{'?'}
matches any single character.
On the Macintosh this function currently returns an empty list.
\end{funcdesc}

\begin{funcdesc}{setdefscrollbars}{hflag\, vflag}
Set the flags controlling whether subsequently opened windows will
have horizontal and/or vertical scroll bars.
\end{funcdesc}

\begin{funcdesc}{setdefwinpos}{h\, v}
Set the default window position for windows opened subsequently.
\end{funcdesc}

\begin{funcdesc}{setdefwinsize}{width\, height}
Set the default window size for windows opened subsequently.
\end{funcdesc}

\begin{funcdesc}{getdefscrollbars}{}
Return the flags controlling whether subsequently opened windows will
have horizontal and/or vertical scroll bars.
\end{funcdesc}

\begin{funcdesc}{getdefwinpos}{}
Return the default window position for windows opened subsequently.
\end{funcdesc}

\begin{funcdesc}{getdefwinsize}{}
Return the default window size for windows opened subsequently.
\end{funcdesc}

\begin{funcdesc}{getscrsize}{}
Return the screen size in pixels.
\end{funcdesc}

\begin{funcdesc}{getscrmm}{}
Return the screen size in millimeters.
\end{funcdesc}

\begin{funcdesc}{fetchcolor}{colorname}
Return the pixel value corresponding to the given color name.
Return the default foreground color for unknown color names.
Hint: the following code tests whether you are on a machine that
supports more than two colors:
\bcode\begin{verbatim}
if stdwin.fetchcolor('black') <> \
          stdwin.fetchcolor('red') <> \
          stdwin.fetchcolor('white'):
    print 'color machine'
else:
    print 'monochrome machine'
\end{verbatim}\ecode
\end{funcdesc}

\begin{funcdesc}{setfgcolor}{pixel}
Set the default foreground color.
This will become the default foreground color of windows opened
subsequently, including dialogs.
\end{funcdesc}

\begin{funcdesc}{setbgcolor}{pixel}
Set the default background color.
This will become the default background color of windows opened
subsequently, including dialogs.
\end{funcdesc}

\begin{funcdesc}{getfgcolor}{}
Return the pixel value of the current default foreground color.
\end{funcdesc}

\begin{funcdesc}{getbgcolor}{}
Return the pixel value of the current default background color.
\end{funcdesc}

\begin{funcdesc}{setfont}{fontname}
Set the current default font.
This will become the default font for windows opened subsequently,
and is also used by the text measuring functions \code{textwidth},
\code{textbreak}, \code{lineheight} and \code{baseline} below.
This accepts two more optional parameters, size and style:
Size is the font size (in `points').
Style is a single character specifying the style, as follows:
\code{'b'} = bold,
\code{'i'} = italic,
\code{'o'} = bold + italic,
\code{'u'} = underline;
default style is roman.
Size and style are ignored under X11 but used on the Macintosh.
(Sorry for all this complexity --- a more uniform interface is being designed.)
\end{funcdesc}

\begin{funcdesc}{menucreate}{title}
Create a menu object referring to a global menu (a menu that appears in
all windows).
Methods of menu objects are described below.
Note: normally, menus are created locally; see the window method
\code{menucreate} below.
\strong{Warning:} the menu only appears in a window as long as the object
returned by this call exists.
\end{funcdesc}

\begin{funcdesc}{newbitmap}{width\, height}
Create a new bitmap object of the given dimensions.
Methods of bitmap objects are described below.
Not available on the Macintosh.
\end{funcdesc}

\begin{funcdesc}{fleep}{}
Cause a beep or bell (or perhaps a `visual bell' or flash, hence the
name).
\end{funcdesc}

\begin{funcdesc}{message}{string}
Display a dialog box containing the string.
The user must click OK before the function returns.
\end{funcdesc}

\begin{funcdesc}{askync}{prompt\, default}
Display a dialog that prompts the user to answer a question with yes or
no.
Return 0 for no, 1 for yes.
If the user hits the Return key, the default (which must be 0 or 1) is
returned.
If the user cancels the dialog, the
\code{KeyboardInterrupt}
exception is raised.
\end{funcdesc}

\begin{funcdesc}{askstr}{prompt\, default}
Display a dialog that prompts the user for a string.
If the user hits the Return key, the default string is returned.
If the user cancels the dialog, the
\code{KeyboardInterrupt}
exception is raised.
\end{funcdesc}

\begin{funcdesc}{askfile}{prompt\, default\, new}
Ask the user to specify a filename.
If
\var{new}
is zero it must be an existing file; otherwise, it must be a new file.
If the user cancels the dialog, the
\code{KeyboardInterrupt}
exception is raised.
\end{funcdesc}

\begin{funcdesc}{setcutbuffer}{i\, string}
Store the string in the system's cut buffer number
\var{i},
where it can be found (for pasting) by other applications.
On X11, there are 8 cut buffers (numbered 0..7).
Cut buffer number 0 is the `clipboard' on the Macintosh.
\end{funcdesc}

\begin{funcdesc}{getcutbuffer}{i}
Return the contents of the system's cut buffer number
\var{i}.
\end{funcdesc}

\begin{funcdesc}{rotatecutbuffers}{n}
On X11, rotate the 8 cut buffers by
\var{n}.
Ignored on the Macintosh.
\end{funcdesc}

\begin{funcdesc}{getselection}{i}
Return X11 selection number
\var{i.}
Selections are not cut buffers.
Selection numbers are defined in module
\code{stdwinevents}.
Selection \code{WS_PRIMARY} is the
\dfn{primary}
selection (used by
xterm,
for instance);
selection \code{WS_SECONDARY} is the
\dfn{secondary}
selection; selection \code{WS_CLIPBOARD} is the
\dfn{clipboard}
selection (used by
xclipboard).
On the Macintosh, this always returns an empty string.
\end{funcdesc}

\begin{funcdesc}{resetselection}{i}
Reset selection number
\var{i},
if this process owns it.
(See window method
\code{setselection()}).
\end{funcdesc}

\begin{funcdesc}{baseline}{}
Return the baseline of the current font (defined by STDWIN as the
vertical distance between the baseline and the top of the
characters).
\end{funcdesc}

\begin{funcdesc}{lineheight}{}
Return the total line height of the current font.
\end{funcdesc}

\begin{funcdesc}{textbreak}{str\, width}
Return the number of characters of the string that fit into a space of
\var{width}
bits wide when drawn in the curent font.
\end{funcdesc}

\begin{funcdesc}{textwidth}{str}
Return the width in bits of the string when drawn in the current font.
\end{funcdesc}

\begin{funcdesc}{connectionnumber}{}
\funcline{fileno}{}
(X11 under \UNIX{} only) Return the ``connection number'' used by the
underlying X11 implementation.  (This is normally the file number of
the socket.)  Both functions return the same value;
\code{connectionnumber()} is named after the corresponding function in
X11 and STDWIN, while \code{fileno()} makes it possible to use the
\code{stdwin} module as a ``file'' object parameter to
\code{select.select()}.  Note that if \code{select()} implies that
input is possible on \code{stdwin}, this does not guarantee that an
event is ready --- it may be some internal communication going on
between the X server and the client library.  Thus, you should call
\code{stdwin.pollevent()} until it returns \code{None} to check for
events if you don't want your program to block.  Because of internal
buffering in X11, it is also possible that \code{stdwin.pollevent()}
returns an event while \code{select()} does not find \code{stdwin} to
be ready, so you should read any pending events with
\code{stdwin.pollevent()} until it returns \code{None} before entering
a blocking \code{select()} call.
\ttindex{select}
\end{funcdesc}

\subsection{Window Objects}

Window objects are created by \code{stdwin.open()}.  They are closed
by their \code{close()} method or when they are garbage-collected.
Window objects have the following methods:

\renewcommand{\indexsubitem}{(window method)}

\begin{funcdesc}{begindrawing}{}
Return a drawing object, whose methods (described below) allow drawing
in the window.
\end{funcdesc}

\begin{funcdesc}{change}{rect}
Invalidate the given rectangle; this may cause a draw event.
\end{funcdesc}

\begin{funcdesc}{gettitle}{}
Returns the window's title string.
\end{funcdesc}

\begin{funcdesc}{getdocsize}{}
\begin{sloppypar}
Return a pair of integers giving the size of the document as set by
\code{setdocsize()}.
\end{sloppypar}
\end{funcdesc}

\begin{funcdesc}{getorigin}{}
Return a pair of integers giving the origin of the window with respect
to the document.
\end{funcdesc}

\begin{funcdesc}{gettitle}{}
Return the window's title string.
\end{funcdesc}

\begin{funcdesc}{getwinsize}{}
Return a pair of integers giving the size of the window.
\end{funcdesc}

\begin{funcdesc}{getwinpos}{}
Return a pair of integers giving the position of the window's upper
left corner (relative to the upper left corner of the screen).
\end{funcdesc}

\begin{funcdesc}{menucreate}{title}
Create a menu object referring to a local menu (a menu that appears
only in this window).
Methods of menu objects are described below.
{\bf Warning:} the menu only appears as long as the object
returned by this call exists.
\end{funcdesc}

\begin{funcdesc}{scroll}{rect\, point}
Scroll the given rectangle by the vector given by the point.
\end{funcdesc}

\begin{funcdesc}{setdocsize}{point}
Set the size of the drawing document.
\end{funcdesc}

\begin{funcdesc}{setorigin}{point}
Move the origin of the window (its upper left corner)
to the given point in the document.
\end{funcdesc}

\begin{funcdesc}{setselection}{i\, str}
Attempt to set X11 selection number
\var{i}
to the string
\var{str}.
(See stdwin method
\code{getselection()}
for the meaning of
\var{i}.)
Return true if it succeeds.
If  succeeds, the window ``owns'' the selection until
(a) another application takes ownership of the selection; or
(b) the window is deleted; or
(c) the application clears ownership by calling
\code{stdwin.resetselection(\var{i})}.
When another application takes ownership of the selection, a
\code{WE_LOST_SEL}
event is received for no particular window and with the selection number
as detail.
Ignored on the Macintosh.
\end{funcdesc}

\begin{funcdesc}{settimer}{dsecs}
Schedule a timer event for the window in
\code{\var{dsecs}/10}
seconds.
\end{funcdesc}

\begin{funcdesc}{settitle}{title}
Set the window's title string.
\end{funcdesc}

\begin{funcdesc}{setwincursor}{name}
\begin{sloppypar}
Set the window cursor to a cursor of the given name.
It raises the
\code{RuntimeError}
exception if no cursor of the given name exists.
Suitable names include
\code{'ibeam'},
\code{'arrow'},
\code{'cross'},
\code{'watch'}
and
\code{'plus'}.
On X11, there are many more (see
\file{<X11/cursorfont.h>}).
\end{sloppypar}
\end{funcdesc}

\begin{funcdesc}{setwinpos}{h\, v}
Set the the position of the window's upper left corner (relative to
the upper left corner of the screen).
\end{funcdesc}

\begin{funcdesc}{setwinsize}{width\, height}
Set the window's size.
\end{funcdesc}

\begin{funcdesc}{show}{rect}
Try to ensure that the given rectangle of the document is visible in
the window.
\end{funcdesc}

\begin{funcdesc}{textcreate}{rect}
Create a text-edit object in the document at the given rectangle.
Methods of text-edit objects are described below.
\end{funcdesc}

\begin{funcdesc}{setactive}{}
Attempt to make this window the active window.  If successful, this
will generate a WE_ACTIVATE event (and a WE_DEACTIVATE event in case
another window in this application became inactive).
\end{funcdesc}

\begin{funcdesc}{close}{}
Discard the window object.  It should not be used again.
\end{funcdesc}

\subsection{Drawing Objects}

Drawing objects are created exclusively by the window method
\code{begindrawing()}.
Only one drawing object can exist at any given time; the drawing object
must be deleted to finish drawing.
No drawing object may exist when
\code{stdwin.getevent()}
is called.
Drawing objects have the following methods:

\renewcommand{\indexsubitem}{(drawing method)}

\begin{funcdesc}{box}{rect}
Draw a box just inside a rectangle.
\end{funcdesc}

\begin{funcdesc}{circle}{center\, radius}
Draw a circle with given center point and radius.
\end{funcdesc}

\begin{funcdesc}{elarc}{center\, \(rh\, rv\)\, \(a1\, a2\)}
Draw an elliptical arc with given center point.
\code{(\var{rh}, \var{rv})}
gives the half sizes of the horizontal and vertical radii.
\code{(\var{a1}, \var{a2})}
gives the angles (in degrees) of the begin and end points.
0 degrees is at 3 o'clock, 90 degrees is at 12 o'clock.
\end{funcdesc}

\begin{funcdesc}{erase}{rect}
Erase a rectangle.
\end{funcdesc}

\begin{funcdesc}{fillcircle}{center\, radius}
Draw a filled circle with given center point and radius.
\end{funcdesc}

\begin{funcdesc}{fillelarc}{center\, \(rh\, rv\)\, \(a1\, a2\)}
Draw a filled elliptical arc; arguments as for \code{elarc}.
\end{funcdesc}

\begin{funcdesc}{fillpoly}{points}
Draw a filled polygon given by a list (or tuple) of points.
\end{funcdesc}

\begin{funcdesc}{invert}{rect}
Invert a rectangle.
\end{funcdesc}

\begin{funcdesc}{line}{p1\, p2}
Draw a line from point
\var{p1}
to
\var{p2}.
\end{funcdesc}

\begin{funcdesc}{paint}{rect}
Fill a rectangle.
\end{funcdesc}

\begin{funcdesc}{poly}{points}
Draw the lines connecting the given list (or tuple) of points.
\end{funcdesc}

\begin{funcdesc}{shade}{rect\, percent}
Fill a rectangle with a shading pattern that is about
\var{percent}
percent filled.
\end{funcdesc}

\begin{funcdesc}{text}{p\, str}
Draw a string starting at point p (the point specifies the
top left coordinate of the string).
\end{funcdesc}

\begin{funcdesc}{xorcircle}{center\, radius}
\funcline{xorelarc}{center\, \(rh\, rv\)\, \(a1\, a2\)}
\funcline{xorline}{p1\, p2}
\funcline{xorpoly}{points}
Draw a circle, an elliptical arc, a line or a polygon, respectively,
in XOR mode.
\end{funcdesc}

\begin{funcdesc}{setfgcolor}{}
\funcline{setbgcolor}{}
\funcline{getfgcolor}{}
\funcline{getbgcolor}{}
These functions are similar to the corresponding functions described
above for the
\code{stdwin}
module, but affect or return the colors currently used for drawing
instead of the global default colors.
When a drawing object is created, its colors are set to the window's
default colors, which are in turn initialized from the global default
colors when the window is created.
\end{funcdesc}

\begin{funcdesc}{setfont}{}
\funcline{baseline}{}
\funcline{lineheight}{}
\funcline{textbreak}{}
\funcline{textwidth}{}
These functions are similar to the corresponding functions described
above for the
\code{stdwin}
module, but affect or use the current drawing font instead of
the global default font.
When a drawing object is created, its font is set to the window's
default font, which is in turn initialized from the global default
font when the window is created.
\end{funcdesc}

\begin{funcdesc}{bitmap}{point\, bitmap\, mask}
Draw the \var{bitmap} with its top left corner at \var{point}.
If the optional \var{mask} argument is present, it should be either
the same object as \var{bitmap}, to draw only those bits that are set
in the bitmap, in the foreground color, or \code{None}, to draw all
bits (ones are drawn in the foreground color, zeros in the background
color).
Not available on the Macintosh.
\end{funcdesc}

\begin{funcdesc}{cliprect}{rect}
Set the ``clipping region'' to a rectangle.
The clipping region limits the effect of all drawing operations, until
it is changed again or until the drawing object is closed.  When a
drawing object is created the clipping region is set to the entire
window.  When an object to be drawn falls partly outside the clipping
region, the set of pixels drawn is the intersection of the clipping
region and the set of pixels that would be drawn by the same operation
in the absence of a clipping region.
\end{funcdesc}

\begin{funcdesc}{noclip}{}
Reset the clipping region to the entire window.
\end{funcdesc}

\begin{funcdesc}{close}{}
\funcline{enddrawing}{}
Discard the drawing object.  It should not be used again.
\end{funcdesc}

\subsection{Menu Objects}

A menu object represents a menu.
The menu is destroyed when the menu object is deleted.
The following methods are defined:

\renewcommand{\indexsubitem}{(menu method)}

\begin{funcdesc}{additem}{text\, shortcut}
Add a menu item with given text.
The shortcut must be a string of length 1, or omitted (to specify no
shortcut).
\end{funcdesc}

\begin{funcdesc}{setitem}{i\, text}
Set the text of item number
\var{i}.
\end{funcdesc}

\begin{funcdesc}{enable}{i\, flag}
Enable or disables item
\var{i}.
\end{funcdesc}

\begin{funcdesc}{check}{i\, flag}
Set or clear the
\dfn{check mark}
for item
\var{i}.
\end{funcdesc}

\begin{funcdesc}{close}{}
Discard the menu object.  It should not be used again.
\end{funcdesc}

\subsection{Bitmap Objects}

A bitmap represents a rectangular array of bits.
The top left bit has coordinate (0, 0).
A bitmap can be drawn with the \code{bitmap} method of a drawing object.
Bitmaps are currently not available on the Macintosh.

The following methods are defined:

\renewcommand{\indexsubitem}{(bitmap method)}

\begin{funcdesc}{getsize}{}
Return a tuple representing the width and height of the bitmap.
(This returns the values that have been passed to the \code{newbitmap}
function.)
\end{funcdesc}

\begin{funcdesc}{setbit}{point\, bit}
Set the value of the bit indicated by \var{point} to \var{bit}.
\end{funcdesc}

\begin{funcdesc}{getbit}{point}
Return the value of the bit indicated by \var{point}.
\end{funcdesc}

\begin{funcdesc}{close}{}
Discard the bitmap object.  It should not be used again.
\end{funcdesc}

\subsection{Text-edit Objects}

A text-edit object represents a text-edit block.
For semantics, see the STDWIN documentation for C programmers.
The following methods exist:

\renewcommand{\indexsubitem}{(text-edit method)}

\begin{funcdesc}{arrow}{code}
Pass an arrow event to the text-edit block.
The
\var{code}
must be one of
\code{WC_LEFT},
\code{WC_RIGHT},
\code{WC_UP}
or
\code{WC_DOWN}
(see module
\code{stdwinevents}).
\end{funcdesc}

\begin{funcdesc}{draw}{rect}
Pass a draw event to the text-edit block.
The rectangle specifies the redraw area.
\end{funcdesc}

\begin{funcdesc}{event}{type\, window\, detail}
Pass an event gotten from
\code{stdwin.getevent()}
to the text-edit block.
Return true if the event was handled.
\end{funcdesc}

\begin{funcdesc}{getfocus}{}
Return 2 integers representing the start and end positions of the
focus, usable as slice indices on the string returned by
\code{gettext()}.
\end{funcdesc}

\begin{funcdesc}{getfocustext}{}
Return the text in the focus.
\end{funcdesc}

\begin{funcdesc}{getrect}{}
Return a rectangle giving the actual position of the text-edit block.
(The bottom coordinate may differ from the initial position because
the block automatically shrinks or grows to fit.)
\end{funcdesc}

\begin{funcdesc}{gettext}{}
Return the entire text buffer.
\end{funcdesc}

\begin{funcdesc}{move}{rect}
Specify a new position for the text-edit block in the document.
\end{funcdesc}

\begin{funcdesc}{replace}{str}
Replace the text in the focus by the given string.
The new focus is an insert point at the end of the string.
\end{funcdesc}

\begin{funcdesc}{setfocus}{i\, j}
Specify the new focus.
Out-of-bounds values are silently clipped.
\end{funcdesc}

\begin{funcdesc}{settext}{str}
Replace the entire text buffer by the given string and set the focus
to \code{(0, 0)}.
\end{funcdesc}

\begin{funcdesc}{setview}{rect}
Set the view rectangle to \var{rect}.  If \var{rect} is \code{None},
viewing mode is reset.  In viewing mode, all output from the text-edit
object is clipped to the viewing rectangle.  This may be useful to
implement your own scrolling text subwindow.
\end{funcdesc}

\begin{funcdesc}{close}{}
Discard the text-edit object.  It should not be used again.
\end{funcdesc}

\subsection{Example}
\nodename{STDWIN Example}

Here is a minimal example of using STDWIN in Python.
It creates a window and draws the string ``Hello world'' in the top
left corner of the window.
The window will be correctly redrawn when covered and re-exposed.
The program quits when the close icon or menu item is requested.

\bcode\begin{verbatim}
import stdwin
from stdwinevents import *

def main():
    mywin = stdwin.open('Hello')
    #
    while 1:
        (type, win, detail) = stdwin.getevent()
        if type == WE_DRAW:
            draw = win.begindrawing()
            draw.text((0, 0), 'Hello, world')
            del draw
        elif type == WE_CLOSE:
            break

main()
\end{verbatim}\ecode

\section{Standard Module \sectcode{stdwinevents}}
\stmodindex{stdwinevents}

This module defines constants used by STDWIN for event types
(\code{WE_ACTIVATE} etc.), command codes (\code{WC_LEFT} etc.)
and selection types (\code{WS_PRIMARY} etc.).
Read the file for details.
Suggested usage is

\bcode\begin{verbatim}
>>> from stdwinevents import *
>>> 
\end{verbatim}\ecode

\section{Standard Module \sectcode{rect}}
\stmodindex{rect}

This module contains useful operations on rectangles.
A rectangle is defined as in module
\code{stdwin}:
a pair of points, where a point is a pair of integers.
For example, the rectangle

\bcode\begin{verbatim}
(10, 20), (90, 80)
\end{verbatim}\ecode

is a rectangle whose left, top, right and bottom edges are 10, 20, 90
and 80, respectively.
Note that the positive vertical axis points down (as in
\code{stdwin}).

The module defines the following objects:

\renewcommand{\indexsubitem}{(in module rect)}
\begin{excdesc}{error}
The exception raised by functions in this module when they detect an
error.
The exception argument is a string describing the problem in more
detail.
\end{excdesc}

\begin{datadesc}{empty}
The rectangle returned when some operations return an empty result.
This makes it possible to quickly check whether a result is empty:

\bcode\begin{verbatim}
>>> import rect
>>> r1 = (10, 20), (90, 80)
>>> r2 = (0, 0), (10, 20)
>>> r3 = rect.intersect([r1, r2])
>>> if r3 is rect.empty: print 'Empty intersection'
Empty intersection
>>> 
\end{verbatim}\ecode
\end{datadesc}

\begin{funcdesc}{is_empty}{r}
Returns true if the given rectangle is empty.
A rectangle
\code{(\var{left}, \var{top}), (\var{right}, \var{bottom})}
is empty if
\iftexi
\code{\var{left} >= \var{right}} or \code{\var{top} => \var{bottom}}.
\else
$\var{left} \geq \var{right}$ or $\var{top} \geq \var{bottom}$.
%%JHXXX{\em left~$\geq$~right} or {\em top~$\leq$~bottom}.
\fi
\end{funcdesc}

\begin{funcdesc}{intersect}{list}
Returns the intersection of all rectangles in the list argument.
It may also be called with a tuple argument.
Raises
\code{rect.error}
if the list is empty.
Returns
\code{rect.empty}
if the intersection of the rectangles is empty.
\end{funcdesc}

\begin{funcdesc}{union}{list}
Returns the smallest rectangle that contains all non-empty rectangles in
the list argument.
It may also be called with a tuple argument or with two or more
rectangles as arguments.
Returns
\code{rect.empty}
if the list is empty or all its rectangles are empty.
\end{funcdesc}

\begin{funcdesc}{pointinrect}{point\, rect}
Returns true if the point is inside the rectangle.
By definition, a point
\code{(\var{h}, \var{v})}
is inside a rectangle
\code{(\var{left}, \var{top}), (\var{right}, \var{bottom})} if
\iftexi
\code{\var{left} <= \var{h} < \var{right}} and
\code{\var{top} <= \var{v} < \var{bottom}}.
\else
$\var{left} \leq \var{h} < \var{right}$ and
$\var{top} \leq \var{v} < \var{bottom}$.
\fi
\end{funcdesc}

\begin{funcdesc}{inset}{rect\, \(dh\, dv\)}
Returns a rectangle that lies inside the
\code{rect}
argument by
\var{dh}
pixels horizontally
and
\var{dv}
pixels
vertically.
If
\var{dh}
or
\var{dv}
is negative, the result lies outside
\var{rect}.
\end{funcdesc}

\begin{funcdesc}{rect2geom}{rect}
Converts a rectangle to geometry representation:
\code{(\var{left}, \var{top}), (\var{width}, \var{height})}.
\end{funcdesc}

\begin{funcdesc}{geom2rect}{geom}
Converts a rectangle given in geometry representation back to the
standard rectangle representation
\code{(\var{left}, \var{top}), (\var{right}, \var{bottom})}.
\end{funcdesc}
		% STDWIN ONLY

\chapter{SGI IRIX Specific Services}

The modules described in this chapter provide interfaces to features
that are unique to SGI's IRIX operating system (versions 4 and 5).
			% SGI IRIX ONLY
\section{Built-in Module \sectcode{al}}
\bimodindex{al}

This module provides access to the audio facilities of the SGI Indy
and Indigo workstations.  See section 3A of the IRIX man pages for
details.  You'll need to read those man pages to understand what these
functions do!  Some of the functions are not available in IRIX
releases before 4.0.5.  Again, see the manual to check whether a
specific function is available on your platform.

All functions and methods defined in this module are equivalent to
the C functions with \samp{AL} prefixed to their name.

Symbolic constants from the C header file \file{<audio.h>} are defined
in the standard module \code{AL}, see below.

\strong{Warning:} the current version of the audio library may dump core
when bad argument values are passed rather than returning an error
status.  Unfortunately, since the precise circumstances under which
this may happen are undocumented and hard to check, the Python
interface can provide no protection against this kind of problems.
(One example is specifying an excessive queue size --- there is no
documented upper limit.)

The module defines the following functions:

\renewcommand{\indexsubitem}{(in module al)}

\begin{funcdesc}{openport}{name\, direction\optional{\, config}}
The name and direction arguments are strings.  The optional config
argument is a configuration object as returned by
\code{al.newconfig()}.  The return value is an \dfn{port object};
methods of port objects are described below.
\end{funcdesc}

\begin{funcdesc}{newconfig}{}
The return value is a new \dfn{configuration object}; methods of
configuration objects are described below.
\end{funcdesc}

\begin{funcdesc}{queryparams}{device}
The device argument is an integer.  The return value is a list of
integers containing the data returned by ALqueryparams().
\end{funcdesc}

\begin{funcdesc}{getparams}{device\, list}
The device argument is an integer.  The list argument is a list such
as returned by \code{queryparams}; it is modified in place (!).
\end{funcdesc}

\begin{funcdesc}{setparams}{device\, list}
The device argument is an integer.  The list argument is a list such
as returned by \code{al.queryparams}.
\end{funcdesc}

\subsection{Configuration Objects}

Configuration objects (returned by \code{al.newconfig()} have the
following methods:

\renewcommand{\indexsubitem}{(audio configuration object method)}

\begin{funcdesc}{getqueuesize}{}
Return the queue size.
\end{funcdesc}

\begin{funcdesc}{setqueuesize}{size}
Set the queue size.
\end{funcdesc}

\begin{funcdesc}{getwidth}{}
Get the sample width.
\end{funcdesc}

\begin{funcdesc}{setwidth}{width}
Set the sample width.
\end{funcdesc}

\begin{funcdesc}{getchannels}{}
Get the channel count.
\end{funcdesc}

\begin{funcdesc}{setchannels}{nchannels}
Set the channel count.
\end{funcdesc}

\begin{funcdesc}{getsampfmt}{}
Get the sample format.
\end{funcdesc}

\begin{funcdesc}{setsampfmt}{sampfmt}
Set the sample format.
\end{funcdesc}

\begin{funcdesc}{getfloatmax}{}
Get the maximum value for floating sample formats.
\end{funcdesc}

\begin{funcdesc}{setfloatmax}{floatmax}
Set the maximum value for floating sample formats.
\end{funcdesc}

\subsection{Port Objects}

Port objects (returned by \code{al.openport()} have the following
methods:

\renewcommand{\indexsubitem}{(audio port object method)}

\begin{funcdesc}{closeport}{}
Close the port.
\end{funcdesc}

\begin{funcdesc}{getfd}{}
Return the file descriptor as an int.
\end{funcdesc}

\begin{funcdesc}{getfilled}{}
Return the number of filled samples.
\end{funcdesc}

\begin{funcdesc}{getfillable}{}
Return the number of fillable samples.
\end{funcdesc}

\begin{funcdesc}{readsamps}{nsamples}
Read a number of samples from the queue, blocking if necessary.
Return the data as a string containing the raw data, (e.g., 2 bytes per
sample in big-endian byte order (high byte, low byte) if you have set
the sample width to 2 bytes).
\end{funcdesc}

\begin{funcdesc}{writesamps}{samples}
Write samples into the queue, blocking if necessary.  The samples are
encoded as described for the \code{readsamps} return value.
\end{funcdesc}

\begin{funcdesc}{getfillpoint}{}
Return the `fill point'.
\end{funcdesc}

\begin{funcdesc}{setfillpoint}{fillpoint}
Set the `fill point'.
\end{funcdesc}

\begin{funcdesc}{getconfig}{}
Return a configuration object containing the current configuration of
the port.
\end{funcdesc}

\begin{funcdesc}{setconfig}{config}
Set the configuration from the argument, a configuration object.
\end{funcdesc}

\begin{funcdesc}{getstatus}{list}
Get status information on last error.
\end{funcdesc}

\section{Standard Module \sectcode{AL}}
\nodename{AL (uppercase)}
\stmodindex{AL}

This module defines symbolic constants needed to use the built-in
module \code{al} (see above); they are equivalent to those defined in
the C header file \file{<audio.h>} except that the name prefix
\samp{AL_} is omitted.  Read the module source for a complete list of
the defined names.  Suggested use:

\bcode\begin{verbatim}
import al
from AL import *
\end{verbatim}\ecode

%\section{Built-in Module \sectcode{audio}}
\bimodindex{audio}

\strong{Note:} This module is obsolete, since the hardware to which it
interfaces is obsolete.  For audio on the Indigo or 4D/35, see
built-in module \code{al} above.

This module provides rudimentary access to the audio I/O device
\file{/dev/audio} on the Silicon Graphics Personal IRIS 4D/25;
see {\it audio}(7). It supports the following operations:

\renewcommand{\indexsubitem}{(in module audio)}
\begin{funcdesc}{setoutgain}{n}
Sets the output gain.
\iftexi
\code{0 <= \var{n} < 256}.
\else
$0 \leq \var{n} < 256$.
%%JHXXX Sets the output gain (0-255).
\fi
\end{funcdesc}

\begin{funcdesc}{getoutgain}{}
Returns the output gain.
\end{funcdesc}

\begin{funcdesc}{setrate}{n}
Sets the sampling rate: \code{1} = 32K/sec, \code{2} = 16K/sec,
\code{3} = 8K/sec.
\end{funcdesc}

\begin{funcdesc}{setduration}{n}
Sets the `sound duration' in units of 1/100 seconds.
\end{funcdesc}

\begin{funcdesc}{read}{n}
Reads a chunk of
\var{n}
sampled bytes from the audio input (line in or microphone).
The chunk is returned as a string of length n.
Each byte encodes one sample as a signed 8-bit quantity using linear
encoding.
This string can be converted to numbers using \code{chr2num()} described
below.
\end{funcdesc}

\begin{funcdesc}{write}{buf}
Writes a chunk of samples to the audio output (speaker).
\end{funcdesc}

These operations support asynchronous audio I/O:

\renewcommand{\indexsubitem}{(in module audio)}
\begin{funcdesc}{start_recording}{n}
Starts a second thread (a process with shared memory) that begins reading
\var{n}
bytes from the audio device.
The main thread immediately continues.
\end{funcdesc}

\begin{funcdesc}{wait_recording}{}
Waits for the second thread to finish and returns the data read.
\end{funcdesc}

\begin{funcdesc}{stop_recording}{}
Makes the second thread stop reading as soon as possible.
Returns the data read so far.
\end{funcdesc}

\begin{funcdesc}{poll_recording}{}
Returns true if the second thread has finished reading (so
\code{wait_recording()} would return the data without delay).
\end{funcdesc}

\begin{funcdesc}{start_playing}{}
\funcline{wait_playing}{}
\funcline{stop_playing}{}
\funcline{poll_playing}{}
\begin{sloppypar}
Similar but for output.
\code{stop_playing()}
returns a lower bound for the number of bytes actually played (not very
accurate).
\end{sloppypar}
\end{funcdesc}

The following operations do not affect the audio device but are
implemented in C for efficiency:

\renewcommand{\indexsubitem}{(in module audio)}
\begin{funcdesc}{amplify}{buf\, f1\, f2}
Amplifies a chunk of samples by a variable factor changing from
\code{\var{f1}/256} to \code{\var{f2}/256.}
Negative factors are allowed.
Resulting values that are to large to fit in a byte are clipped.         
\end{funcdesc}

\begin{funcdesc}{reverse}{buf}
Returns a chunk of samples backwards.
\end{funcdesc}

\begin{funcdesc}{add}{buf1\, buf2}
Bytewise adds two chunks of samples.
Bytes that exceed the range are clipped.
If one buffer is shorter, it is assumed to be padded with zeros.
\end{funcdesc}

\begin{funcdesc}{chr2num}{buf}
Converts a string of sampled bytes as returned by \code{read()} into
a list containing the numeric values of the samples.
\end{funcdesc}

\begin{funcdesc}{num2chr}{list}
\begin{sloppypar}
Converts a list as returned by
\code{chr2num()}
back to a buffer acceptable by
\code{write()}.
\end{sloppypar}
\end{funcdesc}

\section{Built-in Module \sectcode{cd}}
\bimodindex{cd}

This module provides an interface to the Silicon Graphics CD library.
It is available only on Silicon Graphics systems.

The way the library works is as follows.  A program opens the CD-ROM
device with \code{cd.open()} and creates a parser to parse the data
from the CD with \code{cd.createparser()}.  The object returned by
\code{cd.open()} can be used to read data from the CD, but also to get
status information for the CD-ROM device, and to get information about
the CD, such as the table of contents.  Data from the CD is passed to
the parser, which parses the frames, and calls any callback
functions that have previously been added.

An audio CD is divided into \dfn{tracks} or \dfn{programs} (the terms
are used interchangeably).  Tracks can be subdivided into
\dfn{indices}.  An audio CD contains a \dfn{table of contents} which
gives the starts of the tracks on the CD.  Index 0 is usually the
pause before the start of a track.  The start of the track as given by
the table of contents is normally the start of index 1.

Positions on a CD can be represented in two ways.  Either a frame
number or a tuple of three values, minutes, seconds and frames.  Most
functions use the latter representation.  Positions can be both
relative to the beginning of the CD, and to the beginning of the
track.

Module \code{cd} defines the following functions and constants:

\renewcommand{\indexsubitem}{(in module cd)}

\begin{funcdesc}{createparser}{}
Create and return an opaque parser object.  The methods of the parser
object are described below.
\end{funcdesc}

\begin{funcdesc}{msftoframe}{min\, sec\, frame}
Converts a \code{(minutes, seconds, frames)} triple representing time
in absolute time code into the corresponding CD frame number.
\end{funcdesc}

\begin{funcdesc}{open}{\optional{device\optional{\, mode}}}
Open the CD-ROM device.  The return value is an opaque player object;
methods of the player object are described below.  The device is the
name of the SCSI device file, e.g. /dev/scsi/sc0d4l0, or \code{None}.
If omited or \code{None}, the hardware inventory is consulted to
locate a CD-ROM drive.  The \code{mode}, if not omited, should be the
string 'r'.
\end{funcdesc}

The module defines the following variables:

\begin{datadesc}{error}
Exception raised on various errors.
\end{datadesc}

\begin{datadesc}{DATASIZE}
The size of one frame's worth of audio data.  This is the size of the
audio data as passed to the callback of type \code{audio}.
\end{datadesc}

\begin{datadesc}{BLOCKSIZE}
The size of one uninterpreted frame of audio data.
\end{datadesc}

The following variables are states as returned by \code{getstatus}:

\begin{datadesc}{READY}
The drive is ready for operation loaded with an audio CD.
\end{datadesc}

\begin{datadesc}{NODISC}
The drive does not have a CD loaded.
\end{datadesc}

\begin{datadesc}{CDROM}
The drive is loaded with a CD-ROM.  Subsequent play or read operations
will return I/O errors.
\end{datadesc}

\begin{datadesc}{ERROR}
An error aoocurred while trying to read the disc or its table of
contents.
\end{datadesc}

\begin{datadesc}{PLAYING}
The drive is in CD player mode playing an audio CD through its audio
jacks.
\end{datadesc}

\begin{datadesc}{PAUSED}
The drive is in CD layer mode with play paused.
\end{datadesc}

\begin{datadesc}{STILL}
The equivalent of \code{PAUSED} on older (non 3301) model Toshiba
CD-ROM drives.  Such drives have never been shipped by SGI.
\end{datadesc}

\begin{datadesc}{audio}
\dataline{pnum}
\dataline{index}
\dataline{ptime}
\dataline{atime}
\dataline{catalog}
\dataline{ident}
\dataline{control}
Integer constants describing the various types of parser callbacks
that can be set by the \code{addcallback()} method of CD parser
objects (see below).
\end{datadesc}

Player objects (returned by \code{cd.open()}) have the following
methods:

\renewcommand{\indexsubitem}{(CD player object method)}

\begin{funcdesc}{allowremoval}{}
Unlocks the eject button on the CD-ROM drive permitting the user to
eject the caddy if desired.
\end{funcdesc}

\begin{funcdesc}{bestreadsize}{}
Returns the best value to use for the \code{num_frames} parameter of
the \code{readda} method.  Best is defined as the value that permits a
continuous flow of data from the CD-ROM drive.
\end{funcdesc}

\begin{funcdesc}{close}{}
Frees the resources associated with the player object.  After calling
\code{close}, the methods of the object should no longer be used.
\end{funcdesc}

\begin{funcdesc}{eject}{}
Ejects the caddy from the CD-ROM drive.
\end{funcdesc}

\begin{funcdesc}{getstatus}{}
Returns information pertaining to the current state of the CD-ROM
drive.  The returned information is a tuple with the following values:
\code{state}, \code{track}, \code{rtime}, \code{atime}, \code{ttime},
\code{first}, \code{last}, \code{scsi_audio}, \code{cur_block}.
\code{rtime} is the time relative to the start of the current track;
\code{atime} is the time relative to the beginning of the disc;
\code{ttime} is the total time on the disc.  For more information on
the meaning of the values, see the manual for CDgetstatus.
The value of \code{state} is one of the following: \code{cd.ERROR},
\code{cd.NODISC}, \code{cd.READY}, \code{cd.PLAYING},
\code{cd.PAUSED}, \code{cd.STILL}, or \code{cd.CDROM}.
\end{funcdesc}

\begin{funcdesc}{gettrackinfo}{track}
Returns information about the specified track.  The returned
information is a tuple consisting of two elements, the start time of
the track and the duration of the track.
\end{funcdesc}

\begin{funcdesc}{msftoblock}{min\, sec\, frame}
Converts a minutes, seconds, frames triple representing a time in
absolute time code into the corresponding logical block number for the
given CD-ROM drive.  You should use \code{cd.msftoframe()} rather than
\code{msftoblock()} for comparing times.  The logical block number
differs from the frame number by an offset required by certain CD-ROM
drives.
\end{funcdesc}

\begin{funcdesc}{play}{start\, play}
Starts playback of an audio CD in the CD-ROM drive at the specified
track.  The audio output appears on the CD-ROM drive's headphone and
audio jacks (if fitted).  Play stops at the end of the disc.
\code{start} is the number of the track at which to start playing the
CD; if \code{play} is 0, the CD will be set to an initial paused
state.  The method \code{togglepause()} can then be used to commence
play.
\end{funcdesc}

\begin{funcdesc}{playabs}{min\, sec\, frame\, play}
Like \code{play()}, except that the start is given in minutes,
seconds, frames instead of a track number.
\end{funcdesc}

\begin{funcdesc}{playtrack}{start\, play}
Like \code{play()}, except that playing stops at the end of the track.
\end{funcdesc}

\begin{funcdesc}{playtrackabs}{track\, min\, sec\, frame\, play}
Like \code{play()}, except that playing begins at the spcified
absolute time and ends at the end of the specified track.
\end{funcdesc}

\begin{funcdesc}{preventremoval}{}
Locks the eject button on the CD-ROM drive thus preventing the user
from arbitrarily ejecting the caddy.
\end{funcdesc}

\begin{funcdesc}{readda}{num_frames}
Reads the specified number of frames from an audio CD mounted in the
CD-ROM drive.  The return value is a string representing the audio
frames.  This string can be passed unaltered to the \code{parseframe}
method of the parser object.
\end{funcdesc}

\begin{funcdesc}{seek}{min\, sec\, frame}
Sets the pointer that indicates the starting point of the next read of
digital audio data from a CD-ROM.  The pointer is set to an absolute
time code location specified in minutes, seconds, and frames.  The
return value is the logical block number to which the pointer has been
set.
\end{funcdesc}

\begin{funcdesc}{seekblock}{block}
Sets the pointer that indicates the starting point of the next read of
digital audio data from a CD-ROM.  The pointer is set to the specified
logical block number.  The return value is the logical block number to
which the pointer has been set.
\end{funcdesc}

\begin{funcdesc}{seektrack}{track}
Sets the pointer that indicates the starting point of the next read of
digital audio data from a CD-ROM.  The pointer is set to the specified
track.  The return value is the logical block number to which the
pointer has been set.
\end{funcdesc}

\begin{funcdesc}{stop}{}
Stops the current playing operation.
\end{funcdesc}

\begin{funcdesc}{togglepause}{}
Pauses the CD if it is playing, and makes it play if it is paused.
\end{funcdesc}

Parser objects (returned by \code{cd.createparser()}) have the
following methods:

\renewcommand{\indexsubitem}{(CD parser object method)}

\begin{funcdesc}{addcallback}{type\, func\, arg}
Adds a callback for the parser.  The parser has callbacks for eight
different types of data in the digital audio data stream.  Constants
for these types are defined at the \code{cd} module level (see above).
The callback is called as follows: \code{func(arg, type, data)}, where
\code{arg} is the user supplied argument, \code{type} is the
particular type of callback, and \code{data} is the data returned for
this \code{type} of callback.  The type of the data depends on the
\code{type} of callback as follows:
\begin{description}
\item[\code{cd.audio}: ]
The argument is a string which can be passed unmodified to
\code{al.writesamps()}.
\item[\code{cd.pnum}: ]
The argument is an integer giving the program (track) number.
\item[\code{cd.index}: ]
The argument is an integer giving the index number.
\item[\code{cd.ptime}: ]
The argument is a tuple consisting of the program time in minutes,
seconds, and frames.
\item[\code{cd.atime}: ]
The argument is a tuple consisting of the absolute time in minutes,
seconds, and frames.
\item[\code{cd.catalog}: ]
The argument is a string of 13 characters, giving the catalog number
of the CD.
\item[\code{cd.ident}: ]
The argument is a string of 12 characters, giving the ISRC
identification number of the recording.  The string consists of two
characters country code, three characters owner code, two characters
giving the year, and five characters giving a serial number.
\item[\code{cd.control}: ]
The argument is an integer giving the control bits from the CD subcode
data.
\end{description}
\end{funcdesc}

\begin{funcdesc}{deleteparser}{}
Deletes the parser and frees the memory it was using.  The object
should not be used after this call.  This call is done automatically
when the last reference to the object is removed.
\end{funcdesc}

\begin{funcdesc}{parseframe}{frame}
Parses one or more frames of digital audio data from a CD such as
returned by \code{readda}.  It determines which subcodes are present
in the data.  If these subcodes have changed since the last frame,
then \code{parseframe} executes a callback of the appropriate type
passing to it the subcode data found in the frame.
Unlike the C function, more than one frame of digital audio data can
be passed to this method.
\end{funcdesc}

\begin{funcdesc}{removecallback}{type}
Removes the callback for the given \code{type}.
\end{funcdesc}

\begin{funcdesc}{resetparser}{}
Resets the fields of the parser used for tracking subcodes to an
initial state.  \code{resetparser} should be called after the disc has
been changed.
\end{funcdesc}

\section{Built-in Module \sectcode{fl}}
\bimodindex{fl}

This module provides an interface to the FORMS Library by Mark
Overmars.  The source for the library can be retrieved by anonymous
ftp from host \samp{ftp.cs.ruu.nl}, directory \file{SGI/FORMS}.  It
was last tested with version 2.0b.

Most functions are literal translations of their C equivalents,
dropping the initial \samp{fl_} from their name.  Constants used by
the library are defined in module \code{FL} described below.

The creation of objects is a little different in Python than in C:
instead of the `current form' maintained by the library to which new
FORMS objects are added, all functions that add a FORMS object to a
form are methods of the Python object representing the form.
Consequently, there are no Python equivalents for the C functions
\code{fl_addto_form} and \code{fl_end_form}, and the equivalent of
\code{fl_bgn_form} is called \code{fl.make_form}.

Watch out for the somewhat confusing terminology: FORMS uses the word
\dfn{object} for the buttons, sliders etc. that you can place in a form.
In Python, `object' means any value.  The Python interface to FORMS
introduces two new Python object types: form objects (representing an
entire form) and FORMS objects (representing one button, slider etc.).
Hopefully this isn't too confusing...

There are no `free objects' in the Python interface to FORMS, nor is
there an easy way to add object classes written in Python.  The FORMS
interface to GL event handling is available, though, so you can mix
FORMS with pure GL windows.

\strong{Please note:} importing \code{fl} implies a call to the GL function
\code{foreground()} and to the FORMS routine \code{fl_init()}.

\subsection{Functions Defined in Module \sectcode{fl}}
\nodename{FL Functions}

Module \code{fl} defines the following functions.  For more information
about what they do, see the description of the equivalent C function
in the FORMS documentation:

\renewcommand{\indexsubitem}{(in module fl)}
\begin{funcdesc}{make_form}{type\, width\, height}
Create a form with given type, width and height.  This returns a
\dfn{form} object, whose methods are described below.
\end{funcdesc}

\begin{funcdesc}{do_forms}{}
The standard FORMS main loop.  Returns a Python object representing
the FORMS object needing interaction, or the special value
\code{FL.EVENT}.
\end{funcdesc}

\begin{funcdesc}{check_forms}{}
Check for FORMS events.  Returns what \code{do_forms} above returns,
or \code{None} if there is no event that immediately needs
interaction.
\end{funcdesc}

\begin{funcdesc}{set_event_call_back}{function}
Set the event callback function.
\end{funcdesc}

\begin{funcdesc}{set_graphics_mode}{rgbmode\, doublebuffering}
Set the graphics modes.
\end{funcdesc}

\begin{funcdesc}{get_rgbmode}{}
Return the current rgb mode.  This is the value of the C global
variable \code{fl_rgbmode}.
\end{funcdesc}

\begin{funcdesc}{show_message}{str1\, str2\, str3}
Show a dialog box with a three-line message and an OK button.
\end{funcdesc}

\begin{funcdesc}{show_question}{str1\, str2\, str3}
Show a dialog box with a three-line message and YES and NO buttons.
It returns \code{1} if the user pressed YES, \code{0} if NO.
\end{funcdesc}

\begin{funcdesc}{show_choice}{str1\, str2\, str3\, but1\optional{\, but2\,
but3}}
Show a dialog box with a three-line message and up to three buttons.
It returns the number of the button clicked by the user
(\code{1}, \code{2} or \code{3}).
\end{funcdesc}

\begin{funcdesc}{show_input}{prompt\, default}
Show a dialog box with a one-line prompt message and text field in
which the user can enter a string.  The second argument is the default
input string.  It returns the string value as edited by the user.
\end{funcdesc}

\begin{funcdesc}{show_file_selector}{message\, directory\, pattern\, default}
Show a dialog box in which the user can select a file.  It returns
the absolute filename selected by the user, or \code{None} if the user
presses Cancel.
\end{funcdesc}

\begin{funcdesc}{get_directory}{}
\funcline{get_pattern}{}
\funcline{get_filename}{}
These functions return the directory, pattern and filename (the tail
part only) selected by the user in the last \code{show_file_selector}
call.
\end{funcdesc}

\begin{funcdesc}{qdevice}{dev}
\funcline{unqdevice}{dev}
\funcline{isqueued}{dev}
\funcline{qtest}{}
\funcline{qread}{}
%\funcline{blkqread}{?}
\funcline{qreset}{}
\funcline{qenter}{dev\, val}
\funcline{get_mouse}{}
\funcline{tie}{button\, valuator1\, valuator2}
These functions are the FORMS interfaces to the corresponding GL
functions.  Use these if you want to handle some GL events yourself
when using \code{fl.do_events}.  When a GL event is detected that
FORMS cannot handle, \code{fl.do_forms()} returns the special value
\code{FL.EVENT} and you should call \code{fl.qread()} to read the
event from the queue.  Don't use the equivalent GL functions!
\end{funcdesc}

\begin{funcdesc}{color}{}
\funcline{mapcolor}{}
\funcline{getmcolor}{}
See the description in the FORMS documentation of \code{fl_color},
\code{fl_mapcolor} and \code{fl_getmcolor}.
\end{funcdesc}

\subsection{Form Objects}

Form objects (returned by \code{fl.make_form()} above) have the
following methods.  Each method corresponds to a C function whose name
is prefixed with \samp{fl_}; and whose first argument is a form
pointer; please refer to the official FORMS documentation for
descriptions.

All the \samp{add_{\rm \ldots}} functions return a Python object representing
the FORMS object.  Methods of FORMS objects are described below.  Most
kinds of FORMS object also have some methods specific to that kind;
these methods are listed here.

\begin{flushleft}
\renewcommand{\indexsubitem}{(form object method)}
\begin{funcdesc}{show_form}{placement\, bordertype\, name}
  Show the form.
\end{funcdesc}

\begin{funcdesc}{hide_form}{}
  Hide the form.
\end{funcdesc}

\begin{funcdesc}{redraw_form}{}
  Redraw the form.
\end{funcdesc}

\begin{funcdesc}{set_form_position}{x\, y}
Set the form's position.
\end{funcdesc}

\begin{funcdesc}{freeze_form}{}
Freeze the form.
\end{funcdesc}

\begin{funcdesc}{unfreeze_form}{}
  Unfreeze the form.
\end{funcdesc}

\begin{funcdesc}{activate_form}{}
  Activate the form.
\end{funcdesc}

\begin{funcdesc}{deactivate_form}{}
  Deactivate the form.
\end{funcdesc}

\begin{funcdesc}{bgn_group}{}
  Begin a new group of objects; return a group object.
\end{funcdesc}

\begin{funcdesc}{end_group}{}
  End the current group of objects.
\end{funcdesc}

\begin{funcdesc}{find_first}{}
  Find the first object in the form.
\end{funcdesc}

\begin{funcdesc}{find_last}{}
  Find the last object in the form.
\end{funcdesc}

%---

\begin{funcdesc}{add_box}{type\, x\, y\, w\, h\, name}
Add a box object to the form.
No extra methods.
\end{funcdesc}

\begin{funcdesc}{add_text}{type\, x\, y\, w\, h\, name}
Add a text object to the form.
No extra methods.
\end{funcdesc}

%\begin{funcdesc}{add_bitmap}{type\, x\, y\, w\, h\, name}
%Add a bitmap object to the form.
%\end{funcdesc}

\begin{funcdesc}{add_clock}{type\, x\, y\, w\, h\, name}
Add a clock object to the form. \\
Method:
\code{get_clock}.
\end{funcdesc}

%---

\begin{funcdesc}{add_button}{type\, x\, y\, w\, h\,  name}
Add a button object to the form. \\
Methods:
\code{get_button},
\code{set_button}.
\end{funcdesc}

\begin{funcdesc}{add_lightbutton}{type\, x\, y\, w\, h\, name}
Add a lightbutton object to the form. \\
Methods:
\code{get_button},
\code{set_button}.
\end{funcdesc}

\begin{funcdesc}{add_roundbutton}{type\, x\, y\, w\, h\, name}
Add a roundbutton object to the form. \\
Methods:
\code{get_button},
\code{set_button}.
\end{funcdesc}

%---

\begin{funcdesc}{add_slider}{type\, x\, y\, w\, h\, name}
Add a slider object to the form. \\
Methods:
\code{set_slider_value},
\code{get_slider_value},
\code{set_slider_bounds},
\code{get_slider_bounds},
\code{set_slider_return},
\code{set_slider_size},
\code{set_slider_precision},
\code{set_slider_step}.
\end{funcdesc}

\begin{funcdesc}{add_valslider}{type\, x\, y\, w\, h\, name}
Add a valslider object to the form. \\
Methods:
\code{set_slider_value},
\code{get_slider_value},
\code{set_slider_bounds},
\code{get_slider_bounds},
\code{set_slider_return},
\code{set_slider_size},
\code{set_slider_precision},
\code{set_slider_step}.
\end{funcdesc}

\begin{funcdesc}{add_dial}{type\, x\, y\, w\, h\, name}
Add a dial object to the form. \\
Methods:
\code{set_dial_value},
\code{get_dial_value},
\code{set_dial_bounds},
\code{get_dial_bounds}.
\end{funcdesc}

\begin{funcdesc}{add_positioner}{type\, x\, y\, w\, h\, name}
Add a positioner object to the form. \\
Methods:
\code{set_positioner_xvalue},
\code{set_positioner_yvalue},
\code{set_positioner_xbounds},
\code{set_positioner_ybounds},
\code{get_positioner_xvalue},
\code{get_positioner_yvalue},
\code{get_positioner_xbounds},
\code{get_positioner_ybounds}.
\end{funcdesc}

\begin{funcdesc}{add_counter}{type\, x\, y\, w\, h\, name}
Add a counter object to the form. \\
Methods:
\code{set_counter_value},
\code{get_counter_value},
\code{set_counter_bounds},
\code{set_counter_step},
\code{set_counter_precision},
\code{set_counter_return}.
\end{funcdesc}

%---

\begin{funcdesc}{add_input}{type\, x\, y\, w\, h\, name}
Add a input object to the form. \\
Methods:
\code{set_input},
\code{get_input},
\code{set_input_color},
\code{set_input_return}.
\end{funcdesc}

%---

\begin{funcdesc}{add_menu}{type\, x\, y\, w\, h\, name}
Add a menu object to the form. \\
Methods:
\code{set_menu},
\code{get_menu},
\code{addto_menu}.
\end{funcdesc}

\begin{funcdesc}{add_choice}{type\, x\, y\, w\, h\, name}
Add a choice object to the form. \\
Methods:
\code{set_choice},
\code{get_choice},
\code{clear_choice},
\code{addto_choice},
\code{replace_choice},
\code{delete_choice},
\code{get_choice_text},
\code{set_choice_fontsize},
\code{set_choice_fontstyle}.
\end{funcdesc}

\begin{funcdesc}{add_browser}{type\, x\, y\, w\, h\, name}
Add a browser object to the form. \\
Methods:
\code{set_browser_topline},
\code{clear_browser},
\code{add_browser_line},
\code{addto_browser},
\code{insert_browser_line},
\code{delete_browser_line},
\code{replace_browser_line},
\code{get_browser_line},
\code{load_browser},
\code{get_browser_maxline},
\code{select_browser_line},
\code{deselect_browser_line},
\code{deselect_browser},
\code{isselected_browser_line},
\code{get_browser},
\code{set_browser_fontsize},
\code{set_browser_fontstyle},
\code{set_browser_specialkey}.
\end{funcdesc}

%---

\begin{funcdesc}{add_timer}{type\, x\, y\, w\, h\, name}
Add a timer object to the form. \\
Methods:
\code{set_timer},
\code{get_timer}.
\end{funcdesc}
\end{flushleft}

Form objects have the following data attributes; see the FORMS
documentation:

\begin{tableiii}{|l|c|l|}{code}{Name}{Type}{Meaning}
  \lineiii{window}{int (read-only)}{GL window id}
  \lineiii{w}{float}{form width}
  \lineiii{h}{float}{form height}
  \lineiii{x}{float}{form x origin}
  \lineiii{y}{float}{form y origin}
  \lineiii{deactivated}{int}{nonzero if form is deactivated}
  \lineiii{visible}{int}{nonzero if form is visible}
  \lineiii{frozen}{int}{nonzero if form is frozen}
  \lineiii{doublebuf}{int}{nonzero if double buffering on}
\end{tableiii}

\subsection{FORMS Objects}

Besides methods specific to particular kinds of FORMS objects, all
FORMS objects also have the following methods:

\renewcommand{\indexsubitem}{(FORMS object method)}
\begin{funcdesc}{set_call_back}{function\, argument}
Set the object's callback function and argument.  When the object
needs interaction, the callback function will be called with two
arguments: the object, and the callback argument.  (FORMS objects
without a callback function are returned by \code{fl.do_forms()} or
\code{fl.check_forms()} when they need interaction.)  Call this method
without arguments to remove the callback function.
\end{funcdesc}

\begin{funcdesc}{delete_object}{}
  Delete the object.
\end{funcdesc}

\begin{funcdesc}{show_object}{}
  Show the object.
\end{funcdesc}

\begin{funcdesc}{hide_object}{}
  Hide the object.
\end{funcdesc}

\begin{funcdesc}{redraw_object}{}
  Redraw the object.
\end{funcdesc}

\begin{funcdesc}{freeze_object}{}
  Freeze the object.
\end{funcdesc}

\begin{funcdesc}{unfreeze_object}{}
  Unfreeze the object.
\end{funcdesc}

%\begin{funcdesc}{handle_object}{} XXX
%\end{funcdesc}

%\begin{funcdesc}{handle_object_direct}{} XXX
%\end{funcdesc}

FORMS objects have these data attributes; see the FORMS documentation:

\begin{tableiii}{|l|c|l|}{code}{Name}{Type}{Meaning}
  \lineiii{objclass}{int (read-only)}{object class}
  \lineiii{type}{int (read-only)}{object type}
  \lineiii{boxtype}{int}{box type}
  \lineiii{x}{float}{x origin}
  \lineiii{y}{float}{y origin}
  \lineiii{w}{float}{width}
  \lineiii{h}{float}{height}
  \lineiii{col1}{int}{primary color}
  \lineiii{col2}{int}{secondary color}
  \lineiii{align}{int}{alignment}
  \lineiii{lcol}{int}{label color}
  \lineiii{lsize}{float}{label font size}
  \lineiii{label}{string}{label string}
  \lineiii{lstyle}{int}{label style}
  \lineiii{pushed}{int (read-only)}{(see FORMS docs)}
  \lineiii{focus}{int (read-only)}{(see FORMS docs)}
  \lineiii{belowmouse}{int (read-only)}{(see FORMS docs)}
  \lineiii{frozen}{int (read-only)}{(see FORMS docs)}
  \lineiii{active}{int (read-only)}{(see FORMS docs)}
  \lineiii{input}{int (read-only)}{(see FORMS docs)}
  \lineiii{visible}{int (read-only)}{(see FORMS docs)}
  \lineiii{radio}{int (read-only)}{(see FORMS docs)}
  \lineiii{automatic}{int (read-only)}{(see FORMS docs)}
\end{tableiii}

\section{Standard Module \sectcode{FL}}
\nodename{FL (uppercase)}
\stmodindex{FL}

This module defines symbolic constants needed to use the built-in
module \code{fl} (see above); they are equivalent to those defined in
the C header file \file{<forms.h>} except that the name prefix
\samp{FL_} is omitted.  Read the module source for a complete list of
the defined names.  Suggested use:

\bcode\begin{verbatim}
import fl
from FL import *
\end{verbatim}\ecode

\section{Standard Module \sectcode{flp}}
\stmodindex{flp}

This module defines functions that can read form definitions created
by the `form designer' (\code{fdesign}) program that comes with the
FORMS library (see module \code{fl} above).

For now, see the file \file{flp.doc} in the Python library source
directory for a description.

XXX A complete description should be inserted here!

\section{Built-in Module \sectcode{fm}}
\bimodindex{fm}

This module provides access to the IRIS {\em Font Manager} library.
It is available only on Silicon Graphics machines.
See also: 4Sight User's Guide, Section 1, Chapter 5: Using the IRIS
Font Manager.

This is not yet a full interface to the IRIS Font Manager.
Among the unsupported features are: matrix operations; cache
operations; character operations (use string operations instead); some
details of font info; individual glyph metrics; and printer matching.

It supports the following operations:

\renewcommand{\indexsubitem}{(in module fm)}
\begin{funcdesc}{init}{}
Initialization function.
Calls \code{fminit()}.
It is normally not necessary to call this function, since it is called
automatically the first time the \code{fm} module is imported.
\end{funcdesc}

\begin{funcdesc}{findfont}{fontname}
Return a font handle object.
Calls \code{fmfindfont(\var{fontname})}.
\end{funcdesc}

\begin{funcdesc}{enumerate}{}
Returns a list of available font names.
This is an interface to \code{fmenumerate()}.
\end{funcdesc}

\begin{funcdesc}{prstr}{string}
Render a string using the current font (see the \code{setfont()} font
handle method below).
Calls \code{fmprstr(\var{string})}.
\end{funcdesc}

\begin{funcdesc}{setpath}{string}
Sets the font search path.
Calls \code{fmsetpath(string)}.
(XXX Does not work!?!)
\end{funcdesc}

\begin{funcdesc}{fontpath}{}
Returns the current font search path.
\end{funcdesc}

Font handle objects support the following operations:

\renewcommand{\indexsubitem}{(font handle method)}
\begin{funcdesc}{scalefont}{factor}
Returns a handle for a scaled version of this font.
Calls \code{fmscalefont(\var{fh}, \var{factor})}.
\end{funcdesc}

\begin{funcdesc}{setfont}{}
Makes this font the current font.
Note: the effect is undone silently when the font handle object is
deleted.
Calls \code{fmsetfont(\var{fh})}.
\end{funcdesc}

\begin{funcdesc}{getfontname}{}
Returns this font's name.
Calls \code{fmgetfontname(\var{fh})}.
\end{funcdesc}

\begin{funcdesc}{getcomment}{}
Returns the comment string associated with this font.
Raises an exception if there is none.
Calls \code{fmgetcomment(\var{fh})}.
\end{funcdesc}

\begin{funcdesc}{getfontinfo}{}
Returns a tuple giving some pertinent data about this font.
This is an interface to \code{fmgetfontinfo()}.
The returned tuple contains the following numbers:
{\tt(\var{printermatched}, \var{fixed_width}, \var{xorig}, \var{yorig},
\var{xsize}, \var{ysize}, \var{height}, \var{nglyphs})}.
\end{funcdesc}

\begin{funcdesc}{getstrwidth}{string}
Returns the width, in pixels, of the string when drawn in this font.
Calls \code{fmgetstrwidth(\var{fh}, \var{string})}.
\end{funcdesc}

\section{Built-in Module \sectcode{gl}}
\bimodindex{gl}

This module provides access to the Silicon Graphics
{\em Graphics Library}.
It is available only on Silicon Graphics machines.

\strong{Warning:}
Some illegal calls to the GL library cause the Python interpreter to dump
core.
In particular, the use of most GL calls is unsafe before the first
window is opened.

The module is too large to document here in its entirety, but the
following should help you to get started.
The parameter conventions for the C functions are translated to Python as
follows:

\begin{itemize}
\item
All (short, long, unsigned) int values are represented by Python
integers.
\item
All float and double values are represented by Python floating point
numbers.
In most cases, Python integers are also allowed.
\item
All arrays are represented by one-dimensional Python lists.
In most cases, tuples are also allowed.
\item
\begin{sloppypar}
All string and character arguments are represented by Python strings,
for instance,
\code{winopen('Hi There!')}
and
\code{rotate(900, 'z')}.
\end{sloppypar}
\item
All (short, long, unsigned) integer arguments or return values that are
only used to specify the length of an array argument are omitted.
For example, the C call

\bcode\begin{verbatim}
lmdef(deftype, index, np, props)
\end{verbatim}\ecode

is translated to Python as

\bcode\begin{verbatim}
lmdef(deftype, index, props)
\end{verbatim}\ecode

\item
Output arguments are omitted from the argument list; they are
transmitted as function return values instead.
If more than one value must be returned, the return value is a tuple.
If the C function has both a regular return value (that is not omitted
because of the previous rule) and an output argument, the return value
comes first in the tuple.
Examples: the C call

\bcode\begin{verbatim}
getmcolor(i, &red, &green, &blue)
\end{verbatim}\ecode

is translated to Python as

\bcode\begin{verbatim}
red, green, blue = getmcolor(i)
\end{verbatim}\ecode

\end{itemize}

The following functions are non-standard or have special argument
conventions:

\renewcommand{\indexsubitem}{(in module gl)}
\begin{funcdesc}{varray}{argument}
%JHXXX the argument-argument added
Equivalent to but faster than a number of
\code{v3d()}
calls.
The \var{argument} is a list (or tuple) of points.
Each point must be a tuple of coordinates
\code{(\var{x}, \var{y}, \var{z})} or \code{(\var{x}, \var{y})}.
The points may be 2- or 3-dimensional but must all have the
same dimension.
Float and int values may be mixed however.
The points are always converted to 3D double precision points
by assuming \code{\var{z} = 0.0} if necessary (as indicated in the man page),
and for each point
\code{v3d()}
is called.
\end{funcdesc}

\begin{funcdesc}{nvarray}{}
Equivalent to but faster than a number of
\code{n3f}
and
\code{v3f}
calls.
The argument is an array (list or tuple) of pairs of normals and points.
Each pair is a tuple of a point and a normal for that point.
Each point or normal must be a tuple of coordinates
\code{(\var{x}, \var{y}, \var{z})}.
Three coordinates must be given.
Float and int values may be mixed.
For each pair,
\code{n3f()}
is called for the normal, and then
\code{v3f()}
is called for the point.
\end{funcdesc}

\begin{funcdesc}{vnarray}{}
Similar to 
\code{nvarray()}
but the pairs have the point first and the normal second.
\end{funcdesc}

\begin{funcdesc}{nurbssurface}{s_k\, t_k\, ctl\, s_ord\, t_ord\, type}
% XXX s_k[], t_k[], ctl[][]
%\itembreak
Defines a nurbs surface.
The dimensions of
\code{\var{ctl}[][]}
are computed as follows:
\code{[len(\var{s_k}) - \var{s_ord}]},
\code{[len(\var{t_k}) - \var{t_ord}]}.
\end{funcdesc}

\begin{funcdesc}{nurbscurve}{knots\, ctlpoints\, order\, type}
Defines a nurbs curve.
The length of ctlpoints is
\code{len(\var{knots}) - \var{order}}.
\end{funcdesc}

\begin{funcdesc}{pwlcurve}{points\, type}
Defines a piecewise-linear curve.
\var{points}
is a list of points.
\var{type}
must be
\code{N_ST}.
\end{funcdesc}

\begin{funcdesc}{pick}{n}
\funcline{select}{n}
The only argument to these functions specifies the desired size of the
pick or select buffer.
\end{funcdesc}

\begin{funcdesc}{endpick}{}
\funcline{endselect}{}
These functions have no arguments.
They return a list of integers representing the used part of the
pick/select buffer.
No method is provided to detect buffer overrun.
\end{funcdesc}

Here is a tiny but complete example GL program in Python:

\bcode\begin{verbatim}
import gl, GL, time

def main():
    gl.foreground()
    gl.prefposition(500, 900, 500, 900)
    w = gl.winopen('CrissCross')
    gl.ortho2(0.0, 400.0, 0.0, 400.0)
    gl.color(GL.WHITE)
    gl.clear()
    gl.color(GL.RED)
    gl.bgnline()
    gl.v2f(0.0, 0.0)
    gl.v2f(400.0, 400.0)
    gl.endline()
    gl.bgnline()
    gl.v2f(400.0, 0.0)
    gl.v2f(0.0, 400.0)
    gl.endline()
    time.sleep(5)

main()
\end{verbatim}\ecode

\section{Standard Modules \sectcode{GL} and \sectcode{DEVICE}}
\nodename{GL and DEVICE}
\stmodindex{GL}
\stmodindex{DEVICE}

These modules define the constants used by the Silicon Graphics
{\em Graphics Library}
that C programmers find in the header files
\file{<gl/gl.h>}
and
\file{<gl/device.h>}.
Read the module source files for details.

\section{Built-in Module \sectcode{imgfile}}
\bimodindex{imgfile}

The imgfile module allows python programs to access SGI imglib image
files (also known as \file{.rgb} files).  The module is far from
complete, but is provided anyway since the functionality that there is
is enough in some cases.  Currently, colormap files are not supported.

The module defines the following variables and functions:

\renewcommand{\indexsubitem}{(in module imgfile)}
\begin{excdesc}{error}
This exception is raised on all errors, such as unsupported file type, etc.
\end{excdesc}

\begin{funcdesc}{getsizes}{file}
This function returns a tuple \code{(\var{x}, \var{y}, \var{z})} where
\var{x} and \var{y} are the size of the image in pixels and
\var{z} is the number of
bytes per pixel. Only 3 byte RGB pixels and 1 byte greyscale pixels
are currently supported.
\end{funcdesc}

\begin{funcdesc}{read}{file}
This function reads and decodes the image on the specified file, and
returns it as a python string. The string has either 1 byte greyscale
pixels or 4 byte RGBA pixels. The bottom left pixel is the first in
the string. This format is suitable to pass to \code{gl.lrectwrite},
for instance.
\end{funcdesc}

\begin{funcdesc}{readscaled}{file\, x\, y\, filter\optional{\, blur}}
This function is identical to read but it returns an image that is
scaled to the given \var{x} and \var{y} sizes. If the \var{filter} and
\var{blur} parameters are omitted scaling is done by
simply dropping or duplicating pixels, so the result will be less than
perfect, especially for computer-generated images.

Alternatively, you can specify a filter to use to smoothen the image
after scaling. The filter forms supported are \code{'impulse'},
\code{'box'}, \code{'triangle'}, \code{'quadratic'} and
\code{'gaussian'}. If a filter is specified \var{blur} is an optional
parameter specifying the blurriness of the filter. It defaults to \code{1.0}.

\code{readscaled} makes no
attempt to keep the aspect ratio correct, so that is the users'
responsibility.
\end{funcdesc}

\begin{funcdesc}{ttob}{flag}
This function sets a global flag which defines whether the scan lines
of the image are read or written from bottom to top (flag is zero,
compatible with SGI GL) or from top to bottom(flag is one,
compatible with X).  The default is zero.
\end{funcdesc}

\begin{funcdesc}{write}{file\, data\, x\, y\, z}
This function writes the RGB or greyscale data in \var{data} to image
file \var{file}. \var{x} and \var{y} give the size of the image,
\var{z} is 1 for 1 byte greyscale images or 3 for RGB images (which are
stored as 4 byte values of which only the lower three bytes are used).
These are the formats returned by \code{gl.lrectread}.
\end{funcdesc}

%\section{Standard Module \sectcode{panel}}
\stmodindex{panel}

\strong{Please note:} The FORMS library, to which the \code{fl} module described
above interfaces, is a simpler and more accessible user interface
library for use with GL than the Panel Module (besides also being by a
Dutch author).

This module should be used instead of the built-in module
\code{pnl}
to interface with the
{\em Panel Library}.

The module is too large to document here in its entirety.
One interesting function:

\renewcommand{\indexsubitem}{(in module panel)}
\begin{funcdesc}{defpanellist}{filename}
Parses a panel description file containing S-expressions written by the
{\em Panel Editor}
that accompanies the Panel Library and creates the described panels.
It returns a list of panel objects.
\end{funcdesc}

\strong{Warning:}
the Python interpreter will dump core if you don't create a GL window
before calling
\code{panel.mkpanel()}
or
\code{panel.defpanellist()}.

\section{Standard Module \sectcode{panelparser}}
\stmodindex{panelparser}

This module defines a self-contained parser for S-expressions as output
by the Panel Editor (which is written in Scheme so it can't help writing
S-expressions).
The relevant function is
\code{panelparser.parse_file(\var{file})}
which has a file object (not a filename!) as argument and returns a list
of parsed S-expressions.
Each S-expression is converted into a Python list, with atoms converted
to Python strings and sub-expressions (recursively) to Python lists.
For more details, read the module file.
% XXXXJH should be funcdesc, I think

\section{Built-in Module \sectcode{pnl}}
\bimodindex{pnl}

This module provides access to the
{\em Panel Library}
built by NASA Ames (to get it, send e-mail to
{\tt panel-request@nas.nasa.gov}).
All access to it should be done through the standard module
\code{panel},
which transparantly exports most functions from
\code{pnl}
but redefines
\code{pnl.dopanel()}.

\strong{Warning:}
the Python interpreter will dump core if you don't create a GL window
before calling
\code{pnl.mkpanel()}.

The module is too large to document here in its entirety.


\chapter{SunOS Specific Services}

The modules described in this chapter provide interfaces to features
that are unique to the SunOS operating system (versions 4 and 5; the
latter is also known as Solaris version 2).

\section{Built-in Module \sectcode{sunaudiodev}}
\bimodindex{sunaudiodev}

This module allows you to access the sun audio interface. The sun
audio hardware is capable of recording and playing back audio data
in U-LAW format with a sample rate of 8K per second. A full
description can be gotten with \samp{man audio}.

The module defines the following variables and functions:

\renewcommand{\indexsubitem}{(in module sunaudiodev)}
\begin{excdesc}{error}
This exception is raised on all errors. The argument is a string
describing what went wrong.
\end{excdesc}

\begin{funcdesc}{open}{mode}
This function opens the audio device and returns a sun audio device
object. This object can then be used to do I/O on. The \var{mode} parameter
is one of \code{'r'} for record-only access, \code{'w'} for play-only
access, \code{'rw'} for both and \code{'control'} for access to the
control device. Since only one process is allowed to have the recorder
or player open at the same time it is a good idea to open the device
only for the activity needed. See the audio manpage for details.
\end{funcdesc}

\subsection{Audio Device Objects}

The audio device objects are returned by \code{open} define the
following methods (except \code{control} objects which only provide
getinfo, setinfo and drain):

\renewcommand{\indexsubitem}{(audio device method)}

\begin{funcdesc}{close}{}
This method explicitly closes the device. It is useful in situations
where deleting the object does not immediately close it since there
are other references to it. A closed device should not be used again.
\end{funcdesc}

\begin{funcdesc}{drain}{}
This method waits until all pending output is processed and then returns.
Calling this method is often not necessary: destroying the object will
automatically close the audio device and this will do an implicit drain.
\end{funcdesc}

\begin{funcdesc}{flush}{}
This method discards all pending output. It can be used avoid the
slow response to a user's stop request (due to buffering of up to one
second of sound).
\end{funcdesc}

\begin{funcdesc}{getinfo}{}
This method retrieves status information like input and output volume,
etc. and returns it in the form of
an audio status object. This object has no methods but it contains a
number of attributes describing the current device status. The names
and meanings of the attributes are described in
\file{/usr/include/sun/audioio.h} and in the audio man page. Member names
are slightly different from their C counterparts: a status object is
only a single structure. Members of the \code{play} substructure have
\samp{o_} prepended to their name and members of the \code{record}
structure have \samp{i_}. So, the C member \code{play.sample_rate} is
accessed as \code{o_sample_rate}, \code{record.gain} as \code{i_gain}
and \code{monitor_gain} plainly as \code{monitor_gain}.
\end{funcdesc}

\begin{funcdesc}{ibufcount}{}
This method returns the number of samples that are buffered on the
recording side, i.e.
the program will not block on a \code{read} call of so many samples.
\end{funcdesc}

\begin{funcdesc}{obufcount}{}
This method returns the number of samples buffered on the playback
side. Unfortunately, this number cannot be used to determine a number
of samples that can be written without blocking since the kernel
output queue length seems to be variable.
\end{funcdesc}

\begin{funcdesc}{read}{size}
This method reads \var{size} samples from the audio input and returns
them as a python string. The function blocks until enough data is available.
\end{funcdesc}

\begin{funcdesc}{setinfo}{status}
This method sets the audio device status parameters. The \var{status}
parameter is an device status object as returned by \code{getinfo} and
possibly modified by the program.
\end{funcdesc}

\begin{funcdesc}{write}{samples}
Write is passed a python string containing audio samples to be played.
If there is enough buffer space free it will immediately return,
otherwise it will block.
\end{funcdesc}

There is a companion module, \code{SUNAUDIODEV}, which defines useful
symbolic constants like \code{MIN_GAIN}, \code{MAX_GAIN},
\code{SPEAKER}, etc. The names of
the constants are the same names as used in the C include file
\file{<sun/audioio.h>}, with the leading string \samp{AUDIO_} stripped.

Useability of the control device is limited at the moment, since there
is no way to use the ``wait for something to happen'' feature the
device provides.
			% SUNOS ONLY

\documentstyle[twoside,11pt,myformat]{report}

% NOTE: this file controls which chapters/sections of the library
% manual are actually printed.  It is easy to customize your manual
% by commenting out sections that you're not interested in.

\title{Python Library Reference}

\author{
	Guido van Rossum \\
	Corporation for National Research Initiatives (CNRI) \\
	1895 Preston White Drive, Reston, Va 20191, USA \\
	E-mail: {\tt guido@CNRI.Reston.Va.US}, {\tt guido@python.org}
}

\date{October 25, 1996 \\ Release 1.4} % XXX update before release!


\makeindex			% tell \index to actually write the .idx file


\begin{document}

\pagenumbering{roman}

\maketitle

Copyright \copyright{} 1991-1995 by Stichting Mathematisch Centrum,
Amsterdam, The Netherlands.

\begin{center}
All Rights Reserved
\end{center}

Permission to use, copy, modify, and distribute this software and its
documentation for any purpose and without fee is hereby granted,
provided that the above copyright notice appear in all copies and that
both that copyright notice and this permission notice appear in
supporting documentation, and that the names of Stichting Mathematisch
Centrum or CWI or Corporation for National Research Initiatives or
CNRI not be used in advertising or publicity pertaining to
distribution of the software without specific, written prior
permission.

While CWI is the initial source for this software, a modified version
is made available by the Corporation for National Research Initiatives
(CNRI) at the Internet address ftp://ftp.python.org.

STICHTING MATHEMATISCH CENTRUM AND CNRI DISCLAIM ALL WARRANTIES WITH
REGARD TO THIS SOFTWARE, INCLUDING ALL IMPLIED WARRANTIES OF
MERCHANTABILITY AND FITNESS, IN NO EVENT SHALL STICHTING MATHEMATISCH
CENTRUM OR CNRI BE LIABLE FOR ANY SPECIAL, INDIRECT OR CONSEQUENTIAL
DAMAGES OR ANY DAMAGES WHATSOEVER RESULTING FROM LOSS OF USE, DATA OR
PROFITS, WHETHER IN AN ACTION OF CONTRACT, NEGLIGENCE OR OTHER
TORTIOUS ACTION, ARISING OUT OF OR IN CONNECTION WITH THE USE OR
PERFORMANCE OF THIS SOFTWARE.


\begin{abstract}

\noindent
Python is an extensible, interpreted, object-oriented programming
language.  It supports a wide range of applications, from simple text
processing scripts to interactive WWW browsers.

While the {\em Python Reference Manual} describes the exact syntax and
semantics of the language, it does not describe the standard library
that is distributed with the language, and which greatly enhances its
immediate usability.  This library contains built-in modules (written
in C) that provide access to system functionality such as file I/O
that would otherwise be inaccessible to Python programmers, as well as
modules written in Python that provide standardized solutions for many
problems that occur in everyday programming.  Some of these modules
are explicitly designed to encourage and enhance the portability of
Python programs.

This library reference manual documents Python's standard library, as
well as many optional library modules (which may or may not be
available, depending on whether the underlying platform supports them
and on the configuration choices made at compile time).  It also
documents the standard types of the language and its built-in
functions and exceptions, many of which are not or incompletely
documented in the Reference Manual.

This manual assumes basic knowledge about the Python language.  For an
informal introduction to Python, see the {\em Python Tutorial}; the
Python Reference Manual remains the highest authority on syntactic and
semantic questions.  Finally, the manual entitled {\em Extending and
Embedding the Python Interpreter} describes how to add new extensions
to Python and how to embed it in other applications.

\end{abstract}

\pagebreak

{
\parskip = 0mm
\tableofcontents
}

\pagebreak

\pagenumbering{arabic}

				% Chapter title:

\chapter{Introduction}

The ``Python library'' contains several different kinds of components.

It contains data types that would normally be considered part of the
``core'' of a language, such as numbers and lists.  For these types,
the Python language core defines the form of literals and places some
constraints on their semantics, but does not fully define the
semantics.  (On the other hand, the language core does define
syntactic properties like the spelling and priorities of operators.)

The library also contains built-in functions and exceptions ---
objects that can be used by all Python code without the need of an
\code{import} statement.  Some of these are defined by the core
language, but many are not essential for the core semantics and are
only described here.

The bulk of the library, however, consists of a collection of modules.
There are many ways to dissect this collection.  Some modules are
written in C and built in to the Python interpreter; others are
written in Python and imported in source form.  Some modules provide
interfaces that are highly specific to Python, like printing a stack
trace; some provide interfaces that are specific to particular
operating systems, like socket I/O; others provide interfaces that are
specific to a particular application domain, like the World-Wide Web.
Some modules are avaiable in all versions and ports of Python; others
are only available when the underlying system supports or requires
them; yet others are available only when a particular configuration
option was chosen at the time when Python was compiled and installed.

This manual is organized ``from the inside out'': it first describes
the built-in data types, then the built-in functions and exceptions,
and finally the modules, grouped in chapters of related modules.  The
ordering of the chapters as well as the ordering of the modules within
each chapter is roughly from most relevant to least important.

This means that if you start reading this manual from the start, and
skip to the next chapter when you get bored, you will get a reasonable
overview of the available modules and application areas that are
supported by the Python library.  Of course, you don't \emph{have} to
read it like a novel --- you can also browse the table of contents (in
front of the manual), or look for a specific function, module or term
in the index (in the back).  And finally, if you enjoy learning about
random subjects, you choose a random page number (see module
\code{rand}) and read a section or two.

Let the show begin!
		% Introduction

\chapter{Built-in Types, Exceptions and Functions}

\nodename{Built-in Objects}

Names for built-in exceptions and functions are found in a separate
symbol table.  This table is searched last when the interpreter looks
up the meaning of a name, so local and global
user-defined names can override built-in names.  Built-in types are
described together here for easy reference.%
\footnote{Most descriptions sorely lack explanations of the exceptions
	that may be raised --- this will be fixed in a future version of
	this manual.}
\indexii{built-in}{types}
\indexii{built-in}{exceptions}
\indexii{built-in}{functions}
\index{symbol table}
\bifuncindex{type}

The tables in this chapter document the priorities of operators by
listing them in order of ascending priority (within a table) and
grouping operators that have the same priority in the same box.
Binary operators of the same priority group from left to right.
(Unary operators group from right to left, but there you have no real
choice.)  See Chapter 5 of the Python Reference Manual for the
complete picture on operator priorities.
			% Built-in Types, Exceptions and Functions
\section{Built-in Types}

The following sections describe the standard types that are built into
the interpreter.  These are the numeric types, sequence types, and
several others, including types themselves.  There is no explicit
Boolean type; use integers instead.
\indexii{built-in}{types}
\indexii{Boolean}{type}

Some operations are supported by several object types; in particular,
all objects can be compared, tested for truth value, and converted to
a string (with the \code{`{\rm \ldots}`} notation).  The latter conversion is
implicitly used when an object is written by the \code{print} statement.
\stindex{print}

\subsection{Truth Value Testing}

Any object can be tested for truth value, for use in an \code{if} or
\code{while} condition or as operand of the Boolean operations below.
The following values are considered false:
\stindex{if}
\stindex{while}
\indexii{truth}{value}
\indexii{Boolean}{operations}
\index{false}

\begin{itemize}
\renewcommand{\indexsubitem}{(Built-in object)}

\item	\code{None}
	\ttindex{None}

\item	zero of any numeric type, e.g., \code{0}, \code{0L}, \code{0.0}.

\item	any empty sequence, e.g., \code{''}, \code{()}, \code{[]}.

\item	any empty mapping, e.g., \code{\{\}}.

\item	instances of user-defined classes, if the class defines a
	\code{__nonzero__()} or \code{__len__()} method, when that
	method returns zero.

\end{itemize}

All other values are considered true --- so objects of many types are
always true.
\index{true}

Operations and built-in functions that have a Boolean result always
return \code{0} for false and \code{1} for true, unless otherwise
stated.  (Important exception: the Boolean operations \samp{or} and
\samp{and} always return one of their operands.)

\subsection{Boolean Operations}

These are the Boolean operations, ordered by ascending priority:
\indexii{Boolean}{operations}

\begin{tableiii}{|c|l|c|}{code}{Operation}{Result}{Notes}
  \lineiii{\var{x} or \var{y}}{if \var{x} is false, then \var{y}, else \var{x}}{(1)}
  \hline
  \lineiii{\var{x} and \var{y}}{if \var{x} is false, then \var{x}, else \var{y}}{(1)}
  \hline
  \lineiii{not \var{x}}{if \var{x} is false, then \code{1}, else \code{0}}{(2)}
\end{tableiii}
\opindex{and}
\opindex{or}
\opindex{not}

\noindent
Notes:

\begin{description}

\item[(1)]
These only evaluate their second argument if needed for their outcome.

\item[(2)]
\samp{not} has a lower priority than non-Boolean operators, so e.g.
\code{not a == b} is interpreted as \code{not(a == b)}, and
\code{a == not b} is a syntax error.

\end{description}

\subsection{Comparisons}

Comparison operations are supported by all objects.  They all have the
same priority (which is higher than that of the Boolean operations).
Comparisons can be chained arbitrarily, e.g. \code{x < y <= z} is
equivalent to \code{x < y and y <= z}, except that \code{y} is
evaluated only once (but in both cases \code{z} is not evaluated at
all when \code{x < y} is found to be false).
\indexii{chaining}{comparisons}

This table summarizes the comparison operations:

\begin{tableiii}{|c|l|c|}{code}{Operation}{Meaning}{Notes}
  \lineiii{<}{strictly less than}{}
  \lineiii{<=}{less than or equal}{}
  \lineiii{>}{strictly greater than}{}
  \lineiii{>=}{greater than or equal}{}
  \lineiii{==}{equal}{}
  \lineiii{<>}{not equal}{(1)}
  \lineiii{!=}{not equal}{(1)}
  \lineiii{is}{object identity}{}
  \lineiii{is not}{negated object identity}{}
\end{tableiii}
\indexii{operator}{comparison}
\opindex{==} % XXX *All* others have funny characters < ! >
\opindex{is}
\opindex{is not}

\noindent
Notes:

\begin{description}

\item[(1)]
\code{<>} and \code{!=} are alternate spellings for the same operator.
(I couldn't choose between \ABC{} and \C{}! :-)
\indexii{\ABC{}}{language}
\indexii{\C{}}{language}

\end{description}

Objects of different types, except different numeric types, never
compare equal; such objects are ordered consistently but arbitrarily
(so that sorting a heterogeneous array yields a consistent result).
Furthermore, some types (e.g., windows) support only a degenerate
notion of comparison where any two objects of that type are unequal.
Again, such objects are ordered arbitrarily but consistently.
\indexii{types}{numeric}
\indexii{objects}{comparing}

(Implementation note: objects of different types except numbers are
ordered by their type names; objects of the same types that don't
support proper comparison are ordered by their address.)

Two more operations with the same syntactic priority, \code{in} and
\code{not in}, are supported only by sequence types (below).
\opindex{in}
\opindex{not in}

\subsection{Numeric Types}

There are three numeric types: \dfn{plain integers}, \dfn{long integers}, and
\dfn{floating point numbers}.  Plain integers (also just called \dfn{integers})
are implemented using \code{long} in \C{}, which gives them at least 32
bits of precision.  Long integers have unlimited precision.  Floating
point numbers are implemented using \code{double} in \C{}.  All bets on
their precision are off unless you happen to know the machine you are
working with.
\indexii{numeric}{types}
\indexii{integer}{types}
\indexii{integer}{type}
\indexiii{long}{integer}{type}
\indexii{floating point}{type}
\indexii{\C{}}{language}

Numbers are created by numeric literals or as the result of built-in
functions and operators.  Unadorned integer literals (including hex
and octal numbers) yield plain integers.  Integer literals with an \samp{L}
or \samp{l} suffix yield long integers
(\samp{L} is preferred because \code{1l} looks too much like eleven!).
Numeric literals containing a decimal point or an exponent sign yield
floating point numbers.
\indexii{numeric}{literals}
\indexii{integer}{literals}
\indexiii{long}{integer}{literals}
\indexii{floating point}{literals}
\indexii{hexadecimal}{literals}
\indexii{octal}{literals}

Python fully supports mixed arithmetic: when a binary arithmetic
operator has operands of different numeric types, the operand with the
``smaller'' type is converted to that of the other, where plain
integer is smaller than long integer is smaller than floating point.
Comparisons between numbers of mixed type use the same rule.%
\footnote{As a consequence, the list \code{[1, 2]} is considered equal
	to \code{[1.0, 2.0]}, and similar for tuples.}
The functions \code{int()}, \code{long()} and \code{float()} can be used
to coerce numbers to a specific type.
\index{arithmetic}
\bifuncindex{int}
\bifuncindex{long}
\bifuncindex{float}

All numeric types support the following operations, sorted by
ascending priority (operations in the same box have the same
priority; all numeric operations have a higher priority than
comparison operations):

\begin{tableiii}{|c|l|c|}{code}{Operation}{Result}{Notes}
  \lineiii{\var{x} + \var{y}}{sum of \var{x} and \var{y}}{}
  \lineiii{\var{x} - \var{y}}{difference of \var{x} and \var{y}}{}
  \hline
  \lineiii{\var{x} * \var{y}}{product of \var{x} and \var{y}}{}
  \lineiii{\var{x} / \var{y}}{quotient of \var{x} and \var{y}}{(1)}
  \lineiii{\var{x} \%{} \var{y}}{remainder of \code{\var{x} / \var{y}}}{}
  \hline
  \lineiii{-\var{x}}{\var{x} negated}{}
  \lineiii{+\var{x}}{\var{x} unchanged}{}
  \hline
  \lineiii{abs(\var{x})}{absolute value of \var{x}}{}
  \lineiii{int(\var{x})}{\var{x} converted to integer}{(2)}
  \lineiii{long(\var{x})}{\var{x} converted to long integer}{(2)}
  \lineiii{float(\var{x})}{\var{x} converted to floating point}{}
  \lineiii{divmod(\var{x}, \var{y})}{the pair \code{(\var{x} / \var{y}, \var{x} \%{} \var{y})}}{(3)}
  \lineiii{pow(\var{x}, \var{y})}{\var{x} to the power \var{y}}{}
\end{tableiii}
\indexiii{operations on}{numeric}{types}

\noindent
Notes:
\begin{description}

\item[(1)]
For (plain or long) integer division, the result is an integer.
The result is always rounded towards minus infinity: 1/2 is 0, 
(-1)/2 is -1, 1/(-2) is -1, and (-1)/(-2) is 0.
\indexii{integer}{division}
\indexiii{long}{integer}{division}

\item[(2)]
Conversion from floating point to (long or plain) integer may round or
truncate as in \C{}; see functions \code{floor()} and \code{ceil()} in
module \code{math} for well-defined conversions.
\bifuncindex{floor}
\bifuncindex{ceil}
\indexii{numeric}{conversions}
\stmodindex{math}
\indexii{\C{}}{language}

\item[(3)]
See the section on built-in functions for an exact definition.

\end{description}
% XXXJH exceptions: overflow (when? what operations?) zerodivision

\subsubsection{Bit-string Operations on Integer Types}
\nodename{Bit-string Operations}

Plain and long integer types support additional operations that make
sense only for bit-strings.  Negative numbers are treated as their 2's
complement value (for long integers, this assumes a sufficiently large
number of bits that no overflow occurs during the operation).

The priorities of the binary bit-wise operations are all lower than
the numeric operations and higher than the comparisons; the unary
operation \samp{~} has the same priority as the other unary numeric
operations (\samp{+} and \samp{-}).

This table lists the bit-string operations sorted in ascending
priority (operations in the same box have the same priority):

\begin{tableiii}{|c|l|c|}{code}{Operation}{Result}{Notes}
  \lineiii{\var{x} | \var{y}}{bitwise \dfn{or} of \var{x} and \var{y}}{}
  \hline
  \lineiii{\var{x} \^{} \var{y}}{bitwise \dfn{exclusive or} of \var{x} and \var{y}}{}
  \hline
  \lineiii{\var{x} \&{} \var{y}}{bitwise \dfn{and} of \var{x} and \var{y}}{}
  \hline
  \lineiii{\var{x} << \var{n}}{\var{x} shifted left by \var{n} bits}{(1), (2)}
  \lineiii{\var{x} >> \var{n}}{\var{x} shifted right by \var{n} bits}{(1), (3)}
  \hline
  \hline
  \lineiii{\~\var{x}}{the bits of \var{x} inverted}{}
\end{tableiii}
\indexiii{operations on}{integer}{types}
\indexii{bit-string}{operations}
\indexii{shifting}{operations}
\indexii{masking}{operations}

\noindent
Notes:
\begin{description}
\item[(1)] Negative shift counts are illegal.
\item[(2)] A left shift by \var{n} bits is equivalent to
multiplication by \code{pow(2, \var{n})} without overflow check.
\item[(3)] A right shift by \var{n} bits is equivalent to
division by \code{pow(2, \var{n})} without overflow check.
\end{description}

\subsection{Sequence Types}

There are three sequence types: strings, lists and tuples.

Strings literals are written in single or double quotes:
\code{'xyzzy'}, \code{"frobozz"}.  See Chapter 2 of the Python
Reference Manual for more about string literals.  Lists are
constructed with square brackets, separating items with commas:
\code{[a, b, c]}.  Tuples are constructed by the comma operator (not
within square brackets), with or without enclosing parentheses, but an
empty tuple must have the enclosing parentheses, e.g.,
\code{a, b, c} or \code{()}.  A single item tuple must have a trailing
comma, e.g., \code{(d,)}.
\indexii{sequence}{types}
\indexii{string}{type}
\indexii{tuple}{type}
\indexii{list}{type}

Sequence types support the following operations.  The \samp{in} and
\samp{not\,in} operations have the same priorities as the comparison
operations.  The \samp{+} and \samp{*} operations have the same
priority as the corresponding numeric operations.\footnote{They must
have since the parser can't tell the type of the operands.}

This table lists the sequence operations sorted in ascending priority
(operations in the same box have the same priority).  In the table,
\var{s} and \var{t} are sequences of the same type; \var{n}, \var{i}
and \var{j} are integers:

\begin{tableiii}{|c|l|c|}{code}{Operation}{Result}{Notes}
  \lineiii{\var{x} in \var{s}}{\code{1} if an item of \var{s} is equal to \var{x}, else \code{0}}{}
  \lineiii{\var{x} not in \var{s}}{\code{0} if an item of \var{s} is
equal to \var{x}, else \code{1}}{}
  \hline
  \lineiii{\var{s} + \var{t}}{the concatenation of \var{s} and \var{t}}{}
  \hline
  \lineiii{\var{s} * \var{n}{\rm ,} \var{n} * \var{s}}{\var{n} copies of \var{s} concatenated}{}
  \hline
  \lineiii{\var{s}[\var{i}]}{\var{i}'th item of \var{s}, origin 0}{(1)}
  \lineiii{\var{s}[\var{i}:\var{j}]}{slice of \var{s} from \var{i} to \var{j}}{(1), (2)}
  \hline
  \lineiii{len(\var{s})}{length of \var{s}}{}
  \lineiii{min(\var{s})}{smallest item of \var{s}}{}
  \lineiii{max(\var{s})}{largest item of \var{s}}{}
\end{tableiii}
\indexiii{operations on}{sequence}{types}
\bifuncindex{len}
\bifuncindex{min}
\bifuncindex{max}
\indexii{concatenation}{operation}
\indexii{repetition}{operation}
\indexii{subscript}{operation}
\indexii{slice}{operation}
\opindex{in}
\opindex{not in}

\noindent
Notes:

\begin{description}
  
\item[(1)] If \var{i} or \var{j} is negative, the index is relative to
  the end of the string, i.e., \code{len(\var{s}) + \var{i}} or
  \code{len(\var{s}) + \var{j}} is substituted.  But note that \code{-0} is
  still \code{0}.
  
\item[(2)] The slice of \var{s} from \var{i} to \var{j} is defined as
  the sequence of items with index \var{k} such that \code{\var{i} <=
  \var{k} < \var{j}}.  If \var{i} or \var{j} is greater than
  \code{len(\var{s})}, use \code{len(\var{s})}.  If \var{i} is omitted,
  use \code{0}.  If \var{j} is omitted, use \code{len(\var{s})}.  If
  \var{i} is greater than or equal to \var{j}, the slice is empty.

\end{description}

\subsubsection{More String Operations}

String objects have one unique built-in operation: the \code{\%}
operator (modulo) with a string left argument interprets this string
as a C sprintf format string to be applied to the right argument, and
returns the string resulting from this formatting operation.

The right argument should be a tuple with one item for each argument
required by the format string; if the string requires a single
argument, the right argument may also be a single non-tuple object.%
\footnote{A tuple object in this case should be a singleton.}
The following format characters are understood:
\%, c, s, i, d, u, o, x, X, e, E, f, g, G.
Width and precision may be a * to specify that an integer argument
specifies the actual width or precision.  The flag characters -, +,
blank, \# and 0 are understood.  The size specifiers h, l or L may be
present but are ignored.  The \code{\%s} conversion takes any Python
object and converts it to a string using \code{str()} before
formatting it.  The ANSI features \code{\%p} and \code{\%n}
are not supported.  Since Python strings have an explicit length,
\code{\%s} conversions don't assume that \code{'\e0'} is the end of
the string.

For safety reasons, floating point precisions are clipped to 50;
\code{\%f} conversions for numbers whose absolute value is over 1e25
are replaced by \code{\%g} conversions.%
\footnote{These numbers are fairly arbitrary.  They are intended to
avoid printing endless strings of meaningless digits without hampering
correct use and without having to know the exact precision of floating
point values on a particular machine.}
All other errors raise exceptions.

If the right argument is a dictionary (or any kind of mapping), then
the formats in the string must have a parenthesized key into that
dictionary inserted immediately after the \code{\%} character, and
each format formats the corresponding entry from the mapping.  E.g.
\begin{verbatim}
    >>> count = 2
    >>> language = 'Python'
    >>> print '%(language)s has %(count)03d quote types.' % vars()
    Python has 002 quote types.
    >>> 
\end{verbatim}
In this case no * specifiers may occur in a format (since they
require a sequential parameter list).

Additional string operations are defined in standard module
\code{string} and in built-in module \code{regex}.
\index{string}
\index{regex}

\subsubsection{Mutable Sequence Types}

List objects support additional operations that allow in-place
modification of the object.
These operations would be supported by other mutable sequence types
(when added to the language) as well.
Strings and tuples are immutable sequence types and such objects cannot
be modified once created.
The following operations are defined on mutable sequence types (where
\var{x} is an arbitrary object):
\indexiii{mutable}{sequence}{types}
\indexii{list}{type}

\begin{tableiii}{|c|l|c|}{code}{Operation}{Result}{Notes}
  \lineiii{\var{s}[\var{i}] = \var{x}}
	{item \var{i} of \var{s} is replaced by \var{x}}{}
  \lineiii{\var{s}[\var{i}:\var{j}] = \var{t}}
  	{slice of \var{s} from \var{i} to \var{j} is replaced by \var{t}}{}
  \lineiii{del \var{s}[\var{i}:\var{j}]}
	{same as \code{\var{s}[\var{i}:\var{j}] = []}}{}
  \lineiii{\var{s}.append(\var{x})}
	{same as \code{\var{s}[len(\var{s}):len(\var{s})] = [\var{x}]}}{}
  \lineiii{\var{s}.count(\var{x})}
	{return number of \var{i}'s for which \code{\var{s}[\var{i}] == \var{x}}}{}
  \lineiii{\var{s}.index(\var{x})}
	{return smallest \var{i} such that \code{\var{s}[\var{i}] == \var{x}}}{(1)}
  \lineiii{\var{s}.insert(\var{i}, \var{x})}
	{same as \code{\var{s}[\var{i}:\var{i}] = [\var{x}]}
	  if \code{\var{i} >= 0}}{}
  \lineiii{\var{s}.remove(\var{x})}
	{same as \code{del \var{s}[\var{s}.index(\var{x})]}}{(1)}
  \lineiii{\var{s}.reverse()}
	{reverses the items of \var{s} in place}{}
  \lineiii{\var{s}.sort()}
	{permutes the items of \var{s} to satisfy
        \code{\var{s}[\var{i}] <= \var{s}[\var{j}]},
        for \code{\var{i} < \var{j}}}{(2)}
\end{tableiii}
\indexiv{operations on}{mutable}{sequence}{types}
\indexiii{operations on}{sequence}{types}
\indexiii{operations on}{list}{type}
\indexii{subscript}{assignment}
\indexii{slice}{assignment}
\stindex{del}
\renewcommand{\indexsubitem}{(list method)}
\ttindex{append}
\ttindex{count}
\ttindex{index}
\ttindex{insert}
\ttindex{remove}
\ttindex{reverse}
\ttindex{sort}

\noindent
Notes:
\begin{description}
\item[(1)] Raises an exception when \var{x} is not found in \var{s}.
  
\item[(2)] The \code{sort()} method takes an optional argument
  specifying a comparison function of two arguments (list items) which
  should return \code{-1}, \code{0} or \code{1} depending on whether the
  first argument is considered smaller than, equal to, or larger than the
  second argument.  Note that this slows the sorting process down
  considerably; e.g. to sort a list in reverse order it is much faster
  to use calls to \code{sort()} and \code{reverse()} than to use
  \code{sort()} with a comparison function that reverses the ordering of
  the elements.
\end{description}

\subsection{Mapping Types}

A \dfn{mapping} object maps values of one type (the key type) to
arbitrary objects.  Mappings are mutable objects.  There is currently
only one standard mapping type, the \dfn{dictionary}.  A dictionary's keys are
almost arbitrary values.  The only types of values not acceptable as
keys are values containing lists or dictionaries or other mutable
types that are compared by value rather than by object identity.
Numeric types used for keys obey the normal rules for numeric
comparison: if two numbers compare equal (e.g. 1 and 1.0) then they
can be used interchangeably to index the same dictionary entry.

\indexii{mapping}{types}
\indexii{dictionary}{type}

Dictionaries are created by placing a comma-separated list of
\code{\var{key}:\,\var{value}} pairs within braces, for example:
\code{\{'jack':\,4098, 'sjoerd':\,4127\}} or
\code{\{4098:\,'jack', 4127:\,'sjoerd'\}}.

The following operations are defined on mappings (where \var{a} is a
mapping, \var{k} is a key and \var{x} is an arbitrary object):

\begin{tableiii}{|c|l|c|}{code}{Operation}{Result}{Notes}
  \lineiii{len(\var{a})}{the number of items in \var{a}}{}
  \lineiii{\var{a}[\var{k}]}{the item of \var{a} with key \var{k}}{(1)}
  \lineiii{\var{a}[\var{k}] = \var{x}}{set \code{\var{a}[\var{k}]} to \var{x}}{}
  \lineiii{del \var{a}[\var{k}]}{remove \code{\var{a}[\var{k}]} from \var{a}}{(1)}
  \lineiii{\var{a}.items()}{a copy of \var{a}'s list of (key, item) pairs}{(2)}
  \lineiii{\var{a}.keys()}{a copy of \var{a}'s list of keys}{(2)}
  \lineiii{\var{a}.values()}{a copy of \var{a}'s list of values}{(2)}
  \lineiii{\var{a}.has_key(\var{k})}{\code{1} if \var{a} has a key \var{k}, else \code{0}}{}
\end{tableiii}
\indexiii{operations on}{mapping}{types}
\indexiii{operations on}{dictionary}{type}
\stindex{del}
\bifuncindex{len}
\renewcommand{\indexsubitem}{(dictionary method)}
\ttindex{keys}
\ttindex{has_key}

\noindent
Notes:
\begin{description}
\item[(1)] Raises an exception if \var{k} is not in the map.

\item[(2)] Keys and values are listed in random order.
\end{description}

\subsection{Other Built-in Types}

The interpreter supports several other kinds of objects.
Most of these support only one or two operations.

\subsubsection{Modules}

The only special operation on a module is attribute access:
\code{\var{m}.\var{name}}, where \var{m} is a module and \var{name} accesses
a name defined in \var{m}'s symbol table.  Module attributes can be
assigned to.  (Note that the \code{import} statement is not, strictly
spoken, an operation on a module object; \code{import \var{foo}} does not
require a module object named \var{foo} to exist, rather it requires
an (external) \emph{definition} for a module named \var{foo}
somewhere.)

A special member of every module is \code{__dict__}.
This is the dictionary containing the module's symbol table.
Modifying this dictionary will actually change the module's symbol
table, but direct assignment to the \code{__dict__} attribute is not
possible (i.e., you can write \code{\var{m}.__dict__['a'] = 1}, which
defines \code{\var{m}.a} to be \code{1}, but you can't write \code{\var{m}.__dict__ = \{\}}.

Modules are written like this: \code{<module 'sys'>}.

\subsubsection{Classes and Class Instances}
\nodename{Classes and Instances}

(See Chapters 3 and 7 of the Python Reference Manual for these.)

\subsubsection{Functions}

Function objects are created by function definitions.  The only
operation on a function object is to call it:
\code{\var{func}(\var{argument-list})}.

There are really two flavors of function objects: built-in functions
and user-defined functions.  Both support the same operation (to call
the function), but the implementation is different, hence the
different object types.

The implementation adds two special read-only attributes:
\code{\var{f}.func_code} is a function's \dfn{code object} (see below) and
\code{\var{f}.func_globals} is the dictionary used as the function's
global name space (this is the same as \code{\var{m}.__dict__} where
\var{m} is the module in which the function \var{f} was defined).

\subsubsection{Methods}
\obindex{method}

Methods are functions that are called using the attribute notation.
There are two flavors: built-in methods (such as \code{append()} on
lists) and class instance methods.  Built-in methods are described
with the types that support them.

The implementation adds two special read-only attributes to class
instance methods: \code{\var{m}.im_self} is the object whose method this
is, and \code{\var{m}.im_func} is the function implementing the method.
Calling \code{\var{m}(\var{arg-1}, \var{arg-2}, {\rm \ldots},
\var{arg-n})} is completely equivalent to calling
\code{\var{m}.im_func(\var{m}.im_self, \var{arg-1}, \var{arg-2}, {\rm
\ldots}, \var{arg-n})}.

(See the Python Reference Manual for more info.)

\subsubsection{Code Objects}
\obindex{code}

Code objects are used by the implementation to represent
``pseudo-compiled'' executable Python code such as a function body.
They differ from function objects because they don't contain a
reference to their global execution environment.  Code objects are
returned by the built-in \code{compile()} function and can be
extracted from function objects through their \code{func_code}
attribute.
\bifuncindex{compile}
\ttindex{func_code}

A code object can be executed or evaluated by passing it (instead of a
source string) to the \code{exec} statement or the built-in
\code{eval()} function.
\stindex{exec}
\bifuncindex{eval}

(See the Python Reference Manual for more info.)

\subsubsection{Type Objects}

Type objects represent the various object types.  An object's type is
accessed by the built-in function \code{type()}.  There are no special
operations on types.  The standard module \code{types} defines names
for all standard built-in types.
\bifuncindex{type}
\stmodindex{types}

Types are written like this: \code{<type 'int'>}.

\subsubsection{The Null Object}

This object is returned by functions that don't explicitly return a
value.  It supports no special operations.  There is exactly one null
object, named \code{None} (a built-in name).

It is written as \code{None}.

\subsubsection{File Objects}

File objects are implemented using \C{}'s \code{stdio} package and can be
created with the built-in function \code{open()} described under
Built-in Functions below.  They are also returned by some other
built-in functions and methods, e.g.\ \code{posix.popen()} and
\code{posix.fdopen()} and the \code{makefile()} method of socket
objects.
\bifuncindex{open}

When a file operation fails for an I/O-related reason, the exception
\code{IOError} is raised.  This includes situations where the
operation is not defined for some reason, like \code{seek()} on a tty
device or writing a file opened for reading.

Files have the following methods:


\renewcommand{\indexsubitem}{(file method)}

\begin{funcdesc}{close}{}
  Close the file.  A closed file cannot be read or written anymore.
\end{funcdesc}

\begin{funcdesc}{flush}{}
  Flush the internal buffer, like \code{stdio}'s \code{fflush()}.
\end{funcdesc}

\begin{funcdesc}{isatty}{}
  Return \code{1} if the file is connected to a tty(-like) device, else
  \code{0}.
\end{funcdesc}

\begin{funcdesc}{read}{\optional{size}}
  Read at most \var{size} bytes from the file (less if the read hits
  \EOF{} or no more data is immediately available on a pipe, tty or
  similar device).  If the \var{size} argument is negative or omitted,
  read all data until \EOF{} is reached.  The bytes are returned as a string
  object.  An empty string is returned when \EOF{} is encountered
  immediately.  (For certain files, like ttys, it makes sense to
  continue reading after an \EOF{} is hit.)
\end{funcdesc}

\begin{funcdesc}{readline}{\optional{size}}
  Read one entire line from the file.  A trailing newline character is
  kept in the string%
\footnote{The advantage of leaving the newline on is that an empty string 
	can be returned to mean \EOF{} without being ambiguous.  Another 
	advantage is that (in cases where it might matter, e.g. if you 
	want to make an exact copy of a file while scanning its lines) 
	you can tell whether the last line of a file ended in a newline
	or not (yes this happens!).}
  (but may be absent when a file ends with an
  incomplete line).  If the \var{size} argument is present and
  non-negative, it is a maximum byte count (including the trailing
  newline) and an incomplete line may be returned.
  An empty string is returned when \EOF{} is hit
  immediately.  Note: unlike \code{stdio}'s \code{fgets()}, the returned
  string contains null characters (\code{'\e 0'}) if they occurred in the
  input.
\end{funcdesc}

\begin{funcdesc}{readlines}{}
  Read until \EOF{} using \code{readline()} and return a list containing
  the lines thus read.
\end{funcdesc}

\begin{funcdesc}{seek}{offset\, whence}
  Set the file's current position, like \code{stdio}'s \code{fseek()}.
  The \var{whence} argument is optional and defaults to \code{0}
  (absolute file positioning); other values are \code{1} (seek
  relative to the current position) and \code{2} (seek relative to the
  file's end).  There is no return value.
\end{funcdesc}

\begin{funcdesc}{tell}{}
  Return the file's current position, like \code{stdio}'s \code{ftell()}.
\end{funcdesc}

\begin{funcdesc}{truncate}{\optional{size}}
Truncate the file's size.  If the optional size argument present, the
file is truncated to (at most) that size.  The size defaults to the
current position.  Availability of this function depends on the
operating system version (e.g., not all \UNIX{} versions support this
operation).
\end{funcdesc}

\begin{funcdesc}{write}{str}
Write a string to the file.  There is no return value.  Note: due to
buffering, the string may not actually show up in the file until
the \code{flush()} or \code{close()} method is called.
\end{funcdesc}

\begin{funcdesc}{writelines}{list}
Write a list of strings to the file.  There is no return value.
(The name is intended to match \code{readlines}; \code{writelines}
does not add line separators.)
\end{funcdesc}

\subsubsection{Internal Objects}

(See the Python Reference Manual for these.)

\subsection{Special Attributes}

The implementation adds a few special read-only attributes to several
object types, where they are relevant:

\begin{itemize}

\item
\code{\var{x}.__dict__} is a dictionary of some sort used to store an
object's (writable) attributes;

\item
\code{\var{x}.__methods__} lists the methods of many built-in object types,
e.g., \code{[].__methods__} yields
\code{['append', 'count', 'index', 'insert', 'remove', 'reverse', 'sort']};

\item
\code{\var{x}.__members__} lists data attributes;

\item
\code{\var{x}.__class__} is the class to which a class instance belongs;

\item
\code{\var{x}.__bases__} is the tuple of base classes of a class object.

\end{itemize}

\section{Built-in Exceptions}

Exceptions are string objects.  Two distinct string objects with the
same value are different exceptions.  This is done to force programmers
to use exception names rather than their string value when specifying
exception handlers.  The string value of all built-in exceptions is
their name, but this is not a requirement for user-defined exceptions
or exceptions defined by library modules.

The following exceptions can be generated by the interpreter or
built-in functions.  Except where mentioned, they have an `associated
value' indicating the detailed cause of the error.  This may be a
string or a tuple containing several items of information (e.g., an
error code and a string explaining the code).

User code can raise built-in exceptions.  This can be used to test an
exception handler or to report an error condition `just like' the
situation in which the interpreter raises the same exception; but
beware that there is nothing to prevent user code from raising an
inappropriate error.

\renewcommand{\indexsubitem}{(built-in exception)}

\begin{excdesc}{AttributeError}
% xref to attribute reference?
  Raised when an attribute reference or assignment fails.  (When an
  object does not support attribute references or attribute assignments
  at all, \code{TypeError} is raised.)
\end{excdesc}

\begin{excdesc}{EOFError}
% XXXJH xrefs here
  Raised when one of the built-in functions (\code{input()} or
  \code{raw_input()}) hits an end-of-file condition (\EOF{}) without
  reading any data.
% XXXJH xrefs here
  (N.B.: the \code{read()} and \code{readline()} methods of file
  objects return an empty string when they hit \EOF{}.)  No associated value.
\end{excdesc}

\begin{excdesc}{IOError}
% XXXJH xrefs here
  Raised when an I/O operation (such as a \code{print} statement, the
  built-in \code{open()} function or a method of a file object) fails
  for an I/O-related reason, e.g., `file not found', `disk full'.
\end{excdesc}

\begin{excdesc}{ImportError}
% XXXJH xref to import statement?
  Raised when an \code{import} statement fails to find the module
  definition or when a \code{from {\rm \ldots} import} fails to find a
  name that is to be imported.
\end{excdesc}

\begin{excdesc}{IndexError}
% XXXJH xref to sequences
  Raised when a sequence subscript is out of range.  (Slice indices are
  silently truncated to fall in the allowed range; if an index is not a
  plain integer, \code{TypeError} is raised.)
\end{excdesc}

\begin{excdesc}{KeyError}
% XXXJH xref to mapping objects?
  Raised when a mapping (dictionary) key is not found in the set of
  existing keys.
\end{excdesc}

\begin{excdesc}{KeyboardInterrupt}
  Raised when the user hits the interrupt key (normally
  \kbd{Control-C} or
\key{DEL}).  During execution, a check for interrupts is made regularly.
% XXXJH xrefs here
  Interrupts typed when a built-in function \code{input()} or
  \code{raw_input()}) is waiting for input also raise this exception.  No
  associated value.
\end{excdesc}

\begin{excdesc}{MemoryError}
  Raised when an operation runs out of memory but the situation may
  still be rescued (by deleting some objects).  The associated value is
  a string indicating what kind of (internal) operation ran out of memory.
  Note that because of the underlying memory management architecture
  (\C{}'s \code{malloc()} function), the interpreter may not always be able
  to completely recover from this situation; it nevertheless raises an
  exception so that a stack traceback can be printed, in case a run-away
  program was the cause.
\end{excdesc}

\begin{excdesc}{NameError}
  Raised when a local or global name is not found.  This applies only
  to unqualified names.  The associated value is the name that could
  not be found.
\end{excdesc}

\begin{excdesc}{OverflowError}
% XXXJH reference to long's and/or int's?
  Raised when the result of an arithmetic operation is too large to be
  represented.  This cannot occur for long integers (which would rather
  raise \code{MemoryError} than give up).  Because of the lack of
  standardization of floating point exception handling in \C{}, most
  floating point operations also aren't checked.  For plain integers,
  all operations that can overflow are checked except left shift, where
  typical applications prefer to drop bits than raise an exception.
\end{excdesc}

\begin{excdesc}{RuntimeError}
  Raised when an error is detected that doesn't fall in any of the
  other categories.  The associated value is a string indicating what
  precisely went wrong.  (This exception is a relic from a previous
  version of the interpreter; it is not used any more except by some
  extension modules that haven't been converted to define their own
  exceptions yet.)
\end{excdesc}

\begin{excdesc}{SyntaxError}
% XXXJH xref to these functions?
  Raised when the parser encounters a syntax error.  This may occur in
  an \code{import} statement, in an \code{exec} statement, in a call
  to the built-in function \code{eval()} or \code{input()}, or
  when reading the initial script or standard input (also
  interactively).
\end{excdesc}

\begin{excdesc}{SystemError}
  Raised when the interpreter finds an internal error, but the
  situation does not look so serious to cause it to abandon all hope.
  The associated value is a string indicating what went wrong (in
  low-level terms).
  
  You should report this to the author or maintainer of your Python
  interpreter.  Be sure to report the version string of the Python
  interpreter (\code{sys.version}; it is also printed at the start of an
  interactive Python session), the exact error message (the exception's
  associated value) and if possible the source of the program that
  triggered the error.
\end{excdesc}

\begin{excdesc}{SystemExit}
% XXXJH xref to module sys?
  This exception is raised by the \code{sys.exit()} function.  When it
  is not handled, the Python interpreter exits; no stack traceback is
  printed.  If the associated value is a plain integer, it specifies the
  system exit status (passed to \C{}'s \code{exit()} function); if it is
  \code{None}, the exit status is zero; if it has another type (such as
  a string), the object's value is printed and the exit status is one.
  
  A call to \code{sys.exit} is translated into an exception so that
  clean-up handlers (\code{finally} clauses of \code{try} statements)
  can be executed, and so that a debugger can execute a script without
  running the risk of losing control.  The \code{posix._exit()} function
  can be used if it is absolutely positively necessary to exit
  immediately (e.g., after a \code{fork()} in the child process).
\end{excdesc}

\begin{excdesc}{TypeError}
  Raised when a built-in operation or function is applied to an object
  of inappropriate type.  The associated value is a string giving
  details about the type mismatch.
\end{excdesc}

\begin{excdesc}{ValueError}
  Raised when a built-in operation or function receives an argument
  that has the right type but an inappropriate value, and the
  situation is not described by a more precise exception such as
  \code{IndexError}.
\end{excdesc}

\begin{excdesc}{ZeroDivisionError}
  Raised when the second argument of a division or modulo operation is
  zero.  The associated value is a string indicating the type of the
  operands and the operation.
\end{excdesc}

\section{Built-in Functions}

The Python interpreter has a number of functions built into it that
are always available.  They are listed here in alphabetical order.


\renewcommand{\indexsubitem}{(built-in function)}
\begin{funcdesc}{abs}{x}
  Return the absolute value of a number.  The argument may be a plain
  or long integer or a floating point number.
\end{funcdesc}

\begin{funcdesc}{apply}{function\, args\optional{, keywords}}
The \var{function} argument must be a callable object (a user-defined or
built-in function or method, or a class object) and the \var{args}
argument must be a tuple.  The \var{function} is called with
\var{args} as argument list; the number of arguments is the the length
of the tuple.  (This is different from just calling
\code{\var{func}(\var{args})}, since in that case there is always
exactly one argument.)
If the optional \var{keywords} argument is present, it must be a
dictionary whose keys are strings.  It specifies keyword arguments to
be added to the end of the the argument list.
\end{funcdesc}

\begin{funcdesc}{chr}{i}
  Return a string of one character whose \ASCII{} code is the integer
  \var{i}, e.g., \code{chr(97)} returns the string \code{'a'}.  This is the
  inverse of \code{ord()}.  The argument must be in the range [0..255],
  inclusive.
\end{funcdesc}

\begin{funcdesc}{cmp}{x\, y}
  Compare the two objects \var{x} and \var{y} and return an integer
  according to the outcome.  The return value is negative if \code{\var{x}
  < \var{y}}, zero if \code{\var{x} == \var{y}} and strictly positive if
  \code{\var{x} > \var{y}}.
\end{funcdesc}

\begin{funcdesc}{coerce}{x\, y}
  Return a tuple consisting of the two numeric arguments converted to
  a common type, using the same rules as used by arithmetic
  operations.
\end{funcdesc}

\begin{funcdesc}{compile}{string\, filename\, kind}
  Compile the \var{string} into a code object.  Code objects can be
  executed by an \code{exec} statement or evaluated by a call to
  \code{eval()}.  The \var{filename} argument should
  give the file from which the code was read; pass e.g. \code{'<string>'}
  if it wasn't read from a file.  The \var{kind} argument specifies
  what kind of code must be compiled; it can be \code{'exec'} if
  \var{string} consists of a sequence of statements, \code{'eval'}
  if it consists of a single expression, or \code{'single'} if
  it consists of a single interactive statement (in the latter case,
  expression statements that evaluate to something else than
  \code{None} will printed).
\end{funcdesc}

\begin{funcdesc}{delattr}{object\, name}
  This is a relative of \code{setattr}.  The arguments are an
  object and a string.  The string must be the name
  of one of the object's attributes.  The function deletes
  the named attribute, provided the object allows it.  For example,
  \code{delattr(\var{x}, '\var{foobar}')} is equivalent to
  \code{del \var{x}.\var{foobar}}.
\end{funcdesc}

\begin{funcdesc}{dir}{}
  Without arguments, return the list of names in the current local
  symbol table.  With a module, class or class instance object as
  argument (or anything else that has a \code{__dict__} attribute),
  returns the list of names in that object's attribute dictionary.
  The resulting list is sorted.  For example:

\bcode\begin{verbatim}
>>> import sys
>>> dir()
['sys']
>>> dir(sys)
['argv', 'exit', 'modules', 'path', 'stderr', 'stdin', 'stdout']
>>> 
\end{verbatim}\ecode
\end{funcdesc}

\begin{funcdesc}{divmod}{a\, b}
  Take two numbers as arguments and return a pair of integers
  consisting of their integer quotient and remainder.  With mixed
  operand types, the rules for binary arithmetic operators apply.  For
  plain and long integers, the result is the same as
  \code{(\var{a} / \var{b}, \var{a} \%{} \var{b})}.
  For floating point numbers the result is the same as
  \code{(math.floor(\var{a} / \var{b}), \var{a} \%{} \var{b})}.
\end{funcdesc}

\begin{funcdesc}{eval}{expression\optional{\, globals\optional{\, locals}}}
  The arguments are a string and two optional dictionaries.  The
  \var{expression} argument is parsed and evaluated as a Python
  expression (technically speaking, a condition list) using the
  \var{globals} and \var{locals} dictionaries as global and local name
  space.  If the \var{locals} dictionary is omitted it defaults to
  the \var{globals} dictionary.  If both dictionaries are omitted, the
  expression is executed in the environment where \code{eval} is
  called.  The return value is the result of the evaluated expression.
  Syntax errors are reported as exceptions.  Example:

\bcode\begin{verbatim}
>>> x = 1
>>> print eval('x+1')
2
>>> 
\end{verbatim}\ecode

  This function can also be used to execute arbitrary code objects
  (e.g.\ created by \code{compile()}).  In this case pass a code
  object instead of a string.  The code object must have been compiled
  passing \code{'eval'} to the \var{kind} argument.

  Hints: dynamic execution of statements is supported by the
  \code{exec} statement.  Execution of statements from a file is
  supported by the \code{execfile()} function.  The \code{globals()}
  and \code{locals()} functions returns the current global and local
  dictionary, respectively, which may be useful
  to pass around for use by \code{eval()} or \code{execfile()}.

\end{funcdesc}

\begin{funcdesc}{execfile}{file\optional{\, globals\optional{\, locals}}}
  This function is similar to the
  \code{exec} statement, but parses a file instead of a string.  It is
  different from the \code{import} statement in that it does not use
  the module administration --- it reads the file unconditionally and
  does not create a new module.\footnote{It is used relatively rarely
  so does not warrant being made into a statement.}

  The arguments are a file name and two optional dictionaries.  The
  file is parsed and evaluated as a sequence of Python statements
  (similarly to a module) using the \var{globals} and \var{locals}
  dictionaries as global and local name space.  If the \var{locals}
  dictionary is omitted it defaults to the \var{globals} dictionary.
  If both dictionaries are omitted, the expression is executed in the
  environment where \code{execfile()} is called.  The return value is
  \code{None}.
\end{funcdesc}

\begin{funcdesc}{filter}{function\, list}
Construct a list from those elements of \var{list} for which
\var{function} returns true.  If \var{list} is a string or a tuple,
the result also has that type; otherwise it is always a list.  If
\var{function} is \code{None}, the identity function is assumed,
i.e.\ all elements of \var{list} that are false (zero or empty) are
removed.
\end{funcdesc}

\begin{funcdesc}{float}{x}
  Convert a number to floating point.  The argument may be a plain or
  long integer or a floating point number.
\end{funcdesc}

\begin{funcdesc}{getattr}{object\, name}
  The arguments are an object and a string.  The string must be the
  name
  of one of the object's attributes.  The result is the value of that
  attribute.  For example, \code{getattr(\var{x}, '\var{foobar}')} is equivalent to
  \code{\var{x}.\var{foobar}}.
\end{funcdesc}

\begin{funcdesc}{globals}{}
Return a dictionary representing the current global symbol table.
This is always the dictionary of the current module (inside a
function or method, this is the module where it is defined, not the
module from which it is called).
\end{funcdesc}

\begin{funcdesc}{hasattr}{object\, name}
  The arguments are an object and a string.  The result is 1 if the
  string is the name of one of the object's attributes, 0 if not.
  (This is implemented by calling \code{getattr(object, name)} and
  seeing whether it raises an exception or not.)
\end{funcdesc}

\begin{funcdesc}{hash}{object}
  Return the hash value of the object (if it has one).  Hash values
  are 32-bit integers.  They are used to quickly compare dictionary
  keys during a dictionary lookup.  Numeric values that compare equal
  have the same hash value (even if they are of different types, e.g.
  1 and 1.0).
\end{funcdesc}

\begin{funcdesc}{hex}{x}
  Convert an integer number (of any size) to a hexadecimal string.
  The result is a valid Python expression.
\end{funcdesc}

\begin{funcdesc}{id}{object}
  Return the `identity' of an object.  This is an integer which is
  guaranteed to be unique and constant for this object during its
  lifetime.  (Two objects whose lifetimes are disjunct may have the
  same id() value.)  (Implementation note: this is the address of the
  object.)
\end{funcdesc}

\begin{funcdesc}{input}{\optional{prompt}}
  Almost equivalent to \code{eval(raw_input(\var{prompt}))}.  Like
  \code{raw_input()}, the \var{prompt} argument is optional.  The difference
  is that a long input expression may be broken over multiple lines using
  the backslash convention.
\end{funcdesc}

\begin{funcdesc}{int}{x}
  Convert a number to a plain integer.  The argument may be a plain or
  long integer or a floating point number.  Conversion of floating
  point numbers to integers is defined by the C semantics; normally
  the conversion truncates towards zero.\footnote{This is ugly --- the
  language definition should require truncation towards zero.}
\end{funcdesc}

\begin{funcdesc}{len}{s}
  Return the length (the number of items) of an object.  The argument
  may be a sequence (string, tuple or list) or a mapping (dictionary).
\end{funcdesc}

\begin{funcdesc}{locals}{}
Return a dictionary representing the current local symbol table.
Inside a function, modifying this dictionary does not always have the
desired effect.
\end{funcdesc}

\begin{funcdesc}{long}{x}
  Convert a number to a long integer.  The argument may be a plain or
  long integer or a floating point number.
\end{funcdesc}

\begin{funcdesc}{map}{function\, list\, ...}
Apply \var{function} to every item of \var{list} and return a list
of the results.  If additional \var{list} arguments are passed, 
\var{function} must take that many arguments and is applied to
the items of all lists in parallel; if a list is shorter than another
it is assumed to be extended with \code{None} items.  If
\var{function} is \code{None}, the identity function is assumed; if
there are multiple list arguments, \code{map} returns a list
consisting of tuples containing the corresponding items from all lists
(i.e. a kind of transpose operation).  The \var{list} arguments may be
any kind of sequence; the result is always a list.
\end{funcdesc}

\begin{funcdesc}{max}{s}
  Return the largest item of a non-empty sequence (string, tuple or
  list).
\end{funcdesc}

\begin{funcdesc}{min}{s}
  Return the smallest item of a non-empty sequence (string, tuple or
  list).
\end{funcdesc}

\begin{funcdesc}{oct}{x}
  Convert an integer number (of any size) to an octal string.  The
  result is a valid Python expression.
\end{funcdesc}

\begin{funcdesc}{open}{filename\optional{\, mode\optional{\, bufsize}}}
  Return a new file object (described earlier under Built-in Types).
  The first two arguments are the same as for \code{stdio}'s
  \code{fopen()}: \var{filename} is the file name to be opened,
  \var{mode} indicates how the file is to be opened: \code{'r'} for
  reading, \code{'w'} for writing (truncating an existing file), and
  \code{'a'} opens it for appending (which on {\em some} \UNIX{}
  systems means that {\em all} writes append to the end of the file,
  regardless of the current seek position).
  Modes \code{'r+'}, \code{'w+'} and
  \code{'a+'} open the file for updating, provided the underlying
  \code{stdio} library understands this.  On systems that differentiate
  between binary and text files, \code{'b'} appended to the mode opens
  the file in binary mode.  If the file cannot be opened, \code{IOError}
  is raised.
If \var{mode} is omitted, it defaults to \code{'r'}.
The optional \var{bufsize} argument specifies the file's desired
buffer size: 0 means unbuffered, 1 means line buffered, any other
positive value means use a buffer of (approximately) that size.  A
negative \var{bufsize} means to use the system default, which is
usually line buffered for for tty devices and fully buffered for other
files.%
\footnote{Specifying a buffer size currently has no effect on systems
that don't have \code{setvbuf()}.  The interface to specify the buffer
size is not done using a method that calls \code{setvbuf()}, because
that may dump core when called after any I/O has been performed, and
there's no reliable way to determine whether this is the case.}
\end{funcdesc}

\begin{funcdesc}{ord}{c}
  Return the \ASCII{} value of a string of one character.  E.g.,
  \code{ord('a')} returns the integer \code{97}.  This is the inverse of
  \code{chr()}.
\end{funcdesc}

\begin{funcdesc}{pow}{x\, y\optional{\, z}}
  Return \var{x} to the power \var{y}; if \var{z} is present, return
  \var{x} to the power \var{y}, modulo \var{z} (computed more
  efficiently than \code{pow(\var{x}, \var{y}) \% \var{z}}).
  The arguments must have
  numeric types.  With mixed operand types, the rules for binary
  arithmetic operators apply.  The effective operand type is also the
  type of the result; if the result is not expressible in this type, the
  function raises an exception; e.g., \code{pow(2, -1)} or \code{pow(2,
  35000)} is not allowed.
\end{funcdesc}

\begin{funcdesc}{range}{\optional{start\,} end\optional{\, step}}
  This is a versatile function to create lists containing arithmetic
  progressions.  It is most often used in \code{for} loops.  The
  arguments must be plain integers.  If the \var{step} argument is
  omitted, it defaults to \code{1}.  If the \var{start} argument is
  omitted, it defaults to \code{0}.  The full form returns a list of
  plain integers \code{[\var{start}, \var{start} + \var{step},
  \var{start} + 2 * \var{step}, \ldots]}.  If \var{step} is positive,
  the last element is the largest \code{\var{start} + \var{i} *
  \var{step}} less than \var{end}; if \var{step} is negative, the last
  element is the largest \code{\var{start} + \var{i} * \var{step}}
  greater than \var{end}.  \var{step} must not be zero (or else an
  exception is raised).  Example:

\bcode\begin{verbatim}
>>> range(10)
[0, 1, 2, 3, 4, 5, 6, 7, 8, 9]
>>> range(1, 11)
[1, 2, 3, 4, 5, 6, 7, 8, 9, 10]
>>> range(0, 30, 5)
[0, 5, 10, 15, 20, 25]
>>> range(0, 10, 3)
[0, 3, 6, 9]
>>> range(0, -10, -1)
[0, -1, -2, -3, -4, -5, -6, -7, -8, -9]
>>> range(0)
[]
>>> range(1, 0)
[]
>>> 
\end{verbatim}\ecode
\end{funcdesc}

\begin{funcdesc}{raw_input}{\optional{prompt}}
  If the \var{prompt} argument is present, it is written to standard output
  without a trailing newline.  The function then reads a line from input,
  converts it to a string (stripping a trailing newline), and returns that.
  When \EOF{} is read, \code{EOFError} is raised. Example:

\bcode\begin{verbatim}
>>> s = raw_input('--> ')
--> Monty Python's Flying Circus
>>> s
"Monty Python's Flying Circus"
>>> 
\end{verbatim}\ecode
\end{funcdesc}

\begin{funcdesc}{reduce}{function\, list\optional{\, initializer}}
Apply the binary \var{function} to the items of \var{list} so as to
reduce the list to a single value.  E.g.,
\code{reduce(lambda x, y: x*y, \var{list}, 1)} returns the product of
the elements of \var{list}.  The optional \var{initializer} can be
thought of as being prepended to \var{list} so as to allow reduction
of an empty \var{list}.  The \var{list} arguments may be any kind of
sequence.
\end{funcdesc}

\begin{funcdesc}{reload}{module}
Re-parse and re-initialize an already imported \var{module}.  The
argument must be a module object, so it must have been successfully
imported before.  This is useful if you have edited the module source
file using an external editor and want to try out the new version
without leaving the Python interpreter.  The return value is the
module object (i.e.\ the same as the \var{module} argument).

There are a number of caveats:

If a module is syntactically correct but its initialization fails, the
first \code{import} statement for it does not bind its name locally,
but does store a (partially initialized) module object in
\code{sys.modules}.  To reload the module you must first
\code{import} it again (this will bind the name to the partially
initialized module object) before you can \code{reload()} it.

When a module is reloaded, its dictionary (containing the module's
global variables) is retained.  Redefinitions of names will override
the old definitions, so this is generally not a problem.  If the new
version of a module does not define a name that was defined by the old
version, the old definition remains.  This feature can be used to the
module's advantage if it maintains a global table or cache of objects
--- with a \code{try} statement it can test for the table's presence
and skip its initialization if desired.

It is legal though generally not very useful to reload built-in or
dynamically loaded modules, except for \code{sys}, \code{__main__} and
\code{__builtin__}.  In certain cases, however, extension modules are
not designed to be initialized more than once, and may fail in
arbitrary ways when reloaded.

If a module imports objects from another module using \code{from}
{\ldots} \code{import} {\ldots}, calling \code{reload()} for the other
module does not redefine the objects imported from it --- one way
around this is to re-execute the \code{from} statement, another is to
use \code{import} and qualified names (\var{module}.\var{name})
instead.

If a module instantiates instances of a class, reloading the module
that defines the class does not affect the method definitions of the
instances --- they continue to use the old class definition.  The same
is true for derived classes.
\end{funcdesc}

\begin{funcdesc}{repr}{object}
Return a string containing a printable representation of an object.
This is the same value yielded by conversions (reverse quotes).
It is sometimes useful to be able to access this operation as an
ordinary function.  For many types, this function makes an attempt
to return a string that would yield an object with the same value
when passed to \code{eval()}.
\end{funcdesc}

\begin{funcdesc}{round}{x\, n}
  Return the floating point value \var{x} rounded to \var{n} digits
  after the decimal point.  If \var{n} is omitted, it defaults to zero.
  The result is a floating point number.  Values are rounded to the
  closest multiple of 10 to the power minus \var{n}; if two multiples
  are equally close, rounding is done away from 0 (so e.g.
  \code{round(0.5)} is \code{1.0} and \code{round(-0.5)} is \code{-1.0}).
\end{funcdesc}

\begin{funcdesc}{setattr}{object\, name\, value}
  This is the counterpart of \code{getattr}.  The arguments are an
  object, a string and an arbitrary value.  The string must be the name
  of one of the object's attributes.  The function assigns the value to
  the attribute, provided the object allows it.  For example,
  \code{setattr(\var{x}, '\var{foobar}', 123)} is equivalent to
  \code{\var{x}.\var{foobar} = 123}.
\end{funcdesc}

\begin{funcdesc}{str}{object}
Return a string containing a nicely printable representation of an
object.  For strings, this returns the string itself.  The difference
with \code{repr(\var{object})} is that \code{str(\var{object})} does not
always attempt to return a string that is acceptable to \code{eval()};
its goal is to return a printable string.
\end{funcdesc}

\begin{funcdesc}{tuple}{sequence}
Return a tuple whose items are the same and in the same order as
\var{sequence}'s items.  If \var{sequence} is alread a tuple, it
is returned unchanged.  For instance, \code{tuple('abc')} returns
returns \code{('a', 'b', 'c')} and \code{tuple([1, 2, 3])} returns
\code{(1, 2, 3)}.
\end{funcdesc}

\begin{funcdesc}{type}{object}
Return the type of an \var{object}.  The return value is a type
object.  The standard module \code{types} defines names for all
built-in types.
\stmodindex{types}
\obindex{type}
For instance:

\bcode\begin{verbatim}
>>> import types
>>> if type(x) == types.StringType: print "It's a string"
\end{verbatim}\ecode
\end{funcdesc}

\begin{funcdesc}{vars}{\optional{object}}
Without arguments, return a dictionary corresponding to the current
local symbol table.  With a module, class or class instance object as
argument (or anything else that has a \code{__dict__} attribute),
returns a dictionary corresponding to the object's symbol table.
The returned dictionary should not be modified: the effects on the
corresponding symbol table are undefined.%
\footnote{In the current implementation, local variable bindings
cannot normally be affected this way, but variables retrieved from
other scopes (e.g. modules) can be.  This may change.}
\end{funcdesc}

\begin{funcdesc}{xrange}{\optional{start\,} end\optional{\, step}}
This function is very similar to \code{range()}, but returns an
``xrange object'' instead of a list.  This is an opaque sequence type
which yields the same values as the corresponding list, without
actually storing them all simultaneously.  The advantage of
\code{xrange()} over \code{range()} is minimal (since \code{xrange()}
still has to create the values when asked for them) except when a very
large range is used on a memory-starved machine (e.g. MS-DOS) or when all
of the range's elements are never used (e.g. when the loop is usually
terminated with \code{break}).
\end{funcdesc}


\chapter{Python Services}

The modules described in this chapter provide a wide range of services
related to the Python interpreter and its interaction with its
environment.  Here's an overview:

\begin{description}

\item[sys]
--- Access system specific parameters and functions.

\item[types]
--- Names for all built-in types.

\item[traceback]
--- Print or retrieve a stack traceback.

\item[pickle]
--- Convert Python objects to streams of bytes and back.

\item[shelve]
--- Python object persistency.

\item[copy]
--- Shallow and deep copy operations.

\item[marshal]
--- Convert Python objects to streams of bytes and back (with
different constraints).

\item[imp]
--- Access the implementation of the \code{import} statement.

\item[parser]
--- Retrieve and submit parse trees from and to the runtime support
environment.

\item[__builtin__]
--- The set of built-in functions.

\item[__main__]
--- The environment where the top-level script is run.

\end{description}
		% Python Services
\section{Built-in Module \sectcode{sys}}

\bimodindex{sys}
This module provides access to some variables used or maintained by the
interpreter and to functions that interact strongly with the interpreter.
It is always available.

\renewcommand{\indexsubitem}{(in module sys)}

\begin{datadesc}{argv}
  The list of command line arguments passed to a Python script.
  \code{sys.argv[0]} is the script name (it is operating system
  dependent whether this is a full pathname or not).
  If the command was executed using the \samp{-c} command line option
  to the interpreter, \code{sys.argv[0]} is set to the string
  \code{"-c"}.
  If no script name was passed to the Python interpreter,
  \code{sys.argv} has zero length.
\end{datadesc}

\begin{datadesc}{builtin_module_names}
  A list of strings giving the names of all modules that are compiled
  into this Python interpreter.  (This information is not available in
  any other way --- \code{sys.modules.keys()} only lists the imported
  modules.)
\end{datadesc}

\begin{datadesc}{exc_type}
\dataline{exc_value}
\dataline{exc_traceback}
  These three variables are not always defined; they are set when an
  exception handler (an \code{except} clause of a \code{try} statement) is
  invoked.  Their meaning is: \code{exc_type} gets the exception type of
  the exception being handled; \code{exc_value} gets the exception
  parameter (its \dfn{associated value} or the second argument to
  \code{raise}); \code{exc_traceback} gets a traceback object (see the
  Reference Manual) which
  encapsulates the call stack at the point where the exception
  originally occurred.
\obindex{traceback}
\end{datadesc}

\begin{funcdesc}{exit}{n}
  Exit from Python with numeric exit status \var{n}.  This is
  implemented by raising the \code{SystemExit} exception, so cleanup
  actions specified by \code{finally} clauses of \code{try} statements
  are honored, and it is possible to catch the exit attempt at an outer
  level.
\end{funcdesc}

\begin{datadesc}{exitfunc}
  This value is not actually defined by the module, but can be set by
  the user (or by a program) to specify a clean-up action at program
  exit.  When set, it should be a parameterless function.  This function
  will be called when the interpreter exits in any way (except when a
  fatal error occurs: in that case the interpreter's internal state
  cannot be trusted).
\end{datadesc}

\begin{datadesc}{last_type}
\dataline{last_value}
\dataline{last_traceback}
  These three variables are not always defined; they are set when an
  exception is not handled and the interpreter prints an error message
  and a stack traceback.  Their intended use is to allow an interactive
  user to import a debugger module and engage in post-mortem debugging
  without having to re-execute the command that caused the error (which
  may be hard to reproduce).  The meaning of the variables is the same
  as that of \code{exc_type}, \code{exc_value} and \code{exc_tracaback},
  respectively.
\end{datadesc}

\begin{datadesc}{modules}
  Gives the list of modules that have already been loaded.
  This can be manipulated to force reloading of modules and other tricks.
\end{datadesc}

\begin{datadesc}{path}
  A list of strings that specifies the search path for modules.
  Initialized from the environment variable \code{PYTHONPATH}, or an
  installation-dependent default.
\end{datadesc}

\begin{datadesc}{platform}
This string contains a platform identifier.  This can be used to
append platform-specific components to \code{sys.path}, for instance.
\end{datadesc}

\begin{datadesc}{ps1}
\dataline{ps2}
  Strings specifying the primary and secondary prompt of the
  interpreter.  These are only defined if the interpreter is in
  interactive mode.  Their initial values in this case are
  \code{'>>> '} and \code{'... '}.
\end{datadesc}

\begin{funcdesc}{setcheckinterval}{interval}
Set the interpreter's ``check interval''.  This integer value
determines how often the interpreter checks for periodic things such
as thread switches and signal handlers.  The default is 10, meaning
the check is performed every 10 Python virtual instructions.  Setting
it to a larger value may increase performance for programs using
threads.  Setting it to a value $\leq 0$ checks every virtual instruction,
maximizing responsiveness as well as overhead.
\end{funcdesc}

\begin{funcdesc}{settrace}{tracefunc}
  Set the system's trace function, which allows you to implement a
  Python source code debugger in Python.  See section ``How It Works''
  in the chapter on the Python Debugger.
\end{funcdesc}
\index{trace function}
\index{debugger}

\begin{funcdesc}{setprofile}{profilefunc}
  Set the system's profile function, which allows you to implement a
  Python source code profiler in Python.  See the chapter on the
  Python Profiler.  The system's profile function
  is called similarly to the system's trace function (see
  \code{sys.settrace}), but it isn't called for each executed line of
  code (only on call and return and when an exception occurs).  Also,
  its return value is not used, so it can just return \code{None}.
\end{funcdesc}
\index{profile function}
\index{profiler}

\begin{datadesc}{stdin}
\dataline{stdout}
\dataline{stderr}
  File objects corresponding to the interpreter's standard input,
  output and error streams.  \code{sys.stdin} is used for all
  interpreter input except for scripts but including calls to
  \code{input()} and \code{raw_input()}.  \code{sys.stdout} is used
  for the output of \code{print} and expression statements and for the
  prompts of \code{input()} and \code{raw_input()}.  The interpreter's
  own prompts and (almost all of) its error messages go to
  \code{sys.stderr}.  \code{sys.stdout} and \code{sys.stderr} needn't
  be built-in file objects: any object is acceptable as long as it has
  a \code{write} method that takes a string argument.  (Changing these
  objects doesn't affect the standard I/O streams of processes
  executed by \code{popen()}, \code{system()} or the \code{exec*()}
  family of functions in the \code{os} module.)
\stmodindex{os}
\end{datadesc}

\begin{datadesc}{tracebacklimit}
When this variable is set to an integer value, it determines the
maximum number of levels of traceback information printed when an
unhandled exception occurs.  The default is 1000.  When set to 0 or
less, all traceback information is suppressed and only the exception
type and value are printed.
\end{datadesc}

\section{Standard Module \sectcode{types}}
\stmodindex{types}

\renewcommand{\indexsubitem}{(in module types)}

This module defines names for all object types that are used by the
standard Python interpreter (but not for the types defined by various
extension modules).  It is safe to use ``\code{from types import *}'' ---
the module does not export any other names besides the ones listed
here.  New names exported by future versions of this module will
all end in \code{Type}.

Typical use is for functions that do different things depending on
their argument types, like the following:

\begin{verbatim}
from types import *
def delete(list, item):
    if type(item) is IntType:
       del list[item]
    else:
       list.remove(item)
\end{verbatim}

The module defines the following names:

\begin{datadesc}{NoneType}
The type of \code{None}.
\end{datadesc}

\begin{datadesc}{TypeType}
The type of type objects (such as returned by \code{type()}).
\end{datadesc}

\begin{datadesc}{IntType}
The type of integers (e.g. \code{1}).
\end{datadesc}

\begin{datadesc}{LongType}
The type of long integers (e.g. \code{1L}).
\end{datadesc}

\begin{datadesc}{FloatType}
The type of floating point numbers (e.g. \code{1.0}).
\end{datadesc}

\begin{datadesc}{StringType}
The type of character strings (e.g. \code{'Spam'}).
\end{datadesc}

\begin{datadesc}{TupleType}
The type of tuples (e.g. \code{(1, 2, 3, 'Spam')}).
\end{datadesc}

\begin{datadesc}{ListType}
The type of lists (e.g. \code{[0, 1, 2, 3]}).
\end{datadesc}

\begin{datadesc}{DictType}
The type of dictionaries (e.g. \code{\{'Bacon': 1, 'Ham': 0\}}).
\end{datadesc}

\begin{datadesc}{DictionaryType}
An alternative name for \code{DictType}.
\end{datadesc}

\begin{datadesc}{FunctionType}
The type of user-defined functions and lambdas.
\end{datadesc}

\begin{datadesc}{LambdaType}
	An alternative name for \code{FunctionType}.
\end{datadesc}

\begin{datadesc}{CodeType}
The type for code objects such as returned by \code{compile()}.
\end{datadesc}

\begin{datadesc}{ClassType}
The type of user-defined classes.
\end{datadesc}

\begin{datadesc}{InstanceType}
The type of instances of user-defined classes.
\end{datadesc}

\begin{datadesc}{MethodType}
The type of methods of user-defined class instances.
\end{datadesc}

\begin{datadesc}{UnboundMethodType}
An alternative name for \code{MethodType}.
\end{datadesc}

\begin{datadesc}{BuiltinFunctionType}
The type of built-in functions like \code{len} or \code{sys.exit}.
\end{datadesc}

\begin{datadesc}{BuiltinMethodType}
An alternative name for \code{BuiltinFunction}.
\end{datadesc}

\begin{datadesc}{ModuleType}
The type of modules.
\end{datadesc}

\begin{datadesc}{FileType}
The type of open file objects such as \code{sys.stdout}.
\end{datadesc}

\begin{datadesc}{XRangeType}
The type of range objects returned by \code{xrange()}.
\end{datadesc}

\begin{datadesc}{TracebackType}
The type of traceback objects such as found in \code{sys.exc_traceback}.
\end{datadesc}

\begin{datadesc}{FrameType}
The type of frame objects such as found in \code{tb.tb_frame} if
\code{tb} is a traceback object.
\end{datadesc}
		% types is already taken :-(
\section{Standard Module \sectcode{traceback}}
\stmodindex{traceback}

\renewcommand{\indexsubitem}{(in module traceback)}

This module provides a standard interface to format and print stack
traces of Python programs.  It exactly mimics the behavior of the
Python interpreter when it prints a stack trace.  This is useful when
you want to print stack traces under program control, e.g. in a
``wrapper'' around the interpreter.

The module uses traceback objects --- this is the object type
that is stored in the variables \code{sys.exc_traceback} and
\code{sys.last_traceback}.

The module defines the following functions:

\begin{funcdesc}{print_tb}{traceback\optional{\, limit}}
Print up to \var{limit} stack trace entries from \var{traceback}.  If
\var{limit} is omitted or \code{None}, all entries are printed.
\end{funcdesc}

\begin{funcdesc}{extract_tb}{traceback\optional{\, limit}}
Return a list of up to \var{limit} ``pre-processed'' stack trace
entries extracted from \var{traceback}.  It is useful for alternate
formatting of stack traces.  If \var{limit} is omitted or \code{None},
all entries are extracted.  A ``pre-processed'' stack trace entry is a
quadruple (\var{filename}, \var{line number}, \var{function name},
\var{line text}) representing the information that is usually printed
for a stack trace.  The \var{line text} is a string with leading and
trailing whitespace stripped; if the source is not available it is
\code{None}.
\end{funcdesc}

\begin{funcdesc}{print_exception}{type\, value\, traceback\optional{\, limit}}
Print exception information and up to \var{limit} stack trace entries
from \var{traceback}.  This differs from \code{print_tb} in the
following ways: (1) if \var{traceback} is not \code{None}, it prints a
header ``\code{Traceback (innermost last):}''; (2) it prints the
exception \var{type} and \var{value} after the stack trace; (3) if
\var{type} is \code{SyntaxError} and \var{value} has the appropriate
format, it prints the line where the syntax error occurred with a
caret indication the approximate position of the error.
\end{funcdesc}

\begin{funcdesc}{print_exc}{\optional{limit}}
This is a shorthand for \code{print_exception(sys.exc_type,}
\code{sys.exc_value,} \code{sys.exc_traceback,} \code{limit)}.
\end{funcdesc}

\begin{funcdesc}{print_last}{\optional{limit}}
This is a shorthand for \code{print_exception(sys.last_type,}
\code{sys.last_value,} \code{sys.last_traceback,} \code{limit)}.
\end{funcdesc}

\section{Standard Module \sectcode{pickle}}
\stmodindex{pickle}
\index{persistency}
\indexii{persistent}{objects}
\indexii{serializing}{objects}
\indexii{marshalling}{objects}
\indexii{flattening}{objects}
\indexii{pickling}{objects}

\renewcommand{\indexsubitem}{(in module pickle)}

The \code{pickle} module implements a basic but powerful algorithm for
``pickling'' (a.k.a.\ serializing, marshalling or flattening) nearly
arbitrary Python objects.  This is the act of converting objects to a
stream of bytes (and back: ``unpickling'').
This is a more primitive notion than
persistency --- although \code{pickle} reads and writes file objects,
it does not handle the issue of naming persistent objects, nor the
(even more complicated) area of concurrent access to persistent
objects.  The \code{pickle} module can transform a complex object into
a byte stream and it can transform the byte stream into an object with
the same internal structure.  The most obvious thing to do with these
byte streams is to write them onto a file, but it is also conceivable
to send them across a network or store them in a database.  The module
\code{shelve} provides a simple interface to pickle and unpickle
objects on ``dbm''-style database files.
\stmodindex{shelve}

Unlike the built-in module \code{marshal}, \code{pickle} handles the
following correctly:
\stmodindex{marshal}

\begin{itemize}

\item recursive objects (objects containing references to themselves)

\item object sharing (references to the same object in different places)

\item user-defined classes and their instances

\end{itemize}

The data format used by \code{pickle} is Python-specific.  This has
the advantage that there are no restrictions imposed by external
standards such as CORBA (which probably can't represent pointer
sharing or recursive objects); however it means that non-Python
programs may not be able to reconstruct pickled Python objects.

The \code{pickle} data format uses a printable \ASCII{} representation.
This is slightly more voluminous than a binary representation.
However, small integers actually take {\em less} space when
represented as minimal-size decimal strings than when represented as
32-bit binary numbers, and strings are only much longer if they
contain many control characters or 8-bit characters.  The big
advantage of using printable \ASCII{} (and of some other characteristics
of \code{pickle}'s representation) is that for debugging or recovery
purposes it is possible for a human to read the pickled file with a
standard text editor.  (I could have gone a step further and used a
notation like S-expressions, but the parser
(currently written in Python) would have been
considerably more complicated and slower, and the files would probably
have become much larger.)

The \code{pickle} module doesn't handle code objects, which the
\code{marshal} module does.  I suppose \code{pickle} could, and maybe
it should, but there's probably no great need for it right now (as
long as \code{marshal} continues to be used for reading and writing
code objects), and at least this avoids the possibility of smuggling
Trojan horses into a program.
\stmodindex{marshal}

For the benefit of persistency modules written using \code{pickle}, it
supports the notion of a reference to an object outside the pickled
data stream.  Such objects are referenced by a name, which is an
arbitrary string of printable \ASCII{} characters.  The resolution of
such names is not defined by the \code{pickle} module --- the
persistent object module will have to implement a method
\code{persistent_load}.  To write references to persistent objects,
the persistent module must define a method \code{persistent_id} which
returns either \code{None} or the persistent ID of the object.

There are some restrictions on the pickling of class instances.

First of all, the class must be defined at the top level in a module.

\renewcommand{\indexsubitem}{(pickle protocol)}

Next, it must normally be possible to create class instances by
calling the class without arguments.  Usually, this is best
accomplished by providing default values for all arguments to its
\code{__init__} method (if it has one).  If this is undesirable, the
class can define a method \code{__getinitargs__()}, which should
return a {\em tuple} containing the arguments to be passed to the
class constructor (\code{__init__()}).
\ttindex{__getinitargs__}
\ttindex{__init__}

Classes can further influence how their instances are pickled --- if the class
defines the method \code{__getstate__()}, it is called and the return
state is pickled as the contents for the instance, and if the class
defines the method \code{__setstate__()}, it is called with the
unpickled state.  (Note that these methods can also be used to
implement copying class instances.)  If there is no
\code{__getstate__()} method, the instance's \code{__dict__} is
pickled.  If there is no \code{__setstate__()} method, the pickled
object must be a dictionary and its items are assigned to the new
instance's dictionary.  (If a class defines both \code{__getstate__()}
and \code{__setstate__()}, the state object needn't be a dictionary
--- these methods can do what they want.)  This protocol is also used
by the shallow and deep copying operations defined in the \code{copy}
module.
\ttindex{__getstate__}
\ttindex{__setstate__}
\ttindex{__dict__}

Note that when class instances are pickled, their class's code and
data are not pickled along with them.  Only the instance data are
pickled.  This is done on purpose, so you can fix bugs in a class or
add methods and still load objects that were created with an earlier
version of the class.  If you plan to have long-lived objects that
will see many versions of a class, it may be worthwhile to put a version
number in the objects so that suitable conversions can be made by the
class's \code{__setstate__()} method.

When a class itself is pickled, only its name is pickled --- the class
definition is not pickled, but re-imported by the unpickling process.
Therefore, the restriction that the class must be defined at the top
level in a module applies to pickled classes as well.

\renewcommand{\indexsubitem}{(in module pickle)}

The interface can be summarized as follows.

To pickle an object \code{x} onto a file \code{f}, open for writing:

\begin{verbatim}
p = pickle.Pickler(f)
p.dump(x)
\end{verbatim}

A shorthand for this is:

\begin{verbatim}
pickle.dump(x, f)
\end{verbatim}

To unpickle an object \code{x} from a file \code{f}, open for reading:

\begin{verbatim}
u = pickle.Unpickler(f)
x = u.load()
\end{verbatim}

A shorthand is:

\begin{verbatim}
x = pickle.load(f)
\end{verbatim}

The \code{Pickler} class only calls the method \code{f.write} with a
string argument.  The \code{Unpickler} calls the methods \code{f.read}
(with an integer argument) and \code{f.readline} (without argument),
both returning a string.  It is explicitly allowed to pass non-file
objects here, as long as they have the right methods.
\ttindex{Unpickler}
\ttindex{Pickler}

The following types can be pickled:
\begin{itemize}

\item \code{None}

\item integers, long integers, floating point numbers

\item strings

\item tuples, lists and dictionaries containing only picklable objects

\item classes that are defined at the top level in a module

\item instances of such classes whose \code{__dict__} or
\code{__setstate__()} is picklable

\end{itemize}

Attempts to pickle unpicklable objects will raise the
\code{PicklingError} exception; when this happens, an unspecified
number of bytes may have been written to the file.

It is possible to make multiple calls to the \code{dump()} method of
the same \code{Pickler} instance.  These must then be matched to the
same number of calls to the \code{load()} instance of the
corresponding \code{Unpickler} instance.  If the same object is
pickled by multiple \code{dump()} calls, the \code{load()} will all
yield references to the same object.  {\em Warning}: this is intended
for pickling multiple objects without intervening modifications to the
objects or their parts.  If you modify an object and then pickle it
again using the same \code{Pickler} instance, the object is not
pickled again --- a reference to it is pickled and the
\code{Unpickler} will return the old value, not the modified one.
(There are two problems here: (a) detecting changes, and (b)
marshalling a minimal set of changes.  I have no answers.  Garbage
Collection may also become a problem here.)

Apart from the \code{Pickler} and \code{Unpickler} classes, the
module defines the following functions, and an exception:

\begin{funcdesc}{dump}{object\, file}
Write a pickled representation of \var{obect} to the open file object
\var{file}.  This is equivalent to \code{Pickler(file).dump(object)}.
\end{funcdesc}

\begin{funcdesc}{load}{file}
Read a pickled object from the open file object \var{file}.  This is
equivalent to \code{Unpickler(file).load()}.
\end{funcdesc}

\begin{funcdesc}{dumps}{object}
Return the pickled representation of the object as a string, instead
of writing it to a file.
\end{funcdesc}

\begin{funcdesc}{loads}{string}
Read a pickled object from a string instead of a file.  Characters in
the string past the pickled object's representation are ignored.
\end{funcdesc}

\begin{excdesc}{PicklingError}
This exception is raised when an unpicklable object is passed to
\code{Pickler.dump()}.
\end{excdesc}

\section{Standard Module \sectcode{shelve}}
\stmodindex{shelve}
\stmodindex{pickle}
\bimodindex{dbm}
\bimodindex{gdbm}

A ``shelf'' is a persistent, dictionary-like object.  The difference
with ``dbm'' databases is that the values (not the keys!) in a shelf
can be essentially arbitrary Python objects --- anything that the
\code{pickle} module can handle.  This includes most class instances,
recursive data types, and objects containing lots of shared
sub-objects.  The keys are ordinary strings.

To summarize the interface (\code{key} is a string, \code{data} is an
arbitrary object):

\begin{verbatim}
import shelve

d = shelve.open(filename) # open, with (g)dbm filename -- no suffix

d[key] = data   # store data at key (overwrites old data if
                # using an existing key)
data = d[key]   # retrieve data at key (raise KeyError if no
                # such key)
del d[key]      # delete data stored at key (raises KeyError
                # if no such key)
flag = d.has_key(key)   # true if the key exists
list = d.keys() # a list of all existing keys (slow!)

d.close()       # close it
\end{verbatim}

Restrictions:

\begin{itemize}

\item
The choice of which database package will be used (e.g. dbm or gdbm)
depends on which interface is available.  Therefore it isn't safe to
open the database directly using dbm.  The database is also
(unfortunately) subject to the limitations of dbm, if it is used ---
this means that (the pickled representation of) the objects stored in
the database should be fairly small, and in rare cases key collisions
may cause the database to refuse updates.

\item
Dependent on the implementation, closing a persistent dictionary may
or may not be necessary to flush changes to disk.

\item
The \code{shelve} module does not support {\em concurrent} read/write
access to shelved objects.  (Multiple simultaneous read accesses are
safe.)  When a program has a shelf open for writing, no other program
should have it open for reading or writing.  \UNIX{} file locking can
be used to solve this, but this differs across \UNIX{} versions and
requires knowledge about the database implementation used.

\end{itemize}

\section{Standard Module \sectcode{copy}}
\stmodindex{copy}
\renewcommand{\indexsubitem}{(copy function)}
\ttindex{copy}
\ttindex{deepcopy}

This module provides generic (shallow and deep) copying operations.

Interface summary:

\begin{verbatim}
import copy

x = copy.copy(y)        # make a shallow copy of y
x = copy.deepcopy(y)    # make a deep copy of y
\end{verbatim}

For module specific errors, \code{copy.error} is raised.

The difference between shallow and deep copying is only relevant for
compound objects (objects that contain other objects, like lists or
class instances):

\begin{itemize}

\item
A {\em shallow copy} constructs a new compound object and then (to the
extent possible) inserts {\em references} into it to the objects found
in the original.

\item
A {\em deep copy} constructs a new compound object and then,
recursively, inserts {\em copies} into it of the objects found in the
original.

\end{itemize}

Two problems often exist with deep copy operations that don't exist
with shallow copy operations:

\begin{itemize}

\item
Recursive objects (compound objects that, directly or indirectly,
contain a reference to themselves) may cause a recursive loop.

\item
Because deep copy copies {\em everything} it may copy too much, e.g.\
administrative data structures that should be shared even between
copies.

\end{itemize}

Python's \code{deepcopy()} operation avoids these problems by:

\begin{itemize}

\item
keeping a table of objects already copied during the current
copying pass; and

\item
letting user-defined classes override the copying operation or the
set of components copied.

\end{itemize}

This version does not copy types like module, class, function, method,
nor stack trace, stack frame, nor file, socket, window, nor array, nor
any similar types.

Classes can use the same interfaces to control copying that they use
to control pickling: they can define methods called
\code{__getinitargs__()}, \code{__getstate__()} and
\code{__setstate__()}.  See the description of module \code{pickle}
for information on these methods.
\stmodindex{pickle}
\renewcommand{\indexsubitem}{(copy protocol)}
\ttindex{__getinitargs__}
\ttindex{__getstate__}
\ttindex{__setstate__}

\section{Built-in Module \sectcode{marshal}}

\bimodindex{marshal}
This module contains functions that can read and write Python
values in a binary format.  The format is specific to Python, but
independent of machine architecture issues (e.g., you can write a
Python value to a file on a PC, transport the file to a Sun, and read
it back there).  Details of the format are undocumented on purpose;
it may change between Python versions (although it rarely does).%
\footnote{The name of this module stems from a bit of terminology used
by the designers of Modula-3 (amongst others), who use the term
``marshalling'' for shipping of data around in a self-contained form.
Strictly speaking, ``to marshal'' means to convert some data from
internal to external form (in an RPC buffer for instance) and
``unmarshalling'' for the reverse process.}

This is not a general ``persistency'' module.  For general persistency
and transfer of Python objects through RPC calls, see the modules
\code{pickle} and \code{shelve}.  The \code{marshal} module exists
mainly to support reading and writing the ``pseudo-compiled'' code for
Python modules of \samp{.pyc} files.
\stmodindex{pickle}
\stmodindex{shelve}
\obindex{code}

Not all Python object types are supported; in general, only objects
whose value is independent from a particular invocation of Python can
be written and read by this module.  The following types are supported:
\code{None}, integers, long integers, floating point numbers,
strings, tuples, lists, dictionaries, and code objects, where it
should be understood that tuples, lists and dictionaries are only
supported as long as the values contained therein are themselves
supported; and recursive lists and dictionaries should not be written
(they will cause infinite loops).

{\bf Caveat:} On machines where C's \code{long int} type has more than
32 bits (such as the DEC Alpha), it
is possible to create plain Python integers that are longer than 32
bits.  Since the current \code{marshal} module uses 32 bits to
transfer plain Python integers, such values are silently truncated.
This particularly affects the use of very long integer literals in
Python modules --- these will be accepted by the parser on such
machines, but will be silently be truncated when the module is read
from the \code{.pyc} instead.%
\footnote{A solution would be to refuse such literals in the parser,
since they are inherently non-portable.  Another solution would be to
let the \code{marshal} module raise an exception when an integer value
would be truncated.  At least one of these solutions will be
implemented in a future version.}

There are functions that read/write files as well as functions
operating on strings.

The module defines these functions:

\renewcommand{\indexsubitem}{(in module marshal)}

\begin{funcdesc}{dump}{value\, file}
  Write the value on the open file.  The value must be a supported
  type.  The file must be an open file object such as
  \code{sys.stdout} or returned by \code{open()} or
  \code{posix.popen()}.
  
  If the value has (or contains an object that has) an unsupported type,
  a \code{ValueError} exception is raised -- but garbage data will also
  be written to the file.  The object will not be properly read back by
  \code{load()}.
\end{funcdesc}

\begin{funcdesc}{load}{file}
  Read one value from the open file and return it.  If no valid value
  is read, raise \code{EOFError}, \code{ValueError} or
  \code{TypeError}.  The file must be an open file object.

  Warning: If an object containing an unsupported type was marshalled
  with \code{dump()}, \code{load()} will substitute \code{None} for the
  unmarshallable type.
\end{funcdesc}

\begin{funcdesc}{dumps}{value}
  Return the string that would be written to a file by
  \code{dump(value, file)}.  The value must be a supported type.
  Raise a \code{ValueError} exception if value has (or contains an
  object that has) an unsupported type.
\end{funcdesc}

\begin{funcdesc}{loads}{string}
  Convert the string to a value.  If no valid value is found, raise
  \code{EOFError}, \code{ValueError} or \code{TypeError}.  Extra
  characters in the string are ignored.
\end{funcdesc}

\section{Built-in Module \sectcode{imp}}
\bimodindex{imp}
\index{import}

This module provides an interface to the mechanisms used to implement
the \code{import} statement.  It defines the following constants and
functions:

\renewcommand{\indexsubitem}{(in module imp)}

\begin{funcdesc}{get_magic}{}
Return the magic string value used to recognize byte-compiled code
files (``\code{.pyc} files'').
\end{funcdesc}

\begin{funcdesc}{get_suffixes}{}
Return a list of triples, each describing a particular type of file.
Each triple has the form \code{(\var{suffix}, \var{mode},
\var{type})}, where \var{suffix} is a string to be appended to the
module name to form the filename to search for, \var{mode} is the mode
string to pass to the built-in \code{open} function to open the file
(this can be \code{'r'} for text files or \code{'rb'} for binary
files), and \var{type} is the file type, which has one of the values
\code{PY_SOURCE}, \code{PY_COMPILED} or \code{C_EXTENSION}, defined
below.  (System-dependent values may also be returned.)
\end{funcdesc}

\begin{funcdesc}{find_module}{name\, \optional{path}}
Try to find the module \var{name} on the search path \var{path}.  The
default \var{path} is \code{sys.path}.  The return value is a triple
\code{(\var{file}, \var{pathname}, \var{description})} where
\var{file} is an open file object positioned at the beginning,
\var{pathname} is the pathname of the
file found, and \var{description} is a triple as contained in the list
returned by \code{get_suffixes} describing the kind of file found.
\end{funcdesc}

\begin{funcdesc}{init_builtin}{name}
Initialize the built-in module called \var{name} and return its module
object.  If the module was already initialized, it will be initialized
{\em again}.  A few modules cannot be initialized twice --- attempting
to initialize these again will raise an \code{ImportError} exception.
If there is no
built-in module called \var{name}, \code{None} is returned.
\end{funcdesc}

\begin{funcdesc}{init_frozen}{name}
Initialize the frozen module called \var{name} and return its module
object.  If the module was already initialized, it will be initialized
{\em again}.  If there is no frozen module called \var{name},
\code{None} is returned.  (Frozen modules are modules written in
Python whose compiled byte-code object is incorporated into a
custom-built Python interpreter by Python's \code{freeze} utility.
See \code{Tools/freeze} for now.)
\end{funcdesc}

\begin{funcdesc}{is_builtin}{name}
Return \code{1} if there is a built-in module called \var{name} which can be
initialized again.  Return \code{-1} if there is a built-in module
called \var{name} which cannot be initialized again (see
\code{init_builtin}).  Return \code{0} if there is no built-in module
called \var{name}.
\end{funcdesc}

\begin{funcdesc}{is_frozen}{name}
Return \code{1} if there is a frozen module (see \code{init_frozen})
called \var{name}, \code{0} if there is no such module.
\end{funcdesc}

\begin{funcdesc}{load_compiled}{name\, pathname\, file}
Load and initialize a module implemented as a byte-compiled code file
and return its module object.  If the module was already initialized,
it will be initialized {\em again}.  The \var{name} argument is used
to create or access a module object.  The \var{pathname} argument
points to the byte-compiled code file.  The \var{file}
argument is the byte-compiled code file, open for reading in binary
mode, from the beginning.
It must currently be a real file object, not a
user-defined class emulating a file.
\end{funcdesc}

\begin{funcdesc}{load_dynamic}{name\, pathname\, \optional{file}}
Load and initialize a module implemented as a dynamically loadable
shared library and return its module object.  If the module was
already initialized, it will be initialized {\em again}.  Some modules
don't like that and may raise an exception.  The \var{pathname}
argument must point to the shared library.  The \var{name} argument is
used to construct the name of the initialization function: an external
C function called \code{init\var{name}()} in the shared library is
called.  The optional \var{file} argment is ignored.  (Note: using
shared libraries is highly system dependent, and not all systems
support it.)
\end{funcdesc}

\begin{funcdesc}{load_source}{name\, pathname\, file}
Load and initialize a module implemented as a Python source file and
return its module object.  If the module was already initialized, it
will be initialized {\em again}.  The \var{name} argument is used to
create or access a module object.  The \var{pathname} argument points
to the source file.  The \var{file} argument is the source
file, open for reading as text, from the beginning.
It must currently be a real file
object, not a user-defined class emulating a file.  Note that if a
properly matching byte-compiled file (with suffix \code{.pyc}) exists,
it will be used instead of parsing the given source file.
\end{funcdesc}

\begin{funcdesc}{new_module}{name}
Return a new empty module object called \var{name}.  This object is
{\em not} inserted in \code{sys.modules}.
\end{funcdesc}

The following constants with integer values, defined in the module,
are used to indicate the search result of \code{imp.find_module}.

\begin{datadesc}{SEARCH_ERROR}
The module was not found.
\end{datadesc}

\begin{datadesc}{PY_SOURCE}
The module was found as a source file.
\end{datadesc}

\begin{datadesc}{PY_COMPILED}
The module was found as a compiled code object file.
\end{datadesc}

\begin{datadesc}{C_EXTENSION}
The module was found as dynamically loadable shared library.
\end{datadesc}

\subsection{Examples}
The following function emulates the default import statement:

\begin{verbatim}
import imp
import sys

def __import__(name, globals=None, locals=None, fromlist=None):
    # Fast path: see if the module has already been imported.
    if sys.modules.has_key(name):
        return sys.modules[name]

    # If any of the following calls raises an exception,
    # there's a problem we can't handle -- let the caller handle it.

    # See if it's a built-in module.
    m = imp.init_builtin(name)
    if m:
        return m

    # See if it's a frozen module.
    m = imp.init_frozen(name)
    if m:
        return m

    # Search the default path (i.e. sys.path).
    fp, pathname, (suffix, mode, type) = imp.find_module(name)

    # See what we got.
    try:
        if type == imp.C_EXTENSION:
            return imp.load_dynamic(name, pathname)
        if type == imp.PY_SOURCE:
            return imp.load_source(name, pathname, fp)
        if type == imp.PY_COMPILED:
            return imp.load_compiled(name, pathname, fp)

        # Shouldn't get here at all.
        raise ImportError, '%s: unknown module type (%d)' % (name, type)
    finally:
        # Since we may exit via an exception, close fp explicitly.
        fp.close()
\end{verbatim}

% libparser.tex
%
% Introductory documentation for the new parser built-in module.
%
% Copyright 1995 Virginia Polytechnic Institute and State University
% and Fred L. Drake, Jr.  This copyright notice must be distributed on
% all copies, but this document otherwise may be distributed as part
% of the Python distribution.  No fee may be charged for this document
% in any representation, either on paper or electronically.  This
% restriction does not affect other elements in a distributed package
% in any way.
%

\section{Built-in Module \sectcode{parser}}
\bimodindex{parser}

The \code{parser} module provides an interface to Python's internal
parser and byte-code compiler.  The primary purpose for this interface
is to allow Python code to edit the parse tree of a Python expression
and create executable code from this.  This is better than trying
to parse and modify an arbitrary Python code fragment as a string
because parsing is performed in a manner identical to the code
forming the application.  It is also faster.

There are a few things to note about this module which are important
to making use of the data structures created.  This is not a tutorial
on editing the parse trees for Python code, but some examples of using
the \code{parser} module are presented.

Most importantly, a good understanding of the Python grammar processed
by the internal parser is required.  For full information on the
language syntax, refer to the Language Reference.  The parser itself
is created from a grammar specification defined in the file
\file{Grammar/Grammar} in the standard Python distribution.  The parse
trees stored in the ``AST objects'' created by this module are the
actual output from the internal parser when created by the
\code{expr()} or \code{suite()} functions, described below.  The AST
objects created by \code{sequence2ast()} faithfully simulate those
structures.  Be aware that the values of the sequences which are
considered ``correct'' will vary from one version of Python to another
as the formal grammar for the language is revised.  However,
transporting code from one Python version to another as source text
will always allow correct parse trees to be created in the target
version, with the only restriction being that migrating to an older
version of the interpreter will not support more recent language
constructs.  The parse trees are not typically compatible from one
version to another, whereas source code has always been
forward-compatible.

Each element of the sequences returned by \code{ast2list} or
\code{ast2tuple()} has a simple form.  Sequences representing
non-terminal elements in the grammar always have a length greater than
one.  The first element is an integer which identifies a production in
the grammar.  These integers are given symbolic names in the C header
file \file{Include/graminit.h} and the Python module
\file{Lib/symbol.py}.  Each additional element of the sequence represents
a component of the production as recognized in the input string: these
are always sequences which have the same form as the parent.  An
important aspect of this structure which should be noted is that
keywords used to identify the parent node type, such as the keyword
\code{if} in an \code{if_stmt}, are included in the node tree without
any special treatment.  For example, the \code{if} keyword is
represented by the tuple \code{(1, 'if')}, where \code{1} is the
numeric value associated with all \code{NAME} tokens, including
variable and function names defined by the user.  In an alternate form
returned when line number information is requested, the same token
might be represented as \code{(1, 'if', 12)}, where the \code{12}
represents the line number at which the terminal symbol was found.

Terminal elements are represented in much the same way, but without
any child elements and the addition of the source text which was
identified.  The example of the \code{if} keyword above is
representative.  The various types of terminal symbols are defined in
the C header file \file{Include/token.h} and the Python module
\file{Lib/token.py}.

The AST objects are not required to support the functionality of this
module, but are provided for three purposes: to allow an application
to amortize the cost of processing complex parse trees, to provide a
parse tree representation which conserves memory space when compared
to the Python list or tuple representation, and to ease the creation
of additional modules in C which manipulate parse trees.  A simple
``wrapper'' class may be created in Python to hide the use of AST
objects; the \code{AST} library module provides a variety of such
classes.

The \code{parser} module defines functions for a few distinct
purposes.  The most important purposes are to create AST objects and
to convert AST objects to other representations such as parse trees
and compiled code objects, but there are also functions which serve to
query the type of parse tree represented by an AST object.

\renewcommand{\indexsubitem}{(in module parser)}


\subsection{Creating AST Objects}

AST objects may be created from source code or from a parse tree.
When creating an AST object from source, different functions are used
to create the \code{'eval'} and \code{'exec'} forms.

\begin{funcdesc}{expr}{string}
The \code{expr()} function parses the parameter \code{\var{string}}
as if it were an input to \code{compile(\var{string}, 'eval')}.  If
the parse succeeds, an AST object is created to hold the internal
parse tree representation, otherwise an appropriate exception is
thrown.
\end{funcdesc}

\begin{funcdesc}{suite}{string}
The \code{suite()} function parses the parameter \code{\var{string}}
as if it were an input to \code{compile(\var{string}, 'exec')}.  If
the parse succeeds, an AST object is created to hold the internal
parse tree representation, otherwise an appropriate exception is
thrown.
\end{funcdesc}

\begin{funcdesc}{sequence2ast}{sequence}
This function accepts a parse tree represented as a sequence and
builds an internal representation if possible.  If it can validate
that the tree conforms to the Python grammar and all nodes are valid
node types in the host version of Python, an AST object is created
from the internal representation and returned to the called.  If there
is a problem creating the internal representation, or if the tree
cannot be validated, a \code{ParserError} exception is thrown.  An AST
object created this way should not be assumed to compile correctly;
normal exceptions thrown by compilation may still be initiated when
the AST object is passed to \code{compileast()}.  This may indicate
problems not related to syntax (such as a \code{MemoryError}
exception), but may also be due to constructs such as the result of
parsing \code{del f(0)}, which escapes the Python parser but is
checked by the bytecode compiler.

Sequences representing terminal tokens may be represented as either
two-element lists of the form \code{(1, 'name')} or as three-element
lists of the form \code{(1, 'name', 56)}.  If the third element is
present, it is assumed to be a valid line number.  The line number
may be specified for any subset of the terminal symbols in the input
tree.
\end{funcdesc}

\begin{funcdesc}{tuple2ast}{sequence}
This is the same function as \code{sequence2ast()}.  This entry point
is maintained for backward compatibility.
\end{funcdesc}


\subsection{Converting AST Objects}

AST objects, regardless of the input used to create them, may be
converted to parse trees represented as list- or tuple- trees, or may
be compiled into executable code objects.  Parse trees may be
extracted with or without line numbering information.

\begin{funcdesc}{ast2list}{ast\optional{\, line_info\code{ = 0}}}
This function accepts an AST object from the caller in
\code{\var{ast}} and returns a Python list representing the
equivelent parse tree.  The resulting list representation can be used
for inspection or the creation of a new parse tree in list form.  This
function does not fail so long as memory is available to build the
list representation.  If the parse tree will only be used for
inspection, \code{ast2tuple()} should be used instead to reduce memory
consumption and fragmentation.  When the list representation is
required, this function is significantly faster than retrieving a
tuple representation and converting that to nested lists.

If \code{\var{line_info}} is true, line number information will be
included for all terminal tokens as a third element of the list
representing the token.  This information is omitted if the flag is
false or omitted.
\end{funcdesc}

\begin{funcdesc}{ast2tuple}{ast\optional{\, line_info\code{ = 0}}}
This function accepts an AST object from the caller in
\code{\var{ast}} and returns a Python tuple representing the
equivelent parse tree.  Other than returning a tuple instead of a
list, this function is identical to \code{ast2list()}.

If \code{\var{line_info}} is true, line number information will be
included for all terminal tokens as a third element of the list
representing the token.  This information is omitted if the flag is
false or omitted.
\end{funcdesc}

\begin{funcdesc}{compileast}{ast\optional{\, filename\code{ = '<ast>'}}}
The Python byte compiler can be invoked on an AST object to produce
code objects which can be used as part of an \code{exec} statement or
a call to the built-in \code{eval()} function.  This function provides
the interface to the compiler, passing the internal parse tree from
\code{\var{ast}} to the parser, using the source file name specified
by the \code{\var{filename}} parameter.  The default value supplied
for \code{\var{filename}} indicates that the source was an AST object.

Compiling an AST object may result in exceptions related to
compilation; an example would be a \code{SyntaxError} caused by the
parse tree for \code{del f(0)}: this statement is considered legal
within the formal grammar for Python but is not a legal language
construct.  The \code{SyntaxError} raised for this condition is
actually generated by the Python byte-compiler normally, which is why
it can be raised at this point by the \code{parser} module.  Most
causes of compilation failure can be diagnosed programmatically by
inspection of the parse tree.
\end{funcdesc}


\subsection{Queries on AST Objects}

Two functions are provided which allow an application to determine if
an AST was create as an expression or a suite.  Neither of these
functions can be used to determine if an AST was created from source
code via \code{expr()} or \code{suite()} or from a parse tree via
\code{sequence2ast()}.

\begin{funcdesc}{isexpr}{ast}
When \code{\var{ast}} represents an \code{'eval'} form, this function
returns a true value (\code{1}), otherwise it returns false
(\code{0}).  This is useful, since code objects normally cannot be
queried for this information using existing built-in functions.  Note
that the code objects created by \code{compileast()} cannot be queried
like this either, and are identical to those created by the built-in
\code{compile()} function.
\end{funcdesc}


\begin{funcdesc}{issuite}{ast}
This function mirrors \code{isexpr()} in that it reports whether an
AST object represents an \code{'exec'} form, commonly known as a
``suite.''  It is not safe to assume that this function is equivelent
to \code{not isexpr(\var{ast})}, as additional syntactic fragments may
be supported in the future.
\end{funcdesc}


\subsection{Exceptions and Error Handling}

The parser module defines a single exception, but may also pass other
built-in exceptions from other portions of the Python runtime
environment.  See each function for information about the exceptions
it can raise.

\begin{excdesc}{ParserError}
Exception raised when a failure occurs within the parser module.  This
is generally produced for validation failures rather than the built in
\code{SyntaxError} thrown during normal parsing.
The exception argument is either a string describing the reason of the
failure or a tuple containing a sequence causing the failure from a parse
tree passed to \code{sequence2ast()} and an explanatory string.  Calls to
\code{sequence2ast()} need to be able to handle either type of exception,
while calls to other functions in the module will only need to be
aware of the simple string values.
\end{excdesc}

Note that the functions \code{compileast()}, \code{expr()}, and
\code{suite()} may throw exceptions which are normally thrown by the
parsing and compilation process.  These include the built in
exceptions \code{MemoryError}, \code{OverflowError},
\code{SyntaxError}, and \code{SystemError}.  In these cases, these
exceptions carry all the meaning normally associated with them.  Refer
to the descriptions of each function for detailed information.


\subsection{AST Objects}

AST objects returned by \code{expr()}, \code{suite()}, and
\code{sequence2ast()} have no methods of their own.
Some of the functions defined which accept an AST object as their
first argument may change to object methods in the future.  The type
of these objects is available as \code{ASTType} in the module.

Ordered and equality comparisons are supported between AST objects.


\subsection{Examples}

The parser modules allows operations to be performed on the parse tree
of Python source code before the bytecode is generated, and provides
for inspection of the parse tree for information gathering purposes.
Two examples are presented.  The simple example demonstrates emulation
of the \code{compile()} built-in function and the complex example
shows the use of a parse tree for information discovery.

\subsubsection{Emulation of \sectcode{compile()}}

While many useful operations may take place between parsing and
bytecode generation, the simplest operation is to do nothing.  For
this purpose, using the \code{parser} module to produce an
intermediate data structure is equivelent to the code

\begin{verbatim}
>>> code = compile('a + 5', 'eval')
>>> a = 5
>>> eval(code)
10
\end{verbatim}

The equivelent operation using the \code{parser} module is somewhat
longer, and allows the intermediate internal parse tree to be retained
as an AST object:

\begin{verbatim}
>>> import parser
>>> ast = parser.expr('a + 5')
>>> code = parser.compileast(ast)
>>> a = 5
>>> eval(code)
10
\end{verbatim}

An application which needs both AST and code objects can package this
code into readily available functions:

\begin{verbatim}
import parser

def load_suite(source_string):
    ast = parser.suite(source_string)
    code = parser.compileast(ast)
    return ast, code

def load_expression(source_string):
    ast = parser.expr(source_string)
    code = parser.compileast(ast)
    return ast, code
\end{verbatim}

\subsubsection{Information Discovery}

Some applications benefit from direct access to the parse tree.  The
remainder of this section demonstrates how the parse tree provides
access to module documentation defined in docstrings without requiring
that the code being examined be loaded into a running interpreter via
\code{import}.  This can be very useful for performing analyses of
untrusted code.

Generally, the example will demonstrate how the parse tree may be
traversed to distill interesting information.  Two functions and a set
of classes are developed which provide programmatic access to high
level function and class definitions provided by a module.  The
classes extract information from the parse tree and provide access to
the information at a useful semantic level, one function provides a
simple low-level pattern matching capability, and the other function
defines a high-level interface to the classes by handling file
operations on behalf of the caller.  All source files mentioned here
which are not part of the Python installation are located in the
\file{Demo/parser/} directory of the distribution.

The dynamic nature of Python allows the programmer a great deal of
flexibility, but most modules need only a limited measure of this when
defining classes, functions, and methods.  In this example, the only
definitions that will be considered are those which are defined in the
top level of their context, e.g., a function defined by a \code{def}
statement at column zero of a module, but not a function defined
within a branch of an \code{if} ... \code{else} construct, though
there are some good reasons for doing so in some situations.  Nesting
of definitions will be handled by the code developed in the example.

To construct the upper-level extraction methods, we need to know what
the parse tree structure looks like and how much of it we actually
need to be concerned about.  Python uses a moderately deep parse tree
so there are a large number of intermediate nodes.  It is important to
read and understand the formal grammar used by Python.  This is
specified in the file \file{Grammar/Grammar} in the distribution.
Consider the simplest case of interest when searching for docstrings:
a module consisting of a docstring and nothing else.  (See file
\file{docstring.py}.)

\begin{verbatim}
"""Some documentation.
"""
\end{verbatim}

Using the interpreter to take a look at the parse tree, we find a
bewildering mass of numbers and parentheses, with the documentation
buried deep in nested tuples.

\begin{verbatim}
>>> import parser
>>> import pprint
>>> ast = parser.suite(open('docstring.py').read())
>>> tup = parser.ast2tuple(ast)
>>> pprint.pprint(tup)
(257,
 (264,
  (265,
   (266,
    (267,
     (307,
      (287,
       (288,
        (289,
         (290,
          (292,
           (293,
            (294,
             (295,
              (296,
               (297,
                (298,
                 (299,
                  (300, (3, '"""Some documentation.\012"""'))))))))))))))))),
   (4, ''))),
 (4, ''),
 (0, ''))
\end{verbatim}

The numbers at the first element of each node in the tree are the node
types; they map directly to terminal and non-terminal symbols in the
grammar.  Unfortunately, they are represented as integers in the
internal representation, and the Python structures generated do not
change that.  However, the \code{symbol} and \code{token} modules
provide symbolic names for the node types and dictionaries which map
from the integers to the symbolic names for the node types.

In the output presented above, the outermost tuple contains four
elements: the integer \code{257} and three additional tuples.  Node
type \code{257} has the symbolic name \code{file_input}.  Each of
these inner tuples contains an integer as the first element; these
integers, \code{264}, \code{4}, and \code{0}, represent the node types
\code{stmt}, \code{NEWLINE}, and \code{ENDMARKER}, respectively.
Note that these values may change depending on the version of Python
you are using; consult \file{symbol.py} and \file{token.py} for
details of the mapping.  It should be fairly clear that the outermost
node is related primarily to the input source rather than the contents
of the file, and may be disregarded for the moment.  The \code{stmt}
node is much more interesting.  In particular, all docstrings are
found in subtrees which are formed exactly as this node is formed,
with the only difference being the string itself.  The association
between the docstring in a similar tree and the defined entity (class,
function, or module) which it describes is given by the position of
the docstring subtree within the tree defining the described
structure.

By replacing the actual docstring with something to signify a variable
component of the tree, we allow a simple pattern matching approach to
check any given subtree for equivelence to the general pattern for
docstrings.  Since the example demonstrates information extraction, we
can safely require that the tree be in tuple form rather than list
form, allowing a simple variable representation to be
\code{['variable_name']}.  A simple recursive function can implement
the pattern matching, returning a boolean and a dictionary of variable
name to value mappings.  (See file \file{example.py}.)

\begin{verbatim}
from types import ListType, TupleType

def match(pattern, data, vars=None):
    if vars is None:
        vars = {}
    if type(pattern) is ListType:
        vars[pattern[0]] = data
        return 1, vars
    if type(pattern) is not TupleType:
        return (pattern == data), vars
    if len(data) != len(pattern):
        return 0, vars
    for pattern, data in map(None, pattern, data):
        same, vars = match(pattern, data, vars)
        if not same:
            break
    return same, vars
\end{verbatim}

Using this simple representation for syntactic variables and the symbolic
node types, the pattern for the candidate docstring subtrees becomes
fairly readable.  (See file \file{example.py}.)

\begin{verbatim}
import symbol
import token

DOCSTRING_STMT_PATTERN = (
    symbol.stmt,
    (symbol.simple_stmt,
     (symbol.small_stmt,
      (symbol.expr_stmt,
       (symbol.testlist,
        (symbol.test,
         (symbol.and_test,
          (symbol.not_test,
           (symbol.comparison,
            (symbol.expr,
             (symbol.xor_expr,
              (symbol.and_expr,
               (symbol.shift_expr,
                (symbol.arith_expr,
                 (symbol.term,
                  (symbol.factor,
                   (symbol.power,
                    (symbol.atom,
                     (token.STRING, ['docstring'])
                     )))))))))))))))),
     (token.NEWLINE, '')
     ))
\end{verbatim}

Using the \code{match()} function with this pattern, extracting the
module docstring from the parse tree created previously is easy:

\begin{verbatim}
>>> found, vars = match(DOCSTRING_STMT_PATTERN, tup[1])
>>> found
1
>>> vars
{'docstring': '"""Some documentation.\012"""'}
\end{verbatim}

Once specific data can be extracted from a location where it is
expected, the question of where information can be expected
needs to be answered.  When dealing with docstrings, the answer is
fairly simple: the docstring is the first \code{stmt} node in a code
block (\code{file_input} or \code{suite} node types).  A module
consists of a single \code{file_input} node, and class and function
definitions each contain exactly one \code{suite} node.  Classes and
functions are readily identified as subtrees of code block nodes which
start with \code{(stmt, (compound_stmt, (classdef, ...} or
\code{(stmt, (compound_stmt, (funcdef, ...}.  Note that these subtrees
cannot be matched by \code{match()} since it does not support multiple
sibling nodes to match without regard to number.  A more elaborate
matching function could be used to overcome this limitation, but this
is sufficient for the example.

Given the ability to determine whether a statement might be a
docstring and extract the actual string from the statement, some work
needs to be performed to walk the parse tree for an entire module and
extract information about the names defined in each context of the
module and associate any docstrings with the names.  The code to
perform this work is not complicated, but bears some explanation.

The public interface to the classes is straightforward and should
probably be somewhat more flexible.  Each ``major'' block of the
module is described by an object providing several methods for inquiry
and a constructor which accepts at least the subtree of the complete
parse tree which it represents.  The \code{ModuleInfo} constructor
accepts an optional \code{\var{name}} parameter since it cannot
otherwise determine the name of the module.

The public classes include \code{ClassInfo}, \code{FunctionInfo},
and \code{ModuleInfo}.  All objects provide the
methods \code{get_name()}, \code{get_docstring()},
\code{get_class_names()}, and \code{get_class_info()}.  The
\code{ClassInfo} objects support \code{get_method_names()} and
\code{get_method_info()} while the other classes provide
\code{get_function_names()} and \code{get_function_info()}.

Within each of the forms of code block that the public classes
represent, most of the required information is in the same form and is
accessed in the same way, with classes having the distinction that
functions defined at the top level are referred to as ``methods.''
Since the difference in nomenclature reflects a real semantic
distinction from functions defined outside of a class, the
implementation needs to maintain the distinction.
Hence, most of the functionality of the public classes can be
implemented in a common base class, \code{SuiteInfoBase}, with the
accessors for function and method information provided elsewhere.
Note that there is only one class which represents function and method
information; this paralels the use of the \code{def} statement to
define both types of elements.

Most of the accessor functions are declared in \code{SuiteInfoBase}
and do not need to be overriden by subclasses.  More importantly, the
extraction of most information from a parse tree is handled through a
method called by the \code{SuiteInfoBase} constructor.  The example
code for most of the classes is clear when read alongside the formal
grammar, but the method which recursively creates new information
objects requires further examination.  Here is the relevant part of
the \code{SuiteInfoBase} definition from \file{example.py}:

\begin{verbatim}
class SuiteInfoBase:
    _docstring = ''
    _name = ''

    def __init__(self, tree = None):
        self._class_info = {}
        self._function_info = {}
        if tree:
            self._extract_info(tree)

    def _extract_info(self, tree):
        # extract docstring
        if len(tree) == 2:
            found, vars = match(DOCSTRING_STMT_PATTERN[1], tree[1])
        else:
            found, vars = match(DOCSTRING_STMT_PATTERN, tree[3])
        if found:
            self._docstring = eval(vars['docstring'])
        # discover inner definitions
        for node in tree[1:]:
            found, vars = match(COMPOUND_STMT_PATTERN, node)
            if found:
                cstmt = vars['compound']
                if cstmt[0] == symbol.funcdef:
                    name = cstmt[2][1]
                    self._function_info[name] = FunctionInfo(cstmt)
                elif cstmt[0] == symbol.classdef:
                    name = cstmt[2][1]
                    self._class_info[name] = ClassInfo(cstmt)
\end{verbatim}

After initializing some internal state, the constructor calls the
\code{_extract_info()} method.  This method performs the bulk of the
information extraction which takes place in the entire example.  The
extraction has two distinct phases: the location of the docstring for
the parse tree passed in, and the discovery of additional definitions
within the code block represented by the parse tree.

The initial \code{if} test determines whether the nested suite is of
the ``short form'' or the ``long form.''  The short form is used when
the code block is on the same line as the definition of the code
block, as in

\begin{verbatim}
def square(x): "Square an argument."; return x ** 2
\end{verbatim}

while the long form uses an indented block and allows nested
definitions:

\begin{verbatim}
def make_power(exp):
    "Make a function that raises an argument to the exponent `exp'."
    def raiser(x, y=exp):
        return x ** y
    return raiser
\end{verbatim}

When the short form is used, the code block may contain a docstring as
the first, and possibly only, \code{small_stmt} element.  The
extraction of such a docstring is slightly different and requires only
a portion of the complete pattern used in the more common case.  As
implemented, the docstring will only be found if there is only
one \code{small_stmt} node in the \code{simple_stmt} node.  Since most
functions and methods which use the short form do not provide a
docstring, this may be considered sufficient.  The extraction of the
docstring proceeds using the \code{match()} function as described
above, and the value of the docstring is stored as an attribute of the
\code{SuiteInfoBase} object.

After docstring extraction, a simple definition discovery
algorithm operates on the \code{stmt} nodes of the \code{suite} node.  The
special case of the short form is not tested; since there are no
\code{stmt} nodes in the short form, the algorithm will silently skip
the single \code{simple_stmt} node and correctly not discover any
nested definitions.

Each statement in the code block is categorized as
a class definition, function or method definition, or
something else.  For the definition statements, the name of the
element defined is extracted and a representation object
appropriate to the definition is created with the defining subtree
passed as an argument to the constructor.  The repesentation objects
are stored in instance variables and may be retrieved by name using
the appropriate accessor methods.

The public classes provide any accessors required which are more
specific than those provided by the \code{SuiteInfoBase} class, but
the real extraction algorithm remains common to all forms of code
blocks.  A high-level function can be used to extract the complete set
of information from a source file.  (See file \file{example.py}.)

\begin{verbatim}
def get_docs(fileName):
    source = open(fileName).read()
    import os
    basename = os.path.basename(os.path.splitext(fileName)[0])
    import parser
    ast = parser.suite(source)
    tup = parser.ast2tuple(ast)
    return ModuleInfo(tup, basename)
\end{verbatim}

This provides an easy-to-use interface to the documentation of a
module.  If information is required which is not extracted by the code
of this example, the code may be extended at clearly defined points to
provide additional capabilities.


%%
%%  end of file

\section{Built-in Module \sectcode{__builtin__}}
\bimodindex{__builtin__}

This module provides direct access to all `built-in' identifiers of
Python; e.g. \code{__builtin__.open} is the full name for the built-in
function \code{open}.  See the section on Built-in Functions in the
previous chapter.
		% really __builtin__
\section{Built-in Module \sectcode{__main__}}

\bimodindex{__main__}
This module represents the (otherwise anonymous) scope in which the
interpreter's main program executes --- commands read either from
standard input or from a script file.
			% really __main__

\chapter{String Services}

The modules described in this chapter provide a wide range of string
manipulation operations.  Here's an overview:

\begin{description}

\item[string]
--- Common string operations.

\item[regex]
--- Regular expression search and match operations.

\item[regsub]
--- Substitution and splitting operations that use regular expressions.

\item[struct]
--- Interpret strings as packed binary data.

\end{description}
		% String Services
\section{Standard Module \sectcode{string}}

\stmodindex{string}

This module defines some constants useful for checking character
classes and some useful string functions.  See the modules
\code{regex} and \code{regsub} for string functions based on regular
expressions.

The constants defined in this module are are:

\renewcommand{\indexsubitem}{(data in module string)}
\begin{datadesc}{digits}
  The string \code{'0123456789'}.
\end{datadesc}

\begin{datadesc}{hexdigits}
  The string \code{'0123456789abcdefABCDEF'}.
\end{datadesc}

\begin{datadesc}{letters}
  The concatenation of the strings \code{lowercase} and
  \code{uppercase} described below.
\end{datadesc}

\begin{datadesc}{lowercase}
  A string containing all the characters that are considered lowercase
  letters.  On most systems this is the string
  \code{'abcdefghijklmnopqrstuvwxyz'}.  Do not change its definition ---
  the effect on the routines \code{upper} and \code{swapcase} is
  undefined.
\end{datadesc}

\begin{datadesc}{octdigits}
  The string \code{'01234567'}.
\end{datadesc}

\begin{datadesc}{uppercase}
  A string containing all the characters that are considered uppercase
  letters.  On most systems this is the string
  \code{'ABCDEFGHIJKLMNOPQRSTUVWXYZ'}.  Do not change its definition ---
  the effect on the routines \code{lower} and \code{swapcase} is
  undefined.
\end{datadesc}

\begin{datadesc}{whitespace}
  A string containing all characters that are considered whitespace.
  On most systems this includes the characters space, tab, linefeed,
  return, formfeed, and vertical tab.  Do not change its definition ---
  the effect on the routines \code{strip} and \code{split} is
  undefined.
\end{datadesc}

The functions defined in this module are:

\renewcommand{\indexsubitem}{(in module string)}

\begin{funcdesc}{atof}{s}
Convert a string to a floating point number.  The string must have
the standard syntax for a floating point literal in Python, optionally
preceded by a sign (\samp{+} or \samp{-}).
\end{funcdesc}

\begin{funcdesc}{atoi}{s\optional{\, base}}
Convert string \var{s} to an integer in the given \var{base}.  The
string must consist of one or more digits, optionally preceded by a
sign (\samp{+} or \samp{-}).  The \var{base} defaults to 10.  If it is
0, a default base is chosen depending on the leading characters of the
string (after stripping the sign): \samp{0x} or \samp{0X} means 16,
\samp{0} means 8, anything else means 10.  If \var{base} is 16, a
leading \samp{0x} or \samp{0X} is always accepted.  (Note: for a more
flexible interpretation of numeric literals, use the built-in function
\code{eval()}.)
\bifuncindex{eval}
\end{funcdesc}

\begin{funcdesc}{atol}{s\optional{\, base}}
Convert string \var{s} to a long integer in the given \var{base}.  The
string must consist of one or more digits, optionally preceded by a
sign (\samp{+} or \samp{-}).  The \var{base} argument has the same
meaning as for \code{atoi()}.  A trailing \samp{l} or \samp{L} is not
allowed, except if the base is 0.
\end{funcdesc}

\begin{funcdesc}{capitalize}{word}
Capitalize the first character of the argument.
\end{funcdesc}

\begin{funcdesc}{capwords}{s}
Split the argument into words using \code{split}, capitalize each word
using \code{capitalize}, and join the capitalized words using
\code{join}.  Note that this replaces runs of whitespace characters by
a single space.  (See also \code{regsub.capwords()} for a version
that doesn't change the delimiters, and lets you specify a word
separator.)
\end{funcdesc}

\begin{funcdesc}{expandtabs}{s\, tabsize}
Expand tabs in a string, i.e.\ replace them by one or more spaces,
depending on the current column and the given tab size.  The column
number is reset to zero after each newline occurring in the string.
This doesn't understand other non-printing characters or escape
sequences.
\end{funcdesc}

\begin{funcdesc}{find}{s\, sub\optional{\, start}}
Return the lowest index in \var{s} not smaller than \var{start} where the
substring \var{sub} is found.  Return \code{-1} when \var{sub}
does not occur as a substring of \var{s} with index at least \var{start}.
If \var{start} is omitted, it defaults to \code{0}.  If \var{start} is
negative, \code{len(\var{s})} is added.
\end{funcdesc}

\begin{funcdesc}{rfind}{s\, sub\optional{\, start}}
Like \code{find} but find the highest index.
\end{funcdesc}

\begin{funcdesc}{index}{s\, sub\optional{\, start}}
Like \code{find} but raise \code{ValueError} when the substring is
not found.
\end{funcdesc}

\begin{funcdesc}{rindex}{s\, sub\optional{\, start}}
Like \code{rfind} but raise \code{ValueError} when the substring is
not found.
\end{funcdesc}

\begin{funcdesc}{count}{s\, sub\optional{\, start}}
Return the number of (non-overlapping) occurrences of substring
\var{sub} in string \var{s} with index at least \var{start}.
If \var{start} is omitted, it defaults to \code{0}.  If \var{start} is
negative, \code{len(\var{s})} is added.
\end{funcdesc}

\begin{funcdesc}{lower}{s}
Convert letters to lower case.
\end{funcdesc}

\begin{funcdesc}{maketrans}{from, to}
Return a translation table suitable for passing to \code{string.translate}
or \code{regex.compile}, that will map each character in \var{from} 
into the character at the same position in \var{to}; \var{from} and
\var{to} must have the same length. 
\end{funcdesc}

\begin{funcdesc}{split}{s\optional{\, sep\optional{\, maxsplit}}}
Return a list of the words of the string \var{s}.  If the optional
second argument \var{sep} is absent or \code{None}, the words are
separated by arbitrary strings of whitespace characters (space, tab,
newline, return, formfeed).  If the second argument \var{sep} is
present and not \code{None}, it specifies a string to be used as the
word separator.  The returned list will then have one more items than
the number of non-overlapping occurrences of the separator in the
string.  The optional third argument \var{maxsplit} defaults to 0.  If
it is nonzero, at most \var{maxsplit} number of splits occur, and the
remainder of the string is returned as the final element of the list
(thus, the list will have at most \code{\var{maxsplit}+1} elements).
(See also \code{regsub.split()} for a version that allows specifying a
regular expression as the separator.)
\end{funcdesc}

\begin{funcdesc}{splitfields}{s\optional{\, sep\optional{\, maxsplit}}}
This function behaves identical to \code{split}.  (In the past,
\code{split} was only used with one argument, while \code{splitfields}
was only used with two arguments.)
\end{funcdesc}

\begin{funcdesc}{join}{words\optional{\, sep}}
Concatenate a list or tuple of words with intervening occurrences of
\var{sep}.  The default value for \var{sep} is a single space character.
It is always true that
\code{string.join(string.split(\var{s}, \var{sep}), \var{sep})}
equals \var{s}.
\end{funcdesc}

\begin{funcdesc}{joinfields}{words\optional{\, sep}}
This function behaves identical to \code{join}.  (In the past,
\code{join} was only used with one argument, while \code{joinfields}
was only used with two arguments.)
\end{funcdesc}

\begin{funcdesc}{lstrip}{s}
Remove leading whitespace from the string \var{s}.
\end{funcdesc}

\begin{funcdesc}{rstrip}{s}
Remove trailing whitespace from the string \var{s}.
\end{funcdesc}

\begin{funcdesc}{strip}{s}
Remove leading and trailing whitespace from the string \var{s}.
\end{funcdesc}

\begin{funcdesc}{swapcase}{s}
Convert lower case letters to upper case and vice versa.
\end{funcdesc}

\begin{funcdesc}{translate}{s, table\optional{, deletechars}}
Delete all characters from \var{s} that are in \var{deletechars} (if present), and 
then translate the characters using \var{table}, which must be
a 256-character string giving the translation for each character
value, indexed by its ordinal.  
\end{funcdesc}

\begin{funcdesc}{upper}{s}
Convert letters to upper case.
\end{funcdesc}

\begin{funcdesc}{ljust}{s\, width}
\funcline{rjust}{s\, width}
\funcline{center}{s\, width}
These functions respectively left-justify, right-justify and center a
string in a field of given width.
They return a string that is at least
\var{width}
characters wide, created by padding the string
\var{s}
with spaces until the given width on the right, left or both sides.
The string is never truncated.
\end{funcdesc}

\begin{funcdesc}{zfill}{s\, width}
Pad a numeric string on the left with zero digits until the given
width is reached.  Strings starting with a sign are handled correctly.
\end{funcdesc}

This module is implemented in Python.  Much of its functionality has
been reimplemented in the built-in module \code{strop}.  However, you
should \emph{never} import the latter module directly.  When
\code{string} discovers that \code{strop} exists, it transparently
replaces parts of itself with the implementation from \code{strop}.
After initialization, there is \emph{no} overhead in using
\code{string} instead of \code{strop}.
\bimodindex{strop}

\section{Built-in Module \sectcode{regex}}

\bimodindex{regex}
This module provides regular expression matching operations similar to
those found in Emacs.  It is always available.

By default the patterns are Emacs-style regular expressions
(with one exception).  There is
a way to change the syntax to match that of several well-known
\UNIX{} utilities.  The exception is that Emacs' \samp{\e s}
pattern is not supported, since the original implementation references
the Emacs syntax tables.

This module is 8-bit clean: both patterns and strings may contain null
bytes and characters whose high bit is set.

\strong{Please note:} There is a little-known fact about Python string
literals which means that you don't usually have to worry about
doubling backslashes, even though they are used to escape special
characters in string literals as well as in regular expressions.  This
is because Python doesn't remove backslashes from string literals if
they are followed by an unrecognized escape character.
\emph{However}, if you want to include a literal \dfn{backslash} in a
regular expression represented as a string literal, you have to
\emph{quadruple} it.  E.g.\  to extract \LaTeX\ \samp{\e section\{{\rm
\ldots}\}} headers from a document, you can use this pattern:
\code{'\e \e \e \e section\{\e (.*\e )\}'}.  \emph{Another exception:}
the escape sequece \samp{\e b} is significant in string literals
(where it means the ASCII bell character) as well as in Emacs regular
expressions (where it stands for a word boundary), so in order to
search for a word boundary, you should use the pattern \code{'\e \e b'}.
Similarly, a backslash followed by a digit 0-7 should be doubled to
avoid interpretation as an octal escape.

\subsection{Regular Expressions}

A regular expression (or RE) specifies a set of strings that matches
it; the functions in this module let you check if a particular string
matches a given regular expression (or if a given regular expression
matches a particular string, which comes down to the same thing).

Regular expressions can be concatenated to form new regular
expressions; if \emph{A} and \emph{B} are both regular expressions,
then \emph{AB} is also an regular expression.  If a string \emph{p}
matches A and another string \emph{q} matches B, the string \emph{pq}
will match AB.  Thus, complex expressions can easily be constructed
from simpler ones like the primitives described here.  For details of
the theory and implementation of regular expressions, consult almost
any textbook about compiler construction.

% XXX The reference could be made more specific, say to 
% "Compilers: Principles, Techniques and Tools", by Alfred V. Aho, 
% Ravi Sethi, and Jeffrey D. Ullman, or some FA text.   

A brief explanation of the format of regular expressions follows.

Regular expressions can contain both special and ordinary characters.
Ordinary characters, like '\code{A}', '\code{a}', or '\code{0}', are
the simplest regular expressions; they simply match themselves.  You
can concatenate ordinary characters, so '\code{last}' matches the
characters 'last'.  (In the rest of this section, we'll write RE's in
\code{this special font}, usually without quotes, and strings to be
matched 'in single quotes'.)

Special characters either stand for classes of ordinary characters, or
affect how the regular expressions around them are interpreted.

The special characters are:
\begin{itemize}
\item[\code{.}]{(Dot.)  Matches any character except a newline.}
\item[\code{\^}]{(Caret.)  Matches the start of the string.}
\item[\code{\$}]{Matches the end of the string.  
\code{foo} matches both 'foo' and 'foobar', while the regular
expression '\code{foo\$}' matches only 'foo'.}
\item[\code{*}] Causes the resulting RE to
match 0 or more repetitions of the preceding RE.  \code{ab*} will
match 'a', 'ab', or 'a' followed by any number of 'b's.
\item[\code{+}] Causes the
resulting RE to match 1 or more repetitions of the preceding RE.
\code{ab+} will match 'a' followed by any non-zero number of 'b's; it
will not match just 'a'.
\item[\code{?}] Causes the resulting RE to
match 0 or 1 repetitions of the preceding RE.  \code{ab?} will
match either 'a' or 'ab'.

\item[\code{\e}] Either escapes special characters (permitting you to match
characters like '*?+\&\$'), or signals a special sequence; special
sequences are discussed below.  Remember that Python also uses the
backslash as an escape sequence in string literals; if the escape
sequence isn't recognized by Python's parser, the backslash and
subsequent character are included in the resulting string.  However,
if Python would recognize the resulting sequence, the backslash should
be repeated twice.  

\item[\code{[]}] Used to indicate a set of characters.  Characters can
be listed individually, or a range is indicated by giving two
characters and separating them by a '-'.  Special characters are
not active inside sets.  For example, \code{[akm\$]}
will match any of the characters 'a', 'k', 'm', or '\$'; \code{[a-z]} will
match any lowercase letter.  

If you want to include a \code{]} inside a
set, it must be the first character of the set; to include a \code{-},
place it as the first or last character. 

Characters \emph{not} within a range can be matched by including a
\code{\^} as the first character of the set; \code{\^} elsewhere will
simply match the '\code{\^}' character.  
\end{itemize}

The special sequences consist of '\code{\e}' and a character
from the list below.  If the ordinary character is not on the list,
then the resulting RE will match the second character.  For example,
\code{\e\$} matches the character '\$'.  Ones where the backslash
should be doubled are indicated.

\begin{itemize}
\item[\code{\e|}]\code{A\e|B}, where A and B can be arbitrary REs,
creates a regular expression that will match either A or B.  This can
be used inside groups (see below) as well.
%
\item[\code{\e( \e)}]{Indicates the start and end of a group; the
contents of a group can be matched later in the string with the
\code{\e \[1-9]} special sequence, described next.}
%
{\fulllineitems\item[\code{\e \e 1, ... \e \e 7, \e 8, \e 9}]
{Matches the contents of the group of the same
number.  For example, \code{\e (.+\e ) \e \e 1} matches 'the the' or
'55 55', but not 'the end' (note the space after the group).  This
special sequence can only be used to match one of the first 9 groups;
groups with higher numbers can be matched using the \code{\e v}
sequence.  (\code{\e 8} and \code{\e 9} don't need a double backslash
because they are not octal digits.)}}
%
\item[\code{\e \e b}]{Matches the empty string, but only at the
beginning or end of a word.  A word is defined as a sequence of
alphanumeric characters, so the end of a word is indicated by
whitespace or a non-alphanumeric character.}
%
\item[\code{\e B}]{Matches the empty string, but when it is \emph{not} at the
beginning or end of a word.} 
%
\item[\code{\e v}]{Must be followed by a two digit decimal number, and
matches the contents of the group of the same number.  The group number must be between 1 and 99, inclusive.}
%
\item[\code{\e w}]Matches any alphanumeric character; this is
equivalent to the set \code{[a-zA-Z0-9]}.
%
\item[\code{\e W}]{Matches any non-alphanumeric character; this is
equivalent to the set \code{[\^a-zA-Z0-9]}.} 
\item[\code{\e <}]{Matches the empty string, but only at the beginning of a
word.  A word is defined as a sequence of alphanumeric characters, so
the end of a word is indicated by whitespace or a non-alphanumeric 
character.}
\item[\code{\e >}]{Matches the empty string, but only at the end of a
word.}

\item[\code{\e \e \e \e}]{Matches a literal backslash.}

% In Emacs, the following two are start of buffer/end of buffer.  In
% Python they seem to be synonyms for ^$.
\item[\code{\e `}]{Like \code{\^}, this only matches at the start of the
string.}
\item[\code{\e \e '}] Like \code{\$}, this only matches at the end of the
string.
% end of buffer
\end{itemize}

\subsection{Module Contents}

The module defines these functions, and an exception:

\renewcommand{\indexsubitem}{(in module regex)}

\begin{funcdesc}{match}{pattern\, string}
  Return how many characters at the beginning of \var{string} match
  the regular expression \var{pattern}.  Return \code{-1} if the
  string does not match the pattern (this is different from a
  zero-length match!).
\end{funcdesc}

\begin{funcdesc}{search}{pattern\, string}
  Return the first position in \var{string} that matches the regular
  expression \var{pattern}.  Return \code{-1} if no position in the string
  matches the pattern (this is different from a zero-length match
  anywhere!).
\end{funcdesc}

\begin{funcdesc}{compile}{pattern\optional{\, translate}}
  Compile a regular expression pattern into a regular expression
  object, which can be used for matching using its \code{match} and
  \code{search} methods, described below.  The optional argument
  \var{translate}, if present, must be a 256-character string
  indicating how characters (both of the pattern and of the strings to
  be matched) are translated before comparing them; the \code{i}-th
  element of the string gives the translation for the character with
  \ASCII{} code \code{i}.  This can be used to implement
  case-insensitive matching; see the \code{casefold} data item below.

  The sequence

\bcode\begin{verbatim}
prog = regex.compile(pat)
result = prog.match(str)
\end{verbatim}\ecode

is equivalent to

\bcode\begin{verbatim}
result = regex.match(pat, str)
\end{verbatim}\ecode

but the version using \code{compile()} is more efficient when multiple
regular expressions are used concurrently in a single program.  (The
compiled version of the last pattern passed to \code{regex.match()} or
\code{regex.search()} is cached, so programs that use only a single
regular expression at a time needn't worry about compiling regular
expressions.)
\end{funcdesc}

\begin{funcdesc}{set_syntax}{flags}
  Set the syntax to be used by future calls to \code{compile},
  \code{match} and \code{search}.  (Already compiled expression objects
  are not affected.)  The argument is an integer which is the OR of
  several flag bits.  The return value is the previous value of
  the syntax flags.  Names for the flags are defined in the standard
  module \code{regex_syntax}; read the file \file{regex_syntax.py} for
  more information.
\end{funcdesc}

\begin{funcdesc}{symcomp}{pattern\optional{\, translate}}
This is like \code{compile}, but supports symbolic group names: if a
parenthesis-enclosed group begins with a group name in angular
brackets, e.g. \code{'\e(<id>[a-z][a-z0-9]*\e)'}, the group can
be referenced by its name in arguments to the \code{group} method of
the resulting compiled regular expression object, like this:
\code{p.group('id')}.  Group names may contain alphanumeric characters
and \code{'_'} only.
\end{funcdesc}

\begin{excdesc}{error}
  Exception raised when a string passed to one of the functions here
  is not a valid regular expression (e.g., unmatched parentheses) or
  when some other error occurs during compilation or matching.  (It is
  never an error if a string contains no match for a pattern.)
\end{excdesc}

\begin{datadesc}{casefold}
A string suitable to pass as \var{translate} argument to
\code{compile} to map all upper case characters to their lowercase
equivalents.
\end{datadesc}

\noindent
Compiled regular expression objects support these methods:

\renewcommand{\indexsubitem}{(regex method)}
\begin{funcdesc}{match}{string\optional{\, pos}}
  Return how many characters at the beginning of \var{string} match
  the compiled regular expression.  Return \code{-1} if the string
  does not match the pattern (this is different from a zero-length
  match!).
  
  The optional second parameter \var{pos} gives an index in the string
  where the search is to start; it defaults to \code{0}.  This is not
  completely equivalent to slicing the string; the \code{'\^'} pattern
  character matches at the real begin of the string and at positions
  just after a newline, not necessarily at the index where the search
  is to start.
\end{funcdesc}

\begin{funcdesc}{search}{string\optional{\, pos}}
  Return the first position in \var{string} that matches the regular
  expression \code{pattern}.  Return \code{-1} if no position in the
  string matches the pattern (this is different from a zero-length
  match anywhere!).
  
  The optional second parameter has the same meaning as for the
  \code{match} method.
\end{funcdesc}

\begin{funcdesc}{group}{index\, index\, ...}
This method is only valid when the last call to the \code{match}
or \code{search} method found a match.  It returns one or more
groups of the match.  If there is a single \var{index} argument,
the result is a single string; if there are multiple arguments, the
result is a tuple with one item per argument.  If the \var{index} is
zero, the corresponding return value is the entire matching string; if
it is in the inclusive range [1..99], it is the string matching the
the corresponding parenthesized group (using the default syntax,
groups are parenthesized using \code{\\(} and \code{\\)}).  If no
such group exists, the corresponding result is \code{None}.

If the regular expression was compiled by \code{symcomp} instead of
\code{compile}, the \var{index} arguments may also be strings
identifying groups by their group name.
\end{funcdesc}

\noindent
Compiled regular expressions support these data attributes:

\renewcommand{\indexsubitem}{(regex attribute)}

\begin{datadesc}{regs}
When the last call to the \code{match} or \code{search} method found a
match, this is a tuple of pairs of indices corresponding to the
beginning and end of all parenthesized groups in the pattern.  Indices
are relative to the string argument passed to \code{match} or
\code{search}.  The 0-th tuple gives the beginning and end or the
whole pattern.  When the last match or search failed, this is
\code{None}.
\end{datadesc}

\begin{datadesc}{last}
When the last call to the \code{match} or \code{search} method found a
match, this is the string argument passed to that method.  When the
last match or search failed, this is \code{None}.
\end{datadesc}

\begin{datadesc}{translate}
This is the value of the \var{translate} argument to
\code{regex.compile} that created this regular expression object.  If
the \var{translate} argument was omitted in the \code{regex.compile}
call, this is \code{None}.
\end{datadesc}

\begin{datadesc}{givenpat}
The regular expression pattern as passed to \code{compile} or
\code{symcomp}.
\end{datadesc}

\begin{datadesc}{realpat}
The regular expression after stripping the group names for regular
expressions compiled with \code{symcomp}.  Same as \code{givenpat}
otherwise.
\end{datadesc}

\begin{datadesc}{groupindex}
A dictionary giving the mapping from symbolic group names to numerical
group indices for regular expressions compiled with \code{symcomp}.
\code{None} otherwise.
\end{datadesc}

\section{Standard Module \sectcode{regsub}}

\stmodindex{regsub}
This module defines a number of functions useful for working with
regular expressions (see built-in module \code{regex}).

Warning: these functions are not thread-safe.

\renewcommand{\indexsubitem}{(in module regsub)}

\begin{funcdesc}{sub}{pat\, repl\, str}
Replace the first occurrence of pattern \var{pat} in string
\var{str} by replacement \var{repl}.  If the pattern isn't found,
the string is returned unchanged.  The pattern may be a string or an
already compiled pattern.  The replacement may contain references
\samp{\e \var{digit}} to subpatterns and escaped backslashes.
\end{funcdesc}

\begin{funcdesc}{gsub}{pat\, repl\, str}
Replace all (non-overlapping) occurrences of pattern \var{pat} in
string \var{str} by replacement \var{repl}.  The same rules as for
\code{sub()} apply.  Empty matches for the pattern are replaced only
when not adjacent to a previous match, so e.g.
\code{gsub('', '-', 'abc')} returns \code{'-a-b-c-'}.
\end{funcdesc}

\begin{funcdesc}{split}{str\, pat\optional{\, maxsplit}}
Split the string \var{str} in fields separated by delimiters matching
the pattern \var{pat}, and return a list containing the fields.  Only
non-empty matches for the pattern are considered, so e.g.
\code{split('a:b', ':*')} returns \code{['a', 'b']} and
\code{split('abc', '')} returns \code{['abc']}.  The \var{maxsplit}
defaults to 0. If it is nonzero, only \var{maxsplit} number of splits
occur, and the remainder of the string is returned as the final
element of the list.
\end{funcdesc}

\begin{funcdesc}{splitx}{str\, pat\optional{\, maxsplit}}
Split the string \var{str} in fields separated by delimiters matching
the pattern \var{pat}, and return a list containing the fields as well
as the separators.  For example, \code{splitx('a:::b', ':*')} returns
\code{['a', ':::', 'b']}.  Otherwise, this function behaves the same
as \code{split}.
\end{funcdesc}

\begin{funcdesc}{capwords}{s\optional{\, pat}}
Capitalize words separated by optional pattern \var{pat}.  The default
pattern uses any characters except letters, digits and underscores as
word delimiters.  Capitalization is done by changing the first
character of each word to upper case.
\end{funcdesc}

\section{Built-in Module \sectcode{struct}}
\bimodindex{struct}
\indexii{C}{structures}

This module performs conversions between Python values and C
structs represented as Python strings.  It uses \dfn{format strings}
(explained below) as compact descriptions of the lay-out of the C
structs and the intended conversion to/from Python values.

See also built-in module \code{array}.
\bimodindex{array}

The module defines the following exception and functions:

\renewcommand{\indexsubitem}{(in module struct)}
\begin{excdesc}{error}
  Exception raised on various occasions; argument is a string
  describing what is wrong.
\end{excdesc}

\begin{funcdesc}{pack}{fmt\, v1\, v2\, {\rm \ldots}}
  Return a string containing the values
  \code{\var{v1}, \var{v2}, {\rm \ldots}} packed according to the given
  format.  The arguments must match the values required by the format
  exactly.
\end{funcdesc}

\begin{funcdesc}{unpack}{fmt\, string}
  Unpack the string (presumably packed by \code{pack(\var{fmt}, {\rm \ldots})})
  according to the given format.  The result is a tuple even if it
  contains exactly one item.  The string must contain exactly the
  amount of data required by the format (i.e.  \code{len(\var{string})} must
  equal \code{calcsize(\var{fmt})}).
\end{funcdesc}

\begin{funcdesc}{calcsize}{fmt}
  Return the size of the struct (and hence of the string)
  corresponding to the given format.
\end{funcdesc}

Format characters have the following meaning; the conversion between C
and Python values should be obvious given their types:

\begin{tableiii}{|c|l|l|}{samp}{Format}{C}{Python}
  \lineiii{x}{pad byte}{no value}
  \lineiii{c}{char}{string of length 1}
  \lineiii{b}{signed char}{integer}
  \lineiii{h}{short}{integer}
  \lineiii{i}{int}{integer}
  \lineiii{l}{long}{integer}
  \lineiii{f}{float}{float}
  \lineiii{d}{double}{float}
\end{tableiii}

A format character may be preceded by an integral repeat count; e.g.\
the format string \code{'4h'} means exactly the same as \code{'hhhh'}.

C numbers are represented in the machine's native format and byte
order, and properly aligned by skipping pad bytes if necessary
(according to the rules used by the C compiler).

Examples (all on a big-endian machine):

\bcode\begin{verbatim}
pack('hhl', 1, 2, 3) == '\000\001\000\002\000\000\000\003'
unpack('hhl', '\000\001\000\002\000\000\000\003') == (1, 2, 3)
calcsize('hhl') == 8
\end{verbatim}\ecode

Hint: to align the end of a structure to the alignment requirement of
a particular type, end the format with the code for that type with a
repeat count of zero, e.g.\ the format \code{'llh0l'} specifies two
pad bytes at the end, assuming longs are aligned on 4-byte boundaries.

(More format characters are planned, e.g.\ \code{'s'} for character
arrays, upper case for unsigned variants, and a way to specify the
byte order, which is useful for [de]constructing network packets and
reading/writing portable binary file formats like TIFF and AIFF.)


\chapter{Miscellaneous Services}

The modules described in this chapter provide miscellaneous services
that are available in all Python versions.  Here's an overview:

\begin{description}

\item[math]
--- Mathematical functions (\code{sin()} etc.).

\item[rand]
--- Integer random number generator.

\item[whrandom]
--- Floating point random number generator.

\item[array]
--- Efficient arrays of uniformly typed numeric values.

\end{description}
			% Miscellaneous Services
\section{Built-in Module \sectcode{math}}

\bimodindex{math}
\renewcommand{\indexsubitem}{(in module math)}
This module is always available.
It provides access to the mathematical functions defined by the C
standard.
They are:
\iftexi
\begin{funcdesc}{acos}{x}
\funcline{asin}{x}
\funcline{atan}{x}
\funcline{atan2}{x, y}
\funcline{ceil}{x}
\funcline{cos}{x}
\funcline{cosh}{x}
\funcline{exp}{x}
\funcline{fabs}{x}
\funcline{floor}{x}
\funcline{fmod}{x, y}
\funcline{frexp}{x}
\funcline{hypot}{x, y}
\funcline{ldexp}{x, y}
\funcline{log}{x}
\funcline{log10}{x}
\funcline{modf}{x}
\funcline{pow}{x, y}
\funcline{sin}{x}
\funcline{sinh}{x}
\funcline{sqrt}{x}
\funcline{tan}{x}
\funcline{tanh}{x}
\end{funcdesc}
\else
\code{acos(\varvars{x})},
\code{asin(\varvars{x})},
\code{atan(\varvars{x})},
\code{atan2(\varvars{x\, y})},
\code{ceil(\varvars{x})},
\code{cos(\varvars{x})},
\code{cosh(\varvars{x})},
\code{exp(\varvars{x})},
\code{fabs(\varvars{x})},
\code{floor(\varvars{x})},
\code{fmod(\varvars{x\, y})},
\code{frexp(\varvars{x})},
\code{hypot(\varvars{x\, y})},
\code{ldexp(\varvars{x\, y})},
\code{log(\varvars{x})},
\code{log10(\varvars{x})},
\code{modf(\varvars{x})},
\code{pow(\varvars{x\, y})},
\code{sin(\varvars{x})},
\code{sinh(\varvars{x})},
\code{sqrt(\varvars{x})},
\code{tan(\varvars{x})},
\code{tanh(\varvars{x})}.
\fi

Note that \code{frexp} and \code{modf} have a different call/return
pattern than their C equivalents: they take a single argument and
return a pair of values, rather than returning their second return
value through an `output parameter' (there is no such thing in Python).

The module also defines two mathematical constants:
\iftexi
\begin{datadesc}{pi}
\dataline{e}
\end{datadesc}
\else
\code{pi} and \code{e}.
\fi

\section{Standard Module \sectcode{rand}}

\stmodindex{rand} This module implements a pseudo-random number
generator with an interface similar to \code{rand()} in C\@.  It defines
the following functions:

\renewcommand{\indexsubitem}{(in module rand)}
\begin{funcdesc}{rand}{}
Returns an integer random number in the range [0 ... 32768).
\end{funcdesc}

\begin{funcdesc}{choice}{s}
Returns a random element from the sequence (string, tuple or list)
\var{s}.
\end{funcdesc}

\begin{funcdesc}{srand}{seed}
Initializes the random number generator with the given integral seed.
When the module is first imported, the random number is initialized with
the current time.
\end{funcdesc}

\section{Standard Module \sectcode{whrandom}}

\stmodindex{whrandom}
This module implements a Wichmann-Hill pseudo-random number generator.
It defines the following functions:

\renewcommand{\indexsubitem}{(in module whrandom)}
\begin{funcdesc}{random}{}
Returns the next random floating point number in the range [0.0 ... 1.0).
\end{funcdesc}

\begin{funcdesc}{seed}{x\, y\, z}
Initializes the random number generator from the integers
\var{x},
\var{y}
and
\var{z}.
When the module is first imported, the random number is initialized
using values derived from the current time.
\end{funcdesc}

\section{Built-in Module \sectcode{array}}
\bimodindex{array}
\index{arrays}

This module defines a new object type which can efficiently represent
an array of basic values: characters, integers, floating point
numbers.  Arrays are sequence types and behave very much like lists,
except that the type of objects stored in them is constrained.  The
type is specified at object creation time by using a \dfn{type code},
which is a single character.  The following type codes are defined:

\begin{tableiii}{|c|c|c|}{code}{Typecode}{Type}{Minimal size in bytes}
\lineiii{'c'}{character}{1}
\lineiii{'b'}{signed integer}{1}
\lineiii{'h'}{signed integer}{2}
\lineiii{'i'}{signed integer}{2}
\lineiii{'l'}{signed integer}{4}
\lineiii{'f'}{floating point}{4}
\lineiii{'d'}{floating point}{8}
\end{tableiii}

The actual representation of values is determined by the machine
architecture (strictly speaking, by the C implementation).  The actual
size can be accessed through the \var{itemsize} attribute.

See also built-in module \code{struct}.
\bimodindex{struct}

The module defines the following function:

\renewcommand{\indexsubitem}{(in module array)}

\begin{funcdesc}{array}{typecode\optional{\, initializer}}
Return a new array whose items are restricted by \var{typecode}, and
initialized from the optional \var{initializer} value, which must be a
list or a string.  The list or string is passed to the new array's
\code{fromlist()} or \code{fromstring()} method (see below) to add
initial items to the array.
\end{funcdesc}

Array objects support the following data items and methods:

\begin{datadesc}{typecode}
The typecode character used to create the array.
\end{datadesc}

\begin{datadesc}{itemsize}
The length in bytes of one array item in the internal representation.
\end{datadesc}

\begin{funcdesc}{append}{x}
Append a new item with value \var{x} to the end of the array.
\end{funcdesc}

\begin{funcdesc}{byteswap}{x}
``Byteswap'' all items of the array.  This is only supported for
integer values.  It is useful when reading data from a file written
on a machine with a different byte order.
\end{funcdesc}

\begin{funcdesc}{fromfile}{f\, n}
Read \var{n} items (as machine values) from the file object \var{f}
and append them to the end of the array.  If less than \var{n} items
are available, \code{EOFError} is raised, but the items that were
available are still inserted into the array.  \var{f} must be a real
built-in file object; something else with a \code{read()} method won't
do.
\end{funcdesc}

\begin{funcdesc}{fromlist}{list}
Append items from the list.  This is equivalent to
\code{for x in \var{list}:\ a.append(x)}
except that if there is a type error, the array is unchanged.
\end{funcdesc}

\begin{funcdesc}{fromstring}{s}
Appends items from the string, interpreting the string as an
array of machine values (i.e. as if it had been read from a
file using the \code{fromfile()} method).
\end{funcdesc}

\begin{funcdesc}{insert}{i\, x}
Insert a new item with value \var{x} in the array before position
\var{i}.
\end{funcdesc}

\begin{funcdesc}{tofile}{f}
Write all items (as machine values) to the file object \var{f}.
\end{funcdesc}

\begin{funcdesc}{tolist}{}
Convert the array to an ordinary list with the same items.
\end{funcdesc}

\begin{funcdesc}{tostring}{}
Convert the array to an array of machine values and return the
string representation (the same sequence of bytes that would
be written to a file by the \code{tofile()} method.)
\end{funcdesc}

When an array object is printed or converted to a string, it is
represented as \code{array(\var{typecode}, \var{initializer})}.  The
\var{initializer} is omitted if the array is empty, otherwise it is a
string if the \var{typecode} is \code{'c'}, otherwise it is a list of
numbers.  The string is guaranteed to be able to be converted back to
an array with the same type and value using reverse quotes
(\code{``}).  Examples:

\bcode\begin{verbatim}
array('l')
array('c', 'hello world')
array('l', [1, 2, 3, 4, 5])
array('d', [1.0, 2.0, 3.14])
\end{verbatim}\ecode


\chapter{Generic Operating System Services}

The modules described in this chapter provide interfaces to operating
system features that are available on (almost) all operating systems,
such as files and a clock.  The interfaces are generally modelled
after the \UNIX{} or C interfaces but they are available on most other
systems as well.  Here's an overview:

\begin{description}

\item[os]
--- Miscellaneous OS interfaces.

\item[time]
--- Time access and conversions.

\item[getopt]
--- Parser for command line options.

\item[tempfile]
--- Generate temporary file names.

\end{description}
		% Generic Operating System Services
\section{Standard Module \sectcode{os}}

\stmodindex{os}
This module provides a more portable way of using operating system
(OS) dependent functionality than importing an OS dependent built-in
module like \code{posix}.

When the optional built-in module \code{posix} is available, this
module exports the same functions and data as \code{posix}; otherwise,
it searches for an OS dependent built-in module like \code{mac} and
exports the same functions and data as found there.  The design of all
Python's built-in OS dependent modules is such that as long as the same
functionality is available, it uses the same interface; e.g., the
function \code{os.stat(\var{file})} returns stat info about a \var{file} in a
format compatible with the POSIX interface.

Extensions peculiar to a particular OS are also available through the
\code{os} module, but using them is of course a threat to portability!

Note that after the first time \code{os} is imported, there is \emph{no}
performance penalty in using functions from \code{os} instead of
directly from the OS dependent built-in module, so there should be
\emph{no} reason not to use \code{os}!

In addition to whatever the correct OS dependent module exports, the
following variables and functions are always exported by \code{os}:

\renewcommand{\indexsubitem}{(in module os)}

\begin{datadesc}{name}
The name of the OS dependent module imported.  The following names
have currently been registered: \code{'posix'}, \code{'nt'},
\code{'dos'}, \code{'mac'}.
\end{datadesc}

\begin{datadesc}{path}
The corresponding OS dependent standard module for pathname
operations, e.g., \code{posixpath} or \code{macpath}.  Thus, (given
the proper imports), \code{os.path.split(\var{file})} is equivalent to but
more portable than \code{posixpath.split(\var{file})}.
\end{datadesc}

\begin{datadesc}{curdir}
The constant string used by the OS to refer to the current directory,
e.g. \code{'.'} for POSIX or \code{':'} for the Mac.
\end{datadesc}

\begin{datadesc}{pardir}
The constant string used by the OS to refer to the parent directory,
e.g. \code{'..'} for POSIX or \code{'::'} for the Mac.
\end{datadesc}

\begin{datadesc}{sep}
The character used by the OS to separate pathname components, e.g.\
\code{'/'} for POSIX or \code{':'} for the Mac.  Note that knowing this
is not sufficient to be able to parse or concatenate pathnames---better
use \code{os.path.split()} and \code{os.path.join()}---but it is
occasionally useful.
\end{datadesc}

\begin{datadesc}{pathsep}
The character conventionally used by the OS to separate search patch
components (as in \code{\$PATH}), e.g.\ \code{':'} for POSIX or
\code{';'} for MS-DOS.
\end{datadesc}

\begin{datadesc}{defpath}
The default search path used by \code{os.exec*p*()} if the environment
doesn't have a \code{'PATH'} key.
\end{datadesc}

\begin{funcdesc}{execl}{path\, arg0\, arg1\, ...}
This is equivalent to
\code{os.execv(\var{path}, (\var{arg0}, \var{arg1}, ...))}.
\end{funcdesc}

\begin{funcdesc}{execle}{path\, arg0\, arg1\, ...\, env}
This is equivalent to
\code{os.execve(\var{path}, (\var{arg0}, \var{arg1}, ...), \var{env})}.
\end{funcdesc}

\begin{funcdesc}{execlp}{path\, arg0\, arg1\, ...}
This is equivalent to
\code{os.execvp(\var{path}, (\var{arg0}, \var{arg1}, ...))}.
\end{funcdesc}

\begin{funcdesc}{execvp}{path\, args}
This is like \code{os.execv(\var{path}, \var{args})} but duplicates
the shell's actions in searching for an executable file in a list of
directories.  The directory list is obtained from
\code{os.environ['PATH']}.
\end{funcdesc}

\begin{funcdesc}{execvpe}{path\, args\, env}
This is a cross between \code{os.execve()} and \code{os.execvp()}.
The directory list is obtained from \code{\var{env}['PATH']}.
\end{funcdesc}

(The functions \code{os.execv()} and \code{execve()} are not
documented here, since they are implemented by the OS dependent
module.  If the OS dependent module doesn't define either of these,
the functions that rely on it will raise an exception.  They are
documented in the section on module \code{posix}, together with all
other functions that \code{os} imports from the OS dependent module.)

\section{Built-in Module \sectcode{time}}

\bimodindex{time}
This module provides various time-related functions.
It is always available.

An explanation of some terminology and conventions is in order.

\begin{itemize}

\item
The ``epoch'' is the point where the time starts.  On January 1st of that
year, at 0 hours, the ``time since the epoch'' is zero.  For UNIX, the
epoch is 1970.  To find out what the epoch is, look at \code{gmtime(0)}.

\item
UTC is Coordinated Universal Time (formerly known as Greenwich Mean
Time).  The acronym UTC is not a mistake but a compromise between
English and French.

\item
DST is Daylight Saving Time, an adjustment of the timezone by
(usually) one hour during part of the year.  DST rules are magic
(determined by local law) and can change from year to year.  The C
library has a table containing the local rules (often it is read from
a system file for flexibility) and is the only source of True Wisdom
in this respect.

\item
The precision of the various real-time functions may be less than
suggested by the units in which their value or argument is expressed.
E.g.\ on most UNIX systems, the clock ``ticks'' only 50 or 100 times a
second, and on the Mac, times are only accurate to whole seconds.

\item
The time tuple as returned by \code{gmtime()} and \code{localtime()},
or as accpted by \code{mktime()} is a tuple of 9
integers: year (e.g.\ 1993), month (1--12), day (1--31), hour
(0--23), minute (0--59), second (0--59), weekday (0--6, monday is 0),
Julian day (1--366) and daylight savings flag (-1, 0  or 1).
Note that unlike the C structure, the month value is a range of 1-12, not
0-11.  A year value of $<$ 100 will typically be silently converted to
1900 $+$ year value.  A -1 argument as daylight savings flag, passed to
\code{mktime()} will usually result in the correct daylight savings
state to be filled in.


\end{itemize}

The module defines the following functions and data items:

\renewcommand{\indexsubitem}{(in module time)}

\begin{datadesc}{altzone}
The offset of the local DST timezone, in seconds west of the 0th
meridian, if one is defined.  Negative if the local DST timezone is
east of the 0th meridian (as in Western Europe, including the UK).
Only use this if \code{daylight} is nonzero.
\end{datadesc}

\begin{funcdesc}{asctime}{tuple}
Convert a tuple representing a time as returned by \code{gmtime()} or
\code{localtime()} to a 24-character string of the following form:
\code{'Sun Jun 20 23:21:05 1993'}.  Note: unlike the C function of
the same name, there is no trailing newline.
\end{funcdesc}

\begin{funcdesc}{clock}{}
Return the current CPU time as a floating point number expressed in
seconds.  The precision, and in fact the very definiton of the meaning
of ``CPU time'', depends on that of the C function of the same name.
\end{funcdesc}

\begin{funcdesc}{ctime}{secs}
Convert a time expressed in seconds since the epoch to a string
representing local time.  \code{ctime(t)} is equivalent to
\code{asctime(localtime(t))}.
\end{funcdesc}

\begin{datadesc}{daylight}
Nonzero if a DST timezone is defined.
\end{datadesc}

\begin{funcdesc}{gmtime}{secs}
Convert a time expressed in seconds since the epoch to a time tuple
in UTC in which the dst flag is always zero.  Fractions of a second are
ignored.
\end{funcdesc}

\begin{funcdesc}{localtime}{secs}
Like \code{gmtime} but converts to local time.  The dst flag is set
to 1 when DST applies to the given time.
\end{funcdesc}

\begin{funcdesc}{mktime}{tuple}
This is the inverse function of \code{localtime}.  Its argument is the
full 9-tuple (since the dst flag is needed --- pass -1 as the dst flag if
it is unknown) which expresses the time
in \em{local} time, not UTC.  It returns a floating
point number, for compatibility with \code{time.time()}.  If the input
value can't be represented as a valid time, OverflowError is raised.
\end{funcdesc}

\begin{funcdesc}{sleep}{secs}
Suspend execution for the given number of seconds.  The argument may
be a floating point number to indicate a more precise sleep time.
\end{funcdesc}

\begin{funcdesc}{strftime}{format, tuple}
Convert a tuple representing a time as returned by \code{gmtime()} or
\code{localtime()} to a string as specified by the format argument.

      The following directives, shown without the optional field width and
      precision specification, are replaced by the indicated characters:

\begin{tabular}{lp{25em}}
           \%a  &      Locale's abbreviated weekday name. \\
           \%A  &      Locale's full weekday name. \\
           \%b  &      Locale's abbreviated month name. \\
           \%B  &      Locale's full month name. \\
           \%c  &      Locale's appropriate date and time representation. \\
           \%d  &      Day of the month as a decimal number [01,31]. \\
           \%E  &      Locale's combined Emperor/Era name and year. \\
           \%H  &      Hour (24-hour clock) as a decimal number [00,23]. \\
           \%I  &      Hour (12-hour clock) as a decimal number [01,12]. \\
           \%j  &      Day of the year as a decimal number [001,366]. \\
           \%m  &      Month as a decimal number [01,12]. \\
           \%M  &      Minute as a decimal number [00,59]. \\
           \%n  &      New-line character. \\
           \%N  &      Locale's Emperor/Era name. \\
           \%o  &      Locale's Emperor/Era year. \\
           \%p  &      Locale's equivalent of either AM or PM. \\
           \%S  &      Second as a decimal number [00,61]. \\
           \%t  &      Tab character. \\
           \%U  &      Week number of the year (Sunday as the first day of the
                     week) as a decimal number [00,53].  All days in a new
                     year preceding the first Sunday are considered to be in
                     week 0. \\
           \%w  &      Weekday as a decimal number [0(Sunday),6]. \\
           \%W  &      Week number of the year (Monday as the first day of the
                     week) as a decimal number [00,53].  All days in a new
                     year preceding the first Sunday are considered to be in
                     week 0. \\
           \%x  &      Locale's appropriate date representation. \\
           \%X  &      Locale's appropriate time representation. \\
           \%y  &      Year without century as a decimal number [00,99]. \\
           \%Y  &      Year with century as a decimal number. \\
           \%Z  &      Time zone name (or by no characters if no time zone
                     exists). \\
           \%\%  &     \% \\
\end{tabular}

      An optional field width and precision specification can immediately
      follow the initial \% of a directive in the following order: \\

\begin{tabular}{lp{25em}}
      [-|0]w  &       the decimal digit string w specifies a minimum field
                     width in which the result of the conversion is right-
                     or left-justified.  It is right-justified (with space
                     padding) by default.  If the optional flag `-' is
                     specified, it is left-justified with space padding on
                     the right.  If the optional flag `0' is specified, it
                     is right-justified and padded with zeros on the left. \\
      .p      &       the decimal digit string p specifies the minimum number
                     of digits to appear for the d, H, I, j, m, M, o, S, U,
                     w, W, y and Y directives, and the maximum number of
                     characters to be used from the a, A, b, B, c, D, E, F,
                     h, n, N, p, r, t, T, x, X, z, Z, and % directives.  In
                     the first case, if a directive supplies fewer digits
                     than specified by the precision, it will be expanded
                     with leading zeros.  In the second case, if a directive
                     supplies more characters than specified by the
                     precision, excess characters will truncated on the
                     right.
\end{tabular}

      If no field width or precision is specified for a d, H, I, m, M, S, U,
      W, y, or j directive, a default of .2 is used for all but j for which
      .3 is used.

\end{funcdesc}

\begin{funcdesc}{time}{}
Return the time as a floating point number expressed in seconds since
the epoch, in UTC.  Note that even though the time is always returned
as a floating point number, not all systems provide time with a better
precision than 1 second.
\end{funcdesc}

\begin{datadesc}{timezone}
The offset of the local (non-DST) timezone, in seconds west of the 0th
meridian (i.e. negative in most of Western Europe, positive in the US,
zero in the UK).
\end{datadesc}

\begin{datadesc}{tzname}
A tuple of two strings: the first is the name of the local non-DST
timezone, the second is the name of the local DST timezone.  If no DST
timezone is defined, the second string should not be used.
\end{datadesc}


\section{Standard Module \sectcode{getopt}}

\stmodindex{getopt}
This module helps scripts to parse the command line arguments in
\code{sys.argv}.
It supports the same conventions as the \UNIX{}
\code{getopt()}
function (including the special meanings of arguments of the form
\samp{-} and \samp{--}).  Long options similar to those supported by
GNU software may be used as well via an optional third argument.
It defines the function
\code{getopt.getopt(args, options [, long_options])}
and the exception
\code{getopt.error}.

The first argument to
\code{getopt()}
is the argument list passed to the script with its first element
chopped off (i.e.,
\code{sys.argv[1:]}).
The second argument is the string of option letters that the
script wants to recognize, with options that require an argument
followed by a colon (i.e., the same format that \UNIX{}
\code{getopt()}
uses).
The third option, if specified, is a list of strings with the names of
the long options which should be supported.  The leading \code{'--'}
characters should not be included in the option name.  Options which
require an argument should be followed by an equal sign (\code{'='}).
The return value consists of two elements: the first is a list of
option-and-value pairs; the second is the list of program arguments
left after the option list was stripped (this is a trailing slice of the
first argument).
Each option-and-value pair returned has the option as its first element,
prefixed with a hyphen (e.g.,
\code{'-x'}),
and the option argument as its second element, or an empty string if the
option has no argument.
The options occur in the list in the same order in which they were
found, thus allowing multiple occurrences.  Long and short options may
be mixed.

An example using only \UNIX{} style options:

\bcode\begin{verbatim}
>>> import getopt, string
>>> args = string.split('-a -b -cfoo -d bar a1 a2')
>>> args
['-a', '-b', '-cfoo', '-d', 'bar', 'a1', 'a2']
>>> optlist, args = getopt.getopt(args, 'abc:d:')
>>> optlist
[('-a', ''), ('-b', ''), ('-c', 'foo'), ('-d', 'bar')]
>>> args
['a1', 'a2']
>>> 
\end{verbatim}\ecode

Using long option names is equally easy:

\bcode\begin{verbatim}
>>> s = '--condition=foo --testing --output-file abc.def -x a1 a2'
>>> args = string.split(s)
>>> args
['--condition=foo', '--testing', '--output-file', 'abc.def', '-x', 'a1', 'a2']
>>> optlist, args = getopt.getopt(args, 'x', [
...     'condition=', 'output-file=', 'testing'])
>>> optlist
[('--condition', 'foo'), ('--testing', ''), ('--output-file', 'abc.def'), ('-x', '')]
>>> args
['a1', 'a2']
>>> 
\end{verbatim}\ecode

The exception
\code{getopt.error = 'getopt.error'}
is raised when an unrecognized option is found in the argument list or
when an option requiring an argument is given none.
The argument to the exception is a string indicating the cause of the
error.  For long options, an argument given to an option which does
not require one will also cause this exception to be raised.

\section{Standard Module \sectcode{tempfile}}
\stmodindex{tempfile}
\indexii{temporary}{file name}
\indexii{temporary}{file}

\renewcommand{\indexsubitem}{(in module tempfile)}

This module generates temporary file names.  It is not \UNIX{} specific,
but it may require some help on non-\UNIX{} systems.

Note: the modules does not create temporary files, nor does it
automatically remove them when the current process exits or dies.

The module defines a single user-callable function:

\begin{funcdesc}{mktemp}{}
Return a unique temporary filename.  This is an absolute pathname of a
file that does not exist at the time the call is made.  No two calls
will return the same filename.
\end{funcdesc}

The module uses two global variables that tell it how to construct a
temporary name.  The caller may assign values to them; by default they
are initialized at the first call to \code{mktemp()}.

\begin{datadesc}{tempdir}
When set to a value other than \code{None}, this variable defines the
directory in which filenames returned by \code{mktemp()} reside.  The
default is taken from the environment variable \code{TMPDIR}; if this
is not set, either \code{/usr/tmp} is used (on \UNIX{}), or the current
working directory (all other systems).  No check is made to see
whether its value is valid.
\end{datadesc}
\ttindex{TMPDIR}

\begin{datadesc}{template}
When set to a value other than \code{None}, this variable defines the
prefix of the final component of the filenames returned by
\code{mktemp()}.  A string of decimal digits is added to generate
unique filenames.  The default is either ``\code{@\var{pid}.}'' where
\var{pid} is the current process ID (on \UNIX{}), or ``\code{tmp}'' (all
other systems).
\end{datadesc}

Warning: if a \UNIX{} process uses \code{mktemp()}, then calls
\code{fork()} and both parent and child continue to use
\code{mktemp()}, the processes will generate conflicting temporary
names.  To resolve this, the child process should assign \code{None}
to \code{template}, to force recomputing the default on the next call
to \code{mktemp()}.

\section{Standard Module \sectcode{errno}}
\stmodindex{errno}

\renewcommand{\indexsubitem}{(in module errno)}

This module makes available standard errno system symbols.
The value of each symbol is the corresponding integer value.
The names and descriptions are borrowed from linux/include/errno.h,
which should be pretty all-inclusive.  Of the following list, symbols
that are not used on the current platform are not defined by the
module.

Symbols available can include:
\begin{datadesc}{EPERM} Operation not permitted \end{datadesc}
\begin{datadesc}{ENOENT} No such file or directory \end{datadesc}
\begin{datadesc}{ESRCH} No such process \end{datadesc}
\begin{datadesc}{EINTR} Interrupted system call \end{datadesc}
\begin{datadesc}{EIO} I/O error \end{datadesc}
\begin{datadesc}{ENXIO} No such device or address \end{datadesc}
\begin{datadesc}{E2BIG} Arg list too long \end{datadesc}
\begin{datadesc}{ENOEXEC} Exec format error \end{datadesc}
\begin{datadesc}{EBADF} Bad file number \end{datadesc}
\begin{datadesc}{ECHILD} No child processes \end{datadesc}
\begin{datadesc}{EAGAIN} Try again \end{datadesc}
\begin{datadesc}{ENOMEM} Out of memory \end{datadesc}
\begin{datadesc}{EACCES} Permission denied \end{datadesc}
\begin{datadesc}{EFAULT} Bad address \end{datadesc}
\begin{datadesc}{ENOTBLK} Block device required \end{datadesc}
\begin{datadesc}{EBUSY} Device or resource busy \end{datadesc}
\begin{datadesc}{EEXIST} File exists \end{datadesc}
\begin{datadesc}{EXDEV} Cross-device link \end{datadesc}
\begin{datadesc}{ENODEV} No such device \end{datadesc}
\begin{datadesc}{ENOTDIR} Not a directory \end{datadesc}
\begin{datadesc}{EISDIR} Is a directory \end{datadesc}
\begin{datadesc}{EINVAL} Invalid argument \end{datadesc}
\begin{datadesc}{ENFILE} File table overflow \end{datadesc}
\begin{datadesc}{EMFILE} Too many open files \end{datadesc}
\begin{datadesc}{ENOTTY} Not a typewriter \end{datadesc}
\begin{datadesc}{ETXTBSY} Text file busy \end{datadesc}
\begin{datadesc}{EFBIG} File too large \end{datadesc}
\begin{datadesc}{ENOSPC} No space left on device \end{datadesc}
\begin{datadesc}{ESPIPE} Illegal seek \end{datadesc}
\begin{datadesc}{EROFS} Read-only file system \end{datadesc}
\begin{datadesc}{EMLINK} Too many links \end{datadesc}
\begin{datadesc}{EPIPE} Broken pipe \end{datadesc}
\begin{datadesc}{EDOM} Math argument out of domain of func \end{datadesc}
\begin{datadesc}{ERANGE} Math result not representable \end{datadesc}
\begin{datadesc}{EDEADLK} Resource deadlock would occur \end{datadesc}
\begin{datadesc}{ENAMETOOLONG} File name too long \end{datadesc}
\begin{datadesc}{ENOLCK} No record locks available \end{datadesc}
\begin{datadesc}{ENOSYS} Function not implemented \end{datadesc}
\begin{datadesc}{ENOTEMPTY} Directory not empty \end{datadesc}
\begin{datadesc}{ELOOP} Too many symbolic links encountered \end{datadesc}
\begin{datadesc}{EWOULDBLOCK} Operation would block \end{datadesc}
\begin{datadesc}{ENOMSG} No message of desired type \end{datadesc}
\begin{datadesc}{EIDRM} Identifier removed \end{datadesc}
\begin{datadesc}{ECHRNG} Channel number out of range \end{datadesc}
\begin{datadesc}{EL2NSYNC} Level 2 not synchronized \end{datadesc}
\begin{datadesc}{EL3HLT} Level 3 halted \end{datadesc}
\begin{datadesc}{EL3RST} Level 3 reset \end{datadesc}
\begin{datadesc}{ELNRNG} Link number out of range \end{datadesc}
\begin{datadesc}{EUNATCH} Protocol driver not attached \end{datadesc}
\begin{datadesc}{ENOCSI} No CSI structure available \end{datadesc}
\begin{datadesc}{EL2HLT} Level 2 halted \end{datadesc}
\begin{datadesc}{EBADE} Invalid exchange \end{datadesc}
\begin{datadesc}{EBADR} Invalid request descriptor \end{datadesc}
\begin{datadesc}{EXFULL} Exchange full \end{datadesc}
\begin{datadesc}{ENOANO} No anode \end{datadesc}
\begin{datadesc}{EBADRQC} Invalid request code \end{datadesc}
\begin{datadesc}{EBADSLT} Invalid slot \end{datadesc}
\begin{datadesc}{EDEADLOCK} File locking deadlock error \end{datadesc}
\begin{datadesc}{EBFONT} Bad font file format \end{datadesc}
\begin{datadesc}{ENOSTR} Device not a stream \end{datadesc}
\begin{datadesc}{ENODATA} No data available \end{datadesc}
\begin{datadesc}{ETIME} Timer expired \end{datadesc}
\begin{datadesc}{ENOSR} Out of streams resources \end{datadesc}
\begin{datadesc}{ENONET} Machine is not on the network \end{datadesc}
\begin{datadesc}{ENOPKG} Package not installed \end{datadesc}
\begin{datadesc}{EREMOTE} Object is remote \end{datadesc}
\begin{datadesc}{ENOLINK} Link has been severed \end{datadesc}
\begin{datadesc}{EADV} Advertise error \end{datadesc}
\begin{datadesc}{ESRMNT} Srmount error \end{datadesc}
\begin{datadesc}{ECOMM} Communication error on send \end{datadesc}
\begin{datadesc}{EPROTO} Protocol error \end{datadesc}
\begin{datadesc}{EMULTIHOP} Multihop attempted \end{datadesc}
\begin{datadesc}{EDOTDOT} RFS specific error \end{datadesc}
\begin{datadesc}{EBADMSG} Not a data message \end{datadesc}
\begin{datadesc}{EOVERFLOW} Value too large for defined data type \end{datadesc}
\begin{datadesc}{ENOTUNIQ} Name not unique on network \end{datadesc}
\begin{datadesc}{EBADFD} File descriptor in bad state \end{datadesc}
\begin{datadesc}{EREMCHG} Remote address changed \end{datadesc}
\begin{datadesc}{ELIBACC} Can not access a needed shared library \end{datadesc}
\begin{datadesc}{ELIBBAD} Accessing a corrupted shared library \end{datadesc}
\begin{datadesc}{ELIBSCN} .lib section in a.out corrupted \end{datadesc}
\begin{datadesc}{ELIBMAX} Attempting to link in too many shared libraries \end{datadesc}
\begin{datadesc}{ELIBEXEC} Cannot exec a shared library directly \end{datadesc}
\begin{datadesc}{EILSEQ} Illegal byte sequence \end{datadesc}
\begin{datadesc}{ERESTART} Interrupted system call should be restarted \end{datadesc}
\begin{datadesc}{ESTRPIPE} Streams pipe error \end{datadesc}
\begin{datadesc}{EUSERS} Too many users \end{datadesc}
\begin{datadesc}{ENOTSOCK} Socket operation on non-socket \end{datadesc}
\begin{datadesc}{EDESTADDRREQ} Destination address required \end{datadesc}
\begin{datadesc}{EMSGSIZE} Message too long \end{datadesc}
\begin{datadesc}{EPROTOTYPE} Protocol wrong type for socket \end{datadesc}
\begin{datadesc}{ENOPROTOOPT} Protocol not available \end{datadesc}
\begin{datadesc}{EPROTONOSUPPORT} Protocol not supported \end{datadesc}
\begin{datadesc}{ESOCKTNOSUPPORT} Socket type not supported \end{datadesc}
\begin{datadesc}{EOPNOTSUPP} Operation not supported on transport endpoint \end{datadesc}
\begin{datadesc}{EPFNOSUPPORT} Protocol family not supported \end{datadesc}
\begin{datadesc}{EAFNOSUPPORT} Address family not supported by protocol \end{datadesc}
\begin{datadesc}{EADDRINUSE} Address already in use \end{datadesc}
\begin{datadesc}{EADDRNOTAVAIL} Cannot assign requested address \end{datadesc}
\begin{datadesc}{ENETDOWN} Network is down \end{datadesc}
\begin{datadesc}{ENETUNREACH} Network is unreachable \end{datadesc}
\begin{datadesc}{ENETRESET} Network dropped connection because of reset \end{datadesc}
\begin{datadesc}{ECONNABORTED} Software caused connection abort \end{datadesc}
\begin{datadesc}{ECONNRESET} Connection reset by peer \end{datadesc}
\begin{datadesc}{ENOBUFS} No buffer space available \end{datadesc}
\begin{datadesc}{EISCONN} Transport endpoint is already connected \end{datadesc}
\begin{datadesc}{ENOTCONN} Transport endpoint is not connected \end{datadesc}
\begin{datadesc}{ESHUTDOWN} Cannot send after transport endpoint shutdown \end{datadesc}
\begin{datadesc}{ETOOMANYREFS} Too many references: cannot splice \end{datadesc}
\begin{datadesc}{ETIMEDOUT} Connection timed out \end{datadesc}
\begin{datadesc}{ECONNREFUSED} Connection refused \end{datadesc}
\begin{datadesc}{EHOSTDOWN} Host is down \end{datadesc}
\begin{datadesc}{EHOSTUNREACH} No route to host \end{datadesc}
\begin{datadesc}{EALREADY} Operation already in progress \end{datadesc}
\begin{datadesc}{EINPROGRESS} Operation now in progress \end{datadesc}
\begin{datadesc}{ESTALE} Stale NFS file handle \end{datadesc}
\begin{datadesc}{EUCLEAN} Structure needs cleaning \end{datadesc}
\begin{datadesc}{ENOTNAM} Not a XENIX named type file \end{datadesc}
\begin{datadesc}{ENAVAIL} No XENIX semaphores available \end{datadesc}
\begin{datadesc}{EISNAM} Is a named type file \end{datadesc}
\begin{datadesc}{EREMOTEIO} Remote I/O error \end{datadesc}
\begin{datadesc}{EDQUOT} Quota exceeded \end{datadesc}



\chapter{Optional Operating System Services}

The modules described in this chapter provide interfaces to operating
system features that are available on selected operating systems only.
The interfaces are generally modelled after the \UNIX{} or C
interfaces but they are available on some other systems as well
(e.g. Windows or NT).  Here's an overview:

\begin{description}

\item[signal]
--- Set handlers for asynchronous events.

\item[socket]
--- Low-level networking interface.

\item[select]
--- Wait for I/O completion on multiple streams.

\item[thread]
--- Create multiple threads of control within one namespace.

\end{description}
		% Optional Operating System Services
\section{Built-in Module \sectcode{signal}}

\bimodindex{signal}
This module provides mechanisms to use signal handlers in Python.
Some general rules for working with signals handlers:

\begin{itemize}

\item
A handler for a particular signal, once set, remains installed until
it is explicitly reset (i.e. Python emulates the BSD style interface
regardless of the underlying implementation), with the exception of
the handler for \code{SIGCHLD}, which follows the underlying
implementation.

\item
There is no way to ``block'' signals temporarily from critical
sections (since this is not supported by all \UNIX{} flavors).

\item
Although Python signal handlers are called asynchronously as far as
the Python user is concerned, they can only occur between the
``atomic'' instructions of the Python interpreter.  This means that
signals arriving during long calculations implemented purely in C
(e.g.\ regular expression matches on large bodies of text) may be
delayed for an arbitrary amount of time.

\item
When a signal arrives during an I/O operation, it is possible that the
I/O operation raises an exception after the signal handler returns.
This is dependent on the underlying \UNIX{} system's semantics regarding
interrupted system calls.

\item
Because the C signal handler always returns, it makes little sense to
catch synchronous errors like \code{SIGFPE} or \code{SIGSEGV}.

\item
Python installs a small number of signal handlers by default:
\code{SIGPIPE} is ignored (so write errors on pipes and sockets can be
reported as ordinary Python exceptions), \code{SIGINT} is translated
into a \code{KeyboardInterrupt} exception, and \code{SIGTERM} is
caught so that necessary cleanup (especially \code{sys.exitfunc}) can
be performed before actually terminating.  All of these can be
overridden.

\item
Some care must be taken if both signals and threads are used in the
same program.  The fundamental thing to remember in using signals and
threads simultaneously is:\ always perform \code{signal()} operations
in the main thread of execution.  Any thread can perform an
\code{alarm()}, \code{getsignal()}, or \code{pause()}; only the main
thread can set a new signal handler, and the main thread will be the
only one to receive signals (this is enforced by the Python signal
module, even if the underlying thread implementation supports sending
signals to individual threads).  This means that signals can't be used
as a means of interthread communication.  Use locks instead.

\end{itemize}

The variables defined in the signal module are:

\renewcommand{\indexsubitem}{(in module signal)}
\begin{datadesc}{SIG_DFL}
  This is one of two standard signal handling options; it will simply
  perform the default function for the signal.  For example, on most
  systems the default action for SIGQUIT is to dump core and exit,
  while the default action for SIGCLD is to simply ignore it.
\end{datadesc}

\begin{datadesc}{SIG_IGN}
  This is another standard signal handler, which will simply ignore
  the given signal.
\end{datadesc}

\begin{datadesc}{SIG*}
  All the signal numbers are defined symbolically.  For example, the
  hangup signal is defined as \code{signal.SIGHUP}; the variable names
  are identical to the names used in C programs, as found in
  \file{signal.h}.
  The \UNIX{} man page for \file{signal} lists the existing signals (on
  some systems this is \file{signal(2)}, on others the list is in
  \file{signal(7)}).
  Note that not all systems define the same set of signal names; only
  those names defined by the system are defined by this module.
\end{datadesc}

\begin{datadesc}{NSIG}
  One more than the number of the highest signal number.
\end{datadesc}

The signal module defines the following functions:

\begin{funcdesc}{alarm}{time}
  If \var{time} is non-zero, this function requests that a
  \code{SIGALRM} signal be sent to the process in \var{time} seconds.
  Any previously scheduled alarm is canceled (i.e.\ only one alarm can
  be scheduled at any time).  The returned value is then the number of
  seconds before any previously set alarm was to have been delivered.
  If \var{time} is zero, no alarm id scheduled, and any scheduled
  alarm is canceled.  The return value is the number of seconds
  remaining before a previously scheduled alarm.  If the return value
  is zero, no alarm is currently scheduled.  (See the \UNIX{} man page
  \code{alarm(2)}.)
\end{funcdesc}

\begin{funcdesc}{getsignal}{signalnum}
  Return the current signal handler for the signal \var{signalnum}.
  The returned value may be a callable Python object, or one of the
  special values \code{signal.SIG_IGN}, \code{signal.SIG_DFL} or
  \code{None}.  Here, \code{signal.SIG_IGN} means that the signal was
  previously ignored, \code{signal.SIG_DFL} means that the default way
  of handling the signal was previously in use, and \code{None} means
  that the previous signal handler was not installed from Python.
\end{funcdesc}

\begin{funcdesc}{pause}{}
  Cause the process to sleep until a signal is received; the
  appropriate handler will then be called.  Returns nothing.  (See the
  \UNIX{} man page \code{signal(2)}.)
\end{funcdesc}

\begin{funcdesc}{signal}{signalnum\, handler}
  Set the handler for signal \var{signalnum} to the function
  \var{handler}.  \var{handler} can be any callable Python object, or
  one of the special values \code{signal.SIG_IGN} or
  \code{signal.SIG_DFL}.  The previous signal handler will be returned
  (see the description of \code{getsignal()} above).  (See the \UNIX{}
  man page \code{signal(2)}.)

  When threads are enabled, this function can only be called from the
  main thread; attempting to call it from other threads will cause a
  \code{ValueError} exception to be raised.

  The \var{handler} is called with two arguments: the signal number
  and the current stack frame (\code{None} or a frame object; see the
  reference manual for a description of frame objects).
\obindex{frame}
\end{funcdesc}

\section{Built-in Module \sectcode{socket}}

\bimodindex{socket}
This module provides access to the BSD {\em socket} interface.
It is available on \UNIX{} systems that support this interface.

For an introduction to socket programming (in C), see the following
papers: \emph{An Introductory 4.3BSD Interprocess Communication
Tutorial}, by Stuart Sechrest and \emph{An Advanced 4.3BSD Interprocess
Communication Tutorial}, by Samuel J.  Leffler et al, both in the
\UNIX{} Programmer's Manual, Supplementary Documents 1 (sections PS1:7
and PS1:8).  The \UNIX{} manual pages for the various socket-related
system calls are also a valuable source of information on the details of
socket semantics.

The Python interface is a straightforward transliteration of the
\UNIX{} system call and library interface for sockets to Python's
object-oriented style: the \code{socket()} function returns a
\dfn{socket object} whose methods implement the various socket system
calls.  Parameter types are somewhat higer-level than in the C
interface: as with \code{read()} and \code{write()} operations on Python
files, buffer allocation on receive operations is automatic, and
buffer length is implicit on send operations.

Socket addresses are represented as a single string for the
\code{AF_UNIX} address family and as a pair
\code{(\var{host}, \var{port})} for the \code{AF_INET} address family,
where \var{host} is a string representing
either a hostname in Internet domain notation like
\code{'daring.cwi.nl'} or an IP address like \code{'100.50.200.5'},
and \var{port} is an integral port number.  Other address families are
currently not supported.  The address format required by a particular
socket object is automatically selected based on the address family
specified when the socket object was created.

All errors raise exceptions.  The normal exceptions for invalid
argument types and out-of-memory conditions can be raised; errors
related to socket or address semantics raise the error \code{socket.error}.

Non-blocking mode is supported through the \code{setblocking()}
method.

The module \code{socket} exports the following constants and functions:

\renewcommand{\indexsubitem}{(in module socket)}
\begin{excdesc}{error}
This exception is raised for socket- or address-related errors.
The accompanying value is either a string telling what went wrong or a
pair \code{(\var{errno}, \var{string})}
representing an error returned by a system
call, similar to the value accompanying \code{posix.error}.
\end{excdesc}

\begin{datadesc}{AF_UNIX}
\dataline{AF_INET}
These constants represent the address (and protocol) families,
used for the first argument to \code{socket()}.  If the \code{AF_UNIX}
constant is not defined then this protocol is unsupported.
\end{datadesc}

\begin{datadesc}{SOCK_STREAM}
\dataline{SOCK_DGRAM}
\dataline{SOCK_RAW}
\dataline{SOCK_RDM}
\dataline{SOCK_SEQPACKET}
These constants represent the socket types,
used for the second argument to \code{socket()}.
(Only \code{SOCK_STREAM} and
\code{SOCK_DGRAM} appear to be generally useful.)
\end{datadesc}

\begin{datadesc}{SO_*}
\dataline{SOMAXCONN}
\dataline{MSG_*}
\dataline{SOL_*}
\dataline{IPPROTO_*}
\dataline{IPPORT_*}
\dataline{INADDR_*}
\dataline{IP_*}
Many constants of these forms, documented in the \UNIX{} documentation on
sockets and/or the IP protocol, are also defined in the socket module.
They are generally used in arguments to the \code{setsockopt} and
\code{getsockopt} methods of socket objects.  In most cases, only
those symbols that are defined in the \UNIX{} header files are defined;
for a few symbols, default values are provided.
\end{datadesc}

\begin{funcdesc}{gethostbyname}{hostname}
Translate a host name to IP address format.  The IP address is
returned as a string, e.g.,  \code{'100.50.200.5'}.  If the host name
is an IP address itself it is returned unchanged.
\end{funcdesc}

\begin{funcdesc}{gethostname}{}
Return a string containing the hostname of the machine where 
the Python interpreter is currently executing.  If you want to know the
current machine's IP address, use
\code{socket.gethostbyname(socket.gethostname())}.
\end{funcdesc}

\begin{funcdesc}{gethostbyaddr}{ip_address}
Return a triple \code{(hostname, aliaslist, ipaddrlist)} where
\code{hostname} is the primary host name responding to the given
\var{ip_address}, \code{aliaslist} is a (possibly empty) list of
alternative host names for the same address, and \code{ipaddrlist} is
a list of IP addresses for the same interface on the same
host (most likely containing only a single address).
\end{funcdesc}

\begin{funcdesc}{getservbyname}{servicename\, protocolname}
Translate an Internet service name and protocol name to a port number
for that service.  The protocol name should be \code{'tcp'} or
\code{'udp'}.
\end{funcdesc}

\begin{funcdesc}{socket}{family\, type\optional{\, proto}}
Create a new socket using the given address family, socket type and
protocol number.  The address family should be \code{AF_INET} or
\code{AF_UNIX}.  The socket type should be \code{SOCK_STREAM},
\code{SOCK_DGRAM} or perhaps one of the other \samp{SOCK_} constants.
The protocol number is usually zero and may be omitted in that case.
\end{funcdesc}

\begin{funcdesc}{fromfd}{fd\, family\, type\optional{\, proto}}
Build a socket object from an existing file descriptor (an integer as
returned by a file object's \code{fileno} method).  Address family,
socket type and protocol number are as for the \code{socket} function
above.  The file descriptor should refer to a socket, but this is not
checked --- subsequent operations on the object may fail if the file
descriptor is invalid.  This function is rarely needed, but can be
used to get or set socket options on a socket passed to a program as
standard input or output (e.g.\ a server started by the \UNIX{} inet
daemon).
\end{funcdesc}

\subsection{Socket Objects}

\noindent
Socket objects have the following methods.  Except for
\code{makefile()} these correspond to \UNIX{} system calls applicable to
sockets.

\renewcommand{\indexsubitem}{(socket method)}
\begin{funcdesc}{accept}{}
Accept a connection.
The socket must be bound to an address and listening for connections.
The return value is a pair \code{(\var{conn}, \var{address})}
where \var{conn} is a \emph{new} socket object usable to send and
receive data on the connection, and \var{address} is the address bound
to the socket on the other end of the connection.
\end{funcdesc}

\begin{funcdesc}{bind}{address}
Bind the socket to \var{address}.  The socket must not already be bound.
(The format of \var{address} depends on the address family --- see above.)
\end{funcdesc}

\begin{funcdesc}{close}{}
Close the socket.  All future operations on the socket object will fail.
The remote end will receive no more data (after queued data is flushed).
Sockets are automatically closed when they are garbage-collected.
\end{funcdesc}

\begin{funcdesc}{connect}{address}
Connect to a remote socket at \var{address}.
(The format of \var{address} depends on the address family --- see above.)
\end{funcdesc}

\begin{funcdesc}{fileno}{}
Return the socket's file descriptor (a small integer).  This is useful
with \code{select}.
\end{funcdesc}

\begin{funcdesc}{getpeername}{}
Return the remote address to which the socket is connected.  This is
useful to find out the port number of a remote IP socket, for instance.
(The format of the address returned depends on the address family ---
see above.)  On some systems this function is not supported.
\end{funcdesc}

\begin{funcdesc}{getsockname}{}
Return the socket's own address.  This is useful to find out the port
number of an IP socket, for instance.
(The format of the address returned depends on the address family ---
see above.)
\end{funcdesc}

\begin{funcdesc}{getsockopt}{level\, optname\optional{\, buflen}}
Return the value of the given socket option (see the \UNIX{} man page
{\it getsockopt}(2)).  The needed symbolic constants (\code{SO_*} etc.)
are defined in this module.  If \var{buflen}
is absent, an integer option is assumed and its integer value
is returned by the function.  If \var{buflen} is present, it specifies
the maximum length of the buffer used to receive the option in, and
this buffer is returned as a string.  It is up to the caller to decode
the contents of the buffer (see the optional built-in module
\code{struct} for a way to decode C structures encoded as strings).
\end{funcdesc}

\begin{funcdesc}{listen}{backlog}
Listen for connections made to the socket.  The \var{backlog} argument
specifies the maximum number of queued connections and should be at
least 1; the maximum value is system-dependent (usually 5).
\end{funcdesc}

\begin{funcdesc}{makefile}{\optional{mode\optional{\, bufsize}}}
Return a \dfn{file object} associated with the socket.  (File objects
were described earlier under Built-in Types.)  The file object
references a \code{dup()}ped version of the socket file descriptor, so
the file object and socket object may be closed or garbage-collected
independently.  The optional \var{mode} and \var{bufsize} arguments
are interpreted the same way as by the built-in
\code{open()} function.
\end{funcdesc}

\begin{funcdesc}{recv}{bufsize\optional{\, flags}}
Receive data from the socket.  The return value is a string representing
the data received.  The maximum amount of data to be received
at once is specified by \var{bufsize}.  See the \UNIX{} manual page
for the meaning of the optional argument \var{flags}; it defaults to
zero.
\end{funcdesc}

\begin{funcdesc}{recvfrom}{bufsize\optional{\, flags}}
Receive data from the socket.  The return value is a pair
\code{(\var{string}, \var{address})} where \var{string} is a string
representing the data received and \var{address} is the address of the
socket sending the data.  The optional \var{flags} argument has the
same meaning as for \code{recv()} above.
(The format of \var{address} depends on the address family --- see above.)
\end{funcdesc}

\begin{funcdesc}{send}{string\optional{\, flags}}
Send data to the socket.  The socket must be connected to a remote
socket.  The optional \var{flags} argument has the same meaning as for
\code{recv()} above.  Return the number of bytes sent.
\end{funcdesc}

\begin{funcdesc}{sendto}{string\optional{\, flags}\, address}
Send data to the socket.  The socket should not be connected to a
remote socket, since the destination socket is specified by
\code{address}.  The optional \var{flags} argument has the same
meaning as for \code{recv()} above.  Return the number of bytes sent.
(The format of \var{address} depends on the address family --- see above.)
\end{funcdesc}

\begin{funcdesc}{setblocking}{flag}
Set blocking or non-blocking mode of the socket: if \var{flag} is 0,
the socket is set to non-blocking, else to blocking mode.  Initially
all sockets are in blocking mode.  In non-blocking mode, if a
\code{recv} call doesn't find any data, or if a \code{send} call can't
immediately dispose of the data, a \code{socket.error} exception is
raised; in blocking mode, the calls block until they can proceed.
\end{funcdesc}

\begin{funcdesc}{setsockopt}{level\, optname\, value}
Set the value of the given socket option (see the \UNIX{} man page
{\it setsockopt}(2)).  The needed symbolic constants are defined in
the \code{socket} module (\code{SO_*} etc.).  The value can be an
integer or a string representing a buffer.  In the latter case it is
up to the caller to ensure that the string contains the proper bits
(see the optional built-in module
\code{struct} for a way to encode C structures as strings).
\end{funcdesc}

\begin{funcdesc}{shutdown}{how}
Shut down one or both halves of the connection.  If \var{how} is \code{0},
further receives are disallowed.  If \var{how} is \code{1}, further sends are
disallowed.  If \var{how} is \code{2}, further sends and receives are
disallowed.
\end{funcdesc}

Note that there are no methods \code{read()} or \code{write()}; use
\code{recv()} and \code{send()} without \var{flags} argument instead.

\subsection{Example}
\nodename{Socket Example}

Here are two minimal example programs using the TCP/IP protocol:\ a
server that echoes all data that it receives back (servicing only one
client), and a client using it.  Note that a server must perform the
sequence \code{socket}, \code{bind}, \code{listen}, \code{accept}
(possibly repeating the \code{accept} to service more than one client),
while a client only needs the sequence \code{socket}, \code{connect}.
Also note that the server does not \code{send}/\code{receive} on the
socket it is listening on but on the new socket returned by
\code{accept}.

\bcode\begin{verbatim}
# Echo server program
from socket import *
HOST = ''                 # Symbolic name meaning the local host
PORT = 50007              # Arbitrary non-privileged server
s = socket(AF_INET, SOCK_STREAM)
s.bind(HOST, PORT)
s.listen(1)
conn, addr = s.accept()
print 'Connected by', addr
while 1:
    data = conn.recv(1024)
    if not data: break
    conn.send(data)
conn.close()
\end{verbatim}\ecode

\bcode\begin{verbatim}
# Echo client program
from socket import *
HOST = 'daring.cwi.nl'    # The remote host
PORT = 50007              # The same port as used by the server
s = socket(AF_INET, SOCK_STREAM)
s.connect(HOST, PORT)
s.send('Hello, world')
data = s.recv(1024)
s.close()
print 'Received', `data`
\end{verbatim}\ecode

\section{Built-in Module \sectcode{select}}
\bimodindex{select}

This module provides access to the function \code{select} available in
most \UNIX{} versions.  It defines the following:

\renewcommand{\indexsubitem}{(in module select)}
\begin{excdesc}{error}
The exception raised when an error occurs.  The accompanying value is
a pair containing the numeric error code from \code{errno} and the
corresponding string, as would be printed by the C function
\code{perror()}.
\end{excdesc}

\begin{funcdesc}{select}{iwtd\, owtd\, ewtd\optional{\, timeout}}
This is a straightforward interface to the \UNIX{} \code{select()}
system call.  The first three arguments are lists of `waitable
objects': either integers representing \UNIX{} file descriptors or
objects with a parameterless method named \code{fileno()} returning
such an integer.  The three lists of waitable objects are for input,
output and `exceptional conditions', respectively.  Empty lists are
allowed.  The optional \var{timeout} argument specifies a time-out as a
floating point number in seconds.  When the \var{timeout} argument
is omitted the function blocks until at least one file descriptor is
ready.  A time-out value of zero specifies a poll and never blocks.

The return value is a triple of lists of objects that are ready:
subsets of the first three arguments.  When the time-out is reached
without a file descriptor becoming ready, three empty lists are
returned.

Amongst the acceptable object types in the lists are Python file
objects (e.g. \code{sys.stdin}, or objects returned by \code{open()}
or \code{posix.popen()}), socket objects returned by
\code{socket.socket()}, and the module \code{stdwin} which happens to
define a function \code{fileno()} for just this purpose.  You may
also define a \dfn{wrapper} class yourself, as long as it has an
appropriate \code{fileno()} method (that really returns a \UNIX{} file
descriptor, not just a random integer).
\end{funcdesc}
\ttindex{socket}
\ttindex{stdwin}

\section{Built-in Module \sectcode{thread}}
\bimodindex{thread}

This module provides low-level primitives for working with multiple
threads (a.k.a.\ \dfn{light-weight processes} or \dfn{tasks}) --- multiple
threads of control sharing their global data space.  For
synchronization, simple locks (a.k.a.\ \dfn{mutexes} or \dfn{binary
semaphores}) are provided.

The module is optional and supported on SGI IRIX 4.x and 5.x and Sun
Solaris 2.x systems, as well as on systems that have a PTHREAD
implementation (e.g.\ KSR).

It defines the following constant and functions:

\renewcommand{\indexsubitem}{(in module thread)}
\begin{excdesc}{error}
Raised on thread-specific errors.
\end{excdesc}

\begin{funcdesc}{start_new_thread}{func\, arg}
Start a new thread.  The thread executes the function \var{func}
with the argument list \var{arg} (which must be a tuple).  When the
function returns, the thread silently exits.  When the function
terminates with an unhandled exception, a stack trace is printed and
then the thread exits (but other threads continue to run).
\end{funcdesc}

\begin{funcdesc}{exit}{}
This is a shorthand for \code{thread.exit_thread()}.
\end{funcdesc}

\begin{funcdesc}{exit_thread}{}
Raise the \code{SystemExit} exception.  When not caught, this will
cause the thread to exit silently.
\end{funcdesc}

%\begin{funcdesc}{exit_prog}{status}
%Exit all threads and report the value of the integer argument
%\var{status} as the exit status of the entire program.
%\strong{Caveat:} code in pending \code{finally} clauses, in this thread
%or in other threads, is not executed.
%\end{funcdesc}

\begin{funcdesc}{allocate_lock}{}
Return a new lock object.  Methods of locks are described below.  The
lock is initially unlocked.
\end{funcdesc}

\begin{funcdesc}{get_ident}{}
Return the `thread identifier' of the current thread.  This is a
nonzero integer.  Its value has no direct meaning; it is intended as a
magic cookie to be used e.g. to index a dictionary of thread-specific
data.  Thread identifiers may be recycled when a thread exits and
another thread is created.
\end{funcdesc}

Lock objects have the following methods:

\renewcommand{\indexsubitem}{(lock method)}
\begin{funcdesc}{acquire}{\optional{waitflag}}
Without the optional argument, this method acquires the lock
unconditionally, if necessary waiting until it is released by another
thread (only one thread at a time can acquire a lock --- that's their
reason for existence), and returns \code{None}.  If the integer
\var{waitflag} argument is present, the action depends on its value:\
if it is zero, the lock is only acquired if it can be acquired
immediately without waiting, while if it is nonzero, the lock is
acquired unconditionally as before.  If an argument is present, the
return value is 1 if the lock is acquired successfully, 0 if not.
\end{funcdesc}

\begin{funcdesc}{release}{}
Releases the lock.  The lock must have been acquired earlier, but not
necessarily by the same thread.
\end{funcdesc}

\begin{funcdesc}{locked}{}
Return the status of the lock:\ 1 if it has been acquired by some
thread, 0 if not.
\end{funcdesc}

{\bf Caveats:}

\begin{itemize}
\item
Threads interact strangely with interrupts: the
\code{KeyboardInterrupt} exception will be received by an arbitrary
thread.  (When the \code{signal} module is available, interrupts
always go to the main thread.)

\item
Calling \code{sys.exit()} or raising the \code{SystemExit} is
equivalent to calling \code{thread.exit_thread()}.

\item
Not all built-in functions that may block waiting for I/O allow other
threads to run.  (The most popular ones (\code{sleep}, \code{read},
\code{select}) work as expected.)

\end{itemize}


\chapter{UNIX Specific Services}

The modules described in this chapter provide interfaces to features
that are unique to the \UNIX{} operating system, or in some cases to
some or many variants of it.  Here's an overview:

\begin{description}

\item[posix]
--- The most common Posix system calls (normally used via module \code{os}).

\item[posixpath]
--- Common Posix pathname manipulations (normally used via \code{os.path}).

\item[pwd]
--- The password database (\code{getpwnam()} and friends).

\item[grp]
--- The group database (\code{getgrnam()} and friends).

\item[crypt]
--- The (\code{crypt()} function used to check Unix passwords).

\item[dbm]
--- The standard ``database'' interface, based on \code{ndbm}.

\item[gdbm]
--- GNU's reinterpretation of dbm.

\item[termios]
--- Posix style tty control.

\item[fcntl]
--- The \code{fcntl()} and \code{ioctl()} system calls.

\item[posixfile]
--- A file-like object with support for locking.

\end{description}
			% UNIX Specific Services
\section{Built-in Module \sectcode{posix}}
\bimodindex{posix}

This module provides access to operating system functionality that is
standardized by the C Standard and the POSIX standard (a thinly disguised
\UNIX{} interface).

\strong{Do not import this module directly.}  Instead, import the
module \code{os}, which provides a \emph{portable} version of this
interface.  On \UNIX{}, the \code{os} module provides a superset of
the \code{posix} interface.  On non-\UNIX{} operating systems the
\code{posix} module is not available, but a subset is always available
through the \code{os} interface.  Once \code{os} is imported, there is
\emph{no} performance penalty in using it instead of
\code{posix}.
\stmodindex{os}

The descriptions below are very terse; refer to the
corresponding \UNIX{} manual entry for more information.  Arguments
called \var{path} refer to a pathname given as a string.

Errors are reported as exceptions; the usual exceptions are given
for type errors, while errors reported by the system calls raise
\code{posix.error}, described below.

Module \code{posix} defines the following data items:

\renewcommand{\indexsubitem}{(data in module posix)}
\begin{datadesc}{environ}
A dictionary representing the string environment at the time
the interpreter was started.
For example,
\code{posix.environ['HOME']}
is the pathname of your home directory, equivalent to
\code{getenv("HOME")}
in C.
Modifying this dictionary does not affect the string environment
passed on by \code{execv()}, \code{popen()} or \code{system()}; if you
need to change the environment, pass \code{environ} to \code{execve()}
or add variable assignments and export statements to the command
string for \code{system()} or \code{popen()}.%
\footnote{The problem with automatically passing on \code{environ} is
that there is no portable way of changing the environment.}
\end{datadesc}

\renewcommand{\indexsubitem}{(exception in module posix)}
\begin{excdesc}{error}
This exception is raised when a POSIX function returns a
POSIX-related error (e.g., not for illegal argument types).  Its
string value is \code{'posix.error'}.  The accompanying value is a
pair containing the numeric error code from \code{errno} and the
corresponding string, as would be printed by the C function
\code{perror()}.
\end{excdesc}

It defines the following functions and constants:

\renewcommand{\indexsubitem}{(in module posix)}
\begin{funcdesc}{chdir}{path}
Change the current working directory to \var{path}.
\end{funcdesc}

\begin{funcdesc}{chmod}{path\, mode}
Change the mode of \var{path} to the numeric \var{mode}.
\end{funcdesc}

\begin{funcdesc}{chown}{path\, uid, gid}
Change the owner and group id of \var{path} to the numeric \var{uid}
and \var{gid}.
(Not on MS-DOS.)
\end{funcdesc}

\begin{funcdesc}{close}{fd}
Close file descriptor \var{fd}.

Note: this function is intended for low-level I/O and must be applied
to a file descriptor as returned by \code{posix.open()} or
\code{posix.pipe()}.  To close a ``file object'' returned by the
built-in function \code{open} or by \code{posix.popen} or
\code{posix.fdopen}, use its \code{close()} method.
\end{funcdesc}

\begin{funcdesc}{dup}{fd}
Return a duplicate of file descriptor \var{fd}.
\end{funcdesc}

\begin{funcdesc}{dup2}{fd\, fd2}
Duplicate file descriptor \var{fd} to \var{fd2}, closing the latter
first if necessary.  Return \code{None}.
\end{funcdesc}

\begin{funcdesc}{execv}{path\, args}
Execute the executable \var{path} with argument list \var{args},
replacing the current process (i.e., the Python interpreter).
The argument list may be a tuple or list of strings.
(Not on MS-DOS.)
\end{funcdesc}

\begin{funcdesc}{execve}{path\, args\, env}
Execute the executable \var{path} with argument list \var{args},
and environment \var{env},
replacing the current process (i.e., the Python interpreter).
The argument list may be a tuple or list of strings.
The environment must be a dictionary mapping strings to strings.
(Not on MS-DOS.)
\end{funcdesc}

\begin{funcdesc}{_exit}{n}
Exit to the system with status \var{n}, without calling cleanup
handlers, flushing stdio buffers, etc.
(Not on MS-DOS.)

Note: the standard way to exit is \code{sys.exit(\var{n})}.
\code{posix._exit()} should normally only be used in the child process
after a \code{fork()}.
\end{funcdesc}

\begin{funcdesc}{fdopen}{fd\optional{\, mode\optional{\, bufsize}}}
Return an open file object connected to the file descriptor \var{fd}.
The \var{mode} and \var{bufsize} arguments have the same meaning as
the corresponding arguments to the built-in \code{open()} function.
\end{funcdesc}

\begin{funcdesc}{fork}{}
Fork a child process.  Return 0 in the child, the child's process id
in the parent.
(Not on MS-DOS.)
\end{funcdesc}

\begin{funcdesc}{fstat}{fd}
Return status for file descriptor \var{fd}, like \code{stat()}.
\end{funcdesc}

\begin{funcdesc}{getcwd}{}
Return a string representing the current working directory.
\end{funcdesc}

\begin{funcdesc}{getegid}{}
Return the current process's effective group id.
(Not on MS-DOS.)
\end{funcdesc}

\begin{funcdesc}{geteuid}{}
Return the current process's effective user id.
(Not on MS-DOS.)
\end{funcdesc}

\begin{funcdesc}{getgid}{}
Return the current process's group id.
(Not on MS-DOS.)
\end{funcdesc}

\begin{funcdesc}{getpgrp}{}
Return the current process group id.
(Not on MS-DOS.)
\end{funcdesc}

\begin{funcdesc}{getpid}{}
Return the current process id.
(Not on MS-DOS.)
\end{funcdesc}

\begin{funcdesc}{getppid}{}
Return the parent's process id.
(Not on MS-DOS.)
\end{funcdesc}

\begin{funcdesc}{getuid}{}
Return the current process's user id.
(Not on MS-DOS.)
\end{funcdesc}

\begin{funcdesc}{kill}{pid\, sig}
Kill the process \var{pid} with signal \var{sig}.
(Not on MS-DOS.)
\end{funcdesc}

\begin{funcdesc}{link}{src\, dst}
Create a hard link pointing to \var{src} named \var{dst}.
(Not on MS-DOS.)
\end{funcdesc}

\begin{funcdesc}{listdir}{path}
Return a list containing the names of the entries in the directory.
The list is in arbitrary order.  It does not include the special
entries \code{'.'} and \code{'..'} even if they are present in the
directory.
\end{funcdesc}

\begin{funcdesc}{lseek}{fd\, pos\, how}
Set the current position of file descriptor \var{fd} to position
\var{pos}, modified by \var{how}: 0 to set the position relative to
the beginning of the file; 1 to set it relative to the current
position; 2 to set it relative to the end of the file.
\end{funcdesc}

\begin{funcdesc}{lstat}{path}
Like \code{stat()}, but do not follow symbolic links.  (On systems
without symbolic links, this is identical to \code{posix.stat}.)
\end{funcdesc}

\begin{funcdesc}{mkfifo}{path\optional{\, mode}}
Create a FIFO (a POSIX named pipe) named \var{path} with numeric mode
\var{mode}.  The default \var{mode} is 0666 (octal).  The current
umask value is first masked out from the mode.
(Not on MS-DOS.)

FIFOs are pipes that can be accessed like regular files.  FIFOs exist
until they are deleted (for example with \code{os.unlink}).
Generally, FIFOs are used as rendez-vous between ``client'' and
``server'' type processes: the server opens the FIFO for reading, and
the client opens it for writing.  Note that \code{mkfifo()} doesn't
open the FIFO -- it just creates the rendez-vous point.
\end{funcdesc}

\begin{funcdesc}{mkdir}{path\optional{\, mode}}
Create a directory named \var{path} with numeric mode \var{mode}.
The default \var{mode} is 0777 (octal).  On some systems, \var{mode}
is ignored.  Where it is used, the current umask value is first
masked out.
\end{funcdesc}

\begin{funcdesc}{nice}{increment}
Add \var{incr} to the process' ``niceness''.  Return the new niceness.
(Not on MS-DOS.)
\end{funcdesc}

\begin{funcdesc}{open}{file\, flags\, mode}
Open the file \var{file} and set various flags according to
\var{flags} and possibly its mode according to \var{mode}.
Return the file descriptor for the newly opened file.

Note: this function is intended for low-level I/O.  For normal usage,
use the built-in function \code{open}, which returns a ``file object''
with \code{read()} and  \code{write()} methods (and many more).
\end{funcdesc}

\begin{funcdesc}{pipe}{}
Create a pipe.  Return a pair of file descriptors \code{(r, w)}
usable for reading and writing, respectively.
(Not on MS-DOS.)
\end{funcdesc}

\begin{funcdesc}{plock}{op}
Lock program segments into memory.  The value of \var{op}
(defined in \code{<sys/lock.h>}) determines which segments are locked.
(Not on MS-DOS.)
\end{funcdesc}

\begin{funcdesc}{popen}{command\optional{\, mode\optional{\, bufsize}}}
Open a pipe to or from \var{command}.  The return value is an open
file object connected to the pipe, which can be read or written
depending on whether \var{mode} is \code{'r'} (default) or \code{'w'}.
The \var{bufsize} argument has the same meaning as the corresponding
argument to the built-in \code{open()} function.
(Not on MS-DOS.)
\end{funcdesc}

\begin{funcdesc}{read}{fd\, n}
Read at most \var{n} bytes from file descriptor \var{fd}.
Return a string containing the bytes read.

Note: this function is intended for low-level I/O and must be applied
to a file descriptor as returned by \code{posix.open()} or
\code{posix.pipe()}.  To read a ``file object'' returned by the
built-in function \code{open} or by \code{posix.popen} or
\code{posix.fdopen}, or \code{sys.stdin}, use its
\code{read()} or \code{readline()} methods.
\end{funcdesc}

\begin{funcdesc}{readlink}{path}
Return a string representing the path to which the symbolic link
points.  (On systems without symbolic links, this always raises
\code{posix.error}.)
\end{funcdesc}

\begin{funcdesc}{remove}{path}
Remove the file \var{path}.  See \code{rmdir} below to remove a directory.
\end{funcdesc}

\begin{funcdesc}{rename}{src\, dst}
Rename the file or directory \var{src} to \var{dst}.
\end{funcdesc}

\begin{funcdesc}{rmdir}{path}
Remove the directory \var{path}.
\end{funcdesc}

\begin{funcdesc}{setgid}{gid}
Set the current process's group id.
(Not on MS-DOS.)
\end{funcdesc}

\begin{funcdesc}{setpgrp}{}
Calls the system call \code{setpgrp()} or \code{setpgrp(0, 0)}
depending on which version is implemented (if any).  See the {\UNIX}
manual for the semantics.
(Not on MS-DOS.)
\end{funcdesc}

\begin{funcdesc}{setpgid}{pid\, pgrp}
Calls the system call \code{setpgid()}.  See the {\UNIX} manual for
the semantics.
(Not on MS-DOS.)
\end{funcdesc}

\begin{funcdesc}{setsid}{}
Calls the system call \code{setsid()}.  See the {\UNIX} manual for the
semantics.
(Not on MS-DOS.)
\end{funcdesc}

\begin{funcdesc}{setuid}{uid}
Set the current process's user id.
(Not on MS-DOS.)
\end{funcdesc}

\begin{funcdesc}{stat}{path}
Perform a {\em stat} system call on the given path.  The return value
is a tuple of at least 10 integers giving the most important (and
portable) members of the {\em stat} structure, in the order
\code{st_mode},
\code{st_ino},
\code{st_dev},
\code{st_nlink},
\code{st_uid},
\code{st_gid},
\code{st_size},
\code{st_atime},
\code{st_mtime},
\code{st_ctime}.
More items may be added at the end by some implementations.
(On MS-DOS, some items are filled with dummy values.)

Note: The standard module \code{stat} defines functions and constants
that are useful for extracting information from a stat structure.
\end{funcdesc}

\begin{funcdesc}{symlink}{src\, dst}
Create a symbolic link pointing to \var{src} named \var{dst}.  (On
systems without symbolic links, this always raises
\code{posix.error}.)
\end{funcdesc}

\begin{funcdesc}{system}{command}
Execute the command (a string) in a subshell.  This is implemented by
calling the Standard C function \code{system()}, and has the same
limitations.  Changes to \code{posix.environ}, \code{sys.stdin} etc.\ are
not reflected in the environment of the executed command.  The return
value is the exit status of the process as returned by Standard C
\code{system()}.
\end{funcdesc}

\begin{funcdesc}{tcgetpgrp}{fd}
Return the process group associated with the terminal given by
\var{fd} (an open file descriptor as returned by \code{posix.open()}).
(Not on MS-DOS.)
\end{funcdesc}

\begin{funcdesc}{tcsetpgrp}{fd\, pg}
Set the process group associated with the terminal given by
\var{fd} (an open file descriptor as returned by \code{posix.open()})
to \var{pg}.
(Not on MS-DOS.)
\end{funcdesc}

\begin{funcdesc}{times}{}
Return a 5-tuple of floating point numbers indicating accumulated (CPU
or other)
times, in seconds.  The items are: user time, system time, children's
user time, children's system time, and elapsed real time since a fixed
point in the past, in that order.  See the \UNIX{}
manual page {\it times}(2).  (Not on MS-DOS.)
\end{funcdesc}

\begin{funcdesc}{umask}{mask}
Set the current numeric umask and returns the previous umask.
(Not on MS-DOS.)
\end{funcdesc}

\begin{funcdesc}{uname}{}
Return a 5-tuple containing information identifying the current
operating system.  The tuple contains 5 strings:
\code{(\var{sysname}, \var{nodename}, \var{release}, \var{version}, \var{machine})}.
Some systems truncate the nodename to 8
characters or to the leading component; a better way to get the
hostname is \code{socket.gethostname()}.  (Not on MS-DOS, nor on older
\UNIX{} systems.)
\end{funcdesc}

\begin{funcdesc}{unlink}{path}
Remove the file \var{path}.  This is the same function as \code{remove};
the \code{unlink} name is its traditional \UNIX{} name.
\end{funcdesc}

\begin{funcdesc}{utime}{path\, \(atime\, mtime\)}
Set the access and modified time of the file to the given values.
(The second argument is a tuple of two items.)
\end{funcdesc}

\begin{funcdesc}{wait}{}
Wait for completion of a child process, and return a tuple containing
its pid and exit status indication (encoded as by \UNIX{}).
(Not on MS-DOS.)
\end{funcdesc}

\begin{funcdesc}{waitpid}{pid\, options}
Wait for completion of a child process given by proces id, and return
a tuple containing its pid and exit status indication (encoded as by
\UNIX{}).  The semantics of the call are affected by the value of
the integer options, which should be 0 for normal operation.  (If the
system does not support \code{waitpid()}, this always raises
\code{posix.error}.  Not on MS-DOS.)
\end{funcdesc}

\begin{funcdesc}{write}{fd\, str}
Write the string \var{str} to file descriptor \var{fd}.
Return the number of bytes actually written.

Note: this function is intended for low-level I/O and must be applied
to a file descriptor as returned by \code{posix.open()} or
\code{posix.pipe()}.  To write a ``file object'' returned by the
built-in function \code{open} or by \code{posix.popen} or
\code{posix.fdopen}, or \code{sys.stdout} or \code{sys.stderr}, use
its \code{write()} method.
\end{funcdesc}

\begin{datadesc}{WNOHANG}
The option for \code{waitpid()} to avoid hanging if no child process
status is available immediately.
\end{datadesc}

\section{Standard Module \sectcode{posixpath}}
\stmodindex{posixpath}

This module implements some useful functions on POSIX pathnames.

\strong{Do not import this module directly.}  Instead, import the
module \code{os} and use \code{os.path}.
\stmodindex{os}

\renewcommand{\indexsubitem}{(in module posixpath)}

\begin{funcdesc}{basename}{p}
Return the base name of pathname
\var{p}.
This is the second half of the pair returned by
\code{posixpath.split(\var{p})}.
\end{funcdesc}

\begin{funcdesc}{commonprefix}{list}
Return the longest string that is a prefix of all strings in
\var{list}.
If
\var{list}
is empty, return the empty string (\code{''}).
\end{funcdesc}

\begin{funcdesc}{exists}{p}
Return true if
\var{p}
refers to an existing path.
\end{funcdesc}

\begin{funcdesc}{expanduser}{p}
Return the argument with an initial component of \samp{\~} or
\samp{\~\var{user}} replaced by that \var{user}'s home directory.  An
initial \samp{\~{}} is replaced by the environment variable \code{\${}HOME};
an initial \samp{\~\var{user}} is looked up in the password directory through
the built-in module \code{pwd}.  If the expansion fails, or if the
path does not begin with a tilde, the path is returned unchanged.
\end{funcdesc}

\begin{funcdesc}{expandvars}{p}
Return the argument with environment variables expanded.  Substrings
of the form \samp{\$\var{name}} or \samp{\$\{\var{name}\}} are
replaced by the value of environment variable \var{name}.  Malformed
variable names and references to non-existing variables are left
unchanged.
\end{funcdesc}

\begin{funcdesc}{isabs}{p}
Return true if \var{p} is an absolute pathname (begins with a slash).
\end{funcdesc}

\begin{funcdesc}{isfile}{p}
Return true if \var{p} is an existing regular file.  This follows
symbolic links, so both \code{islink()} and \code{isfile()} can be true for the same
path.
\end{funcdesc}

\begin{funcdesc}{isdir}{p}
Return true if \var{p} is an existing directory.  This follows
symbolic links, so both \code{islink()} and \code{isdir()} can be true for the same
path.
\end{funcdesc}

\begin{funcdesc}{islink}{p}
Return true if
\var{p}
refers to a directory entry that is a symbolic link.
Always false if symbolic links are not supported.
\end{funcdesc}

\begin{funcdesc}{ismount}{p}
Return true if pathname \var{p} is a \dfn{mount point}: a point in a
file system where a different file system has been mounted.  The
function checks whether \var{p}'s parent, \file{\var{p}/..}, is on a
different device than \var{p}, or whether \file{\var{p}/..} and
\var{p} point to the same i-node on the same device --- this should
detect mount points for all \UNIX{} and POSIX variants.
\end{funcdesc}

\begin{funcdesc}{join}{p\, q}
Join the paths
\var{p}
and
\var{q} intelligently:
If
\var{q}
is an absolute path, the return value is
\var{q}.
Otherwise, the concatenation of
\var{p}
and
\var{q}
is returned, with a slash (\code{'/'}) inserted unless
\var{p}
is empty or ends in a slash.
\end{funcdesc}

\begin{funcdesc}{normcase}{p}
Normalize the case of a pathname.  This returns the path unchanged;
however, a similar function in \code{macpath} converts upper case to
lower case.
\end{funcdesc}

\begin{funcdesc}{samefile}{p\, q}
Return true if both pathname arguments refer to the same file or directory
(as indicated by device number and i-node number).
Raise an exception if a stat call on either pathname fails.
\end{funcdesc}

\begin{funcdesc}{split}{p}
Split the pathname \var{p} in a pair \code{(\var{head}, \var{tail})},
where \var{tail} is the last pathname component and \var{head} is
everything leading up to that.  The \var{tail} part will never contain
a slash; if \var{p} ends in a slash, \var{tail} will be empty.  If
there is no slash in \var{p}, \var{head} will be empty.  If \var{p} is
empty, both \var{head} and \var{tail} are empty.  Trailing slashes are
stripped from \var{head} unless it is the root (one or more slashes
only).  In nearly all cases, \code{join(\var{head}, \var{tail})}
equals \var{p} (the only exception being when there were multiple
slashes separating \var{head} from \var{tail}).
\end{funcdesc}

\begin{funcdesc}{splitext}{p}
Split the pathname \var{p} in a pair \code{(\var{root}, \var{ext})}
such that \code{\var{root} + \var{ext} == \var{p}},
and \var{ext} is empty or begins with a period and contains
at most one period.
\end{funcdesc}

\begin{funcdesc}{walk}{p\, visit\, arg}
Calls the function \var{visit} with arguments
\code{(\var{arg}, \var{dirname}, \var{names})} for each directory in the
directory tree rooted at \var{p} (including \var{p} itself, if it is a
directory).  The argument \var{dirname} specifies the visited directory,
the argument \var{names} lists the files in the directory (gotten from
\code{posix.listdir(\var{dirname})}, so including \samp{.} and
\samp{..}).  The \var{visit} function may modify \var{names} to
influence the set of directories visited below \var{dirname}, e.g., to
avoid visiting certain parts of the tree.  (The object referred to by
\var{names} must be modified in place, using \code{del} or slice
assignment.)
\end{funcdesc}
		% == posixpath
\section{Built-in Module \sectcode{pwd}}

\bimodindex{pwd}
This module provides access to the \UNIX{} password database.
It is available on all \UNIX{} versions.

Password database entries are reported as 7-tuples containing the
following items from the password database (see \file{<pwd.h>}), in order:
\code{pw_name},
\code{pw_passwd},
\code{pw_uid},
\code{pw_gid},
\code{pw_gecos},
\code{pw_dir},
\code{pw_shell}.
The uid and gid items are integers, all others are strings.
An exception is raised if the entry asked for cannot be found.

It defines the following items:

\renewcommand{\indexsubitem}{(in module pwd)}
\begin{funcdesc}{getpwuid}{uid}
Return the password database entry for the given numeric user ID.
\end{funcdesc}

\begin{funcdesc}{getpwnam}{name}
Return the password database entry for the given user name.
\end{funcdesc}

\begin{funcdesc}{getpwall}{}
Return a list of all available password database entries, in arbitrary order.
\end{funcdesc}

\section{Built-in Module \sectcode{grp}}

\bimodindex{grp}
This module provides access to the \UNIX{} group database.
It is available on all \UNIX{} versions.

Group database entries are reported as 4-tuples containing the
following items from the group database (see \file{<grp.h>}), in order:
\code{gr_name},
\code{gr_passwd},
\code{gr_gid},
\code{gr_mem}.
The gid is an integer, name and password are strings, and the member
list is a list of strings.
(Note that most users are not explicitly listed as members of the
group they are in according to the password database.)
An exception is raised if the entry asked for cannot be found.

It defines the following items:

\renewcommand{\indexsubitem}{(in module grp)}
\begin{funcdesc}{getgrgid}{gid}
Return the group database entry for the given numeric group ID.
\end{funcdesc}

\begin{funcdesc}{getgrnam}{name}
Return the group database entry for the given group name.
\end{funcdesc}

\begin{funcdesc}{getgrall}{}
Return a list of all available group entries, in arbitrary order.
\end{funcdesc}

\section{Built-in module {\tt crypt}}
\bimodindex{crypt}

This module implements an interface to the crypt({\bf 3}) routine,
which is a one-way hash function based upon a modified DES algorithm;
see the Unix man page for further details.  Possible uses include
allowing Python scripts to accept typed passwords from the user, or
attempting to crack Unix passwords with a dictionary.
\index{crypt(3)}

\begin{funcdesc}{crypt}{word\, salt} 
\var{word} will usually be a user's password.  \var{salt} is a
2-character string which will be used to select one of 4096 variations
of DES.  The characters in \var{salt} must be either \code{.},
\code{/}, or an alphanumeric character.  Returns the hashed password
as a string, which will be composed of characters from the same
alphabet as the salt.
\end{funcdesc}

The module and documentation were written by Steve Majewski.
\index{Majewski, Steve}

\section{Built-in Module \sectcode{dbm}}
\bimodindex{dbm}

The \code{dbm} module provides an interface to the {\UNIX}
\code{(n)dbm} library.  Dbm objects behave like mappings
(dictionaries), except that keys and values are always strings.
Printing a dbm object doesn't print the keys and values, and the
\code{items()} and \code{values()} methods are not supported.

See also the \code{gdbm} module, which provides a similar interface
using the GNU GDBM library.
\bimodindex{gdbm}

The module defines the following constant and functions:

\renewcommand{\indexsubitem}{(in module dbm)}
\begin{excdesc}{error}
Raised on dbm-specific errors, such as I/O errors. \code{KeyError} is
raised for general mapping errors like specifying an incorrect key.
\end{excdesc}

\begin{funcdesc}{open}{filename\, \optional{flag\, \optional{mode}}}
Open a dbm database and return a dbm object.  The \var{filename}
argument is the name of the database file (without the \file{.dir} or
\file{.pag} extensions).

The optional \var{flag} argument can be
\code{'r'} (to open an existing database for reading only --- default),
\code{'w'} (to open an existing database for reading and writing),
\code{'c'} (which creates the database if it doesn't exist), or
\code{'n'} (which always creates a new empty database).

The optional \var{mode} argument is the \UNIX{} mode of the file, used
only when the database has to be created.  It defaults to octal
\code{0666}.
\end{funcdesc}

\section{Built-in Module \sectcode{gdbm}}
\bimodindex{gdbm}

This module is nearly identical to the \code{dbm} module, but uses
GDBM instead.  Its interface is identical, and not repeated here.

Warning: the file formats created by gdbm and dbm are incompatible.
\bimodindex{dbm}

\section{Built-in Module \sectcode{termios}}
\bimodindex{termios}
\indexii{Posix}{I/O control}
\indexii{tty}{I/O control}

\renewcommand{\indexsubitem}{(in module termios)}

This module provides an interface to the Posix calls for tty I/O
control.  For a complete description of these calls, see the Posix or
\UNIX{} manual pages.  It is only available for those \UNIX{} versions
that support Posix \code{termios} style tty I/O control (and then
only if configured at installation time).

All functions in this module take a file descriptor \var{fd} as their
first argument.  This must be an integer file descriptor, such as
returned by \code{sys.stdin.fileno()}.

This module should be used in conjunction with the \code{TERMIOS}
module, which defines the relevant symbolic constants (see the next
section).

The module defines the following functions:

\begin{funcdesc}{tcgetattr}{fd}
Return a list containing the tty attributes for file descriptor
\var{fd}, as follows: \code{[\var{iflag}, \var{oflag}, \var{cflag},
\var{lflag}, \var{ispeed}, \var{ospeed}, \var{cc}]} where \var{cc} is
a list of the tty special characters (each a string of length 1,
except the items with indices \code{VMIN} and \code{VTIME}, which are
integers when these fields are defined).  The interpretation of the
flags and the speeds as well as the indexing in the \var{cc} array
must be done using the symbolic constants defined in the
\code{TERMIOS} module.
\end{funcdesc}

\begin{funcdesc}{tcsetattr}{fd\, when\, attributes}
Set the tty attributes for file descriptor \var{fd} from the
\var{attributes}, which is a list like the one returned by
\code{tcgetattr()}.  The \var{when} argument determines when the
attributes are changed: \code{TERMIOS.TCSANOW} to change immediately,
\code{TERMIOS.TCSADRAIN} to change after transmitting all queued
output, or \code{TERMIOS.TCSAFLUSH} to change after transmitting all
queued output and discarding all queued input.
\end{funcdesc}

\begin{funcdesc}{tcsendbreak}{fd\, duration}
Send a break on file descriptor \var{fd}.  A zero \var{duration} sends
a break for 0.25--0.5 seconds; a nonzero \var{duration} has a system
dependent meaning.
\end{funcdesc}

\begin{funcdesc}{tcdrain}{fd}
Wait until all output written to file descriptor \var{fd} has been
transmitted.
\end{funcdesc}

\begin{funcdesc}{tcflush}{fd\, queue}
Discard queued data on file descriptor \var{fd}.  The \var{queue}
selector specifies which queue: \code{TERMIOS.TCIFLUSH} for the input
queue, \code{TERMIOS.TCOFLUSH} for the output queue, or
\code{TERMIOS.TCIOFLUSH} for both queues.
\end{funcdesc}

\begin{funcdesc}{tcflow}{fd\, action}
Suspend or resume input or output on file descriptor \var{fd}.  The
\var{action} argument can be \code{TERMIOS.TCOOFF} to suspend output,
\code{TERMIOS.TCOON} to restart output, \code{TERMIOS.TCIOFF} to
suspend input, or \code{TERMIOS.TCION} to restart input.
\end{funcdesc}

\subsection{Example}
\nodename{termios Example}

Here's a function that prompts for a password with echoing turned off.
Note the technique using a separate \code{termios.tcgetattr()} call
and a \code{try {\ldots} finally} statement to ensure that the old tty
attributes are restored exactly no matter what happens:

\begin{verbatim}
def getpass(prompt = "Password: "):
    import termios, TERMIOS, sys
    fd = sys.stdin.fileno()
    old = termios.tcgetattr(fd)
    new = termios.tcgetattr(fd)
    new[3] = new[3] & ~TERMIOS.ECHO          # lflags
    try:
        termios.tcsetattr(fd, TERMIOS.TCSADRAIN, new)
        passwd = raw_input(prompt)
    finally:
        termios.tcsetattr(fd, TERMIOS.TCSADRAIN, old)
    return passwd
\end{verbatim}


\section{Standard Module \sectcode{TERMIOS}}
\stmodindex{TERMIOS}
\indexii{Posix}{I/O control}
\indexii{tty}{I/O control}

\renewcommand{\indexsubitem}{(in module TERMIOS)}

This module defines the symbolic constants required to use the
\code{termios} module (see the previous section).  See the Posix or
\UNIX{} manual pages (or the source) for a list of those constants.

Note: this module resides in a system-dependent subdirectory of the
Python library directory.  You may have to generate it for your
particular system using the script \file{Tools/scripts/h2py.py}.

% Manual text by Jaap Vermeulen
\section{Built-in Module \sectcode{fcntl}}
\bimodindex{fcntl}
\indexii{\UNIX{}}{file control}
\indexii{\UNIX{}}{I/O control}

This module performs file control and I/O control on file descriptors.
It is an interface to the \dfn{fcntl()} and \dfn{ioctl()} \UNIX{} routines.
File descriptors can be obtained with the \dfn{fileno()} method of a
file or socket object.

The module defines the following functions:

\renewcommand{\indexsubitem}{(in module struct)}

\begin{funcdesc}{fcntl}{fd\, op\optional{\, arg}}
  Perform the requested operation on file descriptor \code{\var{fd}}.
  The operation is defined by \code{\var{op}} and is operating system
  dependent.  Typically these codes can be retrieved from the library
  module \code{FCNTL}. The argument \code{\var{arg}} is optional, and
  defaults to the integer value \code{0}.  When
  it is present, it can either be an integer value, or a string.  With
  the argument missing or an integer value, the return value of this
  function is the integer return value of the real \code{fcntl()}
  call.  When the argument is a string it represents a binary
  structure, e.g.\ created by \code{struct.pack()}. The binary data is
  copied to a buffer whose address is passed to the real \code{fcntl()}
  call.  The return value after a successful call is the contents of
  the buffer, converted to a string object.  In case the
  \code{fcntl()} fails, an \code{IOError} will be raised.
\end{funcdesc}

\begin{funcdesc}{ioctl}{fd\, op\, arg}
  This function is identical to the \code{fcntl()} function, except
  that the operations are typically defined in the library module
  \code{IOCTL}.
\end{funcdesc}

\begin{funcdesc}{flock}{fd\, op}
Perform the lock operation \var{op} on file descriptor \var{fd}.
See the Unix manual for details.  (On some systems, this function is
emulated using \code{fcntl}.)
\end{funcdesc}

\begin{funcdesc}{lockf}{fd\, code\, \optional{len\, \optional{start\, \optional{whence}}}}
This is a wrapper around the \code{F_SETLK} and \code{F_SETLKW}
\code{fcntl()} calls.  See the Unix manual for details.
\end{funcdesc}

If the library modules \code{FCNTL} or \code{IOCTL} are missing, you
can find the opcodes in the C include files \code{sys/fcntl} and
\code{sys/ioctl}. You can create the modules yourself with the h2py
script, found in the \code{Tools/scripts} directory.
\stmodindex{FCNTL}
\stmodindex{IOCTL}

Examples (all on a SVR4 compliant system):

\bcode\begin{verbatim}
import struct, FCNTL

file = open(...)
rv = fcntl(file.fileno(), FCNTL.O_NDELAY, 1)

lockdata = struct.pack('hhllhh', FCNTL.F_WRLCK, 0, 0, 0, 0, 0)
rv = fcntl(file.fileno(), FCNTL.F_SETLKW, lockdata)
\end{verbatim}\ecode

Note that in the first example the return value variable \code{rv} will
hold an integer value; in the second example it will hold a string
value.  The structure lay-out for the \var{lockadata} variable is
system dependent -- therefore using the \code{flock()} call may be
better.

% Manual text and implementation by Jaap Vermeulen
\section{Standard Module \sectcode{posixfile}}
\bimodindex{posixfile}
\indexii{posix}{file object}

\emph{Note:} This module will become obsolete in a future release.
The locking operation that it provides is done better and more
portably by the \code{fcntl.lockf()} call.

This module implements some additional functionality over the built-in
file objects.  In particular, it implements file locking, control over
the file flags, and an easy interface to duplicate the file object.
The module defines a new file object, the posixfile object.  It
has all the standard file object methods and adds the methods
described below.  This module only works for certain flavors of
\UNIX{}, since it uses \code{fcntl()} for file locking.

To instantiate a posixfile object, use the \code{open()} function in
the posixfile module.  The resulting object looks and feels roughly
the same as a standard file object.

The posixfile module defines the following constants:

\renewcommand{\indexsubitem}{(in module posixfile)}
\begin{datadesc}{SEEK_SET}
offset is calculated from the start of the file
\end{datadesc}

\begin{datadesc}{SEEK_CUR}
offset is calculated from the current position in the file
\end{datadesc}

\begin{datadesc}{SEEK_END}
offset is calculated from the end of the file
\end{datadesc}

The posixfile module defines the following functions:

\renewcommand{\indexsubitem}{(in module posixfile)}

\begin{funcdesc}{open}{filename\optional{\, mode\optional{\, bufsize}}}
 Create a new posixfile object with the given filename and mode.  The
 \var{filename}, \var{mode} and \var{bufsize} arguments are
 interpreted the same way as by the built-in \code{open()} function.
\end{funcdesc}

\begin{funcdesc}{fileopen}{fileobject}
 Create a new posixfile object with the given standard file object.
 The resulting object has the same filename and mode as the original
 file object.
\end{funcdesc}

The posixfile object defines the following additional methods:

\renewcommand{\indexsubitem}{(posixfile method)}
\begin{funcdesc}{lock}{fmt\, \optional{len\optional{\, start
\optional{\, whence}}}}
 Lock the specified section of the file that the file object is
 referring to.  The format is explained
 below in a table.  The \var{len} argument specifies the length of the
 section that should be locked. The default is \code{0}. \var{start}
 specifies the starting offset of the section, where the default is
 \code{0}.  The \var{whence} argument specifies where the offset is
 relative to. It accepts one of the constants \code{SEEK_SET},
 \code{SEEK_CUR} or \code{SEEK_END}.  The default is \code{SEEK_SET}.
 For more information about the arguments refer to the fcntl
 manual page on your system.
\end{funcdesc}

\begin{funcdesc}{flags}{\optional{flags}}
 Set the specified flags for the file that the file object is referring
 to.  The new flags are ORed with the old flags, unless specified
 otherwise.  The format is explained below in a table.  Without
 the \var{flags} argument
 a string indicating the current flags is returned (this is
 the same as the '?' modifier).  For more information about the flags
 refer to the fcntl manual page on your system.
\end{funcdesc}

\begin{funcdesc}{dup}{}
 Duplicate the file object and the underlying file pointer and file
 descriptor.  The resulting object behaves as if it were newly
 opened.
\end{funcdesc}

\begin{funcdesc}{dup2}{fd}
 Duplicate the file object and the underlying file pointer and file
 descriptor.  The new object will have the given file descriptor.
 Otherwise the resulting object behaves as if it were newly opened.
\end{funcdesc}

\begin{funcdesc}{file}{}
 Return the standard file object that the posixfile object is based
 on.  This is sometimes necessary for functions that insist on a
 standard file object.
\end{funcdesc}

All methods return \code{IOError} when the request fails.

Format characters for the \code{lock()} method have the following meaning:

\begin{tableiii}{|c|l|c|}{samp}{Format}{Meaning}{}
  \lineiii{u}{unlock the specified region}{}
  \lineiii{r}{request a read lock for the specified section}{}
  \lineiii{w}{request a write lock for the specified section}{}
\end{tableiii}

In addition the following modifiers can be added to the format:

\begin{tableiii}{|c|l|c|}{samp}{Modifier}{Meaning}{Notes}
  \lineiii{|}{wait until the lock has been granted}{}
  \lineiii{?}{return the first lock conflicting with the requested lock, or
              \code{None} if there is no conflict.}{(1)} 
\end{tableiii}

Note:

(1) The lock returned is in the format \code{(mode, len, start,
whence, pid)} where mode is a character representing the type of lock
('r' or 'w').  This modifier prevents a request from being granted; it
is for query purposes only.

Format character for the \code{flags()} method have the following meaning:

\begin{tableiii}{|c|l|c|}{samp}{Format}{Meaning}{}
  \lineiii{a}{append only flag}{}
  \lineiii{c}{close on exec flag}{}
  \lineiii{n}{no delay flag (also called non-blocking flag)}{}
  \lineiii{s}{synchronization flag}{}
\end{tableiii}

In addition the following modifiers can be added to the format:

\begin{tableiii}{|c|l|c|}{samp}{Modifier}{Meaning}{Notes}
  \lineiii{!}{turn the specified flags 'off', instead of the default 'on'}{(1)}
  \lineiii{=}{replace the flags, instead of the default 'OR' operation}{(1)}
  \lineiii{?}{return a string in which the characters represent the flags that
  are set.}{(2)}
\end{tableiii}

Note:

(1) The \code{!} and \code{=} modifiers are mutually exclusive.

(2) This string represents the flags after they may have been altered
by the same call.

Examples:

\bcode\begin{verbatim}
from posixfile import *

file = open('/tmp/test', 'w')
file.lock('w|')
...
file.lock('u')
file.close()
\end{verbatim}\ecode

\section{Built-in Module \sectcode{syslog}}
\bimodindex{syslog}

This module provides an interface to the Unix \code{syslog} library
routines.  Refer to the \UNIX{} manual pages for a detailed description
of the \code{syslog} facility.

The module defines the following functions:

\begin{funcdesc}{syslog}{\optional{priority\,} message}
Send the string \var{message} to the system logger.
A trailing newline is added if necessary.
Each message is tagged with a priority composed of a \var{facility} and
a \var{level}.
The optional \var{priority} argument, which defaults to
\code{(LOG_USER | LOG_INFO)}, determines the message priority.
\end{funcdesc}

\begin{funcdesc}{openlog}{ident\, \optional{logopt\, \optional{facility}}}
Logging options other than the defaults can be set by explicitly opening
the log file with \code{openlog()} prior to calling \code{syslog()}.
The defaults are (usually) \var{ident} = \samp{syslog}, \var{logopt} = 0,
\var{facility} = \code{LOG_USER}.
The \var{ident} argument is a string which is prepended to every message.
The optional \var{logopt} argument is a bit field - see below for possible
values to combine.
The optional \var{facility} argument sets the default facility for messages
which do not have a facility explicitly encoded.
\end{funcdesc}

\begin{funcdesc}{closelog}{}
Close the log file.
\end{funcdesc}

\begin{funcdesc}{setlogmask}{maskpri}
This function set the priority mask to \var{maskpri} and returns the
previous mask value.
Calls to \code{syslog} with a priority level not set in \var{maskpri}
are ignored.
The default is to log all priorities.
The function \code{LOG_MASK(\var{pri})} calculates the mask for the
individual priority \var{pri}.
The function \code{LOG_UPTO(\var{pri})} calculates the mask for all priorities
up to and including \var{pri}.
\end{funcdesc}

The module defines the following constants:

\begin{description}

\item[Priority levels (high to low):]

\code{LOG_EMERG}, \code{LOG_ALERT}, \code{LOG_CRIT}, \code{LOG_ERR},
\code{LOG_WARNING}, \code{LOG_NOTICE}, \code{LOG_INFO}, \code{LOG_DEBUG}.

\item[Facilities:]

\code{LOG_KERN}, \code{LOG_USER}, \code{LOG_MAIL}, \code{LOG_DAEMON},
\code{LOG_AUTH}, \code{LOG_LPR}, \code{LOG_NEWS}, \code{LOG_UUCP},
\code{LOG_CRON} and \code{LOG_LOCAL0} to \code{LOG_LOCAL7}.

\item[Log options:]

\code{LOG_PID}, \code{LOG_CONS}, \code{LOG_NDELAY}, \code{LOG_NOWAIT}
and \code{LOG_PERROR} if defined in \file{syslog.h}.

\end{description}


\chapter{The Python Debugger}
\stmodindex{pdb}
\index{debugging}

\renewcommand{\indexsubitem}{(in module pdb)}

The module \code{pdb} defines an interactive source code debugger for
Python programs.  It supports setting breakpoints and single stepping
at the source line level, inspection of stack frames, source code
listing, and evaluation of arbitrary Python code in the context of any
stack frame.  It also supports post-mortem debugging and can be called
under program control.

The debugger is extensible --- it is actually defined as a class
\code{Pdb}.  This is currently undocumented but easily understood by
reading the source.  The extension interface uses the (also
undocumented) modules \code{bdb} and \code{cmd}.
\ttindex{Pdb}
\ttindex{bdb}
\ttindex{cmd}

A primitive windowing version of the debugger also exists --- this is
module \code{wdb}, which requires STDWIN (see the chapter on STDWIN
specific modules).
\index{stdwin}
\ttindex{wdb}

The debugger's prompt is ``\code{(Pdb) }''.
Typical usage to run a program under control of the debugger is:

\begin{verbatim}
>>> import pdb
>>> import mymodule
>>> pdb.run('mymodule.test()')
> <string>(0)?()
(Pdb) continue
> <string>(1)?()
(Pdb) continue
NameError: 'spam'
> <string>(1)?()
(Pdb) 
\end{verbatim}

Typical usage to inspect a crashed program is:

\begin{verbatim}
>>> import pdb
>>> import mymodule
>>> mymodule.test()
Traceback (innermost last):
  File "<stdin>", line 1, in ?
  File "./mymodule.py", line 4, in test
    test2()
  File "./mymodule.py", line 3, in test2
    print spam
NameError: spam
>>> pdb.pm()
> ./mymodule.py(3)test2()
-> print spam
(Pdb) 
\end{verbatim}

The module defines the following functions; each enters the debugger
in a slightly different way:

\begin{funcdesc}{run}{statement\optional{\, globals\optional{\, locals}}}
Execute the \var{statement} (given as a string) under debugger
control.  The debugger prompt appears before any code is executed; you
can set breakpoints and type \code{continue}, or you can step through
the statement using \code{step} or \code{next} (all these commands are
explained below).  The optional \var{globals} and \var{locals}
arguments specify the environment in which the code is executed; by
default the dictionary of the module \code{__main__} is used.  (See
the explanation of the \code{exec} statement or the \code{eval()}
built-in function.)
\end{funcdesc}

\begin{funcdesc}{runeval}{expression\optional{\, globals\optional{\, locals}}}
Evaluate the \var{expression} (given as a a string) under debugger
control.  When \code{runeval()} returns, it returns the value of the
expression.  Otherwise this function is similar to
\code{run()}.
\end{funcdesc}

\begin{funcdesc}{runcall}{function\optional{\, argument\, ...}}
Call the \var{function} (a function or method object, not a string)
with the given arguments.  When \code{runcall()} returns, it returns
whatever the function call returned.  The debugger prompt appears as
soon as the function is entered.
\end{funcdesc}

\begin{funcdesc}{set_trace}{}
Enter the debugger at the calling stack frame.  This is useful to
hard-code a breakpoint at a given point in a program, even if the code
is not otherwise being debugged (e.g. when an assertion fails).
\end{funcdesc}

\begin{funcdesc}{post_mortem}{traceback}
Enter post-mortem debugging of the given \var{traceback} object.
\end{funcdesc}

\begin{funcdesc}{pm}{}
Enter post-mortem debugging of the traceback found in
\code{sys.last_traceback}.
\end{funcdesc}

\section{Debugger Commands}

The debugger recognizes the following commands.  Most commands can be
abbreviated to one or two letters; e.g. ``\code{h(elp)}'' means that
either ``\code{h}'' or ``\code{help}'' can be used to enter the help
command (but not ``\code{he}'' or ``\code{hel}'', nor ``\code{H}'' or
``\code{Help} or ``\code{HELP}'').  Arguments to commands must be
separated by whitespace (spaces or tabs).  Optional arguments are
enclosed in square brackets (``\code{[]}'') in the command syntax; the
square brackets must not be typed.  Alternatives in the command syntax
are separated by a vertical bar (``\code{|}'').

Entering a blank line repeats the last command entered.  Exception: if
the last command was a ``\code{list}'' command, the next 11 lines are
listed.

Commands that the debugger doesn't recognize are assumed to be Python
statements and are executed in the context of the program being
debugged.  Python statements can also be prefixed with an exclamation
point (``\code{!}'').  This is a powerful way to inspect the program
being debugged; it is even possible to change a variable or call a
function.  When an
exception occurs in such a statement, the exception name is printed
but the debugger's state is not changed.

\begin{description}

\item[h(elp) [\var{command}]]

Without argument, print the list of available commands.
With a \var{command} as argument, print help about that command.
``\code{help pdb}'' displays the full documentation file; if the
environment variable \code{PAGER} is defined, the file is piped
through that command instead.  Since the \var{command} argument must be
an identifier, ``\code{help exec}'' must be entered to get help on the
``\code{!}'' command.

\item[w(here)]

Print a stack trace, with the most recent frame at the bottom.
An arrow indicates the current frame, which determines the
context of most commands.

\item[d(own)]

Move the current frame one level down in the stack trace
(to an older frame).

\item[u(p)]

Move the current frame one level up in the stack trace
(to a newer frame).

\item[b(reak) [\var{lineno}\code{|}\var{function}]]

With a \var{lineno} argument, set a break there in the current
file.  With a \var{function} argument, set a break at the entry of
that function.  Without argument, list all breaks.

\item[cl(ear) [\var{lineno}]]

With a \var{lineno} argument, clear that break in the current file.
Without argument, clear all breaks (but first ask confirmation).

\item[s(tep)]

Execute the current line, stop at the first possible occasion
(either in a function that is called or on the next line in the
current function).

\item[n(ext)]

Continue execution until the next line in the current function
is reached or it returns.  (The difference between \code{next} and
\code{step} is that \code{step} stops inside a called function, while
\code{next} executes called functions at (nearly) full speed, only
stopping at the next line in the current function.)

\item[r(eturn)]

Continue execution until the current function returns.

\item[c(ont(inue))]

Continue execution, only stop when a breakpoint is encountered.

\item[l(ist) [\var{first} [, \var{last}]]]

List source code for the current file.  Without arguments, list 11
lines around the current line or continue the previous listing.  With
one argument, list 11 lines around at that line.  With two arguments,
list the given range; if the second argument is less than the first,
it is interpreted as a count.

\item[a(rgs)]

Print the argument list of the current function.

\item[p \var{expression}]

Evaluate the \var{expression} in the current context and print its
value.  (Note: \code{print} can also be used, but is not a debugger
command --- this executes the Python \code{print} statement.)

\item[[!] \var{statement}]

Execute the (one-line) \var{statement} in the context of
the current stack frame.
The exclamation point can be omitted unless the first word
of the statement resembles a debugger command.
To set a global variable, you can prefix the assignment
command with a ``\code{global}'' command on the same line, e.g.:
\begin{verbatim}
(Pdb) global list_options; list_options = ['-l']
(Pdb)
\end{verbatim}

\item[q(uit)]

Quit from the debugger.
The program being executed is aborted.

\end{description}

\section{How It Works}

Some changes were made to the interpreter:

\begin{itemize}
\item sys.settrace(func) sets the global trace function
\item there can also a local trace function (see later)
\end{itemize}

Trace functions have three arguments: (\var{frame}, \var{event}, \var{arg})

\begin{description}

\item[\var{frame}] is the current stack frame

\item[\var{event}] is a string: \code{'call'}, \code{'line'}, \code{'return'}
or \code{'exception'}

\item[\var{arg}] is dependent on the event type

\end{description}

A trace function should return a new trace function or None.
Class methods are accepted (and most useful!) as trace methods.

The events have the following meaning:

\begin{description}

\item[\code{'call'}]
A function is called (or some other code block entered).  The global
trace function is called; arg is the argument list to the function;
the return value specifies the local trace function.

\item[\code{'line'}]
The interpreter is about to execute a new line of code (sometimes
multiple line events on one line exist).  The local trace function is
called; arg in None; the return value specifies the new local trace
function.

\item[\code{'return'}]
A function (or other code block) is about to return.  The local trace
function is called; arg is the value that will be returned.  The trace
function's return value is ignored.

\item[\code{'exception'}]
An exception has occurred.  The local trace function is called; arg is
a triple (exception, value, traceback); the return value specifies the
new local trace function

\end{description}

Note that as an exception is propagated down the chain of callers, an
\code{'exception'} event is generated at each level.

Stack frame objects have the following read-only attributes:

\begin{description}
\item[f_code]      the code object being executed
\item[f_lineno]    the current line number (\code{-1} for \code{'call'} events)
\item[f_back]      the stack frame of the caller, or None
\item[f_locals]    dictionary containing local name bindings
\item[f_globals]   dictionary containing global name bindings
\end{description}

Code objects have the following read-only attributes:

\begin{description}
\item[co_code]     the code string
\item[co_names]    the list of names used by the code
\item[co_consts]   the list of (literal) constants used by the code
\item[co_filename] the filename from which the code was compiled
\end{description}
			% The Python Debugger

\chapter{The Python Profiler}
\stmodindex{profile}
\stmodindex{pstats}

Copyright \copyright\ 1994, by InfoSeek Corporation, all rights reserved.

Written by James Roskind%
\footnote{
Updated and converted to \LaTeX\ by Guido van Rossum.  The references to
the old profiler are left in the text, although it no longer exists.
}

Permission to use, copy, modify, and distribute this Python software
and its associated documentation for any purpose (subject to the
restriction in the following sentence) without fee is hereby granted,
provided that the above copyright notice appears in all copies, and
that both that copyright notice and this permission notice appear in
supporting documentation, and that the name of InfoSeek not be used in
advertising or publicity pertaining to distribution of the software
without specific, written prior permission.  This permission is
explicitly restricted to the copying and modification of the software
to remain in Python, compiled Python, or other languages (such as C)
wherein the modified or derived code is exclusively imported into a
Python module.

INFOSEEK CORPORATION DISCLAIMS ALL WARRANTIES WITH REGARD TO THIS
SOFTWARE, INCLUDING ALL IMPLIED WARRANTIES OF MERCHANTABILITY AND
FITNESS. IN NO EVENT SHALL INFOSEEK CORPORATION BE LIABLE FOR ANY
SPECIAL, INDIRECT OR CONSEQUENTIAL DAMAGES OR ANY DAMAGES WHATSOEVER
RESULTING FROM LOSS OF USE, DATA OR PROFITS, WHETHER IN AN ACTION OF
CONTRACT, NEGLIGENCE OR OTHER TORTIOUS ACTION, ARISING OUT OF OR IN
CONNECTION WITH THE USE OR PERFORMANCE OF THIS SOFTWARE.


The profiler was written after only programming in Python for 3 weeks.
As a result, it is probably clumsy code, but I don't know for sure yet
'cause I'm a beginner :-).  I did work hard to make the code run fast,
so that profiling would be a reasonable thing to do.  I tried not to
repeat code fragments, but I'm sure I did some stuff in really awkward
ways at times.  Please send suggestions for improvements to:
\code{jar@netscape.com}.  I won't promise \emph{any} support.  ...but
I'd appreciate the feedback.


\section{Introduction to the profiler}
\nodename{Profiler Introduction}

A \dfn{profiler} is a program that describes the run time performance
of a program, providing a variety of statistics.  This documentation
describes the profiler functionality provided in the modules
\code{profile} and \code{pstats.}  This profiler provides
\dfn{deterministic profiling} of any Python programs.  It also
provides a series of report generation tools to allow users to rapidly
examine the results of a profile operation.


\section{How Is This Profiler Different From The Old Profiler?}
\nodename{Profiler Changes}

The big changes from old profiling module are that you get more
information, and you pay less CPU time.  It's not a trade-off, it's a
trade-up.

To be specific:

\begin{description}

\item[Bugs removed:]
Local stack frame is no longer molested, execution time is now charged
to correct functions.

\item[Accuracy increased:]
Profiler execution time is no longer charged to user's code,
calibration for platform is supported, file reads are not done \emph{by}
profiler \emph{during} profiling (and charged to user's code!).

\item[Speed increased:]
Overhead CPU cost was reduced by more than a factor of two (perhaps a
factor of five), lightweight profiler module is all that must be
loaded, and the report generating module (\code{pstats}) is not needed
during profiling.

\item[Recursive functions support:]
Cumulative times in recursive functions are correctly calculated;
recursive entries are counted.

\item[Large growth in report generating UI:]
Distinct profiles runs can be added together forming a comprehensive
report; functions that import statistics take arbitrary lists of
files; sorting criteria is now based on keywords (instead of 4 integer
options); reports shows what functions were profiled as well as what
profile file was referenced; output format has been improved.

\end{description}


\section{Instant Users Manual}

This section is provided for users that ``don't want to read the
manual.'' It provides a very brief overview, and allows a user to
rapidly perform profiling on an existing application.

To profile an application with a main entry point of \samp{foo()}, you
would add the following to your module:

\begin{verbatim}
    import profile
    profile.run("foo()")
\end{verbatim}

The above action would cause \samp{foo()} to be run, and a series of
informative lines (the profile) to be printed.  The above approach is
most useful when working with the interpreter.  If you would like to
save the results of a profile into a file for later examination, you
can supply a file name as the second argument to the \code{run()}
function:

\begin{verbatim}
    import profile
    profile.run("foo()", 'fooprof')
\end{verbatim}

When you wish to review the profile, you should use the methods in the
\code{pstats} module.  Typically you would load the statistics data as
follows:

\begin{verbatim}
    import pstats
    p = pstats.Stats('fooprof')
\end{verbatim}

The class \code{Stats} (the above code just created an instance of
this class) has a variety of methods for manipulating and printing the
data that was just read into \samp{p}.  When you ran
\code{profile.run()} above, what was printed was the result of three
method calls:

\begin{verbatim}
    p.strip_dirs().sort_stats(-1).print_stats()
\end{verbatim}

The first method removed the extraneous path from all the module
names. The second method sorted all the entries according to the
standard module/line/name string that is printed (this is to comply
with the semantics of the old profiler).  The third method printed out
all the statistics.  You might try the following sort calls:

\begin{verbatim}
    p.sort_stats('name')
    p.print_stats()
\end{verbatim}

The first call will actually sort the list by function name, and the
second call will print out the statistics.  The following are some
interesting calls to experiment with:

\begin{verbatim}
    p.sort_stats('cumulative').print_stats(10)
\end{verbatim}

This sorts the profile by cumulative time in a function, and then only
prints the ten most significant lines.  If you want to understand what
algorithms are taking time, the above line is what you would use.

If you were looking to see what functions were looping a lot, and
taking a lot of time, you would do:

\begin{verbatim}
    p.sort_stats('time').print_stats(10)
\end{verbatim}

to sort according to time spent within each function, and then print
the statistics for the top ten functions.

You might also try:

\begin{verbatim}
    p.sort_stats('file').print_stats('__init__')
\end{verbatim}

This will sort all the statistics by file name, and then print out
statistics for only the class init methods ('cause they are spelled
with \code{__init__} in them).  As one final example, you could try:

\begin{verbatim}
    p.sort_stats('time', 'cum').print_stats(.5, 'init')
\end{verbatim}

This line sorts statistics with a primary key of time, and a secondary
key of cumulative time, and then prints out some of the statistics.
To be specific, the list is first culled down to 50\% (re: \samp{.5})
of its original size, then only lines containing \code{init} are
maintained, and that sub-sub-list is printed.

If you wondered what functions called the above functions, you could
now (\samp{p} is still sorted according to the last criteria) do:

\begin{verbatim}
    p.print_callers(.5, 'init')
\end{verbatim}

and you would get a list of callers for each of the listed functions. 

If you want more functionality, you're going to have to read the
manual, or guess what the following functions do:

\begin{verbatim}
    p.print_callees()
    p.add('fooprof')
\end{verbatim}


\section{What Is Deterministic Profiling?}
\nodename{Deterministic Profiling}

\dfn{Deterministic profiling} is meant to reflect the fact that all
\dfn{function call}, \dfn{function return}, and \dfn{exception} events
are monitored, and precise timings are made for the intervals between
these events (during which time the user's code is executing).  In
contrast, \dfn{statistical profiling} (which is not done by this
module) randomly samples the effective instruction pointer, and
deduces where time is being spent.  The latter technique traditionally
involves less overhead (as the code does not need to be instrumented),
but provides only relative indications of where time is being spent.

In Python, since there is an interpreter active during execution, the
presence of instrumented code is not required to do deterministic
profiling.  Python automatically provides a \dfn{hook} (optional
callback) for each event.  In addition, the interpreted nature of
Python tends to add so much overhead to execution, that deterministic
profiling tends to only add small processing overhead in typical
applications.  The result is that deterministic profiling is not that
expensive, yet provides extensive run time statistics about the
execution of a Python program.

Call count statistics can be used to identify bugs in code (surprising
counts), and to identify possible inline-expansion points (high call
counts).  Internal time statistics can be used to identify ``hot
loops'' that should be carefully optimized.  Cumulative time
statistics should be used to identify high level errors in the
selection of algorithms.  Note that the unusual handling of cumulative
times in this profiler allows statistics for recursive implementations
of algorithms to be directly compared to iterative implementations.


\section{Reference Manual}

\renewcommand{\indexsubitem}{(profiler function)}

The primary entry point for the profiler is the global function
\code{profile.run()}.  It is typically used to create any profile
information.  The reports are formatted and printed using methods of
the class \code{pstats.Stats}.  The following is a description of all
of these standard entry points and functions.  For a more in-depth
view of some of the code, consider reading the later section on
Profiler Extensions, which includes discussion of how to derive
``better'' profilers from the classes presented, or reading the source
code for these modules.

\begin{funcdesc}{profile.run}{string\optional{\, filename\optional{\, ...}}}

This function takes a single argument that has can be passed to the
\code{exec} statement, and an optional file name.  In all cases this
routine attempts to \code{exec} its first argument, and gather profiling
statistics from the execution. If no file name is present, then this
function automatically prints a simple profiling report, sorted by the
standard name string (file/line/function-name) that is presented in
each line.  The following is a typical output from such a call:

\small{
\begin{verbatim}
      main()
      2706 function calls (2004 primitive calls) in 4.504 CPU seconds

Ordered by: standard name

ncalls  tottime  percall  cumtime  percall filename:lineno(function)
     2    0.006    0.003    0.953    0.477 pobject.py:75(save_objects)
  43/3    0.533    0.012    0.749    0.250 pobject.py:99(evaluate)
 ...
\end{verbatim}
}

The first line indicates that this profile was generated by the call:\\
\code{profile.run('main()')}, and hence the exec'ed string is
\code{'main()'}.  The second line indicates that 2706 calls were
monitored.  Of those calls, 2004 were \dfn{primitive}.  We define
\dfn{primitive} to mean that the call was not induced via recursion.
The next line: \code{Ordered by:\ standard name}, indicates that
the text string in the far right column was used to sort the output.
The column headings include:

\begin{description}

\item[ncalls ]
for the number of calls, 

\item[tottime ]
for the total time spent in the given function (and excluding time
made in calls to sub-functions),

\item[percall ]
is the quotient of \code{tottime} divided by \code{ncalls}

\item[cumtime ]
is the total time spent in this and all subfunctions (i.e., from
invocation till exit). This figure is accurate \emph{even} for recursive
functions.

\item[percall ]
is the quotient of \code{cumtime} divided by primitive calls

\item[filename:lineno(function) ]
provides the respective data of each function

\end{description}

When there are two numbers in the first column (e.g.: \samp{43/3}),
then the latter is the number of primitive calls, and the former is
the actual number of calls.  Note that when the function does not
recurse, these two values are the same, and only the single figure is
printed.

\end{funcdesc}

\begin{funcdesc}{pstats.Stats}{filename\optional{\, ...}}
This class constructor creates an instance of a ``statistics object''
from a \var{filename} (or set of filenames).  \code{Stats} objects are
manipulated by methods, in order to print useful reports.

The file selected by the above constructor must have been created by
the corresponding version of \code{profile}.  To be specific, there is
\emph{NO} file compatibility guaranteed with future versions of this
profiler, and there is no compatibility with files produced by other
profilers (e.g., the old system profiler).

If several files are provided, all the statistics for identical
functions will be coalesced, so that an overall view of several
processes can be considered in a single report.  If additional files
need to be combined with data in an existing \code{Stats} object, the
\code{add()} method can be used.
\end{funcdesc}


\subsection{The \sectcode{Stats} Class}

\renewcommand{\indexsubitem}{(Stats method)}

\begin{funcdesc}{strip_dirs}{}
This method for the \code{Stats} class removes all leading path information
from file names.  It is very useful in reducing the size of the
printout to fit within (close to) 80 columns.  This method modifies
the object, and the stripped information is lost.  After performing a
strip operation, the object is considered to have its entries in a
``random'' order, as it was just after object initialization and
loading.  If \code{strip_dirs()} causes two function names to be
indistinguishable (i.e., they are on the same line of the same
filename, and have the same function name), then the statistics for
these two entries are accumulated into a single entry.
\end{funcdesc}


\begin{funcdesc}{add}{filename\optional{\, ...}}
This method of the \code{Stats} class accumulates additional profiling
information into the current profiling object.  Its arguments should
refer to filenames created by the corresponding version of
\code{profile.run()}.  Statistics for identically named (re: file,
line, name) functions are automatically accumulated into single
function statistics.
\end{funcdesc}

\begin{funcdesc}{sort_stats}{key\optional{\, ...}}
This method modifies the \code{Stats} object by sorting it according to the
supplied criteria.  The argument is typically a string identifying the
basis of a sort (example: \code{"time"} or \code{"name"}).

When more than one key is provided, then additional keys are used as
secondary criteria when the there is equality in all keys selected
before them.  For example, sort_stats('name', 'file') will sort all
the entries according to their function name, and resolve all ties
(identical function names) by sorting by file name.

Abbreviations can be used for any key names, as long as the
abbreviation is unambiguous.  The following are the keys currently
defined: 

\begin{tableii}{|l|l|}{code}{Valid Arg}{Meaning}
\lineii{"calls"}{call count}
\lineii{"cumulative"}{cumulative time}
\lineii{"file"}{file name}
\lineii{"module"}{file name}
\lineii{"pcalls"}{primitive call count}
\lineii{"line"}{line number}
\lineii{"name"}{function name}
\lineii{"nfl"}{name/file/line}
\lineii{"stdname"}{standard name}
\lineii{"time"}{internal time}
\end{tableii}

Note that all sorts on statistics are in descending order (placing
most time consuming items first), where as name, file, and line number
searches are in ascending order (i.e., alphabetical). The subtle
distinction between \code{"nfl"} and \code{"stdname"} is that the
standard name is a sort of the name as printed, which means that the
embedded line numbers get compared in an odd way.  For example, lines
3, 20, and 40 would (if the file names were the same) appear in the
string order 20, 3 and 40.  In contrast, \code{"nfl"} does a numeric
compare of the line numbers.  In fact, \code{sort_stats("nfl")} is the
same as \code{sort_stats("name", "file", "line")}.

For compatibility with the old profiler, the numeric arguments
\samp{-1}, \samp{0}, \samp{1}, and \samp{2} are permitted.  They are
interpreted as \code{"stdname"}, \code{"calls"}, \code{"time"}, and
\code{"cumulative"} respectively.  If this old style format (numeric)
is used, only one sort key (the numeric key) will be used, and
additional arguments will be silently ignored.
\end{funcdesc}


\begin{funcdesc}{reverse_order}{}
This method for the \code{Stats} class reverses the ordering of the basic
list within the object.  This method is provided primarily for
compatibility with the old profiler.  Its utility is questionable
now that ascending vs descending order is properly selected based on
the sort key of choice.
\end{funcdesc}

\begin{funcdesc}{print_stats}{restriction\optional{\, ...}}
This method for the \code{Stats} class prints out a report as described
in the \code{profile.run()} definition.

The order of the printing is based on the last \code{sort_stats()}
operation done on the object (subject to caveats in \code{add()} and
\code{strip_dirs())}.

The arguments provided (if any) can be used to limit the list down to
the significant entries.  Initially, the list is taken to be the
complete set of profiled functions.  Each restriction is either an
integer (to select a count of lines), or a decimal fraction between
0.0 and 1.0 inclusive (to select a percentage of lines), or a regular
expression (to pattern match the standard name that is printed).  If
several restrictions are provided, then they are applied sequentially.
For example:

\begin{verbatim}
    print_stats(.1, "foo:")
\end{verbatim}

would first limit the printing to first 10\% of list, and then only
print functions that were part of filename \samp{.*foo:}.  In
contrast, the command:

\begin{verbatim}
    print_stats("foo:", .1)
\end{verbatim}

would limit the list to all functions having file names \samp{.*foo:},
and then proceed to only print the first 10\% of them.
\end{funcdesc}


\begin{funcdesc}{print_callers}{restrictions\optional{\, ...}}
This method for the \code{Stats} class prints a list of all functions
that called each function in the profiled database.  The ordering is
identical to that provided by \code{print_stats()}, and the definition
of the restricting argument is also identical.  For convenience, a
number is shown in parentheses after each caller to show how many
times this specific call was made.  A second non-parenthesized number
is the cumulative time spent in the function at the right.
\end{funcdesc}

\begin{funcdesc}{print_callees}{restrictions\optional{\, ...}}
This method for the \code{Stats} class prints a list of all function
that were called by the indicated function.  Aside from this reversal
of direction of calls (re: called vs was called by), the arguments and
ordering are identical to the \code{print_callers()} method.
\end{funcdesc}

\begin{funcdesc}{ignore}{}
This method of the \code{Stats} class is used to dispose of the value
returned by earlier methods.  All standard methods in this class
return the instance that is being processed, so that the commands can
be strung together.  For example:

\begin{verbatim}
pstats.Stats('foofile').strip_dirs().sort_stats('cum') \
                       .print_stats().ignore()
\end{verbatim}

would perform all the indicated functions, but it would not return
the final reference to the \code{Stats} instance.%
\footnote{
This was once necessary, when Python would print any unused expression
result that was not \code{None}.  The method is still defined for
backward compatibility.
}
\end{funcdesc}


\section{Limitations}

There are two fundamental limitations on this profiler.  The first is
that it relies on the Python interpreter to dispatch \dfn{call},
\dfn{return}, and \dfn{exception} events.  Compiled C code does not
get interpreted, and hence is ``invisible'' to the profiler.  All time
spent in C code (including builtin functions) will be charged to the
Python function that invoked the C code.  If the C code calls out
to some native Python code, then those calls will be profiled
properly.

The second limitation has to do with accuracy of timing information.
There is a fundamental problem with deterministic profilers involving
accuracy.  The most obvious restriction is that the underlying ``clock''
is only ticking at a rate (typically) of about .001 seconds.  Hence no
measurements will be more accurate that that underlying clock.  If
enough measurements are taken, then the ``error'' will tend to average
out. Unfortunately, removing this first error induces a second source
of error...

The second problem is that it ``takes a while'' from when an event is
dispatched until the profiler's call to get the time actually
\emph{gets} the state of the clock.  Similarly, there is a certain lag
when exiting the profiler event handler from the time that the clock's
value was obtained (and then squirreled away), until the user's code
is once again executing.  As a result, functions that are called many
times, or call many functions, will typically accumulate this error.
The error that accumulates in this fashion is typically less than the
accuracy of the clock (i.e., less than one clock tick), but it
\emph{can} accumulate and become very significant.  This profiler
provides a means of calibrating itself for a given platform so that
this error can be probabilistically (i.e., on the average) removed.
After the profiler is calibrated, it will be more accurate (in a least
square sense), but it will sometimes produce negative numbers (when
call counts are exceptionally low, and the gods of probability work
against you :-). )  Do \emph{NOT} be alarmed by negative numbers in
the profile.  They should \emph{only} appear if you have calibrated
your profiler, and the results are actually better than without
calibration.


\section{Calibration}

The profiler class has a hard coded constant that is added to each
event handling time to compensate for the overhead of calling the time
function, and socking away the results.  The following procedure can
be used to obtain this constant for a given platform (see discussion
in section Limitations above).

\begin{verbatim}
    import profile
    pr = profile.Profile()
    pr.calibrate(100)
    pr.calibrate(100)
    pr.calibrate(100)
\end{verbatim}

The argument to calibrate() is the number of times to try to do the
sample calls to get the CPU times.  If your computer is \emph{very}
fast, you might have to do:

\begin{verbatim}
    pr.calibrate(1000)
\end{verbatim}

or even:

\begin{verbatim}
    pr.calibrate(10000)
\end{verbatim}

The object of this exercise is to get a fairly consistent result.
When you have a consistent answer, you are ready to use that number in
the source code.  For a Sun Sparcstation 1000 running Solaris 2.3, the
magical number is about .00053.  If you have a choice, you are better
off with a smaller constant, and your results will ``less often'' show
up as negative in profile statistics.

The following shows how the trace_dispatch() method in the Profile
class should be modified to install the calibration constant on a Sun
Sparcstation 1000:

\begin{verbatim}
    def trace_dispatch(self, frame, event, arg):
        t = self.timer()
        t = t[0] + t[1] - self.t - .00053 # Calibration constant

        if self.dispatch[event](frame,t):
            t = self.timer()
            self.t = t[0] + t[1]
        else:
            r = self.timer()
            self.t = r[0] + r[1] - t # put back unrecorded delta
        return
\end{verbatim}

Note that if there is no calibration constant, then the line
containing the callibration constant should simply say:

\begin{verbatim}
        t = t[0] + t[1] - self.t  # no calibration constant
\end{verbatim}

You can also achieve the same results using a derived class (and the
profiler will actually run equally fast!!), but the above method is
the simplest to use.  I could have made the profiler ``self
calibrating'', but it would have made the initialization of the
profiler class slower, and would have required some \emph{very} fancy
coding, or else the use of a variable where the constant \samp{.00053}
was placed in the code shown.  This is a \strong{VERY} critical
performance section, and there is no reason to use a variable lookup
at this point, when a constant can be used.


\section{Extensions --- Deriving Better Profilers}
\nodename{Profiler Extensions}

The \code{Profile} class of module \code{profile} was written so that
derived classes could be developed to extend the profiler.  Rather
than describing all the details of such an effort, I'll just present
the following two examples of derived classes that can be used to do
profiling.  If the reader is an avid Python programmer, then it should
be possible to use these as a model and create similar (and perchance
better) profile classes.

If all you want to do is change how the timer is called, or which
timer function is used, then the basic class has an option for that in
the constructor for the class.  Consider passing the name of a
function to call into the constructor:

\begin{verbatim}
    pr = profile.Profile(your_time_func)
\end{verbatim}

The resulting profiler will call \code{your_time_func()} instead of
\code{os.times()}.  The function should return either a single number
or a list of numbers (like what \code{os.times()} returns).  If the
function returns a single time number, or the list of returned numbers
has length 2, then you will get an especially fast version of the
dispatch routine.

Be warned that you \emph{should} calibrate the profiler class for the
timer function that you choose.  For most machines, a timer that
returns a lone integer value will provide the best results in terms of
low overhead during profiling.  (os.times is \emph{pretty} bad, 'cause
it returns a tuple of floating point values, so all arithmetic is
floating point in the profiler!).  If you want to substitute a
better timer in the cleanest fashion, you should derive a class, and
simply put in the replacement dispatch method that better handles your
timer call, along with the appropriate calibration constant :-).


\subsection{OldProfile Class}

The following derived profiler simulates the old style profiler,
providing errant results on recursive functions. The reason for the
usefulness of this profiler is that it runs faster (i.e., less
overhead) than the old profiler.  It still creates all the caller
stats, and is quite useful when there is \emph{no} recursion in the
user's code.  It is also a lot more accurate than the old profiler, as
it does not charge all its overhead time to the user's code.

\begin{verbatim}
class OldProfile(Profile):

    def trace_dispatch_exception(self, frame, t):
        rt, rtt, rct, rfn, rframe, rcur = self.cur
        if rcur and not rframe is frame:
            return self.trace_dispatch_return(rframe, t)
        return 0

    def trace_dispatch_call(self, frame, t):
        fn = `frame.f_code`
        
        self.cur = (t, 0, 0, fn, frame, self.cur)
        if self.timings.has_key(fn):
            tt, ct, callers = self.timings[fn]
            self.timings[fn] = tt, ct, callers
        else:
            self.timings[fn] = 0, 0, {}
        return 1

    def trace_dispatch_return(self, frame, t):
        rt, rtt, rct, rfn, frame, rcur = self.cur
        rtt = rtt + t
        sft = rtt + rct

        pt, ptt, pct, pfn, pframe, pcur = rcur
        self.cur = pt, ptt+rt, pct+sft, pfn, pframe, pcur

        tt, ct, callers = self.timings[rfn]
        if callers.has_key(pfn):
            callers[pfn] = callers[pfn] + 1
        else:
            callers[pfn] = 1
        self.timings[rfn] = tt+rtt, ct + sft, callers

        return 1


    def snapshot_stats(self):
        self.stats = {}
        for func in self.timings.keys():
            tt, ct, callers = self.timings[func]
            nor_func = self.func_normalize(func)
            nor_callers = {}
            nc = 0
            for func_caller in callers.keys():
                nor_callers[self.func_normalize(func_caller)]=\
                      callers[func_caller]
                nc = nc + callers[func_caller]
            self.stats[nor_func] = nc, nc, tt, ct, nor_callers
\end{verbatim}
        

\subsection{HotProfile Class}

This profiler is the fastest derived profile example.  It does not
calculate caller-callee relationships, and does not calculate
cumulative time under a function.  It only calculates time spent in a
function, so it runs very quickly (re: very low overhead).  In truth,
the basic profiler is so fast, that is probably not worth the savings
to give up the data, but this class still provides a nice example.

\begin{verbatim}
class HotProfile(Profile):

    def trace_dispatch_exception(self, frame, t):
        rt, rtt, rfn, rframe, rcur = self.cur
        if rcur and not rframe is frame:
            return self.trace_dispatch_return(rframe, t)
        return 0

    def trace_dispatch_call(self, frame, t):
        self.cur = (t, 0, frame, self.cur)
        return 1

    def trace_dispatch_return(self, frame, t):
        rt, rtt, frame, rcur = self.cur

        rfn = `frame.f_code`

        pt, ptt, pframe, pcur = rcur
        self.cur = pt, ptt+rt, pframe, pcur

        if self.timings.has_key(rfn):
            nc, tt = self.timings[rfn]
            self.timings[rfn] = nc + 1, rt + rtt + tt
        else:
            self.timings[rfn] =      1, rt + rtt

        return 1


    def snapshot_stats(self):
        self.stats = {}
        for func in self.timings.keys():
            nc, tt = self.timings[func]
            nor_func = self.func_normalize(func)
            self.stats[nor_func] = nc, nc, tt, 0, {}
\end{verbatim}
		% The Python Profiler

\chapter{Internet and WWW Services}
\nodename{Internet and WWW}
\index{WWW}
\index{Internet}
\index{World-Wide Web}

The modules described in this chapter provide various services to
World-Wide Web (WWW) clients and/or services, and a few modules
related to news and email.  They are all implemented in Python.  Some
of these modules require the presence of the system-dependent module
\code{sockets}, which is currently only fully supported on Unix and
Windows NT.  Here is an overview:

\begin{description}

\item[cgi]
--- Common Gateway Interface, used to interpret forms in server-side
scripts.

\item[urllib]
--- Open an arbitrary object given by URL (requires sockets).

\item[httplib]
--- HTTP protocol client (requires sockets).

\item[ftplib]
--- FTP protocol client (requires sockets).

\item[gopherlib]
--- Gopher protocol client (requires sockets).

\item[nntplib]
--- NNTP protocol client (requires sockets).

\item[urlparse]
--- Parse a URL string into a tuple (addressing scheme identifier, network
location, path, parameters, query string, fragment identifier).

\item[sgmllib]
--- Only as much of an SGML parser as needed to parse HTML.

\item[htmllib]
--- A (slow) parser for HTML documents.

\item[formatter]
--- Generic output formatter and device interface.

\item[rfc822]
--- Parse RFC-822 style mail headers.

\item[mimetools]
--- Tools for parsing MIME style message bodies.

\end{description}
			% Internet and WWW Services
\section{Standard Module \sectcode{cgi}}
\stmodindex{cgi}
\indexii{WWW}{server}
\indexii{CGI}{protocol}
\indexii{HTTP}{protocol}
\indexii{MIME}{headers}
\index{URL}

\renewcommand{\indexsubitem}{(in module cgi)}

Support module for CGI (Common Gateway Interface) scripts.

This module defines a number of utilities for use by CGI scripts
written in Python.

\subsection{Introduction}
\nodename{Introduction to the CGI module}

A CGI script is invoked by an HTTP server, usually to process user
input submitted through an HTML \code{<FORM>} or \code{<ISINPUT>} element.

Most often, CGI scripts live in the server's special \code{cgi-bin}
directory.  The HTTP server places all sorts of information about the
request (such as the client's hostname, the requested URL, the query
string, and lots of other goodies) in the script's shell environment,
executes the script, and sends the script's output back to the client.

The script's input is connected to the client too, and sometimes the
form data is read this way; at other times the form data is passed via
the ``query string'' part of the URL.  This module (\code{cgi.py}) is intended
to take care of the different cases and provide a simpler interface to
the Python script.  It also provides a number of utilities that help
in debugging scripts, and the latest addition is support for file
uploads from a form (if your browser supports it -- Grail 0.3 and
Netscape 2.0 do).

The output of a CGI script should consist of two sections, separated
by a blank line.  The first section contains a number of headers,
telling the client what kind of data is following.  Python code to
generate a minimal header section looks like this:

\begin{verbatim}
	print "Content-type: text/html"	# HTML is following
	print				# blank line, end of headers
\end{verbatim}

The second section is usually HTML, which allows the client software
to display nicely formatted text with header, in-line images, etc.
Here's Python code that prints a simple piece of HTML:

\begin{verbatim}
	print "<TITLE>CGI script output</TITLE>"
	print "<H1>This is my first CGI script</H1>"
	print "Hello, world!"
\end{verbatim}

(It may not be fully legal HTML according to the letter of the
standard, but any browser will understand it.)

\subsection{Using the cgi module}
\nodename{Using the cgi module}

Begin by writing \code{import cgi}.  Don't use \code{from cgi import *} -- the
module defines all sorts of names for its own use or for backward 
compatibility that you don't want in your namespace.

It's best to use the \code{FieldStorage} class.  The other classes define in this 
module are provided mostly for backward compatibility.  Instantiate it 
exactly once, without arguments.  This reads the form contents from 
standard input or the environment (depending on the value of various 
environment variables set according to the CGI standard).  Since it may 
consume standard input, it should be instantiated only once.

The \code{FieldStorage} instance can be accessed as if it were a Python 
dictionary.  For instance, the following code (which assumes that the 
\code{Content-type} header and blank line have already been printed) checks that 
the fields \code{name} and \code{addr} are both set to a non-empty string:

\begin{verbatim}
	form = cgi.FieldStorage()
	form_ok = 0
	if form.has_key("name") and form.has_key("addr"):
		if form["name"].value != "" and form["addr"].value != "":
			form_ok = 1
	if not form_ok:
		print "<H1>Error</H1>"
		print "Please fill in the name and addr fields."
		return
	...further form processing here...
\end{verbatim}

Here the fields, accessed through \code{form[key]}, are themselves instances
of \code{FieldStorage} (or \code{MiniFieldStorage}, depending on the form encoding).

If the submitted form data contains more than one field with the same
name, the object retrieved by \code{form[key]} is not a \code{(Mini)FieldStorage}
instance but a list of such instances.  If you expect this possibility
(i.e., when your HTML form comtains multiple fields with the same
name), use the \code{type()} function to determine whether you have a single
instance or a list of instances.  For example, here's code that
concatenates any number of username fields, separated by commas:

\begin{verbatim}
	username = form["username"]
	if type(username) is type([]):
		# Multiple username fields specified
		usernames = ""
		for item in username:
			if usernames:
				# Next item -- insert comma
				usernames = usernames + "," + item.value
			else:
				# First item -- don't insert comma
				usernames = item.value
	else:
		# Single username field specified
		usernames = username.value
\end{verbatim}

If a field represents an uploaded file, the value attribute reads the 
entire file in memory as a string.  This may not be what you want.  You can 
test for an uploaded file by testing either the filename attribute or the 
file attribute.  You can then read the data at leasure from the file 
attribute:

\begin{verbatim}
	fileitem = form["userfile"]
	if fileitem.file:
		# It's an uploaded file; count lines
		linecount = 0
		while 1:
			line = fileitem.file.readline()
			if not line: break
			linecount = linecount + 1
\end{verbatim}

The file upload draft standard entertains the possibility of uploading
multiple files from one field (using a recursive \code{multipart/*}
encoding).  When this occurs, the item will be a dictionary-like
FieldStorage item.  This can be determined by testing its type
attribute, which should have the value \code{multipart/form-data} (or
perhaps another string beginning with \code{multipart/}  It this case, it
can be iterated over recursively just like the top-level form object.

When a form is submitted in the ``old'' format (as the query string or as a 
single data part of type \code{application/x-www-form-urlencoded}), the items 
will actually be instances of the class \code{MiniFieldStorage}.  In this case,
the list, file and filename attributes are always \code{None}.


\subsection{Old classes}

These classes, present in earlier versions of the \code{cgi} module, are still 
supported for backward compatibility.  New applications should use the
FieldStorage class.

\code{SvFormContentDict}: single value form content as dictionary; assumes each 
field name occurs in the form only once.

\code{FormContentDict}: multiple value form content as dictionary (the form
items are lists of values).  Useful if your form contains multiple
fields with the same name.

Other classes (\code{FormContent}, \code{InterpFormContentDict}) are present for
backwards compatibility with really old applications only.  If you still 
use these and would be inconvenienced when they disappeared from a next 
version of this module, drop me a note.


\subsection{Functions}

These are useful if you want more control, or if you want to employ
some of the algorithms implemented in this module in other
circumstances.

\begin{funcdesc}{parse}{fp}: Parse a query in the environment or from a file (default \code{sys.stdin}).
\end{funcdesc}

\begin{funcdesc}{parse_qs}{qs}: parse a query string given as a string argument (data of type 
\code{application/x-www-form-urlencoded}).
\end{funcdesc}

\begin{funcdesc}{parse_multipart}{fp\, pdict}: parse input of type \code{multipart/form-data} (for 
file uploads).  Arguments are \code{fp} for the input file and 
    \code{pdict} for the dictionary containing other parameters of \code{content-type} header

    Returns a dictionary just like \code{parse_qs()}: keys are the field names, each 
    value is a list of values for that field.  This is easy to use but not 
    much good if you are expecting megabytes to be uploaded -- in that case, 
    use the \code{FieldStorage} class instead which is much more flexible.  Note 
    that \code{content-type} is the raw, unparsed contents of the \code{content-type} 
    header.

    Note that this does not parse nested multipart parts -- use \code{FieldStorage} for 
    that.
\end{funcdesc}

\begin{funcdesc}{parse_header}{string}: parse a header like \code{Content-type} into a main
content-type and a dictionary of parameters.
\end{funcdesc}

\begin{funcdesc}{test}{}: robust test CGI script, usable as main program.
    Writes minimal HTTP headers and formats all information provided to
    the script in HTML form.
\end{funcdesc}

\begin{funcdesc}{print_environ}{}: format the shell environment in HTML.
\end{funcdesc}

\begin{funcdesc}{print_form}{form}: format a form in HTML.
\end{funcdesc}

\begin{funcdesc}{print_directory}{}: format the current directory in HTML.
\end{funcdesc}

\begin{funcdesc}{print_environ_usage}{}: print a list of useful (used by CGI) environment variables in
HTML.
\end{funcdesc}

\begin{funcdesc}{escape}{}: convert the characters ``\code{\&}'', ``\code{<}'' and ``\code{>}'' to HTML-safe
sequences.  Use this if you need to display text that might contain
such characters in HTML.  To translate URLs for inclusion in the HREF
attribute of an \code{<A>} tag, use \code{urllib.quote()}.
\end{funcdesc}


\subsection{Caring about security}

There's one important rule: if you invoke an external program (e.g.
via the \code{os.system()} or \code{os.popen()} functions), make very sure you don't
pass arbitrary strings received from the client to the shell.  This is
a well-known security hole whereby clever hackers anywhere on the web
can exploit a gullible CGI script to invoke arbitrary shell commands.
Even parts of the URL or field names cannot be trusted, since the
request doesn't have to come from your form!

To be on the safe side, if you must pass a string gotten from a form
to a shell command, you should make sure the string contains only
alphanumeric characters, dashes, underscores, and periods.


\subsection{Installing your CGI script on a Unix system}

Read the documentation for your HTTP server and check with your local
system administrator to find the directory where CGI scripts should be
installed; usually this is in a directory \code{cgi-bin} in the server tree.

Make sure that your script is readable and executable by ``others''; the
Unix file mode should be 755 (use \code{chmod 755 filename}).  Make sure
that the first line of the script contains \code{\#!} starting in column 1
followed by the pathname of the Python interpreter, for instance:

\begin{verbatim}
	#!/usr/local/bin/python
\end{verbatim}

Make sure the Python interpreter exists and is executable by ``others''.

Make sure that any files your script needs to read or write are
readable or writable, respectively, by ``others'' -- their mode should
be 644 for readable and 666 for writable.  This is because, for
security reasons, the HTTP server executes your script as user
``nobody'', without any special privileges.  It can only read (write,
execute) files that everybody can read (write, execute).  The current
directory at execution time is also different (it is usually the
server's cgi-bin directory) and the set of environment variables is
also different from what you get at login.  in particular, don't count
on the shell's search path for executables (\code{\$PATH}) or the Python
module search path (\code{\$PYTHONPATH}) to be set to anything interesting.

If you need to load modules from a directory which is not on Python's
default module search path, you can change the path in your script,
before importing other modules, e.g.:

\begin{verbatim}
	import sys
	sys.path.insert(0, "/usr/home/joe/lib/python")
	sys.path.insert(0, "/usr/local/lib/python")
\end{verbatim}

(This way, the directory inserted last will be searched first!)

Instructions for non-Unix systems will vary; check your HTTP server's
documentation (it will usually have a section on CGI scripts).


\subsection{Testing your CGI script}

Unfortunately, a CGI script will generally not run when you try it
from the command line, and a script that works perfectly from the
command line may fail mysteriously when run from the server.  There's
one reason why you should still test your script from the command
line: if it contains a syntax error, the python interpreter won't
execute it at all, and the HTTP server will most likely send a cryptic
error to the client.

Assuming your script has no syntax errors, yet it does not work, you
have no choice but to read the next section:


\subsection{Debugging CGI scripts}

First of all, check for trivial installation errors -- reading the
section above on installing your CGI script carefully can save you a
lot of time.  If you wonder whether you have understood the
installation procedure correctly, try installing a copy of this module
file (\code{cgi.py}) as a CGI script.  When invoked as a script, the file
will dump its environment and the contents of the form in HTML form.
Give it the right mode etc, and send it a request.  If it's installed
in the standard \code{cgi-bin} directory, it should be possible to send it a
request by entering a URL into your browser of the form:

\begin{verbatim}
	http://yourhostname/cgi-bin/cgi.py?name=Joe+Blow&addr=At+Home
\end{verbatim}

If this gives an error of type 404, the server cannot find the script
-- perhaps you need to install it in a different directory.  If it
gives another error (e.g.  500), there's an installation problem that
you should fix before trying to go any further.  If you get a nicely
formatted listing of the environment and form content (in this
example, the fields should be listed as ``addr'' with value ``At Home''
and ``name'' with value ``Joe Blow''), the \code{cgi.py} script has been
installed correctly.  If you follow the same procedure for your own
script, you should now be able to debug it.

The next step could be to call the \code{cgi} module's test() function from
your script: replace its main code with the single statement

\begin{verbatim}
	cgi.test()
\end{verbatim}
	
This should produce the same results as those gotten from installing
the \code{cgi.py} file itself.

When an ordinary Python script raises an unhandled exception
(e.g. because of a typo in a module name, a file that can't be opened,
etc.), the Python interpreter prints a nice traceback and exits.
While the Python interpreter will still do this when your CGI script
raises an exception, most likely the traceback will end up in one of
the HTTP server's log file, or be discarded altogether.

Fortunately, once you have managed to get your script to execute
*some* code, it is easy to catch exceptions and cause a traceback to
be printed.  The \code{test()} function below in this module is an example.
Here are the rules:

\begin{enumerate}
	\item Import the traceback module (before entering the
	   try-except!)
	
	\item Make sure you finish printing the headers and the blank
	   line early
	
	\item Assign \code{sys.stderr} to \code{sys.stdout}
	
	\item Wrap all remaining code in a try-except statement
	
	\item In the except clause, call \code{traceback.print_exc()}
\end{enumerate}

For example:

\begin{verbatim}
	import sys
	import traceback
	print "Content-type: text/html"
	print
	sys.stderr = sys.stdout
	try:
		...your code here...
	except:
		print "\n\n<PRE>"
		traceback.print_exc()
\end{verbatim}

Notes: The assignment to \code{sys.stderr} is needed because the traceback
prints to \code{sys.stderr}.  The \code{print "$\backslash$n$\backslash$n<PRE>"} statement is necessary to
disable the word wrapping in HTML.

If you suspect that there may be a problem in importing the traceback
module, you can use an even more robust approach (which only uses
built-in modules):

\begin{verbatim}
	import sys
	sys.stderr = sys.stdout
	print "Content-type: text/plain"
	print
	...your code here...
\end{verbatim}

This relies on the Python interpreter to print the traceback.  The
content type of the output is set to plain text, which disables all
HTML processing.  If your script works, the raw HTML will be displayed
by your client.  If it raises an exception, most likely after the
first two lines have been printed, a traceback will be displayed.
Because no HTML interpretation is going on, the traceback will
readable.


\subsection{Common problems and solutions}

\begin{itemize}
\item Most HTTP servers buffer the output from CGI scripts until the
script is completed.  This means that it is not possible to display a
progress report on the client's display while the script is running.

\item Check the installation instructions above.

\item Check the HTTP server's log files.  (\code{tail -f logfile} in a separate
window may be useful!)

\item Always check a script for syntax errors first, by doing something
like \code{python script.py}.

\item When using any of the debugging techniques, don't forget to add
\code{import sys} to the top of the script.

\item When invoking external programs, make sure they can be found.
Usually, this means using absolute path names -- \code{\$PATH} is usually not
set to a very useful value in a CGI script.

\item When reading or writing external files, make sure they can be read
or written by every user on the system.

\item Don't try to give a CGI script a set-uid mode.  This doesn't work on
most systems, and is a security liability as well.
\end{itemize}


\section{Standard Module \sectcode{urllib}}
\stmodindex{urllib}
\index{WWW}
\index{World-Wide Web}
\index{URL}

\renewcommand{\indexsubitem}{(in module urllib)}

This module provides a high-level interface for fetching data across
the World-Wide Web.  In particular, the \code{urlopen} function is
similar to the built-in function \code{open}, but accepts URLs
(Universal Resource Locators) instead of filenames.  Some restrictions
apply --- it can only open URLs for reading, and no seek operations
are available.

it defines the following public functions:

\begin{funcdesc}{urlopen}{url}
Open a network object denoted by a URL for reading.  If the URL does
not have a scheme identifier, or if it has \samp{file:} as its scheme
identifier, this opens a local file; otherwise it opens a socket to a
server somewhere on the network.  If the connection cannot be made, or
if the server returns an error code, the \code{IOError} exception is
raised.  If all went well, a file-like object is returned.  This
supports the following methods: \code{read()}, \code{readline()},
\code{readlines()}, \code{fileno()}, \code{close()} and \code{info()}.
Except for the last one, these methods have the same interface as for
file objects --- see the section on File Objects earlier in this
manual.  (It's not a built-in file object, however, so it can't be
used at those few places where a true built-in file object is
required.)

The \code{info()} method returns an instance of the class
\code{rfc822.Message} containing the headers received from the server,
if the protocol uses such headers (currently the only supported
protocol that uses this is HTTP).  See the description of the
\code{rfc822} module.
\end{funcdesc}

\begin{funcdesc}{urlretrieve}{url}
Copy a network object denoted by a URL to a local file, if necessary.
If the URL points to a local file, or a valid cached copy of the
object exists, the object is not copied.  Return a tuple (\var{filename},
\var{headers}) where \var{filename} is the local file name under which
the object can be found, and \var{headers} is either \code{None} (for
a local object) or whatever the \code{info()} method of the object
returned by \code{urlopen()} returned (for a remote object, possibly
cached).  Exceptions are the same as for \code{urlopen()}.
\end{funcdesc}

\begin{funcdesc}{urlcleanup}{}
Clear the cache that may have been built up by previous calls to
\code{urlretrieve()}.
\end{funcdesc}

\begin{funcdesc}{quote}{string\optional{\, addsafe}}
Replace special characters in \var{string} using the \code{\%xx} escape.
Letters, digits, and the characters ``\code{_,.-}'' are never quoted.
The optional \var{addsafe} parameter specifies additional characters
that should not be quoted --- its default value is \code{'/'}.

Example: \code{quote('/\~conolly/')} yields \code{'/\%7econnolly/'}.
\end{funcdesc}

\begin{funcdesc}{unquote}{string}
Replace \samp{\%xx} escapes by their single-character equivalent.

Example: \code{unquote('/\%7Econnolly/')} yields \code{'/\~connolly/'}.
\end{funcdesc}

Restrictions:

\begin{itemize}

\item
Currently, only the following protocols are supported: HTTP, (versions
0.9 and 1.0), Gopher (but not Gopher-+), FTP, and local files.
\index{HTTP}
\index{Gopher}
\index{FTP}

\item
The caching feature of \code{urlretrieve()} has been disabled until I
find the time to hack proper processing of Expiration time headers.

\item
There should be a function to query whether a particular URL is in
the cache.

\item
For backward compatibility, if a URL appears to point to a local file
but the file can't be opened, the URL is re-interpreted using the FTP
protocol.  This can sometimes cause confusing error messages.

\item
The \code{urlopen()} and \code{urlretrieve()} functions can cause
arbitrarily long delays while waiting for a network connection to be
set up.  This means that it is difficult to build an interactive
web client using these functions without using threads.

\item
The data returned by \code{urlopen()} or \code{urlretrieve()} is the
raw data returned by the server.  This may be binary data (e.g. an
image), plain text or (for example) HTML.  The HTTP protocol provides
type information in the reply header, which can be inspected by
looking at the \code{Content-type} header.  For the Gopher protocol,
type information is encoded in the URL; there is currently no easy way
to extract it.  If the returned data is HTML, you can use the module
\code{htmllib} to parse it.
\index{HTML}
\index{HTTP}
\index{Gopher}
\stmodindex{htmllib}

\item
Although the \code{urllib} module contains (undocumented) routines to
parse and unparse URL strings, the recommended interface for URL
manipulation is in module \code{urlparse}.
\stmodindex{urlparse}

\end{itemize}

\section{Standard Module \sectcode{httplib}}
\stmodindex{httplib}
\index{HTTP}

\renewcommand{\indexsubitem}{(in module httplib)}

This module defines a class which implements the client side of the
HTTP protocol.  It is normally not used directly --- the module
\code{urllib} uses it to handle URLs that use HTTP.
\stmodindex{urllib}

The module defines one class, \code{HTTP}.  An \code{HTTP} instance
represents one transaction with an HTTP server.  It should be
instantiated passing it a host and optional port number.  If no port
number is passed, the port is extracted from the host string if it has
the form \code{host:port}, else the default HTTP port (80) is used.
If no host is passed, no connection is made, and the \code{connect}
method should be used to connect to a server.  For example, the
following calls all create instances that connect to the server at the
same host and port:

\begin{verbatim}
>>> h1 = httplib.HTTP('www.cwi.nl')
>>> h2 = httplib.HTTP('www.cwi.nl:80')
>>> h3 = httplib.HTTP('www.cwi.nl', 80)
\end{verbatim}

Once an \code{HTTP} instance has been connected to an HTTP server, it
should be used as follows:

\begin{enumerate}

\item[1.] Make exactly one call to the \code{putrequest()} method.

\item[2.] Make zero or more calls to the \code{putheader()} method.

\item[3.] Call the \code{endheaders()} method (this can be omitted if
step 4 makes no calls).

\item[4.] Optional calls to the \code{send()} method.

\item[5.] Call the \code{getreply()} method.

\item[6.] Call the \code{getfile()} method and read the data off the
file object that it returns.

\end{enumerate}

\subsection{HTTP Objects}

\code{HTTP} instances have the following methods:

\renewcommand{\indexsubitem}{(HTTP method)}

\begin{funcdesc}{set_debuglevel}{level}
Set the debugging level (the amount of debugging output printed).
The default debug level is \code{0}, meaning no debugging output is
printed.
\end{funcdesc}

\begin{funcdesc}{connect}{host\optional{\, port}}
Connect to the server given by \var{host} and \var{port}.  See the
intro for the default port.  This should be called directly only if
the instance was instantiated without passing a host.
\end{funcdesc}

\begin{funcdesc}{send}{data}
Send data to the server.  This should be used directly only after the
\code{endheaders()} method has been called and before
\code{getreply()} has been called.
\end{funcdesc}

\begin{funcdesc}{putrequest}{request\, selector}
This should be the first call after the connection to the server has
been made.  It sends a line to the server consisting of the
\var{request} string, the \var{selector} string, and the HTTP version
(\code{HTTP/1.0}).
\end{funcdesc}

\begin{funcdesc}{putheader}{header\, argument\optional{\, ...}}
Send an RFC-822 style header to the server.  It sends a line to the
server consisting of the header, a colon and a space, and the first
argument.  If more arguments are given, continuation lines are sent,
each consisting of a tab and an argument.
\end{funcdesc}

\begin{funcdesc}{endheaders}{}
Send a blank line to the server, signalling the end of the headers.
\end{funcdesc}

\begin{funcdesc}{getreply}{}
Complete the request by shutting down the sending end of the socket,
read the reply from the server, and return a triple (\var{replycode},
\var{message}, \var{headers}).  Here \var{replycode} is the integer
reply code from the request (e.g.\ \code{200} if the request was
handled properly); \var{message} is the message string corresponding
to the reply code; and \var{header} is an instance of the class
\code{rfc822.Message} containing the headers received from the server.
See the description of the \code{rfc822} module.
\stmodindex{rfc822}
\end{funcdesc}

\begin{funcdesc}{getfile}{}
Return a file object from which the data returned by the server can be
read, using the \code{read()}, \code{readline()} or \code{readlines()}
methods.
\end{funcdesc}

\subsection{Example}
\nodename{HTTP Example}

Here is an example session:

\begin{verbatim}
>>> import httplib
>>> h = httplib.HTTP('www.cwi.nl')
>>> h.putrequest('GET', '/index.html')
>>> h.putheader('Accept', 'text/html')
>>> h.putheader('Accept', 'text/plain')
>>> h.endheaders()
>>> errcode, errmsg, headers = h.getreply()
>>> print errcode # Should be 200
>>> f = h.getfile()
>>> data f.read() # Get the raw HTML
>>> f.close()
>>> 
\end{verbatim}

\section{Standard Module \sectcode{ftplib}}
\stmodindex{ftplib}

\renewcommand{\indexsubitem}{(in module ftplib)}

This module defines the class \code{FTP} and a few related items.  The
\code{FTP} class implements the client side of the FTP protocol.  You
can use this to write Python programs that perform a variety of
automated FTP jobs, such as mirroring other ftp servers.  It is also
used by the module \code{urllib} to handle URLs that use FTP.  For
more information on FTP (File Transfer Protocol), see Internet RFC
959.

Here's a sample session using the \code{ftplib} module:

\begin{verbatim}
>>> from ftplib import FTP
>>> ftp = FTP('ftp.cwi.nl')   # connect to host, default port
>>> ftp.login()               # user anonymous, passwd user@hostname
>>> ftp.retrlines('LIST')     # list directory contents
total 24418
drwxrwsr-x   5 ftp-usr  pdmaint     1536 Mar 20 09:48 .
dr-xr-srwt 105 ftp-usr  pdmaint     1536 Mar 21 14:32 ..
-rw-r--r--   1 ftp-usr  pdmaint     5305 Mar 20 09:48 INDEX
 .
 .
 .
>>> ftp.quit()
\end{verbatim}

The module defines the following items:

\begin{funcdesc}{FTP}{\optional{host\optional{\, user\, passwd\, acct}}}
Return a new instance of the \code{FTP} class.  When
\var{host} is given, the method call \code{connect(\var{host})} is
made.  When \var{user} is given, additionally the method call
\code{login(\var{user}, \var{passwd}, \var{acct})} is made (where
\var{passwd} and \var{acct} default to the empty string when not given).
\end{funcdesc}

\begin{datadesc}{all_errors}
The set of all exceptions (as a tuple) that methods of \code{FTP}
instances may raise as a result of problems with the FTP connection
(as opposed to programming errors made by the caller).  This set
includes the four exceptions listed below as well as
\code{socket.error} and \code{IOError}.
\end{datadesc}

\begin{excdesc}{error_reply}
Exception raised when an unexpected reply is received from the server.
\end{excdesc}

\begin{excdesc}{error_temp}
Exception raised when an error code in the range 400--499 is received.
\end{excdesc}

\begin{excdesc}{error_perm}
Exception raised when an error code in the range 500--599 is received.
\end{excdesc}

\begin{excdesc}{error_proto}
Exception raised when a reply is received from the server that does
not begin with a digit in the range 1--5.
\end{excdesc}

\subsection{FTP Objects}

FTP instances have the following methods:

\renewcommand{\indexsubitem}{(FTP object method)}

\begin{funcdesc}{set_debuglevel}{level}
Set the instance's debugging level.  This controls the amount of
debugging output printed.  The default, 0, produces no debugging
output.  A value of 1 produces a moderate amount of debugging output,
generally a single line per request.  A value of 2 or higher produces
the maximum amount of debugging output, logging each line sent and
received on the control connection.
\end{funcdesc}

\begin{funcdesc}{connect}{host\optional{\, port}}
Connect to the given host and port.  The default port number is 21, as
specified by the FTP protocol specification.  It is rarely needed to
specify a different port number.  This function should be called only
once for each instance; it should not be called at all if a host was
given when the instance was created.  All other methods can only be
used after a connection has been made.
\end{funcdesc}

\begin{funcdesc}{getwelcome}{}
Return the welcome message sent by the server in reply to the initial
connection.  (This message sometimes contains disclaimers or help
information that may be relevant to the user.)
\end{funcdesc}

\begin{funcdesc}{login}{\optional{user\optional{\, passwd\optional{\, acct}}}}
Log in as the given \var{user}.  The \var{passwd} and \var{acct}
parameters are optional and default to the empty string.  If no
\var{user} is specified, it defaults to \samp{anonymous}.  If
\var{user} is \code{anonymous}, the default \var{passwd} is
\samp{\var{realuser}@\var{host}} where \var{realuser} is the real user
name (glanced from the \samp{LOGNAME} or \samp{USER} environment
variable) and \var{host} is the hostname as returned by
\code{socket.gethostname()}.  This function should be called only
once for each instance, after a connection has been established; it
should not be called at all if a host and user were given when the
instance was created.  Most FTP commands are only allowed after the
client has logged in.
\end{funcdesc}

\begin{funcdesc}{abort}{}
Abort a file transfer that is in progress.  Using this does not always
work, but it's worth a try.
\end{funcdesc}

\begin{funcdesc}{sendcmd}{command}
Send a simple command string to the server and return the response
string.
\end{funcdesc}

\begin{funcdesc}{voidcmd}{command}
Send a simple command string to the server and handle the response.
Return nothing if a response code in the range 200--299 is received.
Raise an exception otherwise.
\end{funcdesc}

\begin{funcdesc}{retrbinary}{command\, callback\, maxblocksize}
Retrieve a file in binary transfer mode.  \var{command} should be an
appropriate \samp{RETR} command, i.e.\ \code{"RETR \var{filename}"}.
The \var{callback} function is called for each block of data received,
with a single string argument giving the data block.
The \var{maxblocksize} argument specifies the maximum block size
(which may not be the actual size of the data blocks passed to
\var{callback}).
\end{funcdesc}

\begin{funcdesc}{retrlines}{command\optional{\, callback}}
Retrieve a file or directory listing in \ASCII{} transfer mode.
var{command} should be an appropriate \samp{RETR} command (see
\code{retrbinary()} or a \samp{LIST} command (usually just the string
\code{"LIST"}).  The \var{callback} function is called for each line,
with the trailing CRLF stripped.  The default \var{callback} prints
the line to \code{sys.stdout}.
\end{funcdesc}

\begin{funcdesc}{storbinary}{command\, file\, blocksize}
Store a file in binary transfer mode.  \var{command} should be an
appropriate \samp{STOR} command, i.e.\ \code{"STOR \var{filename}"}.
\var{file} is an open file object which is read until EOF using its
\code{read()} method in blocks of size \var{blocksize} to provide the
data to be stored.
\end{funcdesc}

\begin{funcdesc}{storlines}{command\, file}
Store a file in \ASCII{} transfer mode.  \var{command} should be an
appropriate \samp{STOR} command (see \code{storbinary()}).  Lines are
read until EOF from the open file object \var{file} using its
\code{readline()} method to privide the data to be stored.
\end{funcdesc}

\begin{funcdesc}{nlst}{argument\optional{\, \ldots}}
Return a list of files as returned by the \samp{NLST} command.  The
optional var{argument} is a directory to list (default is the current
server directory).  Multiple arguments can be used to pass
non-standard options to the \samp{NLST} command.
\end{funcdesc}

\begin{funcdesc}{dir}{argument\optional{\, \ldots}}
Return a directory listing as returned by the \samp{LIST} command, as
a list of lines.  The optional var{argument} is a directory to list
(default is the current server directory).  Multiple arguments can be
used to pass non-standard options to the \samp{LIST} command.  If the
last argument is a function, it is used as a \var{callback} function
as for \code{retrlines()}.
\end{funcdesc}

\begin{funcdesc}{rename}{fromname\, toname}
Rename file \var{fromname} on the server to \var{toname}.
\end{funcdesc}

\begin{funcdesc}{cwd}{pathname}
Set the current directory on the server.
\end{funcdesc}

\begin{funcdesc}{mkd}{pathname}
Create a new directory on the server.
\end{funcdesc}

\begin{funcdesc}{pwd}{}
Return the pathname of the current directory on the server.
\end{funcdesc}

\begin{funcdesc}{quit}{}
Send a \samp{QUIT} command to the server and close the connection.
This is the ``polite'' way to close a connection, but it may raise an
exception of the server reponds with an error to the \code{QUIT}
command.
\end{funcdesc}

\begin{funcdesc}{close}{}
Close the connection unilaterally.  This should not be applied to an
already closed connection (e.g.\ after a successful call to
\code{quit()}.
\end{funcdesc}

\section{Standard Module \sectcode{gopherlib}}
\stmodindex{gopherlib}

\renewcommand{\indexsubitem}{(in module gopherlib)}

This module provides a minimal implementation of client side of the
the Gopher protocol.  It is used by the module \code{urllib} to handle
URLs that use the Gopher protocol.

The module defines the following functions:

\begin{funcdesc}{send_selector}{selector\, host\optional{\, port}}
Send a \var{selector} string to the gopher server at \var{host} and
\var{port} (default 70).  Return an open file object from which the
returned document can be read.
\end{funcdesc}

\begin{funcdesc}{send_query}{selector\, query\, host\optional{\, port}}
Send a \var{selector} string and a \var{query} string to a gopher
server at \var{host} and \var{port} (default 70).  Return an open file
object from which the returned document can be read.
\end{funcdesc}

Note that the data returned by the Gopher server can be of any type,
depending on the first character of the selector string.  If the data
is text (first character of the selector is \samp{0}), lines are
terminated by CRLF, and the data is terminated by a line consisting of
a single \samp{.}, and a leading \samp{.} should be stripped from
lines that begin with \samp{..}.  Directory listings (first charactger
of the selector is \samp{1}) are transferred using the same protocol.

\section{Standard Module \sectcode{nntplib}}
\stmodindex{nntplib}

\renewcommand{\indexsubitem}{(in module nntplib)}

This module defines the class \code{NNTP} which implements the client
side of the NNTP protocol.  It can be used to implement a news reader
or poster, or automated news processors.  For more information on NNTP
(Network News Transfer Protocol), see Internet RFC 977.

Here are two small examples of how it can be used.  To list some
statistics about a newsgroup and print the subjects of the last 10
articles:

\small{
\begin{verbatim}
>>> s = NNTP('news.cwi.nl')
>>> resp, count, first, last, name = s.group('comp.lang.python')
>>> print 'Group', name, 'has', count, 'articles, range', first, 'to', last
Group comp.lang.python has 59 articles, range 3742 to 3803
>>> resp, subs = s.xhdr('subject', first + '-' + last)
>>> for id, sub in subs[-10:]: print id, sub
... 
3792 Re: Removing elements from a list while iterating...
3793 Re: Who likes Info files?
3794 Emacs and doc strings
3795 a few questions about the Mac implementation
3796 Re: executable python scripts
3797 Re: executable python scripts
3798 Re: a few questions about the Mac implementation 
3799 Re: PROPOSAL: A Generic Python Object Interface for Python C Modules
3802 Re: executable python scripts 
3803 Re: POSIX wait and SIGCHLD
>>> s.quit()
'205 news.cwi.nl closing connection.  Goodbye.'
>>> 
\end{verbatim}
}

To post an article from a file (this assumes that the article has
valid headers):

\begin{verbatim}
>>> s = NNTP('news.cwi.nl')
>>> f = open('/tmp/article')
>>> s.post(f)
'240 Article posted successfully.'
>>> s.quit()
'205 news.cwi.nl closing connection.  Goodbye.'
>>> 
\end{verbatim}

The module itself defines the following items:

\begin{funcdesc}{NNTP}{host\optional{\, port}}
Return a new instance of the \code{NNTP} class, representing a
connection to the NNTP server running on host \var{host}, listening at
port \var{port}.  The default \var{port} is 119.
\end{funcdesc}

\begin{excdesc}{error_reply}
Exception raised when an unexpected reply is received from the server.
\end{excdesc}

\begin{excdesc}{error_temp}
Exception raised when an error code in the range 400--499 is received.
\end{excdesc}

\begin{excdesc}{error_perm}
Exception raised when an error code in the range 500--599 is received.
\end{excdesc}

\begin{excdesc}{error_proto}
Exception raised when a reply is received from the server that does
not begin with a digit in the range 1--5.
\end{excdesc}

\subsection{NNTP Objects}

NNTP instances have the following methods.  The \var{response} that is
returned as the first item in the return tuple of almost all methods
is the server's response: a string beginning with a three-digit code.
If the server's response indicates an error, the method raises one of
the above exceptions.

\renewcommand{\indexsubitem}{(NNTP object method)}

\begin{funcdesc}{getwelcome}{}
Return the welcome message sent by the server in reply to the initial
connection.  (This message sometimes contains disclaimers or help
information that may be relevant to the user.)
\end{funcdesc}

\begin{funcdesc}{set_debuglevel}{level}
Set the instance's debugging level.  This controls the amount of
debugging output printed.  The default, 0, produces no debugging
output.  A value of 1 produces a moderate amount of debugging output,
generally a single line per request or response.  A value of 2 or
higher produces the maximum amount of debugging output, logging each
line sent and received on the connection (including message text).
\end{funcdesc}

\begin{funcdesc}{newgroups}{date\, time}
Send a \samp{NEWGROUPS} command.  The \var{date} argument should be a
string of the form \code{"\var{yy}\var{mm}\var{dd}"} indicating the
date, and \var{time} should be a string of the form
\code{"\var{hh}\var{mm}\var{ss}"} indicating the time.  Return a pair
\code{(\var{response}, \var{groups})} where \var{groups} is a list of
group names that are new since the given date and time.
\end{funcdesc}

\begin{funcdesc}{newnews}{group\, date\, time}
Send a \samp{NEWNEWS} command.  Here, \var{group} is a group name or
\code{"*"}, and \var{date} and \var{time} have the same meaning as for
\code{newgroups()}.  Return a pair \code{(\var{response},
\var{articles})} where \var{articles} is a list of article ids.
\end{funcdesc}

\begin{funcdesc}{list}{}
Send a \samp{LIST} command.  Return a pair \code{(\var{response},
\var{list})} where \var{list} is a list of tuples.  Each tuple has the
form \code{(\var{group}, \var{last}, \var{first}, \var{flag})}, where
\var{group} is a group name, \var{last} and \var{first} are the last
and first article numbers (as strings), and \var{flag} is \code{'y'}
if posting is allowed, \code{'n'} if not, and \code{'m'} if the
newsgroup is moderated.  (Note the ordering: \var{last}, \var{first}.)
\end{funcdesc}

\begin{funcdesc}{group}{name}
Send a \samp{GROUP} command, where \var{name} is the group name.
Return a tuple \code{(\var{response}, \var{count}, \var{first},
\var{last}, \var{name})} where \var{count} is the (estimated) number
of articles in the group, \var{first} is the first article number in
the group, \var{last} is the last article number in the group, and
\var{name} is the group name.  The numbers are returned as strings.
\end{funcdesc}

\begin{funcdesc}{help}{}
Send a \samp{HELP} command.  Return a pair \code{(\var{response},
\var{list})} where \var{list} is a list of help strings.
\end{funcdesc}

\begin{funcdesc}{stat}{id}
Send a \samp{STAT} command, where \var{id} is the message id (enclosed
in \samp{<} and \samp{>}) or an article number (as a string).
Return a triple \code{(var{response}, \var{number}, \var{id})} where
\var{number} is the article number (as a string) and \var{id} is the
article id  (enclosed in \samp{<} and \samp{>}).
\end{funcdesc}

\begin{funcdesc}{next}{}
Send a \samp{NEXT} command.  Return as for \code{stat()}.
\end{funcdesc}

\begin{funcdesc}{last}{}
Send a \samp{LAST} command.  Return as for \code{stat()}.
\end{funcdesc}

\begin{funcdesc}{head}{id}
Send a \samp{HEAD} command, where \var{id} has the same meaning as for
\code{stat()}.  Return a pair \code{(\var{response}, \var{list})}
where \var{list} is a list of the article's headers (an uninterpreted
list of lines, without trailing newlines).
\end{funcdesc}

\begin{funcdesc}{body}{id}
Send a \samp{BODY} command, where \var{id} has the same meaning as for
\code{stat()}.  Return a pair \code{(\var{response}, \var{list})}
where \var{list} is a list of the article's body text (an
uninterpreted list of lines, without trailing newlines).
\end{funcdesc}

\begin{funcdesc}{article}{id}
Send a \samp{ARTICLE} command, where \var{id} has the same meaning as
for \code{stat()}.  Return a pair \code{(\var{response}, \var{list})}
where \var{list} is a list of the article's header and body text (an
uninterpreted list of lines, without trailing newlines).
\end{funcdesc}

\begin{funcdesc}{slave}{}
Send a \samp{SLAVE} command.  Return the server's \var{response}.
\end{funcdesc}

\begin{funcdesc}{xhdr}{header\, string}
Send an \samp{XHDR} command.  This command is not defined in the RFC
but is a common extension.  The \var{header} argument is a header
keyword, e.g. \code{"subject"}.  The \var{string} argument should have
the form \code{"\var{first}-\var{last}"} where \var{first} and
\var{last} are the first and last article numbers to search.  Return a
pair \code{(\var{response}, \var{list})}, where \var{list} is a list of
pairs \code{(\var{id}, \var{text})}, where \var{id} is an article id
(as a string) and \var{text} is the text of the requested header for
that article.
\end{funcdesc}

\begin{funcdesc}{post}{file}
Post an article using the \samp{POST} command.  The \var{file}
argument is an open file object which is read until EOF using its
\code{readline()} method.  It should be a well-formed news article,
including the required headers.  The \code{post()} method
automatically escapes lines beginning with \samp{.}.
\end{funcdesc}

\begin{funcdesc}{ihave}{id\, file}
Send an \samp{IHAVE} command.  If the response is not an error, treat
\var{file} exactly as for the \code{post()} method.
\end{funcdesc}

\begin{funcdesc}{quit}{}
Send a \samp{QUIT} command and close the connection.  Once this method
has been called, no other methods of the NNTP object should be called.
\end{funcdesc}

\section{Standard Module \sectcode{urlparse}}
\stmodindex{urlparse}
\index{WWW}
\index{World-Wide Web}
\index{URL}
\indexii{URL}{parsing}
\indexii{relative}{URL}

\renewcommand{\indexsubitem}{(in module urlparse)}

This module defines a standard interface to break URL strings up in
components (addessing scheme, network location, path etc.), to combine
the components back into a URL string, and to convert a ``relative
URL'' to an absolute URL given a ``base URL''.

The module has been designed to match the current Internet draft on
Relative Uniform Resource Locators (and discovered a bug in an earlier
draft!).

It defines the following functions:

\begin{funcdesc}{urlparse}{urlstring\optional{\,
default_scheme\optional{\, allow_fragments}}}
Parse a URL into 6 components, returning a 6-tuple: (addressing
scheme, network location, path, parameters, query, fragment
identifier).  This corresponds to the general structure of a URL:
\code{\var{scheme}://\var{netloc}/\var{path};\var{parameters}?\var{query}\#\var{fragment}}.
Each tuple item is a string, possibly empty.
The components are not broken up in smaller parts (e.g. the network
location is a single string), and \% escapes are not expanded.
The delimiters as shown above are not part of the tuple items,
except for a leading slash in the \var{path} component, which is
retained if present.

Example:

\begin{verbatim}
urlparse('http://www.cwi.nl:80/%7Eguido/Python.html')
\end{verbatim}

yields the tuple

\begin{verbatim}
('http', 'www.cwi.nl:80', '/%7Eguido/Python.html', '', '', '')
\end{verbatim}

If the \var{default_scheme} argument is specified, it gives the
default addressing scheme, to be used only if the URL string does not
specify one.  The default value for this argument is the empty string.

If the \var{allow_fragments} argument is zero, fragment identifiers
are not allowed, even if the URL's addressing scheme normally does
support them.  The default value for this argument is \code{1}.
\end{funcdesc}

\begin{funcdesc}{urlunparse}{tuple}
Construct a URL string from a tuple as returned by \code{urlparse}.
This may result in a slightly different, but equivalent URL, if the
URL that was parsed originally had redundant delimiters, e.g. a ? with
an empty query (the draft states that these are equivalent).
\end{funcdesc}

\begin{funcdesc}{urljoin}{base\, url\optional{\, allow_fragments}}
Construct a full (``absolute'') URL by combining a ``base URL''
(\var{base}) with a ``relative URL'' (\var{url}).  Informally, this
uses components of the base URL, in particular the addressing scheme,
the network location and (part of) the path, to provide missing
components in the relative URL.

Example:

\begin{verbatim}
urljoin('http://www.cwi.nl/%7Eguido/Python.html', 'FAQ.html')
\end{verbatim}

yields the string

\begin{verbatim}
'http://www.cwi.nl/%7Eguido/FAQ.html'
\end{verbatim}

The \var{allow_fragments} argument has the same meaning as for
\code{urlparse}.
\end{funcdesc}

\section{Standard Module \sectcode{sgmllib}}
\stmodindex{sgmllib}
\index{SGML}

This module defines a class \code{SGMLParser} which serves as the
basis for parsing text files formatted in SGML (Standard Generalized
Mark-up Language).  In fact, it does not provide a full SGML parser
--- it only parses SGML insofar as it is used by HTML, and the module
only exists as a base for the \code{htmllib} module.
\stmodindex{htmllib}

In particular, the parser is hardcoded to recognize the following
constructs:

\begin{itemize}

\item
Opening and closing tags of the form
``\code{<\var{tag} \var{attr}="\var{value}" ...>}'' and
``\code{</\var{tag}>}'', respectively.

\item
Numeric character references of the form ``\code{\&\#\var{name};}''.

\item
Entity references of the form ``\code{\&\var{name};}''.

\item
SGML comments of the form ``\code{<!--\var{text}-->}''.  Note that
spaces, tabs, and newlines are allowed between the trailing
``\code{>}'' and the immediately preceeding ``\code{--}''.

\end{itemize}

The \code{SGMLParser} class must be instantiated without arguments.
It has the following interface methods:

\renewcommand{\indexsubitem}{({\tt SGMLParser} method)}

\begin{funcdesc}{reset}{}
Reset the instance.  Loses all unprocessed data.  This is called
implicitly at instantiation time.
\end{funcdesc}

\begin{funcdesc}{setnomoretags}{}
Stop processing tags.  Treat all following input as literal input
(CDATA).  (This is only provided so the HTML tag \code{<PLAINTEXT>}
can be implemented.)
\end{funcdesc}

\begin{funcdesc}{setliteral}{}
Enter literal mode (CDATA mode).
\end{funcdesc}

\begin{funcdesc}{feed}{data}
Feed some text to the parser.  It is processed insofar as it consists
of complete elements; incomplete data is buffered until more data is
fed or \code{close()} is called.
\end{funcdesc}

\begin{funcdesc}{close}{}
Force processing of all buffered data as if it were followed by an
end-of-file mark.  This method may be redefined by a derived class to
define additional processing at the end of the input, but the
redefined version should always call \code{SGMLParser.close()}.
\end{funcdesc}

\begin{funcdesc}{handle_starttag}{tag\, method\, attributes}
This method is called to handle start tags for which either a
\code{start_\var{tag}()} or \code{do_\var{tag}()} method has been
defined.  The \code{tag} argument is the name of the tag converted to
lower case, and the \code{method} argument is the bound method which
should be used to support semantic interpretation of the start tag.
The \var{attributes} argument is a list of (\var{name}, \var{value})
pairs containing the attributes found inside the tag's \code{<>}
brackets.  The \var{name} has been translated to lower case and double
quotes and backslashes in the \var{value} have been interpreted.  For
instance, for the tag \code{<A HREF="http://www.cwi.nl/">}, this
method would be called as \code{unknown_starttag('a', [('href',
'http://www.cwi.nl/')])}.  The base implementation simply calls
\code{method} with \code{attributes} as the only argument.
\end{funcdesc}

\begin{funcdesc}{handle_endtag}{tag\, method}

This method is called to handle endtags for which an
\code{end_\var{tag}()} method has been defined.  The \code{tag}
argument is the name of the tag converted to lower case, and the
\code{method} argument is the bound method which should be used to
support semantic interpretation of the end tag.  If no
\code{end_\var{tag}()} method is defined for the closing element, this
handler is not called.  The base implementation simply calls
\code{method}.
\end{funcdesc}

\begin{funcdesc}{handle_data}{data}
This method is called to process arbitrary data.  It is intended to be
overridden by a derived class; the base class implementation does
nothing.
\end{funcdesc}

\begin{funcdesc}{handle_charref}{ref}
This method is called to process a character reference of the form
``\code{\&\#\var{ref};}''.  In the base implementation, \var{ref} must
be a decimal number in the
range 0-255.  It translates the character to \ASCII{} and calls the
method \code{handle_data()} with the character as argument.  If
\var{ref} is invalid or out of range, the method
\code{unknown_charref(\var{ref})} is called to handle the error.  A
subclass must override this method to provide support for named
character entities.
\end{funcdesc}

\begin{funcdesc}{handle_entityref}{ref}
This method is called to process a general entity reference of the form
``\code{\&\var{ref};}'' where \var{ref} is an general entity
reference.  It looks for \var{ref} in the instance (or class)
variable \code{entitydefs} which should be a mapping from entity names
to corresponding translations.
If a translation is found, it calls the method \code{handle_data()}
with the translation; otherwise, it calls the method
\code{unknown_entityref(\var{ref})}.  The default \code{entitydefs}
defines translations for \code{\&amp;}, \code{\&apos}, \code{\&gt;},
\code{\&lt;}, and \code{\&quot;}.
\end{funcdesc}

\begin{funcdesc}{handle_comment}{comment}
This method is called when a comment is encountered.  The
\code{comment} argument is a string containing the text between the
``\code{<!--}'' and ``\code{-->}'' delimiters, but not the delimiters
themselves.  For example, the comment ``\code{<!--text-->}'' will
cause this method to be called with the argument \code{'text'}.  The
default method does nothing.
\end{funcdesc}

\begin{funcdesc}{report_unbalanced}{tag}
This method is called when an end tag is found which does not
correspond to any open element.
\end{funcdesc}

\begin{funcdesc}{unknown_starttag}{tag\, attributes}
This method is called to process an unknown start tag.  It is intended
to be overridden by a derived class; the base class implementation
does nothing.
\end{funcdesc}

\begin{funcdesc}{unknown_endtag}{tag}
This method is called to process an unknown end tag.  It is intended
to be overridden by a derived class; the base class implementation
does nothing.
\end{funcdesc}

\begin{funcdesc}{unknown_charref}{ref}
This method is called to process unresolvable numeric character
references.  It is intended to be overridden by a derived class; the
base class implementation does nothing.
\end{funcdesc}

\begin{funcdesc}{unknown_entityref}{ref}
This method is called to process an unknown entity reference.  It is
intended to be overridden by a derived class; the base class
implementation does nothing.
\end{funcdesc}

Apart from overriding or extending the methods listed above, derived
classes may also define methods of the following form to define
processing of specific tags.  Tag names in the input stream are case
independent; the \var{tag} occurring in method names must be in lower
case:

\begin{funcdesc}{start_\var{tag}}{attributes}
This method is called to process an opening tag \var{tag}.  It has
preference over \code{do_\var{tag}()}.  The \var{attributes} argument
has the same meaning as described for \code{handle_starttag()} above.
\end{funcdesc}

\begin{funcdesc}{do_\var{tag}}{attributes}
This method is called to process an opening tag \var{tag} that does
not come with a matching closing tag.  The \var{attributes} argument
has the same meaning as described for \code{handle_starttag()} above.
\end{funcdesc}

\begin{funcdesc}{end_\var{tag}}{}
This method is called to process a closing tag \var{tag}.
\end{funcdesc}

Note that the parser maintains a stack of open elements for which no
end tag has been found yet.  Only tags processed by
\code{start_\var{tag}()} are pushed on this stack.  Definition of an
\code{end_\var{tag}()} method is optional for these tags.  For tags
processed by \code{do_\var{tag}()} or by \code{unknown_tag()}, no
\code{end_\var{tag}()} method must be defined; if defined, it will not
be used.  If both \code{start_\var{tag}()} and \code{do_\var{tag}()}
methods exist for a tag, the \code{start_\var{tag}()} method takes
precedence.

\section{Standard Module \sectcode{htmllib}}
\stmodindex{htmllib}
\index{HTML}
\index{hypertext}

\renewcommand{\indexsubitem}{(in module htmllib)}

This module defines a class which can serve as a base for parsing text
files formatted in the HyperText Mark-up Language (HTML).  The class
is not directly concerned with I/O --- it must be provided with input
in string form via a method, and makes calls to methods of a
``formatter'' object in order to produce output.  The
\code{HTMLParser} class is designed to be used as a base class for
other classes in order to add functionality, and allows most of its
methods to be extended or overridden.  In turn, this class is derived
from and extends the \code{SGMLParser} class defined in module
\code{sgmllib}.  Two implementations of formatter objects are
provided in the \code{formatter} module; refer to the documentation
for that module for information on the formatter interface.
\index{SGML}
\stmodindex{sgmllib}
\ttindex{SGMLParser}
\index{formatter}
\stmodindex{formatter}

The following is a summary of the interface defined by
\code{sgmllib.SGMLParser}:

\begin{itemize}

\item
The interface to feed data to an instance is through the \code{feed()}
method, which takes a string argument.  This can be called with as
little or as much text at a time as desired; \code{p.feed(a);
p.feed(b)} has the same effect as \code{p.feed(a+b)}.  When the data
contains complete HTML tags, these are processed immediately;
incomplete elements are saved in a buffer.  To force processing of all
unprocessed data, call the \code{close()} method.

For example, to parse the entire contents of a file, use:
\begin{verbatim}
parser.feed(open('myfile.html').read())
parser.close()
\end{verbatim}

\item
The interface to define semantics for HTML tags is very simple: derive
a class and define methods called \code{start_\var{tag}()},
\code{end_\var{tag}()}, or \code{do_\var{tag}()}.  The parser will
call these at appropriate moments: \code{start_\var{tag}} or
\code{do_\var{tag}} is called when an opening tag of the form
\code{<\var{tag} ...>} is encountered; \code{end_\var{tag}} is called
when a closing tag of the form \code{<\var{tag}>} is encountered.  If
an opening tag requires a corresponding closing tag, like \code{<H1>}
... \code{</H1>}, the class should define the \code{start_\var{tag}}
method; if a tag requires no closing tag, like \code{<P>}, the class
should define the \code{do_\var{tag}} method.

\end{itemize}

The module defines a single class:

\begin{funcdesc}{HTMLParser}{formatter}
This is the basic HTML parser class.  It supports all entity names
required by the HTML 2.0 specification (RFC 1866).  It also defines
handlers for all HTML 2.0 and many HTML 3.0 and 3.2 elements.
\end{funcdesc}

In addition to tag methods, the \code{HTMLParser} class provides some
additional methods and instance variables for use within tag methods.

\renewcommand{\indexsubitem}{({\tt HTMLParser} method)}

\begin{datadesc}{formatter}
This is the formatter instance associated with the parser.
\end{datadesc}

\begin{datadesc}{nofill}
Boolean flag which should be true when whitespace should not be
collapsed, or false when it should be.  In general, this should only
be true when character data is to be treated as ``preformatted'' text,
as within a \code{<PRE>} element.  The default value is false.  This
affects the operation of \code{handle_data()} and \code{save_end()}.
\end{datadesc}

\begin{funcdesc}{anchor_bgn}{href\, name\, type}
This method is called at the start of an anchor region.  The arguments
correspond to the attributes of the \code{<A>} tag with the same
names.  The default implementation maintains a list of hyperlinks
(defined by the \code{href} argument) within the document.  The list
of hyperlinks is available as the data attribute \code{anchorlist}.
\end{funcdesc}

\begin{funcdesc}{anchor_end}{}
This method is called at the end of an anchor region.  The default
implementation adds a textual footnote marker using an index into the
list of hyperlinks created by \code{anchor_bgn()}.
\end{funcdesc}

\begin{funcdesc}{handle_image}{source\, alt\optional{\, ismap\optional{\, align\optional{\, width\optional{\, height}}}}}
This method is called to handle images.  The default implementation
simply passes the \code{alt} value to the \code{handle_data()}
method.
\end{funcdesc}

\begin{funcdesc}{save_bgn}{}
Begins saving character data in a buffer instead of sending it to the
formatter object.  Retrieve the stored data via \code{save_end()}
Use of the \code{save_bgn()} / \code{save_end()} pair may not be
nested.
\end{funcdesc}

\begin{funcdesc}{save_end}{}
Ends buffering character data and returns all data saved since the
preceeding call to \code{save_bgn()}.  If \code{nofill} flag is false,
whitespace is collapsed to single spaces.  A call to this method
without a preceeding call to \code{save_bgn()} will raise a
\code{TypeError} exception.
\end{funcdesc}

\section{Standard Module \sectcode{formatter}}
\stmodindex{formatter}

\renewcommand{\indexsubitem}{(in module formatter)}

This module supports two interface definitions, each with mulitple
implementations.  The \emph{formatter} interface is used by the
\code{HTMLParser} class of the \code{htmllib} module, and the
\emph{writer} interface is required by the formatter interface.

Formatter objects transform an abstract flow of formatting events into
specific output events on writer objects.  Formatters manage several
stack structures to allow various properties of a writer object to be
changed and restored; writers need not be able to handle relative
changes nor any sort of ``change back'' operation.  Specific writer
properties which may be controlled via formatter objects are
horizontal alignment, font, and left margin indentations.  A mechanism
is provided which supports providing arbitrary, non-exclusive style
settings to a writer as well.  Additional interfaces facilitate
formatting events which are not reversible, such as paragraph
separation.

Writer objects encapsulate device interfaces.  Abstract devices, such
as file formats, are supported as well as physical devices.  The
provided implementations all work with abstract devices.  The
interface makes available mechanisms for setting the properties which
formatter objects manage and inserting data into the output.


\subsection{The Formatter Interface}

Interfaces to create formatters are dependent on the specific
formatter class being instantiated.  The interfaces described below
are the required interfaces which all formatters must support once
initialized.

One data element is defined at the module level:

\begin{datadesc}{AS_IS}
Value which can be used in the font specification passed to the
\code{push_font()} method described below, or as the new value to any
other \code{push_\var{property}()} method.  Pushing the \code{AS_IS}
value allows the corresponding \code{pop_\var{property}()} method to
be called without having to track whether the property was changed.
\end{datadesc}

The following attributes are defined for formatter instance objects:

\renewcommand{\indexsubitem}{(formatter object data)}

\begin{datadesc}{writer}
The writer instance with which the formatter interacts.
\end{datadesc}


\renewcommand{\indexsubitem}{(formatter object method)}

\begin{funcdesc}{end_paragraph}{blanklines}
Close any open paragraphs and insert at least \code{blanklines}
before the next paragraph.
\end{funcdesc}

\begin{funcdesc}{add_line_break}{}
Add a hard line break if one does not already exist.  This does not
break the logical paragraph.
\end{funcdesc}

\begin{funcdesc}{add_hor_rule}{*args\, **kw}
Insert a horizontal rule in the output.  A hard break is inserted if
there is data in the current paragraph, but the logical paragraph is
not broken.  The arguments and keywords are passed on to the writer's
\code{send_line_break()} method.
\end{funcdesc}

\begin{funcdesc}{add_flowing_data}{data}
Provide data which should be formatted with collapsed whitespaces.
Whitespace from preceeding and successive calls to
\code{add_flowing_data()} is considered as well when the whitespace
collapse is performed.  The data which is passed to this method is
expected to be word-wrapped by the output device.  Note that any
word-wrapping still must be performed by the writer object due to the
need to rely on device and font information.
\end{funcdesc}

\begin{funcdesc}{add_literal_data}{data}
Provide data which should be passed to the writer unchanged.
Whitespace, including newline and tab characters, are considered legal
in the value of \code{data}.  
\end{funcdesc}

\begin{funcdesc}{add_label_data}{format, counter}
Insert a label which should be placed to the left of the current left
margin.  This should be used for constructing bulleted or numbered
lists.  If the \code{format} value is a string, it is interpreted as a
format specification for \code{counter}, which should be an integer.
The result of this formatting becomes the value of the label; if
\code{format} is not a string it is used as the label value directly.
The label value is passed as the only argument to the writer's
\code{send_label_data()} method.  Interpretation of non-string label
values is dependent on the associated writer.

Format specifications are strings which, in combination with a counter
value, are used to compute label values.  Each character in the format
string is copied to the label value, with some characters recognized
to indicate a transform on the counter value.  Specifically, the
character ``\code{1}'' represents the counter value formatter as an
arabic number, the characters ``\code{A}'' and ``\code{a}'' represent
alphabetic representations of the counter value in upper and lower
case, respectively, and ``\code{I}'' and ``\code{i}'' represent the
counter value in Roman numerals, in upper and lower case.  Note that
the alphabetic and roman transforms require that the counter value be
greater than zero.
\end{funcdesc}

\begin{funcdesc}{flush_softspace}{}
Send any pending whitespace buffered from a previous call to
\code{add_flowing_data()} to the associated writer object.  This
should be called before any direct manipulation of the writer object.
\end{funcdesc}

\begin{funcdesc}{push_alignment}{align}
Push a new alignment setting onto the alignment stack.  This may be
\code{AS_IS} if no change is desired.  If the alignment value is
changed from the previous setting, the writer's \code{new_alignment()}
method is called with the \code{align} value.
\end{funcdesc}

\begin{funcdesc}{pop_alignment}{}
Restore the previous alignment.
\end{funcdesc}

\begin{funcdesc}{push_font}{(size, italic, bold, teletype)}
Change some or all font properties of the writer object.  Properties
which are not set to \code{AS_IS} are set to the values passed in
while others are maintained at their current settings.  The writer's
\code{new_font()} method is called with the fully resolved font
specification.
\end{funcdesc}

\begin{funcdesc}{pop_font}{}
Restore the previous font.
\end{funcdesc}

\begin{funcdesc}{push_margin}{margin}
Increase the number of left margin indentations by one, associating
the logical tag \code{margin} with the new indentation.  The initial
margin level is \code{0}.  Changed values of the logical tag must be
true values; false values other than \code{AS_IS} are not sufficient
to change the margin.
\end{funcdesc}

\begin{funcdesc}{pop_margin}{}
Restore the previous margin.
\end{funcdesc}

\begin{funcdesc}{push_style}{*styles}
Push any number of arbitrary style specifications.  All styles are
pushed onto the styles stack in order.  A tuple representing the
entire stack, including \code{AS_IS} values, is passed to the writer's
\code{new_styles()} method.
\end{funcdesc}

\begin{funcdesc}{pop_style}{\optional{n\code{ = 1}}}
Pop the last \code{n} style specifications passed to
\code{push_style()}.  A tuple representing the revised stack,
including \code{AS_IS} values, is passed to the writer's
\code{new_styles()} method.
\end{funcdesc}

\begin{funcdesc}{set_spacing}{spacing}
Set the spacing style for the writer.
\end{funcdesc}

\begin{funcdesc}{assert_line_data}{\optional{flag\code{ = 1}}}
Inform the formatter that data has been added to the current paragraph
out-of-band.  This should be used when the writer has been manipulated
directly.  The optional \code{flag} argument can be set to false if
the writer manipulations produced a hard line break at the end of the
output.
\end{funcdesc}


\subsection{Formatter Implementations}

Two implementations of formatter objects are provided by this module.
Most applications may use one of these classes without modification or
subclassing.

\renewcommand{\indexsubitem}{(in module formatter)}

\begin{funcdesc}{NullFormatter}{\optional{writer\code{ = None}}}
A formatter which does nothing.  If \code{writer} is omitted, a
\code{NullWriter} instance is created.  No methods of the writer are
called by \code{NullWriter} instances.  Implementations should inherit
from this class if implementing a writer interface but don't need to
inherit any implementation.
\end{funcdesc}

\begin{funcdesc}{AbstractFormatter}{writer}
The standard formatter.  This implementation has demonstrated wide
applicability to many writers, and may be used directly in most
circumstances.  It has been used to implement a full-featured
world-wide web browser.
\end{funcdesc}



\subsection{The Writer Interface}

Interfaces to create writers are dependent on the specific writer
class being instantiated.  The interfaces described below are the
required interfaces which all writers must support once initialized.
Note that while most applications can use the \code{AbstractFormatter}
class as a formatter, the writer must typically be provided by the
application.

\renewcommand{\indexsubitem}{(writer object method)}

\begin{funcdesc}{new_alignment}{align}
Set the alignment style.  The \code{align} value can be any object,
but by convention is a string or \code{None}, where \code{None}
indicates that the writer's ``preferred'' alignment should be used.
Conventional \code{align} values are \code{'left'}, \code{'center'},
\code{'right'}, and \code{'justify'}.
\end{funcdesc}

\begin{funcdesc}{new_font}{font}
Set the font style.  The value of \code{font} will be \code{None},
indicating that the device's default font should be used, or a tuple
of the form (\var{size}, \var{italic}, \var{bold}, \var{teletype}).
Size will be a string indicating the size of font that should be used;
specific strings and their interpretation must be defined by the
application.  The \var{italic}, \var{bold}, and \var{teletype} values
are boolean indicators specifying which of those font attributes
should be used.
\end{funcdesc}

\begin{funcdesc}{new_margin}{margin, level}
Set the margin level to the integer \code{level} and the logical tag
to \code{margin}.  Interpretation of the logical tag is at the
writer's discretion; the only restriction on the value of the logical
tag is that it not be a false value for non-zero values of
\code{level}.
\end{funcdesc}

\begin{funcdesc}{new_spacing}{spacing}
Set the spacing style to \code{spacing}.
\end{funcdesc}

\begin{funcdesc}{new_styles}{styles}
Set additional styles.  The \code{styles} value is a tuple of
arbitrary values; the value \code{AS_IS} should be ignored.  The
\code{styles} tuple may be interpreted either as a set or as a stack
depending on the requirements of the application and writer
implementation.
\end{funcdesc}

\begin{funcdesc}{send_line_break}{}
Break the current line.
\end{funcdesc}

\begin{funcdesc}{send_paragraph}{blankline}
Produce a paragraph separation of at least \code{blankline} blank
lines, or the equivelent.  The \code{blankline} value will be an
integer.
\end{funcdesc}

\begin{funcdesc}{send_hor_rule}{*args\, **kw}
Display a horizontal rule on the output device.  The arguments to this
method are entirely application- and writer-specific, and should be
interpreted with care.  The method implementation may assume that a
line break has already been issued via \code{send_line_break()}.
\end{funcdesc}

\begin{funcdesc}{send_flowing_data}{data}
Output character data which may be word-wrapped and re-flowed as
needed.  Within any sequence of calls to this method, the writer may
assume that spans of multiple whitespace characters have been
collapsed to single space characters.
\end{funcdesc}

\begin{funcdesc}{send_literal_data}{data}
Output character data which has already been formatted
for display.  Generally, this should be interpreted to mean that line
breaks indicated by newline characters should be preserved and no new
line breaks should be introduced.  The data may contain embedded
newline and tab characters, unlike data provided to the
\code{send_formatted_data()} interface.
\end{funcdesc}

\begin{funcdesc}{send_label_data}{data}
Set \code{data} to the left of the current left margin, if possible.
The value of \code{data} is not restricted; treatment of non-string
values is entirely application- and writer-dependent.  This method
will only be called at the beginning of a line.
\end{funcdesc}


\subsection{Writer Implementations}

Three implementations of the writer object interface are provided as
examples by this module.  Most applications will need to derive new
writer classes from the \code{NullWriter} class.

\renewcommand{\indexsubitem}{(in module formatter)}

\begin{funcdesc}{NullWriter}{}
A writer which only provides the interface definition; no actions are
taken on any methods.  This should be the base class for all writers
which do not need to inherit any implementation methods.
\end{funcdesc}

\begin{funcdesc}{AbstractWriter}{}
A writer which can be used in debugging formatters, but not much
else.  Each method simply accounces itself by printing its name and
arguments on standard output.
\end{funcdesc}

\begin{funcdesc}{DumbWriter}{\optional{file\code{ = None}\optional{\, maxcol\code{ = 72}}}}
Simple writer class which writes output on the file object passed in
as \code{file} or, if \code{file} is omitted, on standard output.  The
output is simply word-wrapped to the number of columns specified by
\code{maxcol}.  This class is suitable for reflowing a sequence of
paragraphs.
\end{funcdesc}

\section{Standard Module \sectcode{rfc822}}
\stmodindex{rfc822}

\renewcommand{\indexsubitem}{(in module rfc822)}

This module defines a class, \code{Message}, which represents a
collection of ``email headers'' as defined by the Internet standard
RFC 822.  It is used in various contexts, usually to read such headers
from a file.

A \code{Message} instance is instantiated with an open file object as
parameter.  Instantiation reads headers from the file up to a blank
line and stores them in the instance; after instantiation, the file is
positioned directly after the blank line that terminates the headers.

Input lines as read from the file may either be terminated by CR-LF or
by a single linefeed; a terminating CR-LF is replaced by a single
linefeed before the line is stored.

All header matching is done independent of upper or lower case;
e.g. \code{m['From']}, \code{m['from']} and \code{m['FROM']} all yield
the same result.

\subsection{Message Objects}

A \code{Message} instance has the following methods:

\begin{funcdesc}{rewindbody}{}
Seek to the start of the message body.  This only works if the file
object is seekable.
\end{funcdesc}

\begin{funcdesc}{getallmatchingheaders}{name}
Return a list of lines consisting of all headers matching
\var{name}, if any.  Each physical line, whether it is a continuation
line or not, is a separate list item.  Return the empty list if no
header matches \var{name}.
\end{funcdesc}

\begin{funcdesc}{getfirstmatchingheader}{name}
Return a list of lines comprising the first header matching
\var{name}, and its continuation line(s), if any.  Return \code{None}
if there is no header matching \var{name}.
\end{funcdesc}

\begin{funcdesc}{getrawheader}{name}
Return a single string consisting of the text after the colon in the
first header matching \var{name}.  This includes leading whitespace,
the trailing linefeed, and internal linefeeds and whitespace if there
any continuation line(s) were present.  Return \code{None} if there is
no header matching \var{name}.
\end{funcdesc}

\begin{funcdesc}{getheader}{name}
Like \code{getrawheader(\var{name})}, but strip leading and trailing
whitespace (but not internal whitespace).
\end{funcdesc}

\begin{funcdesc}{getaddr}{name}
Return a pair (full name, email address) parsed from the string
returned by \code{getheader(\var{name})}.  If no header matching
\var{name} exists, return \code{None, None}; otherwise both the full
name and the address are (possibly empty )strings.

Example: If \code{m}'s first \code{From} header contains the string\\
\code{'jack@cwi.nl (Jack Jansen)'}, then
\code{m.getaddr('From')} will yield the pair
\code{('Jack Jansen', 'jack@cwi.nl')}.
If the header contained
\code{'Jack Jansen <jack@cwi.nl>'} instead, it would yield the
exact same result.
\end{funcdesc}

\begin{funcdesc}{getaddrlist}{name}
This is similar to \code{getaddr(\var{list})}, but parses a header
containing a list of email addresses (e.g. a \code{To} header) and
returns a list of (full name, email address) pairs (even if there was
only one address in the header).  If there is no header matching
\var{name}, return an empty list.

XXX The current version of this function is not really correct.  It
yields bogus results if a full name contains a comma.
\end{funcdesc}

\begin{funcdesc}{getdate}{name}
Retrieve a header using \code{getheader} and parse it into a 9-tuple
compatible with \code{time.mktime()}.  If there is no header matching
\var{name}, or it is unparsable, return \code{None}.

Date parsing appears to be a black art, and not all mailers adhere to
the standard.  While it has been tested and found correct on a large
collection of email from many sources, it is still possible that this
function may occasionally yield an incorrect result.
\end{funcdesc}

\code{Message} instances also support a read-only mapping interface.
In particular: \code{m[name]} is the same as \code{m.getheader(name)};
and \code{len(m)}, \code{m.has_key(name)}, \code{m.keys()},
\code{m.values()} and \code{m.items()} act as expected (and
consistently).

Finally, \code{Message} instances have two public instance variables:

\begin{datadesc}{headers}
A list containing the entire set of header lines, in the order in
which they were read.  Each line contains a trailing newline.  The
blank line terminating the headers is not contained in the list.
\end{datadesc}

\begin{datadesc}{fp}
The file object passed at instantiation time.
\end{datadesc}

\section{Standard Module \sectcode{mimetools}}
\stmodindex{mimetools}

\renewcommand{\indexsubitem}{(in module mimetools)}

This module defines a subclass of the class \code{rfc822.Message} and
a number of utility functions that are useful for the manipulation for
MIME style multipart or encoded message.

It defines the following items:

\begin{funcdesc}{Message}{fp}
Return a new instance of the \code{mimetools.Message} class.  This is
a subclass of the \code{rfc822.Message} class, with some additional
methods (see below).
\end{funcdesc}

\begin{funcdesc}{choose_boundary}{}
Return a unique string that has a high likelihood of being usable as a
part boundary.  The string has the form
\code{"\var{hostipaddr}.\var{uid}.\var{pid}.\var{timestamp}.\var{random}"}.
\end{funcdesc}

\begin{funcdesc}{decode}{input\, output\, encoding}
Read data encoded using the allowed MIME \var{encoding} from open file
object \var{input} and write the decoded data to open file object
\var{output}.  Valid values for \var{encoding} include
\code{"base64"}, \code{"quoted-printable"} and \code{"uuencode"}.
\end{funcdesc}

\begin{funcdesc}{encode}{input\, output\, encoding}
Read data from open file object \var{input} and write it encoded using
the allowed MIME \var{encoding} to open file object \var{output}.
Valid values for \var{encoding} are the same as for \code{decode()}.
\end{funcdesc}

\begin{funcdesc}{copyliteral}{input\, output}
Read lines until EOF from open file \var{input} and write them to open
file \var{output}.
\end{funcdesc}

\begin{funcdesc}{copybinary}{input\, output}
Read blocks until EOF from open file \var{input} and write them to open
file \var{output}.  The block size is currently fixed at 8192.
\end{funcdesc}


\subsection{Additional Methods of Message objects}
\nodename{mimetools.Message Methods}

The \code{mimetools.Message} class defines the following methods in
addition to the \code{rfc822.Message} class:

\renewcommand{\indexsubitem}{(mimetool.Message method)}

\begin{funcdesc}{getplist}{}
Return the parameter list of the \code{Content-type} header.  This is
a list if strings.  For parameters of the form
\samp{\var{key}=\var{value}}, \var{key} is converted to lower case but
\var{value} is not.  For example, if the message contains the header
\samp{Content-type: text/html; spam=1; Spam=2; Spam} then
\code{getplist()} will return the Python list \code{['spam=1',
'spam=2', 'Spam']}.
\end{funcdesc}

\begin{funcdesc}{getparam}{name}
Return the \var{value} of the first parameter (as returned by
\code{getplist()} of the form \samp{\var{name}=\var{value}} for the
given \var{name}.  If \var{value} is surrounded by quotes of the form
\var{<...>} or \var{"..."}, these are removed.
\end{funcdesc}

\begin{funcdesc}{getencoding}{}
Return the encoding specified in the \samp{Content-transfer-encoding}
message header.  If no such header exists, return \code{"7bit"}.  The
encoding is converted to lower case.
\end{funcdesc}

\begin{funcdesc}{gettype}{}
Return the message type (of the form \samp{\var{type}/var{subtype}})
as specified in the \samp{Content-type} header.  If no such header
exists, return \code{"text/plain"}.  The type is converted to lower
case.
\end{funcdesc}

\begin{funcdesc}{getmaintype}{}
Return the main type as specified in the \samp{Content-type} header.
If no such header exists, return \code{"text"}.  The main type is
converted to lower case.
\end{funcdesc}

\begin{funcdesc}{getsubtype}{}
Return the subtype as specified in the \samp{Content-type} header.  If
no such header exists, return \code{"plain"}.  The subtype is
converted to lower case.
\end{funcdesc}

\section{Standard module \sectcode{binhex}}
\stmodindex{binhex}

This module encodes and decodes files in binhex4 format, a format
allowing representation of Macintosh files in ASCII. On the macintosh,
both forks of a file and the finder information are encoded (or
decoded), on other platforms only the data fork is handled.

The \code{binhex} module defines the following functions:

\renewcommand{\indexsubitem}{(in module binhex)}

\begin{funcdesc}{binhex}{input\, output}
Convert a binary file with filename \var{input} to binhex file
\var{output}. The \var{output} parameter can either be a filename or a
file-like object (any object supporting a \var{write} and \var{close}
method).
\end{funcdesc}

\begin{funcdesc}{hexbin}{input\optional{\, output}}
Decode a binhex file \var{input}. \var{Input} may be a filename or a
file-like object supporting \var{read} and \var{close} methods.
The resulting file is written to a file named \var{output}, unless the
argument is empty in which case the output filename is read from the
binhex file.
\end{funcdesc}

\subsection{notes}
There is an alternative, more powerful interface to the coder and
decoder, see the source for details.

If you code or decode textfiles on non-Macintosh platforms they will
still use the macintosh newline convention (carriage-return as end of
line).

As of this writing, \var{hexbin} appears to not work in all cases.

\section{Standard module \sectcode{uu}}
\stmodindex{uu}

This module encodes and decodes files in uuencode format, allowing
arbitrary binary data to be transferred over ascii-only connections.
Whereever a file argument is expected, the methods accept either a
pathname (\code{'-'} for stdin/stdout) or a file-like object.

Normally you would pass filenames, but there is one case where you
have to open the file yourself: if you are on a non-unix platform and
your binary file is actually a textfile that you want encoded
unix-compatible you will have to open the file yourself as a textfile,
so newline conversion is performed.

This code was contributed by Lance Ellinghouse, and modified by Jack
Jansen.

The \code{uu} module defines the following functions:

\renewcommand{\indexsubitem}{(in module uu)}

\begin{funcdesc}{encode}{in_file\, out_file\optional{\, name\, mode}}
Uuencode file \var{in_file} into file \var{out_file}.  The uuencoded
file will have the header specifying \var{name} and \var{mode} as the
defaults for the results of decoding the file. The default defaults
are taken from \var{in_file}, or \code{'-'} and \code{0666}
respectively. 
\end{funcdesc}

\begin{funcdesc}{decode}{in_file\optional{\, out_file\, mode}}
This call decodes uuencoded file \var{in_file} placing the result on
file \var{out_file}. If \var{out_file} is a pathname the \var{mode} is
also set. Defaults for \var{out_file} and \var{mode} are taken from
the uuencode header.
\end{funcdesc}

\section{Built-in Module \sectcode{binascii}}	% If implemented in C
\bimodindex{binascii}

The binascii module contains a number of methods to convert between
binary and various ascii-encoded binary representations. Normally, you
will not use these modules directly but use wrapper modules like
\var{uu} or \var{hexbin} in stead, this module solely exists because
bit-manipuation of large amounts of data is slow in python.

The \code{binascii} module defines the following functions:

\renewcommand{\indexsubitem}{(in module binascii)}

\begin{funcdesc}{a2b_uu}{string}
Convert a single line of uuencoded data back to binary and return the
binary data. Lines normally contain 45 (binary) bytes, except for the
last line. Line data may be followed by whitespace.
\end{funcdesc}

\begin{funcdesc}{b2a_uu}{data}
Convert binary data to a line of ascii characters, the return value is
the converted line, including a newline char. The length of \var{data}
should be at most 45.
\end{funcdesc}

\begin{funcdesc}{a2b_base64}{string}
Convert a block of base64 data back to binary and return the
binary data. More than one line may be passed at a time.
\end{funcdesc}

\begin{funcdesc}{b2a_base64}{data}
Convert binary data to a line of ascii characters in base64 coding.
The return value is the converted line, including a newline char.
The length of \var{data} should be at most 57 to adhere to the base64
standard.
\end{funcdesc}

\begin{funcdesc}{a2b_hqx}{string}
Convert binhex4 formatted ascii data to binary, without doing
rle-decompression. The string should contain a complete number of
binary bytes, or (in case of the last portion of the binhex4 data)
have the remaining bits zero.
\end{funcdesc}

\begin{funcdesc}{rledecode_hqx}{data}
Perform RLE-decompression on the data, as per the binhex4
standard. The algorithm uses \code{0x90} after a byte as a repeat
indicator, followed by a count. A count of \code{0} specifies a byte
value of \code{0x90}. The routine returns the decompressed data,
unless data input data ends in an orphaned repeat indicator, in which
case the \var{Incomplete} exception is raised.
\end{funcdesc}

\begin{funcdesc}{rlecode_hqx}{data}
Perform binhex4 style RLE-compression on \var{data} and return the
result.
\end{funcdesc}

\begin{funcdesc}{b2a_hqx}{data}
Perform hexbin4 binary-to-ascii translation and return the resulting
string. The argument should already be rle-coded, and have a length
divisible by 3 (except possibly the last fragment).
\end{funcdesc}

\begin{funcdesc}{crc_hqx}{data, crc}
Compute the binhex4 crc value of \var{data}, starting with an initial
\var{crc} and returning the result.
\end{funcdesc}
 
\begin{excdesc}{Error}
Exception raised on errors. These are usually programming errors.
\end{excdesc}

\begin{excdesc}{Incomplete}
Exception raised on incomplete data. These are usually not programming
errors, but handled by reading a little more data and trying again.
\end{excdesc}

\section{Standard module \sectcode{xdrlib}}
\stmodindex{xdrlib}
\index{XDR}

\renewcommand{\indexsubitem}{(in module xdrlib)}


The \code{xdrlib} module supports the External Data Representation
Standard as described in RFC 1014, written by Sun Microsystems,
Inc. June 1987.  It supports most of the data types described in the
RFC, although some, most notably \code{float} and \code{double} are
only supported on those operating systems that provide an XDR
library.

The \code{xdrlib} module defines two classes, one for packing
variables into XDR representation, and another for unpacking from XDR
representation.  There are also two exception classes.


\subsection{Packer Objects}

\code{Packer} is the class for packing data into XDR representation.
The \code{Packer} class is instantiated with no arguments.

\begin{funcdesc}{get_buffer}{}
Returns the current pack buffer as a string.
\end{funcdesc}

\begin{funcdesc}{reset}{}
Resets the pack buffer to the empty string.
\end{funcdesc}

In general, you can pack any of the most common XDR data types by
calling the appropriate \code{pack_\var{type}} method.  Each method
takes a single argument, the value to pack.  The following simple data
type packing methods are supported: \code{pack_uint}, \code{pack_int},
\code{pack_enum}, \code{pack_bool}, \code{pack_uhyper},
and \code{pack_hyper}.

The following methods pack floating point numbers, however they
require C library support.  Without the optional C built-in module,
both of these methods will raise an \code{xdrlib.ConversionError}
exception.  See the note at the end of this chapter for details.

\begin{funcdesc}{pack_float}{value}
Packs the single-precision floating point number \var{value}.
\end{funcdesc}

\begin{funcdesc}{pack_double}{value}
Packs the double-precision floating point number \var{value}.
\end{funcdesc}

The following methods support packing strings, bytes, and opaque data:

\begin{funcdesc}{pack_fstring}{n\, s}
Packs a fixed length string, \var{s}.  \var{n} is the length of the
string but it is \emph{not} packed into the data buffer.  The string
is padded with null bytes if necessary to guaranteed 4 byte alignment.
\end{funcdesc}

\begin{funcdesc}{pack_fopaque}{n\, data}
Packs a fixed length opaque data stream, similarly to
\code{pack_fstring}.
\end{funcdesc}

\begin{funcdesc}{pack_string}{s}
Packs a variable length string, \var{s}.  The length of the string is
first packed as an unsigned integer, then the string data is packed
with \code{pack_fstring}.
\end{funcdesc}

\begin{funcdesc}{pack_opaque}{data}
Packs a variable length opaque data string, similarly to
\code{pack_string}.
\end{funcdesc}

\begin{funcdesc}{pack_bytes}{bytes}
Packs a variable length byte stream, similarly to \code{pack_string}.
\end{funcdesc}

The following methods support packing arrays and lists:

\begin{funcdesc}{pack_list}{list\, pack_item}
Packs a \var{list} of homogeneous items.  This method is useful for
lists with an indeterminate size; i.e. the size is not available until
the entire list has been walked.  For each item in the list, an
unsigned integer \code{1} is packed first, followed by the data value
from the list.  \var{pack_item} is the function that is called to pack
the individual item.  At the end of the list, an unsigned integer
\code{0} is packed.
\end{funcdesc}

\begin{funcdesc}{pack_farray}{n\, array\, pack_item}
Packs a fixed length list (\var{array}) of homogeneous items.  \var{n}
is the length of the list; it is \emph{not} packed into the buffer,
but a \code{ValueError} exception is raised if \code{len(array)} is not
equal to \var{n}.  As above, \var{pack_item} is the function used to
pack each element.
\end{funcdesc}

\begin{funcdesc}{pack_array}{list\, pack_item}
Packs a variable length \var{list} of homogeneous items.  First, the
length of the list is packed as an unsigned integer, then each element
is packed as in \code{pack_farray} above.
\end{funcdesc}

\subsection{Unpacker Objects}

\code{Unpacker} is the complementary class which unpacks XDR data
values from a string buffer, and has the following methods:

\begin{funcdesc}{__init__}{data}
Instantiates an \code{Unpacker} object with the string buffer
\var{data}.
\end{funcdesc}

\begin{funcdesc}{reset}{data}
Resets the string buffer with the given \var{data}.
\end{funcdesc}

\begin{funcdesc}{get_position}{}
Returns the current unpack position in the data buffer.
\end{funcdesc}

\begin{funcdesc}{set_position}{position}
Sets the data buffer unpack position to \var{position}.  You should be
careful about using \code{get_position()} and \code{set_position()}.
\end{funcdesc}

\begin{funcdesc}{done}{}
Indicates unpack completion.  Raises an \code{xdrlib.Error} exception
if all of the data has not been unpacked.
\end{funcdesc}

In addition, every data type that can be packed with a \code{Packer},
can be unpacked with an \code{Unpacker}.  Unpacking methods are of the
form \code{unpack_\var{type}}, and take no arguments.  They return the
unpacked object.  The same caveats apply for \code{unpack_float} and
\code{unpack_double} as above.

\begin{funcdesc}{unpack_float}{}
Unpacks a single-precision floating point number.
\end{funcdesc}

\begin{funcdesc}{unpack_double}{}
Unpacks a double-precision floating point number, similarly to
\code{unpack_float}.
\end{funcdesc}

In addition, the following methods unpack strings, bytes, and opaque
data:

\begin{funcdesc}{unpack_fstring}{n}
Unpacks and returns a fixed length string.  \var{n} is the number of
characters expected.  Padding with null bytes to guaranteed 4 byte
alignment is assumed.
\end{funcdesc}

\begin{funcdesc}{unpack_fopaque}{n}
Unpacks and returns a fixed length opaque data stream, similarly to
\code{unpack_fstring}.
\end{funcdesc}

\begin{funcdesc}{unpack_string}{}
Unpacks and returns a variable length string.  The length of the
string is first unpacked as an unsigned integer, then the string data
is unpacked with \code{unpack_fstring}.
\end{funcdesc}

\begin{funcdesc}{unpack_opaque}{}
Unpacks and returns a variable length opaque data string, similarly to
\code{unpack_string}.
\end{funcdesc}

\begin{funcdesc}{unpack_bytes}{}
Unpacks and returns a variable length byte stream, similarly to
\code{unpack_string}.
\end{funcdesc}

The following methods support unpacking arrays and lists:

\begin{funcdesc}{unpack_list}{unpack_item}
Unpacks and returns a list of homogeneous items.  The list is unpacked
one element at a time
by first unpacking an unsigned integer flag.  If the flag is \code{1},
then the item is unpacked and appended to the list.  A flag of
\code{0} indicates the end of the list.  \var{unpack_item} is the
function that is called to unpack the items.
\end{funcdesc}

\begin{funcdesc}{unpack_farray}{n\, unpack_item}
Unpacks and returns (as a list) a fixed length array of homogeneous
items.  \var{n} is number of list elements to expect in the buffer.
As above, \var{unpack_item} is the function used to unpack each element.
\end{funcdesc}

\begin{funcdesc}{unpack_array}{unpack_item}
Unpacks and returns a variable length \var{list} of homogeneous items.
First, the length of the list is unpacked as an unsigned integer, then
each element is unpacked as in \code{unpack_farray} above.
\end{funcdesc}

\subsection{Exceptions}

Exceptions in this module are coded as class instances:

\begin{excdesc}{Error}
The base exception class.  \code{Error} has a single public data
member \code{msg} containing the description of the error.
\end{excdesc}

\begin{excdesc}{ConversionError}
Class derived from \code{Error}.  Contains no additional instance
variables.
\end{excdesc}

Here is an example of how you would catch one of these exceptions:

\begin{verbatim}
import xdrlib
p = xdrlib.Packer()
try:
    p.pack_double(8.01)
except xdrlib.ConversionError, instance:
    print 'packing the double failed:', instance.msg
\end{verbatim}

\subsection{Supporting Floating Point Data}

Packing and unpacking floating point data,
i.e. \code{Packer.pack_float}, \code{Packer.pack_double},
\code{Unpacker.unpack_float}, and \code{Unpacker.unpack_double}, are
only supported with the helper built-in \code{_xdr} module, which
relies on your operating system having the appropriate XDR library
routines.

If you have built the Python interpeter with the \code{_xdr} module,
or have built the \code{_xdr} module as a shared library,
\code{xdrlib} will use these to pack and unpack floating point
numbers.  Otherwise, using these routines will raise a
\code{ConversionError} exception.

See the Python installation instructions for details on building the
\code{_xdr} module.


\chapter{Restricted Execution}

In general, Python programs have complete access to the underlying
operating system throug the various functions and classes, For
example, a Python program can open any file for reading and writing by
using the \code{open()} built-in function (provided the underlying OS
gives you permission!).  This is exactly what you want for most
applications.

There exists a class of applications for which this ``openness'' is
inappropriate.  Take Grail: a web browser that accepts ``applets'',
snippets of Python code, from anywhere on the Internet for execution
on the local system.  This can be used to improve the user interface
of forms, for instance.  Since the originator of the code is unknown,
it is obvious that it cannot be trusted with the full resources of the
local machine.

\emph{Restricted execution} is the basic framework in Python that allows
for the segregation of trusted and untrusted code.  It is based on the
notion that trusted Python code (a \emph{supervisor}) can create a
``padded cell' (or environment) with limited permissions, and run the
untrusted code within this cell.  The untrusted code cannot break out
of its cell, and can only interact with sensitive system resources
through interfaces defined and managed by the trusted code.  The term
``restricted execution'' is favored over ``safe-Python''
since true safety is hard to define, and is determined by the way the
restricted environment is created.  Note that the restricted
environments can be nested, with inner cells creating subcells of
lesser, but never greater, privilege.

An interesting aspect of Python's restricted execution model is that
the interfaces presented to untrusted code usually have the same names
as those presented to trusted code.  Therefore no special interfaces
need to be learned to write code designed to run in a restricted
environment.  And because the exact nature of the padded cell is
determined by the supervisor, different restrictions can be imposed,
depending on the application.  For example, it might be deemed
``safe'' for untrusted code to read any file within a specified
directory, but never to write a file.  In this case, the supervisor
may redefine the built-in
\code{open()} function so that it raises an exception whenever the
\var{mode} parameter is \code{'w'}.  It might also perform a
\code{chroot()}-like operation on the \var{filename} parameter, such
that root is always relative to some safe ``sandbox'' area of the
filesystem.  In this case, the untrusted code would still see an
built-in \code{open()} function in its environment, with the same
calling interface.  The semantics would be identical too, with
\code{IOError}s being raised when the supervisor determined that an
unallowable parameter is being used.

The Python run-time determines whether a particular code block is
executing in restricted execution mode based on the identity of the
\code{__builtins__} object in its global variables: if this is (the
dictionary of) the standard \code{__builtin__} module, the code is
deemed to be unrestricted, else it is deemed to be restricted.

Python code executing in restricted mode faces a number of limitations
that are designed to prevent it from escaping from the padded cell.
For instance, the function object attribute \code{func_globals} and the
class and instance object attribute \code{__dict__} are unavailable.

Two modules provide the framework for setting up restricted execution
environments:

\begin{description}

\item[rexec]
--- Basic restricted execution framework.

\item[Bastion]
--- Providing restricted access to objects.

\end{description}

\section{Standard Module \sectcode{rexec}}
\stmodindex{rexec}
\renewcommand{\indexsubitem}{(in module rexec)}

This module contains the \code{RExec} class, which supports
\code{r_exec()}, \code{r_eval()}, \code{r_execfile()}, and
\code{r_import()} methods, which are restricted versions of the standard
Python functions \code{exec()}, \code{eval()}, \code{execfile()}, and
the \code{import} statement.
Code executed in this restricted environment will
only have access to modules and functions that are deemed safe; you
can subclass \code{RExec} to add or remove capabilities as desired.

\emph{Note:} The \code{RExec} class can prevent code from performing
unsafe operations like reading or writing disk files, or using TCP/IP
sockets.  However, it does not protect against code using extremely
large amounts of memory or CPU time.  

\begin{funcdesc}{RExec}{\optional{hooks\optional{\, verbose}}}
Returns an instance of the \code{RExec} class.  

\var{hooks} is an instance of the \code{RHooks} class or a subclass of it.
If it is omitted or \code{None}, the default \code{RHooks} class is
instantiated.
Whenever the RExec module searches for a module (even a built-in one)
or reads a module's code, it doesn't actually go out to the file
system itself.  Rather, it calls methods of an RHooks instance that
was passed to or created by its constructor.  (Actually, the RExec
object doesn't make these calls---they are made by a module loader
object that's part of the RExec object.  This allows another level of
flexibility, e.g. using packages.)

By providing an alternate RHooks object, we can control the
file system accesses made to import a module, without changing the
actual algorithm that controls the order in which those accesses are
made.  For instance, we could substitute an RHooks object that passes
all filesystem requests to a file server elsewhere, via some RPC
mechanism such as ILU.  Grail's applet loader uses this to support
importing applets from a URL for a directory.

If \var{verbose} is true, additional debugging output may be sent to
standard output.
\end{funcdesc}

The RExec class has the following class attributes, which are used by the
\code{__init__} method.  Changing them on an existing instance won't
have any effect; instead, create a subclass of \code{RExec} and assign
them new values in the class definition.  Instances of the new class
will then use those new values.  All these attributes are tuples of
strings.

\renewcommand{\indexsubitem}{(RExec object attribute)}
\begin{datadesc}{nok_builtin_names}
Contains the names of built-in functions which will \emph{not} be
available to programs running in the restricted environment.  The
value for \code{RExec} is \code{('open',} \code{'reload',}
\code{'__import__')}.  (This gives the exceptions, because by far the
majority of built-in functions are harmless.  A subclass that wants to
override this variable should probably start with the value from the
base class and concatenate additional forbidden functions --- when new
dangerous built-in functions are added to Python, they will also be
added to this module.)
\end{datadesc}

\begin{datadesc}{ok_builtin_modules}
Contains the names of built-in modules which can be safely imported.
The value for \code{RExec} is \code{('audioop',} \code{'array',}
\code{'binascii',} \code{'cmath',} \code{'errno',} \code{'imageop',}
\code{'marshal',} \code{'math',} \code{'md5',} \code{'operator',}
\code{'parser',} \code{'regex',} \code{'rotor',} \code{'select',}
\code{'strop',} \code{'struct',} \code{'time')}.  A similar remark
about overriding this variable applies --- use the value from the base
class as a starting point.
\end{datadesc}

\begin{datadesc}{ok_path}
Contains the directories which will be searched when an \code{import}
is performed in the restricted environment.  
The value for \code{RExec} is the same as \code{sys.path} (at the time
the module is loaded) for unrestricted code.
\end{datadesc}

\begin{datadesc}{ok_posix_names}
% Should this be called ok_os_names?
Contains the names of the functions in the \code{os} module which will be
available to programs running in the restricted environment.  The
value for \code{RExec} is \code{('error',} \code{'fstat',}
\code{'listdir',} \code{'lstat',} \code{'readlink',} \code{'stat',}
\code{'times',} \code{'uname',} \code{'getpid',} \code{'getppid',}
\code{'getcwd',} \code{'getuid',} \code{'getgid',} \code{'geteuid',}
\code{'getegid')}.
\end{datadesc}

\begin{datadesc}{ok_sys_names}
Contains the names of the functions and variables in the \code{sys}
module which will be available to programs running in the restricted
environment.  The value for \code{RExec} is \code{('ps1',}
\code{'ps2',} \code{'copyright',} \code{'version',} \code{'platform',}
\code{'exit',} \code{'maxint')}.
\end{datadesc}

RExec instances support the following methods:
\renewcommand{\indexsubitem}{(RExec object method)}

\begin{funcdesc}{r_eval}{code}
\var{code} must either be a string containing a Python expression, or
a compiled code object, which will be evaluated in the restricted
environment's \code{__main__} module.  The value of the expression or
code object will be returned.
\end{funcdesc}

\begin{funcdesc}{r_exec}{code}
\var{code} must either be a string containing one or more lines of
Python code, or a compiled code object, which will be executed in the
restricted environment's \code{__main__} module.
\end{funcdesc}

\begin{funcdesc}{r_execfile}{filename}
Execute the Python code contained in the file \var{filename} in the
restricted environment's \code{__main__} module.
\end{funcdesc}

Methods whose names begin with \code{s_} are similar to the functions
beginning with \code{r_}, but the code will be granted access to
restricted versions of the standard I/O streans \code{sys.stdin},
\code{sys.stderr}, and \code{sys.stdout}.  

\begin{funcdesc}{s_eval}{code}
\var{code} must be a string containing a Python expression, which will
be evaluated in the restricted environment.  
\end{funcdesc}

\begin{funcdesc}{s_exec}{code}
\var{code} must be a string containing one or more lines of Python code,
which will be executed in the restricted environment.  
\end{funcdesc}

\begin{funcdesc}{s_execfile}{code}
Execute the Python code contained in the file \var{filename} in the
restricted environment.
\end{funcdesc}

\code{RExec} objects must also support various methods which will be
implicitly called by code executing in the restricted environment.
Overriding these methods in a subclass is used to change the policies
enforced by a restricted environment.

\begin{funcdesc}{r_import}{modulename\optional{\, globals\, locals\, fromlist}}
Import the module \var{modulename}, raising an \code{ImportError}
exception if the module is considered unsafe.
\end{funcdesc}

\begin{funcdesc}{r_open}{filename\optional{\, mode\optional{\, bufsize}}}
Method called when \code{open()} is called in the restricted
environment.  The arguments are identical to those of \code{open()},
and a file object (or a class instance compatible with file objects)
should be returned.  \code{RExec}'s default behaviour is allow opening
any file for reading, but forbidding any attempt to write a file.  See
the example below for an implementation of a less restrictive
\code{r_open()}.
\end{funcdesc}

\begin{funcdesc}{r_reload}{module}
Reload the module object \var{module}, re-parsing and re-initializing it.  
\end{funcdesc}

\begin{funcdesc}{r_unload}{module}
Unload the module object \var{module} (i.e., remove it from the
restricted environment's \code{sys.modules} dictionary).
\end{funcdesc}

And their equivalents with access to restricted standard I/O streams:

\begin{funcdesc}{s_import}{modulename\optional{\, globals, locals, fromlist}}
Import the module \var{modulename}, raising an \code{ImportError}
exception if the module is considered unsafe.
\end{funcdesc}

\begin{funcdesc}{s_reload}{module}
Reload the module object \var{module}, re-parsing and re-initializing it.  
\end{funcdesc}

\begin{funcdesc}{s_unload}{module}
Unload the module object \var{module}.   
% XXX what are the semantics of this?  
\end{funcdesc}

\subsection{An example}

Let us say that we want a slightly more relaxed policy than the
standard RExec class.  For example, if we're willing to allow files in
\file{/tmp} to be written, we can subclass the \code{RExec} class:

\bcode\begin{verbatim}
class TmpWriterRExec(rexec.RExec):
    def r_open(self, file, mode='r', buf=-1):
        if mode in ('r', 'rb'):
            pass
        elif mode in ('w', 'wb', 'a', 'ab'):
            # check filename : must begin with /tmp/
            if file[:5]!='/tmp/': 
                raise IOError, "can't write outside /tmp"
            elif (string.find(file, '/../') >= 0 or
                 file[:3] == '../' or file[-3:] == '/..'):
                raise IOError, "'..' in filename forbidden"
        else: raise IOError, "Illegal open() mode"
        return open(file, mode, buf)
\end{verbatim}\ecode

Notice that the above code will occasionally forbid a perfectly valid
filename; for example, code in the restricted environment won't be
able to open a file called \file{/tmp/foo/../bar}.  To fix this, the
\code{r_open} method would have to simplify the filename to
\file{/tmp/bar}, which would require splitting apart the filename and
performing various operations on it.  In cases where security is at
stake, it may be preferable to write simple code which is sometimes
overly restrictive, instead of more general code that is also more
complex and may harbor a subtle security hole.

\section{Standard Module \sectcode{Bastion}}
\stmodindex{Bastion}
\renewcommand{\indexsubitem}{(in module Bastion)}

% I'm concerned that the word 'bastion' won't be understood by people
% for whom English is a second language, making the module name
% somewhat mysterious.  Thus, the brief definition... --amk

According to the dictionary, a bastion is ``a fortified area or
position'', or ``something that is considered a stronghold.''  It's a
suitable name for this module, which provides a way to forbid access
to certain attributes of an object.  It must always be used with the
\code{rexec} module, in order to allow restricted-mode programs access
to certain safe attributes of an object, while denying access to
other, unsafe attributes.

% I've punted on the issue of documenting keyword arguments for now.

\begin{funcdesc}{Bastion}{object\optional{\, filter\, name\, class}}
Protect the class instance \var{object}, returning a bastion for the
object.  Any attempt to access one of the object's attributes will
have to be approved by the \var{filter} function; if the access is
denied an AttributeError exception will be raised.

If present, \var{filter} must be a function that accepts a string
containing an attribute name, and returns true if access to that
attribute will be permitted; if \var{filter} returns false, the access
is denied.  The default filter denies access to any function beginning
with an underscore (\code{_}).  The bastion's string representation
will be \code{<Bastion for \var{name}>} if a value for
\var{name} is provided; otherwise, \code{repr(\var{object})} will be used.

\var{class}, if present, would be a subclass of \code{BastionClass};
see the code in \file{bastion.py} for the details.  Overriding the
default \code{BastionClass} will rarely be required.  

\end{funcdesc}


\chapter{Multimedia Services}

The modules described in this chapter implement various algorithms or
interfaces that are mainly useful for multimedia applications.  They
are available at the discretion of the installation.  Here's an overview:

\begin{description}

\item[audioop]
--- Manipulate raw audio data.

\item[imageop]
--- Manipulate raw image data.

\item[aifc]
--- Read and write audio files in AIFF or AIFC format.

\item[jpeg]
--- Read and write image files in compressed JPEG format.

\item[rgbimg]
--- Read and write image files in ``SGI RGB'' format (the module is
\emph{not} SGI specific though)!

\end{description}
			% Multimedia Services
\section{Built-in Module \sectcode{audioop}}
\bimodindex{audioop}

The \code{audioop} module contains some useful operations on sound fragments.
It operates on sound fragments consisting of signed integer samples
8, 16 or 32 bits wide, stored in Python strings.  This is the same
format as used by the \code{al} and \code{sunaudiodev} modules.  All
scalar items are integers, unless specified otherwise.

A few of the more complicated operations only take 16-bit samples,
otherwise the sample size (in bytes) is always a parameter of the operation.

The module defines the following variables and functions:

\renewcommand{\indexsubitem}{(in module audioop)}
\begin{excdesc}{error}
This exception is raised on all errors, such as unknown number of bytes
per sample, etc.
\end{excdesc}

\begin{funcdesc}{add}{fragment1\, fragment2\, width}
Return a fragment which is the addition of the two samples passed as
parameters.  \var{width} is the sample width in bytes, either
\code{1}, \code{2} or \code{4}.  Both fragments should have the same
length.
\end{funcdesc}

\begin{funcdesc}{adpcm2lin}{adpcmfragment\, width\, state}
Decode an Intel/DVI ADPCM coded fragment to a linear fragment.  See
the description of \code{lin2adpcm} for details on ADPCM coding.
Return a tuple \code{(\var{sample}, \var{newstate})} where the sample
has the width specified in \var{width}.
\end{funcdesc}

\begin{funcdesc}{adpcm32lin}{adpcmfragment\, width\, state}
Decode an alternative 3-bit ADPCM code.  See \code{lin2adpcm3} for
details.
\end{funcdesc}

\begin{funcdesc}{avg}{fragment\, width}
Return the average over all samples in the fragment.
\end{funcdesc}

\begin{funcdesc}{avgpp}{fragment\, width}
Return the average peak-peak value over all samples in the fragment.
No filtering is done, so the usefulness of this routine is
questionable.
\end{funcdesc}

\begin{funcdesc}{bias}{fragment\, width\, bias}
Return a fragment that is the original fragment with a bias added to
each sample.
\end{funcdesc}

\begin{funcdesc}{cross}{fragment\, width}
Return the number of zero crossings in the fragment passed as an
argument.
\end{funcdesc}

\begin{funcdesc}{findfactor}{fragment\, reference}
Return a factor \var{F} such that
\code{rms(add(fragment, mul(reference, -F)))} is minimal, i.e.,
return the factor with which you should multiply \var{reference} to
make it match as well as possible to \var{fragment}.  The fragments
should both contain 2-byte samples.

The time taken by this routine is proportional to \code{len(fragment)}. 
\end{funcdesc}

\begin{funcdesc}{findfit}{fragment\, reference}
This routine (which only accepts 2-byte sample fragments)

Try to match \var{reference} as well as possible to a portion of
\var{fragment} (which should be the longer fragment).  This is
(conceptually) done by taking slices out of \var{fragment}, using
\code{findfactor} to compute the best match, and minimizing the
result.  The fragments should both contain 2-byte samples.  Return a
tuple \code{(\var{offset}, \var{factor})} where \var{offset} is the
(integer) offset into \var{fragment} where the optimal match started
and \var{factor} is the (floating-point) factor as per
\code{findfactor}.
\end{funcdesc}

\begin{funcdesc}{findmax}{fragment\, length}
Search \var{fragment} for a slice of length \var{length} samples (not
bytes!)\ with maximum energy, i.e., return \var{i} for which
\code{rms(fragment[i*2:(i+length)*2])} is maximal.  The fragments
should both contain 2-byte samples.

The routine takes time proportional to \code{len(fragment)}.
\end{funcdesc}

\begin{funcdesc}{getsample}{fragment\, width\, index}
Return the value of sample \var{index} from the fragment.
\end{funcdesc}

\begin{funcdesc}{lin2lin}{fragment\, width\, newwidth}
Convert samples between 1-, 2- and 4-byte formats.
\end{funcdesc}

\begin{funcdesc}{lin2adpcm}{fragment\, width\, state}
Convert samples to 4 bit Intel/DVI ADPCM encoding.  ADPCM coding is an
adaptive coding scheme, whereby each 4 bit number is the difference
between one sample and the next, divided by a (varying) step.  The
Intel/DVI ADPCM algorithm has been selected for use by the IMA, so it
may well become a standard.

\code{State} is a tuple containing the state of the coder.  The coder
returns a tuple \code{(\var{adpcmfrag}, \var{newstate})}, and the
\var{newstate} should be passed to the next call of lin2adpcm.  In the
initial call \code{None} can be passed as the state.  \var{adpcmfrag}
is the ADPCM coded fragment packed 2 4-bit values per byte.
\end{funcdesc}

\begin{funcdesc}{lin2adpcm3}{fragment\, width\, state}
This is an alternative ADPCM coder that uses only 3 bits per sample.
It is not compatible with the Intel/DVI ADPCM coder and its output is
not packed (due to laziness on the side of the author).  Its use is
discouraged.
\end{funcdesc}

\begin{funcdesc}{lin2ulaw}{fragment\, width}
Convert samples in the audio fragment to U-LAW encoding and return
this as a Python string.  U-LAW is an audio encoding format whereby
you get a dynamic range of about 14 bits using only 8 bit samples.  It
is used by the Sun audio hardware, among others.
\end{funcdesc}

\begin{funcdesc}{minmax}{fragment\, width}
Return a tuple consisting of the minimum and maximum values of all
samples in the sound fragment.
\end{funcdesc}

\begin{funcdesc}{max}{fragment\, width}
Return the maximum of the {\em absolute value} of all samples in a
fragment.
\end{funcdesc}

\begin{funcdesc}{maxpp}{fragment\, width}
Return the maximum peak-peak value in the sound fragment.
\end{funcdesc}

\begin{funcdesc}{mul}{fragment\, width\, factor}
Return a fragment that has all samples in the original framgent
multiplied by the floating-point value \var{factor}.  Overflow is
silently ignored.
\end{funcdesc}

\begin{funcdesc}{reverse}{fragment\, width}
Reverse the samples in a fragment and returns the modified fragment.
\end{funcdesc}

\begin{funcdesc}{rms}{fragment\, width}
Return the root-mean-square of the fragment, i.e.
\iftexi
the square root of the quotient of the sum of all squared sample value,
divided by the sumber of samples.
\else
% in eqn: sqrt { sum S sub i sup 2  over n }
\begin{displaymath}
\catcode`_=8
\sqrt{\frac{\sum{{S_{i}}^{2}}}{n}}
\end{displaymath}
\fi
This is a measure of the power in an audio signal.
\end{funcdesc}

\begin{funcdesc}{tomono}{fragment\, width\, lfactor\, rfactor} 
Convert a stereo fragment to a mono fragment.  The left channel is
multiplied by \var{lfactor} and the right channel by \var{rfactor}
before adding the two channels to give a mono signal.
\end{funcdesc}

\begin{funcdesc}{tostereo}{fragment\, width\, lfactor\, rfactor}
Generate a stereo fragment from a mono fragment.  Each pair of samples
in the stereo fragment are computed from the mono sample, whereby left
channel samples are multiplied by \var{lfactor} and right channel
samples by \var{rfactor}.
\end{funcdesc}

\begin{funcdesc}{ulaw2lin}{fragment\, width}
Convert sound fragments in ULAW encoding to linearly encoded sound
fragments.  ULAW encoding always uses 8 bits samples, so \var{width}
refers only to the sample width of the output fragment here.
\end{funcdesc}

Note that operations such as \code{mul} or \code{max} make no
distinction between mono and stereo fragments, i.e.\ all samples are
treated equal.  If this is a problem the stereo fragment should be split
into two mono fragments first and recombined later.  Here is an example
of how to do that:
\bcode\begin{verbatim}
def mul_stereo(sample, width, lfactor, rfactor):
    lsample = audioop.tomono(sample, width, 1, 0)
    rsample = audioop.tomono(sample, width, 0, 1)
    lsample = audioop.mul(sample, width, lfactor)
    rsample = audioop.mul(sample, width, rfactor)
    lsample = audioop.tostereo(lsample, width, 1, 0)
    rsample = audioop.tostereo(rsample, width, 0, 1)
    return audioop.add(lsample, rsample, width)
\end{verbatim}\ecode

If you use the ADPCM coder to build network packets and you want your
protocol to be stateless (i.e.\ to be able to tolerate packet loss)
you should not only transmit the data but also the state.  Note that
you should send the \var{initial} state (the one you passed to
\code{lin2adpcm}) along to the decoder, not the final state (as returned by
the coder).  If you want to use \code{struct} to store the state in
binary you can code the first element (the predicted value) in 16 bits
and the second (the delta index) in 8.

The ADPCM coders have never been tried against other ADPCM coders,
only against themselves.  It could well be that I misinterpreted the
standards in which case they will not be interoperable with the
respective standards.

The \code{find...} routines might look a bit funny at first sight.
They are primarily meant to do echo cancellation.  A reasonably
fast way to do this is to pick the most energetic piece of the output
sample, locate that in the input sample and subtract the whole output
sample from the input sample:
\bcode\begin{verbatim}
def echocancel(outputdata, inputdata):
    pos = audioop.findmax(outputdata, 800)    # one tenth second
    out_test = outputdata[pos*2:]
    in_test = inputdata[pos*2:]
    ipos, factor = audioop.findfit(in_test, out_test)
    # Optional (for better cancellation):
    # factor = audioop.findfactor(in_test[ipos*2:ipos*2+len(out_test)], 
    #              out_test)
    prefill = '\0'*(pos+ipos)*2
    postfill = '\0'*(len(inputdata)-len(prefill)-len(outputdata))
    outputdata = prefill + audioop.mul(outputdata,2,-factor) + postfill
    return audioop.add(inputdata, outputdata, 2)
\end{verbatim}\ecode

\section{Built-in Module \sectcode{imageop}}
\bimodindex{imageop}

The \code{imageop} module contains some useful operations on images.
It operates on images consisting of 8 or 32 bit pixels
stored in Python strings.  This is the same format as used
by \code{gl.lrectwrite} and the \code{imgfile} module.

The module defines the following variables and functions:

\renewcommand{\indexsubitem}{(in module imageop)}

\begin{excdesc}{error}
This exception is raised on all errors, such as unknown number of bits
per pixel, etc.
\end{excdesc}


\begin{funcdesc}{crop}{image\, psize\, width\, height\, x0\, y0\, x1\, y1}
Return the selected part of \var{image}, which should by
\var{width} by \var{height} in size and consist of pixels of
\var{psize} bytes. \var{x0}, \var{y0}, \var{x1} and \var{y1} are like
the \code{lrectread} parameters, i.e.\ the boundary is included in the
new image.  The new boundaries need not be inside the picture.  Pixels
that fall outside the old image will have their value set to zero.  If
\var{x0} is bigger than \var{x1} the new image is mirrored.  The same
holds for the y coordinates.
\end{funcdesc}

\begin{funcdesc}{scale}{image\, psize\, width\, height\, newwidth\, newheight}
Return \var{image} scaled to size \var{newwidth} by \var{newheight}.
No interpolation is done, scaling is done by simple-minded pixel
duplication or removal.  Therefore, computer-generated images or
dithered images will not look nice after scaling.
\end{funcdesc}

\begin{funcdesc}{tovideo}{image\, psize\, width\, height}
Run a vertical low-pass filter over an image.  It does so by computing
each destination pixel as the average of two vertically-aligned source
pixels.  The main use of this routine is to forestall excessive
flicker if the image is displayed on a video device that uses
interlacing, hence the name.
\end{funcdesc}

\begin{funcdesc}{grey2mono}{image\, width\, height\, threshold}
Convert a 8-bit deep greyscale image to a 1-bit deep image by
tresholding all the pixels.  The resulting image is tightly packed and
is probably only useful as an argument to \code{mono2grey}.
\end{funcdesc}

\begin{funcdesc}{dither2mono}{image\, width\, height}
Convert an 8-bit greyscale image to a 1-bit monochrome image using a
(simple-minded) dithering algorithm.
\end{funcdesc}

\begin{funcdesc}{mono2grey}{image\, width\, height\, p0\, p1}
Convert a 1-bit monochrome image to an 8 bit greyscale or color image.
All pixels that are zero-valued on input get value \var{p0} on output
and all one-value input pixels get value \var{p1} on output.  To
convert a monochrome black-and-white image to greyscale pass the
values \code{0} and \code{255} respectively.
\end{funcdesc}

\begin{funcdesc}{grey2grey4}{image\, width\, height}
Convert an 8-bit greyscale image to a 4-bit greyscale image without
dithering.
\end{funcdesc}

\begin{funcdesc}{grey2grey2}{image\, width\, height}
Convert an 8-bit greyscale image to a 2-bit greyscale image without
dithering.
\end{funcdesc}

\begin{funcdesc}{dither2grey2}{image\, width\, height}
Convert an 8-bit greyscale image to a 2-bit greyscale image with
dithering.  As for \code{dither2mono}, the dithering algorithm is
currently very simple.
\end{funcdesc}

\begin{funcdesc}{grey42grey}{image\, width\, height}
Convert a 4-bit greyscale image to an 8-bit greyscale image.
\end{funcdesc}

\begin{funcdesc}{grey22grey}{image\, width\, height}
Convert a 2-bit greyscale image to an 8-bit greyscale image.
\end{funcdesc}

\section{Standard Module \sectcode{aifc}}
\stmodindex{aifc}

This module provides support for reading and writing AIFF and AIFF-C
files.  AIFF is Audio Interchange File Format, a format for storing
digital audio samples in a file.  AIFF-C is a newer version of the
format that includes the ability to compress the audio data.

Audio files have a number of parameters that describe the audio data.
The sampling rate or frame rate is the number of times per second the
sound is sampled.  The number of channels indicate if the audio is
mono, stereo, or quadro.  Each frame consists of one sample per
channel.  The sample size is the size in bytes of each sample.  Thus a
frame consists of \var{nchannels}*\var{samplesize} bytes, and a
second's worth of audio consists of
\var{nchannels}*\var{samplesize}*\var{framerate} bytes.

For example, CD quality audio has a sample size of two bytes (16
bits), uses two channels (stereo) and has a frame rate of 44,100
frames/second.  This gives a frame size of 4 bytes (2*2), and a
second's worth occupies 2*2*44100 bytes, i.e.\ 176,400 bytes.

Module \code{aifc} defines the following function:

\renewcommand{\indexsubitem}{(in module aifc)}
\begin{funcdesc}{open}{file\, mode}
Open an AIFF or AIFF-C file and return an object instance with
methods that are described below.  The argument file is either a
string naming a file or a file object.  The mode is either the string
\code{'r'} when the file must be opened for reading, or \code{'w'}
when the file must be opened for writing.  When used for writing, the
file object should be seekable, unless you know ahead of time how many
samples you are going to write in total and use
\code{writeframesraw()} and \code{setnframes()}.
\end{funcdesc}

Objects returned by \code{aifc.open()} when a file is opened for
reading have the following methods:

\renewcommand{\indexsubitem}{(aifc object method)}
\begin{funcdesc}{getnchannels}{}
Return the number of audio channels (1 for mono, 2 for stereo).
\end{funcdesc}

\begin{funcdesc}{getsampwidth}{}
Return the size in bytes of individual samples.
\end{funcdesc}

\begin{funcdesc}{getframerate}{}
Return the sampling rate (number of audio frames per second).
\end{funcdesc}

\begin{funcdesc}{getnframes}{}
Return the number of audio frames in the file.
\end{funcdesc}

\begin{funcdesc}{getcomptype}{}
Return a four-character string describing the type of compression used
in the audio file.  For AIFF files, the returned value is
\code{'NONE'}.
\end{funcdesc}

\begin{funcdesc}{getcompname}{}
Return a human-readable description of the type of compression used in
the audio file.  For AIFF files, the returned value is \code{'not
compressed'}.
\end{funcdesc}

\begin{funcdesc}{getparams}{}
Return a tuple consisting of all of the above values in the above
order.
\end{funcdesc}

\begin{funcdesc}{getmarkers}{}
Return a list of markers in the audio file.  A marker consists of a
tuple of three elements.  The first is the mark ID (an integer), the
second is the mark position in frames from the beginning of the data
(an integer), the third is the name of the mark (a string).
\end{funcdesc}

\begin{funcdesc}{getmark}{id}
Return the tuple as described in \code{getmarkers} for the mark with
the given id.
\end{funcdesc}

\begin{funcdesc}{readframes}{nframes}
Read and return the next \var{nframes} frames from the audio file.  The
returned data is a string containing for each frame the uncompressed
samples of all channels.
\end{funcdesc}

\begin{funcdesc}{rewind}{}
Rewind the read pointer.  The next \code{readframes} will start from
the beginning.
\end{funcdesc}

\begin{funcdesc}{setpos}{pos}
Seek to the specified frame number.
\end{funcdesc}

\begin{funcdesc}{tell}{}
Return the current frame number.
\end{funcdesc}

\begin{funcdesc}{close}{}
Close the AIFF file.  After calling this method, the object can no
longer be used.
\end{funcdesc}

Objects returned by \code{aifc.open()} when a file is opened for
writing have all the above methods, except for \code{readframes} and
\code{setpos}.  In addition the following methods exist.  The
\code{get} methods can only be called after the corresponding
\code{set} methods have been called.  Before the first
\code{writeframes} or \code{writeframesraw}, all parameters except for
the number of frames must be filled in.

\begin{funcdesc}{aiff}{}
Create an AIFF file.  The default is that an AIFF-C file is created,
unless the name of the file ends in '.aiff' in which case the default
is an AIFF file.
\end{funcdesc}

\begin{funcdesc}{aifc}{}
Create an AIFF-C file.  The default is that an AIFF-C file is created,
unless the name of the file ends in '.aiff' in which case the default
is an AIFF file.
\end{funcdesc}

\begin{funcdesc}{setnchannels}{nchannels}
Specify the number of channels in the audio file.
\end{funcdesc}

\begin{funcdesc}{setsampwidth}{width}
Specify the size in bytes of audio samples.
\end{funcdesc}

\begin{funcdesc}{setframerate}{rate}
Specify the sampling frequency in frames per second.
\end{funcdesc}

\begin{funcdesc}{setnframes}{nframes}
Specify the number of frames that are to be written to the audio file.
If this parameter is not set, or not set correctly, the file needs to
support seeking.
\end{funcdesc}

\begin{funcdesc}{setcomptype}{type\, name}
Specify the compression type.  If not specified, the audio data will
not be compressed.  In AIFF files, compression is not possible.  The
name parameter should be a human-readable description of the
compression type, the type parameter should be a four-character
string.  Currently the following compression types are supported:
NONE, ULAW, ALAW, G722.
\end{funcdesc}

\begin{funcdesc}{setparams}{nchannels\, sampwidth\, framerate\, comptype\, compname}
Set all the above parameters at once.  The argument is a tuple
consisting of the various parameters.  This means that it is possible
to use the result of a \code{getparams} call as argument to
\code{setparams}.
\end{funcdesc}

\begin{funcdesc}{setmark}{id\, pos\, name}
Add a mark with the given id (larger than 0), and the given name at
the given position.  This method can be called at any time before
\code{close}.
\end{funcdesc}

\begin{funcdesc}{tell}{}
Return the current write position in the output file.  Useful in
combination with \code{setmark}.
\end{funcdesc}

\begin{funcdesc}{writeframes}{data}
Write data to the output file.  This method can only be called after
the audio file parameters have been set.
\end{funcdesc}

\begin{funcdesc}{writeframesraw}{data}
Like \code{writeframes}, except that the header of the audio file is
not updated.
\end{funcdesc}

\begin{funcdesc}{close}{}
Close the AIFF file.  The header of the file is updated to reflect the
actual size of the audio data. After calling this method, the object
can no longer be used.
\end{funcdesc}

\section{Built-in Module \sectcode{jpeg}}
\bimodindex{jpeg}

The module \code{jpeg} provides access to the jpeg compressor and
decompressor written by the Independent JPEG Group. JPEG is a (draft?)\
standard for compressing pictures.  For details on jpeg or the
Independent JPEG Group software refer to the JPEG standard or the
documentation provided with the software.

The \code{jpeg} module defines these functions:

\renewcommand{\indexsubitem}{(in module jpeg)}
\begin{funcdesc}{compress}{data\, w\, h\, b}
Treat data as a pixmap of width \var{w} and height \var{h}, with \var{b} bytes per
pixel.  The data is in SGI GL order, so the first pixel is in the
lower-left corner. This means that \code{lrectread} return data can
immediately be passed to compress.  Currently only 1 byte and 4 byte
pixels are allowed, the former being treated as greyscale and the
latter as RGB color.  Compress returns a string that contains the
compressed picture, in JFIF format.
\end{funcdesc}

\begin{funcdesc}{decompress}{data}
Data is a string containing a picture in JFIF format. It returns a
tuple
\code{(\var{data}, \var{width}, \var{height}, \var{bytesperpixel})}.
Again, the data is suitable to pass to \code{lrectwrite}.
\end{funcdesc}

\begin{funcdesc}{setoption}{name\, value}
Set various options.  Subsequent compress and decompress calls
will use these options.  The following options are available:
\begin{description}
\item[\code{'forcegray' }]
Force output to be grayscale, even if input is RGB.

\item[\code{'quality' }]
Set the quality of the compressed image to a
value between \code{0} and \code{100} (default is \code{75}).  Compress only.

\item[\code{'optimize' }]
Perform Huffman table optimization.  Takes longer, but results in
smaller compressed image.  Compress only.

\item[\code{'smooth' }]
Perform inter-block smoothing on uncompressed image.  Only useful for
low-quality images.  Decompress only.
\end{description}
\end{funcdesc}

Compress and uncompress raise the error \code{jpeg.error} in case of errors.

\section{Built-in Module \sectcode{rgbimg}}
\bimodindex{rgbimg}

The rgbimg module allows python programs to access SGI imglib image
files (also known as \file{.rgb} files).  The module is far from
complete, but is provided anyway since the functionality that there is
is enough in some cases.  Currently, colormap files are not supported.

The module defines the following variables and functions:

\renewcommand{\indexsubitem}{(in module rgbimg)}
\begin{excdesc}{error}
This exception is raised on all errors, such as unsupported file type, etc.
\end{excdesc}

\begin{funcdesc}{sizeofimage}{file}
This function returns a tuple \code{(\var{x}, \var{y})} where
\var{x} and \var{y} are the size of the image in pixels.
Only 4 byte RGBA pixels, 3 byte RGB pixels, and 1 byte greyscale pixels
are currently supported.
\end{funcdesc}

\begin{funcdesc}{longimagedata}{file}
This function reads and decodes the image on the specified file, and
returns it as a Python string. The string has 4 byte RGBA pixels.
The bottom left pixel is the first in
the string. This format is suitable to pass to \code{gl.lrectwrite},
for instance.
\end{funcdesc}

\begin{funcdesc}{longstoimage}{data\, x\, y\, z\, file}
This function writes the RGBA data in \var{data} to image
file \var{file}. \var{x} and \var{y} give the size of the image.
\var{z} is 1 if the saved image should be 1 byte greyscale, 3 if the
saved image should be 3 byte RGB data, or 4 if the saved images should
be 4 byte RGBA data.  The input data always contains 4 bytes per pixel.
These are the formats returned by \code{gl.lrectread}.
\end{funcdesc}

\begin{funcdesc}{ttob}{flag}
This function sets a global flag which defines whether the scan lines
of the image are read or written from bottom to top (flag is zero,
compatible with SGI GL) or from top to bottom(flag is one,
compatible with X)\@.  The default is zero.
\end{funcdesc}

\section{Standard module \sectcode{imghdr}}
\stmodindex{imghdr}

The \code{imghdr} module determines the type of image contained in a
file or byte stream.

The \code{imghdr} module defines the following function:

\renewcommand{\indexsubitem}{(in module imghdr)}

\begin{funcdesc}{what}{filename\optional{\, h}}
Tests the image data contained in the file named by \var{filename},
and returns a string describing the image type.  If optional \var{h}
is provided, the \var{filename} is ignored and \var{h} is assumed to
contain the byte stream to test.
\end{funcdesc}

The following image types are recognized, as listed below with the
return value from \code{what}:

\begin{enumerate}
\item[``rgb''] SGI ImgLib Files

\item[``gif''] GIF 87a and 89a Files

\item[``pbm''] Portable Bitmap Files

\item[``pgm''] Portable Graymap Files

\item[``ppm''] Portable Pixmap Files

\item[``tiff''] TIFF Files

\item[``rast''] Sun Raster Files

\item[``xbm''] X Bitmap Files

\item[``jpeg''] JPEG data in JIFF format
\end{enumerate}

You can extend the list of file types \code{imghdr} can recognize by
appending to this variable:

\begin{datadesc}{tests}
A list of functions performing the individual tests.  Each function
takes two arguments: the byte-stream and an open file-like object.
When \code{what()} is called with a byte-stream, the file-like
object will be \code{None}.

The test function should return a string describing the image type if
the test succeeded, or \code{None} if it failed.
\end{datadesc}

Example:

\begin{verbatim}
>>> import imghdr
>>> imghdr.what('/tmp/bass.gif')
'gif'
\end{verbatim}


\chapter{Cryptographic Services}
\index{cryptography}

The modules described in this chapter implement various algorithms of
a cryptographic nature.  They are available at the discretion of the
installation.  Here's an overview:

\begin{description}

\item[md5]
--- RSA's MD5 message digest algorithm.

\item[mpz]
--- Interface to the GNU MP library for arbitrary precision arithmetic.

\item[rotor]
--- Enigma-like encryption and decryption.

\end{description}

Hardcore cypherpunks will probably find the cryptographic modules
written by Andrew Kuchling of further interest; the package adds
built-in modules for DES and IDEA encryption, provides a Python module
for reading and decrypting PGP files, and then some.  These modules
are not distributed with Python but available separately.  See the URL
\file{http://www.magnet.com/~amk/python/pct.html} or send email to
\file{amk@magnet.com} for more information.
\index{PGP}
\indexii{DES}{cipher}
\indexii{IDEA}{cipher}
\index{cryptography}
		% Cryptographic Services
\section{Built-in Module \sectcode{md5}}
\bimodindex{md5}

This module implements the interface to RSA's MD5 message digest
algorithm (see also Internet RFC 1321).  Its use is quite
straightforward:\ use the \code{md5.new()} to create an md5 object.
You can now feed this object with arbitrary strings using the
\code{update()} method, and at any point you can ask it for the
\dfn{digest} (a strong kind of 128-bit checksum,
a.k.a. ``fingerprint'') of the contatenation of the strings fed to it
so far using the \code{digest()} method.

For example, to obtain the digest of the string {\tt"Nobody inspects
the spammish repetition"}:

\bcode\begin{verbatim}
>>> import md5
>>> m = md5.new()
>>> m.update("Nobody inspects")
>>> m.update(" the spammish repetition")
>>> m.digest()
'\273d\234\203\335\036\245\311\331\336\311\241\215\360\377\351'
\end{verbatim}\ecode

More condensed:

\bcode\begin{verbatim}
>>> md5.new("Nobody inspects the spammish repetition").digest()
'\273d\234\203\335\036\245\311\331\336\311\241\215\360\377\351'
\end{verbatim}\ecode

\renewcommand{\indexsubitem}{(in module md5)}

\begin{funcdesc}{new}{\optional{arg}}
Return a new md5 object.  If \var{arg} is present, the method call
\code{update(\var{arg})} is made.
\end{funcdesc}

\begin{funcdesc}{md5}{\optional{arg}}
For backward compatibility reasons, this is an alternative name for the
\code{new()} function.
\end{funcdesc}

An md5 object has the following methods:

\renewcommand{\indexsubitem}{(md5 method)}
\begin{funcdesc}{update}{arg}
Update the md5 object with the string \var{arg}.  Repeated calls are
equivalent to a single call with the concatenation of all the
arguments, i.e.\ \code{m.update(a); m.update(b)} is equivalent to
\code{m.update(a+b)}.
\end{funcdesc}

\begin{funcdesc}{digest}{}
Return the digest of the strings passed to the \code{update()}
method so far.  This is an 16-byte string which may contain
non-\ASCII{} characters, including null bytes.
\end{funcdesc}

\begin{funcdesc}{copy}{}
Return a copy (``clone'') of the md5 object.  This can be used to
efficiently compute the digests of strings that share a common initial
substring.
\end{funcdesc}

\section{Built-in Module \sectcode{mpz}}
\bimodindex{mpz}

This is an optional module.  It is only available when Python is
configured to include it, which requires that the GNU MP software is
installed.

This module implements the interface to part of the GNU MP library,
which defines arbitrary precision integer and rational number
arithmetic routines.  Only the interfaces to the \emph{integer}
(\samp{mpz_{\rm \ldots}}) routines are provided. If not stated
otherwise, the description in the GNU MP documentation can be applied.

In general, \dfn{mpz}-numbers can be used just like other standard
Python numbers, e.g.\ you can use the built-in operators like \code{+},
\code{*}, etc., as well as the standard built-in functions like
\code{abs}, \code{int}, \ldots, \code{divmod}, \code{pow}.
\strong{Please note:} the {\it bitwise-xor} operation has been implemented as
a bunch of {\it and}s, {\it invert}s and {\it or}s, because the library
lacks an \code{mpz_xor} function, and I didn't need one.

You create an mpz-number by calling the function called \code{mpz} (see
below for an exact description). An mpz-number is printed like this:
\code{mpz(\var{value})}.

\renewcommand{\indexsubitem}{(in module mpz)}
\begin{funcdesc}{mpz}{value}
  Create a new mpz-number. \var{value} can be an integer, a long,
  another mpz-number, or even a string. If it is a string, it is
  interpreted as an array of radix-256 digits, least significant digit
  first, resulting in a positive number. See also the \code{binary}
  method, described below.
\end{funcdesc}

A number of {\em extra} functions are defined in this module. Non
mpz-arguments are converted to mpz-values first, and the functions
return mpz-numbers.

\begin{funcdesc}{powm}{base\, exponent\, modulus}
  Return \code{pow(\var{base}, \var{exponent}) \%{} \var{modulus}}. If
  \code{\var{exponent} == 0}, return \code{mpz(1)}. In contrast to the
  \C-library function, this version can handle negative exponents.
\end{funcdesc}

\begin{funcdesc}{gcd}{op1\, op2}
  Return the greatest common divisor of \var{op1} and \var{op2}.
\end{funcdesc}

\begin{funcdesc}{gcdext}{a\, b}
  Return a tuple \code{(\var{g}, \var{s}, \var{t})}, such that
  \code{\var{a}*\var{s} + \var{b}*\var{t} == \var{g} == gcd(\var{a}, \var{b})}.
\end{funcdesc}

\begin{funcdesc}{sqrt}{op}
  Return the square root of \var{op}. The result is rounded towards zero.
\end{funcdesc}

\begin{funcdesc}{sqrtrem}{op}
  Return a tuple \code{(\var{root}, \var{remainder})}, such that
  \code{\var{root}*\var{root} + \var{remainder} == \var{op}}.
\end{funcdesc}

\begin{funcdesc}{divm}{numerator\, denominator\, modulus}
  Returns a number \var{q}. such that
  \code{\var{q} * \var{denominator} \%{} \var{modulus} == \var{numerator}}.
  One could also implement this function in Python, using \code{gcdext}.
\end{funcdesc}

An mpz-number has one method:

\renewcommand{\indexsubitem}{(mpz method)}
\begin{funcdesc}{binary}{}
  Convert this mpz-number to a binary string, where the number has been
  stored as an array of radix-256 digits, least significant digit first.

  The mpz-number must have a value greater than or equal to zero,
  otherwise a \code{ValueError}-exception will be raised.
\end{funcdesc}

\section{Built-in Module \sectcode{rotor}}
\bimodindex{rotor}

This module implements a rotor-based encryption algorithm, contributed by
Lance Ellinghouse.  The design is derived from the Enigma device, a machine
used during World War II to encipher messages.  A rotor is simply a
permutation.  For example, if the character `A' is the origin of the rotor,
then a given rotor might map `A' to `L', `B' to `Z', `C' to `G', and so on.
To encrypt, we choose several different rotors, and set the origins of the
rotors to known positions; their initial position is the ciphering key.  To
encipher a character, we permute the original character by the first rotor,
and then apply the second rotor's permutation to the result. We continue
until we've applied all the rotors; the resulting character is our
ciphertext.  We then change the origin of the final rotor by one position,
from `A' to `B'; if the final rotor has made a complete revolution, then we
rotate the next-to-last rotor by one position, and apply the same procedure
recursively.  In other words, after enciphering one character, we advance
the rotors in the same fashion as a car's odometer. Decoding works in the
same way, except we reverse the permutations and apply them in the opposite
order.
\index{Ellinghouse, Lance}
\indexii{Enigma}{cipher}

The available functions in this module are:

\renewcommand{\indexsubitem}{(in module rotor)}
\begin{funcdesc}{newrotor}{key\optional{\, numrotors}}
Return a rotor object. \var{key} is a string containing the encryption key
for the object; it can contain arbitrary binary data. The key will be used
to randomly generate the rotor permutations and their initial positions.
\var{numrotors} is the number of rotor permutations in the returned object;
if it is omitted, a default value of 6 will be used.
\end{funcdesc}

Rotor objects have the following methods:

\renewcommand{\indexsubitem}{(rotor method)}
\begin{funcdesc}{setkey}{}
Reset the rotor to its initial state.
\end{funcdesc}

\begin{funcdesc}{encrypt}{plaintext}
Reset the rotor object to its initial state and encrypt \var{plaintext},
returning a string containing the ciphertext.  The ciphertext is always the
same length as the original plaintext.
\end{funcdesc}

\begin{funcdesc}{encryptmore}{plaintext}
Encrypt \var{plaintext} without resetting the rotor object, and return a
string containing the ciphertext.
\end{funcdesc}

\begin{funcdesc}{decrypt}{ciphertext}
Reset the rotor object to its initial state and decrypt \var{ciphertext},
returning a string containing the ciphertext.  The plaintext string will
always be the same length as the ciphertext.
\end{funcdesc}

\begin{funcdesc}{decryptmore}{ciphertext}
Decrypt \var{ciphertext} without resetting the rotor object, and return a
string containing the ciphertext.
\end{funcdesc}

An example usage:
\bcode\begin{verbatim}
>>> import rotor
>>> rt = rotor.newrotor('key', 12)
>>> rt.encrypt('bar')
'\2534\363'
>>> rt.encryptmore('bar')
'\357\375$'
>>> rt.encrypt('bar')
'\2534\363'
>>> rt.decrypt('\2534\363')
'bar'
>>> rt.decryptmore('\357\375$')
'bar'
>>> rt.decrypt('\357\375$')
'l(\315'
>>> del rt
\end{verbatim}\ecode

The module's code is not an exact simulation of the original Enigma device;
it implements the rotor encryption scheme differently from the original. The
most important difference is that in the original Enigma, there were only 5
or 6 different rotors in existence, and they were applied twice to each
character; the cipher key was the order in which they were placed in the
machine.  The Python rotor module uses the supplied key to initialize a
random number generator; the rotor permutations and their initial positions
are then randomly generated.  The original device only enciphered the
letters of the alphabet, while this module can handle any 8-bit binary data;
it also produces binary output.  This module can also operate with an
arbitrary number of rotors.

The original Enigma cipher was broken in 1944. % XXX: Is this right?
The version implemented here is probably a good deal more difficult to crack
(especially if you use many rotors), but it won't be impossible for
a truly skilful and determined attacker to break the cipher.  So if you want
to keep the NSA out of your files, this rotor cipher may well be unsafe, but
for discouraging casual snooping through your files, it will probably be
just fine, and may be somewhat safer than using the Unix \file{crypt}
command.
\index{National Security Agency}\index{crypt(1)}
% XXX How were Unix commands represented in the docs?



%\chapter{Amoeba Specific Services}

\section{Built-in Module \sectcode{amoeba}}

\bimodindex{amoeba}
This module provides some object types and operations useful for
Amoeba applications.  It is only available on systems that support
Amoeba operations.  RPC errors and other Amoeba errors are reported as
the exception \code{amoeba.error = 'amoeba.error'}.

The module \code{amoeba} defines the following items:

\renewcommand{\indexsubitem}{(in module amoeba)}
\begin{funcdesc}{name_append}{path\, cap}
Stores a capability in the Amoeba directory tree.
Arguments are the pathname (a string) and the capability (a capability
object as returned by
\code{name_lookup()}).
\end{funcdesc}

\begin{funcdesc}{name_delete}{path}
Deletes a capability from the Amoeba directory tree.
Argument is the pathname.
\end{funcdesc}

\begin{funcdesc}{name_lookup}{path}
Looks up a capability.
Argument is the pathname.
Returns a
\dfn{capability}
object, to which various interesting operations apply, described below.
\end{funcdesc}

\begin{funcdesc}{name_replace}{path\, cap}
Replaces a capability in the Amoeba directory tree.
Arguments are the pathname and the new capability.
(This differs from
\code{name_append()}
in the behavior when the pathname already exists:
\code{name_append()}
finds this an error while
\code{name_replace()}
allows it, as its name suggests.)
\end{funcdesc}

\begin{datadesc}{capv}
A table representing the capability environment at the time the
interpreter was started.
(Alas, modifying this table does not affect the capability environment
of the interpreter.)
For example,
\code{amoeba.capv['ROOT']}
is the capability of your root directory, similar to
\code{getcap("ROOT")}
in C.
\end{datadesc}

\begin{excdesc}{error}
The exception raised when an Amoeba function returns an error.
The value accompanying this exception is a pair containing the numeric
error code and the corresponding string, as returned by the C function
\code{err_why()}.
\end{excdesc}

\begin{funcdesc}{timeout}{msecs}
Sets the transaction timeout, in milliseconds.
Returns the previous timeout.
Initially, the timeout is set to 2 seconds by the Python interpreter.
\end{funcdesc}

\subsection{Capability Operations}

Capabilities are written in a convenient \ASCII{} format, also used by the
Amoeba utilities
{\it c2a}(U)
and
{\it a2c}(U).
For example:

\bcode\begin{verbatim}
>>> amoeba.name_lookup('/profile/cap')
aa:1c:95:52:6a:fa/14(ff)/8e:ba:5b:8:11:1a
>>> 
\end{verbatim}\ecode

The following methods are defined for capability objects.

\renewcommand{\indexsubitem}{(capability method)}
\begin{funcdesc}{dir_list}{}
Returns a list of the names of the entries in an Amoeba directory.
\end{funcdesc}

\begin{funcdesc}{b_read}{offset\, maxsize}
Reads (at most)
\var{maxsize}
bytes from a bullet file at offset
\var{offset.}
The data is returned as a string.
EOF is reported as an empty string.
\end{funcdesc}

\begin{funcdesc}{b_size}{}
Returns the size of a bullet file.
\end{funcdesc}

\begin{funcdesc}{dir_append}{}
\funcline{dir_delete}{}\ 
\funcline{dir_lookup}{}\ 
\funcline{dir_replace}{}
Like the corresponding
\samp{name_}*
functions, but with a path relative to the capability.
(For paths beginning with a slash the capability is ignored, since this
is the defined semantics for Amoeba.)
\end{funcdesc}

\begin{funcdesc}{std_info}{}
Returns the standard info string of the object.
\end{funcdesc}

\begin{funcdesc}{tod_gettime}{}
Returns the time (in seconds since the Epoch, in UCT, as for POSIX) from
a time server.
\end{funcdesc}

\begin{funcdesc}{tod_settime}{t}
Sets the time kept by a time server.
\end{funcdesc}
		% AMOEBA ONLY

\chapter{Macintosh Specific Services}

The modules in this chapter are available on the Apple Macintosh only.

Aside from the modules described here there are also interfaces to
various MacOS toolboxes, which are currently not extensively
described. The toolboxes for which modules exist are:
\code{AE} (Apple Events),
\code{Cm} (Component Manager),
\code{Ctl} (Control Manager),
\code{Dlg} (Dialog Manager),
\code{Evt} (Event Manager),
\code{Fm} (Font Manager),
\code{List} (List Manager),
\code{Menu} (Moenu Manager),
\code{Qd} (QuickDraw),
\code{Qt} (QuickTime),
\code{Res} (Resource Manager and Handles),
\code{Scrap} (Scrap Manager),
\code{Snd} (Sound Manager),
\code{TE} (TextEdit),
\code{Waste} (non-Apple TextEdit replacement) and
\code{Win} (Window Manager).

If applicable the module will define a number of Python objects for
the various structures declared by the toolbox, and operations will be
implemented as methods of the object. Other operations will be
implemented as functions in the module. Not all operations possible in
C will also be possible in Python (callbacks are often a problem), and
parameters will occasionally be different in Python (input and output
buffers, especially). All methods and functions have a \code{__doc__}
string describing their arguments and return values, and for
additional description you are referred to Inside Mac or similar
works.

\section{Built-in Module \sectcode{mac}}

\bimodindex{mac}
This module provides a subset of the operating system dependent
functionality provided by the optional built-in module \code{posix}.
It is best accessed through the more portable standard module
\code{os}.

The following functions are available in this module:
\code{chdir},
\code{close},
\code{dup},
\code{fdopen},
\code{getcwd},
\code{lseek},
\code{listdir},
\code{mkdir},
\code{open},
\code{read},
\code{rename},
\code{rmdir},
\code{stat},
\code{sync},
\code{unlink},
\code{write},
as well as the exception \code{error}. Note that the times returned by
\code{stat} are floating-point values, like all time values in
MacPython.

One additional function is available: \code{xstat}. This function
returns the same information as \code{stat}, but with three extra
values appended: the size of the resource fork of the file and its
4-char creator and type.

\section{Standard Module \sectcode{macpath}}

\stmodindex{macpath}
This module provides a subset of the pathname manipulation functions
available from the optional standard module \code{posixpath}.  It is
best accessed through the more portable standard module \code{os}, as
\code{os.path}.

The following functions are available in this module:
\code{normcase},
\code{normpath},
\code{isabs},
\code{join},
\code{split},
\code{isdir},
\code{isfile},
\code{walk},
\code{exists}.
For other functions available in \code{posixpath} dummy counterparts
are available.
			% MACINTOSH ONLY
\section{Built-in Module \sectcode{ctb}}
\bimodindex{ctb}
\renewcommand{\indexsubitem}{(in module ctb)}

This module provides a partial interface to the Macintosh
Communications Toolbox. Currently, only Connection Manager tools are
supported.  It may not be available in all Mac Python versions.

\begin{datadesc}{error}
The exception raised on errors.
\end{datadesc}

\begin{datadesc}{cmData}
\dataline{cmCntl}
\dataline{cmAttn}
Flags for the \var{channel} argument of the \var{Read} and \var{Write}
methods.
\end{datadesc}

\begin{datadesc}{cmFlagsEOM}
End-of-message flag for \var{Read} and \var{Write}.
\end{datadesc}

\begin{datadesc}{choose*}
Values returned by \var{Choose}.
\end{datadesc}

\begin{datadesc}{cmStatus*}
Bits in the status as returned by \var{Status}.
\end{datadesc}

\begin{funcdesc}{available}{}
Return 1 if the communication toolbox is available, zero otherwise.
\end{funcdesc}

\begin{funcdesc}{CMNew}{name\, sizes}
Create a connection object using the connection tool named
\var{name}. \var{sizes} is a 6-tuple given buffer sizes for data in,
data out, control in, control out, attention in and attention out.
Alternatively, passing \code{None} will result in default buffer sizes.
\end{funcdesc}

\subsection{connection object}
For all connection methods that take a \var{timeout} argument, a value
of \code{-1} is indefinite, meaning that the command runs to completion.

\renewcommand{\indexsubitem}{(connection object attribute)}

\begin{datadesc}{callback}
If this member is set to a value other than \code{None} it should point
to a function accepting a single argument (the connection
object). This will make all connection object methods work
asynchronously, with the callback routine being called upon
completion.

{\em Note:} for reasons beyond my understanding the callback routine
is currently never called. You are advised against using asynchronous
calls for the time being.
\end{datadesc}


\renewcommand{\indexsubitem}{(connection object method)}

\begin{funcdesc}{Open}{timeout}
Open an outgoing connection, waiting at most \var{timeout} seconds for
the connection to be established.
\end{funcdesc}

\begin{funcdesc}{Listen}{timeout}
Wait for an incoming connection. Stop waiting after \var{timeout}
seconds. This call is only meaningful to some tools.
\end{funcdesc}

\begin{funcdesc}{accept}{yesno}
Accept (when \var{yesno} is non-zero) or reject an incoming call after
\var{Listen} returned.
\end{funcdesc}

\begin{funcdesc}{Close}{timeout\, now}
Close a connection. When \var{now} is zero, the close is orderly
(i.e.\ outstanding output is flushed, etc.)\ with a timeout of
\var{timeout} seconds. When \var{now} is non-zero the close is
immediate, discarding output.
\end{funcdesc}

\begin{funcdesc}{Read}{len\, chan\, timeout}
Read \var{len} bytes, or until \var{timeout} seconds have passed, from
the channel \var{chan} (which is one of \var{cmData}, \var{cmCntl} or
\var{cmAttn}). Return a 2-tuple:\ the data read and the end-of-message
flag.
\end{funcdesc}

\begin{funcdesc}{Write}{buf\, chan\, timeout\, eom}
Write \var{buf} to channel \var{chan}, aborting after \var{timeout}
seconds. When \var{eom} has the value \var{cmFlagsEOM} an
end-of-message indicator will be written after the data (if this
concept has a meaning for this communication tool). The method returns
the number of bytes written.
\end{funcdesc}

\begin{funcdesc}{Status}{}
Return connection status as the 2-tuple \code{(\var{sizes},
\var{flags})}. \var{sizes} is a 6-tuple giving the actual buffer sizes used
(see \var{CMNew}), \var{flags} is a set of bits describing the state
of the connection.
\end{funcdesc}

\begin{funcdesc}{GetConfig}{}
Return the configuration string of the communication tool. These
configuration strings are tool-dependent, but usually easily parsed
and modified.
\end{funcdesc}

\begin{funcdesc}{SetConfig}{str}
Set the configuration string for the tool. The strings are parsed
left-to-right, with later values taking precedence. This means
individual configuration parameters can be modified by simply appending
something like \code{'baud 4800'} to the end of the string returned by
\var{GetConfig} and passing that to this method. The method returns
the number of characters actually parsed by the tool before it
encountered an error (or completed successfully).
\end{funcdesc}

\begin{funcdesc}{Choose}{}
Present the user with a dialog to choose a communication tool and
configure it. If there is an outstanding connection some choices (like
selecting a different tool) may cause the connection to be
aborted. The return value (one of the \var{choose*} constants) will
indicate this.
\end{funcdesc}

\begin{funcdesc}{Idle}{}
Give the tool a chance to use the processor. You should call this
method regularly.
\end{funcdesc}

\begin{funcdesc}{Abort}{}
Abort an outstanding asynchronous \var{Open} or \var{Listen}.
\end{funcdesc}

\begin{funcdesc}{Reset}{}
Reset a connection. Exact meaning depends on the tool.
\end{funcdesc}

\begin{funcdesc}{Break}{length}
Send a break. Whether this means anything, what it means and
interpretation of the \var{length} parameter depend on the tool in
use.
\end{funcdesc}

\section{Built-in Module \sectcode{macconsole}}
\bimodindex{macconsole}

\renewcommand{\indexsubitem}{(in module macconsole)}

This module is available on the Macintosh, provided Python has been
built using the Think C compiler. It provides an interface to the
Think console package, with which basic text windows can be created.

\begin{datadesc}{options}
An object allowing you to set various options when creating windows,
see below.
\end{datadesc}

\begin{datadesc}{C_ECHO}
\dataline{C_NOECHO}
\dataline{C_CBREAK}
\dataline{C_RAW}
Options for the \code{setmode} method. \var{C_ECHO} and \var{C_CBREAK}
enable character echo, the other two disable it, \var{C_ECHO} and
\var{C_NOECHO} enable line-oriented input (erase/kill processing,
etc).
\end{datadesc}

\begin{funcdesc}{copen}{}
Open a new console window. Return a console window object.
\end{funcdesc}

\begin{funcdesc}{fopen}{fp}
Return the console window object corresponding with the given file
object. \var{fp} should be one of \code{sys.stdin}, \code{sys.stdout} or
\code{sys.stderr}.
\end{funcdesc}

\subsection{macconsole options object}
These options are examined when a window is created:

\renewcommand{\indexsubitem}{(macconsole option)}
\begin{datadesc}{top}
\dataline{left}
The origin of the window.
\end{datadesc}

\begin{datadesc}{nrows}
\dataline{ncols}
The size of the window.
\end{datadesc}

\begin{datadesc}{txFont}
\dataline{txSize}
\dataline{txStyle}
The font, fontsize and fontstyle to be used in the window.
\end{datadesc}

\begin{datadesc}{title}
The title of the window.
\end{datadesc}

\begin{datadesc}{pause_atexit}
If set non-zero, the window will wait for user action before closing.
\end{datadesc}

\subsection{console window object}

\renewcommand{\indexsubitem}{(console window attribute)}

\begin{datadesc}{file}
The file object corresponding to this console window. If the file is
buffered, you should call \code{file.flush()} between \code{write()}
and \code{read()} calls.
\end{datadesc}

\renewcommand{\indexsubitem}{(console window method)}

\begin{funcdesc}{setmode}{mode}
Set the input mode of the console to \var{C_ECHO}, etc.
\end{funcdesc}

\begin{funcdesc}{settabs}{n}
Set the tabsize to \var{n} spaces.
\end{funcdesc}

\begin{funcdesc}{cleos}{}
Clear to end-of-screen.
\end{funcdesc}

\begin{funcdesc}{cleol}{}
Clear to end-of-line.
\end{funcdesc}

\begin{funcdesc}{inverse}{onoff}
Enable inverse-video mode:\ characters with the high bit set are
displayed in inverse video (this disables the upper half of a
non-\ASCII{} character set).
\end{funcdesc}

\begin{funcdesc}{gotoxy}{x\, y}
Set the cursor to position \code{(\var{x}, \var{y})}.
\end{funcdesc}

\begin{funcdesc}{hide}{}
Hide the window, remembering the contents.
\end{funcdesc}

\begin{funcdesc}{show}{}
Show the window again.
\end{funcdesc}

\begin{funcdesc}{echo2printer}{}
Copy everything written to the window to the printer as well.
\end{funcdesc}


\section{Built-in Module \sectcode{macdnr}}
\bimodindex{macdnr}

This module provides an interface to the Macintosh Domain Name
Resolver.  It is usually used in conjunction with the \var{mactcp}
module, to map hostnames to IP-addresses.  It may not be available in
all Mac Python versions.

The \code{macdnr} module defines the following functions:

\renewcommand{\indexsubitem}{(in module macdnr)}

\begin{funcdesc}{Open}{\optional{filename}}
Open the domain name resolver extension.  If \var{filename} is given it
should be the pathname of the extension, otherwise a default is
used.  Normally, this call is not needed since the other calls will
open the extension automatically.
\end{funcdesc}

\begin{funcdesc}{Close}{}
Close the resolver extension.  Again, not needed for normal use.
\end{funcdesc}

\begin{funcdesc}{StrToAddr}{hostname}
Look up the IP address for \var{hostname}.  This call returns a dnr
result object of the ``address'' variation.
\end{funcdesc}

\begin{funcdesc}{AddrToName}{addr}
Do a reverse lookup on the 32-bit integer IP-address
\var{addr}.  Returns a dnr result object of the ``address'' variation.
\end{funcdesc}

\begin{funcdesc}{AddrToStr}{addr}
Convert the 32-bit integer IP-address \var{addr} to a dotted-decimal
string.  Returns the string.
\end{funcdesc}

\begin{funcdesc}{HInfo}{hostname}
Query the nameservers for a \code{HInfo} record for host
\var{hostname}.  These records contain hardware and software
information about the machine in question (if they are available in
the first place).  Returns a dnr result object of the ``hinfo''
variety.
\end{funcdesc}

\begin{funcdesc}{MXInfo}{domain}
Query the nameservers for a mail exchanger for \var{domain}.  This is
the hostname of a host willing to accept SMTP mail for the given
domain.  Returns a dnr result object of the ``mx'' variety.
\end{funcdesc}

\subsection{dnr result object}

Since the DNR calls all execute asynchronously you do not get the
results back immediately.  Instead, you get a dnr result object.  You
can check this object to see whether the query is complete, and access
its attributes to obtain the information when it is.

Alternatively, you can also reference the result attributes directly,
this will result in an implicit wait for the query to complete.

The \var{rtnCode} and \var{cname} attributes are always available, the
others depend on the type of query (address, hinfo or mx).

\renewcommand{\indexsubitem}{(dnr result object method)}

% Add args, as in {arg1\, arg2 \optional{\, arg3}}
\begin{funcdesc}{wait}{}
Wait for the query to complete.
\end{funcdesc}

% Add args, as in {arg1\, arg2 \optional{\, arg3}}
\begin{funcdesc}{isdone}{}
Return 1 if the query is complete.
\end{funcdesc}

\renewcommand{\indexsubitem}{(dnr result object attribute)}

\begin{datadesc}{rtnCode}
The error code returned by the query.
\end{datadesc}

\begin{datadesc}{cname}
The canonical name of the host that was queried.
\end{datadesc}

\begin{datadesc}{ip0}
\dataline{ip1}
\dataline{ip2}
\dataline{ip3}
At most four integer IP addresses for this host.  Unused entries are
zero.  Valid only for address queries.
\end{datadesc}

\begin{datadesc}{cpuType}
\dataline{osType}
Textual strings giving the machine type an OS name.  Valid for hinfo
queries.
\end{datadesc}

\begin{datadesc}{exchange}
The name of a mail-exchanger host.  Valid for mx queries.
\end{datadesc}

\begin{datadesc}{preference}
The preference of this mx record.  Not too useful, since the Macintosh
will only return a single mx record.  Mx queries only.
\end{datadesc}

The simplest way to use the module to convert names to dotted-decimal
strings, without worrying about idle time, etc:
\begin{verbatim}
>>> def gethostname(name):
...     import macdnr
...     dnrr = macdnr.StrToAddr(name)
...     return macdnr.AddrToStr(dnrr.ip0)
\end{verbatim}

\section{Built-in Module \sectcode{macfs}}
\bimodindex{macfs}

\renewcommand{\indexsubitem}{(in module macfs)}

This module provides access to macintosh FSSpec handling, the Alias
Manager, finder aliases and the Standard File package.

Whenever a function or method expects a \var{file} argument, this
argument can be one of three things:\ (1) a full or partial Macintosh
pathname, (2) an FSSpec object or (3) a 3-tuple \code{(wdRefNum,
parID, name)} as described in Inside Mac VI\@. A description of aliases
and the standard file package can also be found there.

\begin{funcdesc}{FSSpec}{file}
Create an FSSpec object for the specified file.
\end{funcdesc}

\begin{funcdesc}{RawFSSpec}{data}
Create an FSSpec object given the raw data for the C structure for the
FSSpec as a string.  This is mainly useful if you have obtained an
FSSpec structure over a network.
\end{funcdesc}

\begin{funcdesc}{RawAlias}{data}
Create an Alias object given the raw data for the C structure for the
alias as a string.  This is mainly useful if you have obtained an
FSSpec structure over a network.
\end{funcdesc}

\begin{funcdesc}{FInfo}{}
Create a zero-filled FInfo object.
\end{funcdesc}

\begin{funcdesc}{ResolveAliasFile}{file}
Resolve an alias file. Returns a 3-tuple \code{(\var{fsspec}, \var{isfolder},
\var{aliased})} where \var{fsspec} is the resulting FSSpec object,
\var{isfolder} is true if \var{fsspec} points to a folder and
\var{aliased} is true if the file was an alias in the first place
(otherwise the FSSpec object for the file itself is returned).
\end{funcdesc}

\begin{funcdesc}{StandardGetFile}{\optional{type\, ...}}
Present the user with a standard ``open input file''
dialog. Optionally, you can pass up to four 4-char file types to limit
the files the user can choose from. The function returns an FSSpec
object and a flag indicating that the user completed the dialog
without cancelling.
\end{funcdesc}

\begin{funcdesc}{PromptGetFile}{prompt\optional{\, type\, ...}}
Similar to \var{StandardGetFile} but allows you to specify a prompt.
\end{funcdesc}

\begin{funcdesc}{StandardPutFile}{prompt\, \optional{default}}
Present the user with a standard ``open output file''
dialog. \var{prompt} is the prompt string, and the optional
\var{default} argument initializes the output file name. The function
returns an FSSpec object and a flag indicating that the user completed
the dialog without cancelling.
\end{funcdesc}

\begin{funcdesc}{GetDirectory}{\optional{prompt}}
Present the user with a non-standard ``select a directory''
dialog. \var{prompt} is the prompt string, and the optional.
Return an FSSpec object and a success-indicator.
\end{funcdesc}

\begin{funcdesc}{SetFolder}{\optional{fsspec}}
Set the folder that is initially presented to the user when one of
the file selection dialogs is presented. \var{Fsspec} should point to
a file in the folder, not the folder itself (the file need not exist,
though). If no argument is passed the folder will be set to the
current directory, i.e. what \code{os.getcwd()} returns.

Note that starting with system 7.5 the user can change Standard File
behaviour with the ``general controls'' controlpanel, thereby making
this call inoperative.
\end{funcdesc}

\begin{funcdesc}{FindFolder}{where\, which\, create}
Locates one of the ``special'' folders that MacOS knows about, such as
the trash or the Preferences folder. \var{Where} is the disk to
search, \var{which} is the 4-char string specifying which folder to
locate. Setting \var{create} causes the folder to be created if it
does not exist. Returns a \code{(vrefnum, dirid)} tuple.

The constants for \var{where} and \var{which} can be obtained from the
standard module \var{MACFS}.
\end{funcdesc}

\begin{funcdesc}{FindApplication}{creator}
Locate the application with 4-char creator code \var{creator}. The
function returns an FSSpec object pointing to the application.
\end{funcdesc}

\subsection{FSSpec objects}

\renewcommand{\indexsubitem}{(FSSpec object attribute)}
\begin{datadesc}{data}
The raw data from the FSSpec object, suitable for passing
to other applications, for instance.
\end{datadesc}

\renewcommand{\indexsubitem}{(FSSpec object method)}
\begin{funcdesc}{as_pathname}{}
Return the full pathname of the file described by the FSSpec object.
\end{funcdesc}

\begin{funcdesc}{as_tuple}{}
Return the \code{(\var{wdRefNum}, \var{parID}, \var{name})} tuple of the file described
by the FSSpec object.
\end{funcdesc}

\begin{funcdesc}{NewAlias}{\optional{file}}
Create an Alias object pointing to the file described by this
FSSpec. If the optional \var{file} parameter is present the alias
will be relative to that file, otherwise it will be absolute.
\end{funcdesc}

\begin{funcdesc}{NewAliasMinimal}{}
Create a minimal alias pointing to this file.
\end{funcdesc}

\begin{funcdesc}{GetCreatorType}{}
Return the 4-char creator and type of the file.
\end{funcdesc}

\begin{funcdesc}{SetCreatorType}{creator\, type}
Set the 4-char creator and type of the file.
\end{funcdesc}

\begin{funcdesc}{GetFInfo}{}
Return a FInfo object describing the finder info for the file.
\end{funcdesc}

\begin{funcdesc}{SetFInfo}{finfo}
Set the finder info for the file to the values specified in the
\var{finfo} object.
\end{funcdesc}

\begin{funcdesc}{GetDates}{}
Return a tuple with three floating point values representing the
creation date, modification date and backup date of the file.
\end{funcdesc}

\begin{funcdesc}{SetDates}{crdate\, moddate\, backupdate}
Set the creation, modification and backup date of the file. The values
are in the standard floating point format used for times throughout
Python.
\end{funcdesc}

\subsection{alias objects}

\renewcommand{\indexsubitem}{(alias object attribute)}
\begin{datadesc}{data}
The raw data for the Alias record, suitable for storing in a resource
or transmitting to other programs.
\end{datadesc}

\renewcommand{\indexsubitem}{(alias object method)}
\begin{funcdesc}{Resolve}{\optional{file}}
Resolve the alias. If the alias was created as a relative alias you
should pass the file relative to which it is. Return the FSSpec for
the file pointed to and a flag indicating whether the alias object
itself was modified during the search process. 
\end{funcdesc}

\begin{funcdesc}{GetInfo}{num}
An interface to the C routine \code{GetAliasInfo()}.
\end{funcdesc}

\begin{funcdesc}{Update}{file\, \optional{file2}}
Update the alias to point to the \var{file} given. If \var{file2} is
present a relative alias will be created.
\end{funcdesc}

Note that it is currently not possible to directly manipulate a resource
as an alias object. Hence, after calling \var{Update} or after
\var{Resolve} indicates that the alias has changed the Python program
is responsible for getting the \var{data} from the alias object and
modifying the resource.


\subsection{FInfo objects}

See Inside Mac for a complete description of what the various fields
mean.

\renewcommand{\indexsubitem}{(FInfo object attribute)}
\begin{datadesc}{Creator}
The 4-char creator code of the file.
\end{datadesc}

\begin{datadesc}{Type}
The 4-char type code of the file.
\end{datadesc}

\begin{datadesc}{Flags}
The finder flags for the file as 16-bit integer. The bit values in
\var{Flags} are defined in standard module \var{MACFS}.
\end{datadesc}

\begin{datadesc}{Location}
A Point giving the position of the file's icon in its folder.
\end{datadesc}

\begin{datadesc}{Fldr}
The folder the file is in (as an integer).
\end{datadesc}

\section{Built-in Module \sectcode{MacOS}}
\bimodindex{MacOS}

\renewcommand{\indexsubitem}{(in module MacOS)}

This module provides access to MacOS specific functionality in the
python interpreter, such as how the interpreter eventloop functions
and the like. Use with care.

Note the capitalisation of the module name, this is a historical
artefact.

\begin{excdesc}{Error}
This exception is raised on MacOS generated errors, either from
functions in this module or from other mac-specific modules like the
toolbox interfaces. The arguments are the integer error code (the
\var{OSErr} value) and a textual description of the error code.
Symbolic names for all known error codes are defined in the standard
module \var{macerrors}.
\end{excdesc}

\begin{funcdesc}{SetHighLevelEventHandler}{handler}
Pass a python function that will be called upon reception of a
high-level event. The previous handler is returned. The handler
function is called with the event as argument.

Note that your event handler is currently only called dependably if
your main event loop is in \var{stdwin}.
\end{funcdesc}

\begin{funcdesc}{AcceptHighLevelEvent}{}
Read a high-level event. The return value is a tuple \code{(sender,
refcon, data)}.
\end{funcdesc}

\begin{funcdesc}{SetScheduleTimes}{fgi\, fgy \optional{\, bgi\, bgy}}
Controls how often the interpreter checks the event queue and how
long it will yield the processor to other processes. \var{fgi}
specifies after how many clicks (one click is one 60th of a second)
the interpreter should check the event queue, and \var{fgy} specifies
for how many clicks the CPU should be yielded when in the
foreground. The optional \var{bgi} and \var{bgy} allow you to specify
different values to use when python runs in the background, otherwise
the background values will be set the the same as the foreground
values. The function returns nothing.

The default values, which are based on minimal empirical testing, are 12, 1, 6
and 2 respectively.
\end{funcdesc}

\begin{funcdesc}{EnableAppswitch}{onoff}
Enable or disable the python event loop, based on the value of
\var{onoff}. The old value is returned. If the event loop is disabled
no time is granted to other applications, checking for command-period
is not performed and it is impossible to switch applications. This
should only be used by programs providing their own complete event
loop.

Note that based on the compiler used to build python it is still
possible to loose events even with the python event loop disabled. If
you have a \code{sys.stdout} window its handler will often also look
in the event queue. Making sure nothing is ever printed works around
this.
\end{funcdesc}

\begin{funcdesc}{HandleEvent}{ev}
Pass the event record \code{ev} back to the python event loop, or
possibly to the handler for the \code{sys.stdout} window (based on the
compiler used to build python). This allows python programs that do
their own event handling to still have some command-period and
window-switching capability.
\end{funcdesc}

\begin{funcdesc}{GetErrorString}{errno}
Return the textual description of MacOS error code \var{errno}.
\end{funcdesc}

\begin{funcdesc}{splash}{resid}
This function will put a splash window
on-screen, with the contents of the DLOG resource specified by
\code{resid}. Calling with a zero argument will remove the splash
screen. This function is useful if you want an applet to post a splash screen
early in initialization without first having to load numerous
extension modules.
\end{funcdesc}

\begin{funcdesc}{DebugStr}{message \optional{\, object}}
Drop to the low-level debugger with message \var{message}. The
optional \var{object} argument is not used, but can easily be
inspected from the debugger.

Note that you should use this function with extreme care: if no
low-level debugger like MacsBug is installed this call will crash your
system. It is intended mainly for developers of Python extension
modules.
\end{funcdesc}

\begin{funcdesc}{openrf}{name \optional{\, mode}}
Open the resource fork of a file. Arguments are the same as for the
builtin function \code{open}. The object returned has file-like
semantics, but it is not a python file object, so there may be subtle
differences.
\end{funcdesc}


\section{Standard module \sectcode{macostools}}
\stmodindex{macostools}

This module contains some convenience routines for file-manipulation
on the Macintosh.

The \code{macostools} module defines the following functions:

\renewcommand{\indexsubitem}{(in module macostools)}

\begin{funcdesc}{copy}{src\, dst\optional{\, createpath, copytimes}}
Copy file \var{src} to \var{dst}. The files can be specified as
pathnames or \code{FSSpec} objects. If \var{createpath} is non-zero
\var{dst} must be a pathname and the folders leading to the
destination are created if necessary.  The method copies data and
resource fork and some finder information (creator, type, flags) and
optionally the creation, modification and backup times (default is to
copy them). Custom icons, comments and icon position are not copied.

If the source is an alias the original to which the alias points is
copied, not the aliasfile.
\end{funcdesc}

\begin{funcdesc}{copytree}{src\, dst}
Recursively copy a file tree from \var{src} to \var{dst}, creating
folders as needed. \var{Src} and \var{dst} should be specified as
pathnames.
\end{funcdesc}

\begin{funcdesc}{mkalias}{src\, dst}
Create a finder alias \var{dst} pointing to \var{src}. Both may be
specified as pathnames or \var{FSSpec} objects.
\end{funcdesc}

\begin{funcdesc}{touched}{dst}
Tell the finder that some bits of finder-information such as creator
or type for file \var{dst} has changed. The file can be specified by
pathname or fsspec. This call should prod the finder into redrawing the
files icon.
\end{funcdesc}

\begin{datadesc}{BUFSIZ}
The buffer size for \code{copy}, default 1 megabyte.
\end{datadesc}

Note that the process of creating finder aliases is not specified in
the Apple documentation. Hence, aliases created with \code{mkalias}
could conceivably have incompatible behaviour in some cases.

\section{Standard module \sectcode{findertools}}
\stmodindex{findertools}

This module contains routines that give Python programs access to some
functionality provided by the finder. They are implemented as wrappers
around the AppleEvent interface to the finder.

All file and folder parameters can be specified either as full
pathnames or as \code{FSSpec} objects.

The \code{findertools} module defines the following functions:

\renewcommand{\indexsubitem}{(in module macostools)}

\begin{funcdesc}{launch}{file}
Tell the finder to launch \var{file}. What launching means depends on the file:
applications are started, folders are opened and documents are opened
in the correct application.
\end{funcdesc}

\begin{funcdesc}{Print}{file}
Tell the finder to print a file (again specified by full pathname or
FSSpec). The behaviour is identical to selecting the file and using
the print command in the finder.
\end{funcdesc}

\begin{funcdesc}{copy}{file, destdir}
Tell the finder to copy a file or folder \var{file} to folder
\var{destdir}. The function returns an \code{Alias} object pointing to
the new file.
\end{funcdesc}

\begin{funcdesc}{move}{file, destdir}
Tell the finder to move a file or folder \var{file} to folder
\var{destdir}. The function returns an \code{Alias} object pointing to
the new file.
\end{funcdesc}

\begin{funcdesc}{sleep}{}
Tell the finder to put the mac to sleep, if your machine supports it.
\end{funcdesc}

\begin{funcdesc}{restart}{}
Tell the finder to perform an orderly restart of the machine.
\end{funcdesc}

\begin{funcdesc}{shutdown}{}
Tell the finder to perform an orderly shutdown of the machine.
\end{funcdesc}

\section{Built-in Module \sectcode{mactcp}}
\bimodindex{mactcp}

\renewcommand{\indexsubitem}{(in module mactcp)}

This module provides an interface to the Macintosh TCP/IP driver
MacTCP\@. There is an accompanying module \code{macdnr} which provides an
interface to the name-server (allowing you to translate hostnames to
ip-addresses), a module \code{MACTCPconst} which has symbolic names for
constants constants used by MacTCP. Since the builtin module
\code{socket} is also available on the mac it is usually easier to use
sockets in stead of the mac-specific MacTCP API.

A complete description of the MacTCP interface can be found in the
Apple MacTCP API documentation.

\begin{funcdesc}{MTU}{}
Return the Maximum Transmit Unit (the packet size) of the network
interface.
\end{funcdesc}

\begin{funcdesc}{IPAddr}{}
Return the 32-bit integer IP address of the network interface.
\end{funcdesc}

\begin{funcdesc}{NetMask}{}
Return the 32-bit integer network mask of the interface.
\end{funcdesc}

\begin{funcdesc}{TCPCreate}{size}
Create a TCP Stream object. \var{size} is the size of the receive
buffer, \code{4096} is suggested by various sources.
\end{funcdesc}

\begin{funcdesc}{UDPCreate}{size, port}
Create a UDP stream object. \var{size} is the size of the receive
buffer (and, hence, the size of the biggest datagram you can receive
on this port). \var{port} is the UDP port number you want to receive
datagrams on, a value of zero will make MacTCP select a free port.
\end{funcdesc}

\subsection{TCP Stream Objects}

\renewcommand{\indexsubitem}{(TCP stream attribute)}

\begin{datadesc}{asr}
When set to a value different than \code{None} this should point to a
function with two integer parameters:\ an event code and a detail. This
function will be called upon network-generated events such as urgent
data arrival. In addition, it is called with eventcode
\code{MACTCP.PassiveOpenDone} when a \code{PassiveOpen} completes. This
is a Python addition to the MacTCP semantics.
It is safe to do further calls from the \code{asr}.
\end{datadesc}

\renewcommand{\indexsubitem}{(TCP stream method)}

\begin{funcdesc}{PassiveOpen}{port}
Wait for an incoming connection on TCP port \var{port} (zero makes the
system pick a free port). The call returns immediately, and you should
use \var{wait} to wait for completion. You should not issue any method
calls other than
\code{wait}, \code{isdone} or \code{GetSockName} before the call
completes.
\end{funcdesc}

\begin{funcdesc}{wait}{}
Wait for \code{PassiveOpen} to complete.
\end{funcdesc}

\begin{funcdesc}{isdone}{}
Return 1 if a \code{PassiveOpen} has completed.
\end{funcdesc}

\begin{funcdesc}{GetSockName}{}
Return the TCP address of this side of a connection as a 2-tuple
\code{(host, port)}, both integers.
\end{funcdesc}

\begin{funcdesc}{ActiveOpen}{lport\, host\, rport}
Open an outgoing connection to TCP address \code{(\var{host}, \var{rport})}. Use
local port \var{lport} (zero makes the system pick a free port). This
call blocks until the connection has been established.
\end{funcdesc}

\begin{funcdesc}{Send}{buf\, push\, urgent}
Send data \var{buf} over the connection. \var{Push} and \var{urgent}
are flags as specified by the TCP standard.
\end{funcdesc}

\begin{funcdesc}{Rcv}{timeout}
Receive data. The call returns when \var{timeout} seconds have passed
or when (according to the MacTCP documentation) ``a reasonable amount
of data has been received''. The return value is a 3-tuple
\code{(\var{data}, \var{urgent}, \var{mark})}. If urgent data is outstanding \code{Rcv}
will always return that before looking at any normal data. The first
call returning urgent data will have the \var{urgent} flag set, the
last will have the \var{mark} flag set.
\end{funcdesc}

\begin{funcdesc}{Close}{}
Tell MacTCP that no more data will be transmitted on this
connection. The call returns when all data has been acknowledged by
the receiving side.
\end{funcdesc}

\begin{funcdesc}{Abort}{}
Forcibly close both sides of a connection, ignoring outstanding data.
\end{funcdesc}

\begin{funcdesc}{Status}{}
Return a TCP status object for this stream giving the current status
(see below).
\end{funcdesc}

\subsection{TCP Status Objects}
This object has no methods, only some members holding information on
the connection. A complete description of all fields in this objects
can be found in the Apple documentation. The most interesting ones are:

\renewcommand{\indexsubitem}{(TCP status attribute)}

\begin{datadesc}{localHost}
\dataline{localPort}
\dataline{remoteHost}
\dataline{remotePort}
The integer IP-addresses and port numbers of both endpoints of the
connection. 
\end{datadesc}

\begin{datadesc}{sendWindow}
The current window size.
\end{datadesc}

\begin{datadesc}{amtUnackedData}
The number of bytes sent but not yet acknowledged. \code{sendWindow -
amtUnackedData} is what you can pass to \code{Send} without blocking.
\end{datadesc}

\begin{datadesc}{amtUnreadData}
The number of bytes received but not yet read (what you can \code{Recv}
without blocking).
\end{datadesc}



\subsection{UDP Stream Objects}
Note that, unlike the name suggests, there is nothing stream-like
about UDP.

\renewcommand{\indexsubitem}{(UDP stream attribute)}

\begin{datadesc}{asr}
The asynchronous service routine to be called on events such as
datagram arrival without outstanding \code{Read} call. The \code{asr} has a
single argument, the event code.
\end{datadesc}

\begin{datadesc}{port}
A read-only member giving the port number of this UDP stream.
\end{datadesc}

\renewcommand{\indexsubitem}{(UDP stream method)}

\begin{funcdesc}{Read}{timeout}
Read a datagram, waiting at most \var{timeout} seconds ($-1$ is
infinite).  Return the data.
\end{funcdesc}

\begin{funcdesc}{Write}{host\, port\, buf}
Send \var{buf} as a datagram to IP-address \var{host}, port
\var{port}.
\end{funcdesc}

\section{Built-in Module \sectcode{macspeech}}
\bimodindex{macspeech}

\renewcommand{\indexsubitem}{(in module macspeech)}

This module provides an interface to the Macintosh Speech Manager,
allowing you to let the Macintosh utter phrases. You need a version of
the speech manager extension (version 1 and 2 have been tested) in
your \code{Extensions} folder for this to work. The module does not
provide full access to all features of the Speech Manager yet.  It may
not be available in all Mac Python versions.

\begin{funcdesc}{Available}{}
Test availability of the Speech Manager extension (and, on the
PowerPC, the Speech Manager shared library). Return 0 or 1. 
\end{funcdesc}

\begin{funcdesc}{Version}{}
Return the (integer) version number of the Speech Manager.
\end{funcdesc}

\begin{funcdesc}{SpeakString}{str}
Utter the string \var{str} using the default voice,
asynchronously. This aborts any speech that may still be active from
prior \code{SpeakString} invocations.
\end{funcdesc}

\begin{funcdesc}{Busy}{}
Return the number of speech channels busy, system-wide.
\end{funcdesc}

\begin{funcdesc}{CountVoices}{}
Return the number of different voices available.
\end{funcdesc}

\begin{funcdesc}{GetIndVoice}{num}
Return a voice object for voice number \var{num}.
\end{funcdesc}

\subsection{voice objects}
Voice objects contain the description of a voice. It is currently not
yet possible to access the parameters of a voice.

\renewcommand{\indexsubitem}{(voice object method)}

\begin{funcdesc}{GetGender}{}
Return the gender of the voice:\ 0 for male, 1 for female and $-1$ for neuter.
\end{funcdesc}

\begin{funcdesc}{NewChannel}{}
Return a new speech channel object using this voice.
\end{funcdesc}

\subsection{speech channel objects}
A speech channel object allows you to speak strings with slightly more
control than \code{SpeakString()}, and allows you to use multiple
speakers at the same time. Please note that channel pitch and rate are
interrelated in some way, so that to make your Macintosh sing you will
have to adjust both.

\renewcommand{\indexsubitem}{(speech channel object method)}
\begin{funcdesc}{SpeakText}{str}
Start uttering the given string.
\end{funcdesc}

\begin{funcdesc}{Stop}{}
Stop babbling.
\end{funcdesc}

\begin{funcdesc}{GetPitch}{}
Return the current pitch of the channel, as a floating-point number.
\end{funcdesc}

\begin{funcdesc}{SetPitch}{pitch}
Set the pitch of the channel.
\end{funcdesc}

\begin{funcdesc}{GetRate}{}
Get the speech rate (utterances per minute) of the channel as a
floating point number.
\end{funcdesc}

\begin{funcdesc}{SetRate}{rate}
Set the speech rate of the channel.
\end{funcdesc}


\section{Standard module \sectcode{EasyDialogs}}
\stmodindex{EasyDialogs}

The \code{EasyDialogs} module contains some simple dialogs for
the Macintosh, modelled after the \code{stdwin} dialogs with similar
names.

The \code{EasyDialogs} module defines the following functions:

\renewcommand{\indexsubitem}{(in module EasyDialogs)}

\begin{funcdesc}{Message}{str}
A modal dialog with the message text \var{str}, which should be at
most 255 characters long, is displayed. Control is returned when the
user clicks ``OK''.
\end{funcdesc}

\begin{funcdesc}{AskString}{prompt\optional{\, default}}
Ask the user to input a string value, in a modal dialog. \var{Prompt}
is the promt message, the optional \var{default} arg is the initial
value for the string. All strings can be at most 255 bytes
long. \var{AskString} returns the string entered or \code{None} in
case the user cancelled.
\end{funcdesc}

\begin{funcdesc}{AskYesNoCancel}{question\optional{\, default}}
Present a dialog with text \var{question} and three buttons labelled
``yes'', ``no'' and ``cancel''. Return \code{1} for yes, \code{0} for
no and \code{-1} for cancel. The default return value chosen by
hitting return is \code{0}. This can be changed with the optional
\var{default} argument.
\end{funcdesc}

\begin{funcdesc}{ProgressBar}{\optional{label\, maxval}}
Display a modeless progress dialog with a thermometer bar. \var{Label}
is the textstring displayed (default ``Working...''), \var{maxval} is
the value at which progress is complete (default 100). The returned
object has one method, \code{set(value)}, which sets the value of the
progress bar. The bar remains visible until the object returned is
discarded.

The progress bar has a ``cancel'' button, but it is currently
non-functional.
\end{funcdesc}

Note that \code{EasyDialogs} does not currently use the notification
manager. This means that displaying dialogs while the program is in
the background will lead to unexpected results and possibly
crashes. Also, all dialogs are modeless and hence expect to be at the
top of the stacking order. This is true when the dialogs are created,
but windows that pop-up later (like a console window) may also result
in crashes.


\section{Standard module \sectcode{FrameWork}}
\stmodindex{FrameWork}

The \code{FrameWork} module contains classes that together provide a
framework for an interactive Macintosh application. The programmer
builds an application by creating subclasses that override various
methods of the bases classes, thereby implementing the functionality
wanted. Overriding functionality can often be done on various
different levels, i.e. to handle clicks in a single dialog window in a
non-standard way it is not necessary to override the complete event
handling.

The \code{FrameWork} is still very much work-in-progress, and the
documentation describes only the most important functionality, and not
in the most logical manner at that. Examine the source or the examples
for more details.

The \code{FrameWork} module defines the following functions:

\renewcommand{\indexsubitem}{(in module FrameWork)}

\begin{funcdesc}{Application}{}
An object representing the complete application. See below for a
description of the methods. The default \code{__init__} routine
creates an empty window dictionary and a menu bar with an apple menu.
\end{funcdesc}

\begin{funcdesc}{MenuBar}{}
An object representing the menubar. This object is usually not created
by the user.
\end{funcdesc}

\begin{funcdesc}{Menu}{bar\, title\optional{\, after}}
An object representing a menu. Upon creation you pass the
\code{MenuBar} the menu appears in, the \var{title} string and a
position (1-based) \var{after} where the menu should appear (default:
at the end).
\end{funcdesc}

\begin{funcdesc}{MenuItem}{menu\, title\optional{\, shortcut\, callback}}
Create a menu item object. The arguments are the menu to crate the
item it, the item title string and optionally the keyboard shortcut
and a callback routine. The callback is called with the arguments
menu-id, item number within menu (1-based), current front window and
the event record.

In stead of a callable object the callback can also be a string. In
this case menu selection causes the lookup of a method in the topmost
window and the application. The method name is the callback string
with \code{'domenu_'} prepended.

Calling the \code{MenuBar} \code{fixmenudimstate} method sets the
correct dimming for all menu items based on the current front window.
\end{funcdesc}

\begin{funcdesc}{Separator}{menu}
Add a separator to the end of a menu.
\end{funcdesc}

\begin{funcdesc}{SubMenu}{menu\, label}
Create a submenu named \var{label} under menu \var{menu}. The menu
object is returned.
\end{funcdesc}

\begin{funcdesc}{Window}{parent}
Creates a (modeless) window. \var{Parent} is the application object to
which the window belongs. The window is not displayed until later.
\end{funcdesc}

\begin{funcdesc}{DialogWindow}{parent}
Creates a modeless dialog window.
\end{funcdesc}

\begin{funcdesc}{windowbounds}{width\, height}
Return a \code{(left, top, right, bottom)} tuple suitable for creation
of a window of given width and height. The window will be staggered
with respect to previous windows, and an attempt is made to keep the
whole window on-screen. The window will however always be exact the
size given, so parts may be offscreen.
\end{funcdesc}

\begin{funcdesc}{setwatchcursor}{}
Set the mouse cursor to a watch.
\end{funcdesc}

\begin{funcdesc}{setarrowcursor}{}
Set the mouse cursor to an arrow.
\end{funcdesc}

\subsection{Application objects}
Application objects have the following methods, among others:

\renewcommand{\indexsubitem}{(Application method)}

\begin{funcdesc}{makeusermenus}{}
Override this method if you need menus in your application. Append the
menus to \code{self.menubar}.
\end{funcdesc}

\begin{funcdesc}{getabouttext}{}
Override this method to return a text string describing your
application. Alternatively, override the \code{do_about} method for
more elaborate about messages.
\end{funcdesc}

\begin{funcdesc}{mainloop}{\optional{mask\, wait}}
This routine is the main event loop, call it to set your application
rolling. \var{Mask} is the mask of events you want to handle,
\var{wait} is the number of ticks you want to leave to other
concurrent application (default 0, which is probably not a good
idea). While raising \code{self} to exit the mainloop is still
supported it is not recommended, call \code{self._quit} instead.

The event loop is split into many small parts, each of which can be
overridden. The default methods take care of dispatching events to
windows and dialogs, handling drags and resizes, Apple Events, events
for non-FrameWork windows, etc.
\end{funcdesc}

\begin{funcdesc}{_quit}{}
Terminate the event \code{mainloop} at the next convenient moment.
\end{funcdesc}

\begin{funcdesc}{do_char}{c\, event}
The user typed character \var{c}. The complete details of the event
can be found in the \var{event} structure. This method can also be
provided in a \code{Window} object, which overrides the
application-wide handler if the window is frontmost.
\end{funcdesc}

\begin{funcdesc}{do_dialogevent}{event}
Called early in the event loop to handle modeless dialog events. The
default method simply dispatches the event to the relevant dialog (not
through the the \code{DialogWindow} object involved). Override if you
need special handling of dialog events (keyboard shortcuts, etc).
\end{funcdesc}

\begin{funcdesc}{idle}{event}
Called by the main event loop when no events are available. The
null-event is passed (so you can look at mouse position, etc).
\end{funcdesc}

\subsection{Window Objects}

Window objects have the following methods, among others:

\renewcommand{\indexsubitem}{(Window method)}

\begin{funcdesc}{open}{}
Override this method to open a window. Store the MacOS window-id in
\code{self.wid} and call \code{self.do_postopen} to register the
window with the parent application.
\end{funcdesc}

\begin{funcdesc}{close}{}
Override this method to do any special processing on window
close. Call \code{self.do_postclose} to cleanup the parent state.
\end{funcdesc}

\begin{funcdesc}{do_postresize}{width\, height\, macoswindowid}
Called after the window is resized. Override if more needs to be done
than calling \code{InvalRect}.
\end{funcdesc}

\begin{funcdesc}{do_contentclick}{local\, modifiers\, event}
The user clicked in the content part of a window. The arguments are
the coordinates (window-relative), the key modifiers and the raw
event.
\end{funcdesc}

\begin{funcdesc}{do_update}{macoswindowid\, event}
An update event for the window was received. Redraw the window.
\end{funcdesc}

\begin{funcdesc}{do_activate}{activate\, event}
The window was activated (\code{activate==1}) or deactivated
(\code{activate==0}). Handle things like focus highlighting, etc.
\end{funcdesc}

\subsection{ControlsWindow Object}

ControlsWindow objects have the following methods besides those of
\code{Window} objects:

\renewcommand{\indexsubitem}{(ControlsWindow method)}

\begin{funcdesc}{do_controlhit}{window\, control\, pcode\, event}
Part \code{pcode} of control \code{control} was hit by the
user. Tracking and such has already been taken care of.
\end{funcdesc}

\subsection{ScrolledWindow Object}

ScrolledWindow objects are ControlsWindow objects with the following
extra methods:

\renewcommand{\indexsubitem}{(ScrolledWindow method)}

\begin{funcdesc}{scrollbars}{\optional{wantx\, wanty}}
Create (or destroy) horizontal and vertical scrollbars. The arguments
specify which you want (default: both). The scrollbars always have
minimum \code{0} and maximum \code{32767}.
\end{funcdesc}

\begin{funcdesc}{getscrollbarvalues}{}
You must supply this method. It should return a tuple \code{x, y}
giving the current position of the scrollbars (between \code{0} and
\code{32767}). You can return \code{None} for either to indicate the
whole document is visible in that direction.
\end{funcdesc}

\begin{funcdesc}{updatescrollbars}{}
Call this method when the document has changed. It will call
\code{getscrollbarvalues} and update the scrollbars.
\end{funcdesc}

\begin{funcdesc}{scrollbar_callback}{which\, what\, value}
Supplied by you and called after user interaction. \code{Which} will
be \code{'x'} or \code{'y'}, \code{what} will be \code{'-'},
\code{'--'}, \code{'set'}, \code{'++'} or \code{'+'}. For
\code{'set'}, \code{value} will contain the new scrollbar position.
\end{funcdesc}

\begin{funcdesc}{scalebarvalues}{absmin\, absmax\, curmin\, curmax}
Auxiliary method to help you calculate values to return from
\code{getscrollbarvalues}. You pass document minimum and maximum value
and topmost (leftmost) and bottommost (rightmost) visible values and
it returns the correct number or \code{None}.
\end{funcdesc}

\begin{funcdesc}{do_activate}{onoff\, event}
Takes care of dimming/highlighting scrollbars when a window becomes
frontmost vv. If you override this method call this one at the end of
your method.
\end{funcdesc}

\begin{funcdesc}{do_postresize}{width\, height\, window}
Moves scrollbars to the correct position. Call this method initially
if you override it.
\end{funcdesc}

\begin{funcdesc}{do_controlhit}{window\, control\, pcode\, event}
Handles scrollbar interaction. If you override it call this method
first, a nonzero return value indicates the hit was in the scrollbars
and has been handled.
\end{funcdesc}

\subsection{DialogWindow Objects}

DialogWindow objects have the following methods besides those of
\code{Window} objects:

\renewcommand{\indexsubitem}{(DialogWindow method)}

\begin{funcdesc}{open}{resid}
Create the dialog window, from the DLOG resource with id
\var{resid}. The dialog object is stored in \code{self.wid}.
\end{funcdesc}

\begin{funcdesc}{do_itemhit}{item\, event}
Item number \var{item} was hit. You are responsible for redrawing
toggle buttons, etc.
\end{funcdesc}

\section{Standard module \sectcode{MiniAEFrame}}
\stmodindex{MiniAEFrame}

The module \var{MiniAEFrame} provides a framework for an application
that can function as an OSA server, i.e. receive and process
AppleEvents. It can be used in conjunction with \var{FrameWork} or
standalone.

This module is temporary, it will eventually be replaced by a module
that handles argument names better and possibly automates making your
application scriptable.

The \var{MiniAEFrame} module defines the following classes:

\renewcommand{\indexsubitem}{(in module MiniAEFrame)}

\begin{funcdesc}{AEServer}{}
A class that handles AppleEvent dispatch. Your application should
subclass this class together with either
\code{MiniAEFrame.MiniApplication} or
\code{FrameWork.Application}. Your \code{__init__} method should call
the \code{__init__} method for both classes.
\end{funcdesc}

\begin{funcdesc}{MiniApplication}{}
A class that is more or less compatible with
\code{FrameWork.Application} but with less functionality. Its
eventloop supports the apple menu, command-dot and AppleEvents, other
events are passed on to the Python interpreter and/or Sioux.
Useful if your application wants to use \code{AEServer} but does not
provide its own windows, etc.
\end{funcdesc}

\subsection{AEServer Objects}

\renewcommand{\indexsubitem}{(AEServer method)}

\begin{funcdesc}{installaehandler}{classe\, type\, callback}
Installs an AppleEvent handler. \code{Classe} and \code{type} are the
four-char OSA Class and Type designators, \code{'****'} wildcards are
allowed. When a matching AppleEvent is received the parameters are
decoded and your callback is invoked.
\end{funcdesc}

\begin{funcdesc}{callback}{_object\, **kwargs}
Your callback is called with the OSA Direct Object as first positional
parameter. The other parameters are passed as keyword arguments, with
the 4-char designator as name. Three extra keyword parameters are
passed: \code{_class} and \code{_type} are the Class and Type
designators and \code{_attributes} is a dictionary with the AppleEvent
attributes.

The return value of your method is packed with
\code{aetools.packevent} and sent as reply.
\end{funcdesc}

Note that there are some serious problems with the current
design. AppleEvents which have non-identifier 4-char designators for
arguments are not implementable, and it is not possible to return an
error to the originator. This will be addressed in a future release.


\chapter{Standard Windowing Interface}

The modules in this chapter are available only on those systems where
the STDWIN library is available.  STDWIN runs on \UNIX{} under X11 and
on the Macintosh.  See CWI report CS-R8817.

\strong{Warning:} Using STDWIN is not recommended for new
applications.  It has never been ported to Microsoft Windows or
Windows NT, and for X11 or the Macintosh it lacks important
functionality --- in particular, it has no tools for the construction
of dialogs.  For most platforms, alternative, native solutions exist
(though none are currently documented in this manual): Tkinter for
\UNIX{} under X11, native Xt with Motif or Athena widgets for \UNIX{}
under X11, Win32 for Windows and Windows NT, and a collection of
native toolkit interfaces for the Macintosh.

\section{Built-in Module \sectcode{stdwin}}
\bimodindex{stdwin}

This module defines several new object types and functions that
provide access to the functionality of STDWIN.

On Unix running X11, it can only be used if the \code{DISPLAY}
environment variable is set or an explicit \samp{-display
\var{displayname}} argument is passed to the Python interpreter.

Functions have names that usually resemble their C STDWIN counterparts
with the initial `w' dropped.
Points are represented by pairs of integers; rectangles
by pairs of points.
For a complete description of STDWIN please refer to the documentation
of STDWIN for C programmers (aforementioned CWI report).

\subsection{Functions Defined in Module \sectcode{stdwin}}
\nodename{STDWIN Functions}

The following functions are defined in the \code{stdwin} module:

\renewcommand{\indexsubitem}{(in module stdwin)}
\begin{funcdesc}{open}{title}
Open a new window whose initial title is given by the string argument.
Return a window object; window object methods are described below.%
\footnote{The Python version of STDWIN does not support draw procedures; all
	drawing requests are reported as draw events.}
\end{funcdesc}

\begin{funcdesc}{getevent}{}
Wait for and return the next event.
An event is returned as a triple: the first element is the event
type, a small integer; the second element is the window object to which
the event applies, or
\code{None}
if it applies to no window in particular;
the third element is type-dependent.
Names for event types and command codes are defined in the standard
module
\code{stdwinevent}.
\end{funcdesc}

\begin{funcdesc}{pollevent}{}
Return the next event, if one is immediately available.
If no event is available, return \code{()}.
\end{funcdesc}

\begin{funcdesc}{getactive}{}
Return the window that is currently active, or \code{None} if no
window is currently active.  (This can be emulated by monitoring
WE_ACTIVATE and WE_DEACTIVATE events.)
\end{funcdesc}

\begin{funcdesc}{listfontnames}{pattern}
Return the list of font names in the system that match the pattern (a
string).  The pattern should normally be \code{'*'}; returns all
available fonts.  If the underlying window system is X11, other
patterns follow the standard X11 font selection syntax (as used e.g.
in resource definitions), i.e. the wildcard character \code{'*'}
matches any sequence of characters (including none) and \code{'?'}
matches any single character.
On the Macintosh this function currently returns an empty list.
\end{funcdesc}

\begin{funcdesc}{setdefscrollbars}{hflag\, vflag}
Set the flags controlling whether subsequently opened windows will
have horizontal and/or vertical scroll bars.
\end{funcdesc}

\begin{funcdesc}{setdefwinpos}{h\, v}
Set the default window position for windows opened subsequently.
\end{funcdesc}

\begin{funcdesc}{setdefwinsize}{width\, height}
Set the default window size for windows opened subsequently.
\end{funcdesc}

\begin{funcdesc}{getdefscrollbars}{}
Return the flags controlling whether subsequently opened windows will
have horizontal and/or vertical scroll bars.
\end{funcdesc}

\begin{funcdesc}{getdefwinpos}{}
Return the default window position for windows opened subsequently.
\end{funcdesc}

\begin{funcdesc}{getdefwinsize}{}
Return the default window size for windows opened subsequently.
\end{funcdesc}

\begin{funcdesc}{getscrsize}{}
Return the screen size in pixels.
\end{funcdesc}

\begin{funcdesc}{getscrmm}{}
Return the screen size in millimeters.
\end{funcdesc}

\begin{funcdesc}{fetchcolor}{colorname}
Return the pixel value corresponding to the given color name.
Return the default foreground color for unknown color names.
Hint: the following code tests whether you are on a machine that
supports more than two colors:
\bcode\begin{verbatim}
if stdwin.fetchcolor('black') <> \
          stdwin.fetchcolor('red') <> \
          stdwin.fetchcolor('white'):
    print 'color machine'
else:
    print 'monochrome machine'
\end{verbatim}\ecode
\end{funcdesc}

\begin{funcdesc}{setfgcolor}{pixel}
Set the default foreground color.
This will become the default foreground color of windows opened
subsequently, including dialogs.
\end{funcdesc}

\begin{funcdesc}{setbgcolor}{pixel}
Set the default background color.
This will become the default background color of windows opened
subsequently, including dialogs.
\end{funcdesc}

\begin{funcdesc}{getfgcolor}{}
Return the pixel value of the current default foreground color.
\end{funcdesc}

\begin{funcdesc}{getbgcolor}{}
Return the pixel value of the current default background color.
\end{funcdesc}

\begin{funcdesc}{setfont}{fontname}
Set the current default font.
This will become the default font for windows opened subsequently,
and is also used by the text measuring functions \code{textwidth},
\code{textbreak}, \code{lineheight} and \code{baseline} below.
This accepts two more optional parameters, size and style:
Size is the font size (in `points').
Style is a single character specifying the style, as follows:
\code{'b'} = bold,
\code{'i'} = italic,
\code{'o'} = bold + italic,
\code{'u'} = underline;
default style is roman.
Size and style are ignored under X11 but used on the Macintosh.
(Sorry for all this complexity --- a more uniform interface is being designed.)
\end{funcdesc}

\begin{funcdesc}{menucreate}{title}
Create a menu object referring to a global menu (a menu that appears in
all windows).
Methods of menu objects are described below.
Note: normally, menus are created locally; see the window method
\code{menucreate} below.
\strong{Warning:} the menu only appears in a window as long as the object
returned by this call exists.
\end{funcdesc}

\begin{funcdesc}{newbitmap}{width\, height}
Create a new bitmap object of the given dimensions.
Methods of bitmap objects are described below.
Not available on the Macintosh.
\end{funcdesc}

\begin{funcdesc}{fleep}{}
Cause a beep or bell (or perhaps a `visual bell' or flash, hence the
name).
\end{funcdesc}

\begin{funcdesc}{message}{string}
Display a dialog box containing the string.
The user must click OK before the function returns.
\end{funcdesc}

\begin{funcdesc}{askync}{prompt\, default}
Display a dialog that prompts the user to answer a question with yes or
no.
Return 0 for no, 1 for yes.
If the user hits the Return key, the default (which must be 0 or 1) is
returned.
If the user cancels the dialog, the
\code{KeyboardInterrupt}
exception is raised.
\end{funcdesc}

\begin{funcdesc}{askstr}{prompt\, default}
Display a dialog that prompts the user for a string.
If the user hits the Return key, the default string is returned.
If the user cancels the dialog, the
\code{KeyboardInterrupt}
exception is raised.
\end{funcdesc}

\begin{funcdesc}{askfile}{prompt\, default\, new}
Ask the user to specify a filename.
If
\var{new}
is zero it must be an existing file; otherwise, it must be a new file.
If the user cancels the dialog, the
\code{KeyboardInterrupt}
exception is raised.
\end{funcdesc}

\begin{funcdesc}{setcutbuffer}{i\, string}
Store the string in the system's cut buffer number
\var{i},
where it can be found (for pasting) by other applications.
On X11, there are 8 cut buffers (numbered 0..7).
Cut buffer number 0 is the `clipboard' on the Macintosh.
\end{funcdesc}

\begin{funcdesc}{getcutbuffer}{i}
Return the contents of the system's cut buffer number
\var{i}.
\end{funcdesc}

\begin{funcdesc}{rotatecutbuffers}{n}
On X11, rotate the 8 cut buffers by
\var{n}.
Ignored on the Macintosh.
\end{funcdesc}

\begin{funcdesc}{getselection}{i}
Return X11 selection number
\var{i.}
Selections are not cut buffers.
Selection numbers are defined in module
\code{stdwinevents}.
Selection \code{WS_PRIMARY} is the
\dfn{primary}
selection (used by
xterm,
for instance);
selection \code{WS_SECONDARY} is the
\dfn{secondary}
selection; selection \code{WS_CLIPBOARD} is the
\dfn{clipboard}
selection (used by
xclipboard).
On the Macintosh, this always returns an empty string.
\end{funcdesc}

\begin{funcdesc}{resetselection}{i}
Reset selection number
\var{i},
if this process owns it.
(See window method
\code{setselection()}).
\end{funcdesc}

\begin{funcdesc}{baseline}{}
Return the baseline of the current font (defined by STDWIN as the
vertical distance between the baseline and the top of the
characters).
\end{funcdesc}

\begin{funcdesc}{lineheight}{}
Return the total line height of the current font.
\end{funcdesc}

\begin{funcdesc}{textbreak}{str\, width}
Return the number of characters of the string that fit into a space of
\var{width}
bits wide when drawn in the curent font.
\end{funcdesc}

\begin{funcdesc}{textwidth}{str}
Return the width in bits of the string when drawn in the current font.
\end{funcdesc}

\begin{funcdesc}{connectionnumber}{}
\funcline{fileno}{}
(X11 under \UNIX{} only) Return the ``connection number'' used by the
underlying X11 implementation.  (This is normally the file number of
the socket.)  Both functions return the same value;
\code{connectionnumber()} is named after the corresponding function in
X11 and STDWIN, while \code{fileno()} makes it possible to use the
\code{stdwin} module as a ``file'' object parameter to
\code{select.select()}.  Note that if \code{select()} implies that
input is possible on \code{stdwin}, this does not guarantee that an
event is ready --- it may be some internal communication going on
between the X server and the client library.  Thus, you should call
\code{stdwin.pollevent()} until it returns \code{None} to check for
events if you don't want your program to block.  Because of internal
buffering in X11, it is also possible that \code{stdwin.pollevent()}
returns an event while \code{select()} does not find \code{stdwin} to
be ready, so you should read any pending events with
\code{stdwin.pollevent()} until it returns \code{None} before entering
a blocking \code{select()} call.
\ttindex{select}
\end{funcdesc}

\subsection{Window Objects}

Window objects are created by \code{stdwin.open()}.  They are closed
by their \code{close()} method or when they are garbage-collected.
Window objects have the following methods:

\renewcommand{\indexsubitem}{(window method)}

\begin{funcdesc}{begindrawing}{}
Return a drawing object, whose methods (described below) allow drawing
in the window.
\end{funcdesc}

\begin{funcdesc}{change}{rect}
Invalidate the given rectangle; this may cause a draw event.
\end{funcdesc}

\begin{funcdesc}{gettitle}{}
Returns the window's title string.
\end{funcdesc}

\begin{funcdesc}{getdocsize}{}
\begin{sloppypar}
Return a pair of integers giving the size of the document as set by
\code{setdocsize()}.
\end{sloppypar}
\end{funcdesc}

\begin{funcdesc}{getorigin}{}
Return a pair of integers giving the origin of the window with respect
to the document.
\end{funcdesc}

\begin{funcdesc}{gettitle}{}
Return the window's title string.
\end{funcdesc}

\begin{funcdesc}{getwinsize}{}
Return a pair of integers giving the size of the window.
\end{funcdesc}

\begin{funcdesc}{getwinpos}{}
Return a pair of integers giving the position of the window's upper
left corner (relative to the upper left corner of the screen).
\end{funcdesc}

\begin{funcdesc}{menucreate}{title}
Create a menu object referring to a local menu (a menu that appears
only in this window).
Methods of menu objects are described below.
{\bf Warning:} the menu only appears as long as the object
returned by this call exists.
\end{funcdesc}

\begin{funcdesc}{scroll}{rect\, point}
Scroll the given rectangle by the vector given by the point.
\end{funcdesc}

\begin{funcdesc}{setdocsize}{point}
Set the size of the drawing document.
\end{funcdesc}

\begin{funcdesc}{setorigin}{point}
Move the origin of the window (its upper left corner)
to the given point in the document.
\end{funcdesc}

\begin{funcdesc}{setselection}{i\, str}
Attempt to set X11 selection number
\var{i}
to the string
\var{str}.
(See stdwin method
\code{getselection()}
for the meaning of
\var{i}.)
Return true if it succeeds.
If  succeeds, the window ``owns'' the selection until
(a) another application takes ownership of the selection; or
(b) the window is deleted; or
(c) the application clears ownership by calling
\code{stdwin.resetselection(\var{i})}.
When another application takes ownership of the selection, a
\code{WE_LOST_SEL}
event is received for no particular window and with the selection number
as detail.
Ignored on the Macintosh.
\end{funcdesc}

\begin{funcdesc}{settimer}{dsecs}
Schedule a timer event for the window in
\code{\var{dsecs}/10}
seconds.
\end{funcdesc}

\begin{funcdesc}{settitle}{title}
Set the window's title string.
\end{funcdesc}

\begin{funcdesc}{setwincursor}{name}
\begin{sloppypar}
Set the window cursor to a cursor of the given name.
It raises the
\code{RuntimeError}
exception if no cursor of the given name exists.
Suitable names include
\code{'ibeam'},
\code{'arrow'},
\code{'cross'},
\code{'watch'}
and
\code{'plus'}.
On X11, there are many more (see
\file{<X11/cursorfont.h>}).
\end{sloppypar}
\end{funcdesc}

\begin{funcdesc}{setwinpos}{h\, v}
Set the the position of the window's upper left corner (relative to
the upper left corner of the screen).
\end{funcdesc}

\begin{funcdesc}{setwinsize}{width\, height}
Set the window's size.
\end{funcdesc}

\begin{funcdesc}{show}{rect}
Try to ensure that the given rectangle of the document is visible in
the window.
\end{funcdesc}

\begin{funcdesc}{textcreate}{rect}
Create a text-edit object in the document at the given rectangle.
Methods of text-edit objects are described below.
\end{funcdesc}

\begin{funcdesc}{setactive}{}
Attempt to make this window the active window.  If successful, this
will generate a WE_ACTIVATE event (and a WE_DEACTIVATE event in case
another window in this application became inactive).
\end{funcdesc}

\begin{funcdesc}{close}{}
Discard the window object.  It should not be used again.
\end{funcdesc}

\subsection{Drawing Objects}

Drawing objects are created exclusively by the window method
\code{begindrawing()}.
Only one drawing object can exist at any given time; the drawing object
must be deleted to finish drawing.
No drawing object may exist when
\code{stdwin.getevent()}
is called.
Drawing objects have the following methods:

\renewcommand{\indexsubitem}{(drawing method)}

\begin{funcdesc}{box}{rect}
Draw a box just inside a rectangle.
\end{funcdesc}

\begin{funcdesc}{circle}{center\, radius}
Draw a circle with given center point and radius.
\end{funcdesc}

\begin{funcdesc}{elarc}{center\, \(rh\, rv\)\, \(a1\, a2\)}
Draw an elliptical arc with given center point.
\code{(\var{rh}, \var{rv})}
gives the half sizes of the horizontal and vertical radii.
\code{(\var{a1}, \var{a2})}
gives the angles (in degrees) of the begin and end points.
0 degrees is at 3 o'clock, 90 degrees is at 12 o'clock.
\end{funcdesc}

\begin{funcdesc}{erase}{rect}
Erase a rectangle.
\end{funcdesc}

\begin{funcdesc}{fillcircle}{center\, radius}
Draw a filled circle with given center point and radius.
\end{funcdesc}

\begin{funcdesc}{fillelarc}{center\, \(rh\, rv\)\, \(a1\, a2\)}
Draw a filled elliptical arc; arguments as for \code{elarc}.
\end{funcdesc}

\begin{funcdesc}{fillpoly}{points}
Draw a filled polygon given by a list (or tuple) of points.
\end{funcdesc}

\begin{funcdesc}{invert}{rect}
Invert a rectangle.
\end{funcdesc}

\begin{funcdesc}{line}{p1\, p2}
Draw a line from point
\var{p1}
to
\var{p2}.
\end{funcdesc}

\begin{funcdesc}{paint}{rect}
Fill a rectangle.
\end{funcdesc}

\begin{funcdesc}{poly}{points}
Draw the lines connecting the given list (or tuple) of points.
\end{funcdesc}

\begin{funcdesc}{shade}{rect\, percent}
Fill a rectangle with a shading pattern that is about
\var{percent}
percent filled.
\end{funcdesc}

\begin{funcdesc}{text}{p\, str}
Draw a string starting at point p (the point specifies the
top left coordinate of the string).
\end{funcdesc}

\begin{funcdesc}{xorcircle}{center\, radius}
\funcline{xorelarc}{center\, \(rh\, rv\)\, \(a1\, a2\)}
\funcline{xorline}{p1\, p2}
\funcline{xorpoly}{points}
Draw a circle, an elliptical arc, a line or a polygon, respectively,
in XOR mode.
\end{funcdesc}

\begin{funcdesc}{setfgcolor}{}
\funcline{setbgcolor}{}
\funcline{getfgcolor}{}
\funcline{getbgcolor}{}
These functions are similar to the corresponding functions described
above for the
\code{stdwin}
module, but affect or return the colors currently used for drawing
instead of the global default colors.
When a drawing object is created, its colors are set to the window's
default colors, which are in turn initialized from the global default
colors when the window is created.
\end{funcdesc}

\begin{funcdesc}{setfont}{}
\funcline{baseline}{}
\funcline{lineheight}{}
\funcline{textbreak}{}
\funcline{textwidth}{}
These functions are similar to the corresponding functions described
above for the
\code{stdwin}
module, but affect or use the current drawing font instead of
the global default font.
When a drawing object is created, its font is set to the window's
default font, which is in turn initialized from the global default
font when the window is created.
\end{funcdesc}

\begin{funcdesc}{bitmap}{point\, bitmap\, mask}
Draw the \var{bitmap} with its top left corner at \var{point}.
If the optional \var{mask} argument is present, it should be either
the same object as \var{bitmap}, to draw only those bits that are set
in the bitmap, in the foreground color, or \code{None}, to draw all
bits (ones are drawn in the foreground color, zeros in the background
color).
Not available on the Macintosh.
\end{funcdesc}

\begin{funcdesc}{cliprect}{rect}
Set the ``clipping region'' to a rectangle.
The clipping region limits the effect of all drawing operations, until
it is changed again or until the drawing object is closed.  When a
drawing object is created the clipping region is set to the entire
window.  When an object to be drawn falls partly outside the clipping
region, the set of pixels drawn is the intersection of the clipping
region and the set of pixels that would be drawn by the same operation
in the absence of a clipping region.
\end{funcdesc}

\begin{funcdesc}{noclip}{}
Reset the clipping region to the entire window.
\end{funcdesc}

\begin{funcdesc}{close}{}
\funcline{enddrawing}{}
Discard the drawing object.  It should not be used again.
\end{funcdesc}

\subsection{Menu Objects}

A menu object represents a menu.
The menu is destroyed when the menu object is deleted.
The following methods are defined:

\renewcommand{\indexsubitem}{(menu method)}

\begin{funcdesc}{additem}{text\, shortcut}
Add a menu item with given text.
The shortcut must be a string of length 1, or omitted (to specify no
shortcut).
\end{funcdesc}

\begin{funcdesc}{setitem}{i\, text}
Set the text of item number
\var{i}.
\end{funcdesc}

\begin{funcdesc}{enable}{i\, flag}
Enable or disables item
\var{i}.
\end{funcdesc}

\begin{funcdesc}{check}{i\, flag}
Set or clear the
\dfn{check mark}
for item
\var{i}.
\end{funcdesc}

\begin{funcdesc}{close}{}
Discard the menu object.  It should not be used again.
\end{funcdesc}

\subsection{Bitmap Objects}

A bitmap represents a rectangular array of bits.
The top left bit has coordinate (0, 0).
A bitmap can be drawn with the \code{bitmap} method of a drawing object.
Bitmaps are currently not available on the Macintosh.

The following methods are defined:

\renewcommand{\indexsubitem}{(bitmap method)}

\begin{funcdesc}{getsize}{}
Return a tuple representing the width and height of the bitmap.
(This returns the values that have been passed to the \code{newbitmap}
function.)
\end{funcdesc}

\begin{funcdesc}{setbit}{point\, bit}
Set the value of the bit indicated by \var{point} to \var{bit}.
\end{funcdesc}

\begin{funcdesc}{getbit}{point}
Return the value of the bit indicated by \var{point}.
\end{funcdesc}

\begin{funcdesc}{close}{}
Discard the bitmap object.  It should not be used again.
\end{funcdesc}

\subsection{Text-edit Objects}

A text-edit object represents a text-edit block.
For semantics, see the STDWIN documentation for C programmers.
The following methods exist:

\renewcommand{\indexsubitem}{(text-edit method)}

\begin{funcdesc}{arrow}{code}
Pass an arrow event to the text-edit block.
The
\var{code}
must be one of
\code{WC_LEFT},
\code{WC_RIGHT},
\code{WC_UP}
or
\code{WC_DOWN}
(see module
\code{stdwinevents}).
\end{funcdesc}

\begin{funcdesc}{draw}{rect}
Pass a draw event to the text-edit block.
The rectangle specifies the redraw area.
\end{funcdesc}

\begin{funcdesc}{event}{type\, window\, detail}
Pass an event gotten from
\code{stdwin.getevent()}
to the text-edit block.
Return true if the event was handled.
\end{funcdesc}

\begin{funcdesc}{getfocus}{}
Return 2 integers representing the start and end positions of the
focus, usable as slice indices on the string returned by
\code{gettext()}.
\end{funcdesc}

\begin{funcdesc}{getfocustext}{}
Return the text in the focus.
\end{funcdesc}

\begin{funcdesc}{getrect}{}
Return a rectangle giving the actual position of the text-edit block.
(The bottom coordinate may differ from the initial position because
the block automatically shrinks or grows to fit.)
\end{funcdesc}

\begin{funcdesc}{gettext}{}
Return the entire text buffer.
\end{funcdesc}

\begin{funcdesc}{move}{rect}
Specify a new position for the text-edit block in the document.
\end{funcdesc}

\begin{funcdesc}{replace}{str}
Replace the text in the focus by the given string.
The new focus is an insert point at the end of the string.
\end{funcdesc}

\begin{funcdesc}{setfocus}{i\, j}
Specify the new focus.
Out-of-bounds values are silently clipped.
\end{funcdesc}

\begin{funcdesc}{settext}{str}
Replace the entire text buffer by the given string and set the focus
to \code{(0, 0)}.
\end{funcdesc}

\begin{funcdesc}{setview}{rect}
Set the view rectangle to \var{rect}.  If \var{rect} is \code{None},
viewing mode is reset.  In viewing mode, all output from the text-edit
object is clipped to the viewing rectangle.  This may be useful to
implement your own scrolling text subwindow.
\end{funcdesc}

\begin{funcdesc}{close}{}
Discard the text-edit object.  It should not be used again.
\end{funcdesc}

\subsection{Example}
\nodename{STDWIN Example}

Here is a minimal example of using STDWIN in Python.
It creates a window and draws the string ``Hello world'' in the top
left corner of the window.
The window will be correctly redrawn when covered and re-exposed.
The program quits when the close icon or menu item is requested.

\bcode\begin{verbatim}
import stdwin
from stdwinevents import *

def main():
    mywin = stdwin.open('Hello')
    #
    while 1:
        (type, win, detail) = stdwin.getevent()
        if type == WE_DRAW:
            draw = win.begindrawing()
            draw.text((0, 0), 'Hello, world')
            del draw
        elif type == WE_CLOSE:
            break

main()
\end{verbatim}\ecode

\section{Standard Module \sectcode{stdwinevents}}
\stmodindex{stdwinevents}

This module defines constants used by STDWIN for event types
(\code{WE_ACTIVATE} etc.), command codes (\code{WC_LEFT} etc.)
and selection types (\code{WS_PRIMARY} etc.).
Read the file for details.
Suggested usage is

\bcode\begin{verbatim}
>>> from stdwinevents import *
>>> 
\end{verbatim}\ecode

\section{Standard Module \sectcode{rect}}
\stmodindex{rect}

This module contains useful operations on rectangles.
A rectangle is defined as in module
\code{stdwin}:
a pair of points, where a point is a pair of integers.
For example, the rectangle

\bcode\begin{verbatim}
(10, 20), (90, 80)
\end{verbatim}\ecode

is a rectangle whose left, top, right and bottom edges are 10, 20, 90
and 80, respectively.
Note that the positive vertical axis points down (as in
\code{stdwin}).

The module defines the following objects:

\renewcommand{\indexsubitem}{(in module rect)}
\begin{excdesc}{error}
The exception raised by functions in this module when they detect an
error.
The exception argument is a string describing the problem in more
detail.
\end{excdesc}

\begin{datadesc}{empty}
The rectangle returned when some operations return an empty result.
This makes it possible to quickly check whether a result is empty:

\bcode\begin{verbatim}
>>> import rect
>>> r1 = (10, 20), (90, 80)
>>> r2 = (0, 0), (10, 20)
>>> r3 = rect.intersect([r1, r2])
>>> if r3 is rect.empty: print 'Empty intersection'
Empty intersection
>>> 
\end{verbatim}\ecode
\end{datadesc}

\begin{funcdesc}{is_empty}{r}
Returns true if the given rectangle is empty.
A rectangle
\code{(\var{left}, \var{top}), (\var{right}, \var{bottom})}
is empty if
\iftexi
\code{\var{left} >= \var{right}} or \code{\var{top} => \var{bottom}}.
\else
$\var{left} \geq \var{right}$ or $\var{top} \geq \var{bottom}$.
%%JHXXX{\em left~$\geq$~right} or {\em top~$\leq$~bottom}.
\fi
\end{funcdesc}

\begin{funcdesc}{intersect}{list}
Returns the intersection of all rectangles in the list argument.
It may also be called with a tuple argument.
Raises
\code{rect.error}
if the list is empty.
Returns
\code{rect.empty}
if the intersection of the rectangles is empty.
\end{funcdesc}

\begin{funcdesc}{union}{list}
Returns the smallest rectangle that contains all non-empty rectangles in
the list argument.
It may also be called with a tuple argument or with two or more
rectangles as arguments.
Returns
\code{rect.empty}
if the list is empty or all its rectangles are empty.
\end{funcdesc}

\begin{funcdesc}{pointinrect}{point\, rect}
Returns true if the point is inside the rectangle.
By definition, a point
\code{(\var{h}, \var{v})}
is inside a rectangle
\code{(\var{left}, \var{top}), (\var{right}, \var{bottom})} if
\iftexi
\code{\var{left} <= \var{h} < \var{right}} and
\code{\var{top} <= \var{v} < \var{bottom}}.
\else
$\var{left} \leq \var{h} < \var{right}$ and
$\var{top} \leq \var{v} < \var{bottom}$.
\fi
\end{funcdesc}

\begin{funcdesc}{inset}{rect\, \(dh\, dv\)}
Returns a rectangle that lies inside the
\code{rect}
argument by
\var{dh}
pixels horizontally
and
\var{dv}
pixels
vertically.
If
\var{dh}
or
\var{dv}
is negative, the result lies outside
\var{rect}.
\end{funcdesc}

\begin{funcdesc}{rect2geom}{rect}
Converts a rectangle to geometry representation:
\code{(\var{left}, \var{top}), (\var{width}, \var{height})}.
\end{funcdesc}

\begin{funcdesc}{geom2rect}{geom}
Converts a rectangle given in geometry representation back to the
standard rectangle representation
\code{(\var{left}, \var{top}), (\var{right}, \var{bottom})}.
\end{funcdesc}
		% STDWIN ONLY

\chapter{SGI IRIX Specific Services}

The modules described in this chapter provide interfaces to features
that are unique to SGI's IRIX operating system (versions 4 and 5).
			% SGI IRIX ONLY
\section{Built-in Module \sectcode{al}}
\bimodindex{al}

This module provides access to the audio facilities of the SGI Indy
and Indigo workstations.  See section 3A of the IRIX man pages for
details.  You'll need to read those man pages to understand what these
functions do!  Some of the functions are not available in IRIX
releases before 4.0.5.  Again, see the manual to check whether a
specific function is available on your platform.

All functions and methods defined in this module are equivalent to
the C functions with \samp{AL} prefixed to their name.

Symbolic constants from the C header file \file{<audio.h>} are defined
in the standard module \code{AL}, see below.

\strong{Warning:} the current version of the audio library may dump core
when bad argument values are passed rather than returning an error
status.  Unfortunately, since the precise circumstances under which
this may happen are undocumented and hard to check, the Python
interface can provide no protection against this kind of problems.
(One example is specifying an excessive queue size --- there is no
documented upper limit.)

The module defines the following functions:

\renewcommand{\indexsubitem}{(in module al)}

\begin{funcdesc}{openport}{name\, direction\optional{\, config}}
The name and direction arguments are strings.  The optional config
argument is a configuration object as returned by
\code{al.newconfig()}.  The return value is an \dfn{port object};
methods of port objects are described below.
\end{funcdesc}

\begin{funcdesc}{newconfig}{}
The return value is a new \dfn{configuration object}; methods of
configuration objects are described below.
\end{funcdesc}

\begin{funcdesc}{queryparams}{device}
The device argument is an integer.  The return value is a list of
integers containing the data returned by ALqueryparams().
\end{funcdesc}

\begin{funcdesc}{getparams}{device\, list}
The device argument is an integer.  The list argument is a list such
as returned by \code{queryparams}; it is modified in place (!).
\end{funcdesc}

\begin{funcdesc}{setparams}{device\, list}
The device argument is an integer.  The list argument is a list such
as returned by \code{al.queryparams}.
\end{funcdesc}

\subsection{Configuration Objects}

Configuration objects (returned by \code{al.newconfig()} have the
following methods:

\renewcommand{\indexsubitem}{(audio configuration object method)}

\begin{funcdesc}{getqueuesize}{}
Return the queue size.
\end{funcdesc}

\begin{funcdesc}{setqueuesize}{size}
Set the queue size.
\end{funcdesc}

\begin{funcdesc}{getwidth}{}
Get the sample width.
\end{funcdesc}

\begin{funcdesc}{setwidth}{width}
Set the sample width.
\end{funcdesc}

\begin{funcdesc}{getchannels}{}
Get the channel count.
\end{funcdesc}

\begin{funcdesc}{setchannels}{nchannels}
Set the channel count.
\end{funcdesc}

\begin{funcdesc}{getsampfmt}{}
Get the sample format.
\end{funcdesc}

\begin{funcdesc}{setsampfmt}{sampfmt}
Set the sample format.
\end{funcdesc}

\begin{funcdesc}{getfloatmax}{}
Get the maximum value for floating sample formats.
\end{funcdesc}

\begin{funcdesc}{setfloatmax}{floatmax}
Set the maximum value for floating sample formats.
\end{funcdesc}

\subsection{Port Objects}

Port objects (returned by \code{al.openport()} have the following
methods:

\renewcommand{\indexsubitem}{(audio port object method)}

\begin{funcdesc}{closeport}{}
Close the port.
\end{funcdesc}

\begin{funcdesc}{getfd}{}
Return the file descriptor as an int.
\end{funcdesc}

\begin{funcdesc}{getfilled}{}
Return the number of filled samples.
\end{funcdesc}

\begin{funcdesc}{getfillable}{}
Return the number of fillable samples.
\end{funcdesc}

\begin{funcdesc}{readsamps}{nsamples}
Read a number of samples from the queue, blocking if necessary.
Return the data as a string containing the raw data, (e.g., 2 bytes per
sample in big-endian byte order (high byte, low byte) if you have set
the sample width to 2 bytes).
\end{funcdesc}

\begin{funcdesc}{writesamps}{samples}
Write samples into the queue, blocking if necessary.  The samples are
encoded as described for the \code{readsamps} return value.
\end{funcdesc}

\begin{funcdesc}{getfillpoint}{}
Return the `fill point'.
\end{funcdesc}

\begin{funcdesc}{setfillpoint}{fillpoint}
Set the `fill point'.
\end{funcdesc}

\begin{funcdesc}{getconfig}{}
Return a configuration object containing the current configuration of
the port.
\end{funcdesc}

\begin{funcdesc}{setconfig}{config}
Set the configuration from the argument, a configuration object.
\end{funcdesc}

\begin{funcdesc}{getstatus}{list}
Get status information on last error.
\end{funcdesc}

\section{Standard Module \sectcode{AL}}
\nodename{AL (uppercase)}
\stmodindex{AL}

This module defines symbolic constants needed to use the built-in
module \code{al} (see above); they are equivalent to those defined in
the C header file \file{<audio.h>} except that the name prefix
\samp{AL_} is omitted.  Read the module source for a complete list of
the defined names.  Suggested use:

\bcode\begin{verbatim}
import al
from AL import *
\end{verbatim}\ecode

%\section{Built-in Module \sectcode{audio}}
\bimodindex{audio}

\strong{Note:} This module is obsolete, since the hardware to which it
interfaces is obsolete.  For audio on the Indigo or 4D/35, see
built-in module \code{al} above.

This module provides rudimentary access to the audio I/O device
\file{/dev/audio} on the Silicon Graphics Personal IRIS 4D/25;
see {\it audio}(7). It supports the following operations:

\renewcommand{\indexsubitem}{(in module audio)}
\begin{funcdesc}{setoutgain}{n}
Sets the output gain.
\iftexi
\code{0 <= \var{n} < 256}.
\else
$0 \leq \var{n} < 256$.
%%JHXXX Sets the output gain (0-255).
\fi
\end{funcdesc}

\begin{funcdesc}{getoutgain}{}
Returns the output gain.
\end{funcdesc}

\begin{funcdesc}{setrate}{n}
Sets the sampling rate: \code{1} = 32K/sec, \code{2} = 16K/sec,
\code{3} = 8K/sec.
\end{funcdesc}

\begin{funcdesc}{setduration}{n}
Sets the `sound duration' in units of 1/100 seconds.
\end{funcdesc}

\begin{funcdesc}{read}{n}
Reads a chunk of
\var{n}
sampled bytes from the audio input (line in or microphone).
The chunk is returned as a string of length n.
Each byte encodes one sample as a signed 8-bit quantity using linear
encoding.
This string can be converted to numbers using \code{chr2num()} described
below.
\end{funcdesc}

\begin{funcdesc}{write}{buf}
Writes a chunk of samples to the audio output (speaker).
\end{funcdesc}

These operations support asynchronous audio I/O:

\renewcommand{\indexsubitem}{(in module audio)}
\begin{funcdesc}{start_recording}{n}
Starts a second thread (a process with shared memory) that begins reading
\var{n}
bytes from the audio device.
The main thread immediately continues.
\end{funcdesc}

\begin{funcdesc}{wait_recording}{}
Waits for the second thread to finish and returns the data read.
\end{funcdesc}

\begin{funcdesc}{stop_recording}{}
Makes the second thread stop reading as soon as possible.
Returns the data read so far.
\end{funcdesc}

\begin{funcdesc}{poll_recording}{}
Returns true if the second thread has finished reading (so
\code{wait_recording()} would return the data without delay).
\end{funcdesc}

\begin{funcdesc}{start_playing}{}
\funcline{wait_playing}{}
\funcline{stop_playing}{}
\funcline{poll_playing}{}
\begin{sloppypar}
Similar but for output.
\code{stop_playing()}
returns a lower bound for the number of bytes actually played (not very
accurate).
\end{sloppypar}
\end{funcdesc}

The following operations do not affect the audio device but are
implemented in C for efficiency:

\renewcommand{\indexsubitem}{(in module audio)}
\begin{funcdesc}{amplify}{buf\, f1\, f2}
Amplifies a chunk of samples by a variable factor changing from
\code{\var{f1}/256} to \code{\var{f2}/256.}
Negative factors are allowed.
Resulting values that are to large to fit in a byte are clipped.         
\end{funcdesc}

\begin{funcdesc}{reverse}{buf}
Returns a chunk of samples backwards.
\end{funcdesc}

\begin{funcdesc}{add}{buf1\, buf2}
Bytewise adds two chunks of samples.
Bytes that exceed the range are clipped.
If one buffer is shorter, it is assumed to be padded with zeros.
\end{funcdesc}

\begin{funcdesc}{chr2num}{buf}
Converts a string of sampled bytes as returned by \code{read()} into
a list containing the numeric values of the samples.
\end{funcdesc}

\begin{funcdesc}{num2chr}{list}
\begin{sloppypar}
Converts a list as returned by
\code{chr2num()}
back to a buffer acceptable by
\code{write()}.
\end{sloppypar}
\end{funcdesc}

\section{Built-in Module \sectcode{cd}}
\bimodindex{cd}

This module provides an interface to the Silicon Graphics CD library.
It is available only on Silicon Graphics systems.

The way the library works is as follows.  A program opens the CD-ROM
device with \code{cd.open()} and creates a parser to parse the data
from the CD with \code{cd.createparser()}.  The object returned by
\code{cd.open()} can be used to read data from the CD, but also to get
status information for the CD-ROM device, and to get information about
the CD, such as the table of contents.  Data from the CD is passed to
the parser, which parses the frames, and calls any callback
functions that have previously been added.

An audio CD is divided into \dfn{tracks} or \dfn{programs} (the terms
are used interchangeably).  Tracks can be subdivided into
\dfn{indices}.  An audio CD contains a \dfn{table of contents} which
gives the starts of the tracks on the CD.  Index 0 is usually the
pause before the start of a track.  The start of the track as given by
the table of contents is normally the start of index 1.

Positions on a CD can be represented in two ways.  Either a frame
number or a tuple of three values, minutes, seconds and frames.  Most
functions use the latter representation.  Positions can be both
relative to the beginning of the CD, and to the beginning of the
track.

Module \code{cd} defines the following functions and constants:

\renewcommand{\indexsubitem}{(in module cd)}

\begin{funcdesc}{createparser}{}
Create and return an opaque parser object.  The methods of the parser
object are described below.
\end{funcdesc}

\begin{funcdesc}{msftoframe}{min\, sec\, frame}
Converts a \code{(minutes, seconds, frames)} triple representing time
in absolute time code into the corresponding CD frame number.
\end{funcdesc}

\begin{funcdesc}{open}{\optional{device\optional{\, mode}}}
Open the CD-ROM device.  The return value is an opaque player object;
methods of the player object are described below.  The device is the
name of the SCSI device file, e.g. /dev/scsi/sc0d4l0, or \code{None}.
If omited or \code{None}, the hardware inventory is consulted to
locate a CD-ROM drive.  The \code{mode}, if not omited, should be the
string 'r'.
\end{funcdesc}

The module defines the following variables:

\begin{datadesc}{error}
Exception raised on various errors.
\end{datadesc}

\begin{datadesc}{DATASIZE}
The size of one frame's worth of audio data.  This is the size of the
audio data as passed to the callback of type \code{audio}.
\end{datadesc}

\begin{datadesc}{BLOCKSIZE}
The size of one uninterpreted frame of audio data.
\end{datadesc}

The following variables are states as returned by \code{getstatus}:

\begin{datadesc}{READY}
The drive is ready for operation loaded with an audio CD.
\end{datadesc}

\begin{datadesc}{NODISC}
The drive does not have a CD loaded.
\end{datadesc}

\begin{datadesc}{CDROM}
The drive is loaded with a CD-ROM.  Subsequent play or read operations
will return I/O errors.
\end{datadesc}

\begin{datadesc}{ERROR}
An error aoocurred while trying to read the disc or its table of
contents.
\end{datadesc}

\begin{datadesc}{PLAYING}
The drive is in CD player mode playing an audio CD through its audio
jacks.
\end{datadesc}

\begin{datadesc}{PAUSED}
The drive is in CD layer mode with play paused.
\end{datadesc}

\begin{datadesc}{STILL}
The equivalent of \code{PAUSED} on older (non 3301) model Toshiba
CD-ROM drives.  Such drives have never been shipped by SGI.
\end{datadesc}

\begin{datadesc}{audio}
\dataline{pnum}
\dataline{index}
\dataline{ptime}
\dataline{atime}
\dataline{catalog}
\dataline{ident}
\dataline{control}
Integer constants describing the various types of parser callbacks
that can be set by the \code{addcallback()} method of CD parser
objects (see below).
\end{datadesc}

Player objects (returned by \code{cd.open()}) have the following
methods:

\renewcommand{\indexsubitem}{(CD player object method)}

\begin{funcdesc}{allowremoval}{}
Unlocks the eject button on the CD-ROM drive permitting the user to
eject the caddy if desired.
\end{funcdesc}

\begin{funcdesc}{bestreadsize}{}
Returns the best value to use for the \code{num_frames} parameter of
the \code{readda} method.  Best is defined as the value that permits a
continuous flow of data from the CD-ROM drive.
\end{funcdesc}

\begin{funcdesc}{close}{}
Frees the resources associated with the player object.  After calling
\code{close}, the methods of the object should no longer be used.
\end{funcdesc}

\begin{funcdesc}{eject}{}
Ejects the caddy from the CD-ROM drive.
\end{funcdesc}

\begin{funcdesc}{getstatus}{}
Returns information pertaining to the current state of the CD-ROM
drive.  The returned information is a tuple with the following values:
\code{state}, \code{track}, \code{rtime}, \code{atime}, \code{ttime},
\code{first}, \code{last}, \code{scsi_audio}, \code{cur_block}.
\code{rtime} is the time relative to the start of the current track;
\code{atime} is the time relative to the beginning of the disc;
\code{ttime} is the total time on the disc.  For more information on
the meaning of the values, see the manual for CDgetstatus.
The value of \code{state} is one of the following: \code{cd.ERROR},
\code{cd.NODISC}, \code{cd.READY}, \code{cd.PLAYING},
\code{cd.PAUSED}, \code{cd.STILL}, or \code{cd.CDROM}.
\end{funcdesc}

\begin{funcdesc}{gettrackinfo}{track}
Returns information about the specified track.  The returned
information is a tuple consisting of two elements, the start time of
the track and the duration of the track.
\end{funcdesc}

\begin{funcdesc}{msftoblock}{min\, sec\, frame}
Converts a minutes, seconds, frames triple representing a time in
absolute time code into the corresponding logical block number for the
given CD-ROM drive.  You should use \code{cd.msftoframe()} rather than
\code{msftoblock()} for comparing times.  The logical block number
differs from the frame number by an offset required by certain CD-ROM
drives.
\end{funcdesc}

\begin{funcdesc}{play}{start\, play}
Starts playback of an audio CD in the CD-ROM drive at the specified
track.  The audio output appears on the CD-ROM drive's headphone and
audio jacks (if fitted).  Play stops at the end of the disc.
\code{start} is the number of the track at which to start playing the
CD; if \code{play} is 0, the CD will be set to an initial paused
state.  The method \code{togglepause()} can then be used to commence
play.
\end{funcdesc}

\begin{funcdesc}{playabs}{min\, sec\, frame\, play}
Like \code{play()}, except that the start is given in minutes,
seconds, frames instead of a track number.
\end{funcdesc}

\begin{funcdesc}{playtrack}{start\, play}
Like \code{play()}, except that playing stops at the end of the track.
\end{funcdesc}

\begin{funcdesc}{playtrackabs}{track\, min\, sec\, frame\, play}
Like \code{play()}, except that playing begins at the spcified
absolute time and ends at the end of the specified track.
\end{funcdesc}

\begin{funcdesc}{preventremoval}{}
Locks the eject button on the CD-ROM drive thus preventing the user
from arbitrarily ejecting the caddy.
\end{funcdesc}

\begin{funcdesc}{readda}{num_frames}
Reads the specified number of frames from an audio CD mounted in the
CD-ROM drive.  The return value is a string representing the audio
frames.  This string can be passed unaltered to the \code{parseframe}
method of the parser object.
\end{funcdesc}

\begin{funcdesc}{seek}{min\, sec\, frame}
Sets the pointer that indicates the starting point of the next read of
digital audio data from a CD-ROM.  The pointer is set to an absolute
time code location specified in minutes, seconds, and frames.  The
return value is the logical block number to which the pointer has been
set.
\end{funcdesc}

\begin{funcdesc}{seekblock}{block}
Sets the pointer that indicates the starting point of the next read of
digital audio data from a CD-ROM.  The pointer is set to the specified
logical block number.  The return value is the logical block number to
which the pointer has been set.
\end{funcdesc}

\begin{funcdesc}{seektrack}{track}
Sets the pointer that indicates the starting point of the next read of
digital audio data from a CD-ROM.  The pointer is set to the specified
track.  The return value is the logical block number to which the
pointer has been set.
\end{funcdesc}

\begin{funcdesc}{stop}{}
Stops the current playing operation.
\end{funcdesc}

\begin{funcdesc}{togglepause}{}
Pauses the CD if it is playing, and makes it play if it is paused.
\end{funcdesc}

Parser objects (returned by \code{cd.createparser()}) have the
following methods:

\renewcommand{\indexsubitem}{(CD parser object method)}

\begin{funcdesc}{addcallback}{type\, func\, arg}
Adds a callback for the parser.  The parser has callbacks for eight
different types of data in the digital audio data stream.  Constants
for these types are defined at the \code{cd} module level (see above).
The callback is called as follows: \code{func(arg, type, data)}, where
\code{arg} is the user supplied argument, \code{type} is the
particular type of callback, and \code{data} is the data returned for
this \code{type} of callback.  The type of the data depends on the
\code{type} of callback as follows:
\begin{description}
\item[\code{cd.audio}: ]
The argument is a string which can be passed unmodified to
\code{al.writesamps()}.
\item[\code{cd.pnum}: ]
The argument is an integer giving the program (track) number.
\item[\code{cd.index}: ]
The argument is an integer giving the index number.
\item[\code{cd.ptime}: ]
The argument is a tuple consisting of the program time in minutes,
seconds, and frames.
\item[\code{cd.atime}: ]
The argument is a tuple consisting of the absolute time in minutes,
seconds, and frames.
\item[\code{cd.catalog}: ]
The argument is a string of 13 characters, giving the catalog number
of the CD.
\item[\code{cd.ident}: ]
The argument is a string of 12 characters, giving the ISRC
identification number of the recording.  The string consists of two
characters country code, three characters owner code, two characters
giving the year, and five characters giving a serial number.
\item[\code{cd.control}: ]
The argument is an integer giving the control bits from the CD subcode
data.
\end{description}
\end{funcdesc}

\begin{funcdesc}{deleteparser}{}
Deletes the parser and frees the memory it was using.  The object
should not be used after this call.  This call is done automatically
when the last reference to the object is removed.
\end{funcdesc}

\begin{funcdesc}{parseframe}{frame}
Parses one or more frames of digital audio data from a CD such as
returned by \code{readda}.  It determines which subcodes are present
in the data.  If these subcodes have changed since the last frame,
then \code{parseframe} executes a callback of the appropriate type
passing to it the subcode data found in the frame.
Unlike the C function, more than one frame of digital audio data can
be passed to this method.
\end{funcdesc}

\begin{funcdesc}{removecallback}{type}
Removes the callback for the given \code{type}.
\end{funcdesc}

\begin{funcdesc}{resetparser}{}
Resets the fields of the parser used for tracking subcodes to an
initial state.  \code{resetparser} should be called after the disc has
been changed.
\end{funcdesc}

\section{Built-in Module \sectcode{fl}}
\bimodindex{fl}

This module provides an interface to the FORMS Library by Mark
Overmars.  The source for the library can be retrieved by anonymous
ftp from host \samp{ftp.cs.ruu.nl}, directory \file{SGI/FORMS}.  It
was last tested with version 2.0b.

Most functions are literal translations of their C equivalents,
dropping the initial \samp{fl_} from their name.  Constants used by
the library are defined in module \code{FL} described below.

The creation of objects is a little different in Python than in C:
instead of the `current form' maintained by the library to which new
FORMS objects are added, all functions that add a FORMS object to a
form are methods of the Python object representing the form.
Consequently, there are no Python equivalents for the C functions
\code{fl_addto_form} and \code{fl_end_form}, and the equivalent of
\code{fl_bgn_form} is called \code{fl.make_form}.

Watch out for the somewhat confusing terminology: FORMS uses the word
\dfn{object} for the buttons, sliders etc. that you can place in a form.
In Python, `object' means any value.  The Python interface to FORMS
introduces two new Python object types: form objects (representing an
entire form) and FORMS objects (representing one button, slider etc.).
Hopefully this isn't too confusing...

There are no `free objects' in the Python interface to FORMS, nor is
there an easy way to add object classes written in Python.  The FORMS
interface to GL event handling is available, though, so you can mix
FORMS with pure GL windows.

\strong{Please note:} importing \code{fl} implies a call to the GL function
\code{foreground()} and to the FORMS routine \code{fl_init()}.

\subsection{Functions Defined in Module \sectcode{fl}}
\nodename{FL Functions}

Module \code{fl} defines the following functions.  For more information
about what they do, see the description of the equivalent C function
in the FORMS documentation:

\renewcommand{\indexsubitem}{(in module fl)}
\begin{funcdesc}{make_form}{type\, width\, height}
Create a form with given type, width and height.  This returns a
\dfn{form} object, whose methods are described below.
\end{funcdesc}

\begin{funcdesc}{do_forms}{}
The standard FORMS main loop.  Returns a Python object representing
the FORMS object needing interaction, or the special value
\code{FL.EVENT}.
\end{funcdesc}

\begin{funcdesc}{check_forms}{}
Check for FORMS events.  Returns what \code{do_forms} above returns,
or \code{None} if there is no event that immediately needs
interaction.
\end{funcdesc}

\begin{funcdesc}{set_event_call_back}{function}
Set the event callback function.
\end{funcdesc}

\begin{funcdesc}{set_graphics_mode}{rgbmode\, doublebuffering}
Set the graphics modes.
\end{funcdesc}

\begin{funcdesc}{get_rgbmode}{}
Return the current rgb mode.  This is the value of the C global
variable \code{fl_rgbmode}.
\end{funcdesc}

\begin{funcdesc}{show_message}{str1\, str2\, str3}
Show a dialog box with a three-line message and an OK button.
\end{funcdesc}

\begin{funcdesc}{show_question}{str1\, str2\, str3}
Show a dialog box with a three-line message and YES and NO buttons.
It returns \code{1} if the user pressed YES, \code{0} if NO.
\end{funcdesc}

\begin{funcdesc}{show_choice}{str1\, str2\, str3\, but1\optional{\, but2\,
but3}}
Show a dialog box with a three-line message and up to three buttons.
It returns the number of the button clicked by the user
(\code{1}, \code{2} or \code{3}).
\end{funcdesc}

\begin{funcdesc}{show_input}{prompt\, default}
Show a dialog box with a one-line prompt message and text field in
which the user can enter a string.  The second argument is the default
input string.  It returns the string value as edited by the user.
\end{funcdesc}

\begin{funcdesc}{show_file_selector}{message\, directory\, pattern\, default}
Show a dialog box in which the user can select a file.  It returns
the absolute filename selected by the user, or \code{None} if the user
presses Cancel.
\end{funcdesc}

\begin{funcdesc}{get_directory}{}
\funcline{get_pattern}{}
\funcline{get_filename}{}
These functions return the directory, pattern and filename (the tail
part only) selected by the user in the last \code{show_file_selector}
call.
\end{funcdesc}

\begin{funcdesc}{qdevice}{dev}
\funcline{unqdevice}{dev}
\funcline{isqueued}{dev}
\funcline{qtest}{}
\funcline{qread}{}
%\funcline{blkqread}{?}
\funcline{qreset}{}
\funcline{qenter}{dev\, val}
\funcline{get_mouse}{}
\funcline{tie}{button\, valuator1\, valuator2}
These functions are the FORMS interfaces to the corresponding GL
functions.  Use these if you want to handle some GL events yourself
when using \code{fl.do_events}.  When a GL event is detected that
FORMS cannot handle, \code{fl.do_forms()} returns the special value
\code{FL.EVENT} and you should call \code{fl.qread()} to read the
event from the queue.  Don't use the equivalent GL functions!
\end{funcdesc}

\begin{funcdesc}{color}{}
\funcline{mapcolor}{}
\funcline{getmcolor}{}
See the description in the FORMS documentation of \code{fl_color},
\code{fl_mapcolor} and \code{fl_getmcolor}.
\end{funcdesc}

\subsection{Form Objects}

Form objects (returned by \code{fl.make_form()} above) have the
following methods.  Each method corresponds to a C function whose name
is prefixed with \samp{fl_}; and whose first argument is a form
pointer; please refer to the official FORMS documentation for
descriptions.

All the \samp{add_{\rm \ldots}} functions return a Python object representing
the FORMS object.  Methods of FORMS objects are described below.  Most
kinds of FORMS object also have some methods specific to that kind;
these methods are listed here.

\begin{flushleft}
\renewcommand{\indexsubitem}{(form object method)}
\begin{funcdesc}{show_form}{placement\, bordertype\, name}
  Show the form.
\end{funcdesc}

\begin{funcdesc}{hide_form}{}
  Hide the form.
\end{funcdesc}

\begin{funcdesc}{redraw_form}{}
  Redraw the form.
\end{funcdesc}

\begin{funcdesc}{set_form_position}{x\, y}
Set the form's position.
\end{funcdesc}

\begin{funcdesc}{freeze_form}{}
Freeze the form.
\end{funcdesc}

\begin{funcdesc}{unfreeze_form}{}
  Unfreeze the form.
\end{funcdesc}

\begin{funcdesc}{activate_form}{}
  Activate the form.
\end{funcdesc}

\begin{funcdesc}{deactivate_form}{}
  Deactivate the form.
\end{funcdesc}

\begin{funcdesc}{bgn_group}{}
  Begin a new group of objects; return a group object.
\end{funcdesc}

\begin{funcdesc}{end_group}{}
  End the current group of objects.
\end{funcdesc}

\begin{funcdesc}{find_first}{}
  Find the first object in the form.
\end{funcdesc}

\begin{funcdesc}{find_last}{}
  Find the last object in the form.
\end{funcdesc}

%---

\begin{funcdesc}{add_box}{type\, x\, y\, w\, h\, name}
Add a box object to the form.
No extra methods.
\end{funcdesc}

\begin{funcdesc}{add_text}{type\, x\, y\, w\, h\, name}
Add a text object to the form.
No extra methods.
\end{funcdesc}

%\begin{funcdesc}{add_bitmap}{type\, x\, y\, w\, h\, name}
%Add a bitmap object to the form.
%\end{funcdesc}

\begin{funcdesc}{add_clock}{type\, x\, y\, w\, h\, name}
Add a clock object to the form. \\
Method:
\code{get_clock}.
\end{funcdesc}

%---

\begin{funcdesc}{add_button}{type\, x\, y\, w\, h\,  name}
Add a button object to the form. \\
Methods:
\code{get_button},
\code{set_button}.
\end{funcdesc}

\begin{funcdesc}{add_lightbutton}{type\, x\, y\, w\, h\, name}
Add a lightbutton object to the form. \\
Methods:
\code{get_button},
\code{set_button}.
\end{funcdesc}

\begin{funcdesc}{add_roundbutton}{type\, x\, y\, w\, h\, name}
Add a roundbutton object to the form. \\
Methods:
\code{get_button},
\code{set_button}.
\end{funcdesc}

%---

\begin{funcdesc}{add_slider}{type\, x\, y\, w\, h\, name}
Add a slider object to the form. \\
Methods:
\code{set_slider_value},
\code{get_slider_value},
\code{set_slider_bounds},
\code{get_slider_bounds},
\code{set_slider_return},
\code{set_slider_size},
\code{set_slider_precision},
\code{set_slider_step}.
\end{funcdesc}

\begin{funcdesc}{add_valslider}{type\, x\, y\, w\, h\, name}
Add a valslider object to the form. \\
Methods:
\code{set_slider_value},
\code{get_slider_value},
\code{set_slider_bounds},
\code{get_slider_bounds},
\code{set_slider_return},
\code{set_slider_size},
\code{set_slider_precision},
\code{set_slider_step}.
\end{funcdesc}

\begin{funcdesc}{add_dial}{type\, x\, y\, w\, h\, name}
Add a dial object to the form. \\
Methods:
\code{set_dial_value},
\code{get_dial_value},
\code{set_dial_bounds},
\code{get_dial_bounds}.
\end{funcdesc}

\begin{funcdesc}{add_positioner}{type\, x\, y\, w\, h\, name}
Add a positioner object to the form. \\
Methods:
\code{set_positioner_xvalue},
\code{set_positioner_yvalue},
\code{set_positioner_xbounds},
\code{set_positioner_ybounds},
\code{get_positioner_xvalue},
\code{get_positioner_yvalue},
\code{get_positioner_xbounds},
\code{get_positioner_ybounds}.
\end{funcdesc}

\begin{funcdesc}{add_counter}{type\, x\, y\, w\, h\, name}
Add a counter object to the form. \\
Methods:
\code{set_counter_value},
\code{get_counter_value},
\code{set_counter_bounds},
\code{set_counter_step},
\code{set_counter_precision},
\code{set_counter_return}.
\end{funcdesc}

%---

\begin{funcdesc}{add_input}{type\, x\, y\, w\, h\, name}
Add a input object to the form. \\
Methods:
\code{set_input},
\code{get_input},
\code{set_input_color},
\code{set_input_return}.
\end{funcdesc}

%---

\begin{funcdesc}{add_menu}{type\, x\, y\, w\, h\, name}
Add a menu object to the form. \\
Methods:
\code{set_menu},
\code{get_menu},
\code{addto_menu}.
\end{funcdesc}

\begin{funcdesc}{add_choice}{type\, x\, y\, w\, h\, name}
Add a choice object to the form. \\
Methods:
\code{set_choice},
\code{get_choice},
\code{clear_choice},
\code{addto_choice},
\code{replace_choice},
\code{delete_choice},
\code{get_choice_text},
\code{set_choice_fontsize},
\code{set_choice_fontstyle}.
\end{funcdesc}

\begin{funcdesc}{add_browser}{type\, x\, y\, w\, h\, name}
Add a browser object to the form. \\
Methods:
\code{set_browser_topline},
\code{clear_browser},
\code{add_browser_line},
\code{addto_browser},
\code{insert_browser_line},
\code{delete_browser_line},
\code{replace_browser_line},
\code{get_browser_line},
\code{load_browser},
\code{get_browser_maxline},
\code{select_browser_line},
\code{deselect_browser_line},
\code{deselect_browser},
\code{isselected_browser_line},
\code{get_browser},
\code{set_browser_fontsize},
\code{set_browser_fontstyle},
\code{set_browser_specialkey}.
\end{funcdesc}

%---

\begin{funcdesc}{add_timer}{type\, x\, y\, w\, h\, name}
Add a timer object to the form. \\
Methods:
\code{set_timer},
\code{get_timer}.
\end{funcdesc}
\end{flushleft}

Form objects have the following data attributes; see the FORMS
documentation:

\begin{tableiii}{|l|c|l|}{code}{Name}{Type}{Meaning}
  \lineiii{window}{int (read-only)}{GL window id}
  \lineiii{w}{float}{form width}
  \lineiii{h}{float}{form height}
  \lineiii{x}{float}{form x origin}
  \lineiii{y}{float}{form y origin}
  \lineiii{deactivated}{int}{nonzero if form is deactivated}
  \lineiii{visible}{int}{nonzero if form is visible}
  \lineiii{frozen}{int}{nonzero if form is frozen}
  \lineiii{doublebuf}{int}{nonzero if double buffering on}
\end{tableiii}

\subsection{FORMS Objects}

Besides methods specific to particular kinds of FORMS objects, all
FORMS objects also have the following methods:

\renewcommand{\indexsubitem}{(FORMS object method)}
\begin{funcdesc}{set_call_back}{function\, argument}
Set the object's callback function and argument.  When the object
needs interaction, the callback function will be called with two
arguments: the object, and the callback argument.  (FORMS objects
without a callback function are returned by \code{fl.do_forms()} or
\code{fl.check_forms()} when they need interaction.)  Call this method
without arguments to remove the callback function.
\end{funcdesc}

\begin{funcdesc}{delete_object}{}
  Delete the object.
\end{funcdesc}

\begin{funcdesc}{show_object}{}
  Show the object.
\end{funcdesc}

\begin{funcdesc}{hide_object}{}
  Hide the object.
\end{funcdesc}

\begin{funcdesc}{redraw_object}{}
  Redraw the object.
\end{funcdesc}

\begin{funcdesc}{freeze_object}{}
  Freeze the object.
\end{funcdesc}

\begin{funcdesc}{unfreeze_object}{}
  Unfreeze the object.
\end{funcdesc}

%\begin{funcdesc}{handle_object}{} XXX
%\end{funcdesc}

%\begin{funcdesc}{handle_object_direct}{} XXX
%\end{funcdesc}

FORMS objects have these data attributes; see the FORMS documentation:

\begin{tableiii}{|l|c|l|}{code}{Name}{Type}{Meaning}
  \lineiii{objclass}{int (read-only)}{object class}
  \lineiii{type}{int (read-only)}{object type}
  \lineiii{boxtype}{int}{box type}
  \lineiii{x}{float}{x origin}
  \lineiii{y}{float}{y origin}
  \lineiii{w}{float}{width}
  \lineiii{h}{float}{height}
  \lineiii{col1}{int}{primary color}
  \lineiii{col2}{int}{secondary color}
  \lineiii{align}{int}{alignment}
  \lineiii{lcol}{int}{label color}
  \lineiii{lsize}{float}{label font size}
  \lineiii{label}{string}{label string}
  \lineiii{lstyle}{int}{label style}
  \lineiii{pushed}{int (read-only)}{(see FORMS docs)}
  \lineiii{focus}{int (read-only)}{(see FORMS docs)}
  \lineiii{belowmouse}{int (read-only)}{(see FORMS docs)}
  \lineiii{frozen}{int (read-only)}{(see FORMS docs)}
  \lineiii{active}{int (read-only)}{(see FORMS docs)}
  \lineiii{input}{int (read-only)}{(see FORMS docs)}
  \lineiii{visible}{int (read-only)}{(see FORMS docs)}
  \lineiii{radio}{int (read-only)}{(see FORMS docs)}
  \lineiii{automatic}{int (read-only)}{(see FORMS docs)}
\end{tableiii}

\section{Standard Module \sectcode{FL}}
\nodename{FL (uppercase)}
\stmodindex{FL}

This module defines symbolic constants needed to use the built-in
module \code{fl} (see above); they are equivalent to those defined in
the C header file \file{<forms.h>} except that the name prefix
\samp{FL_} is omitted.  Read the module source for a complete list of
the defined names.  Suggested use:

\bcode\begin{verbatim}
import fl
from FL import *
\end{verbatim}\ecode

\section{Standard Module \sectcode{flp}}
\stmodindex{flp}

This module defines functions that can read form definitions created
by the `form designer' (\code{fdesign}) program that comes with the
FORMS library (see module \code{fl} above).

For now, see the file \file{flp.doc} in the Python library source
directory for a description.

XXX A complete description should be inserted here!

\section{Built-in Module \sectcode{fm}}
\bimodindex{fm}

This module provides access to the IRIS {\em Font Manager} library.
It is available only on Silicon Graphics machines.
See also: 4Sight User's Guide, Section 1, Chapter 5: Using the IRIS
Font Manager.

This is not yet a full interface to the IRIS Font Manager.
Among the unsupported features are: matrix operations; cache
operations; character operations (use string operations instead); some
details of font info; individual glyph metrics; and printer matching.

It supports the following operations:

\renewcommand{\indexsubitem}{(in module fm)}
\begin{funcdesc}{init}{}
Initialization function.
Calls \code{fminit()}.
It is normally not necessary to call this function, since it is called
automatically the first time the \code{fm} module is imported.
\end{funcdesc}

\begin{funcdesc}{findfont}{fontname}
Return a font handle object.
Calls \code{fmfindfont(\var{fontname})}.
\end{funcdesc}

\begin{funcdesc}{enumerate}{}
Returns a list of available font names.
This is an interface to \code{fmenumerate()}.
\end{funcdesc}

\begin{funcdesc}{prstr}{string}
Render a string using the current font (see the \code{setfont()} font
handle method below).
Calls \code{fmprstr(\var{string})}.
\end{funcdesc}

\begin{funcdesc}{setpath}{string}
Sets the font search path.
Calls \code{fmsetpath(string)}.
(XXX Does not work!?!)
\end{funcdesc}

\begin{funcdesc}{fontpath}{}
Returns the current font search path.
\end{funcdesc}

Font handle objects support the following operations:

\renewcommand{\indexsubitem}{(font handle method)}
\begin{funcdesc}{scalefont}{factor}
Returns a handle for a scaled version of this font.
Calls \code{fmscalefont(\var{fh}, \var{factor})}.
\end{funcdesc}

\begin{funcdesc}{setfont}{}
Makes this font the current font.
Note: the effect is undone silently when the font handle object is
deleted.
Calls \code{fmsetfont(\var{fh})}.
\end{funcdesc}

\begin{funcdesc}{getfontname}{}
Returns this font's name.
Calls \code{fmgetfontname(\var{fh})}.
\end{funcdesc}

\begin{funcdesc}{getcomment}{}
Returns the comment string associated with this font.
Raises an exception if there is none.
Calls \code{fmgetcomment(\var{fh})}.
\end{funcdesc}

\begin{funcdesc}{getfontinfo}{}
Returns a tuple giving some pertinent data about this font.
This is an interface to \code{fmgetfontinfo()}.
The returned tuple contains the following numbers:
{\tt(\var{printermatched}, \var{fixed_width}, \var{xorig}, \var{yorig},
\var{xsize}, \var{ysize}, \var{height}, \var{nglyphs})}.
\end{funcdesc}

\begin{funcdesc}{getstrwidth}{string}
Returns the width, in pixels, of the string when drawn in this font.
Calls \code{fmgetstrwidth(\var{fh}, \var{string})}.
\end{funcdesc}

\section{Built-in Module \sectcode{gl}}
\bimodindex{gl}

This module provides access to the Silicon Graphics
{\em Graphics Library}.
It is available only on Silicon Graphics machines.

\strong{Warning:}
Some illegal calls to the GL library cause the Python interpreter to dump
core.
In particular, the use of most GL calls is unsafe before the first
window is opened.

The module is too large to document here in its entirety, but the
following should help you to get started.
The parameter conventions for the C functions are translated to Python as
follows:

\begin{itemize}
\item
All (short, long, unsigned) int values are represented by Python
integers.
\item
All float and double values are represented by Python floating point
numbers.
In most cases, Python integers are also allowed.
\item
All arrays are represented by one-dimensional Python lists.
In most cases, tuples are also allowed.
\item
\begin{sloppypar}
All string and character arguments are represented by Python strings,
for instance,
\code{winopen('Hi There!')}
and
\code{rotate(900, 'z')}.
\end{sloppypar}
\item
All (short, long, unsigned) integer arguments or return values that are
only used to specify the length of an array argument are omitted.
For example, the C call

\bcode\begin{verbatim}
lmdef(deftype, index, np, props)
\end{verbatim}\ecode

is translated to Python as

\bcode\begin{verbatim}
lmdef(deftype, index, props)
\end{verbatim}\ecode

\item
Output arguments are omitted from the argument list; they are
transmitted as function return values instead.
If more than one value must be returned, the return value is a tuple.
If the C function has both a regular return value (that is not omitted
because of the previous rule) and an output argument, the return value
comes first in the tuple.
Examples: the C call

\bcode\begin{verbatim}
getmcolor(i, &red, &green, &blue)
\end{verbatim}\ecode

is translated to Python as

\bcode\begin{verbatim}
red, green, blue = getmcolor(i)
\end{verbatim}\ecode

\end{itemize}

The following functions are non-standard or have special argument
conventions:

\renewcommand{\indexsubitem}{(in module gl)}
\begin{funcdesc}{varray}{argument}
%JHXXX the argument-argument added
Equivalent to but faster than a number of
\code{v3d()}
calls.
The \var{argument} is a list (or tuple) of points.
Each point must be a tuple of coordinates
\code{(\var{x}, \var{y}, \var{z})} or \code{(\var{x}, \var{y})}.
The points may be 2- or 3-dimensional but must all have the
same dimension.
Float and int values may be mixed however.
The points are always converted to 3D double precision points
by assuming \code{\var{z} = 0.0} if necessary (as indicated in the man page),
and for each point
\code{v3d()}
is called.
\end{funcdesc}

\begin{funcdesc}{nvarray}{}
Equivalent to but faster than a number of
\code{n3f}
and
\code{v3f}
calls.
The argument is an array (list or tuple) of pairs of normals and points.
Each pair is a tuple of a point and a normal for that point.
Each point or normal must be a tuple of coordinates
\code{(\var{x}, \var{y}, \var{z})}.
Three coordinates must be given.
Float and int values may be mixed.
For each pair,
\code{n3f()}
is called for the normal, and then
\code{v3f()}
is called for the point.
\end{funcdesc}

\begin{funcdesc}{vnarray}{}
Similar to 
\code{nvarray()}
but the pairs have the point first and the normal second.
\end{funcdesc}

\begin{funcdesc}{nurbssurface}{s_k\, t_k\, ctl\, s_ord\, t_ord\, type}
% XXX s_k[], t_k[], ctl[][]
%\itembreak
Defines a nurbs surface.
The dimensions of
\code{\var{ctl}[][]}
are computed as follows:
\code{[len(\var{s_k}) - \var{s_ord}]},
\code{[len(\var{t_k}) - \var{t_ord}]}.
\end{funcdesc}

\begin{funcdesc}{nurbscurve}{knots\, ctlpoints\, order\, type}
Defines a nurbs curve.
The length of ctlpoints is
\code{len(\var{knots}) - \var{order}}.
\end{funcdesc}

\begin{funcdesc}{pwlcurve}{points\, type}
Defines a piecewise-linear curve.
\var{points}
is a list of points.
\var{type}
must be
\code{N_ST}.
\end{funcdesc}

\begin{funcdesc}{pick}{n}
\funcline{select}{n}
The only argument to these functions specifies the desired size of the
pick or select buffer.
\end{funcdesc}

\begin{funcdesc}{endpick}{}
\funcline{endselect}{}
These functions have no arguments.
They return a list of integers representing the used part of the
pick/select buffer.
No method is provided to detect buffer overrun.
\end{funcdesc}

Here is a tiny but complete example GL program in Python:

\bcode\begin{verbatim}
import gl, GL, time

def main():
    gl.foreground()
    gl.prefposition(500, 900, 500, 900)
    w = gl.winopen('CrissCross')
    gl.ortho2(0.0, 400.0, 0.0, 400.0)
    gl.color(GL.WHITE)
    gl.clear()
    gl.color(GL.RED)
    gl.bgnline()
    gl.v2f(0.0, 0.0)
    gl.v2f(400.0, 400.0)
    gl.endline()
    gl.bgnline()
    gl.v2f(400.0, 0.0)
    gl.v2f(0.0, 400.0)
    gl.endline()
    time.sleep(5)

main()
\end{verbatim}\ecode

\section{Standard Modules \sectcode{GL} and \sectcode{DEVICE}}
\nodename{GL and DEVICE}
\stmodindex{GL}
\stmodindex{DEVICE}

These modules define the constants used by the Silicon Graphics
{\em Graphics Library}
that C programmers find in the header files
\file{<gl/gl.h>}
and
\file{<gl/device.h>}.
Read the module source files for details.

\section{Built-in Module \sectcode{imgfile}}
\bimodindex{imgfile}

The imgfile module allows python programs to access SGI imglib image
files (also known as \file{.rgb} files).  The module is far from
complete, but is provided anyway since the functionality that there is
is enough in some cases.  Currently, colormap files are not supported.

The module defines the following variables and functions:

\renewcommand{\indexsubitem}{(in module imgfile)}
\begin{excdesc}{error}
This exception is raised on all errors, such as unsupported file type, etc.
\end{excdesc}

\begin{funcdesc}{getsizes}{file}
This function returns a tuple \code{(\var{x}, \var{y}, \var{z})} where
\var{x} and \var{y} are the size of the image in pixels and
\var{z} is the number of
bytes per pixel. Only 3 byte RGB pixels and 1 byte greyscale pixels
are currently supported.
\end{funcdesc}

\begin{funcdesc}{read}{file}
This function reads and decodes the image on the specified file, and
returns it as a python string. The string has either 1 byte greyscale
pixels or 4 byte RGBA pixels. The bottom left pixel is the first in
the string. This format is suitable to pass to \code{gl.lrectwrite},
for instance.
\end{funcdesc}

\begin{funcdesc}{readscaled}{file\, x\, y\, filter\optional{\, blur}}
This function is identical to read but it returns an image that is
scaled to the given \var{x} and \var{y} sizes. If the \var{filter} and
\var{blur} parameters are omitted scaling is done by
simply dropping or duplicating pixels, so the result will be less than
perfect, especially for computer-generated images.

Alternatively, you can specify a filter to use to smoothen the image
after scaling. The filter forms supported are \code{'impulse'},
\code{'box'}, \code{'triangle'}, \code{'quadratic'} and
\code{'gaussian'}. If a filter is specified \var{blur} is an optional
parameter specifying the blurriness of the filter. It defaults to \code{1.0}.

\code{readscaled} makes no
attempt to keep the aspect ratio correct, so that is the users'
responsibility.
\end{funcdesc}

\begin{funcdesc}{ttob}{flag}
This function sets a global flag which defines whether the scan lines
of the image are read or written from bottom to top (flag is zero,
compatible with SGI GL) or from top to bottom(flag is one,
compatible with X).  The default is zero.
\end{funcdesc}

\begin{funcdesc}{write}{file\, data\, x\, y\, z}
This function writes the RGB or greyscale data in \var{data} to image
file \var{file}. \var{x} and \var{y} give the size of the image,
\var{z} is 1 for 1 byte greyscale images or 3 for RGB images (which are
stored as 4 byte values of which only the lower three bytes are used).
These are the formats returned by \code{gl.lrectread}.
\end{funcdesc}

%\section{Standard Module \sectcode{panel}}
\stmodindex{panel}

\strong{Please note:} The FORMS library, to which the \code{fl} module described
above interfaces, is a simpler and more accessible user interface
library for use with GL than the Panel Module (besides also being by a
Dutch author).

This module should be used instead of the built-in module
\code{pnl}
to interface with the
{\em Panel Library}.

The module is too large to document here in its entirety.
One interesting function:

\renewcommand{\indexsubitem}{(in module panel)}
\begin{funcdesc}{defpanellist}{filename}
Parses a panel description file containing S-expressions written by the
{\em Panel Editor}
that accompanies the Panel Library and creates the described panels.
It returns a list of panel objects.
\end{funcdesc}

\strong{Warning:}
the Python interpreter will dump core if you don't create a GL window
before calling
\code{panel.mkpanel()}
or
\code{panel.defpanellist()}.

\section{Standard Module \sectcode{panelparser}}
\stmodindex{panelparser}

This module defines a self-contained parser for S-expressions as output
by the Panel Editor (which is written in Scheme so it can't help writing
S-expressions).
The relevant function is
\code{panelparser.parse_file(\var{file})}
which has a file object (not a filename!) as argument and returns a list
of parsed S-expressions.
Each S-expression is converted into a Python list, with atoms converted
to Python strings and sub-expressions (recursively) to Python lists.
For more details, read the module file.
% XXXXJH should be funcdesc, I think

\section{Built-in Module \sectcode{pnl}}
\bimodindex{pnl}

This module provides access to the
{\em Panel Library}
built by NASA Ames (to get it, send e-mail to
{\tt panel-request@nas.nasa.gov}).
All access to it should be done through the standard module
\code{panel},
which transparantly exports most functions from
\code{pnl}
but redefines
\code{pnl.dopanel()}.

\strong{Warning:}
the Python interpreter will dump core if you don't create a GL window
before calling
\code{pnl.mkpanel()}.

The module is too large to document here in its entirety.


\chapter{SunOS Specific Services}

The modules described in this chapter provide interfaces to features
that are unique to the SunOS operating system (versions 4 and 5; the
latter is also known as Solaris version 2).

\section{Built-in Module \sectcode{sunaudiodev}}
\bimodindex{sunaudiodev}

This module allows you to access the sun audio interface. The sun
audio hardware is capable of recording and playing back audio data
in U-LAW format with a sample rate of 8K per second. A full
description can be gotten with \samp{man audio}.

The module defines the following variables and functions:

\renewcommand{\indexsubitem}{(in module sunaudiodev)}
\begin{excdesc}{error}
This exception is raised on all errors. The argument is a string
describing what went wrong.
\end{excdesc}

\begin{funcdesc}{open}{mode}
This function opens the audio device and returns a sun audio device
object. This object can then be used to do I/O on. The \var{mode} parameter
is one of \code{'r'} for record-only access, \code{'w'} for play-only
access, \code{'rw'} for both and \code{'control'} for access to the
control device. Since only one process is allowed to have the recorder
or player open at the same time it is a good idea to open the device
only for the activity needed. See the audio manpage for details.
\end{funcdesc}

\subsection{Audio Device Objects}

The audio device objects are returned by \code{open} define the
following methods (except \code{control} objects which only provide
getinfo, setinfo and drain):

\renewcommand{\indexsubitem}{(audio device method)}

\begin{funcdesc}{close}{}
This method explicitly closes the device. It is useful in situations
where deleting the object does not immediately close it since there
are other references to it. A closed device should not be used again.
\end{funcdesc}

\begin{funcdesc}{drain}{}
This method waits until all pending output is processed and then returns.
Calling this method is often not necessary: destroying the object will
automatically close the audio device and this will do an implicit drain.
\end{funcdesc}

\begin{funcdesc}{flush}{}
This method discards all pending output. It can be used avoid the
slow response to a user's stop request (due to buffering of up to one
second of sound).
\end{funcdesc}

\begin{funcdesc}{getinfo}{}
This method retrieves status information like input and output volume,
etc. and returns it in the form of
an audio status object. This object has no methods but it contains a
number of attributes describing the current device status. The names
and meanings of the attributes are described in
\file{/usr/include/sun/audioio.h} and in the audio man page. Member names
are slightly different from their C counterparts: a status object is
only a single structure. Members of the \code{play} substructure have
\samp{o_} prepended to their name and members of the \code{record}
structure have \samp{i_}. So, the C member \code{play.sample_rate} is
accessed as \code{o_sample_rate}, \code{record.gain} as \code{i_gain}
and \code{monitor_gain} plainly as \code{monitor_gain}.
\end{funcdesc}

\begin{funcdesc}{ibufcount}{}
This method returns the number of samples that are buffered on the
recording side, i.e.
the program will not block on a \code{read} call of so many samples.
\end{funcdesc}

\begin{funcdesc}{obufcount}{}
This method returns the number of samples buffered on the playback
side. Unfortunately, this number cannot be used to determine a number
of samples that can be written without blocking since the kernel
output queue length seems to be variable.
\end{funcdesc}

\begin{funcdesc}{read}{size}
This method reads \var{size} samples from the audio input and returns
them as a python string. The function blocks until enough data is available.
\end{funcdesc}

\begin{funcdesc}{setinfo}{status}
This method sets the audio device status parameters. The \var{status}
parameter is an device status object as returned by \code{getinfo} and
possibly modified by the program.
\end{funcdesc}

\begin{funcdesc}{write}{samples}
Write is passed a python string containing audio samples to be played.
If there is enough buffer space free it will immediately return,
otherwise it will block.
\end{funcdesc}

There is a companion module, \code{SUNAUDIODEV}, which defines useful
symbolic constants like \code{MIN_GAIN}, \code{MAX_GAIN},
\code{SPEAKER}, etc. The names of
the constants are the same names as used in the C include file
\file{<sun/audioio.h>}, with the leading string \samp{AUDIO_} stripped.

Useability of the control device is limited at the moment, since there
is no way to use the ``wait for something to happen'' feature the
device provides.
			% SUNOS ONLY

\documentstyle[twoside,11pt,myformat]{report}

% NOTE: this file controls which chapters/sections of the library
% manual are actually printed.  It is easy to customize your manual
% by commenting out sections that you're not interested in.

\title{Python Library Reference}

\author{
	Guido van Rossum \\
	Corporation for National Research Initiatives (CNRI) \\
	1895 Preston White Drive, Reston, Va 20191, USA \\
	E-mail: {\tt guido@CNRI.Reston.Va.US}, {\tt guido@python.org}
}

\date{October 25, 1996 \\ Release 1.4} % XXX update before release!


\makeindex			% tell \index to actually write the .idx file


\begin{document}

\pagenumbering{roman}

\maketitle

Copyright \copyright{} 1991-1995 by Stichting Mathematisch Centrum,
Amsterdam, The Netherlands.

\begin{center}
All Rights Reserved
\end{center}

Permission to use, copy, modify, and distribute this software and its
documentation for any purpose and without fee is hereby granted,
provided that the above copyright notice appear in all copies and that
both that copyright notice and this permission notice appear in
supporting documentation, and that the names of Stichting Mathematisch
Centrum or CWI or Corporation for National Research Initiatives or
CNRI not be used in advertising or publicity pertaining to
distribution of the software without specific, written prior
permission.

While CWI is the initial source for this software, a modified version
is made available by the Corporation for National Research Initiatives
(CNRI) at the Internet address ftp://ftp.python.org.

STICHTING MATHEMATISCH CENTRUM AND CNRI DISCLAIM ALL WARRANTIES WITH
REGARD TO THIS SOFTWARE, INCLUDING ALL IMPLIED WARRANTIES OF
MERCHANTABILITY AND FITNESS, IN NO EVENT SHALL STICHTING MATHEMATISCH
CENTRUM OR CNRI BE LIABLE FOR ANY SPECIAL, INDIRECT OR CONSEQUENTIAL
DAMAGES OR ANY DAMAGES WHATSOEVER RESULTING FROM LOSS OF USE, DATA OR
PROFITS, WHETHER IN AN ACTION OF CONTRACT, NEGLIGENCE OR OTHER
TORTIOUS ACTION, ARISING OUT OF OR IN CONNECTION WITH THE USE OR
PERFORMANCE OF THIS SOFTWARE.


\begin{abstract}

\noindent
Python is an extensible, interpreted, object-oriented programming
language.  It supports a wide range of applications, from simple text
processing scripts to interactive WWW browsers.

While the {\em Python Reference Manual} describes the exact syntax and
semantics of the language, it does not describe the standard library
that is distributed with the language, and which greatly enhances its
immediate usability.  This library contains built-in modules (written
in C) that provide access to system functionality such as file I/O
that would otherwise be inaccessible to Python programmers, as well as
modules written in Python that provide standardized solutions for many
problems that occur in everyday programming.  Some of these modules
are explicitly designed to encourage and enhance the portability of
Python programs.

This library reference manual documents Python's standard library, as
well as many optional library modules (which may or may not be
available, depending on whether the underlying platform supports them
and on the configuration choices made at compile time).  It also
documents the standard types of the language and its built-in
functions and exceptions, many of which are not or incompletely
documented in the Reference Manual.

This manual assumes basic knowledge about the Python language.  For an
informal introduction to Python, see the {\em Python Tutorial}; the
Python Reference Manual remains the highest authority on syntactic and
semantic questions.  Finally, the manual entitled {\em Extending and
Embedding the Python Interpreter} describes how to add new extensions
to Python and how to embed it in other applications.

\end{abstract}

\pagebreak

{
\parskip = 0mm
\tableofcontents
}

\pagebreak

\pagenumbering{arabic}

				% Chapter title:

\chapter{Introduction}

The ``Python library'' contains several different kinds of components.

It contains data types that would normally be considered part of the
``core'' of a language, such as numbers and lists.  For these types,
the Python language core defines the form of literals and places some
constraints on their semantics, but does not fully define the
semantics.  (On the other hand, the language core does define
syntactic properties like the spelling and priorities of operators.)

The library also contains built-in functions and exceptions ---
objects that can be used by all Python code without the need of an
\code{import} statement.  Some of these are defined by the core
language, but many are not essential for the core semantics and are
only described here.

The bulk of the library, however, consists of a collection of modules.
There are many ways to dissect this collection.  Some modules are
written in C and built in to the Python interpreter; others are
written in Python and imported in source form.  Some modules provide
interfaces that are highly specific to Python, like printing a stack
trace; some provide interfaces that are specific to particular
operating systems, like socket I/O; others provide interfaces that are
specific to a particular application domain, like the World-Wide Web.
Some modules are avaiable in all versions and ports of Python; others
are only available when the underlying system supports or requires
them; yet others are available only when a particular configuration
option was chosen at the time when Python was compiled and installed.

This manual is organized ``from the inside out'': it first describes
the built-in data types, then the built-in functions and exceptions,
and finally the modules, grouped in chapters of related modules.  The
ordering of the chapters as well as the ordering of the modules within
each chapter is roughly from most relevant to least important.

This means that if you start reading this manual from the start, and
skip to the next chapter when you get bored, you will get a reasonable
overview of the available modules and application areas that are
supported by the Python library.  Of course, you don't \emph{have} to
read it like a novel --- you can also browse the table of contents (in
front of the manual), or look for a specific function, module or term
in the index (in the back).  And finally, if you enjoy learning about
random subjects, you choose a random page number (see module
\code{rand}) and read a section or two.

Let the show begin!
		% Introduction

\chapter{Built-in Types, Exceptions and Functions}

\nodename{Built-in Objects}

Names for built-in exceptions and functions are found in a separate
symbol table.  This table is searched last when the interpreter looks
up the meaning of a name, so local and global
user-defined names can override built-in names.  Built-in types are
described together here for easy reference.%
\footnote{Most descriptions sorely lack explanations of the exceptions
	that may be raised --- this will be fixed in a future version of
	this manual.}
\indexii{built-in}{types}
\indexii{built-in}{exceptions}
\indexii{built-in}{functions}
\index{symbol table}
\bifuncindex{type}

The tables in this chapter document the priorities of operators by
listing them in order of ascending priority (within a table) and
grouping operators that have the same priority in the same box.
Binary operators of the same priority group from left to right.
(Unary operators group from right to left, but there you have no real
choice.)  See Chapter 5 of the Python Reference Manual for the
complete picture on operator priorities.
			% Built-in Types, Exceptions and Functions
\section{Built-in Types}

The following sections describe the standard types that are built into
the interpreter.  These are the numeric types, sequence types, and
several others, including types themselves.  There is no explicit
Boolean type; use integers instead.
\indexii{built-in}{types}
\indexii{Boolean}{type}

Some operations are supported by several object types; in particular,
all objects can be compared, tested for truth value, and converted to
a string (with the \code{`{\rm \ldots}`} notation).  The latter conversion is
implicitly used when an object is written by the \code{print} statement.
\stindex{print}

\subsection{Truth Value Testing}

Any object can be tested for truth value, for use in an \code{if} or
\code{while} condition or as operand of the Boolean operations below.
The following values are considered false:
\stindex{if}
\stindex{while}
\indexii{truth}{value}
\indexii{Boolean}{operations}
\index{false}

\begin{itemize}
\renewcommand{\indexsubitem}{(Built-in object)}

\item	\code{None}
	\ttindex{None}

\item	zero of any numeric type, e.g., \code{0}, \code{0L}, \code{0.0}.

\item	any empty sequence, e.g., \code{''}, \code{()}, \code{[]}.

\item	any empty mapping, e.g., \code{\{\}}.

\item	instances of user-defined classes, if the class defines a
	\code{__nonzero__()} or \code{__len__()} method, when that
	method returns zero.

\end{itemize}

All other values are considered true --- so objects of many types are
always true.
\index{true}

Operations and built-in functions that have a Boolean result always
return \code{0} for false and \code{1} for true, unless otherwise
stated.  (Important exception: the Boolean operations \samp{or} and
\samp{and} always return one of their operands.)

\subsection{Boolean Operations}

These are the Boolean operations, ordered by ascending priority:
\indexii{Boolean}{operations}

\begin{tableiii}{|c|l|c|}{code}{Operation}{Result}{Notes}
  \lineiii{\var{x} or \var{y}}{if \var{x} is false, then \var{y}, else \var{x}}{(1)}
  \hline
  \lineiii{\var{x} and \var{y}}{if \var{x} is false, then \var{x}, else \var{y}}{(1)}
  \hline
  \lineiii{not \var{x}}{if \var{x} is false, then \code{1}, else \code{0}}{(2)}
\end{tableiii}
\opindex{and}
\opindex{or}
\opindex{not}

\noindent
Notes:

\begin{description}

\item[(1)]
These only evaluate their second argument if needed for their outcome.

\item[(2)]
\samp{not} has a lower priority than non-Boolean operators, so e.g.
\code{not a == b} is interpreted as \code{not(a == b)}, and
\code{a == not b} is a syntax error.

\end{description}

\subsection{Comparisons}

Comparison operations are supported by all objects.  They all have the
same priority (which is higher than that of the Boolean operations).
Comparisons can be chained arbitrarily, e.g. \code{x < y <= z} is
equivalent to \code{x < y and y <= z}, except that \code{y} is
evaluated only once (but in both cases \code{z} is not evaluated at
all when \code{x < y} is found to be false).
\indexii{chaining}{comparisons}

This table summarizes the comparison operations:

\begin{tableiii}{|c|l|c|}{code}{Operation}{Meaning}{Notes}
  \lineiii{<}{strictly less than}{}
  \lineiii{<=}{less than or equal}{}
  \lineiii{>}{strictly greater than}{}
  \lineiii{>=}{greater than or equal}{}
  \lineiii{==}{equal}{}
  \lineiii{<>}{not equal}{(1)}
  \lineiii{!=}{not equal}{(1)}
  \lineiii{is}{object identity}{}
  \lineiii{is not}{negated object identity}{}
\end{tableiii}
\indexii{operator}{comparison}
\opindex{==} % XXX *All* others have funny characters < ! >
\opindex{is}
\opindex{is not}

\noindent
Notes:

\begin{description}

\item[(1)]
\code{<>} and \code{!=} are alternate spellings for the same operator.
(I couldn't choose between \ABC{} and \C{}! :-)
\indexii{\ABC{}}{language}
\indexii{\C{}}{language}

\end{description}

Objects of different types, except different numeric types, never
compare equal; such objects are ordered consistently but arbitrarily
(so that sorting a heterogeneous array yields a consistent result).
Furthermore, some types (e.g., windows) support only a degenerate
notion of comparison where any two objects of that type are unequal.
Again, such objects are ordered arbitrarily but consistently.
\indexii{types}{numeric}
\indexii{objects}{comparing}

(Implementation note: objects of different types except numbers are
ordered by their type names; objects of the same types that don't
support proper comparison are ordered by their address.)

Two more operations with the same syntactic priority, \code{in} and
\code{not in}, are supported only by sequence types (below).
\opindex{in}
\opindex{not in}

\subsection{Numeric Types}

There are three numeric types: \dfn{plain integers}, \dfn{long integers}, and
\dfn{floating point numbers}.  Plain integers (also just called \dfn{integers})
are implemented using \code{long} in \C{}, which gives them at least 32
bits of precision.  Long integers have unlimited precision.  Floating
point numbers are implemented using \code{double} in \C{}.  All bets on
their precision are off unless you happen to know the machine you are
working with.
\indexii{numeric}{types}
\indexii{integer}{types}
\indexii{integer}{type}
\indexiii{long}{integer}{type}
\indexii{floating point}{type}
\indexii{\C{}}{language}

Numbers are created by numeric literals or as the result of built-in
functions and operators.  Unadorned integer literals (including hex
and octal numbers) yield plain integers.  Integer literals with an \samp{L}
or \samp{l} suffix yield long integers
(\samp{L} is preferred because \code{1l} looks too much like eleven!).
Numeric literals containing a decimal point or an exponent sign yield
floating point numbers.
\indexii{numeric}{literals}
\indexii{integer}{literals}
\indexiii{long}{integer}{literals}
\indexii{floating point}{literals}
\indexii{hexadecimal}{literals}
\indexii{octal}{literals}

Python fully supports mixed arithmetic: when a binary arithmetic
operator has operands of different numeric types, the operand with the
``smaller'' type is converted to that of the other, where plain
integer is smaller than long integer is smaller than floating point.
Comparisons between numbers of mixed type use the same rule.%
\footnote{As a consequence, the list \code{[1, 2]} is considered equal
	to \code{[1.0, 2.0]}, and similar for tuples.}
The functions \code{int()}, \code{long()} and \code{float()} can be used
to coerce numbers to a specific type.
\index{arithmetic}
\bifuncindex{int}
\bifuncindex{long}
\bifuncindex{float}

All numeric types support the following operations, sorted by
ascending priority (operations in the same box have the same
priority; all numeric operations have a higher priority than
comparison operations):

\begin{tableiii}{|c|l|c|}{code}{Operation}{Result}{Notes}
  \lineiii{\var{x} + \var{y}}{sum of \var{x} and \var{y}}{}
  \lineiii{\var{x} - \var{y}}{difference of \var{x} and \var{y}}{}
  \hline
  \lineiii{\var{x} * \var{y}}{product of \var{x} and \var{y}}{}
  \lineiii{\var{x} / \var{y}}{quotient of \var{x} and \var{y}}{(1)}
  \lineiii{\var{x} \%{} \var{y}}{remainder of \code{\var{x} / \var{y}}}{}
  \hline
  \lineiii{-\var{x}}{\var{x} negated}{}
  \lineiii{+\var{x}}{\var{x} unchanged}{}
  \hline
  \lineiii{abs(\var{x})}{absolute value of \var{x}}{}
  \lineiii{int(\var{x})}{\var{x} converted to integer}{(2)}
  \lineiii{long(\var{x})}{\var{x} converted to long integer}{(2)}
  \lineiii{float(\var{x})}{\var{x} converted to floating point}{}
  \lineiii{divmod(\var{x}, \var{y})}{the pair \code{(\var{x} / \var{y}, \var{x} \%{} \var{y})}}{(3)}
  \lineiii{pow(\var{x}, \var{y})}{\var{x} to the power \var{y}}{}
\end{tableiii}
\indexiii{operations on}{numeric}{types}

\noindent
Notes:
\begin{description}

\item[(1)]
For (plain or long) integer division, the result is an integer.
The result is always rounded towards minus infinity: 1/2 is 0, 
(-1)/2 is -1, 1/(-2) is -1, and (-1)/(-2) is 0.
\indexii{integer}{division}
\indexiii{long}{integer}{division}

\item[(2)]
Conversion from floating point to (long or plain) integer may round or
truncate as in \C{}; see functions \code{floor()} and \code{ceil()} in
module \code{math} for well-defined conversions.
\bifuncindex{floor}
\bifuncindex{ceil}
\indexii{numeric}{conversions}
\stmodindex{math}
\indexii{\C{}}{language}

\item[(3)]
See the section on built-in functions for an exact definition.

\end{description}
% XXXJH exceptions: overflow (when? what operations?) zerodivision

\subsubsection{Bit-string Operations on Integer Types}
\nodename{Bit-string Operations}

Plain and long integer types support additional operations that make
sense only for bit-strings.  Negative numbers are treated as their 2's
complement value (for long integers, this assumes a sufficiently large
number of bits that no overflow occurs during the operation).

The priorities of the binary bit-wise operations are all lower than
the numeric operations and higher than the comparisons; the unary
operation \samp{~} has the same priority as the other unary numeric
operations (\samp{+} and \samp{-}).

This table lists the bit-string operations sorted in ascending
priority (operations in the same box have the same priority):

\begin{tableiii}{|c|l|c|}{code}{Operation}{Result}{Notes}
  \lineiii{\var{x} | \var{y}}{bitwise \dfn{or} of \var{x} and \var{y}}{}
  \hline
  \lineiii{\var{x} \^{} \var{y}}{bitwise \dfn{exclusive or} of \var{x} and \var{y}}{}
  \hline
  \lineiii{\var{x} \&{} \var{y}}{bitwise \dfn{and} of \var{x} and \var{y}}{}
  \hline
  \lineiii{\var{x} << \var{n}}{\var{x} shifted left by \var{n} bits}{(1), (2)}
  \lineiii{\var{x} >> \var{n}}{\var{x} shifted right by \var{n} bits}{(1), (3)}
  \hline
  \hline
  \lineiii{\~\var{x}}{the bits of \var{x} inverted}{}
\end{tableiii}
\indexiii{operations on}{integer}{types}
\indexii{bit-string}{operations}
\indexii{shifting}{operations}
\indexii{masking}{operations}

\noindent
Notes:
\begin{description}
\item[(1)] Negative shift counts are illegal.
\item[(2)] A left shift by \var{n} bits is equivalent to
multiplication by \code{pow(2, \var{n})} without overflow check.
\item[(3)] A right shift by \var{n} bits is equivalent to
division by \code{pow(2, \var{n})} without overflow check.
\end{description}

\subsection{Sequence Types}

There are three sequence types: strings, lists and tuples.

Strings literals are written in single or double quotes:
\code{'xyzzy'}, \code{"frobozz"}.  See Chapter 2 of the Python
Reference Manual for more about string literals.  Lists are
constructed with square brackets, separating items with commas:
\code{[a, b, c]}.  Tuples are constructed by the comma operator (not
within square brackets), with or without enclosing parentheses, but an
empty tuple must have the enclosing parentheses, e.g.,
\code{a, b, c} or \code{()}.  A single item tuple must have a trailing
comma, e.g., \code{(d,)}.
\indexii{sequence}{types}
\indexii{string}{type}
\indexii{tuple}{type}
\indexii{list}{type}

Sequence types support the following operations.  The \samp{in} and
\samp{not\,in} operations have the same priorities as the comparison
operations.  The \samp{+} and \samp{*} operations have the same
priority as the corresponding numeric operations.\footnote{They must
have since the parser can't tell the type of the operands.}

This table lists the sequence operations sorted in ascending priority
(operations in the same box have the same priority).  In the table,
\var{s} and \var{t} are sequences of the same type; \var{n}, \var{i}
and \var{j} are integers:

\begin{tableiii}{|c|l|c|}{code}{Operation}{Result}{Notes}
  \lineiii{\var{x} in \var{s}}{\code{1} if an item of \var{s} is equal to \var{x}, else \code{0}}{}
  \lineiii{\var{x} not in \var{s}}{\code{0} if an item of \var{s} is
equal to \var{x}, else \code{1}}{}
  \hline
  \lineiii{\var{s} + \var{t}}{the concatenation of \var{s} and \var{t}}{}
  \hline
  \lineiii{\var{s} * \var{n}{\rm ,} \var{n} * \var{s}}{\var{n} copies of \var{s} concatenated}{}
  \hline
  \lineiii{\var{s}[\var{i}]}{\var{i}'th item of \var{s}, origin 0}{(1)}
  \lineiii{\var{s}[\var{i}:\var{j}]}{slice of \var{s} from \var{i} to \var{j}}{(1), (2)}
  \hline
  \lineiii{len(\var{s})}{length of \var{s}}{}
  \lineiii{min(\var{s})}{smallest item of \var{s}}{}
  \lineiii{max(\var{s})}{largest item of \var{s}}{}
\end{tableiii}
\indexiii{operations on}{sequence}{types}
\bifuncindex{len}
\bifuncindex{min}
\bifuncindex{max}
\indexii{concatenation}{operation}
\indexii{repetition}{operation}
\indexii{subscript}{operation}
\indexii{slice}{operation}
\opindex{in}
\opindex{not in}

\noindent
Notes:

\begin{description}
  
\item[(1)] If \var{i} or \var{j} is negative, the index is relative to
  the end of the string, i.e., \code{len(\var{s}) + \var{i}} or
  \code{len(\var{s}) + \var{j}} is substituted.  But note that \code{-0} is
  still \code{0}.
  
\item[(2)] The slice of \var{s} from \var{i} to \var{j} is defined as
  the sequence of items with index \var{k} such that \code{\var{i} <=
  \var{k} < \var{j}}.  If \var{i} or \var{j} is greater than
  \code{len(\var{s})}, use \code{len(\var{s})}.  If \var{i} is omitted,
  use \code{0}.  If \var{j} is omitted, use \code{len(\var{s})}.  If
  \var{i} is greater than or equal to \var{j}, the slice is empty.

\end{description}

\subsubsection{More String Operations}

String objects have one unique built-in operation: the \code{\%}
operator (modulo) with a string left argument interprets this string
as a C sprintf format string to be applied to the right argument, and
returns the string resulting from this formatting operation.

The right argument should be a tuple with one item for each argument
required by the format string; if the string requires a single
argument, the right argument may also be a single non-tuple object.%
\footnote{A tuple object in this case should be a singleton.}
The following format characters are understood:
\%, c, s, i, d, u, o, x, X, e, E, f, g, G.
Width and precision may be a * to specify that an integer argument
specifies the actual width or precision.  The flag characters -, +,
blank, \# and 0 are understood.  The size specifiers h, l or L may be
present but are ignored.  The \code{\%s} conversion takes any Python
object and converts it to a string using \code{str()} before
formatting it.  The ANSI features \code{\%p} and \code{\%n}
are not supported.  Since Python strings have an explicit length,
\code{\%s} conversions don't assume that \code{'\e0'} is the end of
the string.

For safety reasons, floating point precisions are clipped to 50;
\code{\%f} conversions for numbers whose absolute value is over 1e25
are replaced by \code{\%g} conversions.%
\footnote{These numbers are fairly arbitrary.  They are intended to
avoid printing endless strings of meaningless digits without hampering
correct use and without having to know the exact precision of floating
point values on a particular machine.}
All other errors raise exceptions.

If the right argument is a dictionary (or any kind of mapping), then
the formats in the string must have a parenthesized key into that
dictionary inserted immediately after the \code{\%} character, and
each format formats the corresponding entry from the mapping.  E.g.
\begin{verbatim}
    >>> count = 2
    >>> language = 'Python'
    >>> print '%(language)s has %(count)03d quote types.' % vars()
    Python has 002 quote types.
    >>> 
\end{verbatim}
In this case no * specifiers may occur in a format (since they
require a sequential parameter list).

Additional string operations are defined in standard module
\code{string} and in built-in module \code{regex}.
\index{string}
\index{regex}

\subsubsection{Mutable Sequence Types}

List objects support additional operations that allow in-place
modification of the object.
These operations would be supported by other mutable sequence types
(when added to the language) as well.
Strings and tuples are immutable sequence types and such objects cannot
be modified once created.
The following operations are defined on mutable sequence types (where
\var{x} is an arbitrary object):
\indexiii{mutable}{sequence}{types}
\indexii{list}{type}

\begin{tableiii}{|c|l|c|}{code}{Operation}{Result}{Notes}
  \lineiii{\var{s}[\var{i}] = \var{x}}
	{item \var{i} of \var{s} is replaced by \var{x}}{}
  \lineiii{\var{s}[\var{i}:\var{j}] = \var{t}}
  	{slice of \var{s} from \var{i} to \var{j} is replaced by \var{t}}{}
  \lineiii{del \var{s}[\var{i}:\var{j}]}
	{same as \code{\var{s}[\var{i}:\var{j}] = []}}{}
  \lineiii{\var{s}.append(\var{x})}
	{same as \code{\var{s}[len(\var{s}):len(\var{s})] = [\var{x}]}}{}
  \lineiii{\var{s}.count(\var{x})}
	{return number of \var{i}'s for which \code{\var{s}[\var{i}] == \var{x}}}{}
  \lineiii{\var{s}.index(\var{x})}
	{return smallest \var{i} such that \code{\var{s}[\var{i}] == \var{x}}}{(1)}
  \lineiii{\var{s}.insert(\var{i}, \var{x})}
	{same as \code{\var{s}[\var{i}:\var{i}] = [\var{x}]}
	  if \code{\var{i} >= 0}}{}
  \lineiii{\var{s}.remove(\var{x})}
	{same as \code{del \var{s}[\var{s}.index(\var{x})]}}{(1)}
  \lineiii{\var{s}.reverse()}
	{reverses the items of \var{s} in place}{}
  \lineiii{\var{s}.sort()}
	{permutes the items of \var{s} to satisfy
        \code{\var{s}[\var{i}] <= \var{s}[\var{j}]},
        for \code{\var{i} < \var{j}}}{(2)}
\end{tableiii}
\indexiv{operations on}{mutable}{sequence}{types}
\indexiii{operations on}{sequence}{types}
\indexiii{operations on}{list}{type}
\indexii{subscript}{assignment}
\indexii{slice}{assignment}
\stindex{del}
\renewcommand{\indexsubitem}{(list method)}
\ttindex{append}
\ttindex{count}
\ttindex{index}
\ttindex{insert}
\ttindex{remove}
\ttindex{reverse}
\ttindex{sort}

\noindent
Notes:
\begin{description}
\item[(1)] Raises an exception when \var{x} is not found in \var{s}.
  
\item[(2)] The \code{sort()} method takes an optional argument
  specifying a comparison function of two arguments (list items) which
  should return \code{-1}, \code{0} or \code{1} depending on whether the
  first argument is considered smaller than, equal to, or larger than the
  second argument.  Note that this slows the sorting process down
  considerably; e.g. to sort a list in reverse order it is much faster
  to use calls to \code{sort()} and \code{reverse()} than to use
  \code{sort()} with a comparison function that reverses the ordering of
  the elements.
\end{description}

\subsection{Mapping Types}

A \dfn{mapping} object maps values of one type (the key type) to
arbitrary objects.  Mappings are mutable objects.  There is currently
only one standard mapping type, the \dfn{dictionary}.  A dictionary's keys are
almost arbitrary values.  The only types of values not acceptable as
keys are values containing lists or dictionaries or other mutable
types that are compared by value rather than by object identity.
Numeric types used for keys obey the normal rules for numeric
comparison: if two numbers compare equal (e.g. 1 and 1.0) then they
can be used interchangeably to index the same dictionary entry.

\indexii{mapping}{types}
\indexii{dictionary}{type}

Dictionaries are created by placing a comma-separated list of
\code{\var{key}:\,\var{value}} pairs within braces, for example:
\code{\{'jack':\,4098, 'sjoerd':\,4127\}} or
\code{\{4098:\,'jack', 4127:\,'sjoerd'\}}.

The following operations are defined on mappings (where \var{a} is a
mapping, \var{k} is a key and \var{x} is an arbitrary object):

\begin{tableiii}{|c|l|c|}{code}{Operation}{Result}{Notes}
  \lineiii{len(\var{a})}{the number of items in \var{a}}{}
  \lineiii{\var{a}[\var{k}]}{the item of \var{a} with key \var{k}}{(1)}
  \lineiii{\var{a}[\var{k}] = \var{x}}{set \code{\var{a}[\var{k}]} to \var{x}}{}
  \lineiii{del \var{a}[\var{k}]}{remove \code{\var{a}[\var{k}]} from \var{a}}{(1)}
  \lineiii{\var{a}.items()}{a copy of \var{a}'s list of (key, item) pairs}{(2)}
  \lineiii{\var{a}.keys()}{a copy of \var{a}'s list of keys}{(2)}
  \lineiii{\var{a}.values()}{a copy of \var{a}'s list of values}{(2)}
  \lineiii{\var{a}.has_key(\var{k})}{\code{1} if \var{a} has a key \var{k}, else \code{0}}{}
\end{tableiii}
\indexiii{operations on}{mapping}{types}
\indexiii{operations on}{dictionary}{type}
\stindex{del}
\bifuncindex{len}
\renewcommand{\indexsubitem}{(dictionary method)}
\ttindex{keys}
\ttindex{has_key}

\noindent
Notes:
\begin{description}
\item[(1)] Raises an exception if \var{k} is not in the map.

\item[(2)] Keys and values are listed in random order.
\end{description}

\subsection{Other Built-in Types}

The interpreter supports several other kinds of objects.
Most of these support only one or two operations.

\subsubsection{Modules}

The only special operation on a module is attribute access:
\code{\var{m}.\var{name}}, where \var{m} is a module and \var{name} accesses
a name defined in \var{m}'s symbol table.  Module attributes can be
assigned to.  (Note that the \code{import} statement is not, strictly
spoken, an operation on a module object; \code{import \var{foo}} does not
require a module object named \var{foo} to exist, rather it requires
an (external) \emph{definition} for a module named \var{foo}
somewhere.)

A special member of every module is \code{__dict__}.
This is the dictionary containing the module's symbol table.
Modifying this dictionary will actually change the module's symbol
table, but direct assignment to the \code{__dict__} attribute is not
possible (i.e., you can write \code{\var{m}.__dict__['a'] = 1}, which
defines \code{\var{m}.a} to be \code{1}, but you can't write \code{\var{m}.__dict__ = \{\}}.

Modules are written like this: \code{<module 'sys'>}.

\subsubsection{Classes and Class Instances}
\nodename{Classes and Instances}

(See Chapters 3 and 7 of the Python Reference Manual for these.)

\subsubsection{Functions}

Function objects are created by function definitions.  The only
operation on a function object is to call it:
\code{\var{func}(\var{argument-list})}.

There are really two flavors of function objects: built-in functions
and user-defined functions.  Both support the same operation (to call
the function), but the implementation is different, hence the
different object types.

The implementation adds two special read-only attributes:
\code{\var{f}.func_code} is a function's \dfn{code object} (see below) and
\code{\var{f}.func_globals} is the dictionary used as the function's
global name space (this is the same as \code{\var{m}.__dict__} where
\var{m} is the module in which the function \var{f} was defined).

\subsubsection{Methods}
\obindex{method}

Methods are functions that are called using the attribute notation.
There are two flavors: built-in methods (such as \code{append()} on
lists) and class instance methods.  Built-in methods are described
with the types that support them.

The implementation adds two special read-only attributes to class
instance methods: \code{\var{m}.im_self} is the object whose method this
is, and \code{\var{m}.im_func} is the function implementing the method.
Calling \code{\var{m}(\var{arg-1}, \var{arg-2}, {\rm \ldots},
\var{arg-n})} is completely equivalent to calling
\code{\var{m}.im_func(\var{m}.im_self, \var{arg-1}, \var{arg-2}, {\rm
\ldots}, \var{arg-n})}.

(See the Python Reference Manual for more info.)

\subsubsection{Code Objects}
\obindex{code}

Code objects are used by the implementation to represent
``pseudo-compiled'' executable Python code such as a function body.
They differ from function objects because they don't contain a
reference to their global execution environment.  Code objects are
returned by the built-in \code{compile()} function and can be
extracted from function objects through their \code{func_code}
attribute.
\bifuncindex{compile}
\ttindex{func_code}

A code object can be executed or evaluated by passing it (instead of a
source string) to the \code{exec} statement or the built-in
\code{eval()} function.
\stindex{exec}
\bifuncindex{eval}

(See the Python Reference Manual for more info.)

\subsubsection{Type Objects}

Type objects represent the various object types.  An object's type is
accessed by the built-in function \code{type()}.  There are no special
operations on types.  The standard module \code{types} defines names
for all standard built-in types.
\bifuncindex{type}
\stmodindex{types}

Types are written like this: \code{<type 'int'>}.

\subsubsection{The Null Object}

This object is returned by functions that don't explicitly return a
value.  It supports no special operations.  There is exactly one null
object, named \code{None} (a built-in name).

It is written as \code{None}.

\subsubsection{File Objects}

File objects are implemented using \C{}'s \code{stdio} package and can be
created with the built-in function \code{open()} described under
Built-in Functions below.  They are also returned by some other
built-in functions and methods, e.g.\ \code{posix.popen()} and
\code{posix.fdopen()} and the \code{makefile()} method of socket
objects.
\bifuncindex{open}

When a file operation fails for an I/O-related reason, the exception
\code{IOError} is raised.  This includes situations where the
operation is not defined for some reason, like \code{seek()} on a tty
device or writing a file opened for reading.

Files have the following methods:


\renewcommand{\indexsubitem}{(file method)}

\begin{funcdesc}{close}{}
  Close the file.  A closed file cannot be read or written anymore.
\end{funcdesc}

\begin{funcdesc}{flush}{}
  Flush the internal buffer, like \code{stdio}'s \code{fflush()}.
\end{funcdesc}

\begin{funcdesc}{isatty}{}
  Return \code{1} if the file is connected to a tty(-like) device, else
  \code{0}.
\end{funcdesc}

\begin{funcdesc}{read}{\optional{size}}
  Read at most \var{size} bytes from the file (less if the read hits
  \EOF{} or no more data is immediately available on a pipe, tty or
  similar device).  If the \var{size} argument is negative or omitted,
  read all data until \EOF{} is reached.  The bytes are returned as a string
  object.  An empty string is returned when \EOF{} is encountered
  immediately.  (For certain files, like ttys, it makes sense to
  continue reading after an \EOF{} is hit.)
\end{funcdesc}

\begin{funcdesc}{readline}{\optional{size}}
  Read one entire line from the file.  A trailing newline character is
  kept in the string%
\footnote{The advantage of leaving the newline on is that an empty string 
	can be returned to mean \EOF{} without being ambiguous.  Another 
	advantage is that (in cases where it might matter, e.g. if you 
	want to make an exact copy of a file while scanning its lines) 
	you can tell whether the last line of a file ended in a newline
	or not (yes this happens!).}
  (but may be absent when a file ends with an
  incomplete line).  If the \var{size} argument is present and
  non-negative, it is a maximum byte count (including the trailing
  newline) and an incomplete line may be returned.
  An empty string is returned when \EOF{} is hit
  immediately.  Note: unlike \code{stdio}'s \code{fgets()}, the returned
  string contains null characters (\code{'\e 0'}) if they occurred in the
  input.
\end{funcdesc}

\begin{funcdesc}{readlines}{}
  Read until \EOF{} using \code{readline()} and return a list containing
  the lines thus read.
\end{funcdesc}

\begin{funcdesc}{seek}{offset\, whence}
  Set the file's current position, like \code{stdio}'s \code{fseek()}.
  The \var{whence} argument is optional and defaults to \code{0}
  (absolute file positioning); other values are \code{1} (seek
  relative to the current position) and \code{2} (seek relative to the
  file's end).  There is no return value.
\end{funcdesc}

\begin{funcdesc}{tell}{}
  Return the file's current position, like \code{stdio}'s \code{ftell()}.
\end{funcdesc}

\begin{funcdesc}{truncate}{\optional{size}}
Truncate the file's size.  If the optional size argument present, the
file is truncated to (at most) that size.  The size defaults to the
current position.  Availability of this function depends on the
operating system version (e.g., not all \UNIX{} versions support this
operation).
\end{funcdesc}

\begin{funcdesc}{write}{str}
Write a string to the file.  There is no return value.  Note: due to
buffering, the string may not actually show up in the file until
the \code{flush()} or \code{close()} method is called.
\end{funcdesc}

\begin{funcdesc}{writelines}{list}
Write a list of strings to the file.  There is no return value.
(The name is intended to match \code{readlines}; \code{writelines}
does not add line separators.)
\end{funcdesc}

\subsubsection{Internal Objects}

(See the Python Reference Manual for these.)

\subsection{Special Attributes}

The implementation adds a few special read-only attributes to several
object types, where they are relevant:

\begin{itemize}

\item
\code{\var{x}.__dict__} is a dictionary of some sort used to store an
object's (writable) attributes;

\item
\code{\var{x}.__methods__} lists the methods of many built-in object types,
e.g., \code{[].__methods__} yields
\code{['append', 'count', 'index', 'insert', 'remove', 'reverse', 'sort']};

\item
\code{\var{x}.__members__} lists data attributes;

\item
\code{\var{x}.__class__} is the class to which a class instance belongs;

\item
\code{\var{x}.__bases__} is the tuple of base classes of a class object.

\end{itemize}

\section{Built-in Exceptions}

Exceptions are string objects.  Two distinct string objects with the
same value are different exceptions.  This is done to force programmers
to use exception names rather than their string value when specifying
exception handlers.  The string value of all built-in exceptions is
their name, but this is not a requirement for user-defined exceptions
or exceptions defined by library modules.

The following exceptions can be generated by the interpreter or
built-in functions.  Except where mentioned, they have an `associated
value' indicating the detailed cause of the error.  This may be a
string or a tuple containing several items of information (e.g., an
error code and a string explaining the code).

User code can raise built-in exceptions.  This can be used to test an
exception handler or to report an error condition `just like' the
situation in which the interpreter raises the same exception; but
beware that there is nothing to prevent user code from raising an
inappropriate error.

\renewcommand{\indexsubitem}{(built-in exception)}

\begin{excdesc}{AttributeError}
% xref to attribute reference?
  Raised when an attribute reference or assignment fails.  (When an
  object does not support attribute references or attribute assignments
  at all, \code{TypeError} is raised.)
\end{excdesc}

\begin{excdesc}{EOFError}
% XXXJH xrefs here
  Raised when one of the built-in functions (\code{input()} or
  \code{raw_input()}) hits an end-of-file condition (\EOF{}) without
  reading any data.
% XXXJH xrefs here
  (N.B.: the \code{read()} and \code{readline()} methods of file
  objects return an empty string when they hit \EOF{}.)  No associated value.
\end{excdesc}

\begin{excdesc}{IOError}
% XXXJH xrefs here
  Raised when an I/O operation (such as a \code{print} statement, the
  built-in \code{open()} function or a method of a file object) fails
  for an I/O-related reason, e.g., `file not found', `disk full'.
\end{excdesc}

\begin{excdesc}{ImportError}
% XXXJH xref to import statement?
  Raised when an \code{import} statement fails to find the module
  definition or when a \code{from {\rm \ldots} import} fails to find a
  name that is to be imported.
\end{excdesc}

\begin{excdesc}{IndexError}
% XXXJH xref to sequences
  Raised when a sequence subscript is out of range.  (Slice indices are
  silently truncated to fall in the allowed range; if an index is not a
  plain integer, \code{TypeError} is raised.)
\end{excdesc}

\begin{excdesc}{KeyError}
% XXXJH xref to mapping objects?
  Raised when a mapping (dictionary) key is not found in the set of
  existing keys.
\end{excdesc}

\begin{excdesc}{KeyboardInterrupt}
  Raised when the user hits the interrupt key (normally
  \kbd{Control-C} or
\key{DEL}).  During execution, a check for interrupts is made regularly.
% XXXJH xrefs here
  Interrupts typed when a built-in function \code{input()} or
  \code{raw_input()}) is waiting for input also raise this exception.  No
  associated value.
\end{excdesc}

\begin{excdesc}{MemoryError}
  Raised when an operation runs out of memory but the situation may
  still be rescued (by deleting some objects).  The associated value is
  a string indicating what kind of (internal) operation ran out of memory.
  Note that because of the underlying memory management architecture
  (\C{}'s \code{malloc()} function), the interpreter may not always be able
  to completely recover from this situation; it nevertheless raises an
  exception so that a stack traceback can be printed, in case a run-away
  program was the cause.
\end{excdesc}

\begin{excdesc}{NameError}
  Raised when a local or global name is not found.  This applies only
  to unqualified names.  The associated value is the name that could
  not be found.
\end{excdesc}

\begin{excdesc}{OverflowError}
% XXXJH reference to long's and/or int's?
  Raised when the result of an arithmetic operation is too large to be
  represented.  This cannot occur for long integers (which would rather
  raise \code{MemoryError} than give up).  Because of the lack of
  standardization of floating point exception handling in \C{}, most
  floating point operations also aren't checked.  For plain integers,
  all operations that can overflow are checked except left shift, where
  typical applications prefer to drop bits than raise an exception.
\end{excdesc}

\begin{excdesc}{RuntimeError}
  Raised when an error is detected that doesn't fall in any of the
  other categories.  The associated value is a string indicating what
  precisely went wrong.  (This exception is a relic from a previous
  version of the interpreter; it is not used any more except by some
  extension modules that haven't been converted to define their own
  exceptions yet.)
\end{excdesc}

\begin{excdesc}{SyntaxError}
% XXXJH xref to these functions?
  Raised when the parser encounters a syntax error.  This may occur in
  an \code{import} statement, in an \code{exec} statement, in a call
  to the built-in function \code{eval()} or \code{input()}, or
  when reading the initial script or standard input (also
  interactively).
\end{excdesc}

\begin{excdesc}{SystemError}
  Raised when the interpreter finds an internal error, but the
  situation does not look so serious to cause it to abandon all hope.
  The associated value is a string indicating what went wrong (in
  low-level terms).
  
  You should report this to the author or maintainer of your Python
  interpreter.  Be sure to report the version string of the Python
  interpreter (\code{sys.version}; it is also printed at the start of an
  interactive Python session), the exact error message (the exception's
  associated value) and if possible the source of the program that
  triggered the error.
\end{excdesc}

\begin{excdesc}{SystemExit}
% XXXJH xref to module sys?
  This exception is raised by the \code{sys.exit()} function.  When it
  is not handled, the Python interpreter exits; no stack traceback is
  printed.  If the associated value is a plain integer, it specifies the
  system exit status (passed to \C{}'s \code{exit()} function); if it is
  \code{None}, the exit status is zero; if it has another type (such as
  a string), the object's value is printed and the exit status is one.
  
  A call to \code{sys.exit} is translated into an exception so that
  clean-up handlers (\code{finally} clauses of \code{try} statements)
  can be executed, and so that a debugger can execute a script without
  running the risk of losing control.  The \code{posix._exit()} function
  can be used if it is absolutely positively necessary to exit
  immediately (e.g., after a \code{fork()} in the child process).
\end{excdesc}

\begin{excdesc}{TypeError}
  Raised when a built-in operation or function is applied to an object
  of inappropriate type.  The associated value is a string giving
  details about the type mismatch.
\end{excdesc}

\begin{excdesc}{ValueError}
  Raised when a built-in operation or function receives an argument
  that has the right type but an inappropriate value, and the
  situation is not described by a more precise exception such as
  \code{IndexError}.
\end{excdesc}

\begin{excdesc}{ZeroDivisionError}
  Raised when the second argument of a division or modulo operation is
  zero.  The associated value is a string indicating the type of the
  operands and the operation.
\end{excdesc}

\section{Built-in Functions}

The Python interpreter has a number of functions built into it that
are always available.  They are listed here in alphabetical order.


\renewcommand{\indexsubitem}{(built-in function)}
\begin{funcdesc}{abs}{x}
  Return the absolute value of a number.  The argument may be a plain
  or long integer or a floating point number.
\end{funcdesc}

\begin{funcdesc}{apply}{function\, args\optional{, keywords}}
The \var{function} argument must be a callable object (a user-defined or
built-in function or method, or a class object) and the \var{args}
argument must be a tuple.  The \var{function} is called with
\var{args} as argument list; the number of arguments is the the length
of the tuple.  (This is different from just calling
\code{\var{func}(\var{args})}, since in that case there is always
exactly one argument.)
If the optional \var{keywords} argument is present, it must be a
dictionary whose keys are strings.  It specifies keyword arguments to
be added to the end of the the argument list.
\end{funcdesc}

\begin{funcdesc}{chr}{i}
  Return a string of one character whose \ASCII{} code is the integer
  \var{i}, e.g., \code{chr(97)} returns the string \code{'a'}.  This is the
  inverse of \code{ord()}.  The argument must be in the range [0..255],
  inclusive.
\end{funcdesc}

\begin{funcdesc}{cmp}{x\, y}
  Compare the two objects \var{x} and \var{y} and return an integer
  according to the outcome.  The return value is negative if \code{\var{x}
  < \var{y}}, zero if \code{\var{x} == \var{y}} and strictly positive if
  \code{\var{x} > \var{y}}.
\end{funcdesc}

\begin{funcdesc}{coerce}{x\, y}
  Return a tuple consisting of the two numeric arguments converted to
  a common type, using the same rules as used by arithmetic
  operations.
\end{funcdesc}

\begin{funcdesc}{compile}{string\, filename\, kind}
  Compile the \var{string} into a code object.  Code objects can be
  executed by an \code{exec} statement or evaluated by a call to
  \code{eval()}.  The \var{filename} argument should
  give the file from which the code was read; pass e.g. \code{'<string>'}
  if it wasn't read from a file.  The \var{kind} argument specifies
  what kind of code must be compiled; it can be \code{'exec'} if
  \var{string} consists of a sequence of statements, \code{'eval'}
  if it consists of a single expression, or \code{'single'} if
  it consists of a single interactive statement (in the latter case,
  expression statements that evaluate to something else than
  \code{None} will printed).
\end{funcdesc}

\begin{funcdesc}{delattr}{object\, name}
  This is a relative of \code{setattr}.  The arguments are an
  object and a string.  The string must be the name
  of one of the object's attributes.  The function deletes
  the named attribute, provided the object allows it.  For example,
  \code{delattr(\var{x}, '\var{foobar}')} is equivalent to
  \code{del \var{x}.\var{foobar}}.
\end{funcdesc}

\begin{funcdesc}{dir}{}
  Without arguments, return the list of names in the current local
  symbol table.  With a module, class or class instance object as
  argument (or anything else that has a \code{__dict__} attribute),
  returns the list of names in that object's attribute dictionary.
  The resulting list is sorted.  For example:

\bcode\begin{verbatim}
>>> import sys
>>> dir()
['sys']
>>> dir(sys)
['argv', 'exit', 'modules', 'path', 'stderr', 'stdin', 'stdout']
>>> 
\end{verbatim}\ecode
\end{funcdesc}

\begin{funcdesc}{divmod}{a\, b}
  Take two numbers as arguments and return a pair of integers
  consisting of their integer quotient and remainder.  With mixed
  operand types, the rules for binary arithmetic operators apply.  For
  plain and long integers, the result is the same as
  \code{(\var{a} / \var{b}, \var{a} \%{} \var{b})}.
  For floating point numbers the result is the same as
  \code{(math.floor(\var{a} / \var{b}), \var{a} \%{} \var{b})}.
\end{funcdesc}

\begin{funcdesc}{eval}{expression\optional{\, globals\optional{\, locals}}}
  The arguments are a string and two optional dictionaries.  The
  \var{expression} argument is parsed and evaluated as a Python
  expression (technically speaking, a condition list) using the
  \var{globals} and \var{locals} dictionaries as global and local name
  space.  If the \var{locals} dictionary is omitted it defaults to
  the \var{globals} dictionary.  If both dictionaries are omitted, the
  expression is executed in the environment where \code{eval} is
  called.  The return value is the result of the evaluated expression.
  Syntax errors are reported as exceptions.  Example:

\bcode\begin{verbatim}
>>> x = 1
>>> print eval('x+1')
2
>>> 
\end{verbatim}\ecode

  This function can also be used to execute arbitrary code objects
  (e.g.\ created by \code{compile()}).  In this case pass a code
  object instead of a string.  The code object must have been compiled
  passing \code{'eval'} to the \var{kind} argument.

  Hints: dynamic execution of statements is supported by the
  \code{exec} statement.  Execution of statements from a file is
  supported by the \code{execfile()} function.  The \code{globals()}
  and \code{locals()} functions returns the current global and local
  dictionary, respectively, which may be useful
  to pass around for use by \code{eval()} or \code{execfile()}.

\end{funcdesc}

\begin{funcdesc}{execfile}{file\optional{\, globals\optional{\, locals}}}
  This function is similar to the
  \code{exec} statement, but parses a file instead of a string.  It is
  different from the \code{import} statement in that it does not use
  the module administration --- it reads the file unconditionally and
  does not create a new module.\footnote{It is used relatively rarely
  so does not warrant being made into a statement.}

  The arguments are a file name and two optional dictionaries.  The
  file is parsed and evaluated as a sequence of Python statements
  (similarly to a module) using the \var{globals} and \var{locals}
  dictionaries as global and local name space.  If the \var{locals}
  dictionary is omitted it defaults to the \var{globals} dictionary.
  If both dictionaries are omitted, the expression is executed in the
  environment where \code{execfile()} is called.  The return value is
  \code{None}.
\end{funcdesc}

\begin{funcdesc}{filter}{function\, list}
Construct a list from those elements of \var{list} for which
\var{function} returns true.  If \var{list} is a string or a tuple,
the result also has that type; otherwise it is always a list.  If
\var{function} is \code{None}, the identity function is assumed,
i.e.\ all elements of \var{list} that are false (zero or empty) are
removed.
\end{funcdesc}

\begin{funcdesc}{float}{x}
  Convert a number to floating point.  The argument may be a plain or
  long integer or a floating point number.
\end{funcdesc}

\begin{funcdesc}{getattr}{object\, name}
  The arguments are an object and a string.  The string must be the
  name
  of one of the object's attributes.  The result is the value of that
  attribute.  For example, \code{getattr(\var{x}, '\var{foobar}')} is equivalent to
  \code{\var{x}.\var{foobar}}.
\end{funcdesc}

\begin{funcdesc}{globals}{}
Return a dictionary representing the current global symbol table.
This is always the dictionary of the current module (inside a
function or method, this is the module where it is defined, not the
module from which it is called).
\end{funcdesc}

\begin{funcdesc}{hasattr}{object\, name}
  The arguments are an object and a string.  The result is 1 if the
  string is the name of one of the object's attributes, 0 if not.
  (This is implemented by calling \code{getattr(object, name)} and
  seeing whether it raises an exception or not.)
\end{funcdesc}

\begin{funcdesc}{hash}{object}
  Return the hash value of the object (if it has one).  Hash values
  are 32-bit integers.  They are used to quickly compare dictionary
  keys during a dictionary lookup.  Numeric values that compare equal
  have the same hash value (even if they are of different types, e.g.
  1 and 1.0).
\end{funcdesc}

\begin{funcdesc}{hex}{x}
  Convert an integer number (of any size) to a hexadecimal string.
  The result is a valid Python expression.
\end{funcdesc}

\begin{funcdesc}{id}{object}
  Return the `identity' of an object.  This is an integer which is
  guaranteed to be unique and constant for this object during its
  lifetime.  (Two objects whose lifetimes are disjunct may have the
  same id() value.)  (Implementation note: this is the address of the
  object.)
\end{funcdesc}

\begin{funcdesc}{input}{\optional{prompt}}
  Almost equivalent to \code{eval(raw_input(\var{prompt}))}.  Like
  \code{raw_input()}, the \var{prompt} argument is optional.  The difference
  is that a long input expression may be broken over multiple lines using
  the backslash convention.
\end{funcdesc}

\begin{funcdesc}{int}{x}
  Convert a number to a plain integer.  The argument may be a plain or
  long integer or a floating point number.  Conversion of floating
  point numbers to integers is defined by the C semantics; normally
  the conversion truncates towards zero.\footnote{This is ugly --- the
  language definition should require truncation towards zero.}
\end{funcdesc}

\begin{funcdesc}{len}{s}
  Return the length (the number of items) of an object.  The argument
  may be a sequence (string, tuple or list) or a mapping (dictionary).
\end{funcdesc}

\begin{funcdesc}{locals}{}
Return a dictionary representing the current local symbol table.
Inside a function, modifying this dictionary does not always have the
desired effect.
\end{funcdesc}

\begin{funcdesc}{long}{x}
  Convert a number to a long integer.  The argument may be a plain or
  long integer or a floating point number.
\end{funcdesc}

\begin{funcdesc}{map}{function\, list\, ...}
Apply \var{function} to every item of \var{list} and return a list
of the results.  If additional \var{list} arguments are passed, 
\var{function} must take that many arguments and is applied to
the items of all lists in parallel; if a list is shorter than another
it is assumed to be extended with \code{None} items.  If
\var{function} is \code{None}, the identity function is assumed; if
there are multiple list arguments, \code{map} returns a list
consisting of tuples containing the corresponding items from all lists
(i.e. a kind of transpose operation).  The \var{list} arguments may be
any kind of sequence; the result is always a list.
\end{funcdesc}

\begin{funcdesc}{max}{s}
  Return the largest item of a non-empty sequence (string, tuple or
  list).
\end{funcdesc}

\begin{funcdesc}{min}{s}
  Return the smallest item of a non-empty sequence (string, tuple or
  list).
\end{funcdesc}

\begin{funcdesc}{oct}{x}
  Convert an integer number (of any size) to an octal string.  The
  result is a valid Python expression.
\end{funcdesc}

\begin{funcdesc}{open}{filename\optional{\, mode\optional{\, bufsize}}}
  Return a new file object (described earlier under Built-in Types).
  The first two arguments are the same as for \code{stdio}'s
  \code{fopen()}: \var{filename} is the file name to be opened,
  \var{mode} indicates how the file is to be opened: \code{'r'} for
  reading, \code{'w'} for writing (truncating an existing file), and
  \code{'a'} opens it for appending (which on {\em some} \UNIX{}
  systems means that {\em all} writes append to the end of the file,
  regardless of the current seek position).
  Modes \code{'r+'}, \code{'w+'} and
  \code{'a+'} open the file for updating, provided the underlying
  \code{stdio} library understands this.  On systems that differentiate
  between binary and text files, \code{'b'} appended to the mode opens
  the file in binary mode.  If the file cannot be opened, \code{IOError}
  is raised.
If \var{mode} is omitted, it defaults to \code{'r'}.
The optional \var{bufsize} argument specifies the file's desired
buffer size: 0 means unbuffered, 1 means line buffered, any other
positive value means use a buffer of (approximately) that size.  A
negative \var{bufsize} means to use the system default, which is
usually line buffered for for tty devices and fully buffered for other
files.%
\footnote{Specifying a buffer size currently has no effect on systems
that don't have \code{setvbuf()}.  The interface to specify the buffer
size is not done using a method that calls \code{setvbuf()}, because
that may dump core when called after any I/O has been performed, and
there's no reliable way to determine whether this is the case.}
\end{funcdesc}

\begin{funcdesc}{ord}{c}
  Return the \ASCII{} value of a string of one character.  E.g.,
  \code{ord('a')} returns the integer \code{97}.  This is the inverse of
  \code{chr()}.
\end{funcdesc}

\begin{funcdesc}{pow}{x\, y\optional{\, z}}
  Return \var{x} to the power \var{y}; if \var{z} is present, return
  \var{x} to the power \var{y}, modulo \var{z} (computed more
  efficiently than \code{pow(\var{x}, \var{y}) \% \var{z}}).
  The arguments must have
  numeric types.  With mixed operand types, the rules for binary
  arithmetic operators apply.  The effective operand type is also the
  type of the result; if the result is not expressible in this type, the
  function raises an exception; e.g., \code{pow(2, -1)} or \code{pow(2,
  35000)} is not allowed.
\end{funcdesc}

\begin{funcdesc}{range}{\optional{start\,} end\optional{\, step}}
  This is a versatile function to create lists containing arithmetic
  progressions.  It is most often used in \code{for} loops.  The
  arguments must be plain integers.  If the \var{step} argument is
  omitted, it defaults to \code{1}.  If the \var{start} argument is
  omitted, it defaults to \code{0}.  The full form returns a list of
  plain integers \code{[\var{start}, \var{start} + \var{step},
  \var{start} + 2 * \var{step}, \ldots]}.  If \var{step} is positive,
  the last element is the largest \code{\var{start} + \var{i} *
  \var{step}} less than \var{end}; if \var{step} is negative, the last
  element is the largest \code{\var{start} + \var{i} * \var{step}}
  greater than \var{end}.  \var{step} must not be zero (or else an
  exception is raised).  Example:

\bcode\begin{verbatim}
>>> range(10)
[0, 1, 2, 3, 4, 5, 6, 7, 8, 9]
>>> range(1, 11)
[1, 2, 3, 4, 5, 6, 7, 8, 9, 10]
>>> range(0, 30, 5)
[0, 5, 10, 15, 20, 25]
>>> range(0, 10, 3)
[0, 3, 6, 9]
>>> range(0, -10, -1)
[0, -1, -2, -3, -4, -5, -6, -7, -8, -9]
>>> range(0)
[]
>>> range(1, 0)
[]
>>> 
\end{verbatim}\ecode
\end{funcdesc}

\begin{funcdesc}{raw_input}{\optional{prompt}}
  If the \var{prompt} argument is present, it is written to standard output
  without a trailing newline.  The function then reads a line from input,
  converts it to a string (stripping a trailing newline), and returns that.
  When \EOF{} is read, \code{EOFError} is raised. Example:

\bcode\begin{verbatim}
>>> s = raw_input('--> ')
--> Monty Python's Flying Circus
>>> s
"Monty Python's Flying Circus"
>>> 
\end{verbatim}\ecode
\end{funcdesc}

\begin{funcdesc}{reduce}{function\, list\optional{\, initializer}}
Apply the binary \var{function} to the items of \var{list} so as to
reduce the list to a single value.  E.g.,
\code{reduce(lambda x, y: x*y, \var{list}, 1)} returns the product of
the elements of \var{list}.  The optional \var{initializer} can be
thought of as being prepended to \var{list} so as to allow reduction
of an empty \var{list}.  The \var{list} arguments may be any kind of
sequence.
\end{funcdesc}

\begin{funcdesc}{reload}{module}
Re-parse and re-initialize an already imported \var{module}.  The
argument must be a module object, so it must have been successfully
imported before.  This is useful if you have edited the module source
file using an external editor and want to try out the new version
without leaving the Python interpreter.  The return value is the
module object (i.e.\ the same as the \var{module} argument).

There are a number of caveats:

If a module is syntactically correct but its initialization fails, the
first \code{import} statement for it does not bind its name locally,
but does store a (partially initialized) module object in
\code{sys.modules}.  To reload the module you must first
\code{import} it again (this will bind the name to the partially
initialized module object) before you can \code{reload()} it.

When a module is reloaded, its dictionary (containing the module's
global variables) is retained.  Redefinitions of names will override
the old definitions, so this is generally not a problem.  If the new
version of a module does not define a name that was defined by the old
version, the old definition remains.  This feature can be used to the
module's advantage if it maintains a global table or cache of objects
--- with a \code{try} statement it can test for the table's presence
and skip its initialization if desired.

It is legal though generally not very useful to reload built-in or
dynamically loaded modules, except for \code{sys}, \code{__main__} and
\code{__builtin__}.  In certain cases, however, extension modules are
not designed to be initialized more than once, and may fail in
arbitrary ways when reloaded.

If a module imports objects from another module using \code{from}
{\ldots} \code{import} {\ldots}, calling \code{reload()} for the other
module does not redefine the objects imported from it --- one way
around this is to re-execute the \code{from} statement, another is to
use \code{import} and qualified names (\var{module}.\var{name})
instead.

If a module instantiates instances of a class, reloading the module
that defines the class does not affect the method definitions of the
instances --- they continue to use the old class definition.  The same
is true for derived classes.
\end{funcdesc}

\begin{funcdesc}{repr}{object}
Return a string containing a printable representation of an object.
This is the same value yielded by conversions (reverse quotes).
It is sometimes useful to be able to access this operation as an
ordinary function.  For many types, this function makes an attempt
to return a string that would yield an object with the same value
when passed to \code{eval()}.
\end{funcdesc}

\begin{funcdesc}{round}{x\, n}
  Return the floating point value \var{x} rounded to \var{n} digits
  after the decimal point.  If \var{n} is omitted, it defaults to zero.
  The result is a floating point number.  Values are rounded to the
  closest multiple of 10 to the power minus \var{n}; if two multiples
  are equally close, rounding is done away from 0 (so e.g.
  \code{round(0.5)} is \code{1.0} and \code{round(-0.5)} is \code{-1.0}).
\end{funcdesc}

\begin{funcdesc}{setattr}{object\, name\, value}
  This is the counterpart of \code{getattr}.  The arguments are an
  object, a string and an arbitrary value.  The string must be the name
  of one of the object's attributes.  The function assigns the value to
  the attribute, provided the object allows it.  For example,
  \code{setattr(\var{x}, '\var{foobar}', 123)} is equivalent to
  \code{\var{x}.\var{foobar} = 123}.
\end{funcdesc}

\begin{funcdesc}{str}{object}
Return a string containing a nicely printable representation of an
object.  For strings, this returns the string itself.  The difference
with \code{repr(\var{object})} is that \code{str(\var{object})} does not
always attempt to return a string that is acceptable to \code{eval()};
its goal is to return a printable string.
\end{funcdesc}

\begin{funcdesc}{tuple}{sequence}
Return a tuple whose items are the same and in the same order as
\var{sequence}'s items.  If \var{sequence} is alread a tuple, it
is returned unchanged.  For instance, \code{tuple('abc')} returns
returns \code{('a', 'b', 'c')} and \code{tuple([1, 2, 3])} returns
\code{(1, 2, 3)}.
\end{funcdesc}

\begin{funcdesc}{type}{object}
Return the type of an \var{object}.  The return value is a type
object.  The standard module \code{types} defines names for all
built-in types.
\stmodindex{types}
\obindex{type}
For instance:

\bcode\begin{verbatim}
>>> import types
>>> if type(x) == types.StringType: print "It's a string"
\end{verbatim}\ecode
\end{funcdesc}

\begin{funcdesc}{vars}{\optional{object}}
Without arguments, return a dictionary corresponding to the current
local symbol table.  With a module, class or class instance object as
argument (or anything else that has a \code{__dict__} attribute),
returns a dictionary corresponding to the object's symbol table.
The returned dictionary should not be modified: the effects on the
corresponding symbol table are undefined.%
\footnote{In the current implementation, local variable bindings
cannot normally be affected this way, but variables retrieved from
other scopes (e.g. modules) can be.  This may change.}
\end{funcdesc}

\begin{funcdesc}{xrange}{\optional{start\,} end\optional{\, step}}
This function is very similar to \code{range()}, but returns an
``xrange object'' instead of a list.  This is an opaque sequence type
which yields the same values as the corresponding list, without
actually storing them all simultaneously.  The advantage of
\code{xrange()} over \code{range()} is minimal (since \code{xrange()}
still has to create the values when asked for them) except when a very
large range is used on a memory-starved machine (e.g. MS-DOS) or when all
of the range's elements are never used (e.g. when the loop is usually
terminated with \code{break}).
\end{funcdesc}


\chapter{Python Services}

The modules described in this chapter provide a wide range of services
related to the Python interpreter and its interaction with its
environment.  Here's an overview:

\begin{description}

\item[sys]
--- Access system specific parameters and functions.

\item[types]
--- Names for all built-in types.

\item[traceback]
--- Print or retrieve a stack traceback.

\item[pickle]
--- Convert Python objects to streams of bytes and back.

\item[shelve]
--- Python object persistency.

\item[copy]
--- Shallow and deep copy operations.

\item[marshal]
--- Convert Python objects to streams of bytes and back (with
different constraints).

\item[imp]
--- Access the implementation of the \code{import} statement.

\item[parser]
--- Retrieve and submit parse trees from and to the runtime support
environment.

\item[__builtin__]
--- The set of built-in functions.

\item[__main__]
--- The environment where the top-level script is run.

\end{description}
		% Python Services
\section{Built-in Module \sectcode{sys}}

\bimodindex{sys}
This module provides access to some variables used or maintained by the
interpreter and to functions that interact strongly with the interpreter.
It is always available.

\renewcommand{\indexsubitem}{(in module sys)}

\begin{datadesc}{argv}
  The list of command line arguments passed to a Python script.
  \code{sys.argv[0]} is the script name (it is operating system
  dependent whether this is a full pathname or not).
  If the command was executed using the \samp{-c} command line option
  to the interpreter, \code{sys.argv[0]} is set to the string
  \code{"-c"}.
  If no script name was passed to the Python interpreter,
  \code{sys.argv} has zero length.
\end{datadesc}

\begin{datadesc}{builtin_module_names}
  A list of strings giving the names of all modules that are compiled
  into this Python interpreter.  (This information is not available in
  any other way --- \code{sys.modules.keys()} only lists the imported
  modules.)
\end{datadesc}

\begin{datadesc}{exc_type}
\dataline{exc_value}
\dataline{exc_traceback}
  These three variables are not always defined; they are set when an
  exception handler (an \code{except} clause of a \code{try} statement) is
  invoked.  Their meaning is: \code{exc_type} gets the exception type of
  the exception being handled; \code{exc_value} gets the exception
  parameter (its \dfn{associated value} or the second argument to
  \code{raise}); \code{exc_traceback} gets a traceback object (see the
  Reference Manual) which
  encapsulates the call stack at the point where the exception
  originally occurred.
\obindex{traceback}
\end{datadesc}

\begin{funcdesc}{exit}{n}
  Exit from Python with numeric exit status \var{n}.  This is
  implemented by raising the \code{SystemExit} exception, so cleanup
  actions specified by \code{finally} clauses of \code{try} statements
  are honored, and it is possible to catch the exit attempt at an outer
  level.
\end{funcdesc}

\begin{datadesc}{exitfunc}
  This value is not actually defined by the module, but can be set by
  the user (or by a program) to specify a clean-up action at program
  exit.  When set, it should be a parameterless function.  This function
  will be called when the interpreter exits in any way (except when a
  fatal error occurs: in that case the interpreter's internal state
  cannot be trusted).
\end{datadesc}

\begin{datadesc}{last_type}
\dataline{last_value}
\dataline{last_traceback}
  These three variables are not always defined; they are set when an
  exception is not handled and the interpreter prints an error message
  and a stack traceback.  Their intended use is to allow an interactive
  user to import a debugger module and engage in post-mortem debugging
  without having to re-execute the command that caused the error (which
  may be hard to reproduce).  The meaning of the variables is the same
  as that of \code{exc_type}, \code{exc_value} and \code{exc_tracaback},
  respectively.
\end{datadesc}

\begin{datadesc}{modules}
  Gives the list of modules that have already been loaded.
  This can be manipulated to force reloading of modules and other tricks.
\end{datadesc}

\begin{datadesc}{path}
  A list of strings that specifies the search path for modules.
  Initialized from the environment variable \code{PYTHONPATH}, or an
  installation-dependent default.
\end{datadesc}

\begin{datadesc}{platform}
This string contains a platform identifier.  This can be used to
append platform-specific components to \code{sys.path}, for instance.
\end{datadesc}

\begin{datadesc}{ps1}
\dataline{ps2}
  Strings specifying the primary and secondary prompt of the
  interpreter.  These are only defined if the interpreter is in
  interactive mode.  Their initial values in this case are
  \code{'>>> '} and \code{'... '}.
\end{datadesc}

\begin{funcdesc}{setcheckinterval}{interval}
Set the interpreter's ``check interval''.  This integer value
determines how often the interpreter checks for periodic things such
as thread switches and signal handlers.  The default is 10, meaning
the check is performed every 10 Python virtual instructions.  Setting
it to a larger value may increase performance for programs using
threads.  Setting it to a value $\leq 0$ checks every virtual instruction,
maximizing responsiveness as well as overhead.
\end{funcdesc}

\begin{funcdesc}{settrace}{tracefunc}
  Set the system's trace function, which allows you to implement a
  Python source code debugger in Python.  See section ``How It Works''
  in the chapter on the Python Debugger.
\end{funcdesc}
\index{trace function}
\index{debugger}

\begin{funcdesc}{setprofile}{profilefunc}
  Set the system's profile function, which allows you to implement a
  Python source code profiler in Python.  See the chapter on the
  Python Profiler.  The system's profile function
  is called similarly to the system's trace function (see
  \code{sys.settrace}), but it isn't called for each executed line of
  code (only on call and return and when an exception occurs).  Also,
  its return value is not used, so it can just return \code{None}.
\end{funcdesc}
\index{profile function}
\index{profiler}

\begin{datadesc}{stdin}
\dataline{stdout}
\dataline{stderr}
  File objects corresponding to the interpreter's standard input,
  output and error streams.  \code{sys.stdin} is used for all
  interpreter input except for scripts but including calls to
  \code{input()} and \code{raw_input()}.  \code{sys.stdout} is used
  for the output of \code{print} and expression statements and for the
  prompts of \code{input()} and \code{raw_input()}.  The interpreter's
  own prompts and (almost all of) its error messages go to
  \code{sys.stderr}.  \code{sys.stdout} and \code{sys.stderr} needn't
  be built-in file objects: any object is acceptable as long as it has
  a \code{write} method that takes a string argument.  (Changing these
  objects doesn't affect the standard I/O streams of processes
  executed by \code{popen()}, \code{system()} or the \code{exec*()}
  family of functions in the \code{os} module.)
\stmodindex{os}
\end{datadesc}

\begin{datadesc}{tracebacklimit}
When this variable is set to an integer value, it determines the
maximum number of levels of traceback information printed when an
unhandled exception occurs.  The default is 1000.  When set to 0 or
less, all traceback information is suppressed and only the exception
type and value are printed.
\end{datadesc}

\section{Standard Module \sectcode{types}}
\stmodindex{types}

\renewcommand{\indexsubitem}{(in module types)}

This module defines names for all object types that are used by the
standard Python interpreter (but not for the types defined by various
extension modules).  It is safe to use ``\code{from types import *}'' ---
the module does not export any other names besides the ones listed
here.  New names exported by future versions of this module will
all end in \code{Type}.

Typical use is for functions that do different things depending on
their argument types, like the following:

\begin{verbatim}
from types import *
def delete(list, item):
    if type(item) is IntType:
       del list[item]
    else:
       list.remove(item)
\end{verbatim}

The module defines the following names:

\begin{datadesc}{NoneType}
The type of \code{None}.
\end{datadesc}

\begin{datadesc}{TypeType}
The type of type objects (such as returned by \code{type()}).
\end{datadesc}

\begin{datadesc}{IntType}
The type of integers (e.g. \code{1}).
\end{datadesc}

\begin{datadesc}{LongType}
The type of long integers (e.g. \code{1L}).
\end{datadesc}

\begin{datadesc}{FloatType}
The type of floating point numbers (e.g. \code{1.0}).
\end{datadesc}

\begin{datadesc}{StringType}
The type of character strings (e.g. \code{'Spam'}).
\end{datadesc}

\begin{datadesc}{TupleType}
The type of tuples (e.g. \code{(1, 2, 3, 'Spam')}).
\end{datadesc}

\begin{datadesc}{ListType}
The type of lists (e.g. \code{[0, 1, 2, 3]}).
\end{datadesc}

\begin{datadesc}{DictType}
The type of dictionaries (e.g. \code{\{'Bacon': 1, 'Ham': 0\}}).
\end{datadesc}

\begin{datadesc}{DictionaryType}
An alternative name for \code{DictType}.
\end{datadesc}

\begin{datadesc}{FunctionType}
The type of user-defined functions and lambdas.
\end{datadesc}

\begin{datadesc}{LambdaType}
	An alternative name for \code{FunctionType}.
\end{datadesc}

\begin{datadesc}{CodeType}
The type for code objects such as returned by \code{compile()}.
\end{datadesc}

\begin{datadesc}{ClassType}
The type of user-defined classes.
\end{datadesc}

\begin{datadesc}{InstanceType}
The type of instances of user-defined classes.
\end{datadesc}

\begin{datadesc}{MethodType}
The type of methods of user-defined class instances.
\end{datadesc}

\begin{datadesc}{UnboundMethodType}
An alternative name for \code{MethodType}.
\end{datadesc}

\begin{datadesc}{BuiltinFunctionType}
The type of built-in functions like \code{len} or \code{sys.exit}.
\end{datadesc}

\begin{datadesc}{BuiltinMethodType}
An alternative name for \code{BuiltinFunction}.
\end{datadesc}

\begin{datadesc}{ModuleType}
The type of modules.
\end{datadesc}

\begin{datadesc}{FileType}
The type of open file objects such as \code{sys.stdout}.
\end{datadesc}

\begin{datadesc}{XRangeType}
The type of range objects returned by \code{xrange()}.
\end{datadesc}

\begin{datadesc}{TracebackType}
The type of traceback objects such as found in \code{sys.exc_traceback}.
\end{datadesc}

\begin{datadesc}{FrameType}
The type of frame objects such as found in \code{tb.tb_frame} if
\code{tb} is a traceback object.
\end{datadesc}
		% types is already taken :-(
\section{Standard Module \sectcode{traceback}}
\stmodindex{traceback}

\renewcommand{\indexsubitem}{(in module traceback)}

This module provides a standard interface to format and print stack
traces of Python programs.  It exactly mimics the behavior of the
Python interpreter when it prints a stack trace.  This is useful when
you want to print stack traces under program control, e.g. in a
``wrapper'' around the interpreter.

The module uses traceback objects --- this is the object type
that is stored in the variables \code{sys.exc_traceback} and
\code{sys.last_traceback}.

The module defines the following functions:

\begin{funcdesc}{print_tb}{traceback\optional{\, limit}}
Print up to \var{limit} stack trace entries from \var{traceback}.  If
\var{limit} is omitted or \code{None}, all entries are printed.
\end{funcdesc}

\begin{funcdesc}{extract_tb}{traceback\optional{\, limit}}
Return a list of up to \var{limit} ``pre-processed'' stack trace
entries extracted from \var{traceback}.  It is useful for alternate
formatting of stack traces.  If \var{limit} is omitted or \code{None},
all entries are extracted.  A ``pre-processed'' stack trace entry is a
quadruple (\var{filename}, \var{line number}, \var{function name},
\var{line text}) representing the information that is usually printed
for a stack trace.  The \var{line text} is a string with leading and
trailing whitespace stripped; if the source is not available it is
\code{None}.
\end{funcdesc}

\begin{funcdesc}{print_exception}{type\, value\, traceback\optional{\, limit}}
Print exception information and up to \var{limit} stack trace entries
from \var{traceback}.  This differs from \code{print_tb} in the
following ways: (1) if \var{traceback} is not \code{None}, it prints a
header ``\code{Traceback (innermost last):}''; (2) it prints the
exception \var{type} and \var{value} after the stack trace; (3) if
\var{type} is \code{SyntaxError} and \var{value} has the appropriate
format, it prints the line where the syntax error occurred with a
caret indication the approximate position of the error.
\end{funcdesc}

\begin{funcdesc}{print_exc}{\optional{limit}}
This is a shorthand for \code{print_exception(sys.exc_type,}
\code{sys.exc_value,} \code{sys.exc_traceback,} \code{limit)}.
\end{funcdesc}

\begin{funcdesc}{print_last}{\optional{limit}}
This is a shorthand for \code{print_exception(sys.last_type,}
\code{sys.last_value,} \code{sys.last_traceback,} \code{limit)}.
\end{funcdesc}

\section{Standard Module \sectcode{pickle}}
\stmodindex{pickle}
\index{persistency}
\indexii{persistent}{objects}
\indexii{serializing}{objects}
\indexii{marshalling}{objects}
\indexii{flattening}{objects}
\indexii{pickling}{objects}

\renewcommand{\indexsubitem}{(in module pickle)}

The \code{pickle} module implements a basic but powerful algorithm for
``pickling'' (a.k.a.\ serializing, marshalling or flattening) nearly
arbitrary Python objects.  This is the act of converting objects to a
stream of bytes (and back: ``unpickling'').
This is a more primitive notion than
persistency --- although \code{pickle} reads and writes file objects,
it does not handle the issue of naming persistent objects, nor the
(even more complicated) area of concurrent access to persistent
objects.  The \code{pickle} module can transform a complex object into
a byte stream and it can transform the byte stream into an object with
the same internal structure.  The most obvious thing to do with these
byte streams is to write them onto a file, but it is also conceivable
to send them across a network or store them in a database.  The module
\code{shelve} provides a simple interface to pickle and unpickle
objects on ``dbm''-style database files.
\stmodindex{shelve}

Unlike the built-in module \code{marshal}, \code{pickle} handles the
following correctly:
\stmodindex{marshal}

\begin{itemize}

\item recursive objects (objects containing references to themselves)

\item object sharing (references to the same object in different places)

\item user-defined classes and their instances

\end{itemize}

The data format used by \code{pickle} is Python-specific.  This has
the advantage that there are no restrictions imposed by external
standards such as CORBA (which probably can't represent pointer
sharing or recursive objects); however it means that non-Python
programs may not be able to reconstruct pickled Python objects.

The \code{pickle} data format uses a printable \ASCII{} representation.
This is slightly more voluminous than a binary representation.
However, small integers actually take {\em less} space when
represented as minimal-size decimal strings than when represented as
32-bit binary numbers, and strings are only much longer if they
contain many control characters or 8-bit characters.  The big
advantage of using printable \ASCII{} (and of some other characteristics
of \code{pickle}'s representation) is that for debugging or recovery
purposes it is possible for a human to read the pickled file with a
standard text editor.  (I could have gone a step further and used a
notation like S-expressions, but the parser
(currently written in Python) would have been
considerably more complicated and slower, and the files would probably
have become much larger.)

The \code{pickle} module doesn't handle code objects, which the
\code{marshal} module does.  I suppose \code{pickle} could, and maybe
it should, but there's probably no great need for it right now (as
long as \code{marshal} continues to be used for reading and writing
code objects), and at least this avoids the possibility of smuggling
Trojan horses into a program.
\stmodindex{marshal}

For the benefit of persistency modules written using \code{pickle}, it
supports the notion of a reference to an object outside the pickled
data stream.  Such objects are referenced by a name, which is an
arbitrary string of printable \ASCII{} characters.  The resolution of
such names is not defined by the \code{pickle} module --- the
persistent object module will have to implement a method
\code{persistent_load}.  To write references to persistent objects,
the persistent module must define a method \code{persistent_id} which
returns either \code{None} or the persistent ID of the object.

There are some restrictions on the pickling of class instances.

First of all, the class must be defined at the top level in a module.

\renewcommand{\indexsubitem}{(pickle protocol)}

Next, it must normally be possible to create class instances by
calling the class without arguments.  Usually, this is best
accomplished by providing default values for all arguments to its
\code{__init__} method (if it has one).  If this is undesirable, the
class can define a method \code{__getinitargs__()}, which should
return a {\em tuple} containing the arguments to be passed to the
class constructor (\code{__init__()}).
\ttindex{__getinitargs__}
\ttindex{__init__}

Classes can further influence how their instances are pickled --- if the class
defines the method \code{__getstate__()}, it is called and the return
state is pickled as the contents for the instance, and if the class
defines the method \code{__setstate__()}, it is called with the
unpickled state.  (Note that these methods can also be used to
implement copying class instances.)  If there is no
\code{__getstate__()} method, the instance's \code{__dict__} is
pickled.  If there is no \code{__setstate__()} method, the pickled
object must be a dictionary and its items are assigned to the new
instance's dictionary.  (If a class defines both \code{__getstate__()}
and \code{__setstate__()}, the state object needn't be a dictionary
--- these methods can do what they want.)  This protocol is also used
by the shallow and deep copying operations defined in the \code{copy}
module.
\ttindex{__getstate__}
\ttindex{__setstate__}
\ttindex{__dict__}

Note that when class instances are pickled, their class's code and
data are not pickled along with them.  Only the instance data are
pickled.  This is done on purpose, so you can fix bugs in a class or
add methods and still load objects that were created with an earlier
version of the class.  If you plan to have long-lived objects that
will see many versions of a class, it may be worthwhile to put a version
number in the objects so that suitable conversions can be made by the
class's \code{__setstate__()} method.

When a class itself is pickled, only its name is pickled --- the class
definition is not pickled, but re-imported by the unpickling process.
Therefore, the restriction that the class must be defined at the top
level in a module applies to pickled classes as well.

\renewcommand{\indexsubitem}{(in module pickle)}

The interface can be summarized as follows.

To pickle an object \code{x} onto a file \code{f}, open for writing:

\begin{verbatim}
p = pickle.Pickler(f)
p.dump(x)
\end{verbatim}

A shorthand for this is:

\begin{verbatim}
pickle.dump(x, f)
\end{verbatim}

To unpickle an object \code{x} from a file \code{f}, open for reading:

\begin{verbatim}
u = pickle.Unpickler(f)
x = u.load()
\end{verbatim}

A shorthand is:

\begin{verbatim}
x = pickle.load(f)
\end{verbatim}

The \code{Pickler} class only calls the method \code{f.write} with a
string argument.  The \code{Unpickler} calls the methods \code{f.read}
(with an integer argument) and \code{f.readline} (without argument),
both returning a string.  It is explicitly allowed to pass non-file
objects here, as long as they have the right methods.
\ttindex{Unpickler}
\ttindex{Pickler}

The following types can be pickled:
\begin{itemize}

\item \code{None}

\item integers, long integers, floating point numbers

\item strings

\item tuples, lists and dictionaries containing only picklable objects

\item classes that are defined at the top level in a module

\item instances of such classes whose \code{__dict__} or
\code{__setstate__()} is picklable

\end{itemize}

Attempts to pickle unpicklable objects will raise the
\code{PicklingError} exception; when this happens, an unspecified
number of bytes may have been written to the file.

It is possible to make multiple calls to the \code{dump()} method of
the same \code{Pickler} instance.  These must then be matched to the
same number of calls to the \code{load()} instance of the
corresponding \code{Unpickler} instance.  If the same object is
pickled by multiple \code{dump()} calls, the \code{load()} will all
yield references to the same object.  {\em Warning}: this is intended
for pickling multiple objects without intervening modifications to the
objects or their parts.  If you modify an object and then pickle it
again using the same \code{Pickler} instance, the object is not
pickled again --- a reference to it is pickled and the
\code{Unpickler} will return the old value, not the modified one.
(There are two problems here: (a) detecting changes, and (b)
marshalling a minimal set of changes.  I have no answers.  Garbage
Collection may also become a problem here.)

Apart from the \code{Pickler} and \code{Unpickler} classes, the
module defines the following functions, and an exception:

\begin{funcdesc}{dump}{object\, file}
Write a pickled representation of \var{obect} to the open file object
\var{file}.  This is equivalent to \code{Pickler(file).dump(object)}.
\end{funcdesc}

\begin{funcdesc}{load}{file}
Read a pickled object from the open file object \var{file}.  This is
equivalent to \code{Unpickler(file).load()}.
\end{funcdesc}

\begin{funcdesc}{dumps}{object}
Return the pickled representation of the object as a string, instead
of writing it to a file.
\end{funcdesc}

\begin{funcdesc}{loads}{string}
Read a pickled object from a string instead of a file.  Characters in
the string past the pickled object's representation are ignored.
\end{funcdesc}

\begin{excdesc}{PicklingError}
This exception is raised when an unpicklable object is passed to
\code{Pickler.dump()}.
\end{excdesc}

\section{Standard Module \sectcode{shelve}}
\stmodindex{shelve}
\stmodindex{pickle}
\bimodindex{dbm}
\bimodindex{gdbm}

A ``shelf'' is a persistent, dictionary-like object.  The difference
with ``dbm'' databases is that the values (not the keys!) in a shelf
can be essentially arbitrary Python objects --- anything that the
\code{pickle} module can handle.  This includes most class instances,
recursive data types, and objects containing lots of shared
sub-objects.  The keys are ordinary strings.

To summarize the interface (\code{key} is a string, \code{data} is an
arbitrary object):

\begin{verbatim}
import shelve

d = shelve.open(filename) # open, with (g)dbm filename -- no suffix

d[key] = data   # store data at key (overwrites old data if
                # using an existing key)
data = d[key]   # retrieve data at key (raise KeyError if no
                # such key)
del d[key]      # delete data stored at key (raises KeyError
                # if no such key)
flag = d.has_key(key)   # true if the key exists
list = d.keys() # a list of all existing keys (slow!)

d.close()       # close it
\end{verbatim}

Restrictions:

\begin{itemize}

\item
The choice of which database package will be used (e.g. dbm or gdbm)
depends on which interface is available.  Therefore it isn't safe to
open the database directly using dbm.  The database is also
(unfortunately) subject to the limitations of dbm, if it is used ---
this means that (the pickled representation of) the objects stored in
the database should be fairly small, and in rare cases key collisions
may cause the database to refuse updates.

\item
Dependent on the implementation, closing a persistent dictionary may
or may not be necessary to flush changes to disk.

\item
The \code{shelve} module does not support {\em concurrent} read/write
access to shelved objects.  (Multiple simultaneous read accesses are
safe.)  When a program has a shelf open for writing, no other program
should have it open for reading or writing.  \UNIX{} file locking can
be used to solve this, but this differs across \UNIX{} versions and
requires knowledge about the database implementation used.

\end{itemize}

\section{Standard Module \sectcode{copy}}
\stmodindex{copy}
\renewcommand{\indexsubitem}{(copy function)}
\ttindex{copy}
\ttindex{deepcopy}

This module provides generic (shallow and deep) copying operations.

Interface summary:

\begin{verbatim}
import copy

x = copy.copy(y)        # make a shallow copy of y
x = copy.deepcopy(y)    # make a deep copy of y
\end{verbatim}

For module specific errors, \code{copy.error} is raised.

The difference between shallow and deep copying is only relevant for
compound objects (objects that contain other objects, like lists or
class instances):

\begin{itemize}

\item
A {\em shallow copy} constructs a new compound object and then (to the
extent possible) inserts {\em references} into it to the objects found
in the original.

\item
A {\em deep copy} constructs a new compound object and then,
recursively, inserts {\em copies} into it of the objects found in the
original.

\end{itemize}

Two problems often exist with deep copy operations that don't exist
with shallow copy operations:

\begin{itemize}

\item
Recursive objects (compound objects that, directly or indirectly,
contain a reference to themselves) may cause a recursive loop.

\item
Because deep copy copies {\em everything} it may copy too much, e.g.\
administrative data structures that should be shared even between
copies.

\end{itemize}

Python's \code{deepcopy()} operation avoids these problems by:

\begin{itemize}

\item
keeping a table of objects already copied during the current
copying pass; and

\item
letting user-defined classes override the copying operation or the
set of components copied.

\end{itemize}

This version does not copy types like module, class, function, method,
nor stack trace, stack frame, nor file, socket, window, nor array, nor
any similar types.

Classes can use the same interfaces to control copying that they use
to control pickling: they can define methods called
\code{__getinitargs__()}, \code{__getstate__()} and
\code{__setstate__()}.  See the description of module \code{pickle}
for information on these methods.
\stmodindex{pickle}
\renewcommand{\indexsubitem}{(copy protocol)}
\ttindex{__getinitargs__}
\ttindex{__getstate__}
\ttindex{__setstate__}

\section{Built-in Module \sectcode{marshal}}

\bimodindex{marshal}
This module contains functions that can read and write Python
values in a binary format.  The format is specific to Python, but
independent of machine architecture issues (e.g., you can write a
Python value to a file on a PC, transport the file to a Sun, and read
it back there).  Details of the format are undocumented on purpose;
it may change between Python versions (although it rarely does).%
\footnote{The name of this module stems from a bit of terminology used
by the designers of Modula-3 (amongst others), who use the term
``marshalling'' for shipping of data around in a self-contained form.
Strictly speaking, ``to marshal'' means to convert some data from
internal to external form (in an RPC buffer for instance) and
``unmarshalling'' for the reverse process.}

This is not a general ``persistency'' module.  For general persistency
and transfer of Python objects through RPC calls, see the modules
\code{pickle} and \code{shelve}.  The \code{marshal} module exists
mainly to support reading and writing the ``pseudo-compiled'' code for
Python modules of \samp{.pyc} files.
\stmodindex{pickle}
\stmodindex{shelve}
\obindex{code}

Not all Python object types are supported; in general, only objects
whose value is independent from a particular invocation of Python can
be written and read by this module.  The following types are supported:
\code{None}, integers, long integers, floating point numbers,
strings, tuples, lists, dictionaries, and code objects, where it
should be understood that tuples, lists and dictionaries are only
supported as long as the values contained therein are themselves
supported; and recursive lists and dictionaries should not be written
(they will cause infinite loops).

{\bf Caveat:} On machines where C's \code{long int} type has more than
32 bits (such as the DEC Alpha), it
is possible to create plain Python integers that are longer than 32
bits.  Since the current \code{marshal} module uses 32 bits to
transfer plain Python integers, such values are silently truncated.
This particularly affects the use of very long integer literals in
Python modules --- these will be accepted by the parser on such
machines, but will be silently be truncated when the module is read
from the \code{.pyc} instead.%
\footnote{A solution would be to refuse such literals in the parser,
since they are inherently non-portable.  Another solution would be to
let the \code{marshal} module raise an exception when an integer value
would be truncated.  At least one of these solutions will be
implemented in a future version.}

There are functions that read/write files as well as functions
operating on strings.

The module defines these functions:

\renewcommand{\indexsubitem}{(in module marshal)}

\begin{funcdesc}{dump}{value\, file}
  Write the value on the open file.  The value must be a supported
  type.  The file must be an open file object such as
  \code{sys.stdout} or returned by \code{open()} or
  \code{posix.popen()}.
  
  If the value has (or contains an object that has) an unsupported type,
  a \code{ValueError} exception is raised -- but garbage data will also
  be written to the file.  The object will not be properly read back by
  \code{load()}.
\end{funcdesc}

\begin{funcdesc}{load}{file}
  Read one value from the open file and return it.  If no valid value
  is read, raise \code{EOFError}, \code{ValueError} or
  \code{TypeError}.  The file must be an open file object.

  Warning: If an object containing an unsupported type was marshalled
  with \code{dump()}, \code{load()} will substitute \code{None} for the
  unmarshallable type.
\end{funcdesc}

\begin{funcdesc}{dumps}{value}
  Return the string that would be written to a file by
  \code{dump(value, file)}.  The value must be a supported type.
  Raise a \code{ValueError} exception if value has (or contains an
  object that has) an unsupported type.
\end{funcdesc}

\begin{funcdesc}{loads}{string}
  Convert the string to a value.  If no valid value is found, raise
  \code{EOFError}, \code{ValueError} or \code{TypeError}.  Extra
  characters in the string are ignored.
\end{funcdesc}

\section{Built-in Module \sectcode{imp}}
\bimodindex{imp}
\index{import}

This module provides an interface to the mechanisms used to implement
the \code{import} statement.  It defines the following constants and
functions:

\renewcommand{\indexsubitem}{(in module imp)}

\begin{funcdesc}{get_magic}{}
Return the magic string value used to recognize byte-compiled code
files (``\code{.pyc} files'').
\end{funcdesc}

\begin{funcdesc}{get_suffixes}{}
Return a list of triples, each describing a particular type of file.
Each triple has the form \code{(\var{suffix}, \var{mode},
\var{type})}, where \var{suffix} is a string to be appended to the
module name to form the filename to search for, \var{mode} is the mode
string to pass to the built-in \code{open} function to open the file
(this can be \code{'r'} for text files or \code{'rb'} for binary
files), and \var{type} is the file type, which has one of the values
\code{PY_SOURCE}, \code{PY_COMPILED} or \code{C_EXTENSION}, defined
below.  (System-dependent values may also be returned.)
\end{funcdesc}

\begin{funcdesc}{find_module}{name\, \optional{path}}
Try to find the module \var{name} on the search path \var{path}.  The
default \var{path} is \code{sys.path}.  The return value is a triple
\code{(\var{file}, \var{pathname}, \var{description})} where
\var{file} is an open file object positioned at the beginning,
\var{pathname} is the pathname of the
file found, and \var{description} is a triple as contained in the list
returned by \code{get_suffixes} describing the kind of file found.
\end{funcdesc}

\begin{funcdesc}{init_builtin}{name}
Initialize the built-in module called \var{name} and return its module
object.  If the module was already initialized, it will be initialized
{\em again}.  A few modules cannot be initialized twice --- attempting
to initialize these again will raise an \code{ImportError} exception.
If there is no
built-in module called \var{name}, \code{None} is returned.
\end{funcdesc}

\begin{funcdesc}{init_frozen}{name}
Initialize the frozen module called \var{name} and return its module
object.  If the module was already initialized, it will be initialized
{\em again}.  If there is no frozen module called \var{name},
\code{None} is returned.  (Frozen modules are modules written in
Python whose compiled byte-code object is incorporated into a
custom-built Python interpreter by Python's \code{freeze} utility.
See \code{Tools/freeze} for now.)
\end{funcdesc}

\begin{funcdesc}{is_builtin}{name}
Return \code{1} if there is a built-in module called \var{name} which can be
initialized again.  Return \code{-1} if there is a built-in module
called \var{name} which cannot be initialized again (see
\code{init_builtin}).  Return \code{0} if there is no built-in module
called \var{name}.
\end{funcdesc}

\begin{funcdesc}{is_frozen}{name}
Return \code{1} if there is a frozen module (see \code{init_frozen})
called \var{name}, \code{0} if there is no such module.
\end{funcdesc}

\begin{funcdesc}{load_compiled}{name\, pathname\, file}
Load and initialize a module implemented as a byte-compiled code file
and return its module object.  If the module was already initialized,
it will be initialized {\em again}.  The \var{name} argument is used
to create or access a module object.  The \var{pathname} argument
points to the byte-compiled code file.  The \var{file}
argument is the byte-compiled code file, open for reading in binary
mode, from the beginning.
It must currently be a real file object, not a
user-defined class emulating a file.
\end{funcdesc}

\begin{funcdesc}{load_dynamic}{name\, pathname\, \optional{file}}
Load and initialize a module implemented as a dynamically loadable
shared library and return its module object.  If the module was
already initialized, it will be initialized {\em again}.  Some modules
don't like that and may raise an exception.  The \var{pathname}
argument must point to the shared library.  The \var{name} argument is
used to construct the name of the initialization function: an external
C function called \code{init\var{name}()} in the shared library is
called.  The optional \var{file} argment is ignored.  (Note: using
shared libraries is highly system dependent, and not all systems
support it.)
\end{funcdesc}

\begin{funcdesc}{load_source}{name\, pathname\, file}
Load and initialize a module implemented as a Python source file and
return its module object.  If the module was already initialized, it
will be initialized {\em again}.  The \var{name} argument is used to
create or access a module object.  The \var{pathname} argument points
to the source file.  The \var{file} argument is the source
file, open for reading as text, from the beginning.
It must currently be a real file
object, not a user-defined class emulating a file.  Note that if a
properly matching byte-compiled file (with suffix \code{.pyc}) exists,
it will be used instead of parsing the given source file.
\end{funcdesc}

\begin{funcdesc}{new_module}{name}
Return a new empty module object called \var{name}.  This object is
{\em not} inserted in \code{sys.modules}.
\end{funcdesc}

The following constants with integer values, defined in the module,
are used to indicate the search result of \code{imp.find_module}.

\begin{datadesc}{SEARCH_ERROR}
The module was not found.
\end{datadesc}

\begin{datadesc}{PY_SOURCE}
The module was found as a source file.
\end{datadesc}

\begin{datadesc}{PY_COMPILED}
The module was found as a compiled code object file.
\end{datadesc}

\begin{datadesc}{C_EXTENSION}
The module was found as dynamically loadable shared library.
\end{datadesc}

\subsection{Examples}
The following function emulates the default import statement:

\begin{verbatim}
import imp
import sys

def __import__(name, globals=None, locals=None, fromlist=None):
    # Fast path: see if the module has already been imported.
    if sys.modules.has_key(name):
        return sys.modules[name]

    # If any of the following calls raises an exception,
    # there's a problem we can't handle -- let the caller handle it.

    # See if it's a built-in module.
    m = imp.init_builtin(name)
    if m:
        return m

    # See if it's a frozen module.
    m = imp.init_frozen(name)
    if m:
        return m

    # Search the default path (i.e. sys.path).
    fp, pathname, (suffix, mode, type) = imp.find_module(name)

    # See what we got.
    try:
        if type == imp.C_EXTENSION:
            return imp.load_dynamic(name, pathname)
        if type == imp.PY_SOURCE:
            return imp.load_source(name, pathname, fp)
        if type == imp.PY_COMPILED:
            return imp.load_compiled(name, pathname, fp)

        # Shouldn't get here at all.
        raise ImportError, '%s: unknown module type (%d)' % (name, type)
    finally:
        # Since we may exit via an exception, close fp explicitly.
        fp.close()
\end{verbatim}

% libparser.tex
%
% Introductory documentation for the new parser built-in module.
%
% Copyright 1995 Virginia Polytechnic Institute and State University
% and Fred L. Drake, Jr.  This copyright notice must be distributed on
% all copies, but this document otherwise may be distributed as part
% of the Python distribution.  No fee may be charged for this document
% in any representation, either on paper or electronically.  This
% restriction does not affect other elements in a distributed package
% in any way.
%

\section{Built-in Module \sectcode{parser}}
\bimodindex{parser}

The \code{parser} module provides an interface to Python's internal
parser and byte-code compiler.  The primary purpose for this interface
is to allow Python code to edit the parse tree of a Python expression
and create executable code from this.  This is better than trying
to parse and modify an arbitrary Python code fragment as a string
because parsing is performed in a manner identical to the code
forming the application.  It is also faster.

There are a few things to note about this module which are important
to making use of the data structures created.  This is not a tutorial
on editing the parse trees for Python code, but some examples of using
the \code{parser} module are presented.

Most importantly, a good understanding of the Python grammar processed
by the internal parser is required.  For full information on the
language syntax, refer to the Language Reference.  The parser itself
is created from a grammar specification defined in the file
\file{Grammar/Grammar} in the standard Python distribution.  The parse
trees stored in the ``AST objects'' created by this module are the
actual output from the internal parser when created by the
\code{expr()} or \code{suite()} functions, described below.  The AST
objects created by \code{sequence2ast()} faithfully simulate those
structures.  Be aware that the values of the sequences which are
considered ``correct'' will vary from one version of Python to another
as the formal grammar for the language is revised.  However,
transporting code from one Python version to another as source text
will always allow correct parse trees to be created in the target
version, with the only restriction being that migrating to an older
version of the interpreter will not support more recent language
constructs.  The parse trees are not typically compatible from one
version to another, whereas source code has always been
forward-compatible.

Each element of the sequences returned by \code{ast2list} or
\code{ast2tuple()} has a simple form.  Sequences representing
non-terminal elements in the grammar always have a length greater than
one.  The first element is an integer which identifies a production in
the grammar.  These integers are given symbolic names in the C header
file \file{Include/graminit.h} and the Python module
\file{Lib/symbol.py}.  Each additional element of the sequence represents
a component of the production as recognized in the input string: these
are always sequences which have the same form as the parent.  An
important aspect of this structure which should be noted is that
keywords used to identify the parent node type, such as the keyword
\code{if} in an \code{if_stmt}, are included in the node tree without
any special treatment.  For example, the \code{if} keyword is
represented by the tuple \code{(1, 'if')}, where \code{1} is the
numeric value associated with all \code{NAME} tokens, including
variable and function names defined by the user.  In an alternate form
returned when line number information is requested, the same token
might be represented as \code{(1, 'if', 12)}, where the \code{12}
represents the line number at which the terminal symbol was found.

Terminal elements are represented in much the same way, but without
any child elements and the addition of the source text which was
identified.  The example of the \code{if} keyword above is
representative.  The various types of terminal symbols are defined in
the C header file \file{Include/token.h} and the Python module
\file{Lib/token.py}.

The AST objects are not required to support the functionality of this
module, but are provided for three purposes: to allow an application
to amortize the cost of processing complex parse trees, to provide a
parse tree representation which conserves memory space when compared
to the Python list or tuple representation, and to ease the creation
of additional modules in C which manipulate parse trees.  A simple
``wrapper'' class may be created in Python to hide the use of AST
objects; the \code{AST} library module provides a variety of such
classes.

The \code{parser} module defines functions for a few distinct
purposes.  The most important purposes are to create AST objects and
to convert AST objects to other representations such as parse trees
and compiled code objects, but there are also functions which serve to
query the type of parse tree represented by an AST object.

\renewcommand{\indexsubitem}{(in module parser)}


\subsection{Creating AST Objects}

AST objects may be created from source code or from a parse tree.
When creating an AST object from source, different functions are used
to create the \code{'eval'} and \code{'exec'} forms.

\begin{funcdesc}{expr}{string}
The \code{expr()} function parses the parameter \code{\var{string}}
as if it were an input to \code{compile(\var{string}, 'eval')}.  If
the parse succeeds, an AST object is created to hold the internal
parse tree representation, otherwise an appropriate exception is
thrown.
\end{funcdesc}

\begin{funcdesc}{suite}{string}
The \code{suite()} function parses the parameter \code{\var{string}}
as if it were an input to \code{compile(\var{string}, 'exec')}.  If
the parse succeeds, an AST object is created to hold the internal
parse tree representation, otherwise an appropriate exception is
thrown.
\end{funcdesc}

\begin{funcdesc}{sequence2ast}{sequence}
This function accepts a parse tree represented as a sequence and
builds an internal representation if possible.  If it can validate
that the tree conforms to the Python grammar and all nodes are valid
node types in the host version of Python, an AST object is created
from the internal representation and returned to the called.  If there
is a problem creating the internal representation, or if the tree
cannot be validated, a \code{ParserError} exception is thrown.  An AST
object created this way should not be assumed to compile correctly;
normal exceptions thrown by compilation may still be initiated when
the AST object is passed to \code{compileast()}.  This may indicate
problems not related to syntax (such as a \code{MemoryError}
exception), but may also be due to constructs such as the result of
parsing \code{del f(0)}, which escapes the Python parser but is
checked by the bytecode compiler.

Sequences representing terminal tokens may be represented as either
two-element lists of the form \code{(1, 'name')} or as three-element
lists of the form \code{(1, 'name', 56)}.  If the third element is
present, it is assumed to be a valid line number.  The line number
may be specified for any subset of the terminal symbols in the input
tree.
\end{funcdesc}

\begin{funcdesc}{tuple2ast}{sequence}
This is the same function as \code{sequence2ast()}.  This entry point
is maintained for backward compatibility.
\end{funcdesc}


\subsection{Converting AST Objects}

AST objects, regardless of the input used to create them, may be
converted to parse trees represented as list- or tuple- trees, or may
be compiled into executable code objects.  Parse trees may be
extracted with or without line numbering information.

\begin{funcdesc}{ast2list}{ast\optional{\, line_info\code{ = 0}}}
This function accepts an AST object from the caller in
\code{\var{ast}} and returns a Python list representing the
equivelent parse tree.  The resulting list representation can be used
for inspection or the creation of a new parse tree in list form.  This
function does not fail so long as memory is available to build the
list representation.  If the parse tree will only be used for
inspection, \code{ast2tuple()} should be used instead to reduce memory
consumption and fragmentation.  When the list representation is
required, this function is significantly faster than retrieving a
tuple representation and converting that to nested lists.

If \code{\var{line_info}} is true, line number information will be
included for all terminal tokens as a third element of the list
representing the token.  This information is omitted if the flag is
false or omitted.
\end{funcdesc}

\begin{funcdesc}{ast2tuple}{ast\optional{\, line_info\code{ = 0}}}
This function accepts an AST object from the caller in
\code{\var{ast}} and returns a Python tuple representing the
equivelent parse tree.  Other than returning a tuple instead of a
list, this function is identical to \code{ast2list()}.

If \code{\var{line_info}} is true, line number information will be
included for all terminal tokens as a third element of the list
representing the token.  This information is omitted if the flag is
false or omitted.
\end{funcdesc}

\begin{funcdesc}{compileast}{ast\optional{\, filename\code{ = '<ast>'}}}
The Python byte compiler can be invoked on an AST object to produce
code objects which can be used as part of an \code{exec} statement or
a call to the built-in \code{eval()} function.  This function provides
the interface to the compiler, passing the internal parse tree from
\code{\var{ast}} to the parser, using the source file name specified
by the \code{\var{filename}} parameter.  The default value supplied
for \code{\var{filename}} indicates that the source was an AST object.

Compiling an AST object may result in exceptions related to
compilation; an example would be a \code{SyntaxError} caused by the
parse tree for \code{del f(0)}: this statement is considered legal
within the formal grammar for Python but is not a legal language
construct.  The \code{SyntaxError} raised for this condition is
actually generated by the Python byte-compiler normally, which is why
it can be raised at this point by the \code{parser} module.  Most
causes of compilation failure can be diagnosed programmatically by
inspection of the parse tree.
\end{funcdesc}


\subsection{Queries on AST Objects}

Two functions are provided which allow an application to determine if
an AST was create as an expression or a suite.  Neither of these
functions can be used to determine if an AST was created from source
code via \code{expr()} or \code{suite()} or from a parse tree via
\code{sequence2ast()}.

\begin{funcdesc}{isexpr}{ast}
When \code{\var{ast}} represents an \code{'eval'} form, this function
returns a true value (\code{1}), otherwise it returns false
(\code{0}).  This is useful, since code objects normally cannot be
queried for this information using existing built-in functions.  Note
that the code objects created by \code{compileast()} cannot be queried
like this either, and are identical to those created by the built-in
\code{compile()} function.
\end{funcdesc}


\begin{funcdesc}{issuite}{ast}
This function mirrors \code{isexpr()} in that it reports whether an
AST object represents an \code{'exec'} form, commonly known as a
``suite.''  It is not safe to assume that this function is equivelent
to \code{not isexpr(\var{ast})}, as additional syntactic fragments may
be supported in the future.
\end{funcdesc}


\subsection{Exceptions and Error Handling}

The parser module defines a single exception, but may also pass other
built-in exceptions from other portions of the Python runtime
environment.  See each function for information about the exceptions
it can raise.

\begin{excdesc}{ParserError}
Exception raised when a failure occurs within the parser module.  This
is generally produced for validation failures rather than the built in
\code{SyntaxError} thrown during normal parsing.
The exception argument is either a string describing the reason of the
failure or a tuple containing a sequence causing the failure from a parse
tree passed to \code{sequence2ast()} and an explanatory string.  Calls to
\code{sequence2ast()} need to be able to handle either type of exception,
while calls to other functions in the module will only need to be
aware of the simple string values.
\end{excdesc}

Note that the functions \code{compileast()}, \code{expr()}, and
\code{suite()} may throw exceptions which are normally thrown by the
parsing and compilation process.  These include the built in
exceptions \code{MemoryError}, \code{OverflowError},
\code{SyntaxError}, and \code{SystemError}.  In these cases, these
exceptions carry all the meaning normally associated with them.  Refer
to the descriptions of each function for detailed information.


\subsection{AST Objects}

AST objects returned by \code{expr()}, \code{suite()}, and
\code{sequence2ast()} have no methods of their own.
Some of the functions defined which accept an AST object as their
first argument may change to object methods in the future.  The type
of these objects is available as \code{ASTType} in the module.

Ordered and equality comparisons are supported between AST objects.


\subsection{Examples}

The parser modules allows operations to be performed on the parse tree
of Python source code before the bytecode is generated, and provides
for inspection of the parse tree for information gathering purposes.
Two examples are presented.  The simple example demonstrates emulation
of the \code{compile()} built-in function and the complex example
shows the use of a parse tree for information discovery.

\subsubsection{Emulation of \sectcode{compile()}}

While many useful operations may take place between parsing and
bytecode generation, the simplest operation is to do nothing.  For
this purpose, using the \code{parser} module to produce an
intermediate data structure is equivelent to the code

\begin{verbatim}
>>> code = compile('a + 5', 'eval')
>>> a = 5
>>> eval(code)
10
\end{verbatim}

The equivelent operation using the \code{parser} module is somewhat
longer, and allows the intermediate internal parse tree to be retained
as an AST object:

\begin{verbatim}
>>> import parser
>>> ast = parser.expr('a + 5')
>>> code = parser.compileast(ast)
>>> a = 5
>>> eval(code)
10
\end{verbatim}

An application which needs both AST and code objects can package this
code into readily available functions:

\begin{verbatim}
import parser

def load_suite(source_string):
    ast = parser.suite(source_string)
    code = parser.compileast(ast)
    return ast, code

def load_expression(source_string):
    ast = parser.expr(source_string)
    code = parser.compileast(ast)
    return ast, code
\end{verbatim}

\subsubsection{Information Discovery}

Some applications benefit from direct access to the parse tree.  The
remainder of this section demonstrates how the parse tree provides
access to module documentation defined in docstrings without requiring
that the code being examined be loaded into a running interpreter via
\code{import}.  This can be very useful for performing analyses of
untrusted code.

Generally, the example will demonstrate how the parse tree may be
traversed to distill interesting information.  Two functions and a set
of classes are developed which provide programmatic access to high
level function and class definitions provided by a module.  The
classes extract information from the parse tree and provide access to
the information at a useful semantic level, one function provides a
simple low-level pattern matching capability, and the other function
defines a high-level interface to the classes by handling file
operations on behalf of the caller.  All source files mentioned here
which are not part of the Python installation are located in the
\file{Demo/parser/} directory of the distribution.

The dynamic nature of Python allows the programmer a great deal of
flexibility, but most modules need only a limited measure of this when
defining classes, functions, and methods.  In this example, the only
definitions that will be considered are those which are defined in the
top level of their context, e.g., a function defined by a \code{def}
statement at column zero of a module, but not a function defined
within a branch of an \code{if} ... \code{else} construct, though
there are some good reasons for doing so in some situations.  Nesting
of definitions will be handled by the code developed in the example.

To construct the upper-level extraction methods, we need to know what
the parse tree structure looks like and how much of it we actually
need to be concerned about.  Python uses a moderately deep parse tree
so there are a large number of intermediate nodes.  It is important to
read and understand the formal grammar used by Python.  This is
specified in the file \file{Grammar/Grammar} in the distribution.
Consider the simplest case of interest when searching for docstrings:
a module consisting of a docstring and nothing else.  (See file
\file{docstring.py}.)

\begin{verbatim}
"""Some documentation.
"""
\end{verbatim}

Using the interpreter to take a look at the parse tree, we find a
bewildering mass of numbers and parentheses, with the documentation
buried deep in nested tuples.

\begin{verbatim}
>>> import parser
>>> import pprint
>>> ast = parser.suite(open('docstring.py').read())
>>> tup = parser.ast2tuple(ast)
>>> pprint.pprint(tup)
(257,
 (264,
  (265,
   (266,
    (267,
     (307,
      (287,
       (288,
        (289,
         (290,
          (292,
           (293,
            (294,
             (295,
              (296,
               (297,
                (298,
                 (299,
                  (300, (3, '"""Some documentation.\012"""'))))))))))))))))),
   (4, ''))),
 (4, ''),
 (0, ''))
\end{verbatim}

The numbers at the first element of each node in the tree are the node
types; they map directly to terminal and non-terminal symbols in the
grammar.  Unfortunately, they are represented as integers in the
internal representation, and the Python structures generated do not
change that.  However, the \code{symbol} and \code{token} modules
provide symbolic names for the node types and dictionaries which map
from the integers to the symbolic names for the node types.

In the output presented above, the outermost tuple contains four
elements: the integer \code{257} and three additional tuples.  Node
type \code{257} has the symbolic name \code{file_input}.  Each of
these inner tuples contains an integer as the first element; these
integers, \code{264}, \code{4}, and \code{0}, represent the node types
\code{stmt}, \code{NEWLINE}, and \code{ENDMARKER}, respectively.
Note that these values may change depending on the version of Python
you are using; consult \file{symbol.py} and \file{token.py} for
details of the mapping.  It should be fairly clear that the outermost
node is related primarily to the input source rather than the contents
of the file, and may be disregarded for the moment.  The \code{stmt}
node is much more interesting.  In particular, all docstrings are
found in subtrees which are formed exactly as this node is formed,
with the only difference being the string itself.  The association
between the docstring in a similar tree and the defined entity (class,
function, or module) which it describes is given by the position of
the docstring subtree within the tree defining the described
structure.

By replacing the actual docstring with something to signify a variable
component of the tree, we allow a simple pattern matching approach to
check any given subtree for equivelence to the general pattern for
docstrings.  Since the example demonstrates information extraction, we
can safely require that the tree be in tuple form rather than list
form, allowing a simple variable representation to be
\code{['variable_name']}.  A simple recursive function can implement
the pattern matching, returning a boolean and a dictionary of variable
name to value mappings.  (See file \file{example.py}.)

\begin{verbatim}
from types import ListType, TupleType

def match(pattern, data, vars=None):
    if vars is None:
        vars = {}
    if type(pattern) is ListType:
        vars[pattern[0]] = data
        return 1, vars
    if type(pattern) is not TupleType:
        return (pattern == data), vars
    if len(data) != len(pattern):
        return 0, vars
    for pattern, data in map(None, pattern, data):
        same, vars = match(pattern, data, vars)
        if not same:
            break
    return same, vars
\end{verbatim}

Using this simple representation for syntactic variables and the symbolic
node types, the pattern for the candidate docstring subtrees becomes
fairly readable.  (See file \file{example.py}.)

\begin{verbatim}
import symbol
import token

DOCSTRING_STMT_PATTERN = (
    symbol.stmt,
    (symbol.simple_stmt,
     (symbol.small_stmt,
      (symbol.expr_stmt,
       (symbol.testlist,
        (symbol.test,
         (symbol.and_test,
          (symbol.not_test,
           (symbol.comparison,
            (symbol.expr,
             (symbol.xor_expr,
              (symbol.and_expr,
               (symbol.shift_expr,
                (symbol.arith_expr,
                 (symbol.term,
                  (symbol.factor,
                   (symbol.power,
                    (symbol.atom,
                     (token.STRING, ['docstring'])
                     )))))))))))))))),
     (token.NEWLINE, '')
     ))
\end{verbatim}

Using the \code{match()} function with this pattern, extracting the
module docstring from the parse tree created previously is easy:

\begin{verbatim}
>>> found, vars = match(DOCSTRING_STMT_PATTERN, tup[1])
>>> found
1
>>> vars
{'docstring': '"""Some documentation.\012"""'}
\end{verbatim}

Once specific data can be extracted from a location where it is
expected, the question of where information can be expected
needs to be answered.  When dealing with docstrings, the answer is
fairly simple: the docstring is the first \code{stmt} node in a code
block (\code{file_input} or \code{suite} node types).  A module
consists of a single \code{file_input} node, and class and function
definitions each contain exactly one \code{suite} node.  Classes and
functions are readily identified as subtrees of code block nodes which
start with \code{(stmt, (compound_stmt, (classdef, ...} or
\code{(stmt, (compound_stmt, (funcdef, ...}.  Note that these subtrees
cannot be matched by \code{match()} since it does not support multiple
sibling nodes to match without regard to number.  A more elaborate
matching function could be used to overcome this limitation, but this
is sufficient for the example.

Given the ability to determine whether a statement might be a
docstring and extract the actual string from the statement, some work
needs to be performed to walk the parse tree for an entire module and
extract information about the names defined in each context of the
module and associate any docstrings with the names.  The code to
perform this work is not complicated, but bears some explanation.

The public interface to the classes is straightforward and should
probably be somewhat more flexible.  Each ``major'' block of the
module is described by an object providing several methods for inquiry
and a constructor which accepts at least the subtree of the complete
parse tree which it represents.  The \code{ModuleInfo} constructor
accepts an optional \code{\var{name}} parameter since it cannot
otherwise determine the name of the module.

The public classes include \code{ClassInfo}, \code{FunctionInfo},
and \code{ModuleInfo}.  All objects provide the
methods \code{get_name()}, \code{get_docstring()},
\code{get_class_names()}, and \code{get_class_info()}.  The
\code{ClassInfo} objects support \code{get_method_names()} and
\code{get_method_info()} while the other classes provide
\code{get_function_names()} and \code{get_function_info()}.

Within each of the forms of code block that the public classes
represent, most of the required information is in the same form and is
accessed in the same way, with classes having the distinction that
functions defined at the top level are referred to as ``methods.''
Since the difference in nomenclature reflects a real semantic
distinction from functions defined outside of a class, the
implementation needs to maintain the distinction.
Hence, most of the functionality of the public classes can be
implemented in a common base class, \code{SuiteInfoBase}, with the
accessors for function and method information provided elsewhere.
Note that there is only one class which represents function and method
information; this paralels the use of the \code{def} statement to
define both types of elements.

Most of the accessor functions are declared in \code{SuiteInfoBase}
and do not need to be overriden by subclasses.  More importantly, the
extraction of most information from a parse tree is handled through a
method called by the \code{SuiteInfoBase} constructor.  The example
code for most of the classes is clear when read alongside the formal
grammar, but the method which recursively creates new information
objects requires further examination.  Here is the relevant part of
the \code{SuiteInfoBase} definition from \file{example.py}:

\begin{verbatim}
class SuiteInfoBase:
    _docstring = ''
    _name = ''

    def __init__(self, tree = None):
        self._class_info = {}
        self._function_info = {}
        if tree:
            self._extract_info(tree)

    def _extract_info(self, tree):
        # extract docstring
        if len(tree) == 2:
            found, vars = match(DOCSTRING_STMT_PATTERN[1], tree[1])
        else:
            found, vars = match(DOCSTRING_STMT_PATTERN, tree[3])
        if found:
            self._docstring = eval(vars['docstring'])
        # discover inner definitions
        for node in tree[1:]:
            found, vars = match(COMPOUND_STMT_PATTERN, node)
            if found:
                cstmt = vars['compound']
                if cstmt[0] == symbol.funcdef:
                    name = cstmt[2][1]
                    self._function_info[name] = FunctionInfo(cstmt)
                elif cstmt[0] == symbol.classdef:
                    name = cstmt[2][1]
                    self._class_info[name] = ClassInfo(cstmt)
\end{verbatim}

After initializing some internal state, the constructor calls the
\code{_extract_info()} method.  This method performs the bulk of the
information extraction which takes place in the entire example.  The
extraction has two distinct phases: the location of the docstring for
the parse tree passed in, and the discovery of additional definitions
within the code block represented by the parse tree.

The initial \code{if} test determines whether the nested suite is of
the ``short form'' or the ``long form.''  The short form is used when
the code block is on the same line as the definition of the code
block, as in

\begin{verbatim}
def square(x): "Square an argument."; return x ** 2
\end{verbatim}

while the long form uses an indented block and allows nested
definitions:

\begin{verbatim}
def make_power(exp):
    "Make a function that raises an argument to the exponent `exp'."
    def raiser(x, y=exp):
        return x ** y
    return raiser
\end{verbatim}

When the short form is used, the code block may contain a docstring as
the first, and possibly only, \code{small_stmt} element.  The
extraction of such a docstring is slightly different and requires only
a portion of the complete pattern used in the more common case.  As
implemented, the docstring will only be found if there is only
one \code{small_stmt} node in the \code{simple_stmt} node.  Since most
functions and methods which use the short form do not provide a
docstring, this may be considered sufficient.  The extraction of the
docstring proceeds using the \code{match()} function as described
above, and the value of the docstring is stored as an attribute of the
\code{SuiteInfoBase} object.

After docstring extraction, a simple definition discovery
algorithm operates on the \code{stmt} nodes of the \code{suite} node.  The
special case of the short form is not tested; since there are no
\code{stmt} nodes in the short form, the algorithm will silently skip
the single \code{simple_stmt} node and correctly not discover any
nested definitions.

Each statement in the code block is categorized as
a class definition, function or method definition, or
something else.  For the definition statements, the name of the
element defined is extracted and a representation object
appropriate to the definition is created with the defining subtree
passed as an argument to the constructor.  The repesentation objects
are stored in instance variables and may be retrieved by name using
the appropriate accessor methods.

The public classes provide any accessors required which are more
specific than those provided by the \code{SuiteInfoBase} class, but
the real extraction algorithm remains common to all forms of code
blocks.  A high-level function can be used to extract the complete set
of information from a source file.  (See file \file{example.py}.)

\begin{verbatim}
def get_docs(fileName):
    source = open(fileName).read()
    import os
    basename = os.path.basename(os.path.splitext(fileName)[0])
    import parser
    ast = parser.suite(source)
    tup = parser.ast2tuple(ast)
    return ModuleInfo(tup, basename)
\end{verbatim}

This provides an easy-to-use interface to the documentation of a
module.  If information is required which is not extracted by the code
of this example, the code may be extended at clearly defined points to
provide additional capabilities.


%%
%%  end of file

\section{Built-in Module \sectcode{__builtin__}}
\bimodindex{__builtin__}

This module provides direct access to all `built-in' identifiers of
Python; e.g. \code{__builtin__.open} is the full name for the built-in
function \code{open}.  See the section on Built-in Functions in the
previous chapter.
		% really __builtin__
\section{Built-in Module \sectcode{__main__}}

\bimodindex{__main__}
This module represents the (otherwise anonymous) scope in which the
interpreter's main program executes --- commands read either from
standard input or from a script file.
			% really __main__

\chapter{String Services}

The modules described in this chapter provide a wide range of string
manipulation operations.  Here's an overview:

\begin{description}

\item[string]
--- Common string operations.

\item[regex]
--- Regular expression search and match operations.

\item[regsub]
--- Substitution and splitting operations that use regular expressions.

\item[struct]
--- Interpret strings as packed binary data.

\end{description}
		% String Services
\section{Standard Module \sectcode{string}}

\stmodindex{string}

This module defines some constants useful for checking character
classes and some useful string functions.  See the modules
\code{regex} and \code{regsub} for string functions based on regular
expressions.

The constants defined in this module are are:

\renewcommand{\indexsubitem}{(data in module string)}
\begin{datadesc}{digits}
  The string \code{'0123456789'}.
\end{datadesc}

\begin{datadesc}{hexdigits}
  The string \code{'0123456789abcdefABCDEF'}.
\end{datadesc}

\begin{datadesc}{letters}
  The concatenation of the strings \code{lowercase} and
  \code{uppercase} described below.
\end{datadesc}

\begin{datadesc}{lowercase}
  A string containing all the characters that are considered lowercase
  letters.  On most systems this is the string
  \code{'abcdefghijklmnopqrstuvwxyz'}.  Do not change its definition ---
  the effect on the routines \code{upper} and \code{swapcase} is
  undefined.
\end{datadesc}

\begin{datadesc}{octdigits}
  The string \code{'01234567'}.
\end{datadesc}

\begin{datadesc}{uppercase}
  A string containing all the characters that are considered uppercase
  letters.  On most systems this is the string
  \code{'ABCDEFGHIJKLMNOPQRSTUVWXYZ'}.  Do not change its definition ---
  the effect on the routines \code{lower} and \code{swapcase} is
  undefined.
\end{datadesc}

\begin{datadesc}{whitespace}
  A string containing all characters that are considered whitespace.
  On most systems this includes the characters space, tab, linefeed,
  return, formfeed, and vertical tab.  Do not change its definition ---
  the effect on the routines \code{strip} and \code{split} is
  undefined.
\end{datadesc}

The functions defined in this module are:

\renewcommand{\indexsubitem}{(in module string)}

\begin{funcdesc}{atof}{s}
Convert a string to a floating point number.  The string must have
the standard syntax for a floating point literal in Python, optionally
preceded by a sign (\samp{+} or \samp{-}).
\end{funcdesc}

\begin{funcdesc}{atoi}{s\optional{\, base}}
Convert string \var{s} to an integer in the given \var{base}.  The
string must consist of one or more digits, optionally preceded by a
sign (\samp{+} or \samp{-}).  The \var{base} defaults to 10.  If it is
0, a default base is chosen depending on the leading characters of the
string (after stripping the sign): \samp{0x} or \samp{0X} means 16,
\samp{0} means 8, anything else means 10.  If \var{base} is 16, a
leading \samp{0x} or \samp{0X} is always accepted.  (Note: for a more
flexible interpretation of numeric literals, use the built-in function
\code{eval()}.)
\bifuncindex{eval}
\end{funcdesc}

\begin{funcdesc}{atol}{s\optional{\, base}}
Convert string \var{s} to a long integer in the given \var{base}.  The
string must consist of one or more digits, optionally preceded by a
sign (\samp{+} or \samp{-}).  The \var{base} argument has the same
meaning as for \code{atoi()}.  A trailing \samp{l} or \samp{L} is not
allowed, except if the base is 0.
\end{funcdesc}

\begin{funcdesc}{capitalize}{word}
Capitalize the first character of the argument.
\end{funcdesc}

\begin{funcdesc}{capwords}{s}
Split the argument into words using \code{split}, capitalize each word
using \code{capitalize}, and join the capitalized words using
\code{join}.  Note that this replaces runs of whitespace characters by
a single space.  (See also \code{regsub.capwords()} for a version
that doesn't change the delimiters, and lets you specify a word
separator.)
\end{funcdesc}

\begin{funcdesc}{expandtabs}{s\, tabsize}
Expand tabs in a string, i.e.\ replace them by one or more spaces,
depending on the current column and the given tab size.  The column
number is reset to zero after each newline occurring in the string.
This doesn't understand other non-printing characters or escape
sequences.
\end{funcdesc}

\begin{funcdesc}{find}{s\, sub\optional{\, start}}
Return the lowest index in \var{s} not smaller than \var{start} where the
substring \var{sub} is found.  Return \code{-1} when \var{sub}
does not occur as a substring of \var{s} with index at least \var{start}.
If \var{start} is omitted, it defaults to \code{0}.  If \var{start} is
negative, \code{len(\var{s})} is added.
\end{funcdesc}

\begin{funcdesc}{rfind}{s\, sub\optional{\, start}}
Like \code{find} but find the highest index.
\end{funcdesc}

\begin{funcdesc}{index}{s\, sub\optional{\, start}}
Like \code{find} but raise \code{ValueError} when the substring is
not found.
\end{funcdesc}

\begin{funcdesc}{rindex}{s\, sub\optional{\, start}}
Like \code{rfind} but raise \code{ValueError} when the substring is
not found.
\end{funcdesc}

\begin{funcdesc}{count}{s\, sub\optional{\, start}}
Return the number of (non-overlapping) occurrences of substring
\var{sub} in string \var{s} with index at least \var{start}.
If \var{start} is omitted, it defaults to \code{0}.  If \var{start} is
negative, \code{len(\var{s})} is added.
\end{funcdesc}

\begin{funcdesc}{lower}{s}
Convert letters to lower case.
\end{funcdesc}

\begin{funcdesc}{maketrans}{from, to}
Return a translation table suitable for passing to \code{string.translate}
or \code{regex.compile}, that will map each character in \var{from} 
into the character at the same position in \var{to}; \var{from} and
\var{to} must have the same length. 
\end{funcdesc}

\begin{funcdesc}{split}{s\optional{\, sep\optional{\, maxsplit}}}
Return a list of the words of the string \var{s}.  If the optional
second argument \var{sep} is absent or \code{None}, the words are
separated by arbitrary strings of whitespace characters (space, tab,
newline, return, formfeed).  If the second argument \var{sep} is
present and not \code{None}, it specifies a string to be used as the
word separator.  The returned list will then have one more items than
the number of non-overlapping occurrences of the separator in the
string.  The optional third argument \var{maxsplit} defaults to 0.  If
it is nonzero, at most \var{maxsplit} number of splits occur, and the
remainder of the string is returned as the final element of the list
(thus, the list will have at most \code{\var{maxsplit}+1} elements).
(See also \code{regsub.split()} for a version that allows specifying a
regular expression as the separator.)
\end{funcdesc}

\begin{funcdesc}{splitfields}{s\optional{\, sep\optional{\, maxsplit}}}
This function behaves identical to \code{split}.  (In the past,
\code{split} was only used with one argument, while \code{splitfields}
was only used with two arguments.)
\end{funcdesc}

\begin{funcdesc}{join}{words\optional{\, sep}}
Concatenate a list or tuple of words with intervening occurrences of
\var{sep}.  The default value for \var{sep} is a single space character.
It is always true that
\code{string.join(string.split(\var{s}, \var{sep}), \var{sep})}
equals \var{s}.
\end{funcdesc}

\begin{funcdesc}{joinfields}{words\optional{\, sep}}
This function behaves identical to \code{join}.  (In the past,
\code{join} was only used with one argument, while \code{joinfields}
was only used with two arguments.)
\end{funcdesc}

\begin{funcdesc}{lstrip}{s}
Remove leading whitespace from the string \var{s}.
\end{funcdesc}

\begin{funcdesc}{rstrip}{s}
Remove trailing whitespace from the string \var{s}.
\end{funcdesc}

\begin{funcdesc}{strip}{s}
Remove leading and trailing whitespace from the string \var{s}.
\end{funcdesc}

\begin{funcdesc}{swapcase}{s}
Convert lower case letters to upper case and vice versa.
\end{funcdesc}

\begin{funcdesc}{translate}{s, table\optional{, deletechars}}
Delete all characters from \var{s} that are in \var{deletechars} (if present), and 
then translate the characters using \var{table}, which must be
a 256-character string giving the translation for each character
value, indexed by its ordinal.  
\end{funcdesc}

\begin{funcdesc}{upper}{s}
Convert letters to upper case.
\end{funcdesc}

\begin{funcdesc}{ljust}{s\, width}
\funcline{rjust}{s\, width}
\funcline{center}{s\, width}
These functions respectively left-justify, right-justify and center a
string in a field of given width.
They return a string that is at least
\var{width}
characters wide, created by padding the string
\var{s}
with spaces until the given width on the right, left or both sides.
The string is never truncated.
\end{funcdesc}

\begin{funcdesc}{zfill}{s\, width}
Pad a numeric string on the left with zero digits until the given
width is reached.  Strings starting with a sign are handled correctly.
\end{funcdesc}

This module is implemented in Python.  Much of its functionality has
been reimplemented in the built-in module \code{strop}.  However, you
should \emph{never} import the latter module directly.  When
\code{string} discovers that \code{strop} exists, it transparently
replaces parts of itself with the implementation from \code{strop}.
After initialization, there is \emph{no} overhead in using
\code{string} instead of \code{strop}.
\bimodindex{strop}

\section{Built-in Module \sectcode{regex}}

\bimodindex{regex}
This module provides regular expression matching operations similar to
those found in Emacs.  It is always available.

By default the patterns are Emacs-style regular expressions
(with one exception).  There is
a way to change the syntax to match that of several well-known
\UNIX{} utilities.  The exception is that Emacs' \samp{\e s}
pattern is not supported, since the original implementation references
the Emacs syntax tables.

This module is 8-bit clean: both patterns and strings may contain null
bytes and characters whose high bit is set.

\strong{Please note:} There is a little-known fact about Python string
literals which means that you don't usually have to worry about
doubling backslashes, even though they are used to escape special
characters in string literals as well as in regular expressions.  This
is because Python doesn't remove backslashes from string literals if
they are followed by an unrecognized escape character.
\emph{However}, if you want to include a literal \dfn{backslash} in a
regular expression represented as a string literal, you have to
\emph{quadruple} it.  E.g.\  to extract \LaTeX\ \samp{\e section\{{\rm
\ldots}\}} headers from a document, you can use this pattern:
\code{'\e \e \e \e section\{\e (.*\e )\}'}.  \emph{Another exception:}
the escape sequece \samp{\e b} is significant in string literals
(where it means the ASCII bell character) as well as in Emacs regular
expressions (where it stands for a word boundary), so in order to
search for a word boundary, you should use the pattern \code{'\e \e b'}.
Similarly, a backslash followed by a digit 0-7 should be doubled to
avoid interpretation as an octal escape.

\subsection{Regular Expressions}

A regular expression (or RE) specifies a set of strings that matches
it; the functions in this module let you check if a particular string
matches a given regular expression (or if a given regular expression
matches a particular string, which comes down to the same thing).

Regular expressions can be concatenated to form new regular
expressions; if \emph{A} and \emph{B} are both regular expressions,
then \emph{AB} is also an regular expression.  If a string \emph{p}
matches A and another string \emph{q} matches B, the string \emph{pq}
will match AB.  Thus, complex expressions can easily be constructed
from simpler ones like the primitives described here.  For details of
the theory and implementation of regular expressions, consult almost
any textbook about compiler construction.

% XXX The reference could be made more specific, say to 
% "Compilers: Principles, Techniques and Tools", by Alfred V. Aho, 
% Ravi Sethi, and Jeffrey D. Ullman, or some FA text.   

A brief explanation of the format of regular expressions follows.

Regular expressions can contain both special and ordinary characters.
Ordinary characters, like '\code{A}', '\code{a}', or '\code{0}', are
the simplest regular expressions; they simply match themselves.  You
can concatenate ordinary characters, so '\code{last}' matches the
characters 'last'.  (In the rest of this section, we'll write RE's in
\code{this special font}, usually without quotes, and strings to be
matched 'in single quotes'.)

Special characters either stand for classes of ordinary characters, or
affect how the regular expressions around them are interpreted.

The special characters are:
\begin{itemize}
\item[\code{.}]{(Dot.)  Matches any character except a newline.}
\item[\code{\^}]{(Caret.)  Matches the start of the string.}
\item[\code{\$}]{Matches the end of the string.  
\code{foo} matches both 'foo' and 'foobar', while the regular
expression '\code{foo\$}' matches only 'foo'.}
\item[\code{*}] Causes the resulting RE to
match 0 or more repetitions of the preceding RE.  \code{ab*} will
match 'a', 'ab', or 'a' followed by any number of 'b's.
\item[\code{+}] Causes the
resulting RE to match 1 or more repetitions of the preceding RE.
\code{ab+} will match 'a' followed by any non-zero number of 'b's; it
will not match just 'a'.
\item[\code{?}] Causes the resulting RE to
match 0 or 1 repetitions of the preceding RE.  \code{ab?} will
match either 'a' or 'ab'.

\item[\code{\e}] Either escapes special characters (permitting you to match
characters like '*?+\&\$'), or signals a special sequence; special
sequences are discussed below.  Remember that Python also uses the
backslash as an escape sequence in string literals; if the escape
sequence isn't recognized by Python's parser, the backslash and
subsequent character are included in the resulting string.  However,
if Python would recognize the resulting sequence, the backslash should
be repeated twice.  

\item[\code{[]}] Used to indicate a set of characters.  Characters can
be listed individually, or a range is indicated by giving two
characters and separating them by a '-'.  Special characters are
not active inside sets.  For example, \code{[akm\$]}
will match any of the characters 'a', 'k', 'm', or '\$'; \code{[a-z]} will
match any lowercase letter.  

If you want to include a \code{]} inside a
set, it must be the first character of the set; to include a \code{-},
place it as the first or last character. 

Characters \emph{not} within a range can be matched by including a
\code{\^} as the first character of the set; \code{\^} elsewhere will
simply match the '\code{\^}' character.  
\end{itemize}

The special sequences consist of '\code{\e}' and a character
from the list below.  If the ordinary character is not on the list,
then the resulting RE will match the second character.  For example,
\code{\e\$} matches the character '\$'.  Ones where the backslash
should be doubled are indicated.

\begin{itemize}
\item[\code{\e|}]\code{A\e|B}, where A and B can be arbitrary REs,
creates a regular expression that will match either A or B.  This can
be used inside groups (see below) as well.
%
\item[\code{\e( \e)}]{Indicates the start and end of a group; the
contents of a group can be matched later in the string with the
\code{\e \[1-9]} special sequence, described next.}
%
{\fulllineitems\item[\code{\e \e 1, ... \e \e 7, \e 8, \e 9}]
{Matches the contents of the group of the same
number.  For example, \code{\e (.+\e ) \e \e 1} matches 'the the' or
'55 55', but not 'the end' (note the space after the group).  This
special sequence can only be used to match one of the first 9 groups;
groups with higher numbers can be matched using the \code{\e v}
sequence.  (\code{\e 8} and \code{\e 9} don't need a double backslash
because they are not octal digits.)}}
%
\item[\code{\e \e b}]{Matches the empty string, but only at the
beginning or end of a word.  A word is defined as a sequence of
alphanumeric characters, so the end of a word is indicated by
whitespace or a non-alphanumeric character.}
%
\item[\code{\e B}]{Matches the empty string, but when it is \emph{not} at the
beginning or end of a word.} 
%
\item[\code{\e v}]{Must be followed by a two digit decimal number, and
matches the contents of the group of the same number.  The group number must be between 1 and 99, inclusive.}
%
\item[\code{\e w}]Matches any alphanumeric character; this is
equivalent to the set \code{[a-zA-Z0-9]}.
%
\item[\code{\e W}]{Matches any non-alphanumeric character; this is
equivalent to the set \code{[\^a-zA-Z0-9]}.} 
\item[\code{\e <}]{Matches the empty string, but only at the beginning of a
word.  A word is defined as a sequence of alphanumeric characters, so
the end of a word is indicated by whitespace or a non-alphanumeric 
character.}
\item[\code{\e >}]{Matches the empty string, but only at the end of a
word.}

\item[\code{\e \e \e \e}]{Matches a literal backslash.}

% In Emacs, the following two are start of buffer/end of buffer.  In
% Python they seem to be synonyms for ^$.
\item[\code{\e `}]{Like \code{\^}, this only matches at the start of the
string.}
\item[\code{\e \e '}] Like \code{\$}, this only matches at the end of the
string.
% end of buffer
\end{itemize}

\subsection{Module Contents}

The module defines these functions, and an exception:

\renewcommand{\indexsubitem}{(in module regex)}

\begin{funcdesc}{match}{pattern\, string}
  Return how many characters at the beginning of \var{string} match
  the regular expression \var{pattern}.  Return \code{-1} if the
  string does not match the pattern (this is different from a
  zero-length match!).
\end{funcdesc}

\begin{funcdesc}{search}{pattern\, string}
  Return the first position in \var{string} that matches the regular
  expression \var{pattern}.  Return \code{-1} if no position in the string
  matches the pattern (this is different from a zero-length match
  anywhere!).
\end{funcdesc}

\begin{funcdesc}{compile}{pattern\optional{\, translate}}
  Compile a regular expression pattern into a regular expression
  object, which can be used for matching using its \code{match} and
  \code{search} methods, described below.  The optional argument
  \var{translate}, if present, must be a 256-character string
  indicating how characters (both of the pattern and of the strings to
  be matched) are translated before comparing them; the \code{i}-th
  element of the string gives the translation for the character with
  \ASCII{} code \code{i}.  This can be used to implement
  case-insensitive matching; see the \code{casefold} data item below.

  The sequence

\bcode\begin{verbatim}
prog = regex.compile(pat)
result = prog.match(str)
\end{verbatim}\ecode

is equivalent to

\bcode\begin{verbatim}
result = regex.match(pat, str)
\end{verbatim}\ecode

but the version using \code{compile()} is more efficient when multiple
regular expressions are used concurrently in a single program.  (The
compiled version of the last pattern passed to \code{regex.match()} or
\code{regex.search()} is cached, so programs that use only a single
regular expression at a time needn't worry about compiling regular
expressions.)
\end{funcdesc}

\begin{funcdesc}{set_syntax}{flags}
  Set the syntax to be used by future calls to \code{compile},
  \code{match} and \code{search}.  (Already compiled expression objects
  are not affected.)  The argument is an integer which is the OR of
  several flag bits.  The return value is the previous value of
  the syntax flags.  Names for the flags are defined in the standard
  module \code{regex_syntax}; read the file \file{regex_syntax.py} for
  more information.
\end{funcdesc}

\begin{funcdesc}{symcomp}{pattern\optional{\, translate}}
This is like \code{compile}, but supports symbolic group names: if a
parenthesis-enclosed group begins with a group name in angular
brackets, e.g. \code{'\e(<id>[a-z][a-z0-9]*\e)'}, the group can
be referenced by its name in arguments to the \code{group} method of
the resulting compiled regular expression object, like this:
\code{p.group('id')}.  Group names may contain alphanumeric characters
and \code{'_'} only.
\end{funcdesc}

\begin{excdesc}{error}
  Exception raised when a string passed to one of the functions here
  is not a valid regular expression (e.g., unmatched parentheses) or
  when some other error occurs during compilation or matching.  (It is
  never an error if a string contains no match for a pattern.)
\end{excdesc}

\begin{datadesc}{casefold}
A string suitable to pass as \var{translate} argument to
\code{compile} to map all upper case characters to their lowercase
equivalents.
\end{datadesc}

\noindent
Compiled regular expression objects support these methods:

\renewcommand{\indexsubitem}{(regex method)}
\begin{funcdesc}{match}{string\optional{\, pos}}
  Return how many characters at the beginning of \var{string} match
  the compiled regular expression.  Return \code{-1} if the string
  does not match the pattern (this is different from a zero-length
  match!).
  
  The optional second parameter \var{pos} gives an index in the string
  where the search is to start; it defaults to \code{0}.  This is not
  completely equivalent to slicing the string; the \code{'\^'} pattern
  character matches at the real begin of the string and at positions
  just after a newline, not necessarily at the index where the search
  is to start.
\end{funcdesc}

\begin{funcdesc}{search}{string\optional{\, pos}}
  Return the first position in \var{string} that matches the regular
  expression \code{pattern}.  Return \code{-1} if no position in the
  string matches the pattern (this is different from a zero-length
  match anywhere!).
  
  The optional second parameter has the same meaning as for the
  \code{match} method.
\end{funcdesc}

\begin{funcdesc}{group}{index\, index\, ...}
This method is only valid when the last call to the \code{match}
or \code{search} method found a match.  It returns one or more
groups of the match.  If there is a single \var{index} argument,
the result is a single string; if there are multiple arguments, the
result is a tuple with one item per argument.  If the \var{index} is
zero, the corresponding return value is the entire matching string; if
it is in the inclusive range [1..99], it is the string matching the
the corresponding parenthesized group (using the default syntax,
groups are parenthesized using \code{\\(} and \code{\\)}).  If no
such group exists, the corresponding result is \code{None}.

If the regular expression was compiled by \code{symcomp} instead of
\code{compile}, the \var{index} arguments may also be strings
identifying groups by their group name.
\end{funcdesc}

\noindent
Compiled regular expressions support these data attributes:

\renewcommand{\indexsubitem}{(regex attribute)}

\begin{datadesc}{regs}
When the last call to the \code{match} or \code{search} method found a
match, this is a tuple of pairs of indices corresponding to the
beginning and end of all parenthesized groups in the pattern.  Indices
are relative to the string argument passed to \code{match} or
\code{search}.  The 0-th tuple gives the beginning and end or the
whole pattern.  When the last match or search failed, this is
\code{None}.
\end{datadesc}

\begin{datadesc}{last}
When the last call to the \code{match} or \code{search} method found a
match, this is the string argument passed to that method.  When the
last match or search failed, this is \code{None}.
\end{datadesc}

\begin{datadesc}{translate}
This is the value of the \var{translate} argument to
\code{regex.compile} that created this regular expression object.  If
the \var{translate} argument was omitted in the \code{regex.compile}
call, this is \code{None}.
\end{datadesc}

\begin{datadesc}{givenpat}
The regular expression pattern as passed to \code{compile} or
\code{symcomp}.
\end{datadesc}

\begin{datadesc}{realpat}
The regular expression after stripping the group names for regular
expressions compiled with \code{symcomp}.  Same as \code{givenpat}
otherwise.
\end{datadesc}

\begin{datadesc}{groupindex}
A dictionary giving the mapping from symbolic group names to numerical
group indices for regular expressions compiled with \code{symcomp}.
\code{None} otherwise.
\end{datadesc}

\section{Standard Module \sectcode{regsub}}

\stmodindex{regsub}
This module defines a number of functions useful for working with
regular expressions (see built-in module \code{regex}).

Warning: these functions are not thread-safe.

\renewcommand{\indexsubitem}{(in module regsub)}

\begin{funcdesc}{sub}{pat\, repl\, str}
Replace the first occurrence of pattern \var{pat} in string
\var{str} by replacement \var{repl}.  If the pattern isn't found,
the string is returned unchanged.  The pattern may be a string or an
already compiled pattern.  The replacement may contain references
\samp{\e \var{digit}} to subpatterns and escaped backslashes.
\end{funcdesc}

\begin{funcdesc}{gsub}{pat\, repl\, str}
Replace all (non-overlapping) occurrences of pattern \var{pat} in
string \var{str} by replacement \var{repl}.  The same rules as for
\code{sub()} apply.  Empty matches for the pattern are replaced only
when not adjacent to a previous match, so e.g.
\code{gsub('', '-', 'abc')} returns \code{'-a-b-c-'}.
\end{funcdesc}

\begin{funcdesc}{split}{str\, pat\optional{\, maxsplit}}
Split the string \var{str} in fields separated by delimiters matching
the pattern \var{pat}, and return a list containing the fields.  Only
non-empty matches for the pattern are considered, so e.g.
\code{split('a:b', ':*')} returns \code{['a', 'b']} and
\code{split('abc', '')} returns \code{['abc']}.  The \var{maxsplit}
defaults to 0. If it is nonzero, only \var{maxsplit} number of splits
occur, and the remainder of the string is returned as the final
element of the list.
\end{funcdesc}

\begin{funcdesc}{splitx}{str\, pat\optional{\, maxsplit}}
Split the string \var{str} in fields separated by delimiters matching
the pattern \var{pat}, and return a list containing the fields as well
as the separators.  For example, \code{splitx('a:::b', ':*')} returns
\code{['a', ':::', 'b']}.  Otherwise, this function behaves the same
as \code{split}.
\end{funcdesc}

\begin{funcdesc}{capwords}{s\optional{\, pat}}
Capitalize words separated by optional pattern \var{pat}.  The default
pattern uses any characters except letters, digits and underscores as
word delimiters.  Capitalization is done by changing the first
character of each word to upper case.
\end{funcdesc}

\section{Built-in Module \sectcode{struct}}
\bimodindex{struct}
\indexii{C}{structures}

This module performs conversions between Python values and C
structs represented as Python strings.  It uses \dfn{format strings}
(explained below) as compact descriptions of the lay-out of the C
structs and the intended conversion to/from Python values.

See also built-in module \code{array}.
\bimodindex{array}

The module defines the following exception and functions:

\renewcommand{\indexsubitem}{(in module struct)}
\begin{excdesc}{error}
  Exception raised on various occasions; argument is a string
  describing what is wrong.
\end{excdesc}

\begin{funcdesc}{pack}{fmt\, v1\, v2\, {\rm \ldots}}
  Return a string containing the values
  \code{\var{v1}, \var{v2}, {\rm \ldots}} packed according to the given
  format.  The arguments must match the values required by the format
  exactly.
\end{funcdesc}

\begin{funcdesc}{unpack}{fmt\, string}
  Unpack the string (presumably packed by \code{pack(\var{fmt}, {\rm \ldots})})
  according to the given format.  The result is a tuple even if it
  contains exactly one item.  The string must contain exactly the
  amount of data required by the format (i.e.  \code{len(\var{string})} must
  equal \code{calcsize(\var{fmt})}).
\end{funcdesc}

\begin{funcdesc}{calcsize}{fmt}
  Return the size of the struct (and hence of the string)
  corresponding to the given format.
\end{funcdesc}

Format characters have the following meaning; the conversion between C
and Python values should be obvious given their types:

\begin{tableiii}{|c|l|l|}{samp}{Format}{C}{Python}
  \lineiii{x}{pad byte}{no value}
  \lineiii{c}{char}{string of length 1}
  \lineiii{b}{signed char}{integer}
  \lineiii{h}{short}{integer}
  \lineiii{i}{int}{integer}
  \lineiii{l}{long}{integer}
  \lineiii{f}{float}{float}
  \lineiii{d}{double}{float}
\end{tableiii}

A format character may be preceded by an integral repeat count; e.g.\
the format string \code{'4h'} means exactly the same as \code{'hhhh'}.

C numbers are represented in the machine's native format and byte
order, and properly aligned by skipping pad bytes if necessary
(according to the rules used by the C compiler).

Examples (all on a big-endian machine):

\bcode\begin{verbatim}
pack('hhl', 1, 2, 3) == '\000\001\000\002\000\000\000\003'
unpack('hhl', '\000\001\000\002\000\000\000\003') == (1, 2, 3)
calcsize('hhl') == 8
\end{verbatim}\ecode

Hint: to align the end of a structure to the alignment requirement of
a particular type, end the format with the code for that type with a
repeat count of zero, e.g.\ the format \code{'llh0l'} specifies two
pad bytes at the end, assuming longs are aligned on 4-byte boundaries.

(More format characters are planned, e.g.\ \code{'s'} for character
arrays, upper case for unsigned variants, and a way to specify the
byte order, which is useful for [de]constructing network packets and
reading/writing portable binary file formats like TIFF and AIFF.)


\chapter{Miscellaneous Services}

The modules described in this chapter provide miscellaneous services
that are available in all Python versions.  Here's an overview:

\begin{description}

\item[math]
--- Mathematical functions (\code{sin()} etc.).

\item[rand]
--- Integer random number generator.

\item[whrandom]
--- Floating point random number generator.

\item[array]
--- Efficient arrays of uniformly typed numeric values.

\end{description}
			% Miscellaneous Services
\section{Built-in Module \sectcode{math}}

\bimodindex{math}
\renewcommand{\indexsubitem}{(in module math)}
This module is always available.
It provides access to the mathematical functions defined by the C
standard.
They are:
\iftexi
\begin{funcdesc}{acos}{x}
\funcline{asin}{x}
\funcline{atan}{x}
\funcline{atan2}{x, y}
\funcline{ceil}{x}
\funcline{cos}{x}
\funcline{cosh}{x}
\funcline{exp}{x}
\funcline{fabs}{x}
\funcline{floor}{x}
\funcline{fmod}{x, y}
\funcline{frexp}{x}
\funcline{hypot}{x, y}
\funcline{ldexp}{x, y}
\funcline{log}{x}
\funcline{log10}{x}
\funcline{modf}{x}
\funcline{pow}{x, y}
\funcline{sin}{x}
\funcline{sinh}{x}
\funcline{sqrt}{x}
\funcline{tan}{x}
\funcline{tanh}{x}
\end{funcdesc}
\else
\code{acos(\varvars{x})},
\code{asin(\varvars{x})},
\code{atan(\varvars{x})},
\code{atan2(\varvars{x\, y})},
\code{ceil(\varvars{x})},
\code{cos(\varvars{x})},
\code{cosh(\varvars{x})},
\code{exp(\varvars{x})},
\code{fabs(\varvars{x})},
\code{floor(\varvars{x})},
\code{fmod(\varvars{x\, y})},
\code{frexp(\varvars{x})},
\code{hypot(\varvars{x\, y})},
\code{ldexp(\varvars{x\, y})},
\code{log(\varvars{x})},
\code{log10(\varvars{x})},
\code{modf(\varvars{x})},
\code{pow(\varvars{x\, y})},
\code{sin(\varvars{x})},
\code{sinh(\varvars{x})},
\code{sqrt(\varvars{x})},
\code{tan(\varvars{x})},
\code{tanh(\varvars{x})}.
\fi

Note that \code{frexp} and \code{modf} have a different call/return
pattern than their C equivalents: they take a single argument and
return a pair of values, rather than returning their second return
value through an `output parameter' (there is no such thing in Python).

The module also defines two mathematical constants:
\iftexi
\begin{datadesc}{pi}
\dataline{e}
\end{datadesc}
\else
\code{pi} and \code{e}.
\fi

\section{Standard Module \sectcode{rand}}

\stmodindex{rand} This module implements a pseudo-random number
generator with an interface similar to \code{rand()} in C\@.  It defines
the following functions:

\renewcommand{\indexsubitem}{(in module rand)}
\begin{funcdesc}{rand}{}
Returns an integer random number in the range [0 ... 32768).
\end{funcdesc}

\begin{funcdesc}{choice}{s}
Returns a random element from the sequence (string, tuple or list)
\var{s}.
\end{funcdesc}

\begin{funcdesc}{srand}{seed}
Initializes the random number generator with the given integral seed.
When the module is first imported, the random number is initialized with
the current time.
\end{funcdesc}

\section{Standard Module \sectcode{whrandom}}

\stmodindex{whrandom}
This module implements a Wichmann-Hill pseudo-random number generator.
It defines the following functions:

\renewcommand{\indexsubitem}{(in module whrandom)}
\begin{funcdesc}{random}{}
Returns the next random floating point number in the range [0.0 ... 1.0).
\end{funcdesc}

\begin{funcdesc}{seed}{x\, y\, z}
Initializes the random number generator from the integers
\var{x},
\var{y}
and
\var{z}.
When the module is first imported, the random number is initialized
using values derived from the current time.
\end{funcdesc}

\section{Built-in Module \sectcode{array}}
\bimodindex{array}
\index{arrays}

This module defines a new object type which can efficiently represent
an array of basic values: characters, integers, floating point
numbers.  Arrays are sequence types and behave very much like lists,
except that the type of objects stored in them is constrained.  The
type is specified at object creation time by using a \dfn{type code},
which is a single character.  The following type codes are defined:

\begin{tableiii}{|c|c|c|}{code}{Typecode}{Type}{Minimal size in bytes}
\lineiii{'c'}{character}{1}
\lineiii{'b'}{signed integer}{1}
\lineiii{'h'}{signed integer}{2}
\lineiii{'i'}{signed integer}{2}
\lineiii{'l'}{signed integer}{4}
\lineiii{'f'}{floating point}{4}
\lineiii{'d'}{floating point}{8}
\end{tableiii}

The actual representation of values is determined by the machine
architecture (strictly speaking, by the C implementation).  The actual
size can be accessed through the \var{itemsize} attribute.

See also built-in module \code{struct}.
\bimodindex{struct}

The module defines the following function:

\renewcommand{\indexsubitem}{(in module array)}

\begin{funcdesc}{array}{typecode\optional{\, initializer}}
Return a new array whose items are restricted by \var{typecode}, and
initialized from the optional \var{initializer} value, which must be a
list or a string.  The list or string is passed to the new array's
\code{fromlist()} or \code{fromstring()} method (see below) to add
initial items to the array.
\end{funcdesc}

Array objects support the following data items and methods:

\begin{datadesc}{typecode}
The typecode character used to create the array.
\end{datadesc}

\begin{datadesc}{itemsize}
The length in bytes of one array item in the internal representation.
\end{datadesc}

\begin{funcdesc}{append}{x}
Append a new item with value \var{x} to the end of the array.
\end{funcdesc}

\begin{funcdesc}{byteswap}{x}
``Byteswap'' all items of the array.  This is only supported for
integer values.  It is useful when reading data from a file written
on a machine with a different byte order.
\end{funcdesc}

\begin{funcdesc}{fromfile}{f\, n}
Read \var{n} items (as machine values) from the file object \var{f}
and append them to the end of the array.  If less than \var{n} items
are available, \code{EOFError} is raised, but the items that were
available are still inserted into the array.  \var{f} must be a real
built-in file object; something else with a \code{read()} method won't
do.
\end{funcdesc}

\begin{funcdesc}{fromlist}{list}
Append items from the list.  This is equivalent to
\code{for x in \var{list}:\ a.append(x)}
except that if there is a type error, the array is unchanged.
\end{funcdesc}

\begin{funcdesc}{fromstring}{s}
Appends items from the string, interpreting the string as an
array of machine values (i.e. as if it had been read from a
file using the \code{fromfile()} method).
\end{funcdesc}

\begin{funcdesc}{insert}{i\, x}
Insert a new item with value \var{x} in the array before position
\var{i}.
\end{funcdesc}

\begin{funcdesc}{tofile}{f}
Write all items (as machine values) to the file object \var{f}.
\end{funcdesc}

\begin{funcdesc}{tolist}{}
Convert the array to an ordinary list with the same items.
\end{funcdesc}

\begin{funcdesc}{tostring}{}
Convert the array to an array of machine values and return the
string representation (the same sequence of bytes that would
be written to a file by the \code{tofile()} method.)
\end{funcdesc}

When an array object is printed or converted to a string, it is
represented as \code{array(\var{typecode}, \var{initializer})}.  The
\var{initializer} is omitted if the array is empty, otherwise it is a
string if the \var{typecode} is \code{'c'}, otherwise it is a list of
numbers.  The string is guaranteed to be able to be converted back to
an array with the same type and value using reverse quotes
(\code{``}).  Examples:

\bcode\begin{verbatim}
array('l')
array('c', 'hello world')
array('l', [1, 2, 3, 4, 5])
array('d', [1.0, 2.0, 3.14])
\end{verbatim}\ecode


\chapter{Generic Operating System Services}

The modules described in this chapter provide interfaces to operating
system features that are available on (almost) all operating systems,
such as files and a clock.  The interfaces are generally modelled
after the \UNIX{} or C interfaces but they are available on most other
systems as well.  Here's an overview:

\begin{description}

\item[os]
--- Miscellaneous OS interfaces.

\item[time]
--- Time access and conversions.

\item[getopt]
--- Parser for command line options.

\item[tempfile]
--- Generate temporary file names.

\end{description}
		% Generic Operating System Services
\section{Standard Module \sectcode{os}}

\stmodindex{os}
This module provides a more portable way of using operating system
(OS) dependent functionality than importing an OS dependent built-in
module like \code{posix}.

When the optional built-in module \code{posix} is available, this
module exports the same functions and data as \code{posix}; otherwise,
it searches for an OS dependent built-in module like \code{mac} and
exports the same functions and data as found there.  The design of all
Python's built-in OS dependent modules is such that as long as the same
functionality is available, it uses the same interface; e.g., the
function \code{os.stat(\var{file})} returns stat info about a \var{file} in a
format compatible with the POSIX interface.

Extensions peculiar to a particular OS are also available through the
\code{os} module, but using them is of course a threat to portability!

Note that after the first time \code{os} is imported, there is \emph{no}
performance penalty in using functions from \code{os} instead of
directly from the OS dependent built-in module, so there should be
\emph{no} reason not to use \code{os}!

In addition to whatever the correct OS dependent module exports, the
following variables and functions are always exported by \code{os}:

\renewcommand{\indexsubitem}{(in module os)}

\begin{datadesc}{name}
The name of the OS dependent module imported.  The following names
have currently been registered: \code{'posix'}, \code{'nt'},
\code{'dos'}, \code{'mac'}.
\end{datadesc}

\begin{datadesc}{path}
The corresponding OS dependent standard module for pathname
operations, e.g., \code{posixpath} or \code{macpath}.  Thus, (given
the proper imports), \code{os.path.split(\var{file})} is equivalent to but
more portable than \code{posixpath.split(\var{file})}.
\end{datadesc}

\begin{datadesc}{curdir}
The constant string used by the OS to refer to the current directory,
e.g. \code{'.'} for POSIX or \code{':'} for the Mac.
\end{datadesc}

\begin{datadesc}{pardir}
The constant string used by the OS to refer to the parent directory,
e.g. \code{'..'} for POSIX or \code{'::'} for the Mac.
\end{datadesc}

\begin{datadesc}{sep}
The character used by the OS to separate pathname components, e.g.\
\code{'/'} for POSIX or \code{':'} for the Mac.  Note that knowing this
is not sufficient to be able to parse or concatenate pathnames---better
use \code{os.path.split()} and \code{os.path.join()}---but it is
occasionally useful.
\end{datadesc}

\begin{datadesc}{pathsep}
The character conventionally used by the OS to separate search patch
components (as in \code{\$PATH}), e.g.\ \code{':'} for POSIX or
\code{';'} for MS-DOS.
\end{datadesc}

\begin{datadesc}{defpath}
The default search path used by \code{os.exec*p*()} if the environment
doesn't have a \code{'PATH'} key.
\end{datadesc}

\begin{funcdesc}{execl}{path\, arg0\, arg1\, ...}
This is equivalent to
\code{os.execv(\var{path}, (\var{arg0}, \var{arg1}, ...))}.
\end{funcdesc}

\begin{funcdesc}{execle}{path\, arg0\, arg1\, ...\, env}
This is equivalent to
\code{os.execve(\var{path}, (\var{arg0}, \var{arg1}, ...), \var{env})}.
\end{funcdesc}

\begin{funcdesc}{execlp}{path\, arg0\, arg1\, ...}
This is equivalent to
\code{os.execvp(\var{path}, (\var{arg0}, \var{arg1}, ...))}.
\end{funcdesc}

\begin{funcdesc}{execvp}{path\, args}
This is like \code{os.execv(\var{path}, \var{args})} but duplicates
the shell's actions in searching for an executable file in a list of
directories.  The directory list is obtained from
\code{os.environ['PATH']}.
\end{funcdesc}

\begin{funcdesc}{execvpe}{path\, args\, env}
This is a cross between \code{os.execve()} and \code{os.execvp()}.
The directory list is obtained from \code{\var{env}['PATH']}.
\end{funcdesc}

(The functions \code{os.execv()} and \code{execve()} are not
documented here, since they are implemented by the OS dependent
module.  If the OS dependent module doesn't define either of these,
the functions that rely on it will raise an exception.  They are
documented in the section on module \code{posix}, together with all
other functions that \code{os} imports from the OS dependent module.)

\section{Built-in Module \sectcode{time}}

\bimodindex{time}
This module provides various time-related functions.
It is always available.

An explanation of some terminology and conventions is in order.

\begin{itemize}

\item
The ``epoch'' is the point where the time starts.  On January 1st of that
year, at 0 hours, the ``time since the epoch'' is zero.  For UNIX, the
epoch is 1970.  To find out what the epoch is, look at \code{gmtime(0)}.

\item
UTC is Coordinated Universal Time (formerly known as Greenwich Mean
Time).  The acronym UTC is not a mistake but a compromise between
English and French.

\item
DST is Daylight Saving Time, an adjustment of the timezone by
(usually) one hour during part of the year.  DST rules are magic
(determined by local law) and can change from year to year.  The C
library has a table containing the local rules (often it is read from
a system file for flexibility) and is the only source of True Wisdom
in this respect.

\item
The precision of the various real-time functions may be less than
suggested by the units in which their value or argument is expressed.
E.g.\ on most UNIX systems, the clock ``ticks'' only 50 or 100 times a
second, and on the Mac, times are only accurate to whole seconds.

\item
The time tuple as returned by \code{gmtime()} and \code{localtime()},
or as accpted by \code{mktime()} is a tuple of 9
integers: year (e.g.\ 1993), month (1--12), day (1--31), hour
(0--23), minute (0--59), second (0--59), weekday (0--6, monday is 0),
Julian day (1--366) and daylight savings flag (-1, 0  or 1).
Note that unlike the C structure, the month value is a range of 1-12, not
0-11.  A year value of $<$ 100 will typically be silently converted to
1900 $+$ year value.  A -1 argument as daylight savings flag, passed to
\code{mktime()} will usually result in the correct daylight savings
state to be filled in.


\end{itemize}

The module defines the following functions and data items:

\renewcommand{\indexsubitem}{(in module time)}

\begin{datadesc}{altzone}
The offset of the local DST timezone, in seconds west of the 0th
meridian, if one is defined.  Negative if the local DST timezone is
east of the 0th meridian (as in Western Europe, including the UK).
Only use this if \code{daylight} is nonzero.
\end{datadesc}

\begin{funcdesc}{asctime}{tuple}
Convert a tuple representing a time as returned by \code{gmtime()} or
\code{localtime()} to a 24-character string of the following form:
\code{'Sun Jun 20 23:21:05 1993'}.  Note: unlike the C function of
the same name, there is no trailing newline.
\end{funcdesc}

\begin{funcdesc}{clock}{}
Return the current CPU time as a floating point number expressed in
seconds.  The precision, and in fact the very definiton of the meaning
of ``CPU time'', depends on that of the C function of the same name.
\end{funcdesc}

\begin{funcdesc}{ctime}{secs}
Convert a time expressed in seconds since the epoch to a string
representing local time.  \code{ctime(t)} is equivalent to
\code{asctime(localtime(t))}.
\end{funcdesc}

\begin{datadesc}{daylight}
Nonzero if a DST timezone is defined.
\end{datadesc}

\begin{funcdesc}{gmtime}{secs}
Convert a time expressed in seconds since the epoch to a time tuple
in UTC in which the dst flag is always zero.  Fractions of a second are
ignored.
\end{funcdesc}

\begin{funcdesc}{localtime}{secs}
Like \code{gmtime} but converts to local time.  The dst flag is set
to 1 when DST applies to the given time.
\end{funcdesc}

\begin{funcdesc}{mktime}{tuple}
This is the inverse function of \code{localtime}.  Its argument is the
full 9-tuple (since the dst flag is needed --- pass -1 as the dst flag if
it is unknown) which expresses the time
in \em{local} time, not UTC.  It returns a floating
point number, for compatibility with \code{time.time()}.  If the input
value can't be represented as a valid time, OverflowError is raised.
\end{funcdesc}

\begin{funcdesc}{sleep}{secs}
Suspend execution for the given number of seconds.  The argument may
be a floating point number to indicate a more precise sleep time.
\end{funcdesc}

\begin{funcdesc}{strftime}{format, tuple}
Convert a tuple representing a time as returned by \code{gmtime()} or
\code{localtime()} to a string as specified by the format argument.

      The following directives, shown without the optional field width and
      precision specification, are replaced by the indicated characters:

\begin{tabular}{lp{25em}}
           \%a  &      Locale's abbreviated weekday name. \\
           \%A  &      Locale's full weekday name. \\
           \%b  &      Locale's abbreviated month name. \\
           \%B  &      Locale's full month name. \\
           \%c  &      Locale's appropriate date and time representation. \\
           \%d  &      Day of the month as a decimal number [01,31]. \\
           \%E  &      Locale's combined Emperor/Era name and year. \\
           \%H  &      Hour (24-hour clock) as a decimal number [00,23]. \\
           \%I  &      Hour (12-hour clock) as a decimal number [01,12]. \\
           \%j  &      Day of the year as a decimal number [001,366]. \\
           \%m  &      Month as a decimal number [01,12]. \\
           \%M  &      Minute as a decimal number [00,59]. \\
           \%n  &      New-line character. \\
           \%N  &      Locale's Emperor/Era name. \\
           \%o  &      Locale's Emperor/Era year. \\
           \%p  &      Locale's equivalent of either AM or PM. \\
           \%S  &      Second as a decimal number [00,61]. \\
           \%t  &      Tab character. \\
           \%U  &      Week number of the year (Sunday as the first day of the
                     week) as a decimal number [00,53].  All days in a new
                     year preceding the first Sunday are considered to be in
                     week 0. \\
           \%w  &      Weekday as a decimal number [0(Sunday),6]. \\
           \%W  &      Week number of the year (Monday as the first day of the
                     week) as a decimal number [00,53].  All days in a new
                     year preceding the first Sunday are considered to be in
                     week 0. \\
           \%x  &      Locale's appropriate date representation. \\
           \%X  &      Locale's appropriate time representation. \\
           \%y  &      Year without century as a decimal number [00,99]. \\
           \%Y  &      Year with century as a decimal number. \\
           \%Z  &      Time zone name (or by no characters if no time zone
                     exists). \\
           \%\%  &     \% \\
\end{tabular}

      An optional field width and precision specification can immediately
      follow the initial \% of a directive in the following order: \\

\begin{tabular}{lp{25em}}
      [-|0]w  &       the decimal digit string w specifies a minimum field
                     width in which the result of the conversion is right-
                     or left-justified.  It is right-justified (with space
                     padding) by default.  If the optional flag `-' is
                     specified, it is left-justified with space padding on
                     the right.  If the optional flag `0' is specified, it
                     is right-justified and padded with zeros on the left. \\
      .p      &       the decimal digit string p specifies the minimum number
                     of digits to appear for the d, H, I, j, m, M, o, S, U,
                     w, W, y and Y directives, and the maximum number of
                     characters to be used from the a, A, b, B, c, D, E, F,
                     h, n, N, p, r, t, T, x, X, z, Z, and % directives.  In
                     the first case, if a directive supplies fewer digits
                     than specified by the precision, it will be expanded
                     with leading zeros.  In the second case, if a directive
                     supplies more characters than specified by the
                     precision, excess characters will truncated on the
                     right.
\end{tabular}

      If no field width or precision is specified for a d, H, I, m, M, S, U,
      W, y, or j directive, a default of .2 is used for all but j for which
      .3 is used.

\end{funcdesc}

\begin{funcdesc}{time}{}
Return the time as a floating point number expressed in seconds since
the epoch, in UTC.  Note that even though the time is always returned
as a floating point number, not all systems provide time with a better
precision than 1 second.
\end{funcdesc}

\begin{datadesc}{timezone}
The offset of the local (non-DST) timezone, in seconds west of the 0th
meridian (i.e. negative in most of Western Europe, positive in the US,
zero in the UK).
\end{datadesc}

\begin{datadesc}{tzname}
A tuple of two strings: the first is the name of the local non-DST
timezone, the second is the name of the local DST timezone.  If no DST
timezone is defined, the second string should not be used.
\end{datadesc}


\section{Standard Module \sectcode{getopt}}

\stmodindex{getopt}
This module helps scripts to parse the command line arguments in
\code{sys.argv}.
It supports the same conventions as the \UNIX{}
\code{getopt()}
function (including the special meanings of arguments of the form
\samp{-} and \samp{--}).  Long options similar to those supported by
GNU software may be used as well via an optional third argument.
It defines the function
\code{getopt.getopt(args, options [, long_options])}
and the exception
\code{getopt.error}.

The first argument to
\code{getopt()}
is the argument list passed to the script with its first element
chopped off (i.e.,
\code{sys.argv[1:]}).
The second argument is the string of option letters that the
script wants to recognize, with options that require an argument
followed by a colon (i.e., the same format that \UNIX{}
\code{getopt()}
uses).
The third option, if specified, is a list of strings with the names of
the long options which should be supported.  The leading \code{'--'}
characters should not be included in the option name.  Options which
require an argument should be followed by an equal sign (\code{'='}).
The return value consists of two elements: the first is a list of
option-and-value pairs; the second is the list of program arguments
left after the option list was stripped (this is a trailing slice of the
first argument).
Each option-and-value pair returned has the option as its first element,
prefixed with a hyphen (e.g.,
\code{'-x'}),
and the option argument as its second element, or an empty string if the
option has no argument.
The options occur in the list in the same order in which they were
found, thus allowing multiple occurrences.  Long and short options may
be mixed.

An example using only \UNIX{} style options:

\bcode\begin{verbatim}
>>> import getopt, string
>>> args = string.split('-a -b -cfoo -d bar a1 a2')
>>> args
['-a', '-b', '-cfoo', '-d', 'bar', 'a1', 'a2']
>>> optlist, args = getopt.getopt(args, 'abc:d:')
>>> optlist
[('-a', ''), ('-b', ''), ('-c', 'foo'), ('-d', 'bar')]
>>> args
['a1', 'a2']
>>> 
\end{verbatim}\ecode

Using long option names is equally easy:

\bcode\begin{verbatim}
>>> s = '--condition=foo --testing --output-file abc.def -x a1 a2'
>>> args = string.split(s)
>>> args
['--condition=foo', '--testing', '--output-file', 'abc.def', '-x', 'a1', 'a2']
>>> optlist, args = getopt.getopt(args, 'x', [
...     'condition=', 'output-file=', 'testing'])
>>> optlist
[('--condition', 'foo'), ('--testing', ''), ('--output-file', 'abc.def'), ('-x', '')]
>>> args
['a1', 'a2']
>>> 
\end{verbatim}\ecode

The exception
\code{getopt.error = 'getopt.error'}
is raised when an unrecognized option is found in the argument list or
when an option requiring an argument is given none.
The argument to the exception is a string indicating the cause of the
error.  For long options, an argument given to an option which does
not require one will also cause this exception to be raised.

\section{Standard Module \sectcode{tempfile}}
\stmodindex{tempfile}
\indexii{temporary}{file name}
\indexii{temporary}{file}

\renewcommand{\indexsubitem}{(in module tempfile)}

This module generates temporary file names.  It is not \UNIX{} specific,
but it may require some help on non-\UNIX{} systems.

Note: the modules does not create temporary files, nor does it
automatically remove them when the current process exits or dies.

The module defines a single user-callable function:

\begin{funcdesc}{mktemp}{}
Return a unique temporary filename.  This is an absolute pathname of a
file that does not exist at the time the call is made.  No two calls
will return the same filename.
\end{funcdesc}

The module uses two global variables that tell it how to construct a
temporary name.  The caller may assign values to them; by default they
are initialized at the first call to \code{mktemp()}.

\begin{datadesc}{tempdir}
When set to a value other than \code{None}, this variable defines the
directory in which filenames returned by \code{mktemp()} reside.  The
default is taken from the environment variable \code{TMPDIR}; if this
is not set, either \code{/usr/tmp} is used (on \UNIX{}), or the current
working directory (all other systems).  No check is made to see
whether its value is valid.
\end{datadesc}
\ttindex{TMPDIR}

\begin{datadesc}{template}
When set to a value other than \code{None}, this variable defines the
prefix of the final component of the filenames returned by
\code{mktemp()}.  A string of decimal digits is added to generate
unique filenames.  The default is either ``\code{@\var{pid}.}'' where
\var{pid} is the current process ID (on \UNIX{}), or ``\code{tmp}'' (all
other systems).
\end{datadesc}

Warning: if a \UNIX{} process uses \code{mktemp()}, then calls
\code{fork()} and both parent and child continue to use
\code{mktemp()}, the processes will generate conflicting temporary
names.  To resolve this, the child process should assign \code{None}
to \code{template}, to force recomputing the default on the next call
to \code{mktemp()}.

\section{Standard Module \sectcode{errno}}
\stmodindex{errno}

\renewcommand{\indexsubitem}{(in module errno)}

This module makes available standard errno system symbols.
The value of each symbol is the corresponding integer value.
The names and descriptions are borrowed from linux/include/errno.h,
which should be pretty all-inclusive.  Of the following list, symbols
that are not used on the current platform are not defined by the
module.

Symbols available can include:
\begin{datadesc}{EPERM} Operation not permitted \end{datadesc}
\begin{datadesc}{ENOENT} No such file or directory \end{datadesc}
\begin{datadesc}{ESRCH} No such process \end{datadesc}
\begin{datadesc}{EINTR} Interrupted system call \end{datadesc}
\begin{datadesc}{EIO} I/O error \end{datadesc}
\begin{datadesc}{ENXIO} No such device or address \end{datadesc}
\begin{datadesc}{E2BIG} Arg list too long \end{datadesc}
\begin{datadesc}{ENOEXEC} Exec format error \end{datadesc}
\begin{datadesc}{EBADF} Bad file number \end{datadesc}
\begin{datadesc}{ECHILD} No child processes \end{datadesc}
\begin{datadesc}{EAGAIN} Try again \end{datadesc}
\begin{datadesc}{ENOMEM} Out of memory \end{datadesc}
\begin{datadesc}{EACCES} Permission denied \end{datadesc}
\begin{datadesc}{EFAULT} Bad address \end{datadesc}
\begin{datadesc}{ENOTBLK} Block device required \end{datadesc}
\begin{datadesc}{EBUSY} Device or resource busy \end{datadesc}
\begin{datadesc}{EEXIST} File exists \end{datadesc}
\begin{datadesc}{EXDEV} Cross-device link \end{datadesc}
\begin{datadesc}{ENODEV} No such device \end{datadesc}
\begin{datadesc}{ENOTDIR} Not a directory \end{datadesc}
\begin{datadesc}{EISDIR} Is a directory \end{datadesc}
\begin{datadesc}{EINVAL} Invalid argument \end{datadesc}
\begin{datadesc}{ENFILE} File table overflow \end{datadesc}
\begin{datadesc}{EMFILE} Too many open files \end{datadesc}
\begin{datadesc}{ENOTTY} Not a typewriter \end{datadesc}
\begin{datadesc}{ETXTBSY} Text file busy \end{datadesc}
\begin{datadesc}{EFBIG} File too large \end{datadesc}
\begin{datadesc}{ENOSPC} No space left on device \end{datadesc}
\begin{datadesc}{ESPIPE} Illegal seek \end{datadesc}
\begin{datadesc}{EROFS} Read-only file system \end{datadesc}
\begin{datadesc}{EMLINK} Too many links \end{datadesc}
\begin{datadesc}{EPIPE} Broken pipe \end{datadesc}
\begin{datadesc}{EDOM} Math argument out of domain of func \end{datadesc}
\begin{datadesc}{ERANGE} Math result not representable \end{datadesc}
\begin{datadesc}{EDEADLK} Resource deadlock would occur \end{datadesc}
\begin{datadesc}{ENAMETOOLONG} File name too long \end{datadesc}
\begin{datadesc}{ENOLCK} No record locks available \end{datadesc}
\begin{datadesc}{ENOSYS} Function not implemented \end{datadesc}
\begin{datadesc}{ENOTEMPTY} Directory not empty \end{datadesc}
\begin{datadesc}{ELOOP} Too many symbolic links encountered \end{datadesc}
\begin{datadesc}{EWOULDBLOCK} Operation would block \end{datadesc}
\begin{datadesc}{ENOMSG} No message of desired type \end{datadesc}
\begin{datadesc}{EIDRM} Identifier removed \end{datadesc}
\begin{datadesc}{ECHRNG} Channel number out of range \end{datadesc}
\begin{datadesc}{EL2NSYNC} Level 2 not synchronized \end{datadesc}
\begin{datadesc}{EL3HLT} Level 3 halted \end{datadesc}
\begin{datadesc}{EL3RST} Level 3 reset \end{datadesc}
\begin{datadesc}{ELNRNG} Link number out of range \end{datadesc}
\begin{datadesc}{EUNATCH} Protocol driver not attached \end{datadesc}
\begin{datadesc}{ENOCSI} No CSI structure available \end{datadesc}
\begin{datadesc}{EL2HLT} Level 2 halted \end{datadesc}
\begin{datadesc}{EBADE} Invalid exchange \end{datadesc}
\begin{datadesc}{EBADR} Invalid request descriptor \end{datadesc}
\begin{datadesc}{EXFULL} Exchange full \end{datadesc}
\begin{datadesc}{ENOANO} No anode \end{datadesc}
\begin{datadesc}{EBADRQC} Invalid request code \end{datadesc}
\begin{datadesc}{EBADSLT} Invalid slot \end{datadesc}
\begin{datadesc}{EDEADLOCK} File locking deadlock error \end{datadesc}
\begin{datadesc}{EBFONT} Bad font file format \end{datadesc}
\begin{datadesc}{ENOSTR} Device not a stream \end{datadesc}
\begin{datadesc}{ENODATA} No data available \end{datadesc}
\begin{datadesc}{ETIME} Timer expired \end{datadesc}
\begin{datadesc}{ENOSR} Out of streams resources \end{datadesc}
\begin{datadesc}{ENONET} Machine is not on the network \end{datadesc}
\begin{datadesc}{ENOPKG} Package not installed \end{datadesc}
\begin{datadesc}{EREMOTE} Object is remote \end{datadesc}
\begin{datadesc}{ENOLINK} Link has been severed \end{datadesc}
\begin{datadesc}{EADV} Advertise error \end{datadesc}
\begin{datadesc}{ESRMNT} Srmount error \end{datadesc}
\begin{datadesc}{ECOMM} Communication error on send \end{datadesc}
\begin{datadesc}{EPROTO} Protocol error \end{datadesc}
\begin{datadesc}{EMULTIHOP} Multihop attempted \end{datadesc}
\begin{datadesc}{EDOTDOT} RFS specific error \end{datadesc}
\begin{datadesc}{EBADMSG} Not a data message \end{datadesc}
\begin{datadesc}{EOVERFLOW} Value too large for defined data type \end{datadesc}
\begin{datadesc}{ENOTUNIQ} Name not unique on network \end{datadesc}
\begin{datadesc}{EBADFD} File descriptor in bad state \end{datadesc}
\begin{datadesc}{EREMCHG} Remote address changed \end{datadesc}
\begin{datadesc}{ELIBACC} Can not access a needed shared library \end{datadesc}
\begin{datadesc}{ELIBBAD} Accessing a corrupted shared library \end{datadesc}
\begin{datadesc}{ELIBSCN} .lib section in a.out corrupted \end{datadesc}
\begin{datadesc}{ELIBMAX} Attempting to link in too many shared libraries \end{datadesc}
\begin{datadesc}{ELIBEXEC} Cannot exec a shared library directly \end{datadesc}
\begin{datadesc}{EILSEQ} Illegal byte sequence \end{datadesc}
\begin{datadesc}{ERESTART} Interrupted system call should be restarted \end{datadesc}
\begin{datadesc}{ESTRPIPE} Streams pipe error \end{datadesc}
\begin{datadesc}{EUSERS} Too many users \end{datadesc}
\begin{datadesc}{ENOTSOCK} Socket operation on non-socket \end{datadesc}
\begin{datadesc}{EDESTADDRREQ} Destination address required \end{datadesc}
\begin{datadesc}{EMSGSIZE} Message too long \end{datadesc}
\begin{datadesc}{EPROTOTYPE} Protocol wrong type for socket \end{datadesc}
\begin{datadesc}{ENOPROTOOPT} Protocol not available \end{datadesc}
\begin{datadesc}{EPROTONOSUPPORT} Protocol not supported \end{datadesc}
\begin{datadesc}{ESOCKTNOSUPPORT} Socket type not supported \end{datadesc}
\begin{datadesc}{EOPNOTSUPP} Operation not supported on transport endpoint \end{datadesc}
\begin{datadesc}{EPFNOSUPPORT} Protocol family not supported \end{datadesc}
\begin{datadesc}{EAFNOSUPPORT} Address family not supported by protocol \end{datadesc}
\begin{datadesc}{EADDRINUSE} Address already in use \end{datadesc}
\begin{datadesc}{EADDRNOTAVAIL} Cannot assign requested address \end{datadesc}
\begin{datadesc}{ENETDOWN} Network is down \end{datadesc}
\begin{datadesc}{ENETUNREACH} Network is unreachable \end{datadesc}
\begin{datadesc}{ENETRESET} Network dropped connection because of reset \end{datadesc}
\begin{datadesc}{ECONNABORTED} Software caused connection abort \end{datadesc}
\begin{datadesc}{ECONNRESET} Connection reset by peer \end{datadesc}
\begin{datadesc}{ENOBUFS} No buffer space available \end{datadesc}
\begin{datadesc}{EISCONN} Transport endpoint is already connected \end{datadesc}
\begin{datadesc}{ENOTCONN} Transport endpoint is not connected \end{datadesc}
\begin{datadesc}{ESHUTDOWN} Cannot send after transport endpoint shutdown \end{datadesc}
\begin{datadesc}{ETOOMANYREFS} Too many references: cannot splice \end{datadesc}
\begin{datadesc}{ETIMEDOUT} Connection timed out \end{datadesc}
\begin{datadesc}{ECONNREFUSED} Connection refused \end{datadesc}
\begin{datadesc}{EHOSTDOWN} Host is down \end{datadesc}
\begin{datadesc}{EHOSTUNREACH} No route to host \end{datadesc}
\begin{datadesc}{EALREADY} Operation already in progress \end{datadesc}
\begin{datadesc}{EINPROGRESS} Operation now in progress \end{datadesc}
\begin{datadesc}{ESTALE} Stale NFS file handle \end{datadesc}
\begin{datadesc}{EUCLEAN} Structure needs cleaning \end{datadesc}
\begin{datadesc}{ENOTNAM} Not a XENIX named type file \end{datadesc}
\begin{datadesc}{ENAVAIL} No XENIX semaphores available \end{datadesc}
\begin{datadesc}{EISNAM} Is a named type file \end{datadesc}
\begin{datadesc}{EREMOTEIO} Remote I/O error \end{datadesc}
\begin{datadesc}{EDQUOT} Quota exceeded \end{datadesc}



\chapter{Optional Operating System Services}

The modules described in this chapter provide interfaces to operating
system features that are available on selected operating systems only.
The interfaces are generally modelled after the \UNIX{} or C
interfaces but they are available on some other systems as well
(e.g. Windows or NT).  Here's an overview:

\begin{description}

\item[signal]
--- Set handlers for asynchronous events.

\item[socket]
--- Low-level networking interface.

\item[select]
--- Wait for I/O completion on multiple streams.

\item[thread]
--- Create multiple threads of control within one namespace.

\end{description}
		% Optional Operating System Services
\section{Built-in Module \sectcode{signal}}

\bimodindex{signal}
This module provides mechanisms to use signal handlers in Python.
Some general rules for working with signals handlers:

\begin{itemize}

\item
A handler for a particular signal, once set, remains installed until
it is explicitly reset (i.e. Python emulates the BSD style interface
regardless of the underlying implementation), with the exception of
the handler for \code{SIGCHLD}, which follows the underlying
implementation.

\item
There is no way to ``block'' signals temporarily from critical
sections (since this is not supported by all \UNIX{} flavors).

\item
Although Python signal handlers are called asynchronously as far as
the Python user is concerned, they can only occur between the
``atomic'' instructions of the Python interpreter.  This means that
signals arriving during long calculations implemented purely in C
(e.g.\ regular expression matches on large bodies of text) may be
delayed for an arbitrary amount of time.

\item
When a signal arrives during an I/O operation, it is possible that the
I/O operation raises an exception after the signal handler returns.
This is dependent on the underlying \UNIX{} system's semantics regarding
interrupted system calls.

\item
Because the C signal handler always returns, it makes little sense to
catch synchronous errors like \code{SIGFPE} or \code{SIGSEGV}.

\item
Python installs a small number of signal handlers by default:
\code{SIGPIPE} is ignored (so write errors on pipes and sockets can be
reported as ordinary Python exceptions), \code{SIGINT} is translated
into a \code{KeyboardInterrupt} exception, and \code{SIGTERM} is
caught so that necessary cleanup (especially \code{sys.exitfunc}) can
be performed before actually terminating.  All of these can be
overridden.

\item
Some care must be taken if both signals and threads are used in the
same program.  The fundamental thing to remember in using signals and
threads simultaneously is:\ always perform \code{signal()} operations
in the main thread of execution.  Any thread can perform an
\code{alarm()}, \code{getsignal()}, or \code{pause()}; only the main
thread can set a new signal handler, and the main thread will be the
only one to receive signals (this is enforced by the Python signal
module, even if the underlying thread implementation supports sending
signals to individual threads).  This means that signals can't be used
as a means of interthread communication.  Use locks instead.

\end{itemize}

The variables defined in the signal module are:

\renewcommand{\indexsubitem}{(in module signal)}
\begin{datadesc}{SIG_DFL}
  This is one of two standard signal handling options; it will simply
  perform the default function for the signal.  For example, on most
  systems the default action for SIGQUIT is to dump core and exit,
  while the default action for SIGCLD is to simply ignore it.
\end{datadesc}

\begin{datadesc}{SIG_IGN}
  This is another standard signal handler, which will simply ignore
  the given signal.
\end{datadesc}

\begin{datadesc}{SIG*}
  All the signal numbers are defined symbolically.  For example, the
  hangup signal is defined as \code{signal.SIGHUP}; the variable names
  are identical to the names used in C programs, as found in
  \file{signal.h}.
  The \UNIX{} man page for \file{signal} lists the existing signals (on
  some systems this is \file{signal(2)}, on others the list is in
  \file{signal(7)}).
  Note that not all systems define the same set of signal names; only
  those names defined by the system are defined by this module.
\end{datadesc}

\begin{datadesc}{NSIG}
  One more than the number of the highest signal number.
\end{datadesc}

The signal module defines the following functions:

\begin{funcdesc}{alarm}{time}
  If \var{time} is non-zero, this function requests that a
  \code{SIGALRM} signal be sent to the process in \var{time} seconds.
  Any previously scheduled alarm is canceled (i.e.\ only one alarm can
  be scheduled at any time).  The returned value is then the number of
  seconds before any previously set alarm was to have been delivered.
  If \var{time} is zero, no alarm id scheduled, and any scheduled
  alarm is canceled.  The return value is the number of seconds
  remaining before a previously scheduled alarm.  If the return value
  is zero, no alarm is currently scheduled.  (See the \UNIX{} man page
  \code{alarm(2)}.)
\end{funcdesc}

\begin{funcdesc}{getsignal}{signalnum}
  Return the current signal handler for the signal \var{signalnum}.
  The returned value may be a callable Python object, or one of the
  special values \code{signal.SIG_IGN}, \code{signal.SIG_DFL} or
  \code{None}.  Here, \code{signal.SIG_IGN} means that the signal was
  previously ignored, \code{signal.SIG_DFL} means that the default way
  of handling the signal was previously in use, and \code{None} means
  that the previous signal handler was not installed from Python.
\end{funcdesc}

\begin{funcdesc}{pause}{}
  Cause the process to sleep until a signal is received; the
  appropriate handler will then be called.  Returns nothing.  (See the
  \UNIX{} man page \code{signal(2)}.)
\end{funcdesc}

\begin{funcdesc}{signal}{signalnum\, handler}
  Set the handler for signal \var{signalnum} to the function
  \var{handler}.  \var{handler} can be any callable Python object, or
  one of the special values \code{signal.SIG_IGN} or
  \code{signal.SIG_DFL}.  The previous signal handler will be returned
  (see the description of \code{getsignal()} above).  (See the \UNIX{}
  man page \code{signal(2)}.)

  When threads are enabled, this function can only be called from the
  main thread; attempting to call it from other threads will cause a
  \code{ValueError} exception to be raised.

  The \var{handler} is called with two arguments: the signal number
  and the current stack frame (\code{None} or a frame object; see the
  reference manual for a description of frame objects).
\obindex{frame}
\end{funcdesc}

\section{Built-in Module \sectcode{socket}}

\bimodindex{socket}
This module provides access to the BSD {\em socket} interface.
It is available on \UNIX{} systems that support this interface.

For an introduction to socket programming (in C), see the following
papers: \emph{An Introductory 4.3BSD Interprocess Communication
Tutorial}, by Stuart Sechrest and \emph{An Advanced 4.3BSD Interprocess
Communication Tutorial}, by Samuel J.  Leffler et al, both in the
\UNIX{} Programmer's Manual, Supplementary Documents 1 (sections PS1:7
and PS1:8).  The \UNIX{} manual pages for the various socket-related
system calls are also a valuable source of information on the details of
socket semantics.

The Python interface is a straightforward transliteration of the
\UNIX{} system call and library interface for sockets to Python's
object-oriented style: the \code{socket()} function returns a
\dfn{socket object} whose methods implement the various socket system
calls.  Parameter types are somewhat higer-level than in the C
interface: as with \code{read()} and \code{write()} operations on Python
files, buffer allocation on receive operations is automatic, and
buffer length is implicit on send operations.

Socket addresses are represented as a single string for the
\code{AF_UNIX} address family and as a pair
\code{(\var{host}, \var{port})} for the \code{AF_INET} address family,
where \var{host} is a string representing
either a hostname in Internet domain notation like
\code{'daring.cwi.nl'} or an IP address like \code{'100.50.200.5'},
and \var{port} is an integral port number.  Other address families are
currently not supported.  The address format required by a particular
socket object is automatically selected based on the address family
specified when the socket object was created.

All errors raise exceptions.  The normal exceptions for invalid
argument types and out-of-memory conditions can be raised; errors
related to socket or address semantics raise the error \code{socket.error}.

Non-blocking mode is supported through the \code{setblocking()}
method.

The module \code{socket} exports the following constants and functions:

\renewcommand{\indexsubitem}{(in module socket)}
\begin{excdesc}{error}
This exception is raised for socket- or address-related errors.
The accompanying value is either a string telling what went wrong or a
pair \code{(\var{errno}, \var{string})}
representing an error returned by a system
call, similar to the value accompanying \code{posix.error}.
\end{excdesc}

\begin{datadesc}{AF_UNIX}
\dataline{AF_INET}
These constants represent the address (and protocol) families,
used for the first argument to \code{socket()}.  If the \code{AF_UNIX}
constant is not defined then this protocol is unsupported.
\end{datadesc}

\begin{datadesc}{SOCK_STREAM}
\dataline{SOCK_DGRAM}
\dataline{SOCK_RAW}
\dataline{SOCK_RDM}
\dataline{SOCK_SEQPACKET}
These constants represent the socket types,
used for the second argument to \code{socket()}.
(Only \code{SOCK_STREAM} and
\code{SOCK_DGRAM} appear to be generally useful.)
\end{datadesc}

\begin{datadesc}{SO_*}
\dataline{SOMAXCONN}
\dataline{MSG_*}
\dataline{SOL_*}
\dataline{IPPROTO_*}
\dataline{IPPORT_*}
\dataline{INADDR_*}
\dataline{IP_*}
Many constants of these forms, documented in the \UNIX{} documentation on
sockets and/or the IP protocol, are also defined in the socket module.
They are generally used in arguments to the \code{setsockopt} and
\code{getsockopt} methods of socket objects.  In most cases, only
those symbols that are defined in the \UNIX{} header files are defined;
for a few symbols, default values are provided.
\end{datadesc}

\begin{funcdesc}{gethostbyname}{hostname}
Translate a host name to IP address format.  The IP address is
returned as a string, e.g.,  \code{'100.50.200.5'}.  If the host name
is an IP address itself it is returned unchanged.
\end{funcdesc}

\begin{funcdesc}{gethostname}{}
Return a string containing the hostname of the machine where 
the Python interpreter is currently executing.  If you want to know the
current machine's IP address, use
\code{socket.gethostbyname(socket.gethostname())}.
\end{funcdesc}

\begin{funcdesc}{gethostbyaddr}{ip_address}
Return a triple \code{(hostname, aliaslist, ipaddrlist)} where
\code{hostname} is the primary host name responding to the given
\var{ip_address}, \code{aliaslist} is a (possibly empty) list of
alternative host names for the same address, and \code{ipaddrlist} is
a list of IP addresses for the same interface on the same
host (most likely containing only a single address).
\end{funcdesc}

\begin{funcdesc}{getservbyname}{servicename\, protocolname}
Translate an Internet service name and protocol name to a port number
for that service.  The protocol name should be \code{'tcp'} or
\code{'udp'}.
\end{funcdesc}

\begin{funcdesc}{socket}{family\, type\optional{\, proto}}
Create a new socket using the given address family, socket type and
protocol number.  The address family should be \code{AF_INET} or
\code{AF_UNIX}.  The socket type should be \code{SOCK_STREAM},
\code{SOCK_DGRAM} or perhaps one of the other \samp{SOCK_} constants.
The protocol number is usually zero and may be omitted in that case.
\end{funcdesc}

\begin{funcdesc}{fromfd}{fd\, family\, type\optional{\, proto}}
Build a socket object from an existing file descriptor (an integer as
returned by a file object's \code{fileno} method).  Address family,
socket type and protocol number are as for the \code{socket} function
above.  The file descriptor should refer to a socket, but this is not
checked --- subsequent operations on the object may fail if the file
descriptor is invalid.  This function is rarely needed, but can be
used to get or set socket options on a socket passed to a program as
standard input or output (e.g.\ a server started by the \UNIX{} inet
daemon).
\end{funcdesc}

\subsection{Socket Objects}

\noindent
Socket objects have the following methods.  Except for
\code{makefile()} these correspond to \UNIX{} system calls applicable to
sockets.

\renewcommand{\indexsubitem}{(socket method)}
\begin{funcdesc}{accept}{}
Accept a connection.
The socket must be bound to an address and listening for connections.
The return value is a pair \code{(\var{conn}, \var{address})}
where \var{conn} is a \emph{new} socket object usable to send and
receive data on the connection, and \var{address} is the address bound
to the socket on the other end of the connection.
\end{funcdesc}

\begin{funcdesc}{bind}{address}
Bind the socket to \var{address}.  The socket must not already be bound.
(The format of \var{address} depends on the address family --- see above.)
\end{funcdesc}

\begin{funcdesc}{close}{}
Close the socket.  All future operations on the socket object will fail.
The remote end will receive no more data (after queued data is flushed).
Sockets are automatically closed when they are garbage-collected.
\end{funcdesc}

\begin{funcdesc}{connect}{address}
Connect to a remote socket at \var{address}.
(The format of \var{address} depends on the address family --- see above.)
\end{funcdesc}

\begin{funcdesc}{fileno}{}
Return the socket's file descriptor (a small integer).  This is useful
with \code{select}.
\end{funcdesc}

\begin{funcdesc}{getpeername}{}
Return the remote address to which the socket is connected.  This is
useful to find out the port number of a remote IP socket, for instance.
(The format of the address returned depends on the address family ---
see above.)  On some systems this function is not supported.
\end{funcdesc}

\begin{funcdesc}{getsockname}{}
Return the socket's own address.  This is useful to find out the port
number of an IP socket, for instance.
(The format of the address returned depends on the address family ---
see above.)
\end{funcdesc}

\begin{funcdesc}{getsockopt}{level\, optname\optional{\, buflen}}
Return the value of the given socket option (see the \UNIX{} man page
{\it getsockopt}(2)).  The needed symbolic constants (\code{SO_*} etc.)
are defined in this module.  If \var{buflen}
is absent, an integer option is assumed and its integer value
is returned by the function.  If \var{buflen} is present, it specifies
the maximum length of the buffer used to receive the option in, and
this buffer is returned as a string.  It is up to the caller to decode
the contents of the buffer (see the optional built-in module
\code{struct} for a way to decode C structures encoded as strings).
\end{funcdesc}

\begin{funcdesc}{listen}{backlog}
Listen for connections made to the socket.  The \var{backlog} argument
specifies the maximum number of queued connections and should be at
least 1; the maximum value is system-dependent (usually 5).
\end{funcdesc}

\begin{funcdesc}{makefile}{\optional{mode\optional{\, bufsize}}}
Return a \dfn{file object} associated with the socket.  (File objects
were described earlier under Built-in Types.)  The file object
references a \code{dup()}ped version of the socket file descriptor, so
the file object and socket object may be closed or garbage-collected
independently.  The optional \var{mode} and \var{bufsize} arguments
are interpreted the same way as by the built-in
\code{open()} function.
\end{funcdesc}

\begin{funcdesc}{recv}{bufsize\optional{\, flags}}
Receive data from the socket.  The return value is a string representing
the data received.  The maximum amount of data to be received
at once is specified by \var{bufsize}.  See the \UNIX{} manual page
for the meaning of the optional argument \var{flags}; it defaults to
zero.
\end{funcdesc}

\begin{funcdesc}{recvfrom}{bufsize\optional{\, flags}}
Receive data from the socket.  The return value is a pair
\code{(\var{string}, \var{address})} where \var{string} is a string
representing the data received and \var{address} is the address of the
socket sending the data.  The optional \var{flags} argument has the
same meaning as for \code{recv()} above.
(The format of \var{address} depends on the address family --- see above.)
\end{funcdesc}

\begin{funcdesc}{send}{string\optional{\, flags}}
Send data to the socket.  The socket must be connected to a remote
socket.  The optional \var{flags} argument has the same meaning as for
\code{recv()} above.  Return the number of bytes sent.
\end{funcdesc}

\begin{funcdesc}{sendto}{string\optional{\, flags}\, address}
Send data to the socket.  The socket should not be connected to a
remote socket, since the destination socket is specified by
\code{address}.  The optional \var{flags} argument has the same
meaning as for \code{recv()} above.  Return the number of bytes sent.
(The format of \var{address} depends on the address family --- see above.)
\end{funcdesc}

\begin{funcdesc}{setblocking}{flag}
Set blocking or non-blocking mode of the socket: if \var{flag} is 0,
the socket is set to non-blocking, else to blocking mode.  Initially
all sockets are in blocking mode.  In non-blocking mode, if a
\code{recv} call doesn't find any data, or if a \code{send} call can't
immediately dispose of the data, a \code{socket.error} exception is
raised; in blocking mode, the calls block until they can proceed.
\end{funcdesc}

\begin{funcdesc}{setsockopt}{level\, optname\, value}
Set the value of the given socket option (see the \UNIX{} man page
{\it setsockopt}(2)).  The needed symbolic constants are defined in
the \code{socket} module (\code{SO_*} etc.).  The value can be an
integer or a string representing a buffer.  In the latter case it is
up to the caller to ensure that the string contains the proper bits
(see the optional built-in module
\code{struct} for a way to encode C structures as strings).
\end{funcdesc}

\begin{funcdesc}{shutdown}{how}
Shut down one or both halves of the connection.  If \var{how} is \code{0},
further receives are disallowed.  If \var{how} is \code{1}, further sends are
disallowed.  If \var{how} is \code{2}, further sends and receives are
disallowed.
\end{funcdesc}

Note that there are no methods \code{read()} or \code{write()}; use
\code{recv()} and \code{send()} without \var{flags} argument instead.

\subsection{Example}
\nodename{Socket Example}

Here are two minimal example programs using the TCP/IP protocol:\ a
server that echoes all data that it receives back (servicing only one
client), and a client using it.  Note that a server must perform the
sequence \code{socket}, \code{bind}, \code{listen}, \code{accept}
(possibly repeating the \code{accept} to service more than one client),
while a client only needs the sequence \code{socket}, \code{connect}.
Also note that the server does not \code{send}/\code{receive} on the
socket it is listening on but on the new socket returned by
\code{accept}.

\bcode\begin{verbatim}
# Echo server program
from socket import *
HOST = ''                 # Symbolic name meaning the local host
PORT = 50007              # Arbitrary non-privileged server
s = socket(AF_INET, SOCK_STREAM)
s.bind(HOST, PORT)
s.listen(1)
conn, addr = s.accept()
print 'Connected by', addr
while 1:
    data = conn.recv(1024)
    if not data: break
    conn.send(data)
conn.close()
\end{verbatim}\ecode

\bcode\begin{verbatim}
# Echo client program
from socket import *
HOST = 'daring.cwi.nl'    # The remote host
PORT = 50007              # The same port as used by the server
s = socket(AF_INET, SOCK_STREAM)
s.connect(HOST, PORT)
s.send('Hello, world')
data = s.recv(1024)
s.close()
print 'Received', `data`
\end{verbatim}\ecode

\section{Built-in Module \sectcode{select}}
\bimodindex{select}

This module provides access to the function \code{select} available in
most \UNIX{} versions.  It defines the following:

\renewcommand{\indexsubitem}{(in module select)}
\begin{excdesc}{error}
The exception raised when an error occurs.  The accompanying value is
a pair containing the numeric error code from \code{errno} and the
corresponding string, as would be printed by the C function
\code{perror()}.
\end{excdesc}

\begin{funcdesc}{select}{iwtd\, owtd\, ewtd\optional{\, timeout}}
This is a straightforward interface to the \UNIX{} \code{select()}
system call.  The first three arguments are lists of `waitable
objects': either integers representing \UNIX{} file descriptors or
objects with a parameterless method named \code{fileno()} returning
such an integer.  The three lists of waitable objects are for input,
output and `exceptional conditions', respectively.  Empty lists are
allowed.  The optional \var{timeout} argument specifies a time-out as a
floating point number in seconds.  When the \var{timeout} argument
is omitted the function blocks until at least one file descriptor is
ready.  A time-out value of zero specifies a poll and never blocks.

The return value is a triple of lists of objects that are ready:
subsets of the first three arguments.  When the time-out is reached
without a file descriptor becoming ready, three empty lists are
returned.

Amongst the acceptable object types in the lists are Python file
objects (e.g. \code{sys.stdin}, or objects returned by \code{open()}
or \code{posix.popen()}), socket objects returned by
\code{socket.socket()}, and the module \code{stdwin} which happens to
define a function \code{fileno()} for just this purpose.  You may
also define a \dfn{wrapper} class yourself, as long as it has an
appropriate \code{fileno()} method (that really returns a \UNIX{} file
descriptor, not just a random integer).
\end{funcdesc}
\ttindex{socket}
\ttindex{stdwin}

\section{Built-in Module \sectcode{thread}}
\bimodindex{thread}

This module provides low-level primitives for working with multiple
threads (a.k.a.\ \dfn{light-weight processes} or \dfn{tasks}) --- multiple
threads of control sharing their global data space.  For
synchronization, simple locks (a.k.a.\ \dfn{mutexes} or \dfn{binary
semaphores}) are provided.

The module is optional and supported on SGI IRIX 4.x and 5.x and Sun
Solaris 2.x systems, as well as on systems that have a PTHREAD
implementation (e.g.\ KSR).

It defines the following constant and functions:

\renewcommand{\indexsubitem}{(in module thread)}
\begin{excdesc}{error}
Raised on thread-specific errors.
\end{excdesc}

\begin{funcdesc}{start_new_thread}{func\, arg}
Start a new thread.  The thread executes the function \var{func}
with the argument list \var{arg} (which must be a tuple).  When the
function returns, the thread silently exits.  When the function
terminates with an unhandled exception, a stack trace is printed and
then the thread exits (but other threads continue to run).
\end{funcdesc}

\begin{funcdesc}{exit}{}
This is a shorthand for \code{thread.exit_thread()}.
\end{funcdesc}

\begin{funcdesc}{exit_thread}{}
Raise the \code{SystemExit} exception.  When not caught, this will
cause the thread to exit silently.
\end{funcdesc}

%\begin{funcdesc}{exit_prog}{status}
%Exit all threads and report the value of the integer argument
%\var{status} as the exit status of the entire program.
%\strong{Caveat:} code in pending \code{finally} clauses, in this thread
%or in other threads, is not executed.
%\end{funcdesc}

\begin{funcdesc}{allocate_lock}{}
Return a new lock object.  Methods of locks are described below.  The
lock is initially unlocked.
\end{funcdesc}

\begin{funcdesc}{get_ident}{}
Return the `thread identifier' of the current thread.  This is a
nonzero integer.  Its value has no direct meaning; it is intended as a
magic cookie to be used e.g. to index a dictionary of thread-specific
data.  Thread identifiers may be recycled when a thread exits and
another thread is created.
\end{funcdesc}

Lock objects have the following methods:

\renewcommand{\indexsubitem}{(lock method)}
\begin{funcdesc}{acquire}{\optional{waitflag}}
Without the optional argument, this method acquires the lock
unconditionally, if necessary waiting until it is released by another
thread (only one thread at a time can acquire a lock --- that's their
reason for existence), and returns \code{None}.  If the integer
\var{waitflag} argument is present, the action depends on its value:\
if it is zero, the lock is only acquired if it can be acquired
immediately without waiting, while if it is nonzero, the lock is
acquired unconditionally as before.  If an argument is present, the
return value is 1 if the lock is acquired successfully, 0 if not.
\end{funcdesc}

\begin{funcdesc}{release}{}
Releases the lock.  The lock must have been acquired earlier, but not
necessarily by the same thread.
\end{funcdesc}

\begin{funcdesc}{locked}{}
Return the status of the lock:\ 1 if it has been acquired by some
thread, 0 if not.
\end{funcdesc}

{\bf Caveats:}

\begin{itemize}
\item
Threads interact strangely with interrupts: the
\code{KeyboardInterrupt} exception will be received by an arbitrary
thread.  (When the \code{signal} module is available, interrupts
always go to the main thread.)

\item
Calling \code{sys.exit()} or raising the \code{SystemExit} is
equivalent to calling \code{thread.exit_thread()}.

\item
Not all built-in functions that may block waiting for I/O allow other
threads to run.  (The most popular ones (\code{sleep}, \code{read},
\code{select}) work as expected.)

\end{itemize}


\chapter{UNIX Specific Services}

The modules described in this chapter provide interfaces to features
that are unique to the \UNIX{} operating system, or in some cases to
some or many variants of it.  Here's an overview:

\begin{description}

\item[posix]
--- The most common Posix system calls (normally used via module \code{os}).

\item[posixpath]
--- Common Posix pathname manipulations (normally used via \code{os.path}).

\item[pwd]
--- The password database (\code{getpwnam()} and friends).

\item[grp]
--- The group database (\code{getgrnam()} and friends).

\item[crypt]
--- The (\code{crypt()} function used to check Unix passwords).

\item[dbm]
--- The standard ``database'' interface, based on \code{ndbm}.

\item[gdbm]
--- GNU's reinterpretation of dbm.

\item[termios]
--- Posix style tty control.

\item[fcntl]
--- The \code{fcntl()} and \code{ioctl()} system calls.

\item[posixfile]
--- A file-like object with support for locking.

\end{description}
			% UNIX Specific Services
\section{Built-in Module \sectcode{posix}}
\bimodindex{posix}

This module provides access to operating system functionality that is
standardized by the C Standard and the POSIX standard (a thinly disguised
\UNIX{} interface).

\strong{Do not import this module directly.}  Instead, import the
module \code{os}, which provides a \emph{portable} version of this
interface.  On \UNIX{}, the \code{os} module provides a superset of
the \code{posix} interface.  On non-\UNIX{} operating systems the
\code{posix} module is not available, but a subset is always available
through the \code{os} interface.  Once \code{os} is imported, there is
\emph{no} performance penalty in using it instead of
\code{posix}.
\stmodindex{os}

The descriptions below are very terse; refer to the
corresponding \UNIX{} manual entry for more information.  Arguments
called \var{path} refer to a pathname given as a string.

Errors are reported as exceptions; the usual exceptions are given
for type errors, while errors reported by the system calls raise
\code{posix.error}, described below.

Module \code{posix} defines the following data items:

\renewcommand{\indexsubitem}{(data in module posix)}
\begin{datadesc}{environ}
A dictionary representing the string environment at the time
the interpreter was started.
For example,
\code{posix.environ['HOME']}
is the pathname of your home directory, equivalent to
\code{getenv("HOME")}
in C.
Modifying this dictionary does not affect the string environment
passed on by \code{execv()}, \code{popen()} or \code{system()}; if you
need to change the environment, pass \code{environ} to \code{execve()}
or add variable assignments and export statements to the command
string for \code{system()} or \code{popen()}.%
\footnote{The problem with automatically passing on \code{environ} is
that there is no portable way of changing the environment.}
\end{datadesc}

\renewcommand{\indexsubitem}{(exception in module posix)}
\begin{excdesc}{error}
This exception is raised when a POSIX function returns a
POSIX-related error (e.g., not for illegal argument types).  Its
string value is \code{'posix.error'}.  The accompanying value is a
pair containing the numeric error code from \code{errno} and the
corresponding string, as would be printed by the C function
\code{perror()}.
\end{excdesc}

It defines the following functions and constants:

\renewcommand{\indexsubitem}{(in module posix)}
\begin{funcdesc}{chdir}{path}
Change the current working directory to \var{path}.
\end{funcdesc}

\begin{funcdesc}{chmod}{path\, mode}
Change the mode of \var{path} to the numeric \var{mode}.
\end{funcdesc}

\begin{funcdesc}{chown}{path\, uid, gid}
Change the owner and group id of \var{path} to the numeric \var{uid}
and \var{gid}.
(Not on MS-DOS.)
\end{funcdesc}

\begin{funcdesc}{close}{fd}
Close file descriptor \var{fd}.

Note: this function is intended for low-level I/O and must be applied
to a file descriptor as returned by \code{posix.open()} or
\code{posix.pipe()}.  To close a ``file object'' returned by the
built-in function \code{open} or by \code{posix.popen} or
\code{posix.fdopen}, use its \code{close()} method.
\end{funcdesc}

\begin{funcdesc}{dup}{fd}
Return a duplicate of file descriptor \var{fd}.
\end{funcdesc}

\begin{funcdesc}{dup2}{fd\, fd2}
Duplicate file descriptor \var{fd} to \var{fd2}, closing the latter
first if necessary.  Return \code{None}.
\end{funcdesc}

\begin{funcdesc}{execv}{path\, args}
Execute the executable \var{path} with argument list \var{args},
replacing the current process (i.e., the Python interpreter).
The argument list may be a tuple or list of strings.
(Not on MS-DOS.)
\end{funcdesc}

\begin{funcdesc}{execve}{path\, args\, env}
Execute the executable \var{path} with argument list \var{args},
and environment \var{env},
replacing the current process (i.e., the Python interpreter).
The argument list may be a tuple or list of strings.
The environment must be a dictionary mapping strings to strings.
(Not on MS-DOS.)
\end{funcdesc}

\begin{funcdesc}{_exit}{n}
Exit to the system with status \var{n}, without calling cleanup
handlers, flushing stdio buffers, etc.
(Not on MS-DOS.)

Note: the standard way to exit is \code{sys.exit(\var{n})}.
\code{posix._exit()} should normally only be used in the child process
after a \code{fork()}.
\end{funcdesc}

\begin{funcdesc}{fdopen}{fd\optional{\, mode\optional{\, bufsize}}}
Return an open file object connected to the file descriptor \var{fd}.
The \var{mode} and \var{bufsize} arguments have the same meaning as
the corresponding arguments to the built-in \code{open()} function.
\end{funcdesc}

\begin{funcdesc}{fork}{}
Fork a child process.  Return 0 in the child, the child's process id
in the parent.
(Not on MS-DOS.)
\end{funcdesc}

\begin{funcdesc}{fstat}{fd}
Return status for file descriptor \var{fd}, like \code{stat()}.
\end{funcdesc}

\begin{funcdesc}{getcwd}{}
Return a string representing the current working directory.
\end{funcdesc}

\begin{funcdesc}{getegid}{}
Return the current process's effective group id.
(Not on MS-DOS.)
\end{funcdesc}

\begin{funcdesc}{geteuid}{}
Return the current process's effective user id.
(Not on MS-DOS.)
\end{funcdesc}

\begin{funcdesc}{getgid}{}
Return the current process's group id.
(Not on MS-DOS.)
\end{funcdesc}

\begin{funcdesc}{getpgrp}{}
Return the current process group id.
(Not on MS-DOS.)
\end{funcdesc}

\begin{funcdesc}{getpid}{}
Return the current process id.
(Not on MS-DOS.)
\end{funcdesc}

\begin{funcdesc}{getppid}{}
Return the parent's process id.
(Not on MS-DOS.)
\end{funcdesc}

\begin{funcdesc}{getuid}{}
Return the current process's user id.
(Not on MS-DOS.)
\end{funcdesc}

\begin{funcdesc}{kill}{pid\, sig}
Kill the process \var{pid} with signal \var{sig}.
(Not on MS-DOS.)
\end{funcdesc}

\begin{funcdesc}{link}{src\, dst}
Create a hard link pointing to \var{src} named \var{dst}.
(Not on MS-DOS.)
\end{funcdesc}

\begin{funcdesc}{listdir}{path}
Return a list containing the names of the entries in the directory.
The list is in arbitrary order.  It does not include the special
entries \code{'.'} and \code{'..'} even if they are present in the
directory.
\end{funcdesc}

\begin{funcdesc}{lseek}{fd\, pos\, how}
Set the current position of file descriptor \var{fd} to position
\var{pos}, modified by \var{how}: 0 to set the position relative to
the beginning of the file; 1 to set it relative to the current
position; 2 to set it relative to the end of the file.
\end{funcdesc}

\begin{funcdesc}{lstat}{path}
Like \code{stat()}, but do not follow symbolic links.  (On systems
without symbolic links, this is identical to \code{posix.stat}.)
\end{funcdesc}

\begin{funcdesc}{mkfifo}{path\optional{\, mode}}
Create a FIFO (a POSIX named pipe) named \var{path} with numeric mode
\var{mode}.  The default \var{mode} is 0666 (octal).  The current
umask value is first masked out from the mode.
(Not on MS-DOS.)

FIFOs are pipes that can be accessed like regular files.  FIFOs exist
until they are deleted (for example with \code{os.unlink}).
Generally, FIFOs are used as rendez-vous between ``client'' and
``server'' type processes: the server opens the FIFO for reading, and
the client opens it for writing.  Note that \code{mkfifo()} doesn't
open the FIFO -- it just creates the rendez-vous point.
\end{funcdesc}

\begin{funcdesc}{mkdir}{path\optional{\, mode}}
Create a directory named \var{path} with numeric mode \var{mode}.
The default \var{mode} is 0777 (octal).  On some systems, \var{mode}
is ignored.  Where it is used, the current umask value is first
masked out.
\end{funcdesc}

\begin{funcdesc}{nice}{increment}
Add \var{incr} to the process' ``niceness''.  Return the new niceness.
(Not on MS-DOS.)
\end{funcdesc}

\begin{funcdesc}{open}{file\, flags\, mode}
Open the file \var{file} and set various flags according to
\var{flags} and possibly its mode according to \var{mode}.
Return the file descriptor for the newly opened file.

Note: this function is intended for low-level I/O.  For normal usage,
use the built-in function \code{open}, which returns a ``file object''
with \code{read()} and  \code{write()} methods (and many more).
\end{funcdesc}

\begin{funcdesc}{pipe}{}
Create a pipe.  Return a pair of file descriptors \code{(r, w)}
usable for reading and writing, respectively.
(Not on MS-DOS.)
\end{funcdesc}

\begin{funcdesc}{plock}{op}
Lock program segments into memory.  The value of \var{op}
(defined in \code{<sys/lock.h>}) determines which segments are locked.
(Not on MS-DOS.)
\end{funcdesc}

\begin{funcdesc}{popen}{command\optional{\, mode\optional{\, bufsize}}}
Open a pipe to or from \var{command}.  The return value is an open
file object connected to the pipe, which can be read or written
depending on whether \var{mode} is \code{'r'} (default) or \code{'w'}.
The \var{bufsize} argument has the same meaning as the corresponding
argument to the built-in \code{open()} function.
(Not on MS-DOS.)
\end{funcdesc}

\begin{funcdesc}{read}{fd\, n}
Read at most \var{n} bytes from file descriptor \var{fd}.
Return a string containing the bytes read.

Note: this function is intended for low-level I/O and must be applied
to a file descriptor as returned by \code{posix.open()} or
\code{posix.pipe()}.  To read a ``file object'' returned by the
built-in function \code{open} or by \code{posix.popen} or
\code{posix.fdopen}, or \code{sys.stdin}, use its
\code{read()} or \code{readline()} methods.
\end{funcdesc}

\begin{funcdesc}{readlink}{path}
Return a string representing the path to which the symbolic link
points.  (On systems without symbolic links, this always raises
\code{posix.error}.)
\end{funcdesc}

\begin{funcdesc}{remove}{path}
Remove the file \var{path}.  See \code{rmdir} below to remove a directory.
\end{funcdesc}

\begin{funcdesc}{rename}{src\, dst}
Rename the file or directory \var{src} to \var{dst}.
\end{funcdesc}

\begin{funcdesc}{rmdir}{path}
Remove the directory \var{path}.
\end{funcdesc}

\begin{funcdesc}{setgid}{gid}
Set the current process's group id.
(Not on MS-DOS.)
\end{funcdesc}

\begin{funcdesc}{setpgrp}{}
Calls the system call \code{setpgrp()} or \code{setpgrp(0, 0)}
depending on which version is implemented (if any).  See the {\UNIX}
manual for the semantics.
(Not on MS-DOS.)
\end{funcdesc}

\begin{funcdesc}{setpgid}{pid\, pgrp}
Calls the system call \code{setpgid()}.  See the {\UNIX} manual for
the semantics.
(Not on MS-DOS.)
\end{funcdesc}

\begin{funcdesc}{setsid}{}
Calls the system call \code{setsid()}.  See the {\UNIX} manual for the
semantics.
(Not on MS-DOS.)
\end{funcdesc}

\begin{funcdesc}{setuid}{uid}
Set the current process's user id.
(Not on MS-DOS.)
\end{funcdesc}

\begin{funcdesc}{stat}{path}
Perform a {\em stat} system call on the given path.  The return value
is a tuple of at least 10 integers giving the most important (and
portable) members of the {\em stat} structure, in the order
\code{st_mode},
\code{st_ino},
\code{st_dev},
\code{st_nlink},
\code{st_uid},
\code{st_gid},
\code{st_size},
\code{st_atime},
\code{st_mtime},
\code{st_ctime}.
More items may be added at the end by some implementations.
(On MS-DOS, some items are filled with dummy values.)

Note: The standard module \code{stat} defines functions and constants
that are useful for extracting information from a stat structure.
\end{funcdesc}

\begin{funcdesc}{symlink}{src\, dst}
Create a symbolic link pointing to \var{src} named \var{dst}.  (On
systems without symbolic links, this always raises
\code{posix.error}.)
\end{funcdesc}

\begin{funcdesc}{system}{command}
Execute the command (a string) in a subshell.  This is implemented by
calling the Standard C function \code{system()}, and has the same
limitations.  Changes to \code{posix.environ}, \code{sys.stdin} etc.\ are
not reflected in the environment of the executed command.  The return
value is the exit status of the process as returned by Standard C
\code{system()}.
\end{funcdesc}

\begin{funcdesc}{tcgetpgrp}{fd}
Return the process group associated with the terminal given by
\var{fd} (an open file descriptor as returned by \code{posix.open()}).
(Not on MS-DOS.)
\end{funcdesc}

\begin{funcdesc}{tcsetpgrp}{fd\, pg}
Set the process group associated with the terminal given by
\var{fd} (an open file descriptor as returned by \code{posix.open()})
to \var{pg}.
(Not on MS-DOS.)
\end{funcdesc}

\begin{funcdesc}{times}{}
Return a 5-tuple of floating point numbers indicating accumulated (CPU
or other)
times, in seconds.  The items are: user time, system time, children's
user time, children's system time, and elapsed real time since a fixed
point in the past, in that order.  See the \UNIX{}
manual page {\it times}(2).  (Not on MS-DOS.)
\end{funcdesc}

\begin{funcdesc}{umask}{mask}
Set the current numeric umask and returns the previous umask.
(Not on MS-DOS.)
\end{funcdesc}

\begin{funcdesc}{uname}{}
Return a 5-tuple containing information identifying the current
operating system.  The tuple contains 5 strings:
\code{(\var{sysname}, \var{nodename}, \var{release}, \var{version}, \var{machine})}.
Some systems truncate the nodename to 8
characters or to the leading component; a better way to get the
hostname is \code{socket.gethostname()}.  (Not on MS-DOS, nor on older
\UNIX{} systems.)
\end{funcdesc}

\begin{funcdesc}{unlink}{path}
Remove the file \var{path}.  This is the same function as \code{remove};
the \code{unlink} name is its traditional \UNIX{} name.
\end{funcdesc}

\begin{funcdesc}{utime}{path\, \(atime\, mtime\)}
Set the access and modified time of the file to the given values.
(The second argument is a tuple of two items.)
\end{funcdesc}

\begin{funcdesc}{wait}{}
Wait for completion of a child process, and return a tuple containing
its pid and exit status indication (encoded as by \UNIX{}).
(Not on MS-DOS.)
\end{funcdesc}

\begin{funcdesc}{waitpid}{pid\, options}
Wait for completion of a child process given by proces id, and return
a tuple containing its pid and exit status indication (encoded as by
\UNIX{}).  The semantics of the call are affected by the value of
the integer options, which should be 0 for normal operation.  (If the
system does not support \code{waitpid()}, this always raises
\code{posix.error}.  Not on MS-DOS.)
\end{funcdesc}

\begin{funcdesc}{write}{fd\, str}
Write the string \var{str} to file descriptor \var{fd}.
Return the number of bytes actually written.

Note: this function is intended for low-level I/O and must be applied
to a file descriptor as returned by \code{posix.open()} or
\code{posix.pipe()}.  To write a ``file object'' returned by the
built-in function \code{open} or by \code{posix.popen} or
\code{posix.fdopen}, or \code{sys.stdout} or \code{sys.stderr}, use
its \code{write()} method.
\end{funcdesc}

\begin{datadesc}{WNOHANG}
The option for \code{waitpid()} to avoid hanging if no child process
status is available immediately.
\end{datadesc}

\section{Standard Module \sectcode{posixpath}}
\stmodindex{posixpath}

This module implements some useful functions on POSIX pathnames.

\strong{Do not import this module directly.}  Instead, import the
module \code{os} and use \code{os.path}.
\stmodindex{os}

\renewcommand{\indexsubitem}{(in module posixpath)}

\begin{funcdesc}{basename}{p}
Return the base name of pathname
\var{p}.
This is the second half of the pair returned by
\code{posixpath.split(\var{p})}.
\end{funcdesc}

\begin{funcdesc}{commonprefix}{list}
Return the longest string that is a prefix of all strings in
\var{list}.
If
\var{list}
is empty, return the empty string (\code{''}).
\end{funcdesc}

\begin{funcdesc}{exists}{p}
Return true if
\var{p}
refers to an existing path.
\end{funcdesc}

\begin{funcdesc}{expanduser}{p}
Return the argument with an initial component of \samp{\~} or
\samp{\~\var{user}} replaced by that \var{user}'s home directory.  An
initial \samp{\~{}} is replaced by the environment variable \code{\${}HOME};
an initial \samp{\~\var{user}} is looked up in the password directory through
the built-in module \code{pwd}.  If the expansion fails, or if the
path does not begin with a tilde, the path is returned unchanged.
\end{funcdesc}

\begin{funcdesc}{expandvars}{p}
Return the argument with environment variables expanded.  Substrings
of the form \samp{\$\var{name}} or \samp{\$\{\var{name}\}} are
replaced by the value of environment variable \var{name}.  Malformed
variable names and references to non-existing variables are left
unchanged.
\end{funcdesc}

\begin{funcdesc}{isabs}{p}
Return true if \var{p} is an absolute pathname (begins with a slash).
\end{funcdesc}

\begin{funcdesc}{isfile}{p}
Return true if \var{p} is an existing regular file.  This follows
symbolic links, so both \code{islink()} and \code{isfile()} can be true for the same
path.
\end{funcdesc}

\begin{funcdesc}{isdir}{p}
Return true if \var{p} is an existing directory.  This follows
symbolic links, so both \code{islink()} and \code{isdir()} can be true for the same
path.
\end{funcdesc}

\begin{funcdesc}{islink}{p}
Return true if
\var{p}
refers to a directory entry that is a symbolic link.
Always false if symbolic links are not supported.
\end{funcdesc}

\begin{funcdesc}{ismount}{p}
Return true if pathname \var{p} is a \dfn{mount point}: a point in a
file system where a different file system has been mounted.  The
function checks whether \var{p}'s parent, \file{\var{p}/..}, is on a
different device than \var{p}, or whether \file{\var{p}/..} and
\var{p} point to the same i-node on the same device --- this should
detect mount points for all \UNIX{} and POSIX variants.
\end{funcdesc}

\begin{funcdesc}{join}{p\, q}
Join the paths
\var{p}
and
\var{q} intelligently:
If
\var{q}
is an absolute path, the return value is
\var{q}.
Otherwise, the concatenation of
\var{p}
and
\var{q}
is returned, with a slash (\code{'/'}) inserted unless
\var{p}
is empty or ends in a slash.
\end{funcdesc}

\begin{funcdesc}{normcase}{p}
Normalize the case of a pathname.  This returns the path unchanged;
however, a similar function in \code{macpath} converts upper case to
lower case.
\end{funcdesc}

\begin{funcdesc}{samefile}{p\, q}
Return true if both pathname arguments refer to the same file or directory
(as indicated by device number and i-node number).
Raise an exception if a stat call on either pathname fails.
\end{funcdesc}

\begin{funcdesc}{split}{p}
Split the pathname \var{p} in a pair \code{(\var{head}, \var{tail})},
where \var{tail} is the last pathname component and \var{head} is
everything leading up to that.  The \var{tail} part will never contain
a slash; if \var{p} ends in a slash, \var{tail} will be empty.  If
there is no slash in \var{p}, \var{head} will be empty.  If \var{p} is
empty, both \var{head} and \var{tail} are empty.  Trailing slashes are
stripped from \var{head} unless it is the root (one or more slashes
only).  In nearly all cases, \code{join(\var{head}, \var{tail})}
equals \var{p} (the only exception being when there were multiple
slashes separating \var{head} from \var{tail}).
\end{funcdesc}

\begin{funcdesc}{splitext}{p}
Split the pathname \var{p} in a pair \code{(\var{root}, \var{ext})}
such that \code{\var{root} + \var{ext} == \var{p}},
and \var{ext} is empty or begins with a period and contains
at most one period.
\end{funcdesc}

\begin{funcdesc}{walk}{p\, visit\, arg}
Calls the function \var{visit} with arguments
\code{(\var{arg}, \var{dirname}, \var{names})} for each directory in the
directory tree rooted at \var{p} (including \var{p} itself, if it is a
directory).  The argument \var{dirname} specifies the visited directory,
the argument \var{names} lists the files in the directory (gotten from
\code{posix.listdir(\var{dirname})}, so including \samp{.} and
\samp{..}).  The \var{visit} function may modify \var{names} to
influence the set of directories visited below \var{dirname}, e.g., to
avoid visiting certain parts of the tree.  (The object referred to by
\var{names} must be modified in place, using \code{del} or slice
assignment.)
\end{funcdesc}
		% == posixpath
\section{Built-in Module \sectcode{pwd}}

\bimodindex{pwd}
This module provides access to the \UNIX{} password database.
It is available on all \UNIX{} versions.

Password database entries are reported as 7-tuples containing the
following items from the password database (see \file{<pwd.h>}), in order:
\code{pw_name},
\code{pw_passwd},
\code{pw_uid},
\code{pw_gid},
\code{pw_gecos},
\code{pw_dir},
\code{pw_shell}.
The uid and gid items are integers, all others are strings.
An exception is raised if the entry asked for cannot be found.

It defines the following items:

\renewcommand{\indexsubitem}{(in module pwd)}
\begin{funcdesc}{getpwuid}{uid}
Return the password database entry for the given numeric user ID.
\end{funcdesc}

\begin{funcdesc}{getpwnam}{name}
Return the password database entry for the given user name.
\end{funcdesc}

\begin{funcdesc}{getpwall}{}
Return a list of all available password database entries, in arbitrary order.
\end{funcdesc}

\section{Built-in Module \sectcode{grp}}

\bimodindex{grp}
This module provides access to the \UNIX{} group database.
It is available on all \UNIX{} versions.

Group database entries are reported as 4-tuples containing the
following items from the group database (see \file{<grp.h>}), in order:
\code{gr_name},
\code{gr_passwd},
\code{gr_gid},
\code{gr_mem}.
The gid is an integer, name and password are strings, and the member
list is a list of strings.
(Note that most users are not explicitly listed as members of the
group they are in according to the password database.)
An exception is raised if the entry asked for cannot be found.

It defines the following items:

\renewcommand{\indexsubitem}{(in module grp)}
\begin{funcdesc}{getgrgid}{gid}
Return the group database entry for the given numeric group ID.
\end{funcdesc}

\begin{funcdesc}{getgrnam}{name}
Return the group database entry for the given group name.
\end{funcdesc}

\begin{funcdesc}{getgrall}{}
Return a list of all available group entries, in arbitrary order.
\end{funcdesc}

\section{Built-in module {\tt crypt}}
\bimodindex{crypt}

This module implements an interface to the crypt({\bf 3}) routine,
which is a one-way hash function based upon a modified DES algorithm;
see the Unix man page for further details.  Possible uses include
allowing Python scripts to accept typed passwords from the user, or
attempting to crack Unix passwords with a dictionary.
\index{crypt(3)}

\begin{funcdesc}{crypt}{word\, salt} 
\var{word} will usually be a user's password.  \var{salt} is a
2-character string which will be used to select one of 4096 variations
of DES.  The characters in \var{salt} must be either \code{.},
\code{/}, or an alphanumeric character.  Returns the hashed password
as a string, which will be composed of characters from the same
alphabet as the salt.
\end{funcdesc}

The module and documentation were written by Steve Majewski.
\index{Majewski, Steve}

\section{Built-in Module \sectcode{dbm}}
\bimodindex{dbm}

The \code{dbm} module provides an interface to the {\UNIX}
\code{(n)dbm} library.  Dbm objects behave like mappings
(dictionaries), except that keys and values are always strings.
Printing a dbm object doesn't print the keys and values, and the
\code{items()} and \code{values()} methods are not supported.

See also the \code{gdbm} module, which provides a similar interface
using the GNU GDBM library.
\bimodindex{gdbm}

The module defines the following constant and functions:

\renewcommand{\indexsubitem}{(in module dbm)}
\begin{excdesc}{error}
Raised on dbm-specific errors, such as I/O errors. \code{KeyError} is
raised for general mapping errors like specifying an incorrect key.
\end{excdesc}

\begin{funcdesc}{open}{filename\, \optional{flag\, \optional{mode}}}
Open a dbm database and return a dbm object.  The \var{filename}
argument is the name of the database file (without the \file{.dir} or
\file{.pag} extensions).

The optional \var{flag} argument can be
\code{'r'} (to open an existing database for reading only --- default),
\code{'w'} (to open an existing database for reading and writing),
\code{'c'} (which creates the database if it doesn't exist), or
\code{'n'} (which always creates a new empty database).

The optional \var{mode} argument is the \UNIX{} mode of the file, used
only when the database has to be created.  It defaults to octal
\code{0666}.
\end{funcdesc}

\section{Built-in Module \sectcode{gdbm}}
\bimodindex{gdbm}

This module is nearly identical to the \code{dbm} module, but uses
GDBM instead.  Its interface is identical, and not repeated here.

Warning: the file formats created by gdbm and dbm are incompatible.
\bimodindex{dbm}

\section{Built-in Module \sectcode{termios}}
\bimodindex{termios}
\indexii{Posix}{I/O control}
\indexii{tty}{I/O control}

\renewcommand{\indexsubitem}{(in module termios)}

This module provides an interface to the Posix calls for tty I/O
control.  For a complete description of these calls, see the Posix or
\UNIX{} manual pages.  It is only available for those \UNIX{} versions
that support Posix \code{termios} style tty I/O control (and then
only if configured at installation time).

All functions in this module take a file descriptor \var{fd} as their
first argument.  This must be an integer file descriptor, such as
returned by \code{sys.stdin.fileno()}.

This module should be used in conjunction with the \code{TERMIOS}
module, which defines the relevant symbolic constants (see the next
section).

The module defines the following functions:

\begin{funcdesc}{tcgetattr}{fd}
Return a list containing the tty attributes for file descriptor
\var{fd}, as follows: \code{[\var{iflag}, \var{oflag}, \var{cflag},
\var{lflag}, \var{ispeed}, \var{ospeed}, \var{cc}]} where \var{cc} is
a list of the tty special characters (each a string of length 1,
except the items with indices \code{VMIN} and \code{VTIME}, which are
integers when these fields are defined).  The interpretation of the
flags and the speeds as well as the indexing in the \var{cc} array
must be done using the symbolic constants defined in the
\code{TERMIOS} module.
\end{funcdesc}

\begin{funcdesc}{tcsetattr}{fd\, when\, attributes}
Set the tty attributes for file descriptor \var{fd} from the
\var{attributes}, which is a list like the one returned by
\code{tcgetattr()}.  The \var{when} argument determines when the
attributes are changed: \code{TERMIOS.TCSANOW} to change immediately,
\code{TERMIOS.TCSADRAIN} to change after transmitting all queued
output, or \code{TERMIOS.TCSAFLUSH} to change after transmitting all
queued output and discarding all queued input.
\end{funcdesc}

\begin{funcdesc}{tcsendbreak}{fd\, duration}
Send a break on file descriptor \var{fd}.  A zero \var{duration} sends
a break for 0.25--0.5 seconds; a nonzero \var{duration} has a system
dependent meaning.
\end{funcdesc}

\begin{funcdesc}{tcdrain}{fd}
Wait until all output written to file descriptor \var{fd} has been
transmitted.
\end{funcdesc}

\begin{funcdesc}{tcflush}{fd\, queue}
Discard queued data on file descriptor \var{fd}.  The \var{queue}
selector specifies which queue: \code{TERMIOS.TCIFLUSH} for the input
queue, \code{TERMIOS.TCOFLUSH} for the output queue, or
\code{TERMIOS.TCIOFLUSH} for both queues.
\end{funcdesc}

\begin{funcdesc}{tcflow}{fd\, action}
Suspend or resume input or output on file descriptor \var{fd}.  The
\var{action} argument can be \code{TERMIOS.TCOOFF} to suspend output,
\code{TERMIOS.TCOON} to restart output, \code{TERMIOS.TCIOFF} to
suspend input, or \code{TERMIOS.TCION} to restart input.
\end{funcdesc}

\subsection{Example}
\nodename{termios Example}

Here's a function that prompts for a password with echoing turned off.
Note the technique using a separate \code{termios.tcgetattr()} call
and a \code{try {\ldots} finally} statement to ensure that the old tty
attributes are restored exactly no matter what happens:

\begin{verbatim}
def getpass(prompt = "Password: "):
    import termios, TERMIOS, sys
    fd = sys.stdin.fileno()
    old = termios.tcgetattr(fd)
    new = termios.tcgetattr(fd)
    new[3] = new[3] & ~TERMIOS.ECHO          # lflags
    try:
        termios.tcsetattr(fd, TERMIOS.TCSADRAIN, new)
        passwd = raw_input(prompt)
    finally:
        termios.tcsetattr(fd, TERMIOS.TCSADRAIN, old)
    return passwd
\end{verbatim}


\section{Standard Module \sectcode{TERMIOS}}
\stmodindex{TERMIOS}
\indexii{Posix}{I/O control}
\indexii{tty}{I/O control}

\renewcommand{\indexsubitem}{(in module TERMIOS)}

This module defines the symbolic constants required to use the
\code{termios} module (see the previous section).  See the Posix or
\UNIX{} manual pages (or the source) for a list of those constants.

Note: this module resides in a system-dependent subdirectory of the
Python library directory.  You may have to generate it for your
particular system using the script \file{Tools/scripts/h2py.py}.

% Manual text by Jaap Vermeulen
\section{Built-in Module \sectcode{fcntl}}
\bimodindex{fcntl}
\indexii{\UNIX{}}{file control}
\indexii{\UNIX{}}{I/O control}

This module performs file control and I/O control on file descriptors.
It is an interface to the \dfn{fcntl()} and \dfn{ioctl()} \UNIX{} routines.
File descriptors can be obtained with the \dfn{fileno()} method of a
file or socket object.

The module defines the following functions:

\renewcommand{\indexsubitem}{(in module struct)}

\begin{funcdesc}{fcntl}{fd\, op\optional{\, arg}}
  Perform the requested operation on file descriptor \code{\var{fd}}.
  The operation is defined by \code{\var{op}} and is operating system
  dependent.  Typically these codes can be retrieved from the library
  module \code{FCNTL}. The argument \code{\var{arg}} is optional, and
  defaults to the integer value \code{0}.  When
  it is present, it can either be an integer value, or a string.  With
  the argument missing or an integer value, the return value of this
  function is the integer return value of the real \code{fcntl()}
  call.  When the argument is a string it represents a binary
  structure, e.g.\ created by \code{struct.pack()}. The binary data is
  copied to a buffer whose address is passed to the real \code{fcntl()}
  call.  The return value after a successful call is the contents of
  the buffer, converted to a string object.  In case the
  \code{fcntl()} fails, an \code{IOError} will be raised.
\end{funcdesc}

\begin{funcdesc}{ioctl}{fd\, op\, arg}
  This function is identical to the \code{fcntl()} function, except
  that the operations are typically defined in the library module
  \code{IOCTL}.
\end{funcdesc}

\begin{funcdesc}{flock}{fd\, op}
Perform the lock operation \var{op} on file descriptor \var{fd}.
See the Unix manual for details.  (On some systems, this function is
emulated using \code{fcntl}.)
\end{funcdesc}

\begin{funcdesc}{lockf}{fd\, code\, \optional{len\, \optional{start\, \optional{whence}}}}
This is a wrapper around the \code{F_SETLK} and \code{F_SETLKW}
\code{fcntl()} calls.  See the Unix manual for details.
\end{funcdesc}

If the library modules \code{FCNTL} or \code{IOCTL} are missing, you
can find the opcodes in the C include files \code{sys/fcntl} and
\code{sys/ioctl}. You can create the modules yourself with the h2py
script, found in the \code{Tools/scripts} directory.
\stmodindex{FCNTL}
\stmodindex{IOCTL}

Examples (all on a SVR4 compliant system):

\bcode\begin{verbatim}
import struct, FCNTL

file = open(...)
rv = fcntl(file.fileno(), FCNTL.O_NDELAY, 1)

lockdata = struct.pack('hhllhh', FCNTL.F_WRLCK, 0, 0, 0, 0, 0)
rv = fcntl(file.fileno(), FCNTL.F_SETLKW, lockdata)
\end{verbatim}\ecode

Note that in the first example the return value variable \code{rv} will
hold an integer value; in the second example it will hold a string
value.  The structure lay-out for the \var{lockadata} variable is
system dependent -- therefore using the \code{flock()} call may be
better.

% Manual text and implementation by Jaap Vermeulen
\section{Standard Module \sectcode{posixfile}}
\bimodindex{posixfile}
\indexii{posix}{file object}

\emph{Note:} This module will become obsolete in a future release.
The locking operation that it provides is done better and more
portably by the \code{fcntl.lockf()} call.

This module implements some additional functionality over the built-in
file objects.  In particular, it implements file locking, control over
the file flags, and an easy interface to duplicate the file object.
The module defines a new file object, the posixfile object.  It
has all the standard file object methods and adds the methods
described below.  This module only works for certain flavors of
\UNIX{}, since it uses \code{fcntl()} for file locking.

To instantiate a posixfile object, use the \code{open()} function in
the posixfile module.  The resulting object looks and feels roughly
the same as a standard file object.

The posixfile module defines the following constants:

\renewcommand{\indexsubitem}{(in module posixfile)}
\begin{datadesc}{SEEK_SET}
offset is calculated from the start of the file
\end{datadesc}

\begin{datadesc}{SEEK_CUR}
offset is calculated from the current position in the file
\end{datadesc}

\begin{datadesc}{SEEK_END}
offset is calculated from the end of the file
\end{datadesc}

The posixfile module defines the following functions:

\renewcommand{\indexsubitem}{(in module posixfile)}

\begin{funcdesc}{open}{filename\optional{\, mode\optional{\, bufsize}}}
 Create a new posixfile object with the given filename and mode.  The
 \var{filename}, \var{mode} and \var{bufsize} arguments are
 interpreted the same way as by the built-in \code{open()} function.
\end{funcdesc}

\begin{funcdesc}{fileopen}{fileobject}
 Create a new posixfile object with the given standard file object.
 The resulting object has the same filename and mode as the original
 file object.
\end{funcdesc}

The posixfile object defines the following additional methods:

\renewcommand{\indexsubitem}{(posixfile method)}
\begin{funcdesc}{lock}{fmt\, \optional{len\optional{\, start
\optional{\, whence}}}}
 Lock the specified section of the file that the file object is
 referring to.  The format is explained
 below in a table.  The \var{len} argument specifies the length of the
 section that should be locked. The default is \code{0}. \var{start}
 specifies the starting offset of the section, where the default is
 \code{0}.  The \var{whence} argument specifies where the offset is
 relative to. It accepts one of the constants \code{SEEK_SET},
 \code{SEEK_CUR} or \code{SEEK_END}.  The default is \code{SEEK_SET}.
 For more information about the arguments refer to the fcntl
 manual page on your system.
\end{funcdesc}

\begin{funcdesc}{flags}{\optional{flags}}
 Set the specified flags for the file that the file object is referring
 to.  The new flags are ORed with the old flags, unless specified
 otherwise.  The format is explained below in a table.  Without
 the \var{flags} argument
 a string indicating the current flags is returned (this is
 the same as the '?' modifier).  For more information about the flags
 refer to the fcntl manual page on your system.
\end{funcdesc}

\begin{funcdesc}{dup}{}
 Duplicate the file object and the underlying file pointer and file
 descriptor.  The resulting object behaves as if it were newly
 opened.
\end{funcdesc}

\begin{funcdesc}{dup2}{fd}
 Duplicate the file object and the underlying file pointer and file
 descriptor.  The new object will have the given file descriptor.
 Otherwise the resulting object behaves as if it were newly opened.
\end{funcdesc}

\begin{funcdesc}{file}{}
 Return the standard file object that the posixfile object is based
 on.  This is sometimes necessary for functions that insist on a
 standard file object.
\end{funcdesc}

All methods return \code{IOError} when the request fails.

Format characters for the \code{lock()} method have the following meaning:

\begin{tableiii}{|c|l|c|}{samp}{Format}{Meaning}{}
  \lineiii{u}{unlock the specified region}{}
  \lineiii{r}{request a read lock for the specified section}{}
  \lineiii{w}{request a write lock for the specified section}{}
\end{tableiii}

In addition the following modifiers can be added to the format:

\begin{tableiii}{|c|l|c|}{samp}{Modifier}{Meaning}{Notes}
  \lineiii{|}{wait until the lock has been granted}{}
  \lineiii{?}{return the first lock conflicting with the requested lock, or
              \code{None} if there is no conflict.}{(1)} 
\end{tableiii}

Note:

(1) The lock returned is in the format \code{(mode, len, start,
whence, pid)} where mode is a character representing the type of lock
('r' or 'w').  This modifier prevents a request from being granted; it
is for query purposes only.

Format character for the \code{flags()} method have the following meaning:

\begin{tableiii}{|c|l|c|}{samp}{Format}{Meaning}{}
  \lineiii{a}{append only flag}{}
  \lineiii{c}{close on exec flag}{}
  \lineiii{n}{no delay flag (also called non-blocking flag)}{}
  \lineiii{s}{synchronization flag}{}
\end{tableiii}

In addition the following modifiers can be added to the format:

\begin{tableiii}{|c|l|c|}{samp}{Modifier}{Meaning}{Notes}
  \lineiii{!}{turn the specified flags 'off', instead of the default 'on'}{(1)}
  \lineiii{=}{replace the flags, instead of the default 'OR' operation}{(1)}
  \lineiii{?}{return a string in which the characters represent the flags that
  are set.}{(2)}
\end{tableiii}

Note:

(1) The \code{!} and \code{=} modifiers are mutually exclusive.

(2) This string represents the flags after they may have been altered
by the same call.

Examples:

\bcode\begin{verbatim}
from posixfile import *

file = open('/tmp/test', 'w')
file.lock('w|')
...
file.lock('u')
file.close()
\end{verbatim}\ecode

\section{Built-in Module \sectcode{syslog}}
\bimodindex{syslog}

This module provides an interface to the Unix \code{syslog} library
routines.  Refer to the \UNIX{} manual pages for a detailed description
of the \code{syslog} facility.

The module defines the following functions:

\begin{funcdesc}{syslog}{\optional{priority\,} message}
Send the string \var{message} to the system logger.
A trailing newline is added if necessary.
Each message is tagged with a priority composed of a \var{facility} and
a \var{level}.
The optional \var{priority} argument, which defaults to
\code{(LOG_USER | LOG_INFO)}, determines the message priority.
\end{funcdesc}

\begin{funcdesc}{openlog}{ident\, \optional{logopt\, \optional{facility}}}
Logging options other than the defaults can be set by explicitly opening
the log file with \code{openlog()} prior to calling \code{syslog()}.
The defaults are (usually) \var{ident} = \samp{syslog}, \var{logopt} = 0,
\var{facility} = \code{LOG_USER}.
The \var{ident} argument is a string which is prepended to every message.
The optional \var{logopt} argument is a bit field - see below for possible
values to combine.
The optional \var{facility} argument sets the default facility for messages
which do not have a facility explicitly encoded.
\end{funcdesc}

\begin{funcdesc}{closelog}{}
Close the log file.
\end{funcdesc}

\begin{funcdesc}{setlogmask}{maskpri}
This function set the priority mask to \var{maskpri} and returns the
previous mask value.
Calls to \code{syslog} with a priority level not set in \var{maskpri}
are ignored.
The default is to log all priorities.
The function \code{LOG_MASK(\var{pri})} calculates the mask for the
individual priority \var{pri}.
The function \code{LOG_UPTO(\var{pri})} calculates the mask for all priorities
up to and including \var{pri}.
\end{funcdesc}

The module defines the following constants:

\begin{description}

\item[Priority levels (high to low):]

\code{LOG_EMERG}, \code{LOG_ALERT}, \code{LOG_CRIT}, \code{LOG_ERR},
\code{LOG_WARNING}, \code{LOG_NOTICE}, \code{LOG_INFO}, \code{LOG_DEBUG}.

\item[Facilities:]

\code{LOG_KERN}, \code{LOG_USER}, \code{LOG_MAIL}, \code{LOG_DAEMON},
\code{LOG_AUTH}, \code{LOG_LPR}, \code{LOG_NEWS}, \code{LOG_UUCP},
\code{LOG_CRON} and \code{LOG_LOCAL0} to \code{LOG_LOCAL7}.

\item[Log options:]

\code{LOG_PID}, \code{LOG_CONS}, \code{LOG_NDELAY}, \code{LOG_NOWAIT}
and \code{LOG_PERROR} if defined in \file{syslog.h}.

\end{description}


\chapter{The Python Debugger}
\stmodindex{pdb}
\index{debugging}

\renewcommand{\indexsubitem}{(in module pdb)}

The module \code{pdb} defines an interactive source code debugger for
Python programs.  It supports setting breakpoints and single stepping
at the source line level, inspection of stack frames, source code
listing, and evaluation of arbitrary Python code in the context of any
stack frame.  It also supports post-mortem debugging and can be called
under program control.

The debugger is extensible --- it is actually defined as a class
\code{Pdb}.  This is currently undocumented but easily understood by
reading the source.  The extension interface uses the (also
undocumented) modules \code{bdb} and \code{cmd}.
\ttindex{Pdb}
\ttindex{bdb}
\ttindex{cmd}

A primitive windowing version of the debugger also exists --- this is
module \code{wdb}, which requires STDWIN (see the chapter on STDWIN
specific modules).
\index{stdwin}
\ttindex{wdb}

The debugger's prompt is ``\code{(Pdb) }''.
Typical usage to run a program under control of the debugger is:

\begin{verbatim}
>>> import pdb
>>> import mymodule
>>> pdb.run('mymodule.test()')
> <string>(0)?()
(Pdb) continue
> <string>(1)?()
(Pdb) continue
NameError: 'spam'
> <string>(1)?()
(Pdb) 
\end{verbatim}

Typical usage to inspect a crashed program is:

\begin{verbatim}
>>> import pdb
>>> import mymodule
>>> mymodule.test()
Traceback (innermost last):
  File "<stdin>", line 1, in ?
  File "./mymodule.py", line 4, in test
    test2()
  File "./mymodule.py", line 3, in test2
    print spam
NameError: spam
>>> pdb.pm()
> ./mymodule.py(3)test2()
-> print spam
(Pdb) 
\end{verbatim}

The module defines the following functions; each enters the debugger
in a slightly different way:

\begin{funcdesc}{run}{statement\optional{\, globals\optional{\, locals}}}
Execute the \var{statement} (given as a string) under debugger
control.  The debugger prompt appears before any code is executed; you
can set breakpoints and type \code{continue}, or you can step through
the statement using \code{step} or \code{next} (all these commands are
explained below).  The optional \var{globals} and \var{locals}
arguments specify the environment in which the code is executed; by
default the dictionary of the module \code{__main__} is used.  (See
the explanation of the \code{exec} statement or the \code{eval()}
built-in function.)
\end{funcdesc}

\begin{funcdesc}{runeval}{expression\optional{\, globals\optional{\, locals}}}
Evaluate the \var{expression} (given as a a string) under debugger
control.  When \code{runeval()} returns, it returns the value of the
expression.  Otherwise this function is similar to
\code{run()}.
\end{funcdesc}

\begin{funcdesc}{runcall}{function\optional{\, argument\, ...}}
Call the \var{function} (a function or method object, not a string)
with the given arguments.  When \code{runcall()} returns, it returns
whatever the function call returned.  The debugger prompt appears as
soon as the function is entered.
\end{funcdesc}

\begin{funcdesc}{set_trace}{}
Enter the debugger at the calling stack frame.  This is useful to
hard-code a breakpoint at a given point in a program, even if the code
is not otherwise being debugged (e.g. when an assertion fails).
\end{funcdesc}

\begin{funcdesc}{post_mortem}{traceback}
Enter post-mortem debugging of the given \var{traceback} object.
\end{funcdesc}

\begin{funcdesc}{pm}{}
Enter post-mortem debugging of the traceback found in
\code{sys.last_traceback}.
\end{funcdesc}

\section{Debugger Commands}

The debugger recognizes the following commands.  Most commands can be
abbreviated to one or two letters; e.g. ``\code{h(elp)}'' means that
either ``\code{h}'' or ``\code{help}'' can be used to enter the help
command (but not ``\code{he}'' or ``\code{hel}'', nor ``\code{H}'' or
``\code{Help} or ``\code{HELP}'').  Arguments to commands must be
separated by whitespace (spaces or tabs).  Optional arguments are
enclosed in square brackets (``\code{[]}'') in the command syntax; the
square brackets must not be typed.  Alternatives in the command syntax
are separated by a vertical bar (``\code{|}'').

Entering a blank line repeats the last command entered.  Exception: if
the last command was a ``\code{list}'' command, the next 11 lines are
listed.

Commands that the debugger doesn't recognize are assumed to be Python
statements and are executed in the context of the program being
debugged.  Python statements can also be prefixed with an exclamation
point (``\code{!}'').  This is a powerful way to inspect the program
being debugged; it is even possible to change a variable or call a
function.  When an
exception occurs in such a statement, the exception name is printed
but the debugger's state is not changed.

\begin{description}

\item[h(elp) [\var{command}]]

Without argument, print the list of available commands.
With a \var{command} as argument, print help about that command.
``\code{help pdb}'' displays the full documentation file; if the
environment variable \code{PAGER} is defined, the file is piped
through that command instead.  Since the \var{command} argument must be
an identifier, ``\code{help exec}'' must be entered to get help on the
``\code{!}'' command.

\item[w(here)]

Print a stack trace, with the most recent frame at the bottom.
An arrow indicates the current frame, which determines the
context of most commands.

\item[d(own)]

Move the current frame one level down in the stack trace
(to an older frame).

\item[u(p)]

Move the current frame one level up in the stack trace
(to a newer frame).

\item[b(reak) [\var{lineno}\code{|}\var{function}]]

With a \var{lineno} argument, set a break there in the current
file.  With a \var{function} argument, set a break at the entry of
that function.  Without argument, list all breaks.

\item[cl(ear) [\var{lineno}]]

With a \var{lineno} argument, clear that break in the current file.
Without argument, clear all breaks (but first ask confirmation).

\item[s(tep)]

Execute the current line, stop at the first possible occasion
(either in a function that is called or on the next line in the
current function).

\item[n(ext)]

Continue execution until the next line in the current function
is reached or it returns.  (The difference between \code{next} and
\code{step} is that \code{step} stops inside a called function, while
\code{next} executes called functions at (nearly) full speed, only
stopping at the next line in the current function.)

\item[r(eturn)]

Continue execution until the current function returns.

\item[c(ont(inue))]

Continue execution, only stop when a breakpoint is encountered.

\item[l(ist) [\var{first} [, \var{last}]]]

List source code for the current file.  Without arguments, list 11
lines around the current line or continue the previous listing.  With
one argument, list 11 lines around at that line.  With two arguments,
list the given range; if the second argument is less than the first,
it is interpreted as a count.

\item[a(rgs)]

Print the argument list of the current function.

\item[p \var{expression}]

Evaluate the \var{expression} in the current context and print its
value.  (Note: \code{print} can also be used, but is not a debugger
command --- this executes the Python \code{print} statement.)

\item[[!] \var{statement}]

Execute the (one-line) \var{statement} in the context of
the current stack frame.
The exclamation point can be omitted unless the first word
of the statement resembles a debugger command.
To set a global variable, you can prefix the assignment
command with a ``\code{global}'' command on the same line, e.g.:
\begin{verbatim}
(Pdb) global list_options; list_options = ['-l']
(Pdb)
\end{verbatim}

\item[q(uit)]

Quit from the debugger.
The program being executed is aborted.

\end{description}

\section{How It Works}

Some changes were made to the interpreter:

\begin{itemize}
\item sys.settrace(func) sets the global trace function
\item there can also a local trace function (see later)
\end{itemize}

Trace functions have three arguments: (\var{frame}, \var{event}, \var{arg})

\begin{description}

\item[\var{frame}] is the current stack frame

\item[\var{event}] is a string: \code{'call'}, \code{'line'}, \code{'return'}
or \code{'exception'}

\item[\var{arg}] is dependent on the event type

\end{description}

A trace function should return a new trace function or None.
Class methods are accepted (and most useful!) as trace methods.

The events have the following meaning:

\begin{description}

\item[\code{'call'}]
A function is called (or some other code block entered).  The global
trace function is called; arg is the argument list to the function;
the return value specifies the local trace function.

\item[\code{'line'}]
The interpreter is about to execute a new line of code (sometimes
multiple line events on one line exist).  The local trace function is
called; arg in None; the return value specifies the new local trace
function.

\item[\code{'return'}]
A function (or other code block) is about to return.  The local trace
function is called; arg is the value that will be returned.  The trace
function's return value is ignored.

\item[\code{'exception'}]
An exception has occurred.  The local trace function is called; arg is
a triple (exception, value, traceback); the return value specifies the
new local trace function

\end{description}

Note that as an exception is propagated down the chain of callers, an
\code{'exception'} event is generated at each level.

Stack frame objects have the following read-only attributes:

\begin{description}
\item[f_code]      the code object being executed
\item[f_lineno]    the current line number (\code{-1} for \code{'call'} events)
\item[f_back]      the stack frame of the caller, or None
\item[f_locals]    dictionary containing local name bindings
\item[f_globals]   dictionary containing global name bindings
\end{description}

Code objects have the following read-only attributes:

\begin{description}
\item[co_code]     the code string
\item[co_names]    the list of names used by the code
\item[co_consts]   the list of (literal) constants used by the code
\item[co_filename] the filename from which the code was compiled
\end{description}
			% The Python Debugger

\chapter{The Python Profiler}
\stmodindex{profile}
\stmodindex{pstats}

Copyright \copyright\ 1994, by InfoSeek Corporation, all rights reserved.

Written by James Roskind%
\footnote{
Updated and converted to \LaTeX\ by Guido van Rossum.  The references to
the old profiler are left in the text, although it no longer exists.
}

Permission to use, copy, modify, and distribute this Python software
and its associated documentation for any purpose (subject to the
restriction in the following sentence) without fee is hereby granted,
provided that the above copyright notice appears in all copies, and
that both that copyright notice and this permission notice appear in
supporting documentation, and that the name of InfoSeek not be used in
advertising or publicity pertaining to distribution of the software
without specific, written prior permission.  This permission is
explicitly restricted to the copying and modification of the software
to remain in Python, compiled Python, or other languages (such as C)
wherein the modified or derived code is exclusively imported into a
Python module.

INFOSEEK CORPORATION DISCLAIMS ALL WARRANTIES WITH REGARD TO THIS
SOFTWARE, INCLUDING ALL IMPLIED WARRANTIES OF MERCHANTABILITY AND
FITNESS. IN NO EVENT SHALL INFOSEEK CORPORATION BE LIABLE FOR ANY
SPECIAL, INDIRECT OR CONSEQUENTIAL DAMAGES OR ANY DAMAGES WHATSOEVER
RESULTING FROM LOSS OF USE, DATA OR PROFITS, WHETHER IN AN ACTION OF
CONTRACT, NEGLIGENCE OR OTHER TORTIOUS ACTION, ARISING OUT OF OR IN
CONNECTION WITH THE USE OR PERFORMANCE OF THIS SOFTWARE.


The profiler was written after only programming in Python for 3 weeks.
As a result, it is probably clumsy code, but I don't know for sure yet
'cause I'm a beginner :-).  I did work hard to make the code run fast,
so that profiling would be a reasonable thing to do.  I tried not to
repeat code fragments, but I'm sure I did some stuff in really awkward
ways at times.  Please send suggestions for improvements to:
\code{jar@netscape.com}.  I won't promise \emph{any} support.  ...but
I'd appreciate the feedback.


\section{Introduction to the profiler}
\nodename{Profiler Introduction}

A \dfn{profiler} is a program that describes the run time performance
of a program, providing a variety of statistics.  This documentation
describes the profiler functionality provided in the modules
\code{profile} and \code{pstats.}  This profiler provides
\dfn{deterministic profiling} of any Python programs.  It also
provides a series of report generation tools to allow users to rapidly
examine the results of a profile operation.


\section{How Is This Profiler Different From The Old Profiler?}
\nodename{Profiler Changes}

The big changes from old profiling module are that you get more
information, and you pay less CPU time.  It's not a trade-off, it's a
trade-up.

To be specific:

\begin{description}

\item[Bugs removed:]
Local stack frame is no longer molested, execution time is now charged
to correct functions.

\item[Accuracy increased:]
Profiler execution time is no longer charged to user's code,
calibration for platform is supported, file reads are not done \emph{by}
profiler \emph{during} profiling (and charged to user's code!).

\item[Speed increased:]
Overhead CPU cost was reduced by more than a factor of two (perhaps a
factor of five), lightweight profiler module is all that must be
loaded, and the report generating module (\code{pstats}) is not needed
during profiling.

\item[Recursive functions support:]
Cumulative times in recursive functions are correctly calculated;
recursive entries are counted.

\item[Large growth in report generating UI:]
Distinct profiles runs can be added together forming a comprehensive
report; functions that import statistics take arbitrary lists of
files; sorting criteria is now based on keywords (instead of 4 integer
options); reports shows what functions were profiled as well as what
profile file was referenced; output format has been improved.

\end{description}


\section{Instant Users Manual}

This section is provided for users that ``don't want to read the
manual.'' It provides a very brief overview, and allows a user to
rapidly perform profiling on an existing application.

To profile an application with a main entry point of \samp{foo()}, you
would add the following to your module:

\begin{verbatim}
    import profile
    profile.run("foo()")
\end{verbatim}

The above action would cause \samp{foo()} to be run, and a series of
informative lines (the profile) to be printed.  The above approach is
most useful when working with the interpreter.  If you would like to
save the results of a profile into a file for later examination, you
can supply a file name as the second argument to the \code{run()}
function:

\begin{verbatim}
    import profile
    profile.run("foo()", 'fooprof')
\end{verbatim}

When you wish to review the profile, you should use the methods in the
\code{pstats} module.  Typically you would load the statistics data as
follows:

\begin{verbatim}
    import pstats
    p = pstats.Stats('fooprof')
\end{verbatim}

The class \code{Stats} (the above code just created an instance of
this class) has a variety of methods for manipulating and printing the
data that was just read into \samp{p}.  When you ran
\code{profile.run()} above, what was printed was the result of three
method calls:

\begin{verbatim}
    p.strip_dirs().sort_stats(-1).print_stats()
\end{verbatim}

The first method removed the extraneous path from all the module
names. The second method sorted all the entries according to the
standard module/line/name string that is printed (this is to comply
with the semantics of the old profiler).  The third method printed out
all the statistics.  You might try the following sort calls:

\begin{verbatim}
    p.sort_stats('name')
    p.print_stats()
\end{verbatim}

The first call will actually sort the list by function name, and the
second call will print out the statistics.  The following are some
interesting calls to experiment with:

\begin{verbatim}
    p.sort_stats('cumulative').print_stats(10)
\end{verbatim}

This sorts the profile by cumulative time in a function, and then only
prints the ten most significant lines.  If you want to understand what
algorithms are taking time, the above line is what you would use.

If you were looking to see what functions were looping a lot, and
taking a lot of time, you would do:

\begin{verbatim}
    p.sort_stats('time').print_stats(10)
\end{verbatim}

to sort according to time spent within each function, and then print
the statistics for the top ten functions.

You might also try:

\begin{verbatim}
    p.sort_stats('file').print_stats('__init__')
\end{verbatim}

This will sort all the statistics by file name, and then print out
statistics for only the class init methods ('cause they are spelled
with \code{__init__} in them).  As one final example, you could try:

\begin{verbatim}
    p.sort_stats('time', 'cum').print_stats(.5, 'init')
\end{verbatim}

This line sorts statistics with a primary key of time, and a secondary
key of cumulative time, and then prints out some of the statistics.
To be specific, the list is first culled down to 50\% (re: \samp{.5})
of its original size, then only lines containing \code{init} are
maintained, and that sub-sub-list is printed.

If you wondered what functions called the above functions, you could
now (\samp{p} is still sorted according to the last criteria) do:

\begin{verbatim}
    p.print_callers(.5, 'init')
\end{verbatim}

and you would get a list of callers for each of the listed functions. 

If you want more functionality, you're going to have to read the
manual, or guess what the following functions do:

\begin{verbatim}
    p.print_callees()
    p.add('fooprof')
\end{verbatim}


\section{What Is Deterministic Profiling?}
\nodename{Deterministic Profiling}

\dfn{Deterministic profiling} is meant to reflect the fact that all
\dfn{function call}, \dfn{function return}, and \dfn{exception} events
are monitored, and precise timings are made for the intervals between
these events (during which time the user's code is executing).  In
contrast, \dfn{statistical profiling} (which is not done by this
module) randomly samples the effective instruction pointer, and
deduces where time is being spent.  The latter technique traditionally
involves less overhead (as the code does not need to be instrumented),
but provides only relative indications of where time is being spent.

In Python, since there is an interpreter active during execution, the
presence of instrumented code is not required to do deterministic
profiling.  Python automatically provides a \dfn{hook} (optional
callback) for each event.  In addition, the interpreted nature of
Python tends to add so much overhead to execution, that deterministic
profiling tends to only add small processing overhead in typical
applications.  The result is that deterministic profiling is not that
expensive, yet provides extensive run time statistics about the
execution of a Python program.

Call count statistics can be used to identify bugs in code (surprising
counts), and to identify possible inline-expansion points (high call
counts).  Internal time statistics can be used to identify ``hot
loops'' that should be carefully optimized.  Cumulative time
statistics should be used to identify high level errors in the
selection of algorithms.  Note that the unusual handling of cumulative
times in this profiler allows statistics for recursive implementations
of algorithms to be directly compared to iterative implementations.


\section{Reference Manual}

\renewcommand{\indexsubitem}{(profiler function)}

The primary entry point for the profiler is the global function
\code{profile.run()}.  It is typically used to create any profile
information.  The reports are formatted and printed using methods of
the class \code{pstats.Stats}.  The following is a description of all
of these standard entry points and functions.  For a more in-depth
view of some of the code, consider reading the later section on
Profiler Extensions, which includes discussion of how to derive
``better'' profilers from the classes presented, or reading the source
code for these modules.

\begin{funcdesc}{profile.run}{string\optional{\, filename\optional{\, ...}}}

This function takes a single argument that has can be passed to the
\code{exec} statement, and an optional file name.  In all cases this
routine attempts to \code{exec} its first argument, and gather profiling
statistics from the execution. If no file name is present, then this
function automatically prints a simple profiling report, sorted by the
standard name string (file/line/function-name) that is presented in
each line.  The following is a typical output from such a call:

\small{
\begin{verbatim}
      main()
      2706 function calls (2004 primitive calls) in 4.504 CPU seconds

Ordered by: standard name

ncalls  tottime  percall  cumtime  percall filename:lineno(function)
     2    0.006    0.003    0.953    0.477 pobject.py:75(save_objects)
  43/3    0.533    0.012    0.749    0.250 pobject.py:99(evaluate)
 ...
\end{verbatim}
}

The first line indicates that this profile was generated by the call:\\
\code{profile.run('main()')}, and hence the exec'ed string is
\code{'main()'}.  The second line indicates that 2706 calls were
monitored.  Of those calls, 2004 were \dfn{primitive}.  We define
\dfn{primitive} to mean that the call was not induced via recursion.
The next line: \code{Ordered by:\ standard name}, indicates that
the text string in the far right column was used to sort the output.
The column headings include:

\begin{description}

\item[ncalls ]
for the number of calls, 

\item[tottime ]
for the total time spent in the given function (and excluding time
made in calls to sub-functions),

\item[percall ]
is the quotient of \code{tottime} divided by \code{ncalls}

\item[cumtime ]
is the total time spent in this and all subfunctions (i.e., from
invocation till exit). This figure is accurate \emph{even} for recursive
functions.

\item[percall ]
is the quotient of \code{cumtime} divided by primitive calls

\item[filename:lineno(function) ]
provides the respective data of each function

\end{description}

When there are two numbers in the first column (e.g.: \samp{43/3}),
then the latter is the number of primitive calls, and the former is
the actual number of calls.  Note that when the function does not
recurse, these two values are the same, and only the single figure is
printed.

\end{funcdesc}

\begin{funcdesc}{pstats.Stats}{filename\optional{\, ...}}
This class constructor creates an instance of a ``statistics object''
from a \var{filename} (or set of filenames).  \code{Stats} objects are
manipulated by methods, in order to print useful reports.

The file selected by the above constructor must have been created by
the corresponding version of \code{profile}.  To be specific, there is
\emph{NO} file compatibility guaranteed with future versions of this
profiler, and there is no compatibility with files produced by other
profilers (e.g., the old system profiler).

If several files are provided, all the statistics for identical
functions will be coalesced, so that an overall view of several
processes can be considered in a single report.  If additional files
need to be combined with data in an existing \code{Stats} object, the
\code{add()} method can be used.
\end{funcdesc}


\subsection{The \sectcode{Stats} Class}

\renewcommand{\indexsubitem}{(Stats method)}

\begin{funcdesc}{strip_dirs}{}
This method for the \code{Stats} class removes all leading path information
from file names.  It is very useful in reducing the size of the
printout to fit within (close to) 80 columns.  This method modifies
the object, and the stripped information is lost.  After performing a
strip operation, the object is considered to have its entries in a
``random'' order, as it was just after object initialization and
loading.  If \code{strip_dirs()} causes two function names to be
indistinguishable (i.e., they are on the same line of the same
filename, and have the same function name), then the statistics for
these two entries are accumulated into a single entry.
\end{funcdesc}


\begin{funcdesc}{add}{filename\optional{\, ...}}
This method of the \code{Stats} class accumulates additional profiling
information into the current profiling object.  Its arguments should
refer to filenames created by the corresponding version of
\code{profile.run()}.  Statistics for identically named (re: file,
line, name) functions are automatically accumulated into single
function statistics.
\end{funcdesc}

\begin{funcdesc}{sort_stats}{key\optional{\, ...}}
This method modifies the \code{Stats} object by sorting it according to the
supplied criteria.  The argument is typically a string identifying the
basis of a sort (example: \code{"time"} or \code{"name"}).

When more than one key is provided, then additional keys are used as
secondary criteria when the there is equality in all keys selected
before them.  For example, sort_stats('name', 'file') will sort all
the entries according to their function name, and resolve all ties
(identical function names) by sorting by file name.

Abbreviations can be used for any key names, as long as the
abbreviation is unambiguous.  The following are the keys currently
defined: 

\begin{tableii}{|l|l|}{code}{Valid Arg}{Meaning}
\lineii{"calls"}{call count}
\lineii{"cumulative"}{cumulative time}
\lineii{"file"}{file name}
\lineii{"module"}{file name}
\lineii{"pcalls"}{primitive call count}
\lineii{"line"}{line number}
\lineii{"name"}{function name}
\lineii{"nfl"}{name/file/line}
\lineii{"stdname"}{standard name}
\lineii{"time"}{internal time}
\end{tableii}

Note that all sorts on statistics are in descending order (placing
most time consuming items first), where as name, file, and line number
searches are in ascending order (i.e., alphabetical). The subtle
distinction between \code{"nfl"} and \code{"stdname"} is that the
standard name is a sort of the name as printed, which means that the
embedded line numbers get compared in an odd way.  For example, lines
3, 20, and 40 would (if the file names were the same) appear in the
string order 20, 3 and 40.  In contrast, \code{"nfl"} does a numeric
compare of the line numbers.  In fact, \code{sort_stats("nfl")} is the
same as \code{sort_stats("name", "file", "line")}.

For compatibility with the old profiler, the numeric arguments
\samp{-1}, \samp{0}, \samp{1}, and \samp{2} are permitted.  They are
interpreted as \code{"stdname"}, \code{"calls"}, \code{"time"}, and
\code{"cumulative"} respectively.  If this old style format (numeric)
is used, only one sort key (the numeric key) will be used, and
additional arguments will be silently ignored.
\end{funcdesc}


\begin{funcdesc}{reverse_order}{}
This method for the \code{Stats} class reverses the ordering of the basic
list within the object.  This method is provided primarily for
compatibility with the old profiler.  Its utility is questionable
now that ascending vs descending order is properly selected based on
the sort key of choice.
\end{funcdesc}

\begin{funcdesc}{print_stats}{restriction\optional{\, ...}}
This method for the \code{Stats} class prints out a report as described
in the \code{profile.run()} definition.

The order of the printing is based on the last \code{sort_stats()}
operation done on the object (subject to caveats in \code{add()} and
\code{strip_dirs())}.

The arguments provided (if any) can be used to limit the list down to
the significant entries.  Initially, the list is taken to be the
complete set of profiled functions.  Each restriction is either an
integer (to select a count of lines), or a decimal fraction between
0.0 and 1.0 inclusive (to select a percentage of lines), or a regular
expression (to pattern match the standard name that is printed).  If
several restrictions are provided, then they are applied sequentially.
For example:

\begin{verbatim}
    print_stats(.1, "foo:")
\end{verbatim}

would first limit the printing to first 10\% of list, and then only
print functions that were part of filename \samp{.*foo:}.  In
contrast, the command:

\begin{verbatim}
    print_stats("foo:", .1)
\end{verbatim}

would limit the list to all functions having file names \samp{.*foo:},
and then proceed to only print the first 10\% of them.
\end{funcdesc}


\begin{funcdesc}{print_callers}{restrictions\optional{\, ...}}
This method for the \code{Stats} class prints a list of all functions
that called each function in the profiled database.  The ordering is
identical to that provided by \code{print_stats()}, and the definition
of the restricting argument is also identical.  For convenience, a
number is shown in parentheses after each caller to show how many
times this specific call was made.  A second non-parenthesized number
is the cumulative time spent in the function at the right.
\end{funcdesc}

\begin{funcdesc}{print_callees}{restrictions\optional{\, ...}}
This method for the \code{Stats} class prints a list of all function
that were called by the indicated function.  Aside from this reversal
of direction of calls (re: called vs was called by), the arguments and
ordering are identical to the \code{print_callers()} method.
\end{funcdesc}

\begin{funcdesc}{ignore}{}
This method of the \code{Stats} class is used to dispose of the value
returned by earlier methods.  All standard methods in this class
return the instance that is being processed, so that the commands can
be strung together.  For example:

\begin{verbatim}
pstats.Stats('foofile').strip_dirs().sort_stats('cum') \
                       .print_stats().ignore()
\end{verbatim}

would perform all the indicated functions, but it would not return
the final reference to the \code{Stats} instance.%
\footnote{
This was once necessary, when Python would print any unused expression
result that was not \code{None}.  The method is still defined for
backward compatibility.
}
\end{funcdesc}


\section{Limitations}

There are two fundamental limitations on this profiler.  The first is
that it relies on the Python interpreter to dispatch \dfn{call},
\dfn{return}, and \dfn{exception} events.  Compiled C code does not
get interpreted, and hence is ``invisible'' to the profiler.  All time
spent in C code (including builtin functions) will be charged to the
Python function that invoked the C code.  If the C code calls out
to some native Python code, then those calls will be profiled
properly.

The second limitation has to do with accuracy of timing information.
There is a fundamental problem with deterministic profilers involving
accuracy.  The most obvious restriction is that the underlying ``clock''
is only ticking at a rate (typically) of about .001 seconds.  Hence no
measurements will be more accurate that that underlying clock.  If
enough measurements are taken, then the ``error'' will tend to average
out. Unfortunately, removing this first error induces a second source
of error...

The second problem is that it ``takes a while'' from when an event is
dispatched until the profiler's call to get the time actually
\emph{gets} the state of the clock.  Similarly, there is a certain lag
when exiting the profiler event handler from the time that the clock's
value was obtained (and then squirreled away), until the user's code
is once again executing.  As a result, functions that are called many
times, or call many functions, will typically accumulate this error.
The error that accumulates in this fashion is typically less than the
accuracy of the clock (i.e., less than one clock tick), but it
\emph{can} accumulate and become very significant.  This profiler
provides a means of calibrating itself for a given platform so that
this error can be probabilistically (i.e., on the average) removed.
After the profiler is calibrated, it will be more accurate (in a least
square sense), but it will sometimes produce negative numbers (when
call counts are exceptionally low, and the gods of probability work
against you :-). )  Do \emph{NOT} be alarmed by negative numbers in
the profile.  They should \emph{only} appear if you have calibrated
your profiler, and the results are actually better than without
calibration.


\section{Calibration}

The profiler class has a hard coded constant that is added to each
event handling time to compensate for the overhead of calling the time
function, and socking away the results.  The following procedure can
be used to obtain this constant for a given platform (see discussion
in section Limitations above).

\begin{verbatim}
    import profile
    pr = profile.Profile()
    pr.calibrate(100)
    pr.calibrate(100)
    pr.calibrate(100)
\end{verbatim}

The argument to calibrate() is the number of times to try to do the
sample calls to get the CPU times.  If your computer is \emph{very}
fast, you might have to do:

\begin{verbatim}
    pr.calibrate(1000)
\end{verbatim}

or even:

\begin{verbatim}
    pr.calibrate(10000)
\end{verbatim}

The object of this exercise is to get a fairly consistent result.
When you have a consistent answer, you are ready to use that number in
the source code.  For a Sun Sparcstation 1000 running Solaris 2.3, the
magical number is about .00053.  If you have a choice, you are better
off with a smaller constant, and your results will ``less often'' show
up as negative in profile statistics.

The following shows how the trace_dispatch() method in the Profile
class should be modified to install the calibration constant on a Sun
Sparcstation 1000:

\begin{verbatim}
    def trace_dispatch(self, frame, event, arg):
        t = self.timer()
        t = t[0] + t[1] - self.t - .00053 # Calibration constant

        if self.dispatch[event](frame,t):
            t = self.timer()
            self.t = t[0] + t[1]
        else:
            r = self.timer()
            self.t = r[0] + r[1] - t # put back unrecorded delta
        return
\end{verbatim}

Note that if there is no calibration constant, then the line
containing the callibration constant should simply say:

\begin{verbatim}
        t = t[0] + t[1] - self.t  # no calibration constant
\end{verbatim}

You can also achieve the same results using a derived class (and the
profiler will actually run equally fast!!), but the above method is
the simplest to use.  I could have made the profiler ``self
calibrating'', but it would have made the initialization of the
profiler class slower, and would have required some \emph{very} fancy
coding, or else the use of a variable where the constant \samp{.00053}
was placed in the code shown.  This is a \strong{VERY} critical
performance section, and there is no reason to use a variable lookup
at this point, when a constant can be used.


\section{Extensions --- Deriving Better Profilers}
\nodename{Profiler Extensions}

The \code{Profile} class of module \code{profile} was written so that
derived classes could be developed to extend the profiler.  Rather
than describing all the details of such an effort, I'll just present
the following two examples of derived classes that can be used to do
profiling.  If the reader is an avid Python programmer, then it should
be possible to use these as a model and create similar (and perchance
better) profile classes.

If all you want to do is change how the timer is called, or which
timer function is used, then the basic class has an option for that in
the constructor for the class.  Consider passing the name of a
function to call into the constructor:

\begin{verbatim}
    pr = profile.Profile(your_time_func)
\end{verbatim}

The resulting profiler will call \code{your_time_func()} instead of
\code{os.times()}.  The function should return either a single number
or a list of numbers (like what \code{os.times()} returns).  If the
function returns a single time number, or the list of returned numbers
has length 2, then you will get an especially fast version of the
dispatch routine.

Be warned that you \emph{should} calibrate the profiler class for the
timer function that you choose.  For most machines, a timer that
returns a lone integer value will provide the best results in terms of
low overhead during profiling.  (os.times is \emph{pretty} bad, 'cause
it returns a tuple of floating point values, so all arithmetic is
floating point in the profiler!).  If you want to substitute a
better timer in the cleanest fashion, you should derive a class, and
simply put in the replacement dispatch method that better handles your
timer call, along with the appropriate calibration constant :-).


\subsection{OldProfile Class}

The following derived profiler simulates the old style profiler,
providing errant results on recursive functions. The reason for the
usefulness of this profiler is that it runs faster (i.e., less
overhead) than the old profiler.  It still creates all the caller
stats, and is quite useful when there is \emph{no} recursion in the
user's code.  It is also a lot more accurate than the old profiler, as
it does not charge all its overhead time to the user's code.

\begin{verbatim}
class OldProfile(Profile):

    def trace_dispatch_exception(self, frame, t):
        rt, rtt, rct, rfn, rframe, rcur = self.cur
        if rcur and not rframe is frame:
            return self.trace_dispatch_return(rframe, t)
        return 0

    def trace_dispatch_call(self, frame, t):
        fn = `frame.f_code`
        
        self.cur = (t, 0, 0, fn, frame, self.cur)
        if self.timings.has_key(fn):
            tt, ct, callers = self.timings[fn]
            self.timings[fn] = tt, ct, callers
        else:
            self.timings[fn] = 0, 0, {}
        return 1

    def trace_dispatch_return(self, frame, t):
        rt, rtt, rct, rfn, frame, rcur = self.cur
        rtt = rtt + t
        sft = rtt + rct

        pt, ptt, pct, pfn, pframe, pcur = rcur
        self.cur = pt, ptt+rt, pct+sft, pfn, pframe, pcur

        tt, ct, callers = self.timings[rfn]
        if callers.has_key(pfn):
            callers[pfn] = callers[pfn] + 1
        else:
            callers[pfn] = 1
        self.timings[rfn] = tt+rtt, ct + sft, callers

        return 1


    def snapshot_stats(self):
        self.stats = {}
        for func in self.timings.keys():
            tt, ct, callers = self.timings[func]
            nor_func = self.func_normalize(func)
            nor_callers = {}
            nc = 0
            for func_caller in callers.keys():
                nor_callers[self.func_normalize(func_caller)]=\
                      callers[func_caller]
                nc = nc + callers[func_caller]
            self.stats[nor_func] = nc, nc, tt, ct, nor_callers
\end{verbatim}
        

\subsection{HotProfile Class}

This profiler is the fastest derived profile example.  It does not
calculate caller-callee relationships, and does not calculate
cumulative time under a function.  It only calculates time spent in a
function, so it runs very quickly (re: very low overhead).  In truth,
the basic profiler is so fast, that is probably not worth the savings
to give up the data, but this class still provides a nice example.

\begin{verbatim}
class HotProfile(Profile):

    def trace_dispatch_exception(self, frame, t):
        rt, rtt, rfn, rframe, rcur = self.cur
        if rcur and not rframe is frame:
            return self.trace_dispatch_return(rframe, t)
        return 0

    def trace_dispatch_call(self, frame, t):
        self.cur = (t, 0, frame, self.cur)
        return 1

    def trace_dispatch_return(self, frame, t):
        rt, rtt, frame, rcur = self.cur

        rfn = `frame.f_code`

        pt, ptt, pframe, pcur = rcur
        self.cur = pt, ptt+rt, pframe, pcur

        if self.timings.has_key(rfn):
            nc, tt = self.timings[rfn]
            self.timings[rfn] = nc + 1, rt + rtt + tt
        else:
            self.timings[rfn] =      1, rt + rtt

        return 1


    def snapshot_stats(self):
        self.stats = {}
        for func in self.timings.keys():
            nc, tt = self.timings[func]
            nor_func = self.func_normalize(func)
            self.stats[nor_func] = nc, nc, tt, 0, {}
\end{verbatim}
		% The Python Profiler

\chapter{Internet and WWW Services}
\nodename{Internet and WWW}
\index{WWW}
\index{Internet}
\index{World-Wide Web}

The modules described in this chapter provide various services to
World-Wide Web (WWW) clients and/or services, and a few modules
related to news and email.  They are all implemented in Python.  Some
of these modules require the presence of the system-dependent module
\code{sockets}, which is currently only fully supported on Unix and
Windows NT.  Here is an overview:

\begin{description}

\item[cgi]
--- Common Gateway Interface, used to interpret forms in server-side
scripts.

\item[urllib]
--- Open an arbitrary object given by URL (requires sockets).

\item[httplib]
--- HTTP protocol client (requires sockets).

\item[ftplib]
--- FTP protocol client (requires sockets).

\item[gopherlib]
--- Gopher protocol client (requires sockets).

\item[nntplib]
--- NNTP protocol client (requires sockets).

\item[urlparse]
--- Parse a URL string into a tuple (addressing scheme identifier, network
location, path, parameters, query string, fragment identifier).

\item[sgmllib]
--- Only as much of an SGML parser as needed to parse HTML.

\item[htmllib]
--- A (slow) parser for HTML documents.

\item[formatter]
--- Generic output formatter and device interface.

\item[rfc822]
--- Parse RFC-822 style mail headers.

\item[mimetools]
--- Tools for parsing MIME style message bodies.

\end{description}
			% Internet and WWW Services
\section{Standard Module \sectcode{cgi}}
\stmodindex{cgi}
\indexii{WWW}{server}
\indexii{CGI}{protocol}
\indexii{HTTP}{protocol}
\indexii{MIME}{headers}
\index{URL}

\renewcommand{\indexsubitem}{(in module cgi)}

Support module for CGI (Common Gateway Interface) scripts.

This module defines a number of utilities for use by CGI scripts
written in Python.

\subsection{Introduction}
\nodename{Introduction to the CGI module}

A CGI script is invoked by an HTTP server, usually to process user
input submitted through an HTML \code{<FORM>} or \code{<ISINPUT>} element.

Most often, CGI scripts live in the server's special \code{cgi-bin}
directory.  The HTTP server places all sorts of information about the
request (such as the client's hostname, the requested URL, the query
string, and lots of other goodies) in the script's shell environment,
executes the script, and sends the script's output back to the client.

The script's input is connected to the client too, and sometimes the
form data is read this way; at other times the form data is passed via
the ``query string'' part of the URL.  This module (\code{cgi.py}) is intended
to take care of the different cases and provide a simpler interface to
the Python script.  It also provides a number of utilities that help
in debugging scripts, and the latest addition is support for file
uploads from a form (if your browser supports it -- Grail 0.3 and
Netscape 2.0 do).

The output of a CGI script should consist of two sections, separated
by a blank line.  The first section contains a number of headers,
telling the client what kind of data is following.  Python code to
generate a minimal header section looks like this:

\begin{verbatim}
	print "Content-type: text/html"	# HTML is following
	print				# blank line, end of headers
\end{verbatim}

The second section is usually HTML, which allows the client software
to display nicely formatted text with header, in-line images, etc.
Here's Python code that prints a simple piece of HTML:

\begin{verbatim}
	print "<TITLE>CGI script output</TITLE>"
	print "<H1>This is my first CGI script</H1>"
	print "Hello, world!"
\end{verbatim}

(It may not be fully legal HTML according to the letter of the
standard, but any browser will understand it.)

\subsection{Using the cgi module}
\nodename{Using the cgi module}

Begin by writing \code{import cgi}.  Don't use \code{from cgi import *} -- the
module defines all sorts of names for its own use or for backward 
compatibility that you don't want in your namespace.

It's best to use the \code{FieldStorage} class.  The other classes define in this 
module are provided mostly for backward compatibility.  Instantiate it 
exactly once, without arguments.  This reads the form contents from 
standard input or the environment (depending on the value of various 
environment variables set according to the CGI standard).  Since it may 
consume standard input, it should be instantiated only once.

The \code{FieldStorage} instance can be accessed as if it were a Python 
dictionary.  For instance, the following code (which assumes that the 
\code{Content-type} header and blank line have already been printed) checks that 
the fields \code{name} and \code{addr} are both set to a non-empty string:

\begin{verbatim}
	form = cgi.FieldStorage()
	form_ok = 0
	if form.has_key("name") and form.has_key("addr"):
		if form["name"].value != "" and form["addr"].value != "":
			form_ok = 1
	if not form_ok:
		print "<H1>Error</H1>"
		print "Please fill in the name and addr fields."
		return
	...further form processing here...
\end{verbatim}

Here the fields, accessed through \code{form[key]}, are themselves instances
of \code{FieldStorage} (or \code{MiniFieldStorage}, depending on the form encoding).

If the submitted form data contains more than one field with the same
name, the object retrieved by \code{form[key]} is not a \code{(Mini)FieldStorage}
instance but a list of such instances.  If you expect this possibility
(i.e., when your HTML form comtains multiple fields with the same
name), use the \code{type()} function to determine whether you have a single
instance or a list of instances.  For example, here's code that
concatenates any number of username fields, separated by commas:

\begin{verbatim}
	username = form["username"]
	if type(username) is type([]):
		# Multiple username fields specified
		usernames = ""
		for item in username:
			if usernames:
				# Next item -- insert comma
				usernames = usernames + "," + item.value
			else:
				# First item -- don't insert comma
				usernames = item.value
	else:
		# Single username field specified
		usernames = username.value
\end{verbatim}

If a field represents an uploaded file, the value attribute reads the 
entire file in memory as a string.  This may not be what you want.  You can 
test for an uploaded file by testing either the filename attribute or the 
file attribute.  You can then read the data at leasure from the file 
attribute:

\begin{verbatim}
	fileitem = form["userfile"]
	if fileitem.file:
		# It's an uploaded file; count lines
		linecount = 0
		while 1:
			line = fileitem.file.readline()
			if not line: break
			linecount = linecount + 1
\end{verbatim}

The file upload draft standard entertains the possibility of uploading
multiple files from one field (using a recursive \code{multipart/*}
encoding).  When this occurs, the item will be a dictionary-like
FieldStorage item.  This can be determined by testing its type
attribute, which should have the value \code{multipart/form-data} (or
perhaps another string beginning with \code{multipart/}  It this case, it
can be iterated over recursively just like the top-level form object.

When a form is submitted in the ``old'' format (as the query string or as a 
single data part of type \code{application/x-www-form-urlencoded}), the items 
will actually be instances of the class \code{MiniFieldStorage}.  In this case,
the list, file and filename attributes are always \code{None}.


\subsection{Old classes}

These classes, present in earlier versions of the \code{cgi} module, are still 
supported for backward compatibility.  New applications should use the
FieldStorage class.

\code{SvFormContentDict}: single value form content as dictionary; assumes each 
field name occurs in the form only once.

\code{FormContentDict}: multiple value form content as dictionary (the form
items are lists of values).  Useful if your form contains multiple
fields with the same name.

Other classes (\code{FormContent}, \code{InterpFormContentDict}) are present for
backwards compatibility with really old applications only.  If you still 
use these and would be inconvenienced when they disappeared from a next 
version of this module, drop me a note.


\subsection{Functions}

These are useful if you want more control, or if you want to employ
some of the algorithms implemented in this module in other
circumstances.

\begin{funcdesc}{parse}{fp}: Parse a query in the environment or from a file (default \code{sys.stdin}).
\end{funcdesc}

\begin{funcdesc}{parse_qs}{qs}: parse a query string given as a string argument (data of type 
\code{application/x-www-form-urlencoded}).
\end{funcdesc}

\begin{funcdesc}{parse_multipart}{fp\, pdict}: parse input of type \code{multipart/form-data} (for 
file uploads).  Arguments are \code{fp} for the input file and 
    \code{pdict} for the dictionary containing other parameters of \code{content-type} header

    Returns a dictionary just like \code{parse_qs()}: keys are the field names, each 
    value is a list of values for that field.  This is easy to use but not 
    much good if you are expecting megabytes to be uploaded -- in that case, 
    use the \code{FieldStorage} class instead which is much more flexible.  Note 
    that \code{content-type} is the raw, unparsed contents of the \code{content-type} 
    header.

    Note that this does not parse nested multipart parts -- use \code{FieldStorage} for 
    that.
\end{funcdesc}

\begin{funcdesc}{parse_header}{string}: parse a header like \code{Content-type} into a main
content-type and a dictionary of parameters.
\end{funcdesc}

\begin{funcdesc}{test}{}: robust test CGI script, usable as main program.
    Writes minimal HTTP headers and formats all information provided to
    the script in HTML form.
\end{funcdesc}

\begin{funcdesc}{print_environ}{}: format the shell environment in HTML.
\end{funcdesc}

\begin{funcdesc}{print_form}{form}: format a form in HTML.
\end{funcdesc}

\begin{funcdesc}{print_directory}{}: format the current directory in HTML.
\end{funcdesc}

\begin{funcdesc}{print_environ_usage}{}: print a list of useful (used by CGI) environment variables in
HTML.
\end{funcdesc}

\begin{funcdesc}{escape}{}: convert the characters ``\code{\&}'', ``\code{<}'' and ``\code{>}'' to HTML-safe
sequences.  Use this if you need to display text that might contain
such characters in HTML.  To translate URLs for inclusion in the HREF
attribute of an \code{<A>} tag, use \code{urllib.quote()}.
\end{funcdesc}


\subsection{Caring about security}

There's one important rule: if you invoke an external program (e.g.
via the \code{os.system()} or \code{os.popen()} functions), make very sure you don't
pass arbitrary strings received from the client to the shell.  This is
a well-known security hole whereby clever hackers anywhere on the web
can exploit a gullible CGI script to invoke arbitrary shell commands.
Even parts of the URL or field names cannot be trusted, since the
request doesn't have to come from your form!

To be on the safe side, if you must pass a string gotten from a form
to a shell command, you should make sure the string contains only
alphanumeric characters, dashes, underscores, and periods.


\subsection{Installing your CGI script on a Unix system}

Read the documentation for your HTTP server and check with your local
system administrator to find the directory where CGI scripts should be
installed; usually this is in a directory \code{cgi-bin} in the server tree.

Make sure that your script is readable and executable by ``others''; the
Unix file mode should be 755 (use \code{chmod 755 filename}).  Make sure
that the first line of the script contains \code{\#!} starting in column 1
followed by the pathname of the Python interpreter, for instance:

\begin{verbatim}
	#!/usr/local/bin/python
\end{verbatim}

Make sure the Python interpreter exists and is executable by ``others''.

Make sure that any files your script needs to read or write are
readable or writable, respectively, by ``others'' -- their mode should
be 644 for readable and 666 for writable.  This is because, for
security reasons, the HTTP server executes your script as user
``nobody'', without any special privileges.  It can only read (write,
execute) files that everybody can read (write, execute).  The current
directory at execution time is also different (it is usually the
server's cgi-bin directory) and the set of environment variables is
also different from what you get at login.  in particular, don't count
on the shell's search path for executables (\code{\$PATH}) or the Python
module search path (\code{\$PYTHONPATH}) to be set to anything interesting.

If you need to load modules from a directory which is not on Python's
default module search path, you can change the path in your script,
before importing other modules, e.g.:

\begin{verbatim}
	import sys
	sys.path.insert(0, "/usr/home/joe/lib/python")
	sys.path.insert(0, "/usr/local/lib/python")
\end{verbatim}

(This way, the directory inserted last will be searched first!)

Instructions for non-Unix systems will vary; check your HTTP server's
documentation (it will usually have a section on CGI scripts).


\subsection{Testing your CGI script}

Unfortunately, a CGI script will generally not run when you try it
from the command line, and a script that works perfectly from the
command line may fail mysteriously when run from the server.  There's
one reason why you should still test your script from the command
line: if it contains a syntax error, the python interpreter won't
execute it at all, and the HTTP server will most likely send a cryptic
error to the client.

Assuming your script has no syntax errors, yet it does not work, you
have no choice but to read the next section:


\subsection{Debugging CGI scripts}

First of all, check for trivial installation errors -- reading the
section above on installing your CGI script carefully can save you a
lot of time.  If you wonder whether you have understood the
installation procedure correctly, try installing a copy of this module
file (\code{cgi.py}) as a CGI script.  When invoked as a script, the file
will dump its environment and the contents of the form in HTML form.
Give it the right mode etc, and send it a request.  If it's installed
in the standard \code{cgi-bin} directory, it should be possible to send it a
request by entering a URL into your browser of the form:

\begin{verbatim}
	http://yourhostname/cgi-bin/cgi.py?name=Joe+Blow&addr=At+Home
\end{verbatim}

If this gives an error of type 404, the server cannot find the script
-- perhaps you need to install it in a different directory.  If it
gives another error (e.g.  500), there's an installation problem that
you should fix before trying to go any further.  If you get a nicely
formatted listing of the environment and form content (in this
example, the fields should be listed as ``addr'' with value ``At Home''
and ``name'' with value ``Joe Blow''), the \code{cgi.py} script has been
installed correctly.  If you follow the same procedure for your own
script, you should now be able to debug it.

The next step could be to call the \code{cgi} module's test() function from
your script: replace its main code with the single statement

\begin{verbatim}
	cgi.test()
\end{verbatim}
	
This should produce the same results as those gotten from installing
the \code{cgi.py} file itself.

When an ordinary Python script raises an unhandled exception
(e.g. because of a typo in a module name, a file that can't be opened,
etc.), the Python interpreter prints a nice traceback and exits.
While the Python interpreter will still do this when your CGI script
raises an exception, most likely the traceback will end up in one of
the HTTP server's log file, or be discarded altogether.

Fortunately, once you have managed to get your script to execute
*some* code, it is easy to catch exceptions and cause a traceback to
be printed.  The \code{test()} function below in this module is an example.
Here are the rules:

\begin{enumerate}
	\item Import the traceback module (before entering the
	   try-except!)
	
	\item Make sure you finish printing the headers and the blank
	   line early
	
	\item Assign \code{sys.stderr} to \code{sys.stdout}
	
	\item Wrap all remaining code in a try-except statement
	
	\item In the except clause, call \code{traceback.print_exc()}
\end{enumerate}

For example:

\begin{verbatim}
	import sys
	import traceback
	print "Content-type: text/html"
	print
	sys.stderr = sys.stdout
	try:
		...your code here...
	except:
		print "\n\n<PRE>"
		traceback.print_exc()
\end{verbatim}

Notes: The assignment to \code{sys.stderr} is needed because the traceback
prints to \code{sys.stderr}.  The \code{print "$\backslash$n$\backslash$n<PRE>"} statement is necessary to
disable the word wrapping in HTML.

If you suspect that there may be a problem in importing the traceback
module, you can use an even more robust approach (which only uses
built-in modules):

\begin{verbatim}
	import sys
	sys.stderr = sys.stdout
	print "Content-type: text/plain"
	print
	...your code here...
\end{verbatim}

This relies on the Python interpreter to print the traceback.  The
content type of the output is set to plain text, which disables all
HTML processing.  If your script works, the raw HTML will be displayed
by your client.  If it raises an exception, most likely after the
first two lines have been printed, a traceback will be displayed.
Because no HTML interpretation is going on, the traceback will
readable.


\subsection{Common problems and solutions}

\begin{itemize}
\item Most HTTP servers buffer the output from CGI scripts until the
script is completed.  This means that it is not possible to display a
progress report on the client's display while the script is running.

\item Check the installation instructions above.

\item Check the HTTP server's log files.  (\code{tail -f logfile} in a separate
window may be useful!)

\item Always check a script for syntax errors first, by doing something
like \code{python script.py}.

\item When using any of the debugging techniques, don't forget to add
\code{import sys} to the top of the script.

\item When invoking external programs, make sure they can be found.
Usually, this means using absolute path names -- \code{\$PATH} is usually not
set to a very useful value in a CGI script.

\item When reading or writing external files, make sure they can be read
or written by every user on the system.

\item Don't try to give a CGI script a set-uid mode.  This doesn't work on
most systems, and is a security liability as well.
\end{itemize}


\section{Standard Module \sectcode{urllib}}
\stmodindex{urllib}
\index{WWW}
\index{World-Wide Web}
\index{URL}

\renewcommand{\indexsubitem}{(in module urllib)}

This module provides a high-level interface for fetching data across
the World-Wide Web.  In particular, the \code{urlopen} function is
similar to the built-in function \code{open}, but accepts URLs
(Universal Resource Locators) instead of filenames.  Some restrictions
apply --- it can only open URLs for reading, and no seek operations
are available.

it defines the following public functions:

\begin{funcdesc}{urlopen}{url}
Open a network object denoted by a URL for reading.  If the URL does
not have a scheme identifier, or if it has \samp{file:} as its scheme
identifier, this opens a local file; otherwise it opens a socket to a
server somewhere on the network.  If the connection cannot be made, or
if the server returns an error code, the \code{IOError} exception is
raised.  If all went well, a file-like object is returned.  This
supports the following methods: \code{read()}, \code{readline()},
\code{readlines()}, \code{fileno()}, \code{close()} and \code{info()}.
Except for the last one, these methods have the same interface as for
file objects --- see the section on File Objects earlier in this
manual.  (It's not a built-in file object, however, so it can't be
used at those few places where a true built-in file object is
required.)

The \code{info()} method returns an instance of the class
\code{rfc822.Message} containing the headers received from the server,
if the protocol uses such headers (currently the only supported
protocol that uses this is HTTP).  See the description of the
\code{rfc822} module.
\end{funcdesc}

\begin{funcdesc}{urlretrieve}{url}
Copy a network object denoted by a URL to a local file, if necessary.
If the URL points to a local file, or a valid cached copy of the
object exists, the object is not copied.  Return a tuple (\var{filename},
\var{headers}) where \var{filename} is the local file name under which
the object can be found, and \var{headers} is either \code{None} (for
a local object) or whatever the \code{info()} method of the object
returned by \code{urlopen()} returned (for a remote object, possibly
cached).  Exceptions are the same as for \code{urlopen()}.
\end{funcdesc}

\begin{funcdesc}{urlcleanup}{}
Clear the cache that may have been built up by previous calls to
\code{urlretrieve()}.
\end{funcdesc}

\begin{funcdesc}{quote}{string\optional{\, addsafe}}
Replace special characters in \var{string} using the \code{\%xx} escape.
Letters, digits, and the characters ``\code{_,.-}'' are never quoted.
The optional \var{addsafe} parameter specifies additional characters
that should not be quoted --- its default value is \code{'/'}.

Example: \code{quote('/\~conolly/')} yields \code{'/\%7econnolly/'}.
\end{funcdesc}

\begin{funcdesc}{unquote}{string}
Replace \samp{\%xx} escapes by their single-character equivalent.

Example: \code{unquote('/\%7Econnolly/')} yields \code{'/\~connolly/'}.
\end{funcdesc}

Restrictions:

\begin{itemize}

\item
Currently, only the following protocols are supported: HTTP, (versions
0.9 and 1.0), Gopher (but not Gopher-+), FTP, and local files.
\index{HTTP}
\index{Gopher}
\index{FTP}

\item
The caching feature of \code{urlretrieve()} has been disabled until I
find the time to hack proper processing of Expiration time headers.

\item
There should be a function to query whether a particular URL is in
the cache.

\item
For backward compatibility, if a URL appears to point to a local file
but the file can't be opened, the URL is re-interpreted using the FTP
protocol.  This can sometimes cause confusing error messages.

\item
The \code{urlopen()} and \code{urlretrieve()} functions can cause
arbitrarily long delays while waiting for a network connection to be
set up.  This means that it is difficult to build an interactive
web client using these functions without using threads.

\item
The data returned by \code{urlopen()} or \code{urlretrieve()} is the
raw data returned by the server.  This may be binary data (e.g. an
image), plain text or (for example) HTML.  The HTTP protocol provides
type information in the reply header, which can be inspected by
looking at the \code{Content-type} header.  For the Gopher protocol,
type information is encoded in the URL; there is currently no easy way
to extract it.  If the returned data is HTML, you can use the module
\code{htmllib} to parse it.
\index{HTML}
\index{HTTP}
\index{Gopher}
\stmodindex{htmllib}

\item
Although the \code{urllib} module contains (undocumented) routines to
parse and unparse URL strings, the recommended interface for URL
manipulation is in module \code{urlparse}.
\stmodindex{urlparse}

\end{itemize}

\section{Standard Module \sectcode{httplib}}
\stmodindex{httplib}
\index{HTTP}

\renewcommand{\indexsubitem}{(in module httplib)}

This module defines a class which implements the client side of the
HTTP protocol.  It is normally not used directly --- the module
\code{urllib} uses it to handle URLs that use HTTP.
\stmodindex{urllib}

The module defines one class, \code{HTTP}.  An \code{HTTP} instance
represents one transaction with an HTTP server.  It should be
instantiated passing it a host and optional port number.  If no port
number is passed, the port is extracted from the host string if it has
the form \code{host:port}, else the default HTTP port (80) is used.
If no host is passed, no connection is made, and the \code{connect}
method should be used to connect to a server.  For example, the
following calls all create instances that connect to the server at the
same host and port:

\begin{verbatim}
>>> h1 = httplib.HTTP('www.cwi.nl')
>>> h2 = httplib.HTTP('www.cwi.nl:80')
>>> h3 = httplib.HTTP('www.cwi.nl', 80)
\end{verbatim}

Once an \code{HTTP} instance has been connected to an HTTP server, it
should be used as follows:

\begin{enumerate}

\item[1.] Make exactly one call to the \code{putrequest()} method.

\item[2.] Make zero or more calls to the \code{putheader()} method.

\item[3.] Call the \code{endheaders()} method (this can be omitted if
step 4 makes no calls).

\item[4.] Optional calls to the \code{send()} method.

\item[5.] Call the \code{getreply()} method.

\item[6.] Call the \code{getfile()} method and read the data off the
file object that it returns.

\end{enumerate}

\subsection{HTTP Objects}

\code{HTTP} instances have the following methods:

\renewcommand{\indexsubitem}{(HTTP method)}

\begin{funcdesc}{set_debuglevel}{level}
Set the debugging level (the amount of debugging output printed).
The default debug level is \code{0}, meaning no debugging output is
printed.
\end{funcdesc}

\begin{funcdesc}{connect}{host\optional{\, port}}
Connect to the server given by \var{host} and \var{port}.  See the
intro for the default port.  This should be called directly only if
the instance was instantiated without passing a host.
\end{funcdesc}

\begin{funcdesc}{send}{data}
Send data to the server.  This should be used directly only after the
\code{endheaders()} method has been called and before
\code{getreply()} has been called.
\end{funcdesc}

\begin{funcdesc}{putrequest}{request\, selector}
This should be the first call after the connection to the server has
been made.  It sends a line to the server consisting of the
\var{request} string, the \var{selector} string, and the HTTP version
(\code{HTTP/1.0}).
\end{funcdesc}

\begin{funcdesc}{putheader}{header\, argument\optional{\, ...}}
Send an RFC-822 style header to the server.  It sends a line to the
server consisting of the header, a colon and a space, and the first
argument.  If more arguments are given, continuation lines are sent,
each consisting of a tab and an argument.
\end{funcdesc}

\begin{funcdesc}{endheaders}{}
Send a blank line to the server, signalling the end of the headers.
\end{funcdesc}

\begin{funcdesc}{getreply}{}
Complete the request by shutting down the sending end of the socket,
read the reply from the server, and return a triple (\var{replycode},
\var{message}, \var{headers}).  Here \var{replycode} is the integer
reply code from the request (e.g.\ \code{200} if the request was
handled properly); \var{message} is the message string corresponding
to the reply code; and \var{header} is an instance of the class
\code{rfc822.Message} containing the headers received from the server.
See the description of the \code{rfc822} module.
\stmodindex{rfc822}
\end{funcdesc}

\begin{funcdesc}{getfile}{}
Return a file object from which the data returned by the server can be
read, using the \code{read()}, \code{readline()} or \code{readlines()}
methods.
\end{funcdesc}

\subsection{Example}
\nodename{HTTP Example}

Here is an example session:

\begin{verbatim}
>>> import httplib
>>> h = httplib.HTTP('www.cwi.nl')
>>> h.putrequest('GET', '/index.html')
>>> h.putheader('Accept', 'text/html')
>>> h.putheader('Accept', 'text/plain')
>>> h.endheaders()
>>> errcode, errmsg, headers = h.getreply()
>>> print errcode # Should be 200
>>> f = h.getfile()
>>> data f.read() # Get the raw HTML
>>> f.close()
>>> 
\end{verbatim}

\section{Standard Module \sectcode{ftplib}}
\stmodindex{ftplib}

\renewcommand{\indexsubitem}{(in module ftplib)}

This module defines the class \code{FTP} and a few related items.  The
\code{FTP} class implements the client side of the FTP protocol.  You
can use this to write Python programs that perform a variety of
automated FTP jobs, such as mirroring other ftp servers.  It is also
used by the module \code{urllib} to handle URLs that use FTP.  For
more information on FTP (File Transfer Protocol), see Internet RFC
959.

Here's a sample session using the \code{ftplib} module:

\begin{verbatim}
>>> from ftplib import FTP
>>> ftp = FTP('ftp.cwi.nl')   # connect to host, default port
>>> ftp.login()               # user anonymous, passwd user@hostname
>>> ftp.retrlines('LIST')     # list directory contents
total 24418
drwxrwsr-x   5 ftp-usr  pdmaint     1536 Mar 20 09:48 .
dr-xr-srwt 105 ftp-usr  pdmaint     1536 Mar 21 14:32 ..
-rw-r--r--   1 ftp-usr  pdmaint     5305 Mar 20 09:48 INDEX
 .
 .
 .
>>> ftp.quit()
\end{verbatim}

The module defines the following items:

\begin{funcdesc}{FTP}{\optional{host\optional{\, user\, passwd\, acct}}}
Return a new instance of the \code{FTP} class.  When
\var{host} is given, the method call \code{connect(\var{host})} is
made.  When \var{user} is given, additionally the method call
\code{login(\var{user}, \var{passwd}, \var{acct})} is made (where
\var{passwd} and \var{acct} default to the empty string when not given).
\end{funcdesc}

\begin{datadesc}{all_errors}
The set of all exceptions (as a tuple) that methods of \code{FTP}
instances may raise as a result of problems with the FTP connection
(as opposed to programming errors made by the caller).  This set
includes the four exceptions listed below as well as
\code{socket.error} and \code{IOError}.
\end{datadesc}

\begin{excdesc}{error_reply}
Exception raised when an unexpected reply is received from the server.
\end{excdesc}

\begin{excdesc}{error_temp}
Exception raised when an error code in the range 400--499 is received.
\end{excdesc}

\begin{excdesc}{error_perm}
Exception raised when an error code in the range 500--599 is received.
\end{excdesc}

\begin{excdesc}{error_proto}
Exception raised when a reply is received from the server that does
not begin with a digit in the range 1--5.
\end{excdesc}

\subsection{FTP Objects}

FTP instances have the following methods:

\renewcommand{\indexsubitem}{(FTP object method)}

\begin{funcdesc}{set_debuglevel}{level}
Set the instance's debugging level.  This controls the amount of
debugging output printed.  The default, 0, produces no debugging
output.  A value of 1 produces a moderate amount of debugging output,
generally a single line per request.  A value of 2 or higher produces
the maximum amount of debugging output, logging each line sent and
received on the control connection.
\end{funcdesc}

\begin{funcdesc}{connect}{host\optional{\, port}}
Connect to the given host and port.  The default port number is 21, as
specified by the FTP protocol specification.  It is rarely needed to
specify a different port number.  This function should be called only
once for each instance; it should not be called at all if a host was
given when the instance was created.  All other methods can only be
used after a connection has been made.
\end{funcdesc}

\begin{funcdesc}{getwelcome}{}
Return the welcome message sent by the server in reply to the initial
connection.  (This message sometimes contains disclaimers or help
information that may be relevant to the user.)
\end{funcdesc}

\begin{funcdesc}{login}{\optional{user\optional{\, passwd\optional{\, acct}}}}
Log in as the given \var{user}.  The \var{passwd} and \var{acct}
parameters are optional and default to the empty string.  If no
\var{user} is specified, it defaults to \samp{anonymous}.  If
\var{user} is \code{anonymous}, the default \var{passwd} is
\samp{\var{realuser}@\var{host}} where \var{realuser} is the real user
name (glanced from the \samp{LOGNAME} or \samp{USER} environment
variable) and \var{host} is the hostname as returned by
\code{socket.gethostname()}.  This function should be called only
once for each instance, after a connection has been established; it
should not be called at all if a host and user were given when the
instance was created.  Most FTP commands are only allowed after the
client has logged in.
\end{funcdesc}

\begin{funcdesc}{abort}{}
Abort a file transfer that is in progress.  Using this does not always
work, but it's worth a try.
\end{funcdesc}

\begin{funcdesc}{sendcmd}{command}
Send a simple command string to the server and return the response
string.
\end{funcdesc}

\begin{funcdesc}{voidcmd}{command}
Send a simple command string to the server and handle the response.
Return nothing if a response code in the range 200--299 is received.
Raise an exception otherwise.
\end{funcdesc}

\begin{funcdesc}{retrbinary}{command\, callback\, maxblocksize}
Retrieve a file in binary transfer mode.  \var{command} should be an
appropriate \samp{RETR} command, i.e.\ \code{"RETR \var{filename}"}.
The \var{callback} function is called for each block of data received,
with a single string argument giving the data block.
The \var{maxblocksize} argument specifies the maximum block size
(which may not be the actual size of the data blocks passed to
\var{callback}).
\end{funcdesc}

\begin{funcdesc}{retrlines}{command\optional{\, callback}}
Retrieve a file or directory listing in \ASCII{} transfer mode.
var{command} should be an appropriate \samp{RETR} command (see
\code{retrbinary()} or a \samp{LIST} command (usually just the string
\code{"LIST"}).  The \var{callback} function is called for each line,
with the trailing CRLF stripped.  The default \var{callback} prints
the line to \code{sys.stdout}.
\end{funcdesc}

\begin{funcdesc}{storbinary}{command\, file\, blocksize}
Store a file in binary transfer mode.  \var{command} should be an
appropriate \samp{STOR} command, i.e.\ \code{"STOR \var{filename}"}.
\var{file} is an open file object which is read until EOF using its
\code{read()} method in blocks of size \var{blocksize} to provide the
data to be stored.
\end{funcdesc}

\begin{funcdesc}{storlines}{command\, file}
Store a file in \ASCII{} transfer mode.  \var{command} should be an
appropriate \samp{STOR} command (see \code{storbinary()}).  Lines are
read until EOF from the open file object \var{file} using its
\code{readline()} method to privide the data to be stored.
\end{funcdesc}

\begin{funcdesc}{nlst}{argument\optional{\, \ldots}}
Return a list of files as returned by the \samp{NLST} command.  The
optional var{argument} is a directory to list (default is the current
server directory).  Multiple arguments can be used to pass
non-standard options to the \samp{NLST} command.
\end{funcdesc}

\begin{funcdesc}{dir}{argument\optional{\, \ldots}}
Return a directory listing as returned by the \samp{LIST} command, as
a list of lines.  The optional var{argument} is a directory to list
(default is the current server directory).  Multiple arguments can be
used to pass non-standard options to the \samp{LIST} command.  If the
last argument is a function, it is used as a \var{callback} function
as for \code{retrlines()}.
\end{funcdesc}

\begin{funcdesc}{rename}{fromname\, toname}
Rename file \var{fromname} on the server to \var{toname}.
\end{funcdesc}

\begin{funcdesc}{cwd}{pathname}
Set the current directory on the server.
\end{funcdesc}

\begin{funcdesc}{mkd}{pathname}
Create a new directory on the server.
\end{funcdesc}

\begin{funcdesc}{pwd}{}
Return the pathname of the current directory on the server.
\end{funcdesc}

\begin{funcdesc}{quit}{}
Send a \samp{QUIT} command to the server and close the connection.
This is the ``polite'' way to close a connection, but it may raise an
exception of the server reponds with an error to the \code{QUIT}
command.
\end{funcdesc}

\begin{funcdesc}{close}{}
Close the connection unilaterally.  This should not be applied to an
already closed connection (e.g.\ after a successful call to
\code{quit()}.
\end{funcdesc}

\section{Standard Module \sectcode{gopherlib}}
\stmodindex{gopherlib}

\renewcommand{\indexsubitem}{(in module gopherlib)}

This module provides a minimal implementation of client side of the
the Gopher protocol.  It is used by the module \code{urllib} to handle
URLs that use the Gopher protocol.

The module defines the following functions:

\begin{funcdesc}{send_selector}{selector\, host\optional{\, port}}
Send a \var{selector} string to the gopher server at \var{host} and
\var{port} (default 70).  Return an open file object from which the
returned document can be read.
\end{funcdesc}

\begin{funcdesc}{send_query}{selector\, query\, host\optional{\, port}}
Send a \var{selector} string and a \var{query} string to a gopher
server at \var{host} and \var{port} (default 70).  Return an open file
object from which the returned document can be read.
\end{funcdesc}

Note that the data returned by the Gopher server can be of any type,
depending on the first character of the selector string.  If the data
is text (first character of the selector is \samp{0}), lines are
terminated by CRLF, and the data is terminated by a line consisting of
a single \samp{.}, and a leading \samp{.} should be stripped from
lines that begin with \samp{..}.  Directory listings (first charactger
of the selector is \samp{1}) are transferred using the same protocol.

\section{Standard Module \sectcode{nntplib}}
\stmodindex{nntplib}

\renewcommand{\indexsubitem}{(in module nntplib)}

This module defines the class \code{NNTP} which implements the client
side of the NNTP protocol.  It can be used to implement a news reader
or poster, or automated news processors.  For more information on NNTP
(Network News Transfer Protocol), see Internet RFC 977.

Here are two small examples of how it can be used.  To list some
statistics about a newsgroup and print the subjects of the last 10
articles:

\small{
\begin{verbatim}
>>> s = NNTP('news.cwi.nl')
>>> resp, count, first, last, name = s.group('comp.lang.python')
>>> print 'Group', name, 'has', count, 'articles, range', first, 'to', last
Group comp.lang.python has 59 articles, range 3742 to 3803
>>> resp, subs = s.xhdr('subject', first + '-' + last)
>>> for id, sub in subs[-10:]: print id, sub
... 
3792 Re: Removing elements from a list while iterating...
3793 Re: Who likes Info files?
3794 Emacs and doc strings
3795 a few questions about the Mac implementation
3796 Re: executable python scripts
3797 Re: executable python scripts
3798 Re: a few questions about the Mac implementation 
3799 Re: PROPOSAL: A Generic Python Object Interface for Python C Modules
3802 Re: executable python scripts 
3803 Re: POSIX wait and SIGCHLD
>>> s.quit()
'205 news.cwi.nl closing connection.  Goodbye.'
>>> 
\end{verbatim}
}

To post an article from a file (this assumes that the article has
valid headers):

\begin{verbatim}
>>> s = NNTP('news.cwi.nl')
>>> f = open('/tmp/article')
>>> s.post(f)
'240 Article posted successfully.'
>>> s.quit()
'205 news.cwi.nl closing connection.  Goodbye.'
>>> 
\end{verbatim}

The module itself defines the following items:

\begin{funcdesc}{NNTP}{host\optional{\, port}}
Return a new instance of the \code{NNTP} class, representing a
connection to the NNTP server running on host \var{host}, listening at
port \var{port}.  The default \var{port} is 119.
\end{funcdesc}

\begin{excdesc}{error_reply}
Exception raised when an unexpected reply is received from the server.
\end{excdesc}

\begin{excdesc}{error_temp}
Exception raised when an error code in the range 400--499 is received.
\end{excdesc}

\begin{excdesc}{error_perm}
Exception raised when an error code in the range 500--599 is received.
\end{excdesc}

\begin{excdesc}{error_proto}
Exception raised when a reply is received from the server that does
not begin with a digit in the range 1--5.
\end{excdesc}

\subsection{NNTP Objects}

NNTP instances have the following methods.  The \var{response} that is
returned as the first item in the return tuple of almost all methods
is the server's response: a string beginning with a three-digit code.
If the server's response indicates an error, the method raises one of
the above exceptions.

\renewcommand{\indexsubitem}{(NNTP object method)}

\begin{funcdesc}{getwelcome}{}
Return the welcome message sent by the server in reply to the initial
connection.  (This message sometimes contains disclaimers or help
information that may be relevant to the user.)
\end{funcdesc}

\begin{funcdesc}{set_debuglevel}{level}
Set the instance's debugging level.  This controls the amount of
debugging output printed.  The default, 0, produces no debugging
output.  A value of 1 produces a moderate amount of debugging output,
generally a single line per request or response.  A value of 2 or
higher produces the maximum amount of debugging output, logging each
line sent and received on the connection (including message text).
\end{funcdesc}

\begin{funcdesc}{newgroups}{date\, time}
Send a \samp{NEWGROUPS} command.  The \var{date} argument should be a
string of the form \code{"\var{yy}\var{mm}\var{dd}"} indicating the
date, and \var{time} should be a string of the form
\code{"\var{hh}\var{mm}\var{ss}"} indicating the time.  Return a pair
\code{(\var{response}, \var{groups})} where \var{groups} is a list of
group names that are new since the given date and time.
\end{funcdesc}

\begin{funcdesc}{newnews}{group\, date\, time}
Send a \samp{NEWNEWS} command.  Here, \var{group} is a group name or
\code{"*"}, and \var{date} and \var{time} have the same meaning as for
\code{newgroups()}.  Return a pair \code{(\var{response},
\var{articles})} where \var{articles} is a list of article ids.
\end{funcdesc}

\begin{funcdesc}{list}{}
Send a \samp{LIST} command.  Return a pair \code{(\var{response},
\var{list})} where \var{list} is a list of tuples.  Each tuple has the
form \code{(\var{group}, \var{last}, \var{first}, \var{flag})}, where
\var{group} is a group name, \var{last} and \var{first} are the last
and first article numbers (as strings), and \var{flag} is \code{'y'}
if posting is allowed, \code{'n'} if not, and \code{'m'} if the
newsgroup is moderated.  (Note the ordering: \var{last}, \var{first}.)
\end{funcdesc}

\begin{funcdesc}{group}{name}
Send a \samp{GROUP} command, where \var{name} is the group name.
Return a tuple \code{(\var{response}, \var{count}, \var{first},
\var{last}, \var{name})} where \var{count} is the (estimated) number
of articles in the group, \var{first} is the first article number in
the group, \var{last} is the last article number in the group, and
\var{name} is the group name.  The numbers are returned as strings.
\end{funcdesc}

\begin{funcdesc}{help}{}
Send a \samp{HELP} command.  Return a pair \code{(\var{response},
\var{list})} where \var{list} is a list of help strings.
\end{funcdesc}

\begin{funcdesc}{stat}{id}
Send a \samp{STAT} command, where \var{id} is the message id (enclosed
in \samp{<} and \samp{>}) or an article number (as a string).
Return a triple \code{(var{response}, \var{number}, \var{id})} where
\var{number} is the article number (as a string) and \var{id} is the
article id  (enclosed in \samp{<} and \samp{>}).
\end{funcdesc}

\begin{funcdesc}{next}{}
Send a \samp{NEXT} command.  Return as for \code{stat()}.
\end{funcdesc}

\begin{funcdesc}{last}{}
Send a \samp{LAST} command.  Return as for \code{stat()}.
\end{funcdesc}

\begin{funcdesc}{head}{id}
Send a \samp{HEAD} command, where \var{id} has the same meaning as for
\code{stat()}.  Return a pair \code{(\var{response}, \var{list})}
where \var{list} is a list of the article's headers (an uninterpreted
list of lines, without trailing newlines).
\end{funcdesc}

\begin{funcdesc}{body}{id}
Send a \samp{BODY} command, where \var{id} has the same meaning as for
\code{stat()}.  Return a pair \code{(\var{response}, \var{list})}
where \var{list} is a list of the article's body text (an
uninterpreted list of lines, without trailing newlines).
\end{funcdesc}

\begin{funcdesc}{article}{id}
Send a \samp{ARTICLE} command, where \var{id} has the same meaning as
for \code{stat()}.  Return a pair \code{(\var{response}, \var{list})}
where \var{list} is a list of the article's header and body text (an
uninterpreted list of lines, without trailing newlines).
\end{funcdesc}

\begin{funcdesc}{slave}{}
Send a \samp{SLAVE} command.  Return the server's \var{response}.
\end{funcdesc}

\begin{funcdesc}{xhdr}{header\, string}
Send an \samp{XHDR} command.  This command is not defined in the RFC
but is a common extension.  The \var{header} argument is a header
keyword, e.g. \code{"subject"}.  The \var{string} argument should have
the form \code{"\var{first}-\var{last}"} where \var{first} and
\var{last} are the first and last article numbers to search.  Return a
pair \code{(\var{response}, \var{list})}, where \var{list} is a list of
pairs \code{(\var{id}, \var{text})}, where \var{id} is an article id
(as a string) and \var{text} is the text of the requested header for
that article.
\end{funcdesc}

\begin{funcdesc}{post}{file}
Post an article using the \samp{POST} command.  The \var{file}
argument is an open file object which is read until EOF using its
\code{readline()} method.  It should be a well-formed news article,
including the required headers.  The \code{post()} method
automatically escapes lines beginning with \samp{.}.
\end{funcdesc}

\begin{funcdesc}{ihave}{id\, file}
Send an \samp{IHAVE} command.  If the response is not an error, treat
\var{file} exactly as for the \code{post()} method.
\end{funcdesc}

\begin{funcdesc}{quit}{}
Send a \samp{QUIT} command and close the connection.  Once this method
has been called, no other methods of the NNTP object should be called.
\end{funcdesc}

\section{Standard Module \sectcode{urlparse}}
\stmodindex{urlparse}
\index{WWW}
\index{World-Wide Web}
\index{URL}
\indexii{URL}{parsing}
\indexii{relative}{URL}

\renewcommand{\indexsubitem}{(in module urlparse)}

This module defines a standard interface to break URL strings up in
components (addessing scheme, network location, path etc.), to combine
the components back into a URL string, and to convert a ``relative
URL'' to an absolute URL given a ``base URL''.

The module has been designed to match the current Internet draft on
Relative Uniform Resource Locators (and discovered a bug in an earlier
draft!).

It defines the following functions:

\begin{funcdesc}{urlparse}{urlstring\optional{\,
default_scheme\optional{\, allow_fragments}}}
Parse a URL into 6 components, returning a 6-tuple: (addressing
scheme, network location, path, parameters, query, fragment
identifier).  This corresponds to the general structure of a URL:
\code{\var{scheme}://\var{netloc}/\var{path};\var{parameters}?\var{query}\#\var{fragment}}.
Each tuple item is a string, possibly empty.
The components are not broken up in smaller parts (e.g. the network
location is a single string), and \% escapes are not expanded.
The delimiters as shown above are not part of the tuple items,
except for a leading slash in the \var{path} component, which is
retained if present.

Example:

\begin{verbatim}
urlparse('http://www.cwi.nl:80/%7Eguido/Python.html')
\end{verbatim}

yields the tuple

\begin{verbatim}
('http', 'www.cwi.nl:80', '/%7Eguido/Python.html', '', '', '')
\end{verbatim}

If the \var{default_scheme} argument is specified, it gives the
default addressing scheme, to be used only if the URL string does not
specify one.  The default value for this argument is the empty string.

If the \var{allow_fragments} argument is zero, fragment identifiers
are not allowed, even if the URL's addressing scheme normally does
support them.  The default value for this argument is \code{1}.
\end{funcdesc}

\begin{funcdesc}{urlunparse}{tuple}
Construct a URL string from a tuple as returned by \code{urlparse}.
This may result in a slightly different, but equivalent URL, if the
URL that was parsed originally had redundant delimiters, e.g. a ? with
an empty query (the draft states that these are equivalent).
\end{funcdesc}

\begin{funcdesc}{urljoin}{base\, url\optional{\, allow_fragments}}
Construct a full (``absolute'') URL by combining a ``base URL''
(\var{base}) with a ``relative URL'' (\var{url}).  Informally, this
uses components of the base URL, in particular the addressing scheme,
the network location and (part of) the path, to provide missing
components in the relative URL.

Example:

\begin{verbatim}
urljoin('http://www.cwi.nl/%7Eguido/Python.html', 'FAQ.html')
\end{verbatim}

yields the string

\begin{verbatim}
'http://www.cwi.nl/%7Eguido/FAQ.html'
\end{verbatim}

The \var{allow_fragments} argument has the same meaning as for
\code{urlparse}.
\end{funcdesc}

\section{Standard Module \sectcode{sgmllib}}
\stmodindex{sgmllib}
\index{SGML}

This module defines a class \code{SGMLParser} which serves as the
basis for parsing text files formatted in SGML (Standard Generalized
Mark-up Language).  In fact, it does not provide a full SGML parser
--- it only parses SGML insofar as it is used by HTML, and the module
only exists as a base for the \code{htmllib} module.
\stmodindex{htmllib}

In particular, the parser is hardcoded to recognize the following
constructs:

\begin{itemize}

\item
Opening and closing tags of the form
``\code{<\var{tag} \var{attr}="\var{value}" ...>}'' and
``\code{</\var{tag}>}'', respectively.

\item
Numeric character references of the form ``\code{\&\#\var{name};}''.

\item
Entity references of the form ``\code{\&\var{name};}''.

\item
SGML comments of the form ``\code{<!--\var{text}-->}''.  Note that
spaces, tabs, and newlines are allowed between the trailing
``\code{>}'' and the immediately preceeding ``\code{--}''.

\end{itemize}

The \code{SGMLParser} class must be instantiated without arguments.
It has the following interface methods:

\renewcommand{\indexsubitem}{({\tt SGMLParser} method)}

\begin{funcdesc}{reset}{}
Reset the instance.  Loses all unprocessed data.  This is called
implicitly at instantiation time.
\end{funcdesc}

\begin{funcdesc}{setnomoretags}{}
Stop processing tags.  Treat all following input as literal input
(CDATA).  (This is only provided so the HTML tag \code{<PLAINTEXT>}
can be implemented.)
\end{funcdesc}

\begin{funcdesc}{setliteral}{}
Enter literal mode (CDATA mode).
\end{funcdesc}

\begin{funcdesc}{feed}{data}
Feed some text to the parser.  It is processed insofar as it consists
of complete elements; incomplete data is buffered until more data is
fed or \code{close()} is called.
\end{funcdesc}

\begin{funcdesc}{close}{}
Force processing of all buffered data as if it were followed by an
end-of-file mark.  This method may be redefined by a derived class to
define additional processing at the end of the input, but the
redefined version should always call \code{SGMLParser.close()}.
\end{funcdesc}

\begin{funcdesc}{handle_starttag}{tag\, method\, attributes}
This method is called to handle start tags for which either a
\code{start_\var{tag}()} or \code{do_\var{tag}()} method has been
defined.  The \code{tag} argument is the name of the tag converted to
lower case, and the \code{method} argument is the bound method which
should be used to support semantic interpretation of the start tag.
The \var{attributes} argument is a list of (\var{name}, \var{value})
pairs containing the attributes found inside the tag's \code{<>}
brackets.  The \var{name} has been translated to lower case and double
quotes and backslashes in the \var{value} have been interpreted.  For
instance, for the tag \code{<A HREF="http://www.cwi.nl/">}, this
method would be called as \code{unknown_starttag('a', [('href',
'http://www.cwi.nl/')])}.  The base implementation simply calls
\code{method} with \code{attributes} as the only argument.
\end{funcdesc}

\begin{funcdesc}{handle_endtag}{tag\, method}

This method is called to handle endtags for which an
\code{end_\var{tag}()} method has been defined.  The \code{tag}
argument is the name of the tag converted to lower case, and the
\code{method} argument is the bound method which should be used to
support semantic interpretation of the end tag.  If no
\code{end_\var{tag}()} method is defined for the closing element, this
handler is not called.  The base implementation simply calls
\code{method}.
\end{funcdesc}

\begin{funcdesc}{handle_data}{data}
This method is called to process arbitrary data.  It is intended to be
overridden by a derived class; the base class implementation does
nothing.
\end{funcdesc}

\begin{funcdesc}{handle_charref}{ref}
This method is called to process a character reference of the form
``\code{\&\#\var{ref};}''.  In the base implementation, \var{ref} must
be a decimal number in the
range 0-255.  It translates the character to \ASCII{} and calls the
method \code{handle_data()} with the character as argument.  If
\var{ref} is invalid or out of range, the method
\code{unknown_charref(\var{ref})} is called to handle the error.  A
subclass must override this method to provide support for named
character entities.
\end{funcdesc}

\begin{funcdesc}{handle_entityref}{ref}
This method is called to process a general entity reference of the form
``\code{\&\var{ref};}'' where \var{ref} is an general entity
reference.  It looks for \var{ref} in the instance (or class)
variable \code{entitydefs} which should be a mapping from entity names
to corresponding translations.
If a translation is found, it calls the method \code{handle_data()}
with the translation; otherwise, it calls the method
\code{unknown_entityref(\var{ref})}.  The default \code{entitydefs}
defines translations for \code{\&amp;}, \code{\&apos}, \code{\&gt;},
\code{\&lt;}, and \code{\&quot;}.
\end{funcdesc}

\begin{funcdesc}{handle_comment}{comment}
This method is called when a comment is encountered.  The
\code{comment} argument is a string containing the text between the
``\code{<!--}'' and ``\code{-->}'' delimiters, but not the delimiters
themselves.  For example, the comment ``\code{<!--text-->}'' will
cause this method to be called with the argument \code{'text'}.  The
default method does nothing.
\end{funcdesc}

\begin{funcdesc}{report_unbalanced}{tag}
This method is called when an end tag is found which does not
correspond to any open element.
\end{funcdesc}

\begin{funcdesc}{unknown_starttag}{tag\, attributes}
This method is called to process an unknown start tag.  It is intended
to be overridden by a derived class; the base class implementation
does nothing.
\end{funcdesc}

\begin{funcdesc}{unknown_endtag}{tag}
This method is called to process an unknown end tag.  It is intended
to be overridden by a derived class; the base class implementation
does nothing.
\end{funcdesc}

\begin{funcdesc}{unknown_charref}{ref}
This method is called to process unresolvable numeric character
references.  It is intended to be overridden by a derived class; the
base class implementation does nothing.
\end{funcdesc}

\begin{funcdesc}{unknown_entityref}{ref}
This method is called to process an unknown entity reference.  It is
intended to be overridden by a derived class; the base class
implementation does nothing.
\end{funcdesc}

Apart from overriding or extending the methods listed above, derived
classes may also define methods of the following form to define
processing of specific tags.  Tag names in the input stream are case
independent; the \var{tag} occurring in method names must be in lower
case:

\begin{funcdesc}{start_\var{tag}}{attributes}
This method is called to process an opening tag \var{tag}.  It has
preference over \code{do_\var{tag}()}.  The \var{attributes} argument
has the same meaning as described for \code{handle_starttag()} above.
\end{funcdesc}

\begin{funcdesc}{do_\var{tag}}{attributes}
This method is called to process an opening tag \var{tag} that does
not come with a matching closing tag.  The \var{attributes} argument
has the same meaning as described for \code{handle_starttag()} above.
\end{funcdesc}

\begin{funcdesc}{end_\var{tag}}{}
This method is called to process a closing tag \var{tag}.
\end{funcdesc}

Note that the parser maintains a stack of open elements for which no
end tag has been found yet.  Only tags processed by
\code{start_\var{tag}()} are pushed on this stack.  Definition of an
\code{end_\var{tag}()} method is optional for these tags.  For tags
processed by \code{do_\var{tag}()} or by \code{unknown_tag()}, no
\code{end_\var{tag}()} method must be defined; if defined, it will not
be used.  If both \code{start_\var{tag}()} and \code{do_\var{tag}()}
methods exist for a tag, the \code{start_\var{tag}()} method takes
precedence.

\section{Standard Module \sectcode{htmllib}}
\stmodindex{htmllib}
\index{HTML}
\index{hypertext}

\renewcommand{\indexsubitem}{(in module htmllib)}

This module defines a class which can serve as a base for parsing text
files formatted in the HyperText Mark-up Language (HTML).  The class
is not directly concerned with I/O --- it must be provided with input
in string form via a method, and makes calls to methods of a
``formatter'' object in order to produce output.  The
\code{HTMLParser} class is designed to be used as a base class for
other classes in order to add functionality, and allows most of its
methods to be extended or overridden.  In turn, this class is derived
from and extends the \code{SGMLParser} class defined in module
\code{sgmllib}.  Two implementations of formatter objects are
provided in the \code{formatter} module; refer to the documentation
for that module for information on the formatter interface.
\index{SGML}
\stmodindex{sgmllib}
\ttindex{SGMLParser}
\index{formatter}
\stmodindex{formatter}

The following is a summary of the interface defined by
\code{sgmllib.SGMLParser}:

\begin{itemize}

\item
The interface to feed data to an instance is through the \code{feed()}
method, which takes a string argument.  This can be called with as
little or as much text at a time as desired; \code{p.feed(a);
p.feed(b)} has the same effect as \code{p.feed(a+b)}.  When the data
contains complete HTML tags, these are processed immediately;
incomplete elements are saved in a buffer.  To force processing of all
unprocessed data, call the \code{close()} method.

For example, to parse the entire contents of a file, use:
\begin{verbatim}
parser.feed(open('myfile.html').read())
parser.close()
\end{verbatim}

\item
The interface to define semantics for HTML tags is very simple: derive
a class and define methods called \code{start_\var{tag}()},
\code{end_\var{tag}()}, or \code{do_\var{tag}()}.  The parser will
call these at appropriate moments: \code{start_\var{tag}} or
\code{do_\var{tag}} is called when an opening tag of the form
\code{<\var{tag} ...>} is encountered; \code{end_\var{tag}} is called
when a closing tag of the form \code{<\var{tag}>} is encountered.  If
an opening tag requires a corresponding closing tag, like \code{<H1>}
... \code{</H1>}, the class should define the \code{start_\var{tag}}
method; if a tag requires no closing tag, like \code{<P>}, the class
should define the \code{do_\var{tag}} method.

\end{itemize}

The module defines a single class:

\begin{funcdesc}{HTMLParser}{formatter}
This is the basic HTML parser class.  It supports all entity names
required by the HTML 2.0 specification (RFC 1866).  It also defines
handlers for all HTML 2.0 and many HTML 3.0 and 3.2 elements.
\end{funcdesc}

In addition to tag methods, the \code{HTMLParser} class provides some
additional methods and instance variables for use within tag methods.

\renewcommand{\indexsubitem}{({\tt HTMLParser} method)}

\begin{datadesc}{formatter}
This is the formatter instance associated with the parser.
\end{datadesc}

\begin{datadesc}{nofill}
Boolean flag which should be true when whitespace should not be
collapsed, or false when it should be.  In general, this should only
be true when character data is to be treated as ``preformatted'' text,
as within a \code{<PRE>} element.  The default value is false.  This
affects the operation of \code{handle_data()} and \code{save_end()}.
\end{datadesc}

\begin{funcdesc}{anchor_bgn}{href\, name\, type}
This method is called at the start of an anchor region.  The arguments
correspond to the attributes of the \code{<A>} tag with the same
names.  The default implementation maintains a list of hyperlinks
(defined by the \code{href} argument) within the document.  The list
of hyperlinks is available as the data attribute \code{anchorlist}.
\end{funcdesc}

\begin{funcdesc}{anchor_end}{}
This method is called at the end of an anchor region.  The default
implementation adds a textual footnote marker using an index into the
list of hyperlinks created by \code{anchor_bgn()}.
\end{funcdesc}

\begin{funcdesc}{handle_image}{source\, alt\optional{\, ismap\optional{\, align\optional{\, width\optional{\, height}}}}}
This method is called to handle images.  The default implementation
simply passes the \code{alt} value to the \code{handle_data()}
method.
\end{funcdesc}

\begin{funcdesc}{save_bgn}{}
Begins saving character data in a buffer instead of sending it to the
formatter object.  Retrieve the stored data via \code{save_end()}
Use of the \code{save_bgn()} / \code{save_end()} pair may not be
nested.
\end{funcdesc}

\begin{funcdesc}{save_end}{}
Ends buffering character data and returns all data saved since the
preceeding call to \code{save_bgn()}.  If \code{nofill} flag is false,
whitespace is collapsed to single spaces.  A call to this method
without a preceeding call to \code{save_bgn()} will raise a
\code{TypeError} exception.
\end{funcdesc}

\section{Standard Module \sectcode{formatter}}
\stmodindex{formatter}

\renewcommand{\indexsubitem}{(in module formatter)}

This module supports two interface definitions, each with mulitple
implementations.  The \emph{formatter} interface is used by the
\code{HTMLParser} class of the \code{htmllib} module, and the
\emph{writer} interface is required by the formatter interface.

Formatter objects transform an abstract flow of formatting events into
specific output events on writer objects.  Formatters manage several
stack structures to allow various properties of a writer object to be
changed and restored; writers need not be able to handle relative
changes nor any sort of ``change back'' operation.  Specific writer
properties which may be controlled via formatter objects are
horizontal alignment, font, and left margin indentations.  A mechanism
is provided which supports providing arbitrary, non-exclusive style
settings to a writer as well.  Additional interfaces facilitate
formatting events which are not reversible, such as paragraph
separation.

Writer objects encapsulate device interfaces.  Abstract devices, such
as file formats, are supported as well as physical devices.  The
provided implementations all work with abstract devices.  The
interface makes available mechanisms for setting the properties which
formatter objects manage and inserting data into the output.


\subsection{The Formatter Interface}

Interfaces to create formatters are dependent on the specific
formatter class being instantiated.  The interfaces described below
are the required interfaces which all formatters must support once
initialized.

One data element is defined at the module level:

\begin{datadesc}{AS_IS}
Value which can be used in the font specification passed to the
\code{push_font()} method described below, or as the new value to any
other \code{push_\var{property}()} method.  Pushing the \code{AS_IS}
value allows the corresponding \code{pop_\var{property}()} method to
be called without having to track whether the property was changed.
\end{datadesc}

The following attributes are defined for formatter instance objects:

\renewcommand{\indexsubitem}{(formatter object data)}

\begin{datadesc}{writer}
The writer instance with which the formatter interacts.
\end{datadesc}


\renewcommand{\indexsubitem}{(formatter object method)}

\begin{funcdesc}{end_paragraph}{blanklines}
Close any open paragraphs and insert at least \code{blanklines}
before the next paragraph.
\end{funcdesc}

\begin{funcdesc}{add_line_break}{}
Add a hard line break if one does not already exist.  This does not
break the logical paragraph.
\end{funcdesc}

\begin{funcdesc}{add_hor_rule}{*args\, **kw}
Insert a horizontal rule in the output.  A hard break is inserted if
there is data in the current paragraph, but the logical paragraph is
not broken.  The arguments and keywords are passed on to the writer's
\code{send_line_break()} method.
\end{funcdesc}

\begin{funcdesc}{add_flowing_data}{data}
Provide data which should be formatted with collapsed whitespaces.
Whitespace from preceeding and successive calls to
\code{add_flowing_data()} is considered as well when the whitespace
collapse is performed.  The data which is passed to this method is
expected to be word-wrapped by the output device.  Note that any
word-wrapping still must be performed by the writer object due to the
need to rely on device and font information.
\end{funcdesc}

\begin{funcdesc}{add_literal_data}{data}
Provide data which should be passed to the writer unchanged.
Whitespace, including newline and tab characters, are considered legal
in the value of \code{data}.  
\end{funcdesc}

\begin{funcdesc}{add_label_data}{format, counter}
Insert a label which should be placed to the left of the current left
margin.  This should be used for constructing bulleted or numbered
lists.  If the \code{format} value is a string, it is interpreted as a
format specification for \code{counter}, which should be an integer.
The result of this formatting becomes the value of the label; if
\code{format} is not a string it is used as the label value directly.
The label value is passed as the only argument to the writer's
\code{send_label_data()} method.  Interpretation of non-string label
values is dependent on the associated writer.

Format specifications are strings which, in combination with a counter
value, are used to compute label values.  Each character in the format
string is copied to the label value, with some characters recognized
to indicate a transform on the counter value.  Specifically, the
character ``\code{1}'' represents the counter value formatter as an
arabic number, the characters ``\code{A}'' and ``\code{a}'' represent
alphabetic representations of the counter value in upper and lower
case, respectively, and ``\code{I}'' and ``\code{i}'' represent the
counter value in Roman numerals, in upper and lower case.  Note that
the alphabetic and roman transforms require that the counter value be
greater than zero.
\end{funcdesc}

\begin{funcdesc}{flush_softspace}{}
Send any pending whitespace buffered from a previous call to
\code{add_flowing_data()} to the associated writer object.  This
should be called before any direct manipulation of the writer object.
\end{funcdesc}

\begin{funcdesc}{push_alignment}{align}
Push a new alignment setting onto the alignment stack.  This may be
\code{AS_IS} if no change is desired.  If the alignment value is
changed from the previous setting, the writer's \code{new_alignment()}
method is called with the \code{align} value.
\end{funcdesc}

\begin{funcdesc}{pop_alignment}{}
Restore the previous alignment.
\end{funcdesc}

\begin{funcdesc}{push_font}{(size, italic, bold, teletype)}
Change some or all font properties of the writer object.  Properties
which are not set to \code{AS_IS} are set to the values passed in
while others are maintained at their current settings.  The writer's
\code{new_font()} method is called with the fully resolved font
specification.
\end{funcdesc}

\begin{funcdesc}{pop_font}{}
Restore the previous font.
\end{funcdesc}

\begin{funcdesc}{push_margin}{margin}
Increase the number of left margin indentations by one, associating
the logical tag \code{margin} with the new indentation.  The initial
margin level is \code{0}.  Changed values of the logical tag must be
true values; false values other than \code{AS_IS} are not sufficient
to change the margin.
\end{funcdesc}

\begin{funcdesc}{pop_margin}{}
Restore the previous margin.
\end{funcdesc}

\begin{funcdesc}{push_style}{*styles}
Push any number of arbitrary style specifications.  All styles are
pushed onto the styles stack in order.  A tuple representing the
entire stack, including \code{AS_IS} values, is passed to the writer's
\code{new_styles()} method.
\end{funcdesc}

\begin{funcdesc}{pop_style}{\optional{n\code{ = 1}}}
Pop the last \code{n} style specifications passed to
\code{push_style()}.  A tuple representing the revised stack,
including \code{AS_IS} values, is passed to the writer's
\code{new_styles()} method.
\end{funcdesc}

\begin{funcdesc}{set_spacing}{spacing}
Set the spacing style for the writer.
\end{funcdesc}

\begin{funcdesc}{assert_line_data}{\optional{flag\code{ = 1}}}
Inform the formatter that data has been added to the current paragraph
out-of-band.  This should be used when the writer has been manipulated
directly.  The optional \code{flag} argument can be set to false if
the writer manipulations produced a hard line break at the end of the
output.
\end{funcdesc}


\subsection{Formatter Implementations}

Two implementations of formatter objects are provided by this module.
Most applications may use one of these classes without modification or
subclassing.

\renewcommand{\indexsubitem}{(in module formatter)}

\begin{funcdesc}{NullFormatter}{\optional{writer\code{ = None}}}
A formatter which does nothing.  If \code{writer} is omitted, a
\code{NullWriter} instance is created.  No methods of the writer are
called by \code{NullWriter} instances.  Implementations should inherit
from this class if implementing a writer interface but don't need to
inherit any implementation.
\end{funcdesc}

\begin{funcdesc}{AbstractFormatter}{writer}
The standard formatter.  This implementation has demonstrated wide
applicability to many writers, and may be used directly in most
circumstances.  It has been used to implement a full-featured
world-wide web browser.
\end{funcdesc}



\subsection{The Writer Interface}

Interfaces to create writers are dependent on the specific writer
class being instantiated.  The interfaces described below are the
required interfaces which all writers must support once initialized.
Note that while most applications can use the \code{AbstractFormatter}
class as a formatter, the writer must typically be provided by the
application.

\renewcommand{\indexsubitem}{(writer object method)}

\begin{funcdesc}{new_alignment}{align}
Set the alignment style.  The \code{align} value can be any object,
but by convention is a string or \code{None}, where \code{None}
indicates that the writer's ``preferred'' alignment should be used.
Conventional \code{align} values are \code{'left'}, \code{'center'},
\code{'right'}, and \code{'justify'}.
\end{funcdesc}

\begin{funcdesc}{new_font}{font}
Set the font style.  The value of \code{font} will be \code{None},
indicating that the device's default font should be used, or a tuple
of the form (\var{size}, \var{italic}, \var{bold}, \var{teletype}).
Size will be a string indicating the size of font that should be used;
specific strings and their interpretation must be defined by the
application.  The \var{italic}, \var{bold}, and \var{teletype} values
are boolean indicators specifying which of those font attributes
should be used.
\end{funcdesc}

\begin{funcdesc}{new_margin}{margin, level}
Set the margin level to the integer \code{level} and the logical tag
to \code{margin}.  Interpretation of the logical tag is at the
writer's discretion; the only restriction on the value of the logical
tag is that it not be a false value for non-zero values of
\code{level}.
\end{funcdesc}

\begin{funcdesc}{new_spacing}{spacing}
Set the spacing style to \code{spacing}.
\end{funcdesc}

\begin{funcdesc}{new_styles}{styles}
Set additional styles.  The \code{styles} value is a tuple of
arbitrary values; the value \code{AS_IS} should be ignored.  The
\code{styles} tuple may be interpreted either as a set or as a stack
depending on the requirements of the application and writer
implementation.
\end{funcdesc}

\begin{funcdesc}{send_line_break}{}
Break the current line.
\end{funcdesc}

\begin{funcdesc}{send_paragraph}{blankline}
Produce a paragraph separation of at least \code{blankline} blank
lines, or the equivelent.  The \code{blankline} value will be an
integer.
\end{funcdesc}

\begin{funcdesc}{send_hor_rule}{*args\, **kw}
Display a horizontal rule on the output device.  The arguments to this
method are entirely application- and writer-specific, and should be
interpreted with care.  The method implementation may assume that a
line break has already been issued via \code{send_line_break()}.
\end{funcdesc}

\begin{funcdesc}{send_flowing_data}{data}
Output character data which may be word-wrapped and re-flowed as
needed.  Within any sequence of calls to this method, the writer may
assume that spans of multiple whitespace characters have been
collapsed to single space characters.
\end{funcdesc}

\begin{funcdesc}{send_literal_data}{data}
Output character data which has already been formatted
for display.  Generally, this should be interpreted to mean that line
breaks indicated by newline characters should be preserved and no new
line breaks should be introduced.  The data may contain embedded
newline and tab characters, unlike data provided to the
\code{send_formatted_data()} interface.
\end{funcdesc}

\begin{funcdesc}{send_label_data}{data}
Set \code{data} to the left of the current left margin, if possible.
The value of \code{data} is not restricted; treatment of non-string
values is entirely application- and writer-dependent.  This method
will only be called at the beginning of a line.
\end{funcdesc}


\subsection{Writer Implementations}

Three implementations of the writer object interface are provided as
examples by this module.  Most applications will need to derive new
writer classes from the \code{NullWriter} class.

\renewcommand{\indexsubitem}{(in module formatter)}

\begin{funcdesc}{NullWriter}{}
A writer which only provides the interface definition; no actions are
taken on any methods.  This should be the base class for all writers
which do not need to inherit any implementation methods.
\end{funcdesc}

\begin{funcdesc}{AbstractWriter}{}
A writer which can be used in debugging formatters, but not much
else.  Each method simply accounces itself by printing its name and
arguments on standard output.
\end{funcdesc}

\begin{funcdesc}{DumbWriter}{\optional{file\code{ = None}\optional{\, maxcol\code{ = 72}}}}
Simple writer class which writes output on the file object passed in
as \code{file} or, if \code{file} is omitted, on standard output.  The
output is simply word-wrapped to the number of columns specified by
\code{maxcol}.  This class is suitable for reflowing a sequence of
paragraphs.
\end{funcdesc}

\section{Standard Module \sectcode{rfc822}}
\stmodindex{rfc822}

\renewcommand{\indexsubitem}{(in module rfc822)}

This module defines a class, \code{Message}, which represents a
collection of ``email headers'' as defined by the Internet standard
RFC 822.  It is used in various contexts, usually to read such headers
from a file.

A \code{Message} instance is instantiated with an open file object as
parameter.  Instantiation reads headers from the file up to a blank
line and stores them in the instance; after instantiation, the file is
positioned directly after the blank line that terminates the headers.

Input lines as read from the file may either be terminated by CR-LF or
by a single linefeed; a terminating CR-LF is replaced by a single
linefeed before the line is stored.

All header matching is done independent of upper or lower case;
e.g. \code{m['From']}, \code{m['from']} and \code{m['FROM']} all yield
the same result.

\subsection{Message Objects}

A \code{Message} instance has the following methods:

\begin{funcdesc}{rewindbody}{}
Seek to the start of the message body.  This only works if the file
object is seekable.
\end{funcdesc}

\begin{funcdesc}{getallmatchingheaders}{name}
Return a list of lines consisting of all headers matching
\var{name}, if any.  Each physical line, whether it is a continuation
line or not, is a separate list item.  Return the empty list if no
header matches \var{name}.
\end{funcdesc}

\begin{funcdesc}{getfirstmatchingheader}{name}
Return a list of lines comprising the first header matching
\var{name}, and its continuation line(s), if any.  Return \code{None}
if there is no header matching \var{name}.
\end{funcdesc}

\begin{funcdesc}{getrawheader}{name}
Return a single string consisting of the text after the colon in the
first header matching \var{name}.  This includes leading whitespace,
the trailing linefeed, and internal linefeeds and whitespace if there
any continuation line(s) were present.  Return \code{None} if there is
no header matching \var{name}.
\end{funcdesc}

\begin{funcdesc}{getheader}{name}
Like \code{getrawheader(\var{name})}, but strip leading and trailing
whitespace (but not internal whitespace).
\end{funcdesc}

\begin{funcdesc}{getaddr}{name}
Return a pair (full name, email address) parsed from the string
returned by \code{getheader(\var{name})}.  If no header matching
\var{name} exists, return \code{None, None}; otherwise both the full
name and the address are (possibly empty )strings.

Example: If \code{m}'s first \code{From} header contains the string\\
\code{'jack@cwi.nl (Jack Jansen)'}, then
\code{m.getaddr('From')} will yield the pair
\code{('Jack Jansen', 'jack@cwi.nl')}.
If the header contained
\code{'Jack Jansen <jack@cwi.nl>'} instead, it would yield the
exact same result.
\end{funcdesc}

\begin{funcdesc}{getaddrlist}{name}
This is similar to \code{getaddr(\var{list})}, but parses a header
containing a list of email addresses (e.g. a \code{To} header) and
returns a list of (full name, email address) pairs (even if there was
only one address in the header).  If there is no header matching
\var{name}, return an empty list.

XXX The current version of this function is not really correct.  It
yields bogus results if a full name contains a comma.
\end{funcdesc}

\begin{funcdesc}{getdate}{name}
Retrieve a header using \code{getheader} and parse it into a 9-tuple
compatible with \code{time.mktime()}.  If there is no header matching
\var{name}, or it is unparsable, return \code{None}.

Date parsing appears to be a black art, and not all mailers adhere to
the standard.  While it has been tested and found correct on a large
collection of email from many sources, it is still possible that this
function may occasionally yield an incorrect result.
\end{funcdesc}

\code{Message} instances also support a read-only mapping interface.
In particular: \code{m[name]} is the same as \code{m.getheader(name)};
and \code{len(m)}, \code{m.has_key(name)}, \code{m.keys()},
\code{m.values()} and \code{m.items()} act as expected (and
consistently).

Finally, \code{Message} instances have two public instance variables:

\begin{datadesc}{headers}
A list containing the entire set of header lines, in the order in
which they were read.  Each line contains a trailing newline.  The
blank line terminating the headers is not contained in the list.
\end{datadesc}

\begin{datadesc}{fp}
The file object passed at instantiation time.
\end{datadesc}

\section{Standard Module \sectcode{mimetools}}
\stmodindex{mimetools}

\renewcommand{\indexsubitem}{(in module mimetools)}

This module defines a subclass of the class \code{rfc822.Message} and
a number of utility functions that are useful for the manipulation for
MIME style multipart or encoded message.

It defines the following items:

\begin{funcdesc}{Message}{fp}
Return a new instance of the \code{mimetools.Message} class.  This is
a subclass of the \code{rfc822.Message} class, with some additional
methods (see below).
\end{funcdesc}

\begin{funcdesc}{choose_boundary}{}
Return a unique string that has a high likelihood of being usable as a
part boundary.  The string has the form
\code{"\var{hostipaddr}.\var{uid}.\var{pid}.\var{timestamp}.\var{random}"}.
\end{funcdesc}

\begin{funcdesc}{decode}{input\, output\, encoding}
Read data encoded using the allowed MIME \var{encoding} from open file
object \var{input} and write the decoded data to open file object
\var{output}.  Valid values for \var{encoding} include
\code{"base64"}, \code{"quoted-printable"} and \code{"uuencode"}.
\end{funcdesc}

\begin{funcdesc}{encode}{input\, output\, encoding}
Read data from open file object \var{input} and write it encoded using
the allowed MIME \var{encoding} to open file object \var{output}.
Valid values for \var{encoding} are the same as for \code{decode()}.
\end{funcdesc}

\begin{funcdesc}{copyliteral}{input\, output}
Read lines until EOF from open file \var{input} and write them to open
file \var{output}.
\end{funcdesc}

\begin{funcdesc}{copybinary}{input\, output}
Read blocks until EOF from open file \var{input} and write them to open
file \var{output}.  The block size is currently fixed at 8192.
\end{funcdesc}


\subsection{Additional Methods of Message objects}
\nodename{mimetools.Message Methods}

The \code{mimetools.Message} class defines the following methods in
addition to the \code{rfc822.Message} class:

\renewcommand{\indexsubitem}{(mimetool.Message method)}

\begin{funcdesc}{getplist}{}
Return the parameter list of the \code{Content-type} header.  This is
a list if strings.  For parameters of the form
\samp{\var{key}=\var{value}}, \var{key} is converted to lower case but
\var{value} is not.  For example, if the message contains the header
\samp{Content-type: text/html; spam=1; Spam=2; Spam} then
\code{getplist()} will return the Python list \code{['spam=1',
'spam=2', 'Spam']}.
\end{funcdesc}

\begin{funcdesc}{getparam}{name}
Return the \var{value} of the first parameter (as returned by
\code{getplist()} of the form \samp{\var{name}=\var{value}} for the
given \var{name}.  If \var{value} is surrounded by quotes of the form
\var{<...>} or \var{"..."}, these are removed.
\end{funcdesc}

\begin{funcdesc}{getencoding}{}
Return the encoding specified in the \samp{Content-transfer-encoding}
message header.  If no such header exists, return \code{"7bit"}.  The
encoding is converted to lower case.
\end{funcdesc}

\begin{funcdesc}{gettype}{}
Return the message type (of the form \samp{\var{type}/var{subtype}})
as specified in the \samp{Content-type} header.  If no such header
exists, return \code{"text/plain"}.  The type is converted to lower
case.
\end{funcdesc}

\begin{funcdesc}{getmaintype}{}
Return the main type as specified in the \samp{Content-type} header.
If no such header exists, return \code{"text"}.  The main type is
converted to lower case.
\end{funcdesc}

\begin{funcdesc}{getsubtype}{}
Return the subtype as specified in the \samp{Content-type} header.  If
no such header exists, return \code{"plain"}.  The subtype is
converted to lower case.
\end{funcdesc}

\section{Standard module \sectcode{binhex}}
\stmodindex{binhex}

This module encodes and decodes files in binhex4 format, a format
allowing representation of Macintosh files in ASCII. On the macintosh,
both forks of a file and the finder information are encoded (or
decoded), on other platforms only the data fork is handled.

The \code{binhex} module defines the following functions:

\renewcommand{\indexsubitem}{(in module binhex)}

\begin{funcdesc}{binhex}{input\, output}
Convert a binary file with filename \var{input} to binhex file
\var{output}. The \var{output} parameter can either be a filename or a
file-like object (any object supporting a \var{write} and \var{close}
method).
\end{funcdesc}

\begin{funcdesc}{hexbin}{input\optional{\, output}}
Decode a binhex file \var{input}. \var{Input} may be a filename or a
file-like object supporting \var{read} and \var{close} methods.
The resulting file is written to a file named \var{output}, unless the
argument is empty in which case the output filename is read from the
binhex file.
\end{funcdesc}

\subsection{notes}
There is an alternative, more powerful interface to the coder and
decoder, see the source for details.

If you code or decode textfiles on non-Macintosh platforms they will
still use the macintosh newline convention (carriage-return as end of
line).

As of this writing, \var{hexbin} appears to not work in all cases.

\section{Standard module \sectcode{uu}}
\stmodindex{uu}

This module encodes and decodes files in uuencode format, allowing
arbitrary binary data to be transferred over ascii-only connections.
Whereever a file argument is expected, the methods accept either a
pathname (\code{'-'} for stdin/stdout) or a file-like object.

Normally you would pass filenames, but there is one case where you
have to open the file yourself: if you are on a non-unix platform and
your binary file is actually a textfile that you want encoded
unix-compatible you will have to open the file yourself as a textfile,
so newline conversion is performed.

This code was contributed by Lance Ellinghouse, and modified by Jack
Jansen.

The \code{uu} module defines the following functions:

\renewcommand{\indexsubitem}{(in module uu)}

\begin{funcdesc}{encode}{in_file\, out_file\optional{\, name\, mode}}
Uuencode file \var{in_file} into file \var{out_file}.  The uuencoded
file will have the header specifying \var{name} and \var{mode} as the
defaults for the results of decoding the file. The default defaults
are taken from \var{in_file}, or \code{'-'} and \code{0666}
respectively. 
\end{funcdesc}

\begin{funcdesc}{decode}{in_file\optional{\, out_file\, mode}}
This call decodes uuencoded file \var{in_file} placing the result on
file \var{out_file}. If \var{out_file} is a pathname the \var{mode} is
also set. Defaults for \var{out_file} and \var{mode} are taken from
the uuencode header.
\end{funcdesc}

\section{Built-in Module \sectcode{binascii}}	% If implemented in C
\bimodindex{binascii}

The binascii module contains a number of methods to convert between
binary and various ascii-encoded binary representations. Normally, you
will not use these modules directly but use wrapper modules like
\var{uu} or \var{hexbin} in stead, this module solely exists because
bit-manipuation of large amounts of data is slow in python.

The \code{binascii} module defines the following functions:

\renewcommand{\indexsubitem}{(in module binascii)}

\begin{funcdesc}{a2b_uu}{string}
Convert a single line of uuencoded data back to binary and return the
binary data. Lines normally contain 45 (binary) bytes, except for the
last line. Line data may be followed by whitespace.
\end{funcdesc}

\begin{funcdesc}{b2a_uu}{data}
Convert binary data to a line of ascii characters, the return value is
the converted line, including a newline char. The length of \var{data}
should be at most 45.
\end{funcdesc}

\begin{funcdesc}{a2b_base64}{string}
Convert a block of base64 data back to binary and return the
binary data. More than one line may be passed at a time.
\end{funcdesc}

\begin{funcdesc}{b2a_base64}{data}
Convert binary data to a line of ascii characters in base64 coding.
The return value is the converted line, including a newline char.
The length of \var{data} should be at most 57 to adhere to the base64
standard.
\end{funcdesc}

\begin{funcdesc}{a2b_hqx}{string}
Convert binhex4 formatted ascii data to binary, without doing
rle-decompression. The string should contain a complete number of
binary bytes, or (in case of the last portion of the binhex4 data)
have the remaining bits zero.
\end{funcdesc}

\begin{funcdesc}{rledecode_hqx}{data}
Perform RLE-decompression on the data, as per the binhex4
standard. The algorithm uses \code{0x90} after a byte as a repeat
indicator, followed by a count. A count of \code{0} specifies a byte
value of \code{0x90}. The routine returns the decompressed data,
unless data input data ends in an orphaned repeat indicator, in which
case the \var{Incomplete} exception is raised.
\end{funcdesc}

\begin{funcdesc}{rlecode_hqx}{data}
Perform binhex4 style RLE-compression on \var{data} and return the
result.
\end{funcdesc}

\begin{funcdesc}{b2a_hqx}{data}
Perform hexbin4 binary-to-ascii translation and return the resulting
string. The argument should already be rle-coded, and have a length
divisible by 3 (except possibly the last fragment).
\end{funcdesc}

\begin{funcdesc}{crc_hqx}{data, crc}
Compute the binhex4 crc value of \var{data}, starting with an initial
\var{crc} and returning the result.
\end{funcdesc}
 
\begin{excdesc}{Error}
Exception raised on errors. These are usually programming errors.
\end{excdesc}

\begin{excdesc}{Incomplete}
Exception raised on incomplete data. These are usually not programming
errors, but handled by reading a little more data and trying again.
\end{excdesc}

\section{Standard module \sectcode{xdrlib}}
\stmodindex{xdrlib}
\index{XDR}

\renewcommand{\indexsubitem}{(in module xdrlib)}


The \code{xdrlib} module supports the External Data Representation
Standard as described in RFC 1014, written by Sun Microsystems,
Inc. June 1987.  It supports most of the data types described in the
RFC, although some, most notably \code{float} and \code{double} are
only supported on those operating systems that provide an XDR
library.

The \code{xdrlib} module defines two classes, one for packing
variables into XDR representation, and another for unpacking from XDR
representation.  There are also two exception classes.


\subsection{Packer Objects}

\code{Packer} is the class for packing data into XDR representation.
The \code{Packer} class is instantiated with no arguments.

\begin{funcdesc}{get_buffer}{}
Returns the current pack buffer as a string.
\end{funcdesc}

\begin{funcdesc}{reset}{}
Resets the pack buffer to the empty string.
\end{funcdesc}

In general, you can pack any of the most common XDR data types by
calling the appropriate \code{pack_\var{type}} method.  Each method
takes a single argument, the value to pack.  The following simple data
type packing methods are supported: \code{pack_uint}, \code{pack_int},
\code{pack_enum}, \code{pack_bool}, \code{pack_uhyper},
and \code{pack_hyper}.

The following methods pack floating point numbers, however they
require C library support.  Without the optional C built-in module,
both of these methods will raise an \code{xdrlib.ConversionError}
exception.  See the note at the end of this chapter for details.

\begin{funcdesc}{pack_float}{value}
Packs the single-precision floating point number \var{value}.
\end{funcdesc}

\begin{funcdesc}{pack_double}{value}
Packs the double-precision floating point number \var{value}.
\end{funcdesc}

The following methods support packing strings, bytes, and opaque data:

\begin{funcdesc}{pack_fstring}{n\, s}
Packs a fixed length string, \var{s}.  \var{n} is the length of the
string but it is \emph{not} packed into the data buffer.  The string
is padded with null bytes if necessary to guaranteed 4 byte alignment.
\end{funcdesc}

\begin{funcdesc}{pack_fopaque}{n\, data}
Packs a fixed length opaque data stream, similarly to
\code{pack_fstring}.
\end{funcdesc}

\begin{funcdesc}{pack_string}{s}
Packs a variable length string, \var{s}.  The length of the string is
first packed as an unsigned integer, then the string data is packed
with \code{pack_fstring}.
\end{funcdesc}

\begin{funcdesc}{pack_opaque}{data}
Packs a variable length opaque data string, similarly to
\code{pack_string}.
\end{funcdesc}

\begin{funcdesc}{pack_bytes}{bytes}
Packs a variable length byte stream, similarly to \code{pack_string}.
\end{funcdesc}

The following methods support packing arrays and lists:

\begin{funcdesc}{pack_list}{list\, pack_item}
Packs a \var{list} of homogeneous items.  This method is useful for
lists with an indeterminate size; i.e. the size is not available until
the entire list has been walked.  For each item in the list, an
unsigned integer \code{1} is packed first, followed by the data value
from the list.  \var{pack_item} is the function that is called to pack
the individual item.  At the end of the list, an unsigned integer
\code{0} is packed.
\end{funcdesc}

\begin{funcdesc}{pack_farray}{n\, array\, pack_item}
Packs a fixed length list (\var{array}) of homogeneous items.  \var{n}
is the length of the list; it is \emph{not} packed into the buffer,
but a \code{ValueError} exception is raised if \code{len(array)} is not
equal to \var{n}.  As above, \var{pack_item} is the function used to
pack each element.
\end{funcdesc}

\begin{funcdesc}{pack_array}{list\, pack_item}
Packs a variable length \var{list} of homogeneous items.  First, the
length of the list is packed as an unsigned integer, then each element
is packed as in \code{pack_farray} above.
\end{funcdesc}

\subsection{Unpacker Objects}

\code{Unpacker} is the complementary class which unpacks XDR data
values from a string buffer, and has the following methods:

\begin{funcdesc}{__init__}{data}
Instantiates an \code{Unpacker} object with the string buffer
\var{data}.
\end{funcdesc}

\begin{funcdesc}{reset}{data}
Resets the string buffer with the given \var{data}.
\end{funcdesc}

\begin{funcdesc}{get_position}{}
Returns the current unpack position in the data buffer.
\end{funcdesc}

\begin{funcdesc}{set_position}{position}
Sets the data buffer unpack position to \var{position}.  You should be
careful about using \code{get_position()} and \code{set_position()}.
\end{funcdesc}

\begin{funcdesc}{done}{}
Indicates unpack completion.  Raises an \code{xdrlib.Error} exception
if all of the data has not been unpacked.
\end{funcdesc}

In addition, every data type that can be packed with a \code{Packer},
can be unpacked with an \code{Unpacker}.  Unpacking methods are of the
form \code{unpack_\var{type}}, and take no arguments.  They return the
unpacked object.  The same caveats apply for \code{unpack_float} and
\code{unpack_double} as above.

\begin{funcdesc}{unpack_float}{}
Unpacks a single-precision floating point number.
\end{funcdesc}

\begin{funcdesc}{unpack_double}{}
Unpacks a double-precision floating point number, similarly to
\code{unpack_float}.
\end{funcdesc}

In addition, the following methods unpack strings, bytes, and opaque
data:

\begin{funcdesc}{unpack_fstring}{n}
Unpacks and returns a fixed length string.  \var{n} is the number of
characters expected.  Padding with null bytes to guaranteed 4 byte
alignment is assumed.
\end{funcdesc}

\begin{funcdesc}{unpack_fopaque}{n}
Unpacks and returns a fixed length opaque data stream, similarly to
\code{unpack_fstring}.
\end{funcdesc}

\begin{funcdesc}{unpack_string}{}
Unpacks and returns a variable length string.  The length of the
string is first unpacked as an unsigned integer, then the string data
is unpacked with \code{unpack_fstring}.
\end{funcdesc}

\begin{funcdesc}{unpack_opaque}{}
Unpacks and returns a variable length opaque data string, similarly to
\code{unpack_string}.
\end{funcdesc}

\begin{funcdesc}{unpack_bytes}{}
Unpacks and returns a variable length byte stream, similarly to
\code{unpack_string}.
\end{funcdesc}

The following methods support unpacking arrays and lists:

\begin{funcdesc}{unpack_list}{unpack_item}
Unpacks and returns a list of homogeneous items.  The list is unpacked
one element at a time
by first unpacking an unsigned integer flag.  If the flag is \code{1},
then the item is unpacked and appended to the list.  A flag of
\code{0} indicates the end of the list.  \var{unpack_item} is the
function that is called to unpack the items.
\end{funcdesc}

\begin{funcdesc}{unpack_farray}{n\, unpack_item}
Unpacks and returns (as a list) a fixed length array of homogeneous
items.  \var{n} is number of list elements to expect in the buffer.
As above, \var{unpack_item} is the function used to unpack each element.
\end{funcdesc}

\begin{funcdesc}{unpack_array}{unpack_item}
Unpacks and returns a variable length \var{list} of homogeneous items.
First, the length of the list is unpacked as an unsigned integer, then
each element is unpacked as in \code{unpack_farray} above.
\end{funcdesc}

\subsection{Exceptions}

Exceptions in this module are coded as class instances:

\begin{excdesc}{Error}
The base exception class.  \code{Error} has a single public data
member \code{msg} containing the description of the error.
\end{excdesc}

\begin{excdesc}{ConversionError}
Class derived from \code{Error}.  Contains no additional instance
variables.
\end{excdesc}

Here is an example of how you would catch one of these exceptions:

\begin{verbatim}
import xdrlib
p = xdrlib.Packer()
try:
    p.pack_double(8.01)
except xdrlib.ConversionError, instance:
    print 'packing the double failed:', instance.msg
\end{verbatim}

\subsection{Supporting Floating Point Data}

Packing and unpacking floating point data,
i.e. \code{Packer.pack_float}, \code{Packer.pack_double},
\code{Unpacker.unpack_float}, and \code{Unpacker.unpack_double}, are
only supported with the helper built-in \code{_xdr} module, which
relies on your operating system having the appropriate XDR library
routines.

If you have built the Python interpeter with the \code{_xdr} module,
or have built the \code{_xdr} module as a shared library,
\code{xdrlib} will use these to pack and unpack floating point
numbers.  Otherwise, using these routines will raise a
\code{ConversionError} exception.

See the Python installation instructions for details on building the
\code{_xdr} module.


\chapter{Restricted Execution}

In general, Python programs have complete access to the underlying
operating system throug the various functions and classes, For
example, a Python program can open any file for reading and writing by
using the \code{open()} built-in function (provided the underlying OS
gives you permission!).  This is exactly what you want for most
applications.

There exists a class of applications for which this ``openness'' is
inappropriate.  Take Grail: a web browser that accepts ``applets'',
snippets of Python code, from anywhere on the Internet for execution
on the local system.  This can be used to improve the user interface
of forms, for instance.  Since the originator of the code is unknown,
it is obvious that it cannot be trusted with the full resources of the
local machine.

\emph{Restricted execution} is the basic framework in Python that allows
for the segregation of trusted and untrusted code.  It is based on the
notion that trusted Python code (a \emph{supervisor}) can create a
``padded cell' (or environment) with limited permissions, and run the
untrusted code within this cell.  The untrusted code cannot break out
of its cell, and can only interact with sensitive system resources
through interfaces defined and managed by the trusted code.  The term
``restricted execution'' is favored over ``safe-Python''
since true safety is hard to define, and is determined by the way the
restricted environment is created.  Note that the restricted
environments can be nested, with inner cells creating subcells of
lesser, but never greater, privilege.

An interesting aspect of Python's restricted execution model is that
the interfaces presented to untrusted code usually have the same names
as those presented to trusted code.  Therefore no special interfaces
need to be learned to write code designed to run in a restricted
environment.  And because the exact nature of the padded cell is
determined by the supervisor, different restrictions can be imposed,
depending on the application.  For example, it might be deemed
``safe'' for untrusted code to read any file within a specified
directory, but never to write a file.  In this case, the supervisor
may redefine the built-in
\code{open()} function so that it raises an exception whenever the
\var{mode} parameter is \code{'w'}.  It might also perform a
\code{chroot()}-like operation on the \var{filename} parameter, such
that root is always relative to some safe ``sandbox'' area of the
filesystem.  In this case, the untrusted code would still see an
built-in \code{open()} function in its environment, with the same
calling interface.  The semantics would be identical too, with
\code{IOError}s being raised when the supervisor determined that an
unallowable parameter is being used.

The Python run-time determines whether a particular code block is
executing in restricted execution mode based on the identity of the
\code{__builtins__} object in its global variables: if this is (the
dictionary of) the standard \code{__builtin__} module, the code is
deemed to be unrestricted, else it is deemed to be restricted.

Python code executing in restricted mode faces a number of limitations
that are designed to prevent it from escaping from the padded cell.
For instance, the function object attribute \code{func_globals} and the
class and instance object attribute \code{__dict__} are unavailable.

Two modules provide the framework for setting up restricted execution
environments:

\begin{description}

\item[rexec]
--- Basic restricted execution framework.

\item[Bastion]
--- Providing restricted access to objects.

\end{description}

\section{Standard Module \sectcode{rexec}}
\stmodindex{rexec}
\renewcommand{\indexsubitem}{(in module rexec)}

This module contains the \code{RExec} class, which supports
\code{r_exec()}, \code{r_eval()}, \code{r_execfile()}, and
\code{r_import()} methods, which are restricted versions of the standard
Python functions \code{exec()}, \code{eval()}, \code{execfile()}, and
the \code{import} statement.
Code executed in this restricted environment will
only have access to modules and functions that are deemed safe; you
can subclass \code{RExec} to add or remove capabilities as desired.

\emph{Note:} The \code{RExec} class can prevent code from performing
unsafe operations like reading or writing disk files, or using TCP/IP
sockets.  However, it does not protect against code using extremely
large amounts of memory or CPU time.  

\begin{funcdesc}{RExec}{\optional{hooks\optional{\, verbose}}}
Returns an instance of the \code{RExec} class.  

\var{hooks} is an instance of the \code{RHooks} class or a subclass of it.
If it is omitted or \code{None}, the default \code{RHooks} class is
instantiated.
Whenever the RExec module searches for a module (even a built-in one)
or reads a module's code, it doesn't actually go out to the file
system itself.  Rather, it calls methods of an RHooks instance that
was passed to or created by its constructor.  (Actually, the RExec
object doesn't make these calls---they are made by a module loader
object that's part of the RExec object.  This allows another level of
flexibility, e.g. using packages.)

By providing an alternate RHooks object, we can control the
file system accesses made to import a module, without changing the
actual algorithm that controls the order in which those accesses are
made.  For instance, we could substitute an RHooks object that passes
all filesystem requests to a file server elsewhere, via some RPC
mechanism such as ILU.  Grail's applet loader uses this to support
importing applets from a URL for a directory.

If \var{verbose} is true, additional debugging output may be sent to
standard output.
\end{funcdesc}

The RExec class has the following class attributes, which are used by the
\code{__init__} method.  Changing them on an existing instance won't
have any effect; instead, create a subclass of \code{RExec} and assign
them new values in the class definition.  Instances of the new class
will then use those new values.  All these attributes are tuples of
strings.

\renewcommand{\indexsubitem}{(RExec object attribute)}
\begin{datadesc}{nok_builtin_names}
Contains the names of built-in functions which will \emph{not} be
available to programs running in the restricted environment.  The
value for \code{RExec} is \code{('open',} \code{'reload',}
\code{'__import__')}.  (This gives the exceptions, because by far the
majority of built-in functions are harmless.  A subclass that wants to
override this variable should probably start with the value from the
base class and concatenate additional forbidden functions --- when new
dangerous built-in functions are added to Python, they will also be
added to this module.)
\end{datadesc}

\begin{datadesc}{ok_builtin_modules}
Contains the names of built-in modules which can be safely imported.
The value for \code{RExec} is \code{('audioop',} \code{'array',}
\code{'binascii',} \code{'cmath',} \code{'errno',} \code{'imageop',}
\code{'marshal',} \code{'math',} \code{'md5',} \code{'operator',}
\code{'parser',} \code{'regex',} \code{'rotor',} \code{'select',}
\code{'strop',} \code{'struct',} \code{'time')}.  A similar remark
about overriding this variable applies --- use the value from the base
class as a starting point.
\end{datadesc}

\begin{datadesc}{ok_path}
Contains the directories which will be searched when an \code{import}
is performed in the restricted environment.  
The value for \code{RExec} is the same as \code{sys.path} (at the time
the module is loaded) for unrestricted code.
\end{datadesc}

\begin{datadesc}{ok_posix_names}
% Should this be called ok_os_names?
Contains the names of the functions in the \code{os} module which will be
available to programs running in the restricted environment.  The
value for \code{RExec} is \code{('error',} \code{'fstat',}
\code{'listdir',} \code{'lstat',} \code{'readlink',} \code{'stat',}
\code{'times',} \code{'uname',} \code{'getpid',} \code{'getppid',}
\code{'getcwd',} \code{'getuid',} \code{'getgid',} \code{'geteuid',}
\code{'getegid')}.
\end{datadesc}

\begin{datadesc}{ok_sys_names}
Contains the names of the functions and variables in the \code{sys}
module which will be available to programs running in the restricted
environment.  The value for \code{RExec} is \code{('ps1',}
\code{'ps2',} \code{'copyright',} \code{'version',} \code{'platform',}
\code{'exit',} \code{'maxint')}.
\end{datadesc}

RExec instances support the following methods:
\renewcommand{\indexsubitem}{(RExec object method)}

\begin{funcdesc}{r_eval}{code}
\var{code} must either be a string containing a Python expression, or
a compiled code object, which will be evaluated in the restricted
environment's \code{__main__} module.  The value of the expression or
code object will be returned.
\end{funcdesc}

\begin{funcdesc}{r_exec}{code}
\var{code} must either be a string containing one or more lines of
Python code, or a compiled code object, which will be executed in the
restricted environment's \code{__main__} module.
\end{funcdesc}

\begin{funcdesc}{r_execfile}{filename}
Execute the Python code contained in the file \var{filename} in the
restricted environment's \code{__main__} module.
\end{funcdesc}

Methods whose names begin with \code{s_} are similar to the functions
beginning with \code{r_}, but the code will be granted access to
restricted versions of the standard I/O streans \code{sys.stdin},
\code{sys.stderr}, and \code{sys.stdout}.  

\begin{funcdesc}{s_eval}{code}
\var{code} must be a string containing a Python expression, which will
be evaluated in the restricted environment.  
\end{funcdesc}

\begin{funcdesc}{s_exec}{code}
\var{code} must be a string containing one or more lines of Python code,
which will be executed in the restricted environment.  
\end{funcdesc}

\begin{funcdesc}{s_execfile}{code}
Execute the Python code contained in the file \var{filename} in the
restricted environment.
\end{funcdesc}

\code{RExec} objects must also support various methods which will be
implicitly called by code executing in the restricted environment.
Overriding these methods in a subclass is used to change the policies
enforced by a restricted environment.

\begin{funcdesc}{r_import}{modulename\optional{\, globals\, locals\, fromlist}}
Import the module \var{modulename}, raising an \code{ImportError}
exception if the module is considered unsafe.
\end{funcdesc}

\begin{funcdesc}{r_open}{filename\optional{\, mode\optional{\, bufsize}}}
Method called when \code{open()} is called in the restricted
environment.  The arguments are identical to those of \code{open()},
and a file object (or a class instance compatible with file objects)
should be returned.  \code{RExec}'s default behaviour is allow opening
any file for reading, but forbidding any attempt to write a file.  See
the example below for an implementation of a less restrictive
\code{r_open()}.
\end{funcdesc}

\begin{funcdesc}{r_reload}{module}
Reload the module object \var{module}, re-parsing and re-initializing it.  
\end{funcdesc}

\begin{funcdesc}{r_unload}{module}
Unload the module object \var{module} (i.e., remove it from the
restricted environment's \code{sys.modules} dictionary).
\end{funcdesc}

And their equivalents with access to restricted standard I/O streams:

\begin{funcdesc}{s_import}{modulename\optional{\, globals, locals, fromlist}}
Import the module \var{modulename}, raising an \code{ImportError}
exception if the module is considered unsafe.
\end{funcdesc}

\begin{funcdesc}{s_reload}{module}
Reload the module object \var{module}, re-parsing and re-initializing it.  
\end{funcdesc}

\begin{funcdesc}{s_unload}{module}
Unload the module object \var{module}.   
% XXX what are the semantics of this?  
\end{funcdesc}

\subsection{An example}

Let us say that we want a slightly more relaxed policy than the
standard RExec class.  For example, if we're willing to allow files in
\file{/tmp} to be written, we can subclass the \code{RExec} class:

\bcode\begin{verbatim}
class TmpWriterRExec(rexec.RExec):
    def r_open(self, file, mode='r', buf=-1):
        if mode in ('r', 'rb'):
            pass
        elif mode in ('w', 'wb', 'a', 'ab'):
            # check filename : must begin with /tmp/
            if file[:5]!='/tmp/': 
                raise IOError, "can't write outside /tmp"
            elif (string.find(file, '/../') >= 0 or
                 file[:3] == '../' or file[-3:] == '/..'):
                raise IOError, "'..' in filename forbidden"
        else: raise IOError, "Illegal open() mode"
        return open(file, mode, buf)
\end{verbatim}\ecode

Notice that the above code will occasionally forbid a perfectly valid
filename; for example, code in the restricted environment won't be
able to open a file called \file{/tmp/foo/../bar}.  To fix this, the
\code{r_open} method would have to simplify the filename to
\file{/tmp/bar}, which would require splitting apart the filename and
performing various operations on it.  In cases where security is at
stake, it may be preferable to write simple code which is sometimes
overly restrictive, instead of more general code that is also more
complex and may harbor a subtle security hole.

\section{Standard Module \sectcode{Bastion}}
\stmodindex{Bastion}
\renewcommand{\indexsubitem}{(in module Bastion)}

% I'm concerned that the word 'bastion' won't be understood by people
% for whom English is a second language, making the module name
% somewhat mysterious.  Thus, the brief definition... --amk

According to the dictionary, a bastion is ``a fortified area or
position'', or ``something that is considered a stronghold.''  It's a
suitable name for this module, which provides a way to forbid access
to certain attributes of an object.  It must always be used with the
\code{rexec} module, in order to allow restricted-mode programs access
to certain safe attributes of an object, while denying access to
other, unsafe attributes.

% I've punted on the issue of documenting keyword arguments for now.

\begin{funcdesc}{Bastion}{object\optional{\, filter\, name\, class}}
Protect the class instance \var{object}, returning a bastion for the
object.  Any attempt to access one of the object's attributes will
have to be approved by the \var{filter} function; if the access is
denied an AttributeError exception will be raised.

If present, \var{filter} must be a function that accepts a string
containing an attribute name, and returns true if access to that
attribute will be permitted; if \var{filter} returns false, the access
is denied.  The default filter denies access to any function beginning
with an underscore (\code{_}).  The bastion's string representation
will be \code{<Bastion for \var{name}>} if a value for
\var{name} is provided; otherwise, \code{repr(\var{object})} will be used.

\var{class}, if present, would be a subclass of \code{BastionClass};
see the code in \file{bastion.py} for the details.  Overriding the
default \code{BastionClass} will rarely be required.  

\end{funcdesc}


\chapter{Multimedia Services}

The modules described in this chapter implement various algorithms or
interfaces that are mainly useful for multimedia applications.  They
are available at the discretion of the installation.  Here's an overview:

\begin{description}

\item[audioop]
--- Manipulate raw audio data.

\item[imageop]
--- Manipulate raw image data.

\item[aifc]
--- Read and write audio files in AIFF or AIFC format.

\item[jpeg]
--- Read and write image files in compressed JPEG format.

\item[rgbimg]
--- Read and write image files in ``SGI RGB'' format (the module is
\emph{not} SGI specific though)!

\end{description}
			% Multimedia Services
\section{Built-in Module \sectcode{audioop}}
\bimodindex{audioop}

The \code{audioop} module contains some useful operations on sound fragments.
It operates on sound fragments consisting of signed integer samples
8, 16 or 32 bits wide, stored in Python strings.  This is the same
format as used by the \code{al} and \code{sunaudiodev} modules.  All
scalar items are integers, unless specified otherwise.

A few of the more complicated operations only take 16-bit samples,
otherwise the sample size (in bytes) is always a parameter of the operation.

The module defines the following variables and functions:

\renewcommand{\indexsubitem}{(in module audioop)}
\begin{excdesc}{error}
This exception is raised on all errors, such as unknown number of bytes
per sample, etc.
\end{excdesc}

\begin{funcdesc}{add}{fragment1\, fragment2\, width}
Return a fragment which is the addition of the two samples passed as
parameters.  \var{width} is the sample width in bytes, either
\code{1}, \code{2} or \code{4}.  Both fragments should have the same
length.
\end{funcdesc}

\begin{funcdesc}{adpcm2lin}{adpcmfragment\, width\, state}
Decode an Intel/DVI ADPCM coded fragment to a linear fragment.  See
the description of \code{lin2adpcm} for details on ADPCM coding.
Return a tuple \code{(\var{sample}, \var{newstate})} where the sample
has the width specified in \var{width}.
\end{funcdesc}

\begin{funcdesc}{adpcm32lin}{adpcmfragment\, width\, state}
Decode an alternative 3-bit ADPCM code.  See \code{lin2adpcm3} for
details.
\end{funcdesc}

\begin{funcdesc}{avg}{fragment\, width}
Return the average over all samples in the fragment.
\end{funcdesc}

\begin{funcdesc}{avgpp}{fragment\, width}
Return the average peak-peak value over all samples in the fragment.
No filtering is done, so the usefulness of this routine is
questionable.
\end{funcdesc}

\begin{funcdesc}{bias}{fragment\, width\, bias}
Return a fragment that is the original fragment with a bias added to
each sample.
\end{funcdesc}

\begin{funcdesc}{cross}{fragment\, width}
Return the number of zero crossings in the fragment passed as an
argument.
\end{funcdesc}

\begin{funcdesc}{findfactor}{fragment\, reference}
Return a factor \var{F} such that
\code{rms(add(fragment, mul(reference, -F)))} is minimal, i.e.,
return the factor with which you should multiply \var{reference} to
make it match as well as possible to \var{fragment}.  The fragments
should both contain 2-byte samples.

The time taken by this routine is proportional to \code{len(fragment)}. 
\end{funcdesc}

\begin{funcdesc}{findfit}{fragment\, reference}
This routine (which only accepts 2-byte sample fragments)

Try to match \var{reference} as well as possible to a portion of
\var{fragment} (which should be the longer fragment).  This is
(conceptually) done by taking slices out of \var{fragment}, using
\code{findfactor} to compute the best match, and minimizing the
result.  The fragments should both contain 2-byte samples.  Return a
tuple \code{(\var{offset}, \var{factor})} where \var{offset} is the
(integer) offset into \var{fragment} where the optimal match started
and \var{factor} is the (floating-point) factor as per
\code{findfactor}.
\end{funcdesc}

\begin{funcdesc}{findmax}{fragment\, length}
Search \var{fragment} for a slice of length \var{length} samples (not
bytes!)\ with maximum energy, i.e., return \var{i} for which
\code{rms(fragment[i*2:(i+length)*2])} is maximal.  The fragments
should both contain 2-byte samples.

The routine takes time proportional to \code{len(fragment)}.
\end{funcdesc}

\begin{funcdesc}{getsample}{fragment\, width\, index}
Return the value of sample \var{index} from the fragment.
\end{funcdesc}

\begin{funcdesc}{lin2lin}{fragment\, width\, newwidth}
Convert samples between 1-, 2- and 4-byte formats.
\end{funcdesc}

\begin{funcdesc}{lin2adpcm}{fragment\, width\, state}
Convert samples to 4 bit Intel/DVI ADPCM encoding.  ADPCM coding is an
adaptive coding scheme, whereby each 4 bit number is the difference
between one sample and the next, divided by a (varying) step.  The
Intel/DVI ADPCM algorithm has been selected for use by the IMA, so it
may well become a standard.

\code{State} is a tuple containing the state of the coder.  The coder
returns a tuple \code{(\var{adpcmfrag}, \var{newstate})}, and the
\var{newstate} should be passed to the next call of lin2adpcm.  In the
initial call \code{None} can be passed as the state.  \var{adpcmfrag}
is the ADPCM coded fragment packed 2 4-bit values per byte.
\end{funcdesc}

\begin{funcdesc}{lin2adpcm3}{fragment\, width\, state}
This is an alternative ADPCM coder that uses only 3 bits per sample.
It is not compatible with the Intel/DVI ADPCM coder and its output is
not packed (due to laziness on the side of the author).  Its use is
discouraged.
\end{funcdesc}

\begin{funcdesc}{lin2ulaw}{fragment\, width}
Convert samples in the audio fragment to U-LAW encoding and return
this as a Python string.  U-LAW is an audio encoding format whereby
you get a dynamic range of about 14 bits using only 8 bit samples.  It
is used by the Sun audio hardware, among others.
\end{funcdesc}

\begin{funcdesc}{minmax}{fragment\, width}
Return a tuple consisting of the minimum and maximum values of all
samples in the sound fragment.
\end{funcdesc}

\begin{funcdesc}{max}{fragment\, width}
Return the maximum of the {\em absolute value} of all samples in a
fragment.
\end{funcdesc}

\begin{funcdesc}{maxpp}{fragment\, width}
Return the maximum peak-peak value in the sound fragment.
\end{funcdesc}

\begin{funcdesc}{mul}{fragment\, width\, factor}
Return a fragment that has all samples in the original framgent
multiplied by the floating-point value \var{factor}.  Overflow is
silently ignored.
\end{funcdesc}

\begin{funcdesc}{reverse}{fragment\, width}
Reverse the samples in a fragment and returns the modified fragment.
\end{funcdesc}

\begin{funcdesc}{rms}{fragment\, width}
Return the root-mean-square of the fragment, i.e.
\iftexi
the square root of the quotient of the sum of all squared sample value,
divided by the sumber of samples.
\else
% in eqn: sqrt { sum S sub i sup 2  over n }
\begin{displaymath}
\catcode`_=8
\sqrt{\frac{\sum{{S_{i}}^{2}}}{n}}
\end{displaymath}
\fi
This is a measure of the power in an audio signal.
\end{funcdesc}

\begin{funcdesc}{tomono}{fragment\, width\, lfactor\, rfactor} 
Convert a stereo fragment to a mono fragment.  The left channel is
multiplied by \var{lfactor} and the right channel by \var{rfactor}
before adding the two channels to give a mono signal.
\end{funcdesc}

\begin{funcdesc}{tostereo}{fragment\, width\, lfactor\, rfactor}
Generate a stereo fragment from a mono fragment.  Each pair of samples
in the stereo fragment are computed from the mono sample, whereby left
channel samples are multiplied by \var{lfactor} and right channel
samples by \var{rfactor}.
\end{funcdesc}

\begin{funcdesc}{ulaw2lin}{fragment\, width}
Convert sound fragments in ULAW encoding to linearly encoded sound
fragments.  ULAW encoding always uses 8 bits samples, so \var{width}
refers only to the sample width of the output fragment here.
\end{funcdesc}

Note that operations such as \code{mul} or \code{max} make no
distinction between mono and stereo fragments, i.e.\ all samples are
treated equal.  If this is a problem the stereo fragment should be split
into two mono fragments first and recombined later.  Here is an example
of how to do that:
\bcode\begin{verbatim}
def mul_stereo(sample, width, lfactor, rfactor):
    lsample = audioop.tomono(sample, width, 1, 0)
    rsample = audioop.tomono(sample, width, 0, 1)
    lsample = audioop.mul(sample, width, lfactor)
    rsample = audioop.mul(sample, width, rfactor)
    lsample = audioop.tostereo(lsample, width, 1, 0)
    rsample = audioop.tostereo(rsample, width, 0, 1)
    return audioop.add(lsample, rsample, width)
\end{verbatim}\ecode

If you use the ADPCM coder to build network packets and you want your
protocol to be stateless (i.e.\ to be able to tolerate packet loss)
you should not only transmit the data but also the state.  Note that
you should send the \var{initial} state (the one you passed to
\code{lin2adpcm}) along to the decoder, not the final state (as returned by
the coder).  If you want to use \code{struct} to store the state in
binary you can code the first element (the predicted value) in 16 bits
and the second (the delta index) in 8.

The ADPCM coders have never been tried against other ADPCM coders,
only against themselves.  It could well be that I misinterpreted the
standards in which case they will not be interoperable with the
respective standards.

The \code{find...} routines might look a bit funny at first sight.
They are primarily meant to do echo cancellation.  A reasonably
fast way to do this is to pick the most energetic piece of the output
sample, locate that in the input sample and subtract the whole output
sample from the input sample:
\bcode\begin{verbatim}
def echocancel(outputdata, inputdata):
    pos = audioop.findmax(outputdata, 800)    # one tenth second
    out_test = outputdata[pos*2:]
    in_test = inputdata[pos*2:]
    ipos, factor = audioop.findfit(in_test, out_test)
    # Optional (for better cancellation):
    # factor = audioop.findfactor(in_test[ipos*2:ipos*2+len(out_test)], 
    #              out_test)
    prefill = '\0'*(pos+ipos)*2
    postfill = '\0'*(len(inputdata)-len(prefill)-len(outputdata))
    outputdata = prefill + audioop.mul(outputdata,2,-factor) + postfill
    return audioop.add(inputdata, outputdata, 2)
\end{verbatim}\ecode

\section{Built-in Module \sectcode{imageop}}
\bimodindex{imageop}

The \code{imageop} module contains some useful operations on images.
It operates on images consisting of 8 or 32 bit pixels
stored in Python strings.  This is the same format as used
by \code{gl.lrectwrite} and the \code{imgfile} module.

The module defines the following variables and functions:

\renewcommand{\indexsubitem}{(in module imageop)}

\begin{excdesc}{error}
This exception is raised on all errors, such as unknown number of bits
per pixel, etc.
\end{excdesc}


\begin{funcdesc}{crop}{image\, psize\, width\, height\, x0\, y0\, x1\, y1}
Return the selected part of \var{image}, which should by
\var{width} by \var{height} in size and consist of pixels of
\var{psize} bytes. \var{x0}, \var{y0}, \var{x1} and \var{y1} are like
the \code{lrectread} parameters, i.e.\ the boundary is included in the
new image.  The new boundaries need not be inside the picture.  Pixels
that fall outside the old image will have their value set to zero.  If
\var{x0} is bigger than \var{x1} the new image is mirrored.  The same
holds for the y coordinates.
\end{funcdesc}

\begin{funcdesc}{scale}{image\, psize\, width\, height\, newwidth\, newheight}
Return \var{image} scaled to size \var{newwidth} by \var{newheight}.
No interpolation is done, scaling is done by simple-minded pixel
duplication or removal.  Therefore, computer-generated images or
dithered images will not look nice after scaling.
\end{funcdesc}

\begin{funcdesc}{tovideo}{image\, psize\, width\, height}
Run a vertical low-pass filter over an image.  It does so by computing
each destination pixel as the average of two vertically-aligned source
pixels.  The main use of this routine is to forestall excessive
flicker if the image is displayed on a video device that uses
interlacing, hence the name.
\end{funcdesc}

\begin{funcdesc}{grey2mono}{image\, width\, height\, threshold}
Convert a 8-bit deep greyscale image to a 1-bit deep image by
tresholding all the pixels.  The resulting image is tightly packed and
is probably only useful as an argument to \code{mono2grey}.
\end{funcdesc}

\begin{funcdesc}{dither2mono}{image\, width\, height}
Convert an 8-bit greyscale image to a 1-bit monochrome image using a
(simple-minded) dithering algorithm.
\end{funcdesc}

\begin{funcdesc}{mono2grey}{image\, width\, height\, p0\, p1}
Convert a 1-bit monochrome image to an 8 bit greyscale or color image.
All pixels that are zero-valued on input get value \var{p0} on output
and all one-value input pixels get value \var{p1} on output.  To
convert a monochrome black-and-white image to greyscale pass the
values \code{0} and \code{255} respectively.
\end{funcdesc}

\begin{funcdesc}{grey2grey4}{image\, width\, height}
Convert an 8-bit greyscale image to a 4-bit greyscale image without
dithering.
\end{funcdesc}

\begin{funcdesc}{grey2grey2}{image\, width\, height}
Convert an 8-bit greyscale image to a 2-bit greyscale image without
dithering.
\end{funcdesc}

\begin{funcdesc}{dither2grey2}{image\, width\, height}
Convert an 8-bit greyscale image to a 2-bit greyscale image with
dithering.  As for \code{dither2mono}, the dithering algorithm is
currently very simple.
\end{funcdesc}

\begin{funcdesc}{grey42grey}{image\, width\, height}
Convert a 4-bit greyscale image to an 8-bit greyscale image.
\end{funcdesc}

\begin{funcdesc}{grey22grey}{image\, width\, height}
Convert a 2-bit greyscale image to an 8-bit greyscale image.
\end{funcdesc}

\section{Standard Module \sectcode{aifc}}
\stmodindex{aifc}

This module provides support for reading and writing AIFF and AIFF-C
files.  AIFF is Audio Interchange File Format, a format for storing
digital audio samples in a file.  AIFF-C is a newer version of the
format that includes the ability to compress the audio data.

Audio files have a number of parameters that describe the audio data.
The sampling rate or frame rate is the number of times per second the
sound is sampled.  The number of channels indicate if the audio is
mono, stereo, or quadro.  Each frame consists of one sample per
channel.  The sample size is the size in bytes of each sample.  Thus a
frame consists of \var{nchannels}*\var{samplesize} bytes, and a
second's worth of audio consists of
\var{nchannels}*\var{samplesize}*\var{framerate} bytes.

For example, CD quality audio has a sample size of two bytes (16
bits), uses two channels (stereo) and has a frame rate of 44,100
frames/second.  This gives a frame size of 4 bytes (2*2), and a
second's worth occupies 2*2*44100 bytes, i.e.\ 176,400 bytes.

Module \code{aifc} defines the following function:

\renewcommand{\indexsubitem}{(in module aifc)}
\begin{funcdesc}{open}{file\, mode}
Open an AIFF or AIFF-C file and return an object instance with
methods that are described below.  The argument file is either a
string naming a file or a file object.  The mode is either the string
\code{'r'} when the file must be opened for reading, or \code{'w'}
when the file must be opened for writing.  When used for writing, the
file object should be seekable, unless you know ahead of time how many
samples you are going to write in total and use
\code{writeframesraw()} and \code{setnframes()}.
\end{funcdesc}

Objects returned by \code{aifc.open()} when a file is opened for
reading have the following methods:

\renewcommand{\indexsubitem}{(aifc object method)}
\begin{funcdesc}{getnchannels}{}
Return the number of audio channels (1 for mono, 2 for stereo).
\end{funcdesc}

\begin{funcdesc}{getsampwidth}{}
Return the size in bytes of individual samples.
\end{funcdesc}

\begin{funcdesc}{getframerate}{}
Return the sampling rate (number of audio frames per second).
\end{funcdesc}

\begin{funcdesc}{getnframes}{}
Return the number of audio frames in the file.
\end{funcdesc}

\begin{funcdesc}{getcomptype}{}
Return a four-character string describing the type of compression used
in the audio file.  For AIFF files, the returned value is
\code{'NONE'}.
\end{funcdesc}

\begin{funcdesc}{getcompname}{}
Return a human-readable description of the type of compression used in
the audio file.  For AIFF files, the returned value is \code{'not
compressed'}.
\end{funcdesc}

\begin{funcdesc}{getparams}{}
Return a tuple consisting of all of the above values in the above
order.
\end{funcdesc}

\begin{funcdesc}{getmarkers}{}
Return a list of markers in the audio file.  A marker consists of a
tuple of three elements.  The first is the mark ID (an integer), the
second is the mark position in frames from the beginning of the data
(an integer), the third is the name of the mark (a string).
\end{funcdesc}

\begin{funcdesc}{getmark}{id}
Return the tuple as described in \code{getmarkers} for the mark with
the given id.
\end{funcdesc}

\begin{funcdesc}{readframes}{nframes}
Read and return the next \var{nframes} frames from the audio file.  The
returned data is a string containing for each frame the uncompressed
samples of all channels.
\end{funcdesc}

\begin{funcdesc}{rewind}{}
Rewind the read pointer.  The next \code{readframes} will start from
the beginning.
\end{funcdesc}

\begin{funcdesc}{setpos}{pos}
Seek to the specified frame number.
\end{funcdesc}

\begin{funcdesc}{tell}{}
Return the current frame number.
\end{funcdesc}

\begin{funcdesc}{close}{}
Close the AIFF file.  After calling this method, the object can no
longer be used.
\end{funcdesc}

Objects returned by \code{aifc.open()} when a file is opened for
writing have all the above methods, except for \code{readframes} and
\code{setpos}.  In addition the following methods exist.  The
\code{get} methods can only be called after the corresponding
\code{set} methods have been called.  Before the first
\code{writeframes} or \code{writeframesraw}, all parameters except for
the number of frames must be filled in.

\begin{funcdesc}{aiff}{}
Create an AIFF file.  The default is that an AIFF-C file is created,
unless the name of the file ends in '.aiff' in which case the default
is an AIFF file.
\end{funcdesc}

\begin{funcdesc}{aifc}{}
Create an AIFF-C file.  The default is that an AIFF-C file is created,
unless the name of the file ends in '.aiff' in which case the default
is an AIFF file.
\end{funcdesc}

\begin{funcdesc}{setnchannels}{nchannels}
Specify the number of channels in the audio file.
\end{funcdesc}

\begin{funcdesc}{setsampwidth}{width}
Specify the size in bytes of audio samples.
\end{funcdesc}

\begin{funcdesc}{setframerate}{rate}
Specify the sampling frequency in frames per second.
\end{funcdesc}

\begin{funcdesc}{setnframes}{nframes}
Specify the number of frames that are to be written to the audio file.
If this parameter is not set, or not set correctly, the file needs to
support seeking.
\end{funcdesc}

\begin{funcdesc}{setcomptype}{type\, name}
Specify the compression type.  If not specified, the audio data will
not be compressed.  In AIFF files, compression is not possible.  The
name parameter should be a human-readable description of the
compression type, the type parameter should be a four-character
string.  Currently the following compression types are supported:
NONE, ULAW, ALAW, G722.
\end{funcdesc}

\begin{funcdesc}{setparams}{nchannels\, sampwidth\, framerate\, comptype\, compname}
Set all the above parameters at once.  The argument is a tuple
consisting of the various parameters.  This means that it is possible
to use the result of a \code{getparams} call as argument to
\code{setparams}.
\end{funcdesc}

\begin{funcdesc}{setmark}{id\, pos\, name}
Add a mark with the given id (larger than 0), and the given name at
the given position.  This method can be called at any time before
\code{close}.
\end{funcdesc}

\begin{funcdesc}{tell}{}
Return the current write position in the output file.  Useful in
combination with \code{setmark}.
\end{funcdesc}

\begin{funcdesc}{writeframes}{data}
Write data to the output file.  This method can only be called after
the audio file parameters have been set.
\end{funcdesc}

\begin{funcdesc}{writeframesraw}{data}
Like \code{writeframes}, except that the header of the audio file is
not updated.
\end{funcdesc}

\begin{funcdesc}{close}{}
Close the AIFF file.  The header of the file is updated to reflect the
actual size of the audio data. After calling this method, the object
can no longer be used.
\end{funcdesc}

\section{Built-in Module \sectcode{jpeg}}
\bimodindex{jpeg}

The module \code{jpeg} provides access to the jpeg compressor and
decompressor written by the Independent JPEG Group. JPEG is a (draft?)\
standard for compressing pictures.  For details on jpeg or the
Independent JPEG Group software refer to the JPEG standard or the
documentation provided with the software.

The \code{jpeg} module defines these functions:

\renewcommand{\indexsubitem}{(in module jpeg)}
\begin{funcdesc}{compress}{data\, w\, h\, b}
Treat data as a pixmap of width \var{w} and height \var{h}, with \var{b} bytes per
pixel.  The data is in SGI GL order, so the first pixel is in the
lower-left corner. This means that \code{lrectread} return data can
immediately be passed to compress.  Currently only 1 byte and 4 byte
pixels are allowed, the former being treated as greyscale and the
latter as RGB color.  Compress returns a string that contains the
compressed picture, in JFIF format.
\end{funcdesc}

\begin{funcdesc}{decompress}{data}
Data is a string containing a picture in JFIF format. It returns a
tuple
\code{(\var{data}, \var{width}, \var{height}, \var{bytesperpixel})}.
Again, the data is suitable to pass to \code{lrectwrite}.
\end{funcdesc}

\begin{funcdesc}{setoption}{name\, value}
Set various options.  Subsequent compress and decompress calls
will use these options.  The following options are available:
\begin{description}
\item[\code{'forcegray' }]
Force output to be grayscale, even if input is RGB.

\item[\code{'quality' }]
Set the quality of the compressed image to a
value between \code{0} and \code{100} (default is \code{75}).  Compress only.

\item[\code{'optimize' }]
Perform Huffman table optimization.  Takes longer, but results in
smaller compressed image.  Compress only.

\item[\code{'smooth' }]
Perform inter-block smoothing on uncompressed image.  Only useful for
low-quality images.  Decompress only.
\end{description}
\end{funcdesc}

Compress and uncompress raise the error \code{jpeg.error} in case of errors.

\section{Built-in Module \sectcode{rgbimg}}
\bimodindex{rgbimg}

The rgbimg module allows python programs to access SGI imglib image
files (also known as \file{.rgb} files).  The module is far from
complete, but is provided anyway since the functionality that there is
is enough in some cases.  Currently, colormap files are not supported.

The module defines the following variables and functions:

\renewcommand{\indexsubitem}{(in module rgbimg)}
\begin{excdesc}{error}
This exception is raised on all errors, such as unsupported file type, etc.
\end{excdesc}

\begin{funcdesc}{sizeofimage}{file}
This function returns a tuple \code{(\var{x}, \var{y})} where
\var{x} and \var{y} are the size of the image in pixels.
Only 4 byte RGBA pixels, 3 byte RGB pixels, and 1 byte greyscale pixels
are currently supported.
\end{funcdesc}

\begin{funcdesc}{longimagedata}{file}
This function reads and decodes the image on the specified file, and
returns it as a Python string. The string has 4 byte RGBA pixels.
The bottom left pixel is the first in
the string. This format is suitable to pass to \code{gl.lrectwrite},
for instance.
\end{funcdesc}

\begin{funcdesc}{longstoimage}{data\, x\, y\, z\, file}
This function writes the RGBA data in \var{data} to image
file \var{file}. \var{x} and \var{y} give the size of the image.
\var{z} is 1 if the saved image should be 1 byte greyscale, 3 if the
saved image should be 3 byte RGB data, or 4 if the saved images should
be 4 byte RGBA data.  The input data always contains 4 bytes per pixel.
These are the formats returned by \code{gl.lrectread}.
\end{funcdesc}

\begin{funcdesc}{ttob}{flag}
This function sets a global flag which defines whether the scan lines
of the image are read or written from bottom to top (flag is zero,
compatible with SGI GL) or from top to bottom(flag is one,
compatible with X)\@.  The default is zero.
\end{funcdesc}

\section{Standard module \sectcode{imghdr}}
\stmodindex{imghdr}

The \code{imghdr} module determines the type of image contained in a
file or byte stream.

The \code{imghdr} module defines the following function:

\renewcommand{\indexsubitem}{(in module imghdr)}

\begin{funcdesc}{what}{filename\optional{\, h}}
Tests the image data contained in the file named by \var{filename},
and returns a string describing the image type.  If optional \var{h}
is provided, the \var{filename} is ignored and \var{h} is assumed to
contain the byte stream to test.
\end{funcdesc}

The following image types are recognized, as listed below with the
return value from \code{what}:

\begin{enumerate}
\item[``rgb''] SGI ImgLib Files

\item[``gif''] GIF 87a and 89a Files

\item[``pbm''] Portable Bitmap Files

\item[``pgm''] Portable Graymap Files

\item[``ppm''] Portable Pixmap Files

\item[``tiff''] TIFF Files

\item[``rast''] Sun Raster Files

\item[``xbm''] X Bitmap Files

\item[``jpeg''] JPEG data in JIFF format
\end{enumerate}

You can extend the list of file types \code{imghdr} can recognize by
appending to this variable:

\begin{datadesc}{tests}
A list of functions performing the individual tests.  Each function
takes two arguments: the byte-stream and an open file-like object.
When \code{what()} is called with a byte-stream, the file-like
object will be \code{None}.

The test function should return a string describing the image type if
the test succeeded, or \code{None} if it failed.
\end{datadesc}

Example:

\begin{verbatim}
>>> import imghdr
>>> imghdr.what('/tmp/bass.gif')
'gif'
\end{verbatim}


\chapter{Cryptographic Services}
\index{cryptography}

The modules described in this chapter implement various algorithms of
a cryptographic nature.  They are available at the discretion of the
installation.  Here's an overview:

\begin{description}

\item[md5]
--- RSA's MD5 message digest algorithm.

\item[mpz]
--- Interface to the GNU MP library for arbitrary precision arithmetic.

\item[rotor]
--- Enigma-like encryption and decryption.

\end{description}

Hardcore cypherpunks will probably find the cryptographic modules
written by Andrew Kuchling of further interest; the package adds
built-in modules for DES and IDEA encryption, provides a Python module
for reading and decrypting PGP files, and then some.  These modules
are not distributed with Python but available separately.  See the URL
\file{http://www.magnet.com/~amk/python/pct.html} or send email to
\file{amk@magnet.com} for more information.
\index{PGP}
\indexii{DES}{cipher}
\indexii{IDEA}{cipher}
\index{cryptography}
		% Cryptographic Services
\section{Built-in Module \sectcode{md5}}
\bimodindex{md5}

This module implements the interface to RSA's MD5 message digest
algorithm (see also Internet RFC 1321).  Its use is quite
straightforward:\ use the \code{md5.new()} to create an md5 object.
You can now feed this object with arbitrary strings using the
\code{update()} method, and at any point you can ask it for the
\dfn{digest} (a strong kind of 128-bit checksum,
a.k.a. ``fingerprint'') of the contatenation of the strings fed to it
so far using the \code{digest()} method.

For example, to obtain the digest of the string {\tt"Nobody inspects
the spammish repetition"}:

\bcode\begin{verbatim}
>>> import md5
>>> m = md5.new()
>>> m.update("Nobody inspects")
>>> m.update(" the spammish repetition")
>>> m.digest()
'\273d\234\203\335\036\245\311\331\336\311\241\215\360\377\351'
\end{verbatim}\ecode

More condensed:

\bcode\begin{verbatim}
>>> md5.new("Nobody inspects the spammish repetition").digest()
'\273d\234\203\335\036\245\311\331\336\311\241\215\360\377\351'
\end{verbatim}\ecode

\renewcommand{\indexsubitem}{(in module md5)}

\begin{funcdesc}{new}{\optional{arg}}
Return a new md5 object.  If \var{arg} is present, the method call
\code{update(\var{arg})} is made.
\end{funcdesc}

\begin{funcdesc}{md5}{\optional{arg}}
For backward compatibility reasons, this is an alternative name for the
\code{new()} function.
\end{funcdesc}

An md5 object has the following methods:

\renewcommand{\indexsubitem}{(md5 method)}
\begin{funcdesc}{update}{arg}
Update the md5 object with the string \var{arg}.  Repeated calls are
equivalent to a single call with the concatenation of all the
arguments, i.e.\ \code{m.update(a); m.update(b)} is equivalent to
\code{m.update(a+b)}.
\end{funcdesc}

\begin{funcdesc}{digest}{}
Return the digest of the strings passed to the \code{update()}
method so far.  This is an 16-byte string which may contain
non-\ASCII{} characters, including null bytes.
\end{funcdesc}

\begin{funcdesc}{copy}{}
Return a copy (``clone'') of the md5 object.  This can be used to
efficiently compute the digests of strings that share a common initial
substring.
\end{funcdesc}

\section{Built-in Module \sectcode{mpz}}
\bimodindex{mpz}

This is an optional module.  It is only available when Python is
configured to include it, which requires that the GNU MP software is
installed.

This module implements the interface to part of the GNU MP library,
which defines arbitrary precision integer and rational number
arithmetic routines.  Only the interfaces to the \emph{integer}
(\samp{mpz_{\rm \ldots}}) routines are provided. If not stated
otherwise, the description in the GNU MP documentation can be applied.

In general, \dfn{mpz}-numbers can be used just like other standard
Python numbers, e.g.\ you can use the built-in operators like \code{+},
\code{*}, etc., as well as the standard built-in functions like
\code{abs}, \code{int}, \ldots, \code{divmod}, \code{pow}.
\strong{Please note:} the {\it bitwise-xor} operation has been implemented as
a bunch of {\it and}s, {\it invert}s and {\it or}s, because the library
lacks an \code{mpz_xor} function, and I didn't need one.

You create an mpz-number by calling the function called \code{mpz} (see
below for an exact description). An mpz-number is printed like this:
\code{mpz(\var{value})}.

\renewcommand{\indexsubitem}{(in module mpz)}
\begin{funcdesc}{mpz}{value}
  Create a new mpz-number. \var{value} can be an integer, a long,
  another mpz-number, or even a string. If it is a string, it is
  interpreted as an array of radix-256 digits, least significant digit
  first, resulting in a positive number. See also the \code{binary}
  method, described below.
\end{funcdesc}

A number of {\em extra} functions are defined in this module. Non
mpz-arguments are converted to mpz-values first, and the functions
return mpz-numbers.

\begin{funcdesc}{powm}{base\, exponent\, modulus}
  Return \code{pow(\var{base}, \var{exponent}) \%{} \var{modulus}}. If
  \code{\var{exponent} == 0}, return \code{mpz(1)}. In contrast to the
  \C-library function, this version can handle negative exponents.
\end{funcdesc}

\begin{funcdesc}{gcd}{op1\, op2}
  Return the greatest common divisor of \var{op1} and \var{op2}.
\end{funcdesc}

\begin{funcdesc}{gcdext}{a\, b}
  Return a tuple \code{(\var{g}, \var{s}, \var{t})}, such that
  \code{\var{a}*\var{s} + \var{b}*\var{t} == \var{g} == gcd(\var{a}, \var{b})}.
\end{funcdesc}

\begin{funcdesc}{sqrt}{op}
  Return the square root of \var{op}. The result is rounded towards zero.
\end{funcdesc}

\begin{funcdesc}{sqrtrem}{op}
  Return a tuple \code{(\var{root}, \var{remainder})}, such that
  \code{\var{root}*\var{root} + \var{remainder} == \var{op}}.
\end{funcdesc}

\begin{funcdesc}{divm}{numerator\, denominator\, modulus}
  Returns a number \var{q}. such that
  \code{\var{q} * \var{denominator} \%{} \var{modulus} == \var{numerator}}.
  One could also implement this function in Python, using \code{gcdext}.
\end{funcdesc}

An mpz-number has one method:

\renewcommand{\indexsubitem}{(mpz method)}
\begin{funcdesc}{binary}{}
  Convert this mpz-number to a binary string, where the number has been
  stored as an array of radix-256 digits, least significant digit first.

  The mpz-number must have a value greater than or equal to zero,
  otherwise a \code{ValueError}-exception will be raised.
\end{funcdesc}

\section{Built-in Module \sectcode{rotor}}
\bimodindex{rotor}

This module implements a rotor-based encryption algorithm, contributed by
Lance Ellinghouse.  The design is derived from the Enigma device, a machine
used during World War II to encipher messages.  A rotor is simply a
permutation.  For example, if the character `A' is the origin of the rotor,
then a given rotor might map `A' to `L', `B' to `Z', `C' to `G', and so on.
To encrypt, we choose several different rotors, and set the origins of the
rotors to known positions; their initial position is the ciphering key.  To
encipher a character, we permute the original character by the first rotor,
and then apply the second rotor's permutation to the result. We continue
until we've applied all the rotors; the resulting character is our
ciphertext.  We then change the origin of the final rotor by one position,
from `A' to `B'; if the final rotor has made a complete revolution, then we
rotate the next-to-last rotor by one position, and apply the same procedure
recursively.  In other words, after enciphering one character, we advance
the rotors in the same fashion as a car's odometer. Decoding works in the
same way, except we reverse the permutations and apply them in the opposite
order.
\index{Ellinghouse, Lance}
\indexii{Enigma}{cipher}

The available functions in this module are:

\renewcommand{\indexsubitem}{(in module rotor)}
\begin{funcdesc}{newrotor}{key\optional{\, numrotors}}
Return a rotor object. \var{key} is a string containing the encryption key
for the object; it can contain arbitrary binary data. The key will be used
to randomly generate the rotor permutations and their initial positions.
\var{numrotors} is the number of rotor permutations in the returned object;
if it is omitted, a default value of 6 will be used.
\end{funcdesc}

Rotor objects have the following methods:

\renewcommand{\indexsubitem}{(rotor method)}
\begin{funcdesc}{setkey}{}
Reset the rotor to its initial state.
\end{funcdesc}

\begin{funcdesc}{encrypt}{plaintext}
Reset the rotor object to its initial state and encrypt \var{plaintext},
returning a string containing the ciphertext.  The ciphertext is always the
same length as the original plaintext.
\end{funcdesc}

\begin{funcdesc}{encryptmore}{plaintext}
Encrypt \var{plaintext} without resetting the rotor object, and return a
string containing the ciphertext.
\end{funcdesc}

\begin{funcdesc}{decrypt}{ciphertext}
Reset the rotor object to its initial state and decrypt \var{ciphertext},
returning a string containing the ciphertext.  The plaintext string will
always be the same length as the ciphertext.
\end{funcdesc}

\begin{funcdesc}{decryptmore}{ciphertext}
Decrypt \var{ciphertext} without resetting the rotor object, and return a
string containing the ciphertext.
\end{funcdesc}

An example usage:
\bcode\begin{verbatim}
>>> import rotor
>>> rt = rotor.newrotor('key', 12)
>>> rt.encrypt('bar')
'\2534\363'
>>> rt.encryptmore('bar')
'\357\375$'
>>> rt.encrypt('bar')
'\2534\363'
>>> rt.decrypt('\2534\363')
'bar'
>>> rt.decryptmore('\357\375$')
'bar'
>>> rt.decrypt('\357\375$')
'l(\315'
>>> del rt
\end{verbatim}\ecode

The module's code is not an exact simulation of the original Enigma device;
it implements the rotor encryption scheme differently from the original. The
most important difference is that in the original Enigma, there were only 5
or 6 different rotors in existence, and they were applied twice to each
character; the cipher key was the order in which they were placed in the
machine.  The Python rotor module uses the supplied key to initialize a
random number generator; the rotor permutations and their initial positions
are then randomly generated.  The original device only enciphered the
letters of the alphabet, while this module can handle any 8-bit binary data;
it also produces binary output.  This module can also operate with an
arbitrary number of rotors.

The original Enigma cipher was broken in 1944. % XXX: Is this right?
The version implemented here is probably a good deal more difficult to crack
(especially if you use many rotors), but it won't be impossible for
a truly skilful and determined attacker to break the cipher.  So if you want
to keep the NSA out of your files, this rotor cipher may well be unsafe, but
for discouraging casual snooping through your files, it will probably be
just fine, and may be somewhat safer than using the Unix \file{crypt}
command.
\index{National Security Agency}\index{crypt(1)}
% XXX How were Unix commands represented in the docs?



%\chapter{Amoeba Specific Services}

\section{Built-in Module \sectcode{amoeba}}

\bimodindex{amoeba}
This module provides some object types and operations useful for
Amoeba applications.  It is only available on systems that support
Amoeba operations.  RPC errors and other Amoeba errors are reported as
the exception \code{amoeba.error = 'amoeba.error'}.

The module \code{amoeba} defines the following items:

\renewcommand{\indexsubitem}{(in module amoeba)}
\begin{funcdesc}{name_append}{path\, cap}
Stores a capability in the Amoeba directory tree.
Arguments are the pathname (a string) and the capability (a capability
object as returned by
\code{name_lookup()}).
\end{funcdesc}

\begin{funcdesc}{name_delete}{path}
Deletes a capability from the Amoeba directory tree.
Argument is the pathname.
\end{funcdesc}

\begin{funcdesc}{name_lookup}{path}
Looks up a capability.
Argument is the pathname.
Returns a
\dfn{capability}
object, to which various interesting operations apply, described below.
\end{funcdesc}

\begin{funcdesc}{name_replace}{path\, cap}
Replaces a capability in the Amoeba directory tree.
Arguments are the pathname and the new capability.
(This differs from
\code{name_append()}
in the behavior when the pathname already exists:
\code{name_append()}
finds this an error while
\code{name_replace()}
allows it, as its name suggests.)
\end{funcdesc}

\begin{datadesc}{capv}
A table representing the capability environment at the time the
interpreter was started.
(Alas, modifying this table does not affect the capability environment
of the interpreter.)
For example,
\code{amoeba.capv['ROOT']}
is the capability of your root directory, similar to
\code{getcap("ROOT")}
in C.
\end{datadesc}

\begin{excdesc}{error}
The exception raised when an Amoeba function returns an error.
The value accompanying this exception is a pair containing the numeric
error code and the corresponding string, as returned by the C function
\code{err_why()}.
\end{excdesc}

\begin{funcdesc}{timeout}{msecs}
Sets the transaction timeout, in milliseconds.
Returns the previous timeout.
Initially, the timeout is set to 2 seconds by the Python interpreter.
\end{funcdesc}

\subsection{Capability Operations}

Capabilities are written in a convenient \ASCII{} format, also used by the
Amoeba utilities
{\it c2a}(U)
and
{\it a2c}(U).
For example:

\bcode\begin{verbatim}
>>> amoeba.name_lookup('/profile/cap')
aa:1c:95:52:6a:fa/14(ff)/8e:ba:5b:8:11:1a
>>> 
\end{verbatim}\ecode

The following methods are defined for capability objects.

\renewcommand{\indexsubitem}{(capability method)}
\begin{funcdesc}{dir_list}{}
Returns a list of the names of the entries in an Amoeba directory.
\end{funcdesc}

\begin{funcdesc}{b_read}{offset\, maxsize}
Reads (at most)
\var{maxsize}
bytes from a bullet file at offset
\var{offset.}
The data is returned as a string.
EOF is reported as an empty string.
\end{funcdesc}

\begin{funcdesc}{b_size}{}
Returns the size of a bullet file.
\end{funcdesc}

\begin{funcdesc}{dir_append}{}
\funcline{dir_delete}{}\ 
\funcline{dir_lookup}{}\ 
\funcline{dir_replace}{}
Like the corresponding
\samp{name_}*
functions, but with a path relative to the capability.
(For paths beginning with a slash the capability is ignored, since this
is the defined semantics for Amoeba.)
\end{funcdesc}

\begin{funcdesc}{std_info}{}
Returns the standard info string of the object.
\end{funcdesc}

\begin{funcdesc}{tod_gettime}{}
Returns the time (in seconds since the Epoch, in UCT, as for POSIX) from
a time server.
\end{funcdesc}

\begin{funcdesc}{tod_settime}{t}
Sets the time kept by a time server.
\end{funcdesc}
		% AMOEBA ONLY

\chapter{Macintosh Specific Services}

The modules in this chapter are available on the Apple Macintosh only.

Aside from the modules described here there are also interfaces to
various MacOS toolboxes, which are currently not extensively
described. The toolboxes for which modules exist are:
\code{AE} (Apple Events),
\code{Cm} (Component Manager),
\code{Ctl} (Control Manager),
\code{Dlg} (Dialog Manager),
\code{Evt} (Event Manager),
\code{Fm} (Font Manager),
\code{List} (List Manager),
\code{Menu} (Moenu Manager),
\code{Qd} (QuickDraw),
\code{Qt} (QuickTime),
\code{Res} (Resource Manager and Handles),
\code{Scrap} (Scrap Manager),
\code{Snd} (Sound Manager),
\code{TE} (TextEdit),
\code{Waste} (non-Apple TextEdit replacement) and
\code{Win} (Window Manager).

If applicable the module will define a number of Python objects for
the various structures declared by the toolbox, and operations will be
implemented as methods of the object. Other operations will be
implemented as functions in the module. Not all operations possible in
C will also be possible in Python (callbacks are often a problem), and
parameters will occasionally be different in Python (input and output
buffers, especially). All methods and functions have a \code{__doc__}
string describing their arguments and return values, and for
additional description you are referred to Inside Mac or similar
works.

\section{Built-in Module \sectcode{mac}}

\bimodindex{mac}
This module provides a subset of the operating system dependent
functionality provided by the optional built-in module \code{posix}.
It is best accessed through the more portable standard module
\code{os}.

The following functions are available in this module:
\code{chdir},
\code{close},
\code{dup},
\code{fdopen},
\code{getcwd},
\code{lseek},
\code{listdir},
\code{mkdir},
\code{open},
\code{read},
\code{rename},
\code{rmdir},
\code{stat},
\code{sync},
\code{unlink},
\code{write},
as well as the exception \code{error}. Note that the times returned by
\code{stat} are floating-point values, like all time values in
MacPython.

One additional function is available: \code{xstat}. This function
returns the same information as \code{stat}, but with three extra
values appended: the size of the resource fork of the file and its
4-char creator and type.

\section{Standard Module \sectcode{macpath}}

\stmodindex{macpath}
This module provides a subset of the pathname manipulation functions
available from the optional standard module \code{posixpath}.  It is
best accessed through the more portable standard module \code{os}, as
\code{os.path}.

The following functions are available in this module:
\code{normcase},
\code{normpath},
\code{isabs},
\code{join},
\code{split},
\code{isdir},
\code{isfile},
\code{walk},
\code{exists}.
For other functions available in \code{posixpath} dummy counterparts
are available.
			% MACINTOSH ONLY
\section{Built-in Module \sectcode{ctb}}
\bimodindex{ctb}
\renewcommand{\indexsubitem}{(in module ctb)}

This module provides a partial interface to the Macintosh
Communications Toolbox. Currently, only Connection Manager tools are
supported.  It may not be available in all Mac Python versions.

\begin{datadesc}{error}
The exception raised on errors.
\end{datadesc}

\begin{datadesc}{cmData}
\dataline{cmCntl}
\dataline{cmAttn}
Flags for the \var{channel} argument of the \var{Read} and \var{Write}
methods.
\end{datadesc}

\begin{datadesc}{cmFlagsEOM}
End-of-message flag for \var{Read} and \var{Write}.
\end{datadesc}

\begin{datadesc}{choose*}
Values returned by \var{Choose}.
\end{datadesc}

\begin{datadesc}{cmStatus*}
Bits in the status as returned by \var{Status}.
\end{datadesc}

\begin{funcdesc}{available}{}
Return 1 if the communication toolbox is available, zero otherwise.
\end{funcdesc}

\begin{funcdesc}{CMNew}{name\, sizes}
Create a connection object using the connection tool named
\var{name}. \var{sizes} is a 6-tuple given buffer sizes for data in,
data out, control in, control out, attention in and attention out.
Alternatively, passing \code{None} will result in default buffer sizes.
\end{funcdesc}

\subsection{connection object}
For all connection methods that take a \var{timeout} argument, a value
of \code{-1} is indefinite, meaning that the command runs to completion.

\renewcommand{\indexsubitem}{(connection object attribute)}

\begin{datadesc}{callback}
If this member is set to a value other than \code{None} it should point
to a function accepting a single argument (the connection
object). This will make all connection object methods work
asynchronously, with the callback routine being called upon
completion.

{\em Note:} for reasons beyond my understanding the callback routine
is currently never called. You are advised against using asynchronous
calls for the time being.
\end{datadesc}


\renewcommand{\indexsubitem}{(connection object method)}

\begin{funcdesc}{Open}{timeout}
Open an outgoing connection, waiting at most \var{timeout} seconds for
the connection to be established.
\end{funcdesc}

\begin{funcdesc}{Listen}{timeout}
Wait for an incoming connection. Stop waiting after \var{timeout}
seconds. This call is only meaningful to some tools.
\end{funcdesc}

\begin{funcdesc}{accept}{yesno}
Accept (when \var{yesno} is non-zero) or reject an incoming call after
\var{Listen} returned.
\end{funcdesc}

\begin{funcdesc}{Close}{timeout\, now}
Close a connection. When \var{now} is zero, the close is orderly
(i.e.\ outstanding output is flushed, etc.)\ with a timeout of
\var{timeout} seconds. When \var{now} is non-zero the close is
immediate, discarding output.
\end{funcdesc}

\begin{funcdesc}{Read}{len\, chan\, timeout}
Read \var{len} bytes, or until \var{timeout} seconds have passed, from
the channel \var{chan} (which is one of \var{cmData}, \var{cmCntl} or
\var{cmAttn}). Return a 2-tuple:\ the data read and the end-of-message
flag.
\end{funcdesc}

\begin{funcdesc}{Write}{buf\, chan\, timeout\, eom}
Write \var{buf} to channel \var{chan}, aborting after \var{timeout}
seconds. When \var{eom} has the value \var{cmFlagsEOM} an
end-of-message indicator will be written after the data (if this
concept has a meaning for this communication tool). The method returns
the number of bytes written.
\end{funcdesc}

\begin{funcdesc}{Status}{}
Return connection status as the 2-tuple \code{(\var{sizes},
\var{flags})}. \var{sizes} is a 6-tuple giving the actual buffer sizes used
(see \var{CMNew}), \var{flags} is a set of bits describing the state
of the connection.
\end{funcdesc}

\begin{funcdesc}{GetConfig}{}
Return the configuration string of the communication tool. These
configuration strings are tool-dependent, but usually easily parsed
and modified.
\end{funcdesc}

\begin{funcdesc}{SetConfig}{str}
Set the configuration string for the tool. The strings are parsed
left-to-right, with later values taking precedence. This means
individual configuration parameters can be modified by simply appending
something like \code{'baud 4800'} to the end of the string returned by
\var{GetConfig} and passing that to this method. The method returns
the number of characters actually parsed by the tool before it
encountered an error (or completed successfully).
\end{funcdesc}

\begin{funcdesc}{Choose}{}
Present the user with a dialog to choose a communication tool and
configure it. If there is an outstanding connection some choices (like
selecting a different tool) may cause the connection to be
aborted. The return value (one of the \var{choose*} constants) will
indicate this.
\end{funcdesc}

\begin{funcdesc}{Idle}{}
Give the tool a chance to use the processor. You should call this
method regularly.
\end{funcdesc}

\begin{funcdesc}{Abort}{}
Abort an outstanding asynchronous \var{Open} or \var{Listen}.
\end{funcdesc}

\begin{funcdesc}{Reset}{}
Reset a connection. Exact meaning depends on the tool.
\end{funcdesc}

\begin{funcdesc}{Break}{length}
Send a break. Whether this means anything, what it means and
interpretation of the \var{length} parameter depend on the tool in
use.
\end{funcdesc}

\section{Built-in Module \sectcode{macconsole}}
\bimodindex{macconsole}

\renewcommand{\indexsubitem}{(in module macconsole)}

This module is available on the Macintosh, provided Python has been
built using the Think C compiler. It provides an interface to the
Think console package, with which basic text windows can be created.

\begin{datadesc}{options}
An object allowing you to set various options when creating windows,
see below.
\end{datadesc}

\begin{datadesc}{C_ECHO}
\dataline{C_NOECHO}
\dataline{C_CBREAK}
\dataline{C_RAW}
Options for the \code{setmode} method. \var{C_ECHO} and \var{C_CBREAK}
enable character echo, the other two disable it, \var{C_ECHO} and
\var{C_NOECHO} enable line-oriented input (erase/kill processing,
etc).
\end{datadesc}

\begin{funcdesc}{copen}{}
Open a new console window. Return a console window object.
\end{funcdesc}

\begin{funcdesc}{fopen}{fp}
Return the console window object corresponding with the given file
object. \var{fp} should be one of \code{sys.stdin}, \code{sys.stdout} or
\code{sys.stderr}.
\end{funcdesc}

\subsection{macconsole options object}
These options are examined when a window is created:

\renewcommand{\indexsubitem}{(macconsole option)}
\begin{datadesc}{top}
\dataline{left}
The origin of the window.
\end{datadesc}

\begin{datadesc}{nrows}
\dataline{ncols}
The size of the window.
\end{datadesc}

\begin{datadesc}{txFont}
\dataline{txSize}
\dataline{txStyle}
The font, fontsize and fontstyle to be used in the window.
\end{datadesc}

\begin{datadesc}{title}
The title of the window.
\end{datadesc}

\begin{datadesc}{pause_atexit}
If set non-zero, the window will wait for user action before closing.
\end{datadesc}

\subsection{console window object}

\renewcommand{\indexsubitem}{(console window attribute)}

\begin{datadesc}{file}
The file object corresponding to this console window. If the file is
buffered, you should call \code{file.flush()} between \code{write()}
and \code{read()} calls.
\end{datadesc}

\renewcommand{\indexsubitem}{(console window method)}

\begin{funcdesc}{setmode}{mode}
Set the input mode of the console to \var{C_ECHO}, etc.
\end{funcdesc}

\begin{funcdesc}{settabs}{n}
Set the tabsize to \var{n} spaces.
\end{funcdesc}

\begin{funcdesc}{cleos}{}
Clear to end-of-screen.
\end{funcdesc}

\begin{funcdesc}{cleol}{}
Clear to end-of-line.
\end{funcdesc}

\begin{funcdesc}{inverse}{onoff}
Enable inverse-video mode:\ characters with the high bit set are
displayed in inverse video (this disables the upper half of a
non-\ASCII{} character set).
\end{funcdesc}

\begin{funcdesc}{gotoxy}{x\, y}
Set the cursor to position \code{(\var{x}, \var{y})}.
\end{funcdesc}

\begin{funcdesc}{hide}{}
Hide the window, remembering the contents.
\end{funcdesc}

\begin{funcdesc}{show}{}
Show the window again.
\end{funcdesc}

\begin{funcdesc}{echo2printer}{}
Copy everything written to the window to the printer as well.
\end{funcdesc}


\section{Built-in Module \sectcode{macdnr}}
\bimodindex{macdnr}

This module provides an interface to the Macintosh Domain Name
Resolver.  It is usually used in conjunction with the \var{mactcp}
module, to map hostnames to IP-addresses.  It may not be available in
all Mac Python versions.

The \code{macdnr} module defines the following functions:

\renewcommand{\indexsubitem}{(in module macdnr)}

\begin{funcdesc}{Open}{\optional{filename}}
Open the domain name resolver extension.  If \var{filename} is given it
should be the pathname of the extension, otherwise a default is
used.  Normally, this call is not needed since the other calls will
open the extension automatically.
\end{funcdesc}

\begin{funcdesc}{Close}{}
Close the resolver extension.  Again, not needed for normal use.
\end{funcdesc}

\begin{funcdesc}{StrToAddr}{hostname}
Look up the IP address for \var{hostname}.  This call returns a dnr
result object of the ``address'' variation.
\end{funcdesc}

\begin{funcdesc}{AddrToName}{addr}
Do a reverse lookup on the 32-bit integer IP-address
\var{addr}.  Returns a dnr result object of the ``address'' variation.
\end{funcdesc}

\begin{funcdesc}{AddrToStr}{addr}
Convert the 32-bit integer IP-address \var{addr} to a dotted-decimal
string.  Returns the string.
\end{funcdesc}

\begin{funcdesc}{HInfo}{hostname}
Query the nameservers for a \code{HInfo} record for host
\var{hostname}.  These records contain hardware and software
information about the machine in question (if they are available in
the first place).  Returns a dnr result object of the ``hinfo''
variety.
\end{funcdesc}

\begin{funcdesc}{MXInfo}{domain}
Query the nameservers for a mail exchanger for \var{domain}.  This is
the hostname of a host willing to accept SMTP mail for the given
domain.  Returns a dnr result object of the ``mx'' variety.
\end{funcdesc}

\subsection{dnr result object}

Since the DNR calls all execute asynchronously you do not get the
results back immediately.  Instead, you get a dnr result object.  You
can check this object to see whether the query is complete, and access
its attributes to obtain the information when it is.

Alternatively, you can also reference the result attributes directly,
this will result in an implicit wait for the query to complete.

The \var{rtnCode} and \var{cname} attributes are always available, the
others depend on the type of query (address, hinfo or mx).

\renewcommand{\indexsubitem}{(dnr result object method)}

% Add args, as in {arg1\, arg2 \optional{\, arg3}}
\begin{funcdesc}{wait}{}
Wait for the query to complete.
\end{funcdesc}

% Add args, as in {arg1\, arg2 \optional{\, arg3}}
\begin{funcdesc}{isdone}{}
Return 1 if the query is complete.
\end{funcdesc}

\renewcommand{\indexsubitem}{(dnr result object attribute)}

\begin{datadesc}{rtnCode}
The error code returned by the query.
\end{datadesc}

\begin{datadesc}{cname}
The canonical name of the host that was queried.
\end{datadesc}

\begin{datadesc}{ip0}
\dataline{ip1}
\dataline{ip2}
\dataline{ip3}
At most four integer IP addresses for this host.  Unused entries are
zero.  Valid only for address queries.
\end{datadesc}

\begin{datadesc}{cpuType}
\dataline{osType}
Textual strings giving the machine type an OS name.  Valid for hinfo
queries.
\end{datadesc}

\begin{datadesc}{exchange}
The name of a mail-exchanger host.  Valid for mx queries.
\end{datadesc}

\begin{datadesc}{preference}
The preference of this mx record.  Not too useful, since the Macintosh
will only return a single mx record.  Mx queries only.
\end{datadesc}

The simplest way to use the module to convert names to dotted-decimal
strings, without worrying about idle time, etc:
\begin{verbatim}
>>> def gethostname(name):
...     import macdnr
...     dnrr = macdnr.StrToAddr(name)
...     return macdnr.AddrToStr(dnrr.ip0)
\end{verbatim}

\section{Built-in Module \sectcode{macfs}}
\bimodindex{macfs}

\renewcommand{\indexsubitem}{(in module macfs)}

This module provides access to macintosh FSSpec handling, the Alias
Manager, finder aliases and the Standard File package.

Whenever a function or method expects a \var{file} argument, this
argument can be one of three things:\ (1) a full or partial Macintosh
pathname, (2) an FSSpec object or (3) a 3-tuple \code{(wdRefNum,
parID, name)} as described in Inside Mac VI\@. A description of aliases
and the standard file package can also be found there.

\begin{funcdesc}{FSSpec}{file}
Create an FSSpec object for the specified file.
\end{funcdesc}

\begin{funcdesc}{RawFSSpec}{data}
Create an FSSpec object given the raw data for the C structure for the
FSSpec as a string.  This is mainly useful if you have obtained an
FSSpec structure over a network.
\end{funcdesc}

\begin{funcdesc}{RawAlias}{data}
Create an Alias object given the raw data for the C structure for the
alias as a string.  This is mainly useful if you have obtained an
FSSpec structure over a network.
\end{funcdesc}

\begin{funcdesc}{FInfo}{}
Create a zero-filled FInfo object.
\end{funcdesc}

\begin{funcdesc}{ResolveAliasFile}{file}
Resolve an alias file. Returns a 3-tuple \code{(\var{fsspec}, \var{isfolder},
\var{aliased})} where \var{fsspec} is the resulting FSSpec object,
\var{isfolder} is true if \var{fsspec} points to a folder and
\var{aliased} is true if the file was an alias in the first place
(otherwise the FSSpec object for the file itself is returned).
\end{funcdesc}

\begin{funcdesc}{StandardGetFile}{\optional{type\, ...}}
Present the user with a standard ``open input file''
dialog. Optionally, you can pass up to four 4-char file types to limit
the files the user can choose from. The function returns an FSSpec
object and a flag indicating that the user completed the dialog
without cancelling.
\end{funcdesc}

\begin{funcdesc}{PromptGetFile}{prompt\optional{\, type\, ...}}
Similar to \var{StandardGetFile} but allows you to specify a prompt.
\end{funcdesc}

\begin{funcdesc}{StandardPutFile}{prompt\, \optional{default}}
Present the user with a standard ``open output file''
dialog. \var{prompt} is the prompt string, and the optional
\var{default} argument initializes the output file name. The function
returns an FSSpec object and a flag indicating that the user completed
the dialog without cancelling.
\end{funcdesc}

\begin{funcdesc}{GetDirectory}{\optional{prompt}}
Present the user with a non-standard ``select a directory''
dialog. \var{prompt} is the prompt string, and the optional.
Return an FSSpec object and a success-indicator.
\end{funcdesc}

\begin{funcdesc}{SetFolder}{\optional{fsspec}}
Set the folder that is initially presented to the user when one of
the file selection dialogs is presented. \var{Fsspec} should point to
a file in the folder, not the folder itself (the file need not exist,
though). If no argument is passed the folder will be set to the
current directory, i.e. what \code{os.getcwd()} returns.

Note that starting with system 7.5 the user can change Standard File
behaviour with the ``general controls'' controlpanel, thereby making
this call inoperative.
\end{funcdesc}

\begin{funcdesc}{FindFolder}{where\, which\, create}
Locates one of the ``special'' folders that MacOS knows about, such as
the trash or the Preferences folder. \var{Where} is the disk to
search, \var{which} is the 4-char string specifying which folder to
locate. Setting \var{create} causes the folder to be created if it
does not exist. Returns a \code{(vrefnum, dirid)} tuple.

The constants for \var{where} and \var{which} can be obtained from the
standard module \var{MACFS}.
\end{funcdesc}

\begin{funcdesc}{FindApplication}{creator}
Locate the application with 4-char creator code \var{creator}. The
function returns an FSSpec object pointing to the application.
\end{funcdesc}

\subsection{FSSpec objects}

\renewcommand{\indexsubitem}{(FSSpec object attribute)}
\begin{datadesc}{data}
The raw data from the FSSpec object, suitable for passing
to other applications, for instance.
\end{datadesc}

\renewcommand{\indexsubitem}{(FSSpec object method)}
\begin{funcdesc}{as_pathname}{}
Return the full pathname of the file described by the FSSpec object.
\end{funcdesc}

\begin{funcdesc}{as_tuple}{}
Return the \code{(\var{wdRefNum}, \var{parID}, \var{name})} tuple of the file described
by the FSSpec object.
\end{funcdesc}

\begin{funcdesc}{NewAlias}{\optional{file}}
Create an Alias object pointing to the file described by this
FSSpec. If the optional \var{file} parameter is present the alias
will be relative to that file, otherwise it will be absolute.
\end{funcdesc}

\begin{funcdesc}{NewAliasMinimal}{}
Create a minimal alias pointing to this file.
\end{funcdesc}

\begin{funcdesc}{GetCreatorType}{}
Return the 4-char creator and type of the file.
\end{funcdesc}

\begin{funcdesc}{SetCreatorType}{creator\, type}
Set the 4-char creator and type of the file.
\end{funcdesc}

\begin{funcdesc}{GetFInfo}{}
Return a FInfo object describing the finder info for the file.
\end{funcdesc}

\begin{funcdesc}{SetFInfo}{finfo}
Set the finder info for the file to the values specified in the
\var{finfo} object.
\end{funcdesc}

\begin{funcdesc}{GetDates}{}
Return a tuple with three floating point values representing the
creation date, modification date and backup date of the file.
\end{funcdesc}

\begin{funcdesc}{SetDates}{crdate\, moddate\, backupdate}
Set the creation, modification and backup date of the file. The values
are in the standard floating point format used for times throughout
Python.
\end{funcdesc}

\subsection{alias objects}

\renewcommand{\indexsubitem}{(alias object attribute)}
\begin{datadesc}{data}
The raw data for the Alias record, suitable for storing in a resource
or transmitting to other programs.
\end{datadesc}

\renewcommand{\indexsubitem}{(alias object method)}
\begin{funcdesc}{Resolve}{\optional{file}}
Resolve the alias. If the alias was created as a relative alias you
should pass the file relative to which it is. Return the FSSpec for
the file pointed to and a flag indicating whether the alias object
itself was modified during the search process. 
\end{funcdesc}

\begin{funcdesc}{GetInfo}{num}
An interface to the C routine \code{GetAliasInfo()}.
\end{funcdesc}

\begin{funcdesc}{Update}{file\, \optional{file2}}
Update the alias to point to the \var{file} given. If \var{file2} is
present a relative alias will be created.
\end{funcdesc}

Note that it is currently not possible to directly manipulate a resource
as an alias object. Hence, after calling \var{Update} or after
\var{Resolve} indicates that the alias has changed the Python program
is responsible for getting the \var{data} from the alias object and
modifying the resource.


\subsection{FInfo objects}

See Inside Mac for a complete description of what the various fields
mean.

\renewcommand{\indexsubitem}{(FInfo object attribute)}
\begin{datadesc}{Creator}
The 4-char creator code of the file.
\end{datadesc}

\begin{datadesc}{Type}
The 4-char type code of the file.
\end{datadesc}

\begin{datadesc}{Flags}
The finder flags for the file as 16-bit integer. The bit values in
\var{Flags} are defined in standard module \var{MACFS}.
\end{datadesc}

\begin{datadesc}{Location}
A Point giving the position of the file's icon in its folder.
\end{datadesc}

\begin{datadesc}{Fldr}
The folder the file is in (as an integer).
\end{datadesc}

\section{Built-in Module \sectcode{MacOS}}
\bimodindex{MacOS}

\renewcommand{\indexsubitem}{(in module MacOS)}

This module provides access to MacOS specific functionality in the
python interpreter, such as how the interpreter eventloop functions
and the like. Use with care.

Note the capitalisation of the module name, this is a historical
artefact.

\begin{excdesc}{Error}
This exception is raised on MacOS generated errors, either from
functions in this module or from other mac-specific modules like the
toolbox interfaces. The arguments are the integer error code (the
\var{OSErr} value) and a textual description of the error code.
Symbolic names for all known error codes are defined in the standard
module \var{macerrors}.
\end{excdesc}

\begin{funcdesc}{SetHighLevelEventHandler}{handler}
Pass a python function that will be called upon reception of a
high-level event. The previous handler is returned. The handler
function is called with the event as argument.

Note that your event handler is currently only called dependably if
your main event loop is in \var{stdwin}.
\end{funcdesc}

\begin{funcdesc}{AcceptHighLevelEvent}{}
Read a high-level event. The return value is a tuple \code{(sender,
refcon, data)}.
\end{funcdesc}

\begin{funcdesc}{SetScheduleTimes}{fgi\, fgy \optional{\, bgi\, bgy}}
Controls how often the interpreter checks the event queue and how
long it will yield the processor to other processes. \var{fgi}
specifies after how many clicks (one click is one 60th of a second)
the interpreter should check the event queue, and \var{fgy} specifies
for how many clicks the CPU should be yielded when in the
foreground. The optional \var{bgi} and \var{bgy} allow you to specify
different values to use when python runs in the background, otherwise
the background values will be set the the same as the foreground
values. The function returns nothing.

The default values, which are based on minimal empirical testing, are 12, 1, 6
and 2 respectively.
\end{funcdesc}

\begin{funcdesc}{EnableAppswitch}{onoff}
Enable or disable the python event loop, based on the value of
\var{onoff}. The old value is returned. If the event loop is disabled
no time is granted to other applications, checking for command-period
is not performed and it is impossible to switch applications. This
should only be used by programs providing their own complete event
loop.

Note that based on the compiler used to build python it is still
possible to loose events even with the python event loop disabled. If
you have a \code{sys.stdout} window its handler will often also look
in the event queue. Making sure nothing is ever printed works around
this.
\end{funcdesc}

\begin{funcdesc}{HandleEvent}{ev}
Pass the event record \code{ev} back to the python event loop, or
possibly to the handler for the \code{sys.stdout} window (based on the
compiler used to build python). This allows python programs that do
their own event handling to still have some command-period and
window-switching capability.
\end{funcdesc}

\begin{funcdesc}{GetErrorString}{errno}
Return the textual description of MacOS error code \var{errno}.
\end{funcdesc}

\begin{funcdesc}{splash}{resid}
This function will put a splash window
on-screen, with the contents of the DLOG resource specified by
\code{resid}. Calling with a zero argument will remove the splash
screen. This function is useful if you want an applet to post a splash screen
early in initialization without first having to load numerous
extension modules.
\end{funcdesc}

\begin{funcdesc}{DebugStr}{message \optional{\, object}}
Drop to the low-level debugger with message \var{message}. The
optional \var{object} argument is not used, but can easily be
inspected from the debugger.

Note that you should use this function with extreme care: if no
low-level debugger like MacsBug is installed this call will crash your
system. It is intended mainly for developers of Python extension
modules.
\end{funcdesc}

\begin{funcdesc}{openrf}{name \optional{\, mode}}
Open the resource fork of a file. Arguments are the same as for the
builtin function \code{open}. The object returned has file-like
semantics, but it is not a python file object, so there may be subtle
differences.
\end{funcdesc}


\section{Standard module \sectcode{macostools}}
\stmodindex{macostools}

This module contains some convenience routines for file-manipulation
on the Macintosh.

The \code{macostools} module defines the following functions:

\renewcommand{\indexsubitem}{(in module macostools)}

\begin{funcdesc}{copy}{src\, dst\optional{\, createpath, copytimes}}
Copy file \var{src} to \var{dst}. The files can be specified as
pathnames or \code{FSSpec} objects. If \var{createpath} is non-zero
\var{dst} must be a pathname and the folders leading to the
destination are created if necessary.  The method copies data and
resource fork and some finder information (creator, type, flags) and
optionally the creation, modification and backup times (default is to
copy them). Custom icons, comments and icon position are not copied.

If the source is an alias the original to which the alias points is
copied, not the aliasfile.
\end{funcdesc}

\begin{funcdesc}{copytree}{src\, dst}
Recursively copy a file tree from \var{src} to \var{dst}, creating
folders as needed. \var{Src} and \var{dst} should be specified as
pathnames.
\end{funcdesc}

\begin{funcdesc}{mkalias}{src\, dst}
Create a finder alias \var{dst} pointing to \var{src}. Both may be
specified as pathnames or \var{FSSpec} objects.
\end{funcdesc}

\begin{funcdesc}{touched}{dst}
Tell the finder that some bits of finder-information such as creator
or type for file \var{dst} has changed. The file can be specified by
pathname or fsspec. This call should prod the finder into redrawing the
files icon.
\end{funcdesc}

\begin{datadesc}{BUFSIZ}
The buffer size for \code{copy}, default 1 megabyte.
\end{datadesc}

Note that the process of creating finder aliases is not specified in
the Apple documentation. Hence, aliases created with \code{mkalias}
could conceivably have incompatible behaviour in some cases.

\section{Standard module \sectcode{findertools}}
\stmodindex{findertools}

This module contains routines that give Python programs access to some
functionality provided by the finder. They are implemented as wrappers
around the AppleEvent interface to the finder.

All file and folder parameters can be specified either as full
pathnames or as \code{FSSpec} objects.

The \code{findertools} module defines the following functions:

\renewcommand{\indexsubitem}{(in module macostools)}

\begin{funcdesc}{launch}{file}
Tell the finder to launch \var{file}. What launching means depends on the file:
applications are started, folders are opened and documents are opened
in the correct application.
\end{funcdesc}

\begin{funcdesc}{Print}{file}
Tell the finder to print a file (again specified by full pathname or
FSSpec). The behaviour is identical to selecting the file and using
the print command in the finder.
\end{funcdesc}

\begin{funcdesc}{copy}{file, destdir}
Tell the finder to copy a file or folder \var{file} to folder
\var{destdir}. The function returns an \code{Alias} object pointing to
the new file.
\end{funcdesc}

\begin{funcdesc}{move}{file, destdir}
Tell the finder to move a file or folder \var{file} to folder
\var{destdir}. The function returns an \code{Alias} object pointing to
the new file.
\end{funcdesc}

\begin{funcdesc}{sleep}{}
Tell the finder to put the mac to sleep, if your machine supports it.
\end{funcdesc}

\begin{funcdesc}{restart}{}
Tell the finder to perform an orderly restart of the machine.
\end{funcdesc}

\begin{funcdesc}{shutdown}{}
Tell the finder to perform an orderly shutdown of the machine.
\end{funcdesc}

\section{Built-in Module \sectcode{mactcp}}
\bimodindex{mactcp}

\renewcommand{\indexsubitem}{(in module mactcp)}

This module provides an interface to the Macintosh TCP/IP driver
MacTCP\@. There is an accompanying module \code{macdnr} which provides an
interface to the name-server (allowing you to translate hostnames to
ip-addresses), a module \code{MACTCPconst} which has symbolic names for
constants constants used by MacTCP. Since the builtin module
\code{socket} is also available on the mac it is usually easier to use
sockets in stead of the mac-specific MacTCP API.

A complete description of the MacTCP interface can be found in the
Apple MacTCP API documentation.

\begin{funcdesc}{MTU}{}
Return the Maximum Transmit Unit (the packet size) of the network
interface.
\end{funcdesc}

\begin{funcdesc}{IPAddr}{}
Return the 32-bit integer IP address of the network interface.
\end{funcdesc}

\begin{funcdesc}{NetMask}{}
Return the 32-bit integer network mask of the interface.
\end{funcdesc}

\begin{funcdesc}{TCPCreate}{size}
Create a TCP Stream object. \var{size} is the size of the receive
buffer, \code{4096} is suggested by various sources.
\end{funcdesc}

\begin{funcdesc}{UDPCreate}{size, port}
Create a UDP stream object. \var{size} is the size of the receive
buffer (and, hence, the size of the biggest datagram you can receive
on this port). \var{port} is the UDP port number you want to receive
datagrams on, a value of zero will make MacTCP select a free port.
\end{funcdesc}

\subsection{TCP Stream Objects}

\renewcommand{\indexsubitem}{(TCP stream attribute)}

\begin{datadesc}{asr}
When set to a value different than \code{None} this should point to a
function with two integer parameters:\ an event code and a detail. This
function will be called upon network-generated events such as urgent
data arrival. In addition, it is called with eventcode
\code{MACTCP.PassiveOpenDone} when a \code{PassiveOpen} completes. This
is a Python addition to the MacTCP semantics.
It is safe to do further calls from the \code{asr}.
\end{datadesc}

\renewcommand{\indexsubitem}{(TCP stream method)}

\begin{funcdesc}{PassiveOpen}{port}
Wait for an incoming connection on TCP port \var{port} (zero makes the
system pick a free port). The call returns immediately, and you should
use \var{wait} to wait for completion. You should not issue any method
calls other than
\code{wait}, \code{isdone} or \code{GetSockName} before the call
completes.
\end{funcdesc}

\begin{funcdesc}{wait}{}
Wait for \code{PassiveOpen} to complete.
\end{funcdesc}

\begin{funcdesc}{isdone}{}
Return 1 if a \code{PassiveOpen} has completed.
\end{funcdesc}

\begin{funcdesc}{GetSockName}{}
Return the TCP address of this side of a connection as a 2-tuple
\code{(host, port)}, both integers.
\end{funcdesc}

\begin{funcdesc}{ActiveOpen}{lport\, host\, rport}
Open an outgoing connection to TCP address \code{(\var{host}, \var{rport})}. Use
local port \var{lport} (zero makes the system pick a free port). This
call blocks until the connection has been established.
\end{funcdesc}

\begin{funcdesc}{Send}{buf\, push\, urgent}
Send data \var{buf} over the connection. \var{Push} and \var{urgent}
are flags as specified by the TCP standard.
\end{funcdesc}

\begin{funcdesc}{Rcv}{timeout}
Receive data. The call returns when \var{timeout} seconds have passed
or when (according to the MacTCP documentation) ``a reasonable amount
of data has been received''. The return value is a 3-tuple
\code{(\var{data}, \var{urgent}, \var{mark})}. If urgent data is outstanding \code{Rcv}
will always return that before looking at any normal data. The first
call returning urgent data will have the \var{urgent} flag set, the
last will have the \var{mark} flag set.
\end{funcdesc}

\begin{funcdesc}{Close}{}
Tell MacTCP that no more data will be transmitted on this
connection. The call returns when all data has been acknowledged by
the receiving side.
\end{funcdesc}

\begin{funcdesc}{Abort}{}
Forcibly close both sides of a connection, ignoring outstanding data.
\end{funcdesc}

\begin{funcdesc}{Status}{}
Return a TCP status object for this stream giving the current status
(see below).
\end{funcdesc}

\subsection{TCP Status Objects}
This object has no methods, only some members holding information on
the connection. A complete description of all fields in this objects
can be found in the Apple documentation. The most interesting ones are:

\renewcommand{\indexsubitem}{(TCP status attribute)}

\begin{datadesc}{localHost}
\dataline{localPort}
\dataline{remoteHost}
\dataline{remotePort}
The integer IP-addresses and port numbers of both endpoints of the
connection. 
\end{datadesc}

\begin{datadesc}{sendWindow}
The current window size.
\end{datadesc}

\begin{datadesc}{amtUnackedData}
The number of bytes sent but not yet acknowledged. \code{sendWindow -
amtUnackedData} is what you can pass to \code{Send} without blocking.
\end{datadesc}

\begin{datadesc}{amtUnreadData}
The number of bytes received but not yet read (what you can \code{Recv}
without blocking).
\end{datadesc}



\subsection{UDP Stream Objects}
Note that, unlike the name suggests, there is nothing stream-like
about UDP.

\renewcommand{\indexsubitem}{(UDP stream attribute)}

\begin{datadesc}{asr}
The asynchronous service routine to be called on events such as
datagram arrival without outstanding \code{Read} call. The \code{asr} has a
single argument, the event code.
\end{datadesc}

\begin{datadesc}{port}
A read-only member giving the port number of this UDP stream.
\end{datadesc}

\renewcommand{\indexsubitem}{(UDP stream method)}

\begin{funcdesc}{Read}{timeout}
Read a datagram, waiting at most \var{timeout} seconds ($-1$ is
infinite).  Return the data.
\end{funcdesc}

\begin{funcdesc}{Write}{host\, port\, buf}
Send \var{buf} as a datagram to IP-address \var{host}, port
\var{port}.
\end{funcdesc}

\section{Built-in Module \sectcode{macspeech}}
\bimodindex{macspeech}

\renewcommand{\indexsubitem}{(in module macspeech)}

This module provides an interface to the Macintosh Speech Manager,
allowing you to let the Macintosh utter phrases. You need a version of
the speech manager extension (version 1 and 2 have been tested) in
your \code{Extensions} folder for this to work. The module does not
provide full access to all features of the Speech Manager yet.  It may
not be available in all Mac Python versions.

\begin{funcdesc}{Available}{}
Test availability of the Speech Manager extension (and, on the
PowerPC, the Speech Manager shared library). Return 0 or 1. 
\end{funcdesc}

\begin{funcdesc}{Version}{}
Return the (integer) version number of the Speech Manager.
\end{funcdesc}

\begin{funcdesc}{SpeakString}{str}
Utter the string \var{str} using the default voice,
asynchronously. This aborts any speech that may still be active from
prior \code{SpeakString} invocations.
\end{funcdesc}

\begin{funcdesc}{Busy}{}
Return the number of speech channels busy, system-wide.
\end{funcdesc}

\begin{funcdesc}{CountVoices}{}
Return the number of different voices available.
\end{funcdesc}

\begin{funcdesc}{GetIndVoice}{num}
Return a voice object for voice number \var{num}.
\end{funcdesc}

\subsection{voice objects}
Voice objects contain the description of a voice. It is currently not
yet possible to access the parameters of a voice.

\renewcommand{\indexsubitem}{(voice object method)}

\begin{funcdesc}{GetGender}{}
Return the gender of the voice:\ 0 for male, 1 for female and $-1$ for neuter.
\end{funcdesc}

\begin{funcdesc}{NewChannel}{}
Return a new speech channel object using this voice.
\end{funcdesc}

\subsection{speech channel objects}
A speech channel object allows you to speak strings with slightly more
control than \code{SpeakString()}, and allows you to use multiple
speakers at the same time. Please note that channel pitch and rate are
interrelated in some way, so that to make your Macintosh sing you will
have to adjust both.

\renewcommand{\indexsubitem}{(speech channel object method)}
\begin{funcdesc}{SpeakText}{str}
Start uttering the given string.
\end{funcdesc}

\begin{funcdesc}{Stop}{}
Stop babbling.
\end{funcdesc}

\begin{funcdesc}{GetPitch}{}
Return the current pitch of the channel, as a floating-point number.
\end{funcdesc}

\begin{funcdesc}{SetPitch}{pitch}
Set the pitch of the channel.
\end{funcdesc}

\begin{funcdesc}{GetRate}{}
Get the speech rate (utterances per minute) of the channel as a
floating point number.
\end{funcdesc}

\begin{funcdesc}{SetRate}{rate}
Set the speech rate of the channel.
\end{funcdesc}


\section{Standard module \sectcode{EasyDialogs}}
\stmodindex{EasyDialogs}

The \code{EasyDialogs} module contains some simple dialogs for
the Macintosh, modelled after the \code{stdwin} dialogs with similar
names.

The \code{EasyDialogs} module defines the following functions:

\renewcommand{\indexsubitem}{(in module EasyDialogs)}

\begin{funcdesc}{Message}{str}
A modal dialog with the message text \var{str}, which should be at
most 255 characters long, is displayed. Control is returned when the
user clicks ``OK''.
\end{funcdesc}

\begin{funcdesc}{AskString}{prompt\optional{\, default}}
Ask the user to input a string value, in a modal dialog. \var{Prompt}
is the promt message, the optional \var{default} arg is the initial
value for the string. All strings can be at most 255 bytes
long. \var{AskString} returns the string entered or \code{None} in
case the user cancelled.
\end{funcdesc}

\begin{funcdesc}{AskYesNoCancel}{question\optional{\, default}}
Present a dialog with text \var{question} and three buttons labelled
``yes'', ``no'' and ``cancel''. Return \code{1} for yes, \code{0} for
no and \code{-1} for cancel. The default return value chosen by
hitting return is \code{0}. This can be changed with the optional
\var{default} argument.
\end{funcdesc}

\begin{funcdesc}{ProgressBar}{\optional{label\, maxval}}
Display a modeless progress dialog with a thermometer bar. \var{Label}
is the textstring displayed (default ``Working...''), \var{maxval} is
the value at which progress is complete (default 100). The returned
object has one method, \code{set(value)}, which sets the value of the
progress bar. The bar remains visible until the object returned is
discarded.

The progress bar has a ``cancel'' button, but it is currently
non-functional.
\end{funcdesc}

Note that \code{EasyDialogs} does not currently use the notification
manager. This means that displaying dialogs while the program is in
the background will lead to unexpected results and possibly
crashes. Also, all dialogs are modeless and hence expect to be at the
top of the stacking order. This is true when the dialogs are created,
but windows that pop-up later (like a console window) may also result
in crashes.


\section{Standard module \sectcode{FrameWork}}
\stmodindex{FrameWork}

The \code{FrameWork} module contains classes that together provide a
framework for an interactive Macintosh application. The programmer
builds an application by creating subclasses that override various
methods of the bases classes, thereby implementing the functionality
wanted. Overriding functionality can often be done on various
different levels, i.e. to handle clicks in a single dialog window in a
non-standard way it is not necessary to override the complete event
handling.

The \code{FrameWork} is still very much work-in-progress, and the
documentation describes only the most important functionality, and not
in the most logical manner at that. Examine the source or the examples
for more details.

The \code{FrameWork} module defines the following functions:

\renewcommand{\indexsubitem}{(in module FrameWork)}

\begin{funcdesc}{Application}{}
An object representing the complete application. See below for a
description of the methods. The default \code{__init__} routine
creates an empty window dictionary and a menu bar with an apple menu.
\end{funcdesc}

\begin{funcdesc}{MenuBar}{}
An object representing the menubar. This object is usually not created
by the user.
\end{funcdesc}

\begin{funcdesc}{Menu}{bar\, title\optional{\, after}}
An object representing a menu. Upon creation you pass the
\code{MenuBar} the menu appears in, the \var{title} string and a
position (1-based) \var{after} where the menu should appear (default:
at the end).
\end{funcdesc}

\begin{funcdesc}{MenuItem}{menu\, title\optional{\, shortcut\, callback}}
Create a menu item object. The arguments are the menu to crate the
item it, the item title string and optionally the keyboard shortcut
and a callback routine. The callback is called with the arguments
menu-id, item number within menu (1-based), current front window and
the event record.

In stead of a callable object the callback can also be a string. In
this case menu selection causes the lookup of a method in the topmost
window and the application. The method name is the callback string
with \code{'domenu_'} prepended.

Calling the \code{MenuBar} \code{fixmenudimstate} method sets the
correct dimming for all menu items based on the current front window.
\end{funcdesc}

\begin{funcdesc}{Separator}{menu}
Add a separator to the end of a menu.
\end{funcdesc}

\begin{funcdesc}{SubMenu}{menu\, label}
Create a submenu named \var{label} under menu \var{menu}. The menu
object is returned.
\end{funcdesc}

\begin{funcdesc}{Window}{parent}
Creates a (modeless) window. \var{Parent} is the application object to
which the window belongs. The window is not displayed until later.
\end{funcdesc}

\begin{funcdesc}{DialogWindow}{parent}
Creates a modeless dialog window.
\end{funcdesc}

\begin{funcdesc}{windowbounds}{width\, height}
Return a \code{(left, top, right, bottom)} tuple suitable for creation
of a window of given width and height. The window will be staggered
with respect to previous windows, and an attempt is made to keep the
whole window on-screen. The window will however always be exact the
size given, so parts may be offscreen.
\end{funcdesc}

\begin{funcdesc}{setwatchcursor}{}
Set the mouse cursor to a watch.
\end{funcdesc}

\begin{funcdesc}{setarrowcursor}{}
Set the mouse cursor to an arrow.
\end{funcdesc}

\subsection{Application objects}
Application objects have the following methods, among others:

\renewcommand{\indexsubitem}{(Application method)}

\begin{funcdesc}{makeusermenus}{}
Override this method if you need menus in your application. Append the
menus to \code{self.menubar}.
\end{funcdesc}

\begin{funcdesc}{getabouttext}{}
Override this method to return a text string describing your
application. Alternatively, override the \code{do_about} method for
more elaborate about messages.
\end{funcdesc}

\begin{funcdesc}{mainloop}{\optional{mask\, wait}}
This routine is the main event loop, call it to set your application
rolling. \var{Mask} is the mask of events you want to handle,
\var{wait} is the number of ticks you want to leave to other
concurrent application (default 0, which is probably not a good
idea). While raising \code{self} to exit the mainloop is still
supported it is not recommended, call \code{self._quit} instead.

The event loop is split into many small parts, each of which can be
overridden. The default methods take care of dispatching events to
windows and dialogs, handling drags and resizes, Apple Events, events
for non-FrameWork windows, etc.
\end{funcdesc}

\begin{funcdesc}{_quit}{}
Terminate the event \code{mainloop} at the next convenient moment.
\end{funcdesc}

\begin{funcdesc}{do_char}{c\, event}
The user typed character \var{c}. The complete details of the event
can be found in the \var{event} structure. This method can also be
provided in a \code{Window} object, which overrides the
application-wide handler if the window is frontmost.
\end{funcdesc}

\begin{funcdesc}{do_dialogevent}{event}
Called early in the event loop to handle modeless dialog events. The
default method simply dispatches the event to the relevant dialog (not
through the the \code{DialogWindow} object involved). Override if you
need special handling of dialog events (keyboard shortcuts, etc).
\end{funcdesc}

\begin{funcdesc}{idle}{event}
Called by the main event loop when no events are available. The
null-event is passed (so you can look at mouse position, etc).
\end{funcdesc}

\subsection{Window Objects}

Window objects have the following methods, among others:

\renewcommand{\indexsubitem}{(Window method)}

\begin{funcdesc}{open}{}
Override this method to open a window. Store the MacOS window-id in
\code{self.wid} and call \code{self.do_postopen} to register the
window with the parent application.
\end{funcdesc}

\begin{funcdesc}{close}{}
Override this method to do any special processing on window
close. Call \code{self.do_postclose} to cleanup the parent state.
\end{funcdesc}

\begin{funcdesc}{do_postresize}{width\, height\, macoswindowid}
Called after the window is resized. Override if more needs to be done
than calling \code{InvalRect}.
\end{funcdesc}

\begin{funcdesc}{do_contentclick}{local\, modifiers\, event}
The user clicked in the content part of a window. The arguments are
the coordinates (window-relative), the key modifiers and the raw
event.
\end{funcdesc}

\begin{funcdesc}{do_update}{macoswindowid\, event}
An update event for the window was received. Redraw the window.
\end{funcdesc}

\begin{funcdesc}{do_activate}{activate\, event}
The window was activated (\code{activate==1}) or deactivated
(\code{activate==0}). Handle things like focus highlighting, etc.
\end{funcdesc}

\subsection{ControlsWindow Object}

ControlsWindow objects have the following methods besides those of
\code{Window} objects:

\renewcommand{\indexsubitem}{(ControlsWindow method)}

\begin{funcdesc}{do_controlhit}{window\, control\, pcode\, event}
Part \code{pcode} of control \code{control} was hit by the
user. Tracking and such has already been taken care of.
\end{funcdesc}

\subsection{ScrolledWindow Object}

ScrolledWindow objects are ControlsWindow objects with the following
extra methods:

\renewcommand{\indexsubitem}{(ScrolledWindow method)}

\begin{funcdesc}{scrollbars}{\optional{wantx\, wanty}}
Create (or destroy) horizontal and vertical scrollbars. The arguments
specify which you want (default: both). The scrollbars always have
minimum \code{0} and maximum \code{32767}.
\end{funcdesc}

\begin{funcdesc}{getscrollbarvalues}{}
You must supply this method. It should return a tuple \code{x, y}
giving the current position of the scrollbars (between \code{0} and
\code{32767}). You can return \code{None} for either to indicate the
whole document is visible in that direction.
\end{funcdesc}

\begin{funcdesc}{updatescrollbars}{}
Call this method when the document has changed. It will call
\code{getscrollbarvalues} and update the scrollbars.
\end{funcdesc}

\begin{funcdesc}{scrollbar_callback}{which\, what\, value}
Supplied by you and called after user interaction. \code{Which} will
be \code{'x'} or \code{'y'}, \code{what} will be \code{'-'},
\code{'--'}, \code{'set'}, \code{'++'} or \code{'+'}. For
\code{'set'}, \code{value} will contain the new scrollbar position.
\end{funcdesc}

\begin{funcdesc}{scalebarvalues}{absmin\, absmax\, curmin\, curmax}
Auxiliary method to help you calculate values to return from
\code{getscrollbarvalues}. You pass document minimum and maximum value
and topmost (leftmost) and bottommost (rightmost) visible values and
it returns the correct number or \code{None}.
\end{funcdesc}

\begin{funcdesc}{do_activate}{onoff\, event}
Takes care of dimming/highlighting scrollbars when a window becomes
frontmost vv. If you override this method call this one at the end of
your method.
\end{funcdesc}

\begin{funcdesc}{do_postresize}{width\, height\, window}
Moves scrollbars to the correct position. Call this method initially
if you override it.
\end{funcdesc}

\begin{funcdesc}{do_controlhit}{window\, control\, pcode\, event}
Handles scrollbar interaction. If you override it call this method
first, a nonzero return value indicates the hit was in the scrollbars
and has been handled.
\end{funcdesc}

\subsection{DialogWindow Objects}

DialogWindow objects have the following methods besides those of
\code{Window} objects:

\renewcommand{\indexsubitem}{(DialogWindow method)}

\begin{funcdesc}{open}{resid}
Create the dialog window, from the DLOG resource with id
\var{resid}. The dialog object is stored in \code{self.wid}.
\end{funcdesc}

\begin{funcdesc}{do_itemhit}{item\, event}
Item number \var{item} was hit. You are responsible for redrawing
toggle buttons, etc.
\end{funcdesc}

\section{Standard module \sectcode{MiniAEFrame}}
\stmodindex{MiniAEFrame}

The module \var{MiniAEFrame} provides a framework for an application
that can function as an OSA server, i.e. receive and process
AppleEvents. It can be used in conjunction with \var{FrameWork} or
standalone.

This module is temporary, it will eventually be replaced by a module
that handles argument names better and possibly automates making your
application scriptable.

The \var{MiniAEFrame} module defines the following classes:

\renewcommand{\indexsubitem}{(in module MiniAEFrame)}

\begin{funcdesc}{AEServer}{}
A class that handles AppleEvent dispatch. Your application should
subclass this class together with either
\code{MiniAEFrame.MiniApplication} or
\code{FrameWork.Application}. Your \code{__init__} method should call
the \code{__init__} method for both classes.
\end{funcdesc}

\begin{funcdesc}{MiniApplication}{}
A class that is more or less compatible with
\code{FrameWork.Application} but with less functionality. Its
eventloop supports the apple menu, command-dot and AppleEvents, other
events are passed on to the Python interpreter and/or Sioux.
Useful if your application wants to use \code{AEServer} but does not
provide its own windows, etc.
\end{funcdesc}

\subsection{AEServer Objects}

\renewcommand{\indexsubitem}{(AEServer method)}

\begin{funcdesc}{installaehandler}{classe\, type\, callback}
Installs an AppleEvent handler. \code{Classe} and \code{type} are the
four-char OSA Class and Type designators, \code{'****'} wildcards are
allowed. When a matching AppleEvent is received the parameters are
decoded and your callback is invoked.
\end{funcdesc}

\begin{funcdesc}{callback}{_object\, **kwargs}
Your callback is called with the OSA Direct Object as first positional
parameter. The other parameters are passed as keyword arguments, with
the 4-char designator as name. Three extra keyword parameters are
passed: \code{_class} and \code{_type} are the Class and Type
designators and \code{_attributes} is a dictionary with the AppleEvent
attributes.

The return value of your method is packed with
\code{aetools.packevent} and sent as reply.
\end{funcdesc}

Note that there are some serious problems with the current
design. AppleEvents which have non-identifier 4-char designators for
arguments are not implementable, and it is not possible to return an
error to the originator. This will be addressed in a future release.


\chapter{Standard Windowing Interface}

The modules in this chapter are available only on those systems where
the STDWIN library is available.  STDWIN runs on \UNIX{} under X11 and
on the Macintosh.  See CWI report CS-R8817.

\strong{Warning:} Using STDWIN is not recommended for new
applications.  It has never been ported to Microsoft Windows or
Windows NT, and for X11 or the Macintosh it lacks important
functionality --- in particular, it has no tools for the construction
of dialogs.  For most platforms, alternative, native solutions exist
(though none are currently documented in this manual): Tkinter for
\UNIX{} under X11, native Xt with Motif or Athena widgets for \UNIX{}
under X11, Win32 for Windows and Windows NT, and a collection of
native toolkit interfaces for the Macintosh.

\section{Built-in Module \sectcode{stdwin}}
\bimodindex{stdwin}

This module defines several new object types and functions that
provide access to the functionality of STDWIN.

On Unix running X11, it can only be used if the \code{DISPLAY}
environment variable is set or an explicit \samp{-display
\var{displayname}} argument is passed to the Python interpreter.

Functions have names that usually resemble their C STDWIN counterparts
with the initial `w' dropped.
Points are represented by pairs of integers; rectangles
by pairs of points.
For a complete description of STDWIN please refer to the documentation
of STDWIN for C programmers (aforementioned CWI report).

\subsection{Functions Defined in Module \sectcode{stdwin}}
\nodename{STDWIN Functions}

The following functions are defined in the \code{stdwin} module:

\renewcommand{\indexsubitem}{(in module stdwin)}
\begin{funcdesc}{open}{title}
Open a new window whose initial title is given by the string argument.
Return a window object; window object methods are described below.%
\footnote{The Python version of STDWIN does not support draw procedures; all
	drawing requests are reported as draw events.}
\end{funcdesc}

\begin{funcdesc}{getevent}{}
Wait for and return the next event.
An event is returned as a triple: the first element is the event
type, a small integer; the second element is the window object to which
the event applies, or
\code{None}
if it applies to no window in particular;
the third element is type-dependent.
Names for event types and command codes are defined in the standard
module
\code{stdwinevent}.
\end{funcdesc}

\begin{funcdesc}{pollevent}{}
Return the next event, if one is immediately available.
If no event is available, return \code{()}.
\end{funcdesc}

\begin{funcdesc}{getactive}{}
Return the window that is currently active, or \code{None} if no
window is currently active.  (This can be emulated by monitoring
WE_ACTIVATE and WE_DEACTIVATE events.)
\end{funcdesc}

\begin{funcdesc}{listfontnames}{pattern}
Return the list of font names in the system that match the pattern (a
string).  The pattern should normally be \code{'*'}; returns all
available fonts.  If the underlying window system is X11, other
patterns follow the standard X11 font selection syntax (as used e.g.
in resource definitions), i.e. the wildcard character \code{'*'}
matches any sequence of characters (including none) and \code{'?'}
matches any single character.
On the Macintosh this function currently returns an empty list.
\end{funcdesc}

\begin{funcdesc}{setdefscrollbars}{hflag\, vflag}
Set the flags controlling whether subsequently opened windows will
have horizontal and/or vertical scroll bars.
\end{funcdesc}

\begin{funcdesc}{setdefwinpos}{h\, v}
Set the default window position for windows opened subsequently.
\end{funcdesc}

\begin{funcdesc}{setdefwinsize}{width\, height}
Set the default window size for windows opened subsequently.
\end{funcdesc}

\begin{funcdesc}{getdefscrollbars}{}
Return the flags controlling whether subsequently opened windows will
have horizontal and/or vertical scroll bars.
\end{funcdesc}

\begin{funcdesc}{getdefwinpos}{}
Return the default window position for windows opened subsequently.
\end{funcdesc}

\begin{funcdesc}{getdefwinsize}{}
Return the default window size for windows opened subsequently.
\end{funcdesc}

\begin{funcdesc}{getscrsize}{}
Return the screen size in pixels.
\end{funcdesc}

\begin{funcdesc}{getscrmm}{}
Return the screen size in millimeters.
\end{funcdesc}

\begin{funcdesc}{fetchcolor}{colorname}
Return the pixel value corresponding to the given color name.
Return the default foreground color for unknown color names.
Hint: the following code tests whether you are on a machine that
supports more than two colors:
\bcode\begin{verbatim}
if stdwin.fetchcolor('black') <> \
          stdwin.fetchcolor('red') <> \
          stdwin.fetchcolor('white'):
    print 'color machine'
else:
    print 'monochrome machine'
\end{verbatim}\ecode
\end{funcdesc}

\begin{funcdesc}{setfgcolor}{pixel}
Set the default foreground color.
This will become the default foreground color of windows opened
subsequently, including dialogs.
\end{funcdesc}

\begin{funcdesc}{setbgcolor}{pixel}
Set the default background color.
This will become the default background color of windows opened
subsequently, including dialogs.
\end{funcdesc}

\begin{funcdesc}{getfgcolor}{}
Return the pixel value of the current default foreground color.
\end{funcdesc}

\begin{funcdesc}{getbgcolor}{}
Return the pixel value of the current default background color.
\end{funcdesc}

\begin{funcdesc}{setfont}{fontname}
Set the current default font.
This will become the default font for windows opened subsequently,
and is also used by the text measuring functions \code{textwidth},
\code{textbreak}, \code{lineheight} and \code{baseline} below.
This accepts two more optional parameters, size and style:
Size is the font size (in `points').
Style is a single character specifying the style, as follows:
\code{'b'} = bold,
\code{'i'} = italic,
\code{'o'} = bold + italic,
\code{'u'} = underline;
default style is roman.
Size and style are ignored under X11 but used on the Macintosh.
(Sorry for all this complexity --- a more uniform interface is being designed.)
\end{funcdesc}

\begin{funcdesc}{menucreate}{title}
Create a menu object referring to a global menu (a menu that appears in
all windows).
Methods of menu objects are described below.
Note: normally, menus are created locally; see the window method
\code{menucreate} below.
\strong{Warning:} the menu only appears in a window as long as the object
returned by this call exists.
\end{funcdesc}

\begin{funcdesc}{newbitmap}{width\, height}
Create a new bitmap object of the given dimensions.
Methods of bitmap objects are described below.
Not available on the Macintosh.
\end{funcdesc}

\begin{funcdesc}{fleep}{}
Cause a beep or bell (or perhaps a `visual bell' or flash, hence the
name).
\end{funcdesc}

\begin{funcdesc}{message}{string}
Display a dialog box containing the string.
The user must click OK before the function returns.
\end{funcdesc}

\begin{funcdesc}{askync}{prompt\, default}
Display a dialog that prompts the user to answer a question with yes or
no.
Return 0 for no, 1 for yes.
If the user hits the Return key, the default (which must be 0 or 1) is
returned.
If the user cancels the dialog, the
\code{KeyboardInterrupt}
exception is raised.
\end{funcdesc}

\begin{funcdesc}{askstr}{prompt\, default}
Display a dialog that prompts the user for a string.
If the user hits the Return key, the default string is returned.
If the user cancels the dialog, the
\code{KeyboardInterrupt}
exception is raised.
\end{funcdesc}

\begin{funcdesc}{askfile}{prompt\, default\, new}
Ask the user to specify a filename.
If
\var{new}
is zero it must be an existing file; otherwise, it must be a new file.
If the user cancels the dialog, the
\code{KeyboardInterrupt}
exception is raised.
\end{funcdesc}

\begin{funcdesc}{setcutbuffer}{i\, string}
Store the string in the system's cut buffer number
\var{i},
where it can be found (for pasting) by other applications.
On X11, there are 8 cut buffers (numbered 0..7).
Cut buffer number 0 is the `clipboard' on the Macintosh.
\end{funcdesc}

\begin{funcdesc}{getcutbuffer}{i}
Return the contents of the system's cut buffer number
\var{i}.
\end{funcdesc}

\begin{funcdesc}{rotatecutbuffers}{n}
On X11, rotate the 8 cut buffers by
\var{n}.
Ignored on the Macintosh.
\end{funcdesc}

\begin{funcdesc}{getselection}{i}
Return X11 selection number
\var{i.}
Selections are not cut buffers.
Selection numbers are defined in module
\code{stdwinevents}.
Selection \code{WS_PRIMARY} is the
\dfn{primary}
selection (used by
xterm,
for instance);
selection \code{WS_SECONDARY} is the
\dfn{secondary}
selection; selection \code{WS_CLIPBOARD} is the
\dfn{clipboard}
selection (used by
xclipboard).
On the Macintosh, this always returns an empty string.
\end{funcdesc}

\begin{funcdesc}{resetselection}{i}
Reset selection number
\var{i},
if this process owns it.
(See window method
\code{setselection()}).
\end{funcdesc}

\begin{funcdesc}{baseline}{}
Return the baseline of the current font (defined by STDWIN as the
vertical distance between the baseline and the top of the
characters).
\end{funcdesc}

\begin{funcdesc}{lineheight}{}
Return the total line height of the current font.
\end{funcdesc}

\begin{funcdesc}{textbreak}{str\, width}
Return the number of characters of the string that fit into a space of
\var{width}
bits wide when drawn in the curent font.
\end{funcdesc}

\begin{funcdesc}{textwidth}{str}
Return the width in bits of the string when drawn in the current font.
\end{funcdesc}

\begin{funcdesc}{connectionnumber}{}
\funcline{fileno}{}
(X11 under \UNIX{} only) Return the ``connection number'' used by the
underlying X11 implementation.  (This is normally the file number of
the socket.)  Both functions return the same value;
\code{connectionnumber()} is named after the corresponding function in
X11 and STDWIN, while \code{fileno()} makes it possible to use the
\code{stdwin} module as a ``file'' object parameter to
\code{select.select()}.  Note that if \code{select()} implies that
input is possible on \code{stdwin}, this does not guarantee that an
event is ready --- it may be some internal communication going on
between the X server and the client library.  Thus, you should call
\code{stdwin.pollevent()} until it returns \code{None} to check for
events if you don't want your program to block.  Because of internal
buffering in X11, it is also possible that \code{stdwin.pollevent()}
returns an event while \code{select()} does not find \code{stdwin} to
be ready, so you should read any pending events with
\code{stdwin.pollevent()} until it returns \code{None} before entering
a blocking \code{select()} call.
\ttindex{select}
\end{funcdesc}

\subsection{Window Objects}

Window objects are created by \code{stdwin.open()}.  They are closed
by their \code{close()} method or when they are garbage-collected.
Window objects have the following methods:

\renewcommand{\indexsubitem}{(window method)}

\begin{funcdesc}{begindrawing}{}
Return a drawing object, whose methods (described below) allow drawing
in the window.
\end{funcdesc}

\begin{funcdesc}{change}{rect}
Invalidate the given rectangle; this may cause a draw event.
\end{funcdesc}

\begin{funcdesc}{gettitle}{}
Returns the window's title string.
\end{funcdesc}

\begin{funcdesc}{getdocsize}{}
\begin{sloppypar}
Return a pair of integers giving the size of the document as set by
\code{setdocsize()}.
\end{sloppypar}
\end{funcdesc}

\begin{funcdesc}{getorigin}{}
Return a pair of integers giving the origin of the window with respect
to the document.
\end{funcdesc}

\begin{funcdesc}{gettitle}{}
Return the window's title string.
\end{funcdesc}

\begin{funcdesc}{getwinsize}{}
Return a pair of integers giving the size of the window.
\end{funcdesc}

\begin{funcdesc}{getwinpos}{}
Return a pair of integers giving the position of the window's upper
left corner (relative to the upper left corner of the screen).
\end{funcdesc}

\begin{funcdesc}{menucreate}{title}
Create a menu object referring to a local menu (a menu that appears
only in this window).
Methods of menu objects are described below.
{\bf Warning:} the menu only appears as long as the object
returned by this call exists.
\end{funcdesc}

\begin{funcdesc}{scroll}{rect\, point}
Scroll the given rectangle by the vector given by the point.
\end{funcdesc}

\begin{funcdesc}{setdocsize}{point}
Set the size of the drawing document.
\end{funcdesc}

\begin{funcdesc}{setorigin}{point}
Move the origin of the window (its upper left corner)
to the given point in the document.
\end{funcdesc}

\begin{funcdesc}{setselection}{i\, str}
Attempt to set X11 selection number
\var{i}
to the string
\var{str}.
(See stdwin method
\code{getselection()}
for the meaning of
\var{i}.)
Return true if it succeeds.
If  succeeds, the window ``owns'' the selection until
(a) another application takes ownership of the selection; or
(b) the window is deleted; or
(c) the application clears ownership by calling
\code{stdwin.resetselection(\var{i})}.
When another application takes ownership of the selection, a
\code{WE_LOST_SEL}
event is received for no particular window and with the selection number
as detail.
Ignored on the Macintosh.
\end{funcdesc}

\begin{funcdesc}{settimer}{dsecs}
Schedule a timer event for the window in
\code{\var{dsecs}/10}
seconds.
\end{funcdesc}

\begin{funcdesc}{settitle}{title}
Set the window's title string.
\end{funcdesc}

\begin{funcdesc}{setwincursor}{name}
\begin{sloppypar}
Set the window cursor to a cursor of the given name.
It raises the
\code{RuntimeError}
exception if no cursor of the given name exists.
Suitable names include
\code{'ibeam'},
\code{'arrow'},
\code{'cross'},
\code{'watch'}
and
\code{'plus'}.
On X11, there are many more (see
\file{<X11/cursorfont.h>}).
\end{sloppypar}
\end{funcdesc}

\begin{funcdesc}{setwinpos}{h\, v}
Set the the position of the window's upper left corner (relative to
the upper left corner of the screen).
\end{funcdesc}

\begin{funcdesc}{setwinsize}{width\, height}
Set the window's size.
\end{funcdesc}

\begin{funcdesc}{show}{rect}
Try to ensure that the given rectangle of the document is visible in
the window.
\end{funcdesc}

\begin{funcdesc}{textcreate}{rect}
Create a text-edit object in the document at the given rectangle.
Methods of text-edit objects are described below.
\end{funcdesc}

\begin{funcdesc}{setactive}{}
Attempt to make this window the active window.  If successful, this
will generate a WE_ACTIVATE event (and a WE_DEACTIVATE event in case
another window in this application became inactive).
\end{funcdesc}

\begin{funcdesc}{close}{}
Discard the window object.  It should not be used again.
\end{funcdesc}

\subsection{Drawing Objects}

Drawing objects are created exclusively by the window method
\code{begindrawing()}.
Only one drawing object can exist at any given time; the drawing object
must be deleted to finish drawing.
No drawing object may exist when
\code{stdwin.getevent()}
is called.
Drawing objects have the following methods:

\renewcommand{\indexsubitem}{(drawing method)}

\begin{funcdesc}{box}{rect}
Draw a box just inside a rectangle.
\end{funcdesc}

\begin{funcdesc}{circle}{center\, radius}
Draw a circle with given center point and radius.
\end{funcdesc}

\begin{funcdesc}{elarc}{center\, \(rh\, rv\)\, \(a1\, a2\)}
Draw an elliptical arc with given center point.
\code{(\var{rh}, \var{rv})}
gives the half sizes of the horizontal and vertical radii.
\code{(\var{a1}, \var{a2})}
gives the angles (in degrees) of the begin and end points.
0 degrees is at 3 o'clock, 90 degrees is at 12 o'clock.
\end{funcdesc}

\begin{funcdesc}{erase}{rect}
Erase a rectangle.
\end{funcdesc}

\begin{funcdesc}{fillcircle}{center\, radius}
Draw a filled circle with given center point and radius.
\end{funcdesc}

\begin{funcdesc}{fillelarc}{center\, \(rh\, rv\)\, \(a1\, a2\)}
Draw a filled elliptical arc; arguments as for \code{elarc}.
\end{funcdesc}

\begin{funcdesc}{fillpoly}{points}
Draw a filled polygon given by a list (or tuple) of points.
\end{funcdesc}

\begin{funcdesc}{invert}{rect}
Invert a rectangle.
\end{funcdesc}

\begin{funcdesc}{line}{p1\, p2}
Draw a line from point
\var{p1}
to
\var{p2}.
\end{funcdesc}

\begin{funcdesc}{paint}{rect}
Fill a rectangle.
\end{funcdesc}

\begin{funcdesc}{poly}{points}
Draw the lines connecting the given list (or tuple) of points.
\end{funcdesc}

\begin{funcdesc}{shade}{rect\, percent}
Fill a rectangle with a shading pattern that is about
\var{percent}
percent filled.
\end{funcdesc}

\begin{funcdesc}{text}{p\, str}
Draw a string starting at point p (the point specifies the
top left coordinate of the string).
\end{funcdesc}

\begin{funcdesc}{xorcircle}{center\, radius}
\funcline{xorelarc}{center\, \(rh\, rv\)\, \(a1\, a2\)}
\funcline{xorline}{p1\, p2}
\funcline{xorpoly}{points}
Draw a circle, an elliptical arc, a line or a polygon, respectively,
in XOR mode.
\end{funcdesc}

\begin{funcdesc}{setfgcolor}{}
\funcline{setbgcolor}{}
\funcline{getfgcolor}{}
\funcline{getbgcolor}{}
These functions are similar to the corresponding functions described
above for the
\code{stdwin}
module, but affect or return the colors currently used for drawing
instead of the global default colors.
When a drawing object is created, its colors are set to the window's
default colors, which are in turn initialized from the global default
colors when the window is created.
\end{funcdesc}

\begin{funcdesc}{setfont}{}
\funcline{baseline}{}
\funcline{lineheight}{}
\funcline{textbreak}{}
\funcline{textwidth}{}
These functions are similar to the corresponding functions described
above for the
\code{stdwin}
module, but affect or use the current drawing font instead of
the global default font.
When a drawing object is created, its font is set to the window's
default font, which is in turn initialized from the global default
font when the window is created.
\end{funcdesc}

\begin{funcdesc}{bitmap}{point\, bitmap\, mask}
Draw the \var{bitmap} with its top left corner at \var{point}.
If the optional \var{mask} argument is present, it should be either
the same object as \var{bitmap}, to draw only those bits that are set
in the bitmap, in the foreground color, or \code{None}, to draw all
bits (ones are drawn in the foreground color, zeros in the background
color).
Not available on the Macintosh.
\end{funcdesc}

\begin{funcdesc}{cliprect}{rect}
Set the ``clipping region'' to a rectangle.
The clipping region limits the effect of all drawing operations, until
it is changed again or until the drawing object is closed.  When a
drawing object is created the clipping region is set to the entire
window.  When an object to be drawn falls partly outside the clipping
region, the set of pixels drawn is the intersection of the clipping
region and the set of pixels that would be drawn by the same operation
in the absence of a clipping region.
\end{funcdesc}

\begin{funcdesc}{noclip}{}
Reset the clipping region to the entire window.
\end{funcdesc}

\begin{funcdesc}{close}{}
\funcline{enddrawing}{}
Discard the drawing object.  It should not be used again.
\end{funcdesc}

\subsection{Menu Objects}

A menu object represents a menu.
The menu is destroyed when the menu object is deleted.
The following methods are defined:

\renewcommand{\indexsubitem}{(menu method)}

\begin{funcdesc}{additem}{text\, shortcut}
Add a menu item with given text.
The shortcut must be a string of length 1, or omitted (to specify no
shortcut).
\end{funcdesc}

\begin{funcdesc}{setitem}{i\, text}
Set the text of item number
\var{i}.
\end{funcdesc}

\begin{funcdesc}{enable}{i\, flag}
Enable or disables item
\var{i}.
\end{funcdesc}

\begin{funcdesc}{check}{i\, flag}
Set or clear the
\dfn{check mark}
for item
\var{i}.
\end{funcdesc}

\begin{funcdesc}{close}{}
Discard the menu object.  It should not be used again.
\end{funcdesc}

\subsection{Bitmap Objects}

A bitmap represents a rectangular array of bits.
The top left bit has coordinate (0, 0).
A bitmap can be drawn with the \code{bitmap} method of a drawing object.
Bitmaps are currently not available on the Macintosh.

The following methods are defined:

\renewcommand{\indexsubitem}{(bitmap method)}

\begin{funcdesc}{getsize}{}
Return a tuple representing the width and height of the bitmap.
(This returns the values that have been passed to the \code{newbitmap}
function.)
\end{funcdesc}

\begin{funcdesc}{setbit}{point\, bit}
Set the value of the bit indicated by \var{point} to \var{bit}.
\end{funcdesc}

\begin{funcdesc}{getbit}{point}
Return the value of the bit indicated by \var{point}.
\end{funcdesc}

\begin{funcdesc}{close}{}
Discard the bitmap object.  It should not be used again.
\end{funcdesc}

\subsection{Text-edit Objects}

A text-edit object represents a text-edit block.
For semantics, see the STDWIN documentation for C programmers.
The following methods exist:

\renewcommand{\indexsubitem}{(text-edit method)}

\begin{funcdesc}{arrow}{code}
Pass an arrow event to the text-edit block.
The
\var{code}
must be one of
\code{WC_LEFT},
\code{WC_RIGHT},
\code{WC_UP}
or
\code{WC_DOWN}
(see module
\code{stdwinevents}).
\end{funcdesc}

\begin{funcdesc}{draw}{rect}
Pass a draw event to the text-edit block.
The rectangle specifies the redraw area.
\end{funcdesc}

\begin{funcdesc}{event}{type\, window\, detail}
Pass an event gotten from
\code{stdwin.getevent()}
to the text-edit block.
Return true if the event was handled.
\end{funcdesc}

\begin{funcdesc}{getfocus}{}
Return 2 integers representing the start and end positions of the
focus, usable as slice indices on the string returned by
\code{gettext()}.
\end{funcdesc}

\begin{funcdesc}{getfocustext}{}
Return the text in the focus.
\end{funcdesc}

\begin{funcdesc}{getrect}{}
Return a rectangle giving the actual position of the text-edit block.
(The bottom coordinate may differ from the initial position because
the block automatically shrinks or grows to fit.)
\end{funcdesc}

\begin{funcdesc}{gettext}{}
Return the entire text buffer.
\end{funcdesc}

\begin{funcdesc}{move}{rect}
Specify a new position for the text-edit block in the document.
\end{funcdesc}

\begin{funcdesc}{replace}{str}
Replace the text in the focus by the given string.
The new focus is an insert point at the end of the string.
\end{funcdesc}

\begin{funcdesc}{setfocus}{i\, j}
Specify the new focus.
Out-of-bounds values are silently clipped.
\end{funcdesc}

\begin{funcdesc}{settext}{str}
Replace the entire text buffer by the given string and set the focus
to \code{(0, 0)}.
\end{funcdesc}

\begin{funcdesc}{setview}{rect}
Set the view rectangle to \var{rect}.  If \var{rect} is \code{None},
viewing mode is reset.  In viewing mode, all output from the text-edit
object is clipped to the viewing rectangle.  This may be useful to
implement your own scrolling text subwindow.
\end{funcdesc}

\begin{funcdesc}{close}{}
Discard the text-edit object.  It should not be used again.
\end{funcdesc}

\subsection{Example}
\nodename{STDWIN Example}

Here is a minimal example of using STDWIN in Python.
It creates a window and draws the string ``Hello world'' in the top
left corner of the window.
The window will be correctly redrawn when covered and re-exposed.
The program quits when the close icon or menu item is requested.

\bcode\begin{verbatim}
import stdwin
from stdwinevents import *

def main():
    mywin = stdwin.open('Hello')
    #
    while 1:
        (type, win, detail) = stdwin.getevent()
        if type == WE_DRAW:
            draw = win.begindrawing()
            draw.text((0, 0), 'Hello, world')
            del draw
        elif type == WE_CLOSE:
            break

main()
\end{verbatim}\ecode

\section{Standard Module \sectcode{stdwinevents}}
\stmodindex{stdwinevents}

This module defines constants used by STDWIN for event types
(\code{WE_ACTIVATE} etc.), command codes (\code{WC_LEFT} etc.)
and selection types (\code{WS_PRIMARY} etc.).
Read the file for details.
Suggested usage is

\bcode\begin{verbatim}
>>> from stdwinevents import *
>>> 
\end{verbatim}\ecode

\section{Standard Module \sectcode{rect}}
\stmodindex{rect}

This module contains useful operations on rectangles.
A rectangle is defined as in module
\code{stdwin}:
a pair of points, where a point is a pair of integers.
For example, the rectangle

\bcode\begin{verbatim}
(10, 20), (90, 80)
\end{verbatim}\ecode

is a rectangle whose left, top, right and bottom edges are 10, 20, 90
and 80, respectively.
Note that the positive vertical axis points down (as in
\code{stdwin}).

The module defines the following objects:

\renewcommand{\indexsubitem}{(in module rect)}
\begin{excdesc}{error}
The exception raised by functions in this module when they detect an
error.
The exception argument is a string describing the problem in more
detail.
\end{excdesc}

\begin{datadesc}{empty}
The rectangle returned when some operations return an empty result.
This makes it possible to quickly check whether a result is empty:

\bcode\begin{verbatim}
>>> import rect
>>> r1 = (10, 20), (90, 80)
>>> r2 = (0, 0), (10, 20)
>>> r3 = rect.intersect([r1, r2])
>>> if r3 is rect.empty: print 'Empty intersection'
Empty intersection
>>> 
\end{verbatim}\ecode
\end{datadesc}

\begin{funcdesc}{is_empty}{r}
Returns true if the given rectangle is empty.
A rectangle
\code{(\var{left}, \var{top}), (\var{right}, \var{bottom})}
is empty if
\iftexi
\code{\var{left} >= \var{right}} or \code{\var{top} => \var{bottom}}.
\else
$\var{left} \geq \var{right}$ or $\var{top} \geq \var{bottom}$.
%%JHXXX{\em left~$\geq$~right} or {\em top~$\leq$~bottom}.
\fi
\end{funcdesc}

\begin{funcdesc}{intersect}{list}
Returns the intersection of all rectangles in the list argument.
It may also be called with a tuple argument.
Raises
\code{rect.error}
if the list is empty.
Returns
\code{rect.empty}
if the intersection of the rectangles is empty.
\end{funcdesc}

\begin{funcdesc}{union}{list}
Returns the smallest rectangle that contains all non-empty rectangles in
the list argument.
It may also be called with a tuple argument or with two or more
rectangles as arguments.
Returns
\code{rect.empty}
if the list is empty or all its rectangles are empty.
\end{funcdesc}

\begin{funcdesc}{pointinrect}{point\, rect}
Returns true if the point is inside the rectangle.
By definition, a point
\code{(\var{h}, \var{v})}
is inside a rectangle
\code{(\var{left}, \var{top}), (\var{right}, \var{bottom})} if
\iftexi
\code{\var{left} <= \var{h} < \var{right}} and
\code{\var{top} <= \var{v} < \var{bottom}}.
\else
$\var{left} \leq \var{h} < \var{right}$ and
$\var{top} \leq \var{v} < \var{bottom}$.
\fi
\end{funcdesc}

\begin{funcdesc}{inset}{rect\, \(dh\, dv\)}
Returns a rectangle that lies inside the
\code{rect}
argument by
\var{dh}
pixels horizontally
and
\var{dv}
pixels
vertically.
If
\var{dh}
or
\var{dv}
is negative, the result lies outside
\var{rect}.
\end{funcdesc}

\begin{funcdesc}{rect2geom}{rect}
Converts a rectangle to geometry representation:
\code{(\var{left}, \var{top}), (\var{width}, \var{height})}.
\end{funcdesc}

\begin{funcdesc}{geom2rect}{geom}
Converts a rectangle given in geometry representation back to the
standard rectangle representation
\code{(\var{left}, \var{top}), (\var{right}, \var{bottom})}.
\end{funcdesc}
		% STDWIN ONLY

\chapter{SGI IRIX Specific Services}

The modules described in this chapter provide interfaces to features
that are unique to SGI's IRIX operating system (versions 4 and 5).
			% SGI IRIX ONLY
\section{Built-in Module \sectcode{al}}
\bimodindex{al}

This module provides access to the audio facilities of the SGI Indy
and Indigo workstations.  See section 3A of the IRIX man pages for
details.  You'll need to read those man pages to understand what these
functions do!  Some of the functions are not available in IRIX
releases before 4.0.5.  Again, see the manual to check whether a
specific function is available on your platform.

All functions and methods defined in this module are equivalent to
the C functions with \samp{AL} prefixed to their name.

Symbolic constants from the C header file \file{<audio.h>} are defined
in the standard module \code{AL}, see below.

\strong{Warning:} the current version of the audio library may dump core
when bad argument values are passed rather than returning an error
status.  Unfortunately, since the precise circumstances under which
this may happen are undocumented and hard to check, the Python
interface can provide no protection against this kind of problems.
(One example is specifying an excessive queue size --- there is no
documented upper limit.)

The module defines the following functions:

\renewcommand{\indexsubitem}{(in module al)}

\begin{funcdesc}{openport}{name\, direction\optional{\, config}}
The name and direction arguments are strings.  The optional config
argument is a configuration object as returned by
\code{al.newconfig()}.  The return value is an \dfn{port object};
methods of port objects are described below.
\end{funcdesc}

\begin{funcdesc}{newconfig}{}
The return value is a new \dfn{configuration object}; methods of
configuration objects are described below.
\end{funcdesc}

\begin{funcdesc}{queryparams}{device}
The device argument is an integer.  The return value is a list of
integers containing the data returned by ALqueryparams().
\end{funcdesc}

\begin{funcdesc}{getparams}{device\, list}
The device argument is an integer.  The list argument is a list such
as returned by \code{queryparams}; it is modified in place (!).
\end{funcdesc}

\begin{funcdesc}{setparams}{device\, list}
The device argument is an integer.  The list argument is a list such
as returned by \code{al.queryparams}.
\end{funcdesc}

\subsection{Configuration Objects}

Configuration objects (returned by \code{al.newconfig()} have the
following methods:

\renewcommand{\indexsubitem}{(audio configuration object method)}

\begin{funcdesc}{getqueuesize}{}
Return the queue size.
\end{funcdesc}

\begin{funcdesc}{setqueuesize}{size}
Set the queue size.
\end{funcdesc}

\begin{funcdesc}{getwidth}{}
Get the sample width.
\end{funcdesc}

\begin{funcdesc}{setwidth}{width}
Set the sample width.
\end{funcdesc}

\begin{funcdesc}{getchannels}{}
Get the channel count.
\end{funcdesc}

\begin{funcdesc}{setchannels}{nchannels}
Set the channel count.
\end{funcdesc}

\begin{funcdesc}{getsampfmt}{}
Get the sample format.
\end{funcdesc}

\begin{funcdesc}{setsampfmt}{sampfmt}
Set the sample format.
\end{funcdesc}

\begin{funcdesc}{getfloatmax}{}
Get the maximum value for floating sample formats.
\end{funcdesc}

\begin{funcdesc}{setfloatmax}{floatmax}
Set the maximum value for floating sample formats.
\end{funcdesc}

\subsection{Port Objects}

Port objects (returned by \code{al.openport()} have the following
methods:

\renewcommand{\indexsubitem}{(audio port object method)}

\begin{funcdesc}{closeport}{}
Close the port.
\end{funcdesc}

\begin{funcdesc}{getfd}{}
Return the file descriptor as an int.
\end{funcdesc}

\begin{funcdesc}{getfilled}{}
Return the number of filled samples.
\end{funcdesc}

\begin{funcdesc}{getfillable}{}
Return the number of fillable samples.
\end{funcdesc}

\begin{funcdesc}{readsamps}{nsamples}
Read a number of samples from the queue, blocking if necessary.
Return the data as a string containing the raw data, (e.g., 2 bytes per
sample in big-endian byte order (high byte, low byte) if you have set
the sample width to 2 bytes).
\end{funcdesc}

\begin{funcdesc}{writesamps}{samples}
Write samples into the queue, blocking if necessary.  The samples are
encoded as described for the \code{readsamps} return value.
\end{funcdesc}

\begin{funcdesc}{getfillpoint}{}
Return the `fill point'.
\end{funcdesc}

\begin{funcdesc}{setfillpoint}{fillpoint}
Set the `fill point'.
\end{funcdesc}

\begin{funcdesc}{getconfig}{}
Return a configuration object containing the current configuration of
the port.
\end{funcdesc}

\begin{funcdesc}{setconfig}{config}
Set the configuration from the argument, a configuration object.
\end{funcdesc}

\begin{funcdesc}{getstatus}{list}
Get status information on last error.
\end{funcdesc}

\section{Standard Module \sectcode{AL}}
\nodename{AL (uppercase)}
\stmodindex{AL}

This module defines symbolic constants needed to use the built-in
module \code{al} (see above); they are equivalent to those defined in
the C header file \file{<audio.h>} except that the name prefix
\samp{AL_} is omitted.  Read the module source for a complete list of
the defined names.  Suggested use:

\bcode\begin{verbatim}
import al
from AL import *
\end{verbatim}\ecode

%\section{Built-in Module \sectcode{audio}}
\bimodindex{audio}

\strong{Note:} This module is obsolete, since the hardware to which it
interfaces is obsolete.  For audio on the Indigo or 4D/35, see
built-in module \code{al} above.

This module provides rudimentary access to the audio I/O device
\file{/dev/audio} on the Silicon Graphics Personal IRIS 4D/25;
see {\it audio}(7). It supports the following operations:

\renewcommand{\indexsubitem}{(in module audio)}
\begin{funcdesc}{setoutgain}{n}
Sets the output gain.
\iftexi
\code{0 <= \var{n} < 256}.
\else
$0 \leq \var{n} < 256$.
%%JHXXX Sets the output gain (0-255).
\fi
\end{funcdesc}

\begin{funcdesc}{getoutgain}{}
Returns the output gain.
\end{funcdesc}

\begin{funcdesc}{setrate}{n}
Sets the sampling rate: \code{1} = 32K/sec, \code{2} = 16K/sec,
\code{3} = 8K/sec.
\end{funcdesc}

\begin{funcdesc}{setduration}{n}
Sets the `sound duration' in units of 1/100 seconds.
\end{funcdesc}

\begin{funcdesc}{read}{n}
Reads a chunk of
\var{n}
sampled bytes from the audio input (line in or microphone).
The chunk is returned as a string of length n.
Each byte encodes one sample as a signed 8-bit quantity using linear
encoding.
This string can be converted to numbers using \code{chr2num()} described
below.
\end{funcdesc}

\begin{funcdesc}{write}{buf}
Writes a chunk of samples to the audio output (speaker).
\end{funcdesc}

These operations support asynchronous audio I/O:

\renewcommand{\indexsubitem}{(in module audio)}
\begin{funcdesc}{start_recording}{n}
Starts a second thread (a process with shared memory) that begins reading
\var{n}
bytes from the audio device.
The main thread immediately continues.
\end{funcdesc}

\begin{funcdesc}{wait_recording}{}
Waits for the second thread to finish and returns the data read.
\end{funcdesc}

\begin{funcdesc}{stop_recording}{}
Makes the second thread stop reading as soon as possible.
Returns the data read so far.
\end{funcdesc}

\begin{funcdesc}{poll_recording}{}
Returns true if the second thread has finished reading (so
\code{wait_recording()} would return the data without delay).
\end{funcdesc}

\begin{funcdesc}{start_playing}{}
\funcline{wait_playing}{}
\funcline{stop_playing}{}
\funcline{poll_playing}{}
\begin{sloppypar}
Similar but for output.
\code{stop_playing()}
returns a lower bound for the number of bytes actually played (not very
accurate).
\end{sloppypar}
\end{funcdesc}

The following operations do not affect the audio device but are
implemented in C for efficiency:

\renewcommand{\indexsubitem}{(in module audio)}
\begin{funcdesc}{amplify}{buf\, f1\, f2}
Amplifies a chunk of samples by a variable factor changing from
\code{\var{f1}/256} to \code{\var{f2}/256.}
Negative factors are allowed.
Resulting values that are to large to fit in a byte are clipped.         
\end{funcdesc}

\begin{funcdesc}{reverse}{buf}
Returns a chunk of samples backwards.
\end{funcdesc}

\begin{funcdesc}{add}{buf1\, buf2}
Bytewise adds two chunks of samples.
Bytes that exceed the range are clipped.
If one buffer is shorter, it is assumed to be padded with zeros.
\end{funcdesc}

\begin{funcdesc}{chr2num}{buf}
Converts a string of sampled bytes as returned by \code{read()} into
a list containing the numeric values of the samples.
\end{funcdesc}

\begin{funcdesc}{num2chr}{list}
\begin{sloppypar}
Converts a list as returned by
\code{chr2num()}
back to a buffer acceptable by
\code{write()}.
\end{sloppypar}
\end{funcdesc}

\section{Built-in Module \sectcode{cd}}
\bimodindex{cd}

This module provides an interface to the Silicon Graphics CD library.
It is available only on Silicon Graphics systems.

The way the library works is as follows.  A program opens the CD-ROM
device with \code{cd.open()} and creates a parser to parse the data
from the CD with \code{cd.createparser()}.  The object returned by
\code{cd.open()} can be used to read data from the CD, but also to get
status information for the CD-ROM device, and to get information about
the CD, such as the table of contents.  Data from the CD is passed to
the parser, which parses the frames, and calls any callback
functions that have previously been added.

An audio CD is divided into \dfn{tracks} or \dfn{programs} (the terms
are used interchangeably).  Tracks can be subdivided into
\dfn{indices}.  An audio CD contains a \dfn{table of contents} which
gives the starts of the tracks on the CD.  Index 0 is usually the
pause before the start of a track.  The start of the track as given by
the table of contents is normally the start of index 1.

Positions on a CD can be represented in two ways.  Either a frame
number or a tuple of three values, minutes, seconds and frames.  Most
functions use the latter representation.  Positions can be both
relative to the beginning of the CD, and to the beginning of the
track.

Module \code{cd} defines the following functions and constants:

\renewcommand{\indexsubitem}{(in module cd)}

\begin{funcdesc}{createparser}{}
Create and return an opaque parser object.  The methods of the parser
object are described below.
\end{funcdesc}

\begin{funcdesc}{msftoframe}{min\, sec\, frame}
Converts a \code{(minutes, seconds, frames)} triple representing time
in absolute time code into the corresponding CD frame number.
\end{funcdesc}

\begin{funcdesc}{open}{\optional{device\optional{\, mode}}}
Open the CD-ROM device.  The return value is an opaque player object;
methods of the player object are described below.  The device is the
name of the SCSI device file, e.g. /dev/scsi/sc0d4l0, or \code{None}.
If omited or \code{None}, the hardware inventory is consulted to
locate a CD-ROM drive.  The \code{mode}, if not omited, should be the
string 'r'.
\end{funcdesc}

The module defines the following variables:

\begin{datadesc}{error}
Exception raised on various errors.
\end{datadesc}

\begin{datadesc}{DATASIZE}
The size of one frame's worth of audio data.  This is the size of the
audio data as passed to the callback of type \code{audio}.
\end{datadesc}

\begin{datadesc}{BLOCKSIZE}
The size of one uninterpreted frame of audio data.
\end{datadesc}

The following variables are states as returned by \code{getstatus}:

\begin{datadesc}{READY}
The drive is ready for operation loaded with an audio CD.
\end{datadesc}

\begin{datadesc}{NODISC}
The drive does not have a CD loaded.
\end{datadesc}

\begin{datadesc}{CDROM}
The drive is loaded with a CD-ROM.  Subsequent play or read operations
will return I/O errors.
\end{datadesc}

\begin{datadesc}{ERROR}
An error aoocurred while trying to read the disc or its table of
contents.
\end{datadesc}

\begin{datadesc}{PLAYING}
The drive is in CD player mode playing an audio CD through its audio
jacks.
\end{datadesc}

\begin{datadesc}{PAUSED}
The drive is in CD layer mode with play paused.
\end{datadesc}

\begin{datadesc}{STILL}
The equivalent of \code{PAUSED} on older (non 3301) model Toshiba
CD-ROM drives.  Such drives have never been shipped by SGI.
\end{datadesc}

\begin{datadesc}{audio}
\dataline{pnum}
\dataline{index}
\dataline{ptime}
\dataline{atime}
\dataline{catalog}
\dataline{ident}
\dataline{control}
Integer constants describing the various types of parser callbacks
that can be set by the \code{addcallback()} method of CD parser
objects (see below).
\end{datadesc}

Player objects (returned by \code{cd.open()}) have the following
methods:

\renewcommand{\indexsubitem}{(CD player object method)}

\begin{funcdesc}{allowremoval}{}
Unlocks the eject button on the CD-ROM drive permitting the user to
eject the caddy if desired.
\end{funcdesc}

\begin{funcdesc}{bestreadsize}{}
Returns the best value to use for the \code{num_frames} parameter of
the \code{readda} method.  Best is defined as the value that permits a
continuous flow of data from the CD-ROM drive.
\end{funcdesc}

\begin{funcdesc}{close}{}
Frees the resources associated with the player object.  After calling
\code{close}, the methods of the object should no longer be used.
\end{funcdesc}

\begin{funcdesc}{eject}{}
Ejects the caddy from the CD-ROM drive.
\end{funcdesc}

\begin{funcdesc}{getstatus}{}
Returns information pertaining to the current state of the CD-ROM
drive.  The returned information is a tuple with the following values:
\code{state}, \code{track}, \code{rtime}, \code{atime}, \code{ttime},
\code{first}, \code{last}, \code{scsi_audio}, \code{cur_block}.
\code{rtime} is the time relative to the start of the current track;
\code{atime} is the time relative to the beginning of the disc;
\code{ttime} is the total time on the disc.  For more information on
the meaning of the values, see the manual for CDgetstatus.
The value of \code{state} is one of the following: \code{cd.ERROR},
\code{cd.NODISC}, \code{cd.READY}, \code{cd.PLAYING},
\code{cd.PAUSED}, \code{cd.STILL}, or \code{cd.CDROM}.
\end{funcdesc}

\begin{funcdesc}{gettrackinfo}{track}
Returns information about the specified track.  The returned
information is a tuple consisting of two elements, the start time of
the track and the duration of the track.
\end{funcdesc}

\begin{funcdesc}{msftoblock}{min\, sec\, frame}
Converts a minutes, seconds, frames triple representing a time in
absolute time code into the corresponding logical block number for the
given CD-ROM drive.  You should use \code{cd.msftoframe()} rather than
\code{msftoblock()} for comparing times.  The logical block number
differs from the frame number by an offset required by certain CD-ROM
drives.
\end{funcdesc}

\begin{funcdesc}{play}{start\, play}
Starts playback of an audio CD in the CD-ROM drive at the specified
track.  The audio output appears on the CD-ROM drive's headphone and
audio jacks (if fitted).  Play stops at the end of the disc.
\code{start} is the number of the track at which to start playing the
CD; if \code{play} is 0, the CD will be set to an initial paused
state.  The method \code{togglepause()} can then be used to commence
play.
\end{funcdesc}

\begin{funcdesc}{playabs}{min\, sec\, frame\, play}
Like \code{play()}, except that the start is given in minutes,
seconds, frames instead of a track number.
\end{funcdesc}

\begin{funcdesc}{playtrack}{start\, play}
Like \code{play()}, except that playing stops at the end of the track.
\end{funcdesc}

\begin{funcdesc}{playtrackabs}{track\, min\, sec\, frame\, play}
Like \code{play()}, except that playing begins at the spcified
absolute time and ends at the end of the specified track.
\end{funcdesc}

\begin{funcdesc}{preventremoval}{}
Locks the eject button on the CD-ROM drive thus preventing the user
from arbitrarily ejecting the caddy.
\end{funcdesc}

\begin{funcdesc}{readda}{num_frames}
Reads the specified number of frames from an audio CD mounted in the
CD-ROM drive.  The return value is a string representing the audio
frames.  This string can be passed unaltered to the \code{parseframe}
method of the parser object.
\end{funcdesc}

\begin{funcdesc}{seek}{min\, sec\, frame}
Sets the pointer that indicates the starting point of the next read of
digital audio data from a CD-ROM.  The pointer is set to an absolute
time code location specified in minutes, seconds, and frames.  The
return value is the logical block number to which the pointer has been
set.
\end{funcdesc}

\begin{funcdesc}{seekblock}{block}
Sets the pointer that indicates the starting point of the next read of
digital audio data from a CD-ROM.  The pointer is set to the specified
logical block number.  The return value is the logical block number to
which the pointer has been set.
\end{funcdesc}

\begin{funcdesc}{seektrack}{track}
Sets the pointer that indicates the starting point of the next read of
digital audio data from a CD-ROM.  The pointer is set to the specified
track.  The return value is the logical block number to which the
pointer has been set.
\end{funcdesc}

\begin{funcdesc}{stop}{}
Stops the current playing operation.
\end{funcdesc}

\begin{funcdesc}{togglepause}{}
Pauses the CD if it is playing, and makes it play if it is paused.
\end{funcdesc}

Parser objects (returned by \code{cd.createparser()}) have the
following methods:

\renewcommand{\indexsubitem}{(CD parser object method)}

\begin{funcdesc}{addcallback}{type\, func\, arg}
Adds a callback for the parser.  The parser has callbacks for eight
different types of data in the digital audio data stream.  Constants
for these types are defined at the \code{cd} module level (see above).
The callback is called as follows: \code{func(arg, type, data)}, where
\code{arg} is the user supplied argument, \code{type} is the
particular type of callback, and \code{data} is the data returned for
this \code{type} of callback.  The type of the data depends on the
\code{type} of callback as follows:
\begin{description}
\item[\code{cd.audio}: ]
The argument is a string which can be passed unmodified to
\code{al.writesamps()}.
\item[\code{cd.pnum}: ]
The argument is an integer giving the program (track) number.
\item[\code{cd.index}: ]
The argument is an integer giving the index number.
\item[\code{cd.ptime}: ]
The argument is a tuple consisting of the program time in minutes,
seconds, and frames.
\item[\code{cd.atime}: ]
The argument is a tuple consisting of the absolute time in minutes,
seconds, and frames.
\item[\code{cd.catalog}: ]
The argument is a string of 13 characters, giving the catalog number
of the CD.
\item[\code{cd.ident}: ]
The argument is a string of 12 characters, giving the ISRC
identification number of the recording.  The string consists of two
characters country code, three characters owner code, two characters
giving the year, and five characters giving a serial number.
\item[\code{cd.control}: ]
The argument is an integer giving the control bits from the CD subcode
data.
\end{description}
\end{funcdesc}

\begin{funcdesc}{deleteparser}{}
Deletes the parser and frees the memory it was using.  The object
should not be used after this call.  This call is done automatically
when the last reference to the object is removed.
\end{funcdesc}

\begin{funcdesc}{parseframe}{frame}
Parses one or more frames of digital audio data from a CD such as
returned by \code{readda}.  It determines which subcodes are present
in the data.  If these subcodes have changed since the last frame,
then \code{parseframe} executes a callback of the appropriate type
passing to it the subcode data found in the frame.
Unlike the C function, more than one frame of digital audio data can
be passed to this method.
\end{funcdesc}

\begin{funcdesc}{removecallback}{type}
Removes the callback for the given \code{type}.
\end{funcdesc}

\begin{funcdesc}{resetparser}{}
Resets the fields of the parser used for tracking subcodes to an
initial state.  \code{resetparser} should be called after the disc has
been changed.
\end{funcdesc}

\section{Built-in Module \sectcode{fl}}
\bimodindex{fl}

This module provides an interface to the FORMS Library by Mark
Overmars.  The source for the library can be retrieved by anonymous
ftp from host \samp{ftp.cs.ruu.nl}, directory \file{SGI/FORMS}.  It
was last tested with version 2.0b.

Most functions are literal translations of their C equivalents,
dropping the initial \samp{fl_} from their name.  Constants used by
the library are defined in module \code{FL} described below.

The creation of objects is a little different in Python than in C:
instead of the `current form' maintained by the library to which new
FORMS objects are added, all functions that add a FORMS object to a
form are methods of the Python object representing the form.
Consequently, there are no Python equivalents for the C functions
\code{fl_addto_form} and \code{fl_end_form}, and the equivalent of
\code{fl_bgn_form} is called \code{fl.make_form}.

Watch out for the somewhat confusing terminology: FORMS uses the word
\dfn{object} for the buttons, sliders etc. that you can place in a form.
In Python, `object' means any value.  The Python interface to FORMS
introduces two new Python object types: form objects (representing an
entire form) and FORMS objects (representing one button, slider etc.).
Hopefully this isn't too confusing...

There are no `free objects' in the Python interface to FORMS, nor is
there an easy way to add object classes written in Python.  The FORMS
interface to GL event handling is available, though, so you can mix
FORMS with pure GL windows.

\strong{Please note:} importing \code{fl} implies a call to the GL function
\code{foreground()} and to the FORMS routine \code{fl_init()}.

\subsection{Functions Defined in Module \sectcode{fl}}
\nodename{FL Functions}

Module \code{fl} defines the following functions.  For more information
about what they do, see the description of the equivalent C function
in the FORMS documentation:

\renewcommand{\indexsubitem}{(in module fl)}
\begin{funcdesc}{make_form}{type\, width\, height}
Create a form with given type, width and height.  This returns a
\dfn{form} object, whose methods are described below.
\end{funcdesc}

\begin{funcdesc}{do_forms}{}
The standard FORMS main loop.  Returns a Python object representing
the FORMS object needing interaction, or the special value
\code{FL.EVENT}.
\end{funcdesc}

\begin{funcdesc}{check_forms}{}
Check for FORMS events.  Returns what \code{do_forms} above returns,
or \code{None} if there is no event that immediately needs
interaction.
\end{funcdesc}

\begin{funcdesc}{set_event_call_back}{function}
Set the event callback function.
\end{funcdesc}

\begin{funcdesc}{set_graphics_mode}{rgbmode\, doublebuffering}
Set the graphics modes.
\end{funcdesc}

\begin{funcdesc}{get_rgbmode}{}
Return the current rgb mode.  This is the value of the C global
variable \code{fl_rgbmode}.
\end{funcdesc}

\begin{funcdesc}{show_message}{str1\, str2\, str3}
Show a dialog box with a three-line message and an OK button.
\end{funcdesc}

\begin{funcdesc}{show_question}{str1\, str2\, str3}
Show a dialog box with a three-line message and YES and NO buttons.
It returns \code{1} if the user pressed YES, \code{0} if NO.
\end{funcdesc}

\begin{funcdesc}{show_choice}{str1\, str2\, str3\, but1\optional{\, but2\,
but3}}
Show a dialog box with a three-line message and up to three buttons.
It returns the number of the button clicked by the user
(\code{1}, \code{2} or \code{3}).
\end{funcdesc}

\begin{funcdesc}{show_input}{prompt\, default}
Show a dialog box with a one-line prompt message and text field in
which the user can enter a string.  The second argument is the default
input string.  It returns the string value as edited by the user.
\end{funcdesc}

\begin{funcdesc}{show_file_selector}{message\, directory\, pattern\, default}
Show a dialog box in which the user can select a file.  It returns
the absolute filename selected by the user, or \code{None} if the user
presses Cancel.
\end{funcdesc}

\begin{funcdesc}{get_directory}{}
\funcline{get_pattern}{}
\funcline{get_filename}{}
These functions return the directory, pattern and filename (the tail
part only) selected by the user in the last \code{show_file_selector}
call.
\end{funcdesc}

\begin{funcdesc}{qdevice}{dev}
\funcline{unqdevice}{dev}
\funcline{isqueued}{dev}
\funcline{qtest}{}
\funcline{qread}{}
%\funcline{blkqread}{?}
\funcline{qreset}{}
\funcline{qenter}{dev\, val}
\funcline{get_mouse}{}
\funcline{tie}{button\, valuator1\, valuator2}
These functions are the FORMS interfaces to the corresponding GL
functions.  Use these if you want to handle some GL events yourself
when using \code{fl.do_events}.  When a GL event is detected that
FORMS cannot handle, \code{fl.do_forms()} returns the special value
\code{FL.EVENT} and you should call \code{fl.qread()} to read the
event from the queue.  Don't use the equivalent GL functions!
\end{funcdesc}

\begin{funcdesc}{color}{}
\funcline{mapcolor}{}
\funcline{getmcolor}{}
See the description in the FORMS documentation of \code{fl_color},
\code{fl_mapcolor} and \code{fl_getmcolor}.
\end{funcdesc}

\subsection{Form Objects}

Form objects (returned by \code{fl.make_form()} above) have the
following methods.  Each method corresponds to a C function whose name
is prefixed with \samp{fl_}; and whose first argument is a form
pointer; please refer to the official FORMS documentation for
descriptions.

All the \samp{add_{\rm \ldots}} functions return a Python object representing
the FORMS object.  Methods of FORMS objects are described below.  Most
kinds of FORMS object also have some methods specific to that kind;
these methods are listed here.

\begin{flushleft}
\renewcommand{\indexsubitem}{(form object method)}
\begin{funcdesc}{show_form}{placement\, bordertype\, name}
  Show the form.
\end{funcdesc}

\begin{funcdesc}{hide_form}{}
  Hide the form.
\end{funcdesc}

\begin{funcdesc}{redraw_form}{}
  Redraw the form.
\end{funcdesc}

\begin{funcdesc}{set_form_position}{x\, y}
Set the form's position.
\end{funcdesc}

\begin{funcdesc}{freeze_form}{}
Freeze the form.
\end{funcdesc}

\begin{funcdesc}{unfreeze_form}{}
  Unfreeze the form.
\end{funcdesc}

\begin{funcdesc}{activate_form}{}
  Activate the form.
\end{funcdesc}

\begin{funcdesc}{deactivate_form}{}
  Deactivate the form.
\end{funcdesc}

\begin{funcdesc}{bgn_group}{}
  Begin a new group of objects; return a group object.
\end{funcdesc}

\begin{funcdesc}{end_group}{}
  End the current group of objects.
\end{funcdesc}

\begin{funcdesc}{find_first}{}
  Find the first object in the form.
\end{funcdesc}

\begin{funcdesc}{find_last}{}
  Find the last object in the form.
\end{funcdesc}

%---

\begin{funcdesc}{add_box}{type\, x\, y\, w\, h\, name}
Add a box object to the form.
No extra methods.
\end{funcdesc}

\begin{funcdesc}{add_text}{type\, x\, y\, w\, h\, name}
Add a text object to the form.
No extra methods.
\end{funcdesc}

%\begin{funcdesc}{add_bitmap}{type\, x\, y\, w\, h\, name}
%Add a bitmap object to the form.
%\end{funcdesc}

\begin{funcdesc}{add_clock}{type\, x\, y\, w\, h\, name}
Add a clock object to the form. \\
Method:
\code{get_clock}.
\end{funcdesc}

%---

\begin{funcdesc}{add_button}{type\, x\, y\, w\, h\,  name}
Add a button object to the form. \\
Methods:
\code{get_button},
\code{set_button}.
\end{funcdesc}

\begin{funcdesc}{add_lightbutton}{type\, x\, y\, w\, h\, name}
Add a lightbutton object to the form. \\
Methods:
\code{get_button},
\code{set_button}.
\end{funcdesc}

\begin{funcdesc}{add_roundbutton}{type\, x\, y\, w\, h\, name}
Add a roundbutton object to the form. \\
Methods:
\code{get_button},
\code{set_button}.
\end{funcdesc}

%---

\begin{funcdesc}{add_slider}{type\, x\, y\, w\, h\, name}
Add a slider object to the form. \\
Methods:
\code{set_slider_value},
\code{get_slider_value},
\code{set_slider_bounds},
\code{get_slider_bounds},
\code{set_slider_return},
\code{set_slider_size},
\code{set_slider_precision},
\code{set_slider_step}.
\end{funcdesc}

\begin{funcdesc}{add_valslider}{type\, x\, y\, w\, h\, name}
Add a valslider object to the form. \\
Methods:
\code{set_slider_value},
\code{get_slider_value},
\code{set_slider_bounds},
\code{get_slider_bounds},
\code{set_slider_return},
\code{set_slider_size},
\code{set_slider_precision},
\code{set_slider_step}.
\end{funcdesc}

\begin{funcdesc}{add_dial}{type\, x\, y\, w\, h\, name}
Add a dial object to the form. \\
Methods:
\code{set_dial_value},
\code{get_dial_value},
\code{set_dial_bounds},
\code{get_dial_bounds}.
\end{funcdesc}

\begin{funcdesc}{add_positioner}{type\, x\, y\, w\, h\, name}
Add a positioner object to the form. \\
Methods:
\code{set_positioner_xvalue},
\code{set_positioner_yvalue},
\code{set_positioner_xbounds},
\code{set_positioner_ybounds},
\code{get_positioner_xvalue},
\code{get_positioner_yvalue},
\code{get_positioner_xbounds},
\code{get_positioner_ybounds}.
\end{funcdesc}

\begin{funcdesc}{add_counter}{type\, x\, y\, w\, h\, name}
Add a counter object to the form. \\
Methods:
\code{set_counter_value},
\code{get_counter_value},
\code{set_counter_bounds},
\code{set_counter_step},
\code{set_counter_precision},
\code{set_counter_return}.
\end{funcdesc}

%---

\begin{funcdesc}{add_input}{type\, x\, y\, w\, h\, name}
Add a input object to the form. \\
Methods:
\code{set_input},
\code{get_input},
\code{set_input_color},
\code{set_input_return}.
\end{funcdesc}

%---

\begin{funcdesc}{add_menu}{type\, x\, y\, w\, h\, name}
Add a menu object to the form. \\
Methods:
\code{set_menu},
\code{get_menu},
\code{addto_menu}.
\end{funcdesc}

\begin{funcdesc}{add_choice}{type\, x\, y\, w\, h\, name}
Add a choice object to the form. \\
Methods:
\code{set_choice},
\code{get_choice},
\code{clear_choice},
\code{addto_choice},
\code{replace_choice},
\code{delete_choice},
\code{get_choice_text},
\code{set_choice_fontsize},
\code{set_choice_fontstyle}.
\end{funcdesc}

\begin{funcdesc}{add_browser}{type\, x\, y\, w\, h\, name}
Add a browser object to the form. \\
Methods:
\code{set_browser_topline},
\code{clear_browser},
\code{add_browser_line},
\code{addto_browser},
\code{insert_browser_line},
\code{delete_browser_line},
\code{replace_browser_line},
\code{get_browser_line},
\code{load_browser},
\code{get_browser_maxline},
\code{select_browser_line},
\code{deselect_browser_line},
\code{deselect_browser},
\code{isselected_browser_line},
\code{get_browser},
\code{set_browser_fontsize},
\code{set_browser_fontstyle},
\code{set_browser_specialkey}.
\end{funcdesc}

%---

\begin{funcdesc}{add_timer}{type\, x\, y\, w\, h\, name}
Add a timer object to the form. \\
Methods:
\code{set_timer},
\code{get_timer}.
\end{funcdesc}
\end{flushleft}

Form objects have the following data attributes; see the FORMS
documentation:

\begin{tableiii}{|l|c|l|}{code}{Name}{Type}{Meaning}
  \lineiii{window}{int (read-only)}{GL window id}
  \lineiii{w}{float}{form width}
  \lineiii{h}{float}{form height}
  \lineiii{x}{float}{form x origin}
  \lineiii{y}{float}{form y origin}
  \lineiii{deactivated}{int}{nonzero if form is deactivated}
  \lineiii{visible}{int}{nonzero if form is visible}
  \lineiii{frozen}{int}{nonzero if form is frozen}
  \lineiii{doublebuf}{int}{nonzero if double buffering on}
\end{tableiii}

\subsection{FORMS Objects}

Besides methods specific to particular kinds of FORMS objects, all
FORMS objects also have the following methods:

\renewcommand{\indexsubitem}{(FORMS object method)}
\begin{funcdesc}{set_call_back}{function\, argument}
Set the object's callback function and argument.  When the object
needs interaction, the callback function will be called with two
arguments: the object, and the callback argument.  (FORMS objects
without a callback function are returned by \code{fl.do_forms()} or
\code{fl.check_forms()} when they need interaction.)  Call this method
without arguments to remove the callback function.
\end{funcdesc}

\begin{funcdesc}{delete_object}{}
  Delete the object.
\end{funcdesc}

\begin{funcdesc}{show_object}{}
  Show the object.
\end{funcdesc}

\begin{funcdesc}{hide_object}{}
  Hide the object.
\end{funcdesc}

\begin{funcdesc}{redraw_object}{}
  Redraw the object.
\end{funcdesc}

\begin{funcdesc}{freeze_object}{}
  Freeze the object.
\end{funcdesc}

\begin{funcdesc}{unfreeze_object}{}
  Unfreeze the object.
\end{funcdesc}

%\begin{funcdesc}{handle_object}{} XXX
%\end{funcdesc}

%\begin{funcdesc}{handle_object_direct}{} XXX
%\end{funcdesc}

FORMS objects have these data attributes; see the FORMS documentation:

\begin{tableiii}{|l|c|l|}{code}{Name}{Type}{Meaning}
  \lineiii{objclass}{int (read-only)}{object class}
  \lineiii{type}{int (read-only)}{object type}
  \lineiii{boxtype}{int}{box type}
  \lineiii{x}{float}{x origin}
  \lineiii{y}{float}{y origin}
  \lineiii{w}{float}{width}
  \lineiii{h}{float}{height}
  \lineiii{col1}{int}{primary color}
  \lineiii{col2}{int}{secondary color}
  \lineiii{align}{int}{alignment}
  \lineiii{lcol}{int}{label color}
  \lineiii{lsize}{float}{label font size}
  \lineiii{label}{string}{label string}
  \lineiii{lstyle}{int}{label style}
  \lineiii{pushed}{int (read-only)}{(see FORMS docs)}
  \lineiii{focus}{int (read-only)}{(see FORMS docs)}
  \lineiii{belowmouse}{int (read-only)}{(see FORMS docs)}
  \lineiii{frozen}{int (read-only)}{(see FORMS docs)}
  \lineiii{active}{int (read-only)}{(see FORMS docs)}
  \lineiii{input}{int (read-only)}{(see FORMS docs)}
  \lineiii{visible}{int (read-only)}{(see FORMS docs)}
  \lineiii{radio}{int (read-only)}{(see FORMS docs)}
  \lineiii{automatic}{int (read-only)}{(see FORMS docs)}
\end{tableiii}

\section{Standard Module \sectcode{FL}}
\nodename{FL (uppercase)}
\stmodindex{FL}

This module defines symbolic constants needed to use the built-in
module \code{fl} (see above); they are equivalent to those defined in
the C header file \file{<forms.h>} except that the name prefix
\samp{FL_} is omitted.  Read the module source for a complete list of
the defined names.  Suggested use:

\bcode\begin{verbatim}
import fl
from FL import *
\end{verbatim}\ecode

\section{Standard Module \sectcode{flp}}
\stmodindex{flp}

This module defines functions that can read form definitions created
by the `form designer' (\code{fdesign}) program that comes with the
FORMS library (see module \code{fl} above).

For now, see the file \file{flp.doc} in the Python library source
directory for a description.

XXX A complete description should be inserted here!

\section{Built-in Module \sectcode{fm}}
\bimodindex{fm}

This module provides access to the IRIS {\em Font Manager} library.
It is available only on Silicon Graphics machines.
See also: 4Sight User's Guide, Section 1, Chapter 5: Using the IRIS
Font Manager.

This is not yet a full interface to the IRIS Font Manager.
Among the unsupported features are: matrix operations; cache
operations; character operations (use string operations instead); some
details of font info; individual glyph metrics; and printer matching.

It supports the following operations:

\renewcommand{\indexsubitem}{(in module fm)}
\begin{funcdesc}{init}{}
Initialization function.
Calls \code{fminit()}.
It is normally not necessary to call this function, since it is called
automatically the first time the \code{fm} module is imported.
\end{funcdesc}

\begin{funcdesc}{findfont}{fontname}
Return a font handle object.
Calls \code{fmfindfont(\var{fontname})}.
\end{funcdesc}

\begin{funcdesc}{enumerate}{}
Returns a list of available font names.
This is an interface to \code{fmenumerate()}.
\end{funcdesc}

\begin{funcdesc}{prstr}{string}
Render a string using the current font (see the \code{setfont()} font
handle method below).
Calls \code{fmprstr(\var{string})}.
\end{funcdesc}

\begin{funcdesc}{setpath}{string}
Sets the font search path.
Calls \code{fmsetpath(string)}.
(XXX Does not work!?!)
\end{funcdesc}

\begin{funcdesc}{fontpath}{}
Returns the current font search path.
\end{funcdesc}

Font handle objects support the following operations:

\renewcommand{\indexsubitem}{(font handle method)}
\begin{funcdesc}{scalefont}{factor}
Returns a handle for a scaled version of this font.
Calls \code{fmscalefont(\var{fh}, \var{factor})}.
\end{funcdesc}

\begin{funcdesc}{setfont}{}
Makes this font the current font.
Note: the effect is undone silently when the font handle object is
deleted.
Calls \code{fmsetfont(\var{fh})}.
\end{funcdesc}

\begin{funcdesc}{getfontname}{}
Returns this font's name.
Calls \code{fmgetfontname(\var{fh})}.
\end{funcdesc}

\begin{funcdesc}{getcomment}{}
Returns the comment string associated with this font.
Raises an exception if there is none.
Calls \code{fmgetcomment(\var{fh})}.
\end{funcdesc}

\begin{funcdesc}{getfontinfo}{}
Returns a tuple giving some pertinent data about this font.
This is an interface to \code{fmgetfontinfo()}.
The returned tuple contains the following numbers:
{\tt(\var{printermatched}, \var{fixed_width}, \var{xorig}, \var{yorig},
\var{xsize}, \var{ysize}, \var{height}, \var{nglyphs})}.
\end{funcdesc}

\begin{funcdesc}{getstrwidth}{string}
Returns the width, in pixels, of the string when drawn in this font.
Calls \code{fmgetstrwidth(\var{fh}, \var{string})}.
\end{funcdesc}

\section{Built-in Module \sectcode{gl}}
\bimodindex{gl}

This module provides access to the Silicon Graphics
{\em Graphics Library}.
It is available only on Silicon Graphics machines.

\strong{Warning:}
Some illegal calls to the GL library cause the Python interpreter to dump
core.
In particular, the use of most GL calls is unsafe before the first
window is opened.

The module is too large to document here in its entirety, but the
following should help you to get started.
The parameter conventions for the C functions are translated to Python as
follows:

\begin{itemize}
\item
All (short, long, unsigned) int values are represented by Python
integers.
\item
All float and double values are represented by Python floating point
numbers.
In most cases, Python integers are also allowed.
\item
All arrays are represented by one-dimensional Python lists.
In most cases, tuples are also allowed.
\item
\begin{sloppypar}
All string and character arguments are represented by Python strings,
for instance,
\code{winopen('Hi There!')}
and
\code{rotate(900, 'z')}.
\end{sloppypar}
\item
All (short, long, unsigned) integer arguments or return values that are
only used to specify the length of an array argument are omitted.
For example, the C call

\bcode\begin{verbatim}
lmdef(deftype, index, np, props)
\end{verbatim}\ecode

is translated to Python as

\bcode\begin{verbatim}
lmdef(deftype, index, props)
\end{verbatim}\ecode

\item
Output arguments are omitted from the argument list; they are
transmitted as function return values instead.
If more than one value must be returned, the return value is a tuple.
If the C function has both a regular return value (that is not omitted
because of the previous rule) and an output argument, the return value
comes first in the tuple.
Examples: the C call

\bcode\begin{verbatim}
getmcolor(i, &red, &green, &blue)
\end{verbatim}\ecode

is translated to Python as

\bcode\begin{verbatim}
red, green, blue = getmcolor(i)
\end{verbatim}\ecode

\end{itemize}

The following functions are non-standard or have special argument
conventions:

\renewcommand{\indexsubitem}{(in module gl)}
\begin{funcdesc}{varray}{argument}
%JHXXX the argument-argument added
Equivalent to but faster than a number of
\code{v3d()}
calls.
The \var{argument} is a list (or tuple) of points.
Each point must be a tuple of coordinates
\code{(\var{x}, \var{y}, \var{z})} or \code{(\var{x}, \var{y})}.
The points may be 2- or 3-dimensional but must all have the
same dimension.
Float and int values may be mixed however.
The points are always converted to 3D double precision points
by assuming \code{\var{z} = 0.0} if necessary (as indicated in the man page),
and for each point
\code{v3d()}
is called.
\end{funcdesc}

\begin{funcdesc}{nvarray}{}
Equivalent to but faster than a number of
\code{n3f}
and
\code{v3f}
calls.
The argument is an array (list or tuple) of pairs of normals and points.
Each pair is a tuple of a point and a normal for that point.
Each point or normal must be a tuple of coordinates
\code{(\var{x}, \var{y}, \var{z})}.
Three coordinates must be given.
Float and int values may be mixed.
For each pair,
\code{n3f()}
is called for the normal, and then
\code{v3f()}
is called for the point.
\end{funcdesc}

\begin{funcdesc}{vnarray}{}
Similar to 
\code{nvarray()}
but the pairs have the point first and the normal second.
\end{funcdesc}

\begin{funcdesc}{nurbssurface}{s_k\, t_k\, ctl\, s_ord\, t_ord\, type}
% XXX s_k[], t_k[], ctl[][]
%\itembreak
Defines a nurbs surface.
The dimensions of
\code{\var{ctl}[][]}
are computed as follows:
\code{[len(\var{s_k}) - \var{s_ord}]},
\code{[len(\var{t_k}) - \var{t_ord}]}.
\end{funcdesc}

\begin{funcdesc}{nurbscurve}{knots\, ctlpoints\, order\, type}
Defines a nurbs curve.
The length of ctlpoints is
\code{len(\var{knots}) - \var{order}}.
\end{funcdesc}

\begin{funcdesc}{pwlcurve}{points\, type}
Defines a piecewise-linear curve.
\var{points}
is a list of points.
\var{type}
must be
\code{N_ST}.
\end{funcdesc}

\begin{funcdesc}{pick}{n}
\funcline{select}{n}
The only argument to these functions specifies the desired size of the
pick or select buffer.
\end{funcdesc}

\begin{funcdesc}{endpick}{}
\funcline{endselect}{}
These functions have no arguments.
They return a list of integers representing the used part of the
pick/select buffer.
No method is provided to detect buffer overrun.
\end{funcdesc}

Here is a tiny but complete example GL program in Python:

\bcode\begin{verbatim}
import gl, GL, time

def main():
    gl.foreground()
    gl.prefposition(500, 900, 500, 900)
    w = gl.winopen('CrissCross')
    gl.ortho2(0.0, 400.0, 0.0, 400.0)
    gl.color(GL.WHITE)
    gl.clear()
    gl.color(GL.RED)
    gl.bgnline()
    gl.v2f(0.0, 0.0)
    gl.v2f(400.0, 400.0)
    gl.endline()
    gl.bgnline()
    gl.v2f(400.0, 0.0)
    gl.v2f(0.0, 400.0)
    gl.endline()
    time.sleep(5)

main()
\end{verbatim}\ecode

\section{Standard Modules \sectcode{GL} and \sectcode{DEVICE}}
\nodename{GL and DEVICE}
\stmodindex{GL}
\stmodindex{DEVICE}

These modules define the constants used by the Silicon Graphics
{\em Graphics Library}
that C programmers find in the header files
\file{<gl/gl.h>}
and
\file{<gl/device.h>}.
Read the module source files for details.

\section{Built-in Module \sectcode{imgfile}}
\bimodindex{imgfile}

The imgfile module allows python programs to access SGI imglib image
files (also known as \file{.rgb} files).  The module is far from
complete, but is provided anyway since the functionality that there is
is enough in some cases.  Currently, colormap files are not supported.

The module defines the following variables and functions:

\renewcommand{\indexsubitem}{(in module imgfile)}
\begin{excdesc}{error}
This exception is raised on all errors, such as unsupported file type, etc.
\end{excdesc}

\begin{funcdesc}{getsizes}{file}
This function returns a tuple \code{(\var{x}, \var{y}, \var{z})} where
\var{x} and \var{y} are the size of the image in pixels and
\var{z} is the number of
bytes per pixel. Only 3 byte RGB pixels and 1 byte greyscale pixels
are currently supported.
\end{funcdesc}

\begin{funcdesc}{read}{file}
This function reads and decodes the image on the specified file, and
returns it as a python string. The string has either 1 byte greyscale
pixels or 4 byte RGBA pixels. The bottom left pixel is the first in
the string. This format is suitable to pass to \code{gl.lrectwrite},
for instance.
\end{funcdesc}

\begin{funcdesc}{readscaled}{file\, x\, y\, filter\optional{\, blur}}
This function is identical to read but it returns an image that is
scaled to the given \var{x} and \var{y} sizes. If the \var{filter} and
\var{blur} parameters are omitted scaling is done by
simply dropping or duplicating pixels, so the result will be less than
perfect, especially for computer-generated images.

Alternatively, you can specify a filter to use to smoothen the image
after scaling. The filter forms supported are \code{'impulse'},
\code{'box'}, \code{'triangle'}, \code{'quadratic'} and
\code{'gaussian'}. If a filter is specified \var{blur} is an optional
parameter specifying the blurriness of the filter. It defaults to \code{1.0}.

\code{readscaled} makes no
attempt to keep the aspect ratio correct, so that is the users'
responsibility.
\end{funcdesc}

\begin{funcdesc}{ttob}{flag}
This function sets a global flag which defines whether the scan lines
of the image are read or written from bottom to top (flag is zero,
compatible with SGI GL) or from top to bottom(flag is one,
compatible with X).  The default is zero.
\end{funcdesc}

\begin{funcdesc}{write}{file\, data\, x\, y\, z}
This function writes the RGB or greyscale data in \var{data} to image
file \var{file}. \var{x} and \var{y} give the size of the image,
\var{z} is 1 for 1 byte greyscale images or 3 for RGB images (which are
stored as 4 byte values of which only the lower three bytes are used).
These are the formats returned by \code{gl.lrectread}.
\end{funcdesc}

%\section{Standard Module \sectcode{panel}}
\stmodindex{panel}

\strong{Please note:} The FORMS library, to which the \code{fl} module described
above interfaces, is a simpler and more accessible user interface
library for use with GL than the Panel Module (besides also being by a
Dutch author).

This module should be used instead of the built-in module
\code{pnl}
to interface with the
{\em Panel Library}.

The module is too large to document here in its entirety.
One interesting function:

\renewcommand{\indexsubitem}{(in module panel)}
\begin{funcdesc}{defpanellist}{filename}
Parses a panel description file containing S-expressions written by the
{\em Panel Editor}
that accompanies the Panel Library and creates the described panels.
It returns a list of panel objects.
\end{funcdesc}

\strong{Warning:}
the Python interpreter will dump core if you don't create a GL window
before calling
\code{panel.mkpanel()}
or
\code{panel.defpanellist()}.

\section{Standard Module \sectcode{panelparser}}
\stmodindex{panelparser}

This module defines a self-contained parser for S-expressions as output
by the Panel Editor (which is written in Scheme so it can't help writing
S-expressions).
The relevant function is
\code{panelparser.parse_file(\var{file})}
which has a file object (not a filename!) as argument and returns a list
of parsed S-expressions.
Each S-expression is converted into a Python list, with atoms converted
to Python strings and sub-expressions (recursively) to Python lists.
For more details, read the module file.
% XXXXJH should be funcdesc, I think

\section{Built-in Module \sectcode{pnl}}
\bimodindex{pnl}

This module provides access to the
{\em Panel Library}
built by NASA Ames (to get it, send e-mail to
{\tt panel-request@nas.nasa.gov}).
All access to it should be done through the standard module
\code{panel},
which transparantly exports most functions from
\code{pnl}
but redefines
\code{pnl.dopanel()}.

\strong{Warning:}
the Python interpreter will dump core if you don't create a GL window
before calling
\code{pnl.mkpanel()}.

The module is too large to document here in its entirety.


\chapter{SunOS Specific Services}

The modules described in this chapter provide interfaces to features
that are unique to the SunOS operating system (versions 4 and 5; the
latter is also known as Solaris version 2).

\section{Built-in Module \sectcode{sunaudiodev}}
\bimodindex{sunaudiodev}

This module allows you to access the sun audio interface. The sun
audio hardware is capable of recording and playing back audio data
in U-LAW format with a sample rate of 8K per second. A full
description can be gotten with \samp{man audio}.

The module defines the following variables and functions:

\renewcommand{\indexsubitem}{(in module sunaudiodev)}
\begin{excdesc}{error}
This exception is raised on all errors. The argument is a string
describing what went wrong.
\end{excdesc}

\begin{funcdesc}{open}{mode}
This function opens the audio device and returns a sun audio device
object. This object can then be used to do I/O on. The \var{mode} parameter
is one of \code{'r'} for record-only access, \code{'w'} for play-only
access, \code{'rw'} for both and \code{'control'} for access to the
control device. Since only one process is allowed to have the recorder
or player open at the same time it is a good idea to open the device
only for the activity needed. See the audio manpage for details.
\end{funcdesc}

\subsection{Audio Device Objects}

The audio device objects are returned by \code{open} define the
following methods (except \code{control} objects which only provide
getinfo, setinfo and drain):

\renewcommand{\indexsubitem}{(audio device method)}

\begin{funcdesc}{close}{}
This method explicitly closes the device. It is useful in situations
where deleting the object does not immediately close it since there
are other references to it. A closed device should not be used again.
\end{funcdesc}

\begin{funcdesc}{drain}{}
This method waits until all pending output is processed and then returns.
Calling this method is often not necessary: destroying the object will
automatically close the audio device and this will do an implicit drain.
\end{funcdesc}

\begin{funcdesc}{flush}{}
This method discards all pending output. It can be used avoid the
slow response to a user's stop request (due to buffering of up to one
second of sound).
\end{funcdesc}

\begin{funcdesc}{getinfo}{}
This method retrieves status information like input and output volume,
etc. and returns it in the form of
an audio status object. This object has no methods but it contains a
number of attributes describing the current device status. The names
and meanings of the attributes are described in
\file{/usr/include/sun/audioio.h} and in the audio man page. Member names
are slightly different from their C counterparts: a status object is
only a single structure. Members of the \code{play} substructure have
\samp{o_} prepended to their name and members of the \code{record}
structure have \samp{i_}. So, the C member \code{play.sample_rate} is
accessed as \code{o_sample_rate}, \code{record.gain} as \code{i_gain}
and \code{monitor_gain} plainly as \code{monitor_gain}.
\end{funcdesc}

\begin{funcdesc}{ibufcount}{}
This method returns the number of samples that are buffered on the
recording side, i.e.
the program will not block on a \code{read} call of so many samples.
\end{funcdesc}

\begin{funcdesc}{obufcount}{}
This method returns the number of samples buffered on the playback
side. Unfortunately, this number cannot be used to determine a number
of samples that can be written without blocking since the kernel
output queue length seems to be variable.
\end{funcdesc}

\begin{funcdesc}{read}{size}
This method reads \var{size} samples from the audio input and returns
them as a python string. The function blocks until enough data is available.
\end{funcdesc}

\begin{funcdesc}{setinfo}{status}
This method sets the audio device status parameters. The \var{status}
parameter is an device status object as returned by \code{getinfo} and
possibly modified by the program.
\end{funcdesc}

\begin{funcdesc}{write}{samples}
Write is passed a python string containing audio samples to be played.
If there is enough buffer space free it will immediately return,
otherwise it will block.
\end{funcdesc}

There is a companion module, \code{SUNAUDIODEV}, which defines useful
symbolic constants like \code{MIN_GAIN}, \code{MAX_GAIN},
\code{SPEAKER}, etc. The names of
the constants are the same names as used in the C include file
\file{<sun/audioio.h>}, with the leading string \samp{AUDIO_} stripped.

Useability of the control device is limited at the moment, since there
is no way to use the ``wait for something to happen'' feature the
device provides.
			% SUNOS ONLY

\documentstyle[twoside,11pt,myformat]{report}

% NOTE: this file controls which chapters/sections of the library
% manual are actually printed.  It is easy to customize your manual
% by commenting out sections that you're not interested in.

\title{Python Library Reference}

\input{boilerplate}

\makeindex			% tell \index to actually write the .idx file


\begin{document}

\pagenumbering{roman}

\maketitle

\input{copyright}

\begin{abstract}

\noindent
Python is an extensible, interpreted, object-oriented programming
language.  It supports a wide range of applications, from simple text
processing scripts to interactive WWW browsers.

While the {\em Python Reference Manual} describes the exact syntax and
semantics of the language, it does not describe the standard library
that is distributed with the language, and which greatly enhances its
immediate usability.  This library contains built-in modules (written
in C) that provide access to system functionality such as file I/O
that would otherwise be inaccessible to Python programmers, as well as
modules written in Python that provide standardized solutions for many
problems that occur in everyday programming.  Some of these modules
are explicitly designed to encourage and enhance the portability of
Python programs.

This library reference manual documents Python's standard library, as
well as many optional library modules (which may or may not be
available, depending on whether the underlying platform supports them
and on the configuration choices made at compile time).  It also
documents the standard types of the language and its built-in
functions and exceptions, many of which are not or incompletely
documented in the Reference Manual.

This manual assumes basic knowledge about the Python language.  For an
informal introduction to Python, see the {\em Python Tutorial}; the
Python Reference Manual remains the highest authority on syntactic and
semantic questions.  Finally, the manual entitled {\em Extending and
Embedding the Python Interpreter} describes how to add new extensions
to Python and how to embed it in other applications.

\end{abstract}

\pagebreak

{
\parskip = 0mm
\tableofcontents
}

\pagebreak

\pagenumbering{arabic}

				% Chapter title:

\input{libintro}		% Introduction

\input{libobjs}			% Built-in Types, Exceptions and Functions
\input{libtypes}
\input{libexcs}
\input{libfuncs}

\input{libpython}		% Python Services
\input{libsys}
\input{libtypes2}		% types is already taken :-(
\input{libtraceback}
\input{libpickle}
\input{libshelve}
\input{libcopy}
\input{libmarshal}
\input{libimp}
\input{libparser}
\input{libbltin}		% really __builtin__
\input{libmain}			% really __main__

\input{libstrings}		% String Services
\input{libstring}
\input{libregex}
\input{libregsub}
\input{libstruct}

\input{libmisc}			% Miscellaneous Services
\input{libmath}
\input{librand}
\input{libwhrandom}
\input{libarray}

\input{liballos}		% Generic Operating System Services
\input{libos}
\input{libtime}
\input{libgetopt}
\input{libtempfile}
\input{liberrno}

\input{libsomeos}		% Optional Operating System Services
\input{libsignal}
\input{libsocket}
\input{libselect}
\input{libthread}

\input{libunix}			% UNIX Specific Services
\input{libposix}
\input{libppath}		% == posixpath
\input{libpwd}
\input{libgrp}
\input{libcrypt}
\input{libdbm}
\input{libgdbm}
\input{libtermios}
\input{libfcntl}
\input{libposixfile}
\input{libsyslog}

\input{libpdb}			% The Python Debugger

\input{libprofile}		% The Python Profiler

\input{libwww}			% Internet and WWW Services
\input{libcgi}
\input{liburllib}
\input{libhttplib}
\input{libftplib}
\input{libgopherlib}
\input{libnntplib}
\input{liburlparse}
\input{libsgmllib}
\input{libhtmllib}
\input{libformatter}
\input{librfc822}
\input{libmimetools}
\input{libbinascii}
\input{libxdrlib}

\input{librestricted}
\input{librexec}
\input{libbastion}

\input{libmm}			% Multimedia Services
\input{libaudioop}
\input{libimageop}
\input{libaifc}
\input{libjpeg}
\input{librgbimg}
\input{libimghdr}

\input{libcrypto}		% Cryptographic Services
\input{libmd5}
\input{libmpz}
\input{librotor}

%\input{libamoeba}		% AMOEBA ONLY

\input{libmac}			% MACINTOSH ONLY
\input{libctb}
\input{libmacconsole}
\input{libmacdnr}
\input{libmacfs}
\input{libmacos}
\input{libmacostools}
\input{libmactcp}
\input{libmacspeech}
\input{libmacui}

\input{libstdwin}		% STDWIN ONLY

\input{libsgi}			% SGI IRIX ONLY
\input{libal}
%\input{libaudio}
\input{libcd}
\input{libfl}
\input{libfm}
\input{libgl}
\input{libimgfile}
%\input{libpanel}

\input{libsun}			% SUNOS ONLY

\input{lib.ind}			% Index

\end{document}
			% Index

\end{document}
			% Index

\end{document}
			% Index

\end{document}
